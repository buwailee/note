\chapter{Lagrange力学}

在经典力学中,描述一个系统,就是求解系统的运动,而运动就由坐标随时间的演化来描述。设坐标空间为$X$,则运动就可以看出丛$X\times \rr\to \rr$的一个截面。而在场论中,纤维丛的出现则更加自然,因为场就是截面。弦论亦是如此,设$X$是一个二维空间,而$Y$是时空,则一条玻色弦就是$X\times Y\to X$的一个截面。所以,我们期望通过纤维丛$\pi:Y\to X$来描述一个力学系统。

然而,有了一个纤维丛$\pi:Y\to X$自然是不够的,我们还需要一个动力学定律,他将告诉我们系统如何演化。在Lagrange力学中,这往往是通过一个Lagrange密度给出的。方便起见,我们这里处理的情况,一是需要$X$是可定向流形,二是Lagrange密度至多只依赖于场/坐标的一阶导数。这些条件都是可以推广的,关于第一点,在流形上积分只需要密度就行,而不需可定向,至于第二点,给出依赖于高阶导数的Lagrange密度的理论也是可行的。

\section{数学基础}

\begin{para}[仿射空间]
    设$V$是一个矢量空间,而$A$是一个集合,而$\lambda$是一个$V$作为加法群在$A$上的自由可迁(一般是右)作用,此时称$(A,V,\lambda)$是一个仿射空间。时常会直接将$A$称作仿射空间,而$V$被称为底空间,作用直接记作加法。
\end{para}

方便起见,我们会直接称呼$A$为一个仿射空间,其底空间记作$[A]$. 

\begin{para}[仿射映射]
    设$A$和$A'$都是仿射空间,$f:A\to A'$是一个映射,任取$a\in A$,$V\in V$,由于作用是可迁的,所以存在一个唯一的$w\in A'$使得$f(a+v)=f(a)+w$. 这个$w$是$a$和$v$的函数,如果$w$不依赖于$a$且线性依赖于$v$. 则称$f$是一个仿射映射(变换)。从定义,$w$可以写为一个线性变换$[f]:V\to V'$作用在$v$上的结果,所以我们有$f(a+v)=f(a)+[f]v$.
\end{para}

每个矢量空间都是一个仿射空间,上面的仿射映射都可以写作$f(v)=f(0)+[f]v$,即一个线性变换加上一个平移,如果$f(0)=0$,则仿射映射就变成了线性映射。此时,仿射变换的复合写作
\[
    gf(v)=g(f(0)+[f]v)=g(0)+[g]f(0)+[g][f]v.
\]

\begin{para}[仿射表示]
    所有$A\to A$的可逆仿射变换(或称仿射同构)显然构成一个群,记作$\operatorname{GA}(A)$. 类似于表示,群$G$到$\operatorname{GA}(A)$的一个同态被称为$G$在$A$上的一个仿射表示。由于
    \[
        fg(a+v)=f(g(a)+[g]v)=fg(a)+[f][g]v,
    \]
    所以$[fg]=[f][g]$,这就意味着,存在一个典范同态$\operatorname{GA}(A)\to \operatorname{GL}([A])$. 特别地,每一个$G$的仿射表示$\lambda$都定义了一个表示$[\lambda]:g\mapsto [\lambda(g)]$.
\end{para}


\begin{para}[仿射丛]
    类似于矢量丛,我们可以定义仿射丛。设$A$是一个仿射空间,设$\lambda$是一个Lie群$G$在仿射空间$A$上的(光滑)仿射表示,则$(G,\lambda)$-丛$A\to E\xrightarrow{\pi} M$被称为纤维为$A$的仿射丛。此外,由$G$-丛构造定理,我们可以定义出一个$(G,[\lambda])$-矢量丛。它被称为仿射丛的底矢量丛。
\end{para}

一般来说,我们也可以从矢量丛出发构造一个仿射丛,比方说如果一个纤维丛的纤维都是仿射空间,底矢量空间是对应矢量丛的纤维,并且转移函数都是仿射同构,那么它就是一个仿射丛。

\begin{para}[一阶jet丛]
    设$\pi:Y\to X$是一个纤维丛,记$V_yY=\ker\pi_{*y}\subset T_yY$,称之为垂直空间。所谓$Y$上的一阶jet丛$J^1Y$是一个仿射丛,它在点$y$的纤维是满足$\pi_{*y}\gamma=\operatorname{id}$的所有线性映射$\gamma:T_{\pi(y)}X\to T_yY$的集合,这个集合有一个自然的仿射空间结构,而所有$T_{\pi(y)}X\to V_yY$的线性映射构成其底矢量空间。所以一阶jet丛是一个仿射丛。

    一阶jet丛也可以用等价类来构造:设$\varphi$和$\psi$都是$x\in X$附近的局部截面,我们定义等价关系如下:如果$\varphi(x)=\psi(x)$且$\varphi_{*x}=\psi_{*x}$,则$\varphi\sim \psi$. 不难检验这是一个等价关系,那么我们就定义丛$J^1Y$在点$y$的纤维为在点$\pi(x)$附近的局部截面的等价类的集合。
\end{para}

\begin{lem}
    设$\gamma\in (J^1Y)_y$,则$T_yY=\im\gamma\oplus V_yY$.
\end{lem}

\begin{proof}
    从$\pi_{*y}\gamma=\id$,$\ker \gamma=\{0\}$. 首先,设$w\in \im\gamma$是非零矢量,则存在非零矢量$v$使得$w=\gamma(v)$,利用$\pi_{*y}\gamma=\id$立得$\pi_{*y}w=v\neq 0$,所以$w\not\in  V_yY$. 这就意味着$\im\gamma\cap V_yY=\{0\}$. 所以,这两个空间的和也就是它们的直和是$T_yY$的一个子空间。为证明这就是$T_yY$,只需计算维度。

    由于$\ker \gamma=\{0\}$,所以线性代数基本定理告诉我们$\dim \im\gamma=\dim T_xX$. 依然是线性代数基本定理,$\dim T_yY=\dim \ker \pi_{*y}+\dim \im \pi_{*y}=\dim V_yY+\dim T_xX$,所以$\dim T_yY=\dim V_yY+\dim \im\gamma$. 这就告诉我们$T_yY=\im\gamma\oplus V_yY$.
\end{proof}

局部地,如果我们选定了$X$的坐标$x^\mu$和$Y$的纤维坐标$y^A$,这里$\mu$和$A$都是指标,则我们可以得到$T_{\pi(y)}X$和$T_yY$的坐标,进而是$J^1Y$的纤维坐标$v^{A}_\mu$. 事实上,每一个满足$\pi_*\gamma=\id$的
线性映射$\gamma:T_{\pi(y)}X\to T_yY$作用在$\partial_\mu$上都可以唯一写成
\[
    \gamma(\partial_\mu)=\partial_\mu+\gamma^A_{\mu}\partial_A.
\]
所以,我们可将
\begin{equation}
	\gamma^A_{\mu}=\dd y^A(\gamma(\partial_\mu))
\end{equation}
用作$J^1_y Y$的坐标,坐标函数记作$v^A_\mu(\gamma)=\gamma^A_{\mu}$. 

我们还可以将$J^1Y$看成$X$上的一个丛,此时若$\phi:X\to Y$是纤维丛$\pi:Y\to X$的一个截面,则$\phi_{*x}\in J^1_{\phi(x)}Y$,于是$x\mapsto \phi_{*x}$是$J^1Y$的一个截面,当然,更形式的记法应该是$\phi^*J^1Y$的一个截面,我们将其记作$j^1\phi$. 局部地,我们可以将$j^1\phi$写作
\[
    x^\mu \mapsto \left (x^\mu,\phi^A(x^\mu),\partial_\nu \phi^A(x^\mu)\right ),
\]
这里,$\phi^A=y^A\circ \phi$. 如果从$J^1Y\to X$的截面可以写成$j^1\phi$的形式,则称这个截面是完整的(holonomic)。

% \begin{para}[Ehresmann联络]
%     设$\pi:Y\to X$是一个纤维丛,$J^1Y$丛在$Y$上的(局部)截面被称为$Y$的一个(局部)Ehresmann联络。
% \end{para}

% 传统的Ehresmann联络的定义是在每一点(光滑地)给定$T_yY$的一个直和分解,分解为一个横空间$H_yY$与垂直空间$V_yY$的直和,而上一个引理告诉我们,$J^1Y$丛的一个截面就给出了这样一族横空间的选取。

\begin{para}[对偶jet丛]
    设$\pi:Y\to X$是一个纤维丛,其中$\dim X=n$. 我们定义对偶jet丛$J^1Y^*$为一个$Y$上的矢量丛,其在点$y\in Y_x$的纤维是所有从$J^1_yY$到$\Lambda^n_x X$的仿射映射。
\end{para}

局部地,$J^1Y^*$的纤维坐标是$(p,p_A^\mu)$,它对应着仿射映射
\[
    \gamma^A_\mu\mapsto \left (p+p_A^\mu \gamma^A_\mu \right )\dd^n x,
\]
这里
\[
    \dd^n x=\dd x^1\wedge \dd x^2\wedge \cdots \wedge \dd x^n.
\]

类似于jet丛,对偶jet丛也有另一种刻画。考虑矢量丛$\Lambda^n Y\to Y$,而$Z$是其子矢量从,纤维为
\[
    Z_y=\{z\in \Lambda^n_y Y\,:\,\iota(v)\iota(w)z=0\;\;\text{for all $v$, $w\in V_yY$}\},
\]
这里的$\iota$是切矢量与微分形式自然的内积。而这个$Z$就可以自然地等同于我们的$J^1Y^*$. 实际上,每个$Z$的元素都可以唯一写为
\[
    z=p\;\dd^n x+p_A^\mu \,\dd y^A\wedge \dd^{n-1}x_\mu,
\]
这里$\dd^{n-1}x_\mu=\iota(\partial_\mu)\dd^n x$,而$\dd x^\mu$应当理解成坐标映射$x^\mu\circ \pi:Y\to \rr$的外微分,更形式地应该写作$\pi^*\dd x^\mu$. 任取$v\in V_y Y$,我们有$\pi^* \dd x^\mu(v)=\dd x^\mu (\pi_* v)=0$,所以$z$可以唯一写为上面这种形式。

现在,我们可以通过$\langle \Phi(z),\gamma\rangle =\gamma^*z\in \Lambda^{n}_x X$来逐点定义一个映射$\Phi:Z\to J^1Y^*$,其中$z\in Z_y$, $\gamma\in J^1_y Y$. 局部地
\[
    \gamma^*\dd y^A(\partial_\mu)=\dd y^A(\gamma (\partial_\mu))=
    (v^B_\mu\circ \gamma) \dd y^A(\partial_B)=v^A_\mu\circ \gamma,
\]
在注意到$\pi_*\gamma=\id$,即$\gamma^*\pi^*=\id$,这给出了
\[
    \gamma^*\pi^*\dd x^\mu=\dd x^\mu.
\]
因此,局部地,我们有
\[
    \gamma^*z = \left(p+p_A^\mu (v^A_\mu\circ \gamma)\right)\dd^n x,
\]
这就回到了我们前面的局部刻画。$\Phi$的逆同样不难构造,因此$Z$和$J^1Y^*$同构。

\begin{para}[正则形式]
   考虑矢量丛$\tau:\Lambda^n Y\to Y$,任取$z\in \Lambda^n Y$,其为$\tau(z)\in Y$上的$n$-形式,于是可以定义一个$\Lambda^n Y$上的$n$-形式$\Theta_\Lambda$,其在点$z$的值为$(\Theta_\Lambda)_z=\tau^*z$. 这被称为$\Lambda^n Y$上的正则$n$-形式。定义$\Omega_\Lambda=-\dd \Theta_\Lambda$,这被称为$\Lambda^n Y$上的正则$(n+1)$-形式。

    如果把$J^1Y^*$看成$\tau:\Lambda^n Y\to Y$的子丛(即前面的$Z$),含入为$i:J^1Y^*\hookrightarrow \Lambda^n Y$,则我们就可以定义$J^1Y^*$上的正则$n$和$(n+1)$-形式,
    \[
        \Theta=i^*\Theta_\Lambda,\quad \Omega=i^*\Omega_\Lambda,
    \]
    于是$\Omega=-\dd \Theta$.
\end{para}

局部地,
\[
    \Theta=p_A^\mu\,\dd y^A\wedge \dd^{n-1}x_\mu + p\, \dd^n x,
\]
以及
\[
    \Omega=\dd y^A\wedge\dd p_A^\mu\wedge\dd ^{n-1}x_\mu-\dd p\wedge\dd^{n}x.
\]
这里的$p$, $p^\mu_A$, $x^\mu$等都应该理解为$J^1Y^*$上的坐标函数,即$p\tau$, $x^\mu\pi\tau$等,其中$\tau:J^1Y^*\to Y$和$\pi:Y\to X$是丛的典范映射。

\begin{para}[点粒子的例子]
    考虑一个粒子,描述它的运动我们需要时间$X=\rr$以及丛$Y=M\times \rr$,其中$M$是其可以运动的空间,此时,局部地
    \[
        \Theta = p_A\dd y^A+p\dd t,
    \]
    这里$p$就代表(负)能量,而$p_A$代表的是对应坐标$y^A$的动量,进而
    \[
        \Omega=\dd y^A\wedge\dd p_A-\dd p\wedge\dd t.
    \]
    就是在Hamilton力学中熟知的辛形式。实际上,在$n=1$的情况,$\Lambda^1Y$就是余切丛$T^*Y$.
\end{para}

\begin{para}[丛映射]
    设$\pi':Y'\to X'$和$\pi:Y\to X$是两个纤维丛,如果有两个映射$\alpha:Y\to Y'$, $\beta:X\to X'$,满足交换图
    \[
        \xymatrix{
            Y\ar[d]_\pi\ar[r]^\alpha&Y'\ar[d]^{\pi'}\\
            X\ar[r]^\beta&X'
        }
    \]
    则称$(\alpha,\beta)$是一个丛映射,如果$X=X'$且$\beta=\id_X$,则称$\alpha$为一个保持纤维的丛映射。
\end{para}

\begin{para}[丛映射的提升]
    记号同上,记$\Phi=(\alpha,\beta)$为丛映射,其中$\beta$还是一个同胚,我们考虑如下交换图
    \[
        \xymatrix{
            J^1Y \ar[r] \ar@{-->}[d]^{j^1\Phi} &Y\ar[r]^\pi\ar[d]^\alpha&X\ar[d]^\beta\\
            J^1Y' \ar[r] &Y'\ar[r]^{\pi'}&X'
        }
    \]
    其中略去的都是自然构造中出现的箭头,为使$j^1$是一个函子,我们还需构造途中的$j^1\Phi$使得交换图成立。

    任取$\gamma\in j^1_yY$,它是一个线性映射$T_{*x}X\to T_{*y}Y$,其中$x=\pi(y)$. 而丛映射诱导了线性映射$\beta_{*x}:T_xX\to T_{\beta(x)}X'$以及$\alpha_{*y}:T_yY\to T_{\alpha(y)}Y'$. 我们想从这些材料构造一个线性映射$j^1\Phi(\gamma):T_{*\beta(x)}X'\to T_{*\alpha(y)}Y'$,满足$\pi'_*j^1\Phi(\gamma)=\id$,即$j^1\Phi(\gamma)\in j^1_{\beta(y)}Y'$.
    
    注意到,$\beta_{*x}$可逆,此时
    \[
        j^1\Phi(\gamma)=\alpha_{*y}\gamma \beta_{*x}^{-1}:T_{*\beta(x)}X'\to T_{*\alpha(y)}Y'
    \]
    就是所需。实际上,取$\pi'_{*\alpha(y)}$,从交换图可以得到
    \[
        \pi'_{*\alpha(y)}\alpha_{*y}\gamma \beta_{*x}^{-1}=\beta_{*x}\pi_{*y}\gamma \beta_{*x}^{-1},
    \]
    再注意到$\pi_{*y}\gamma=\id$,所以立刻得到$\pi'_{*\alpha(y)}\alpha_{*y}\gamma \beta_{*x}^{-1}=\id$. 

    不难看到,如果$\Phi$, $\Psi$是两个丛映射,则$j^1(\Phi\Psi)=j^1\Phi \circ j^1\Psi$. 所以$j^1$(连同$J^1$)在某些特定的情况下就构成了一个函子,比方说,在对象是所有以$X$为底的纤维丛,而态射是所有保持纤维的丛映射构成的范畴中。

    此外,从构造可以看到,即使无法谈论全局提升,谈论局部提升也是有意义的。这在后面对于矢量场的研究中是重要的,因为矢量场的流不一定是全局定义的,但微分方程的存在唯一性定理保证了其在局部可以良好定义。
\end{para}

回到纤维丛$\pi:Y\to X$的截面$\phi:X\to Y$的情况,前面已经看到,他确定了一个截面$j^1 \phi:x\mapsto \phi_{*x}$. 这里,考虑一个保持纤维的丛映射$\alpha:Y\to Y'$,则我们有两个新的截面$\alpha\phi:X\to Y'$和$j^1(\alpha\phi):X\to J^1Y'$,我们有如下命题。

\begin{lem}
    记号同上,$j^1(\alpha\phi)=j^1\alpha \circ j^1\phi$.
\end{lem}

\begin{proof}
    从定义显然。
\end{proof}

这就告诉我们,至今所有对于$j^1$的使用都是相容的。

\begin{para}[垂直矢量场的提升]
    为方便起见,这里假设所有东西都可以在全局定义,它们全都可以简单地回到局部定义的情况。

    设$\pi:Y\to X$是一个纤维丛,$V$是$Y$上的一个垂直矢量场,即任取$y\in Y$都有$V_y\in V_yY$,它定义了一个流$\sigma^V_t:Y\to Y$,此外,由于
    \[
        \left.\frac{\dd}{\dd \lambda}\right|_{\lambda=0}\pi(\sigma^V_ty)=\pi_{*y}(V_y)=0,
    \]
    所以垂直矢量场的流$\sigma^V_t$保持纤维。继而可以提升为$J^1Y$上的一个流$j^1(\sigma^V_t):J^1Y\to J^1Y$,所以确定了$J^1Y$上的一个矢量场,我们记作$j^1V$.

    下面我们局部计算$j^1V$. 首先注意到,$j^1(\sigma^V_\lambda)$的作用就是将$\gamma\in J^1_yY$变成$(\sigma^V_\lambda)_{*y}\gamma\in J^1_{\sigma^V_\lambda(y)}Y$,它关于$\lambda$在$\lambda=0$的导数就是我们所需的$j^1V$.

    任取$J^1Y$上的函数$f$,我们要计算
    \[
        (j^1V)f=\left.\frac{\dd}{\dd \lambda}\right|_{\lambda=0}f\left((\sigma^V_\lambda)_{*}\gamma\right).
    \]
    取$f=x^\mu\pi\tau$,其中$\pi:Y\to X$, $\tau:J^1Y\to Y$都是丛映射,则
    \[
        x^\mu\pi\tau\left((\sigma^V_\lambda)_{*}\gamma\right)=x^\mu\pi\left(\sigma^V_\lambda(y)\right)=x^\mu\pi(y)=x^\mu,
    \]
    所以这是常函数,进而$(j^1V)(x^\mu\pi\tau)=0$,即不存在$\partial_{\mu}$分量。接着取$f=y^A\tau$,则
    \[
        y^A\tau\left((\sigma^V_\lambda)_{*}\gamma\right)=y^A\left(\sigma^V_\lambda(y)\right),
    \]
    所以它的导数就是$V^A$,所以$\partial_A$分量的系数为$V^A$. 最后取$f=v^A_\mu$,利用式(\theequation),我们有
    \[
        v^A_\mu((\sigma^V_\lambda)_{*}\gamma)=\dd y^A((\sigma^V_\lambda)_{*}\gamma(\partial_\mu))=\left ((\sigma^V_\lambda)^{*}\dd y^A\right )(\gamma_\mu^B\partial_B+\partial_\mu),
    \]
    所以它的导数就是
    \[
        \mathscr L_V\dd y^A(\gamma_\mu^B\partial_B+\partial_\mu)=\dd V^A(\gamma_\mu^B\partial_B+\partial_\mu)=\partial_\mu V^A+\gamma^B_\mu \partial_B V^A,
    \]
    其中$\mathscr L_V$是Lie导数。综上,我们有
    \[
        j^1V=\left(0,V^A,\partial_\mu V^A+\gamma^B_\mu \partial_B V^A\right).
    \]
\end{para}

\section{Lagrange方程}

现在开始,在纤维丛$\pi:Y\to X$中我们将默认$X$是$D=n+1$维的,其坐标为$x^0$, $\dots$, $x^n$,当$n=0$时,就回到了质点力学。注意,这里我们还不需要$X$和$Y$上有什么附加结构。

\begin{para}[Lagrange密度]
    Lagrange密度$\mathcal L$是一个丛映射$\mathcal L:J^1Y\to \Lambda^{n+1}X$,换而言之,成立交换图
    \[
        \xymatrix{
            J^1Y\ar[d]\ar[r]^-{\mathcal L}&\Lambda^{n+1}X\ar[d]\\
            Y\ar[r]&X
        }
    \]
    其中略去标记的箭头都是丛的典范投影。
\end{para}

对点$\gamma\in J^1Y$,其中$\gamma:T_xX\to T_yY$是一个线性映射,而$x=\pi(y)$. 此时,局部地,我们可以将Lagrange密度$\mathcal{L}(\gamma)$写作
\[
    \mathcal{L}(\gamma)=L\left(x^\mu,y^A,\gamma^A_\mu\right)\dd^{n+1}x,
\]
其中$\gamma^A_\mu=v^A_\mu(\gamma)$.

\begin{para}[Legendre变换]
    Legendre变换是一个保持纤维的丛映射$F_{\mathcal L}:J^1Y\to J^1Y^*$,即满足交换图
    \[
        \xymatrix{
        J^1Y\ar[dr]\ar[rr]^-{F_{\mathcal L}}&&J^1Y^*\ar[dl]\\
        &Y&
        }
    \]
    以及
    \[
        F_{\mathcal L}(\gamma):\sigma\mapsto\mathcal{L}(\gamma)+\frac{\dd}{\dd \epsilon}\mathcal L(\gamma+\epsilon(\sigma-\gamma)),
    \]
    其中$\gamma$, $\sigma\in J^1_yY$. 所以$F_{\mathcal L}(\gamma)$从某种角度来看是$\mathcal L$在点$\gamma$的仿射近似。
\end{para}

局部地,我们有
\[
    \mathcal{L}(\gamma)+\frac{\dd}{\dd \epsilon}\mathcal L(\gamma+\epsilon(\sigma-\gamma))=\left(L(\gamma)+\frac{\partial L}{\partial v^A_\mu}(\gamma)(\sigma^A_\mu-\gamma^A_\mu)\right)\dd^{n+1}x,
\]
所以其对应的仿射变换$\sigma^A_\mu\mapsto (p(\gamma)+p_A^\mu(\gamma) \sigma^A_\mu)\dd^{n+1}x$立即给出
\[
    p=L-\frac{\partial L}{\partial v^A_\mu}v^A_\mu,\quad p^\mu_A=\frac{\partial L}{\partial v^A_\mu},
\]
此即经典的Legendre变换。

\begin{para}[Cartan形式]
    已知$J^1Y^*$上的正则$D$-形式$\Theta$,于是,利用Legendre变换$F_{\mathcal L}:J^1Y\to J^1Y^*$可以得到一个$J^1Y$上的$D$-形式
    \[
        \Theta_{\mathcal L}=F_{\mathcal L}^*\Theta,
    \]
    他被称为Cartan形式。Cartan形式的(负)外微分我们记作$\Omega_{\mathcal L}=-\dd \Theta_{\mathcal L}$.
\end{para}

局部地,Cartan形式写作
\[
    \Theta_{\mathcal L}=\frac{\partial L}{\partial v^A_\mu}\dd y^A\wedge \dd^{D-1}x_\mu + \left(L-\frac{\partial L}{\partial v^A_\mu}v^A_\mu\right)\dd^{D} x.
\]

\begin{pro}\label{pro:1.17}
    设$\phi:X\to Y$是纤维丛$\pi:Y\to X$的截面,则
    \[
        \mathcal L(j^1\phi)=(j^1\phi)^*\Theta_{\mathcal L}.
    \]
    其中$j^1\phi$是$\phi^*J^1Y\to X$的由$x\mapsto \phi_{*x}$给出的截面。
\end{pro}

\begin{proof}
    局部地,我们有
    \[
        (j^1\phi)^*\Theta_{\mathcal L}=\frac{\partial L}{\partial v^A_\mu}(j^1\phi)\dd \phi^A\wedge \dd^{D-1}x_\mu + \left(L(j^1\phi)-\frac{\partial L}{\partial v^A_\mu}(j^1\phi)\partial_\mu \phi^A\right)\dd^{D} x,
    \]
    这里我们应用了$v_\mu^A(j^1\phi)=\partial_\mu\phi^A$,进而,通过
    \[
        \dd \phi^A\wedge \dd^{D-1}x_\mu=\partial_\mu \phi^A\dd^{D} x,
    \]
    我们立刻得到
    \[
        (j^1\phi)^*\Theta_{\mathcal L}=L(j^1\phi)\dd^{D} x=\mathcal L(j^1\phi),
    \]
    此即所需。
\end{proof}

换而言之,我们可以通过Cartan形式来还原出Lagrange密度,这也给出了下面关于Lagrangian方程的不同刻画之一。

\begin{thm}\label{Lagrangeeq}
    设我们有Lagrange密度$\mathcal{L}=L\left(x^\mu,y^A,v^A_\mu\right)\dd^{n+1}x$,以及纤维丛$\pi:Y\to X$的截面$\phi$,则以下命题等价:
    \begin{compactenum}
        \item 任取$Y$上的垂直矢量场$V$,垂直指$V_y\in V_yY$,如果它在$X$中的支集紧,定义$\phi_\lambda=\sigma^\lambda_V\circ \phi$,其中$\sigma^\lambda_V$为$V$的流,则\[\left.\frac{\dd}{\dd \lambda}\right|_{\lambda=0}\int_X \mathcal L(j^1\phi_\lambda)=0.\]
        \item $L$满足Lagrange方程:
        \[
            \frac{\partial}{\partial y^A}(j^1\phi)-\frac{\partial}{\partial x^\mu}\left(\frac{\partial L}{\partial v^A_\mu}(j^1\phi)\right)=0.
        \]
        \item 任取$J^1Y$上的矢量场$W$,则 
        \[
            (j^1\phi)^*(\iota(W)\Omega_{\mathcal L})=0.
        \]
    \end{compactenum}
\end{thm}

\begin{proof}
    1和2的等价性完全是经典的。为看出1和3的等价性,我们将$\mathcal L(j^1\phi_\lambda)$表为Cartan形式,于是
    \begin{align*}
        \left.\frac{\dd}{\dd \lambda}\right|_{\lambda=0}\int_X \mathcal L(j^1\phi_\lambda)\;&=\left.\frac{\dd}{\dd \lambda}\right|_{\lambda=0}\int_X \mathcal (j^1\phi_\lambda)^*\Theta_{\mathcal L}
        \\&=\left.\frac{\dd}{\dd \lambda}\right|_{\lambda=0}\int_X \mathcal (j^1\phi)^*(j^1\sigma^V_\lambda)^*\Theta_{\mathcal L}\\
        &=\int_X \mathcal (j^1\phi)^*\mathscr L_{j^1V}\Theta_{\mathcal L},
    \end{align*}
    其中$\mathscr L_{j^1V}$是关于矢量场$j^1V$的Lie导数,$j^1V$是由流$j^1(\sigma_\lambda^V)$给出的矢量场。利用Cartan公式
    \[
        \mathscr L_{j^1V}=\dd \iota(j^1V)+\iota(j^1V)\dd,
    \]
    我们立刻得到
    \[
        \mathscr L_{j^1V}\Theta_{\mathcal L}=-\iota(j^1V)\Omega_{\mathcal L}+\dd\left(\iota(j^1V) \Theta_{\mathcal L}\right),
    \]
    从Stokes定理,第二项积分后为零,所以
    \[
        \left.\frac{\dd}{\dd \lambda}\right|_{\lambda=0}\int_X \mathcal L(j^1\phi_\lambda)=-\int_X \mathcal (j^1\phi)^*(\iota(j^1V)\Omega_{\mathcal L}).
    \]
    所以,我们立刻从3得到了1. 反过来,因为$1$对任意的流都成立,于是上式给出
    \[
        \int_X \mathcal (j^1\phi)^*(\iota(W)\Omega_{\mathcal L})=0,
    \]
    对任意的矢量场$W$成立。因为$W$可以乘以任意一个$X$上的函数,于是,类似于变分基本定理,我们立刻得到被积式为零。
\end{proof}

局部地,经过不长的计算,就可以得到
\[
    (j^1\phi)^*(\iota(j^1V)\Omega_{\mathcal L})=V^A\left[\frac{\partial}{\partial y^A}(j^1\phi)-\frac{\partial}{\partial x^\mu}\left(\frac{\partial L}{\partial v^A_\mu}(j^1\phi)\right)\right]\dd^{n+1}x,
\]
而这也可以直接给出3到2,而2到1是显然的,这也顺便给了1和2的等价性的另一个证明。

\begin{para}[动力学原理]
    设一个系统$\pi:Y\to X$的Lagrange密度为$\mathcal{L}:J^1Y\to \Lambda^{n+1}X$,则截面$\phi:X\to Y$的演化遵循Lagrange方程,或上面命题中的任意一条等价原理。
\end{para}

几乎任何(经典)物理系统都遵循这条原理。

\section{正则变换与流}

在本节中,$Z$都是指$\Lambda^{D}Y$的那个同构于$J^1Y^*$的子丛。

\begin{para}[纤维丛的自同构]
设$Y\to X$是一个纤维丛,则它的一个自同构$\eta$由两个同胚$\eta_Y$和$\eta_X$构成,且满足交换图
\[
    \xymatrix{
        Y \ar[r] \ar[d]^{\eta_Y} &X\ar[d]^{\eta_X}\\
        Y \ar[r] &X
    }
\]
\end{para}

现在,显然我们有两个纤维丛$Z\to Y$和$Y\to X$,它们复合得到了纤维丛$Z\to X$. 
设已有一个$Y\to X$的自同构$(\eta_Y,\eta_X)$,如果$Z\to X$的自同构$(\eta_Z,\eta_X)$使得如下交换图成立
\[
    \xymatrix{
        Z \ar[r] \ar[d]^{\eta_Z} &Y\ar[r]\ar[d]^{\eta_Y}&X\ar[d]^{\eta_X}\\
        Z \ar[r] &Y\ar[r]&X
    }
\]
即$(\eta_Z,\eta_Y)$构成了$Z\to Y$的自同构,此时称$(\eta_Z,\eta_X)$是$(\eta_Y,\eta_X)$的一个提升。方便起见,若没有歧义,则简称为$\eta_Z$是$\eta_Y$的一个提升。

\begin{para}[正则提升]
    设同胚$\eta_Y:Y\to Y$, $\eta_X:X\to X$构成丛映射,则我们可以通过
    \[
        \eta_Z (z)=(\eta_Y^{-1})^*z
    \]
    来定义一个提升$\eta_Z :Z\to Z$. 事实上,任取$z\in Z$以及$v$, $w\in V_yY$,我们都有
    \[
        \iota(v)\iota(w)(\eta_Y^{-1})^*z=(\eta_Y^{-1})^*\left[\iota((\eta_Y^{-1})_*v)\iota((\eta_Y^{-1})_*w)z\right],
    \]
    注意到$(\eta_Y^{-1})_*v)$也在$V_y$中,这是因为
    \[
        \pi_*(\eta_Y^{-1})_*v=(\eta_X^{-1})_*\pi_*v=0,
    \]
    所以立刻可以得到$(\eta_Y^{-1})^*z\in Z$. 这被称为一个正则提升,且这显然是函子性的。
\end{para}

此外,在$J^1Y^*$上,我们也可以通过
\begin{equation}\label{j1y*}
\langle \eta_{J^1Y^*}(z),\gamma\rangle=(\eta_{X}^{-1})^*\langle z,(j^1\eta_Y)^{-1}(\gamma)\rangle
\end{equation}
来定义$J^1Y^*$上的一个同胚,这里$\langle *,*\rangle$是对偶空间中的自然配对。而同构$\Phi:Z\to J^1Y^*$给出了
$\eta_{J^1Y^*}$和$\eta_Z$的关系。

\begin{lem}
$\Phi\eta_Z=\eta_{J^1Y^*}\Phi$.
\end{lem}

\begin{proof}
证明是直接的:
\begin{align*}
	\langle \eta_{J^1Y^*}\Phi (z),\gamma\rangle&=(\eta_{X}^{-1})^*\langle \Phi(z),(j^1\eta_Y)^{-1}(\gamma)\rangle\\
	&=(\eta_{X}^{-1})^*\langle \Phi(z),(\eta_Y)_*^{-1}\gamma(\eta_X)_*\rangle\\
	&=(\eta_{X}^{-1})^*\left((\eta_Y)_*^{-1}\gamma(\eta_X)_*\right)^*z\\
	&=\left((\eta_Y)_*^{-1}\gamma\right)^*z\\
	&=\gamma^*\left(\eta_Z(z)\right)\\
	&=\langle \Phi\eta_Z(z),\gamma\rangle.\qedhere
\end{align*}
\end{proof}

所以,我们可以进一步通过$\Phi$来等同$Z$和$J^1Y^*$,上一个引理告诉我们,这个等同在同胚的提升意义上也是成立的,且成立
\[
	(\eta_{X})^*\langle \eta_{J^1Y^*}(z),\eta_{J^1Y}(\gamma)\rangle=\langle z,\gamma\rangle,
\]
这里$\eta_{J^1Y}:=j^1\eta_Y$.

\begin{para}[协变正则变换]
    设$(\eta_Z,\eta_X)$是$Z\to X$的一个自同构,如果$\eta_Z^*\Omega=\Omega$,则称$\eta_Z$是一个协变正则变换,如果$\eta_Z^*\Theta=\Theta$,则称$\eta_Z$是一个特殊协变正则变换。
\end{para}

% 这里,$\eta_Z^*\Theta=\Theta$应当理解作$\eta_Z^*\Theta_{\eta_Z(z)}=\Theta_{z}$. 
方便起见,在不产生歧义的地方,我们将略去“协变”二字。

\begin{pro}
    丛自同构$\eta_Y$的正则提升$\eta_{Z}$是一个特殊正则变换。
\end{pro}

\begin{proof}
    首先,回顾$(\Theta_\Lambda)_z=\tau^*z$,这里$\tau:\Lambda^{n+1}Y\to Y$是典范映射,以及$\Theta=i^*\Theta_\Lambda$. 按照定义,考虑
    \begin{align*}
        (\eta_{Z}^*i^*\Theta_\Lambda)_z(u_1,\dots,u_N)&=(\Theta_\Lambda)_{\eta_{Z}(z)}((\eta_{Z})_{*}u_1,\dots,(\eta_{Z})_{*}u_N)\\
        &=(\tau^* (\eta_{Z}(z)))_{\eta_{Z}(z)}((\eta_{Z})_{*}u_1,\dots,(\eta_{Z})_{*}u_N)\\ 
        &=\eta_{Z}(z)_{\tau\eta_{Z}(z)}((\tau\eta_{Z})_{*}u_1,\dots,(\tau\eta_{Z})_{*}u_N)
    \end{align*}
    注意到,因为$\eta_Z$是$\eta_Y$的提升,所以$\tau\eta_Z=\eta_Y \tau$,故
    \begin{align*}
        ((\eta_{Z})^*i^*\Theta_\Lambda)_z(u_1,\dots,u_N)&=
        \eta_{Z}(z)_{\eta_{Y}\tau(z)}((\eta_{Y}\tau)_{*}u_1,\dots,(\eta_{Y}\tau)_{*}u_N)\\
        &=((\eta_Y^{-1})^*z)_{\eta_{Y}\tau(z)}((\eta_{Y}\tau)_{*}u_1,\dots,(\eta_{Y}\tau)_{*}u_N)\\
        &=z_{\tau(z)}((\tau)_{*}u_1,\dots,(\tau)_{*}u_N)\\
        &=(\tau^*z)_z(u_1,\dots,u_N)\\
        &=(i^*\Omega_\Lambda)_z(u_1,\dots,u_N)\\
        &=\Omega_z(u_1,\dots,u_N),
    \end{align*}
    于是$(\eta_{Z})^*\Theta=\Theta$. 此即所需。
\end{proof}

现在,设有一个Lie群$G$,任取群元$g$,它在$X$, $Y$, $Z$上的作用$g_X$, $g_Y$和$g_Z$构成了$Z\to Y\to Z$的一个同构。若$g=\exp(t\xi)$,当$t$趋向于零时,上述作用给出了
三个矢量场$\xi_X$, $\xi_Y$和$\xi_Z$. 特别地,如果任取$g\in G$,$g_Z$总是一个正则变换,则
\[
    \mathscr L_{\xi_Z}\Omega=0,
\]
如果是特殊正则变换,则
\[
    \mathscr L_{\xi_Z}\Theta=0.
\]

\begin{para}[流]
    设所有正则变换构成了一个作用在$Z$上的Lie群$G$,其Lie代数记作$\mathfrak g$. 如果映射
    \[
        J:Z\to \mathfrak g^*\otimes \Lambda^n Z,
    \]
    对任意的$\xi\in \mathfrak g$都成立
    \[
    \dd (J(\xi))=\iota(\xi_Z)\Omega,
    \]
    则我们将$J$称之为一个流。在对称性分析中,它的Legendre变换将担当经典的守恒流的作用。
    注意到,如果将$\Omega$类比为辛形式,则上式就是在说$J(\xi)$的“Hamilton矢量场”
    为$\xi_Z$. 这个类比会在下面定义Possion括号时候再次用到。
\end{para}

如果$\eta_Z$都是特殊正则映射,则此时
\[
	J(\xi)=\iota(\xi_Z)\Theta
\]
就定义了一个流$J$. 实际上,此时
\[
	\dd (J(\xi))=\dd \iota(\xi_Z)\Theta=\mathscr L_{\xi_Z}\Theta-\iota(\xi_Z)\dd \Theta=\iota(\xi_Z)\Omega.
\]
这样定义的流被称为特殊流。

\begin{pro}
若$G$在$Z$上的作用是$G$在$Y$上的自同构的正则提升,则此时特殊流可以写作
\[
	J(\xi)(z)=p^*\iota(\xi_Y)z,
\]
其中$p:Z\to Y$为丛的典范投影。
\end{pro}

\begin{proof}
从定义,$\Theta_z=p^*z$,所以
\[
	J(\xi)(z)=\iota(\xi_Z)\Theta_z=\iota(\xi_Z)p^*z=p^*\iota(p_*\xi_Z)z,
\]
此外,任取$y=p(z)\in Y$以及$Y$上的光滑函数$f$,我们有
\[
	(p_*\xi_Z)_{p(z)} f=(\xi_Z)_p (f\circ p)=\lim_{t\to 0}\frac{f(p\exp(t\xi)_Z(z))-f(p(z))}{t}=\lim_{t\to 0}\frac{f(\exp(t\xi)_Y(y))-f(y)}{t}=(\xi_Y)_yf,
\]
其中我们应用了$p\exp(t\xi)_Z=\exp(t\xi)_Yp$,所以矢量场的关系$p_*\xi_Z=\xi_Y$立刻给出了结论。
\end{proof}

局部地,如果$\xi_Y$具有分量$(\xi^\mu,\xi^A)$,而$z$具有分量$(p,p_A^\mu)$,则
\begin{equation}
J(\xi)(z)=(p_A^\mu \xi^A+p\xi^\mu)\dd^{D-1}x_\mu-p_A^\mu \xi^\nu \dd y^A
	\wedge \dd^{D-2}x_{\mu\nu},
\end{equation}
其中
\[
	\dd^{D-2}x_{\mu\nu}=\iota(\partial_\nu)\iota(\partial_\mu)\dd^{D}x.
\]

\begin{pro}
对特殊流,成立$J(\operatorname{Ad}_g^{-1}\xi)=g_Z^*J(\xi)$.
\end{pro}

\begin{proof}
首先,任取$\xi\in \mathfrak g$以及$Z$上的光滑函数$f$
\begin{align*}
((g_Z)_*\xi_Z)_{z} f=(\xi_Z)_{g_Z^{-1}(z)} (f\circ g_Z)&=\lim_{t\to 0}\frac{f(g_Z\exp(t\xi)_Zg_Z^{-1}(z))-f(z)}{t}\\
&=\lim_{t\to 0}\frac{f((g\exp(t\xi)g^{-1})_Z(z))-f(z)}{t}\\
&=(\operatorname{Ad}_g(\xi)_Z)_zf,
\end{align*}
即$(g_Z)_*\xi_Z=\operatorname{Ad}_g(\xi)_Z$. 现在,考虑特殊流
\[
	J(\operatorname{Ad}^{-1}_g(\xi))=\iota(\operatorname{Ad}_g^{-1}(\xi)_Z)\Theta
	=g_Z^*(g^{-1}_Z)^*\iota\left((g^{-1}_Z)_*\xi_Z\right)\Theta=g_Z^*\iota(\xi_Z)(g^{-1}_Z)^*\Theta,
\]
注意到此时$g^{-1}_Z$是特殊正则变换,所以$(g^{-1}_Z)^*\Theta=\Theta$,这就给出了结论。
\end{proof}

在上面命题的式子中取$g=\exp(t\zeta)$,然后对$t$在$t=0$处求导,我们立刻得到了
\[
	J([\xi,\zeta])=\mathscr L_{\zeta_Z}J(\xi)=\dd \iota(\zeta_Z)J(\xi)+\iota(\zeta_Z)\dd J(\xi)=\dd \iota(\zeta_Z)J(\xi)+\iota(\zeta_Z)\iota(\xi_Z) \Omega,
\]
定义Possion括号
\[
	\{J(\xi),J(\zeta)\}=\iota(\zeta_Z)\iota(\xi_Z) \Omega,
\]
则我们就得到了
\[
	\{J(\xi),J(\zeta)\}=J([\xi,\zeta])-\dd \iota({\zeta_Z})J(\xi)=J([\xi,\zeta])+\dd \left(\iota({\zeta_Z})\iota({\xi_Z})\Theta\right).
\]
不难看到,Possion括号是反对称的。

\begin{para}
我们下面来考虑一个最简单的例子,矩阵Lie群$G$在标量场$y^A$上的线性作用$U^A_By^B$,现在设$U=\exp(\lambda t_a)$,
其中$t_a\in \mathfrak{g}$,于是
\[
	(t_a)_Yf=\lim_{\lambda \to 0}\frac{f(\exp(\lambda (t_a)^A_B)y^B)-f(y^A)}{\lambda}=(t_a)^A_By^B\partial_Af,
\]
所以
\[
	(t_a)_Y=(t_a)^A_By^B\partial_A,
\]
而将其带入(\theequation)立刻得到
\[
	J(t_a)(z)=p_A^\mu (t_a)^A_By^B\dd^{D-1}x_\mu,
\]
其中$z=(p,p_A^\mu)$. 此外,考虑两个生成元$t_a$和$t_b$,这里可以直接看出$\iota({(t_a)}_Z)\iota({(t_b)}_Z)\Theta=0$,所以
\[
	\{J(t_a),J(t_b)\}=J([t_a,t_b]),
\]
于是$J$可以看成一个Lie代数同态。
\end{para}

% 将上例中的$J(t_a)(z)$用Legendre变换$F_{\mathcal L}$拉回后,我们立刻得到
% \[
% 	F_{\mathcal L}^*J(t_a)(z)=\frac{\partial L}{\partial v_\mu^A}(t_a)^A_By^B\dd^{D-1}x_\mu,
% \]
% 在具有$\operatorname{SO}(N)$对称性的标量场理论中
% \[
% 	L=\frac 12\partial_\mu \phi^A\partial^\mu \phi_A-\frac {m^2}2\phi^A\phi_A,
% \]
% 这里$\phi_A=\delta_{AB}\phi^B$. 在我们这里,将$y^A$取作$\phi^A$,则$v_\mu^A=\partial_\mu \phi^A$,故
% \[
% 	F_{\mathcal L}^*J(t_a)(z)=\frac{\partial L}{\partial (\partial_\mu \phi^A)}(t_a)^A_B\phi^B\dd^{D-1}x_\mu
% 	=\partial^\mu \phi_A(t_a)^A_B\phi^B\dd^{D-1}x_\mu=j^\mu\dd^{D-1}x_\mu,
% \]
% 其中$j^\mu=\partial^\mu \phi_A(t_a)^A_B\phi^B$就是我们熟知的守恒流。

\section{对称性与Noether定理}

考虑一个群$G$,它作用于$X$和$Y$上,使得任取$g\in G$,其对应的同胚$g_X:X\to X$和$g_Y:Y\to Y$构成一个丛映射(实际是丛同构)。那么,我们可以将其提升到$J^1Y$上,记作$j^1g$或$g_{J^1Y}$.

\begin{para}[Lagrange密度的对称性]
    如果任取$g\in G$以及$\gamma\in J_y^1 Y$,Lagrange密度$\mathcal L$都成立
    \[
        \mathcal L(j^1g(\gamma))=(g_X^{-1})^*\mathcal L(\gamma),
    \]
    则称Lagrange密度$\mathcal L$具有$G$-对称性。
\end{para}

相当一大类理论(比如规范理论)都可以纳入这个框架中,但也有一类理论不在这个框架内,拓扑场论中$\mathcal L$并不是不变的,取而代之的,是正则形式$\Omega_{\mathcal L}$.

\begin{pro}
    如果$\mathcal L$具有$G$-对称性,则
    \begin{compactenum}
        \item 对Legendre变换$F_{\mathcal L}$,交换图
        \[
            \xymatrix{
                J^1Y\ar[r]^-{F_{\mathcal L}}\ar[d]_{j^1g}&J^1Y^*\ar[d]^{g_{j^1Y^*}}\\ 
                J^1Y\ar[r]^-{F_{\mathcal L}}&J^1Y^*
            }
        \]
        对所有$g\in G$都成立。
        \item 对Cartan形式$\Theta_{\mathcal L}$,$(j^1g)^*\Theta_{\mathcal L}=\Theta_{\mathcal L}$对所有$g\in G$都成立。
        \item 对所有的$\xi \in \mathfrak{g}$,映射$J^\mathcal{L}(\xi):=F_{\mathcal L}^*J(\xi):J^1Y\to \Lambda^n(J^1Y)$满足
        \[
        \dd J^\mathcal{L}(\xi)=\iota(\xi_{J^1Y})\Omega_{\mathcal L},
        \]
        其中$\xi_{J^1Y}=j^1\xi_Y$是对应于无穷小生成元$\xi$的矢量场,此外
        \[
        J^\mathcal{L}(\xi)=\iota(\xi_{J^1Y})\Theta_{\mathcal L}.
        \]
    \end{compactenum}
\end{pro}

\begin{proof}
利用式\eqref{j1y*},我们有
\begin{align*}
	\langle g_{J^1Y^*}(F_{\mathcal L}(\gamma)),\gamma'\rangle&=(g_{X}^{-1})^*\langle F_{\mathcal L}(\gamma),(j^1g)^{-1}(\gamma)\rangle\\
&=(g_{X}^{-1})^*\left[
\mathcal L(\gamma)+\left.\frac{\dd}{\dd \epsilon}\right|_{\epsilon=0}\mathcal L(\gamma+\epsilon(j^1g^{-1}(\gamma')-\gamma))
\right]\\
&=
\mathcal L(j^1g(\gamma))+\left.\frac{\dd}{\dd \epsilon}\right|_{\epsilon=0}\mathcal L(j^1g(\gamma)+\epsilon(\gamma'-j^1g(\gamma)))\\
&=\langle F_{\mathcal L}(j^1g(\gamma)),\gamma'\rangle
\end{align*}
所以$g_{J^1Y^*}F_{\mathcal L}=F_{\mathcal L}j^1g$,这就是第一点。

对第二点,利用第一点和定义,我们有
\[
	(j^1g)^*\Theta_{\mathcal L}=(j^1g)^*F_{\mathcal L}^*\Theta=F_{\mathcal L}^*g_{J^1Y^*}^*\Theta,
\]
因为正则提升$g_{J^1Y^*}$是特殊正则变换,所以$g_{J^1Y^*}^*\Theta=\Theta$,这就给出了结论。

对第三点,再一次利用第一点(的无穷小版本),令$g=\exp(t\xi)$后再让$t$趋于零,我们立刻得到
\[
	\xi_{J^1Y^*}=(F_{\mathcal L})_{*}\xi_{J^1Y},
\]
所以
\[
	F_{\mathcal L}^*\iota(\xi_{J^1Y^*})=F_{\mathcal L}^*\iota((F_{\mathcal L})_{*}\xi_{J^1Y})=\iota(\xi_{J^1Y})(F_{\mathcal L})^{*}
\]
以及
\[
	 \dd J^\mathcal{L}(\xi)=F_{\mathcal L}^*\dd J(\xi)=F_{\mathcal L}^*\iota(\xi_{J^1Y^*})\Omega=
	 \iota(\xi_{J^1Y})\Omega_{\mathcal L},
\]
\[
	 J^\mathcal{L}(\xi)=F_{\mathcal L}^*\iota(\xi_{J^1Y^*})\Theta=\iota(\xi_{J^1Y})\Theta_{\mathcal L}.
\]
此即所证。
\end{proof}

\begin{thm}[Noether定理]
    如果$\mathcal L$具有$G$-对称性,则对任意的$\xi\in \mathfrak g$,则
    \[
        \dd \left[ (j^1\phi)^*J^{\mathcal L}(\xi)\right]=0
    \]
    对任意满足Lagrange方程的截面$\phi:X\to Y$都成立。
\end{thm}

\begin{proof}
在Lagrange方程Theorem \ref{Lagrangeeq}
\[
	(j^1\phi)^*(\iota(W)\Omega_{\mathcal L})=0
\]
中,$W$是$J^1Y$上任意的矢量场。这里,我们取$W=\xi_{J^1Y}$,于是
\[
	0=(j^1\phi)^*(\iota(\xi_{J^1Y})\Omega_{\mathcal L})=(j^1\phi)^*\dd J^\mathcal{L}(\xi)= \dd \left[ (j^1\phi)^*J^{\mathcal L}(\xi)\right],
\]
此即所需。
\end{proof}

量$(j^1\phi)^*J^{\mathcal L}(\xi)$一般被称为Noether流,或者守恒流。局部地,可以看到
\[
	(j^1\phi)^*J^{\mathcal L}(\xi)=\left(
\frac{\partial L}{\partial v^A_\mu}(j^1\phi)(\xi^A\circ \phi-\partial_\nu \phi^A\xi^\nu)
+L(j^1\phi)\xi^mu
	\right)\dd^{D-1}x_\mu.
\]