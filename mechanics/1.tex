\chapter{Hamilton力学}
\section{辛形式与Hamilton方程组}

\begin{para}
	设$Z$为实Banach空间,设$\Omega:Z\times Z\to \rr$是$Z$上的一个连续双线性型,如果任取$z_2\in Z$都有$\Omega(z_1,z_2)=0$可以推出$z_1=0$,则称$\Omega$为\textit{非退化形式}\index{非退化形式}或\textit{弱非退化形式}\index{非退化形式!弱非退化形式}。于是,$\Omega$将诱导一个连续线性映射$\Omega^\flat:Z\to Z^*$,满足
	\[
		\Omega^\flat(z_1)(z_2)=\Omega(z_1,z_2).
	\]
	很清楚,$\Omega$是非退化的等价于$\Omega^\flat$是单射。若$\Omega^\flat$还是一个满射,则开映射定理保证了$\Omega^\flat$是一个同胚。
	\begin{quote}\it
	开映射定理:如果$X$和$Y$是Banach空间,$A : X \to Y$是一个满的连续线性算子,那么$A$就是一个开映射。
	\end{quote}
	此时$\Omega^\flat$是一个同构,称$\Omega$为\textit{强非退化形式}\index{非退化形式!强非退化形式}。对有限维而言,弱非退化形式与强非退化形式等价,但一般而言,我们所谓的非退化形式是指弱非退化形式。

	定义矢量空间$Z$上的一个\textit{辛形式}\index{辛形式}$\Omega$是一个非退化反对称双线性型,此时$(Z,\Omega)$被称为一个\textit{辛(矢量)空间}\index{辛空间}。
\end{para}

\begin{exa}
	令$V$是一个矢量空间,再令$Z=V\times V^*$,定义$Z$上的\textit{正则辛形式}\index{辛形式!正则辛形式}$\Omega$为
	\[
	\Omega((v,f),(w,g))=g(v)-f(w),
	\]
	其中$v$, $w\in V$而$f$, $g\in V^*$.

	对于有限维空间,由于$\Omega(z,w)=z^iw^j\Omega_{ij}$,所以给出辛形式等价于给出其矩阵$\Omega_{ij}$. 在我们的例子中,正则辛形式的矩阵写作
	\[
	(\Omega_{ij})=
	\begin{pmatrix}
	0&I\\
	-I&0
	\end{pmatrix}.
	\]

	更一般地,如果$V$和$W$中有一个非退化双线性型$D:V\times W\to \rr$. 则可以在$V\times W$上可以定义一个辛形式如下
	\[
	\Omega((v_1,w_1),(v_2,w_2))=D(v_2,w_1)-D(v_1,w_2).
	\]
	它也被称为正则辛形式。

	另一个例子,考虑一个复Hilbert空间$\mathcal{H}$,上面有一个Hermit内积$\langle *,*\rangle:\mathcal{H}\times \mathcal{H}\to \cc$,按照物理上的习惯,第一个变量是反线性的,第二个变量是线性的,同时成立$\langle v,w\rangle=\overline{\langle w,v\rangle}$. 则$\Omega(\psi_1,\psi_2)=\,\mathrm{Im}\, \langle \psi_1,\psi_2\rangle$是一个$\mathcal{H}$上的(强)辛形式。
\end{exa}

\begin{para}
	设$(Z_1,\Omega_1)$和$(Z_2,\Omega_2)$是两个辛空间,而$f:Z_1\to Z_2$是一个连续可微映射,$\dd f$是它的导数,在点$z$处表现为$\dd f(z):Z_1\to Z_2$,这里我们悄悄引入了赋范空间与它切空间的等同。

	记号上,我们经常用$f_*$或者$f'$来代替$\dd f$,相应地,用$f_{*z}$和$f'(z)$来代替$\dd f(z)$.

	如果对每一个$x$, $y$, $z\in Z_1$都有
	\[
		\Omega_2\left(\dd f(z)x,\dd f(z)y\right)=\Omega_1(x,y),
	\]
	则称$f$是一个\textit{辛变换}\index{辛变换},或者叫\textit{正则变换}\index{正则变换}。辛空间连上辛变换自然构成了一个范畴。
\end{para}

\begin{para}
	设$(Z,\Omega)$是一个辛空间,一个矢量场$X:Z\to Z$被称为Hamilton的,如果对每一个$z\in Z$以及某个$C^1$函数$H:Z\to R$使得
	\[
	\Omega^\flat (X(z))=\dd H(z).
	\]
	如果这样一个函数$H$存在,我们称之为Hamilton函数或者Hamiltonian,而$X$会相应记作$X_H$. 

	如果系统是有限维的,则$\Omega^\flat$是同构,故而,给定一个函数$H$就可以定义出一个相应的矢量场
	\[
		X_H(z)=(\Omega^\flat)^{-1}\left(\dd H(z)\right).
	\]

	经常地,在处理无限维Banach空间时,$H$不必在$Z$上整体有定义,而只在一个稠密子空间上有定义,或者更小的空间上有定义。这点比如在处理不同的可微性时候会出现。以后我们不谈这种东西。

	任取$w\in Z$,由Hamilton矢量场的定义,我们有
	\[
	\Omega(X(z),w)=\Omega^\flat (X(z))w=H_{*z}w=\dd H(z)w.
	\]
	如果已知一个Hamilton矢量场$X$,则可以通过
	\[
	H(z)-H(0)=\int_{0}^1 \dd t\,\frac{\dd H(tz)}{\dd t}=\int_{0}^1 \dd t\,\dd H(tz)z=\int_{0}^1 \dd t\,\Omega(X(tz),z)
	\]
	算出对应的Hamilton函数。
\end{para}

\begin{pro}
	设$(Z,\Omega_Z)$和$(Y,\Omega_Y)$是辛空间,而$f:Z\to Y$是一个微分同胚,则$f$是一个辛变换当且仅当对$Y$上的所有Hamilton矢量场$X_H$成立,$f_*X_{H\circ f}=X_{H}$,即
	\[
		X_H(f(z))=f_{*z}X_{H\circ f}(z).
	\]
\end{pro}

实际上,考虑
\[
	\Omega_Z(X_{H\circ f}(z),v)=(H\circ f)_{*z}v=H_{*f(z)}f_{*z}v=\Omega_Y(X_H(f(z)),f_{*z}v).
\]
由于$f$是微分同胚,$f_*$自然处处是线性同构,所以如果$f$是正则变换,则
\[
	\Omega_Y(f_{*z}X_{H\circ f}(z),f_{*z}v)=\Omega_Z(X_{H\circ f}(z),v)=\Omega_Y(X_H(f(z)),f_{*z}v),
\]
由$\Omega_Y$非退化,可知$X_H(f(z))=f_{*z}X_{H\circ f}(z)$. 反之亦然。

\begin{para}
	设$H$是一个Hamilton函数,而$X_H$是他的Hamilton矢量场,则决定$c:\rr \to Z$是否为$X_H$的积分曲线的微分方程
	\[
	\dot c(t)=X_H(c(t))
	\]
	被称为$H$的Hamilton方程组。

	设$c$是$X_H$的一条积分曲线,则$H$在曲线$c(t)$上保持恒定。实际上,直接对$H\left(c(t)\right)$求导有
	\[
	\frac{\dd}{\dd t}H(c(t))=\dd H(c(t))\dot c(t)=\dd H(c(t))X_H\left(c(t)\right)=\Omega\left(X_H\left(c(t)\right),X_H\left(c(t)\right)\right)=0.
	\]
	Hamilton函数一般被认为是能量,则这个小命题即可认为是能量守恒。
\end{para}

\begin{thm}
	设$X:Z\to Z$是一个矢量场,则$X$是一个Hamilton矢量场,当且仅当,对所有的$x$, $y\in Z$成立
	\[
	\Omega(Ax,y)+\Omega(x,Ay)=0,
	\]
	此时$H(z)$可以取作$\Omega(Az,z)/2$.
\end{thm}

\section{Possion括号}