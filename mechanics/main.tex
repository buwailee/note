\documentclass[10pt]{book}

\usepackage[book]{noteheader}
\usepackage[fontset = fandol]{ctex}
	% \CTEXoptions[today=old]
	% \CTEXoptions[contentsname=Table of Contents]
\usepackage{graphicx,wrapfig,paralist}
	\setlength{\pltopsep}{3pt}
	\setlength{\plpartopsep}{3pt}

\usepackage{titletoc}
	\setcounter{secnumdepth}{5}  
	\setcounter{tocdepth}{5} 

\usepackage{makeidx}
	\makeindex

\usepackage{hyperref}
	\hypersetup{bookmarksnumbered=true}

% \renewcommand\thechapter{\Roman{chapter}} 

\definecolor{shadecolor}{rgb}{0.92,0.92,0.92}

\newcommand{\no}[1]{{$(#1)$}}
% \renewcommand{\not}[1]{#1\!\!\!/}
\newcommand{\rr}{\mathbb{R}}
\newcommand{\zz}{\mathbb{Z}}
\newcommand{\aaa}{\mathfrak{a}}
\newcommand{\pp}{\mathfrak{p}}
\newcommand{\mm}{\mathfrak{m}}
\newcommand{\dd}{\mathrm{d}}
\newcommand{\oo}{\mathcal{O}}
\newcommand{\calf}{\mathcal{F}}
\newcommand{\calg}{\mathcal{G}}
\newcommand{\bbp}{\mathbb{P}}
\newcommand{\bba}{\mathbb{A}}
\newcommand{\osub}{\underset{\mathrm{open}}{\subset}}
\newcommand{\csub}{\underset{\mathrm{closed}}{\subset}}

\DeclareMathOperator{\im}{Im}
\DeclareMathOperator{\Hom}{Hom}
\DeclareMathOperator{\id}{id}
\DeclareMathOperator{\rank}{rank}
\DeclareMathOperator{\tr}{tr}
\DeclareMathOperator{\supp}{supp}
\DeclareMathOperator{\coker}{coker}
\DeclareMathOperator{\codim}{codim}
\DeclareMathOperator{\height}{height}
\DeclareMathOperator{\sign}{sign}

\DeclareMathOperator{\ann}{ann}
\DeclareMathOperator{\Ann}{Ann}
\DeclareMathOperator{\ev}{ev}
	\newcommand{\cc}{\mathbb{C}}

% \setCJKmainfont[
% 	ItalicFont = AdobeKaitiStd-Regular ,
% 	BoldFont = SourceHanSerifSC-Bold ,
% ]{SourceHanSerifSC-Regular}
% \renewcommand{\hyp}{\raisebox{.1em}{-}}
% \renewcommand{\textit}[1]{{\raisebox{-.05ex}{\scalebox{1.05}{\it #1}}}}

\theoremstyle{definition}
\newtheorem{para}{}[chapter]
\newtheorem{exa}[para]{Example}
\renewcommand{\thepara}{\thechapter.\arabic{para}}

\theoremstyle{plain}
\newtheorem{thm}[para]{Theorem}
\newtheorem{pro}[para]{Proposition}
\newtheorem{lem}[para]{Lemma}

\begin{document}
\begin{titlepage}
\setcounter{page}{-1}
\thispagestyle{empty}
	\begin{flushright}
	{\Huge\bfseries 经典力学}\\[\baselineskip]
	% {{\scshape In\:}\Large {\itshape a simple} {\scshape Way}} \par
	{by xxx}\par
	\today
	\end{flushright}
	\vfill
	{\Large\itshape 仅作内部交流用}
\end{titlepage}
\clearpage
\thispagestyle{empty}

\frontmatter
\tableofcontents
\mainmatter
	% !TeX root = ./main.tex

History:

\begin{itemize}
    \item [\qquad\qquad 80':] PT-formula \& ??;
    \item [\qquad\qquad 03':] Witten;
    \item [\qquad\qquad 13':] CHY.
\end{itemize}

At Witten, he gave a formula to express tree amplitudes as a 
integral on the world sheet in a specific theory. This formula 
has the form
\[
    \mathcal M(k_1,\dots,k_n)=\int \frac{d^n\sigma}
    {\operatorname{SL}(2,\cc)}\prod_i\delta(h_i(\sigma,k))f(\sigma,k),
\]
where the integral space is the Riemann surface with ...

Now we can introduce a 1-form 
\[
    \omega^\mu(z)=\sum_{a=1}^n \frac{k_a^\mu}{z-\sigma_a}dz
\]
which is useful to express $k_i$ as a integral in ... by
\[
    \frac{1}{2\pi i}\int_{|z-\sigma_a|=\epsilon}\omega^\mu(z)
    =k_a^\mu.
\]

We want $D$ maps $p^\mu:M\to K$ satisfy
\begin{enumerate}
    \item $p^\mu(\sigma_a) \propto k_a$;
    \item $p^\mu(z) p_\mu(z) = 0$;
    \item $p^\mu(z) dz = \prod_a (z-\sigma_a)\omega^\mu(z)$.
\end{enumerate}
\backmatter
 	% \chapter*{References}
\addcontentsline{toc}{chapter}{References}
把参考文献列出来,就不写那么正式了:
\begin{itemize}

\item Frank W.Warner, GTM 94

\item Daniel Bump, GTM 225

\item R.W.Sharpe, GTM 166

\item V.S.Varadarajan, GTM 102

\item S Kobayashi \& K Nomizu, Foundations of Differential Geometry. Vol. I

\item W.Y.Hsiang, Lectures on Lie Groups

\item Shlomo Sternberg, Semi-Riemann Geometry and General Relativity

\item Yvette Kosmann-Schwarzbach, Groups and Symmetries

\item J.F.Cornwell, Group Theory in Physics

\item Steven Weinberg, The Quantum Theory of Fields. Vol. I

\end{itemize}
\printindex
\end{document}