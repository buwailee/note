\chapter{环簇}

\section{仿射环簇}

\para[环面] 如果一个仿射簇同构于$(\cc^*)^n=(\cc-\{0\})^n=(\spec\cc[x,x^{-1}])^n$,且通过结构给出了
上面的一个乘法结构($(\cc^*)^n$上的乘法为$(a_1,\dots,a_n)(b_1,\dots,b_n)=(a_1b_1,\dots,a_n b_n)$),
则称这个仿射簇为一个环面。

\para[特征] 设$T$是一个环面,同态$\chi:T\to \cc^*$若还是一个群同态,则称其为一个特征。实际上,
众所周知,任何一个$(\cc^*)^n\to \cc^*$的特征都形如
\[
    \chi^a(t_1,\dots,t_n)=t_1^{a_1}\cdots t_n^{a_n},
\]
其中$\{a_i\}$都是整数。于是,所有特征构成的群就同构于加法群$\mathbb Z^n$. 对任意环面$T$,其特征构成的群
也应该是一个自由交换群$M$,他的秩与$T$的维数相同。

\para[单参子群] 对环面$T$,同态$\lambda:\cc^*\to T$若还是一个群同态,则称其为一个单参子群。
任何一个$(\cc^*)^n$上的单参子群同样由一个$\mathbb Z^n$中的向量给出,即
\[
    \lambda^b(t)=(t^{b_1},\dots,t^{b_n}).
\]
实际上,对每个分类,$\cc^*$上的一个特征。对任意环面$T$,其单参子群构成的群
也应该是一个自由交换群$N$,他的秩与$T$的维数相同。

如果环面$T$的单参子群构成的群为$N$,我们通常会用$T_N$来标记$T$.

\para 给定环面$T$,其特征构成群$M$,而单参子群又构成群$N$,那么他们之间在一个自然的配对
$\langle ,\rangle:M\times N\to \mathbb{Z}$. 他是这样定义的:任取特征$\chi^a$和单参子群$\lambda^b$,
那么复合$\chi^a\circ\lambda^b:\cc^*\to \cc^*$就是$\cc^*$上的一个特征,由$t\to t^l$给出,此时定义
$\langle a,b\rangle=l$. 如果应用到环面$(\cc^*)^n$上,很容易看到,
$\langle a,b\rangle=\sum_{i}a_ib_i$就是寻常的内积。

\para[格与仿射半群] 这里的格被定义为有限秩的自由交换群,显然环面$T_N$有两个自然的格,一个是$N$,一个是特征
构成的$M$. 一个半群如果是交换、有限生成且可以被嵌入到一个格中,则称其为仿射半群。给定一个
交换半群$S$,我们可以定义相应的交换半群代数
\[
    \cc[S]=\biggl\{\sum_{s\in S}c_s \chi^s\biggr\},
\]
其中非零$c_s\in \cc$有限,乘法定义为$\chi^s\chi^t=\chi^{s+t}$.

不难看到,则$\cc[S]$是一个整环,且作为$\cc$-代数有限生成。容易检查,
$\cc[\mathbb N^n]=\cc[x_1,\dots,x_n]$, $\cc[\mathbb Z^n]=\cc[x_1,x_1^{-1},\dots,x_n,x_n^{-1}]$. 
后者就是环面的坐标环。一般地,给定一个格$M$,$\spec \cc[M]$将是一个环面。

\para[仿射环面] 令$S\subset M$是一个交换半群,其中$M$是一个格,格$\mathbb ZS$的秩为$s$,则$\spec \cc[S]$被称为一个仿射环面,
其包含一个环面$\spec \cc[S]\cap (\cc^*)^s$作为其Zariski开子集,且这个环的特征格为$\mathbb ZS\subset M$.

假设$S$由一组基$\mathscr A=\{m_1,\dots,m_s\}$生成,那么我们有同态
\[
    \varphi:\cc[x_1,\dots,x_s]\to \cc[M]
\]
这里的$x_i$被映射为$\chi^{m_i}$,于是这给出了概型同态
\[
    \Phi_{\mathscr A}:T_N\to \cc^s,
\]
其具体到点上的行为为
\[
    \Phi_{\mathscr A}(t)=(\chi^{m_1}(t),\dots,\chi^{m_s}(t))\in \cc^s.
\]

现在,我们研究$\Phi_{\mathscr A}$的像,他的闭包是由那些满足$f(\chi^{m_1},\dots,\chi^{m_s})=0$
的多项式$f\in \cc[x_1,\dots,x_s]$定义的,即他的闭包对应的理想是$\ker\varphi$. 同时,我们也注意到
$\varphi$的像就是$\cc[S]$,所以
\[
    \cc[S]\cong \cc[x_1,\dots,x_s]/\ker \varphi
\]
告诉我们$\spec \cc[S]$就是$\Phi_{\mathscr A}$的像的Zariski闭包。下面我们来刻画这个理想$\ker\varphi$.
这当然是一个素理想,因为$\cc[S]$是一个整环。

从格的角度出发,虽然$\mathscr A$的元素个数为$s$,但是$\mathbb Z\mathscr A$的秩不一定为$s$,
这是因为可能存在$\mathbb Z^s$的子格$L$,其中的元素$(l_1,\dots,l_s)$满足$\sum_i l_im_i=0$. 换言之,
即满足短正和列$0\to L\to Z^s\to M$. 将任意的$l\in L$分成正部和负部,即$l=l_+-l_-$,其中$l_+$, $l_-\in \mathbb N^s$,而
\[
    l_+=\sum_{l_i>0}l_ie_i,\quad l_-=-\sum_{l_i<0}l_ie_i.
\]
则上面说的理想$\ker \varphi$可以写作
\[
    \ker\varphi=\langle x^{l_+}-x^{l_-}\,:\, l\in L\rangle=
    \langle x^{a}-x^{b}\,:\, a,b\in \mathbb N^s,\, a-b\in L\rangle.
\]
第二个等号的正确性来自于,如果$a-b=l$,其中$a,b\in \mathbb N^s$,则存在唯一的$c\in \mathbb N^s$
使得$a=c+l_+$, $b=c+l_-$. 此外,
$\langle x^{l_+}-x^{l_-}\,:\, l\in L\rangle\subset \ker\varphi$也是直接的,因为
\[
    \chi^{\sum_{l_i>0} m_il_i}-\chi^{-\sum_{l_i<0} m_il_i}=
    \chi^{\sum_{l_i>0} m_il_i}-\chi^{\sum_{l_i>0} m_il_i}=0.
\]
这里不证明反过来的包含,见教科书。

最后,我们给出可能是最“粗糙”的刻画:仿射环面$V$是一个仿射包含一个环面$T_N$作为其Zariski开子集的仿射簇,
且要求上面的群作用可以延拓到整个$V$上,即群作用$T_N\times T_N\to T_N$可以延拓到$T_N\times V\to V$.

实际上,令$M$为$T_N$的特征格,包含$T_N\subset V$给出了坐标环间的单同态
$\cc[V]\to \cc[M]$,因为$T_N$在$V$里面稠密。现在考虑群作用$T_N\times V\to V$,
任取$t\in T_N$以及$f\in \cc[V]$,则$p\mapsto f(t^{-1}\cdot p)$依然是$V$上的一个
函数,记作$t\cdot f$,还是在坐标环$\cc[V]$中。我们断言,
\[
    \cc[V]=\bigoplus_{\chi^m\in \cc[V]}\cc\cdot\chi^m.
\]
为证明这点,首先注意到右边(记作$A$)肯定包含于$\cc[V]$中。现在取$f\in \cc[V]$,
由于$f\in \cc[M]$,所以可以写成
\[
    f=\sum_{m\in\mathscr A}c_m \chi^m,
\]
其中$\mathscr A$是一个有限集。现在,容易看到环面的特征作为函数,在群作用下就变成
简单的相乘$t\cdot \chi^m=\chi^{m}(t^{-1})\chi^m$,所以向量空间
$B:=\operatorname{span}(\{\chi^m\,:\, m\in \mathscr A\})\cap \cc[V]$是在
$T_N$作用下稳定的,超不出去。
而且由于$B$是有限维的,且$T_N$是交换的,所以$T_N$的作用可以同时对角化,且$B$被
他们的共同特征向量张成。由于他们也在$\cc[M]$中,这些特征向量就是$T_N$的特征,
于是$B$由特征张成。于是,从$f\in B$说明了对每个$m\in \mathscr A$都有
$\chi^m\in \cc[V]$,故$f\in A$.

于是,$\cc[V]=\cc[S]$,其中$S=\{m\in M\,:\, \chi^m\in \cc[V]\}$是一个半群。由于
$\cc[V]$是有限生成的,而上面任意的函数也都可以展开为特征,所以可以由特征有限
生成,这说明$S$是仿射半群。

\begin{pro}
    设$V=\spec(\cc[S])$为一个仿射环簇,$S$是一个仿射半群,则$V$的每个闭点和
    $S\to \cc$的半群同态一一对应,这里将$\cc$看成按复数乘法构成的半群。
\end{pro}

\begin{proof}
    任取闭点$p\in V$,我们通过$m\mapsto \chi^m(p)$定义$S\to \cc$. 反过来,令$\gamma$
    是一个半群同态,那么其可以诱导一个线性映射$\cc[S]\to \cc$. 这也是一个
    $\cc$-代数满同态,他的核是个极大理想给出了$V$中的闭点。
\end{proof}

由于$\spec(\cc[S])$的闭点也完全决定了整个仿射概形,所以可以认为$S\to \cc$的半群
同态决定了整个$\spec(\cc[S])$. 

利用这种刻画,我们也很方便可以描述环面在仿射环簇上的作用。设$p\in \spec(\cc[S])$
为一个闭点,对应于半群同态$\gamma:m\mapsto \gamma(m)$. 那么,作用后的点
$t\cdot p$,就对应于半群同态$m\mapsto \chi^m(t)\gamma(m)$.

\section{锥}

\para[锥] 在$\rr$-向量空间$N_\rr$中的一个凸多面锥由一个有限点集$S\subset N_\rr$
给出
\[
    \operatorname{Cone}(S)=\biggl\{\sum_{u\in S}\lambda_u u\,:\,
    \lambda_u\geq 0\biggr\},
\]
以及设$\operatorname{Cone}(\varnothing)=\{0\}$. 一个锥可以理解为一族穿过原点的
半平面的交,而这些半平面中的平面将是这个锥的面,不同面的交也是面,这样我们就有了
不同维度的面。这些都是非常直观的。

\para[对偶锥] 我们还是用$M_\rr$来表示$N_\rr$的对偶空间,配对为$\langle,\rangle$,
记$\sigma$为$N_\rr$中的一个锥,则
\[
    \sigma^{\vee}=\{m\in M_\rr\,:\,\langle m,u\rangle\geq 0\text{ for all $u\in \sigma$}\}.
\]
以及设$\operatorname{Cone}(\varnothing)=\{0\}$. 这是对偶空间中的锥。

\para[强凸性] 所谓强凸性,即不包含任何的正维数子空间。对于锥$\sigma$而言,
就是为了保证我们考虑的不是二维的半平面,他有很多等价的表述,如
$\sigma\cap(-\sigma)=\{0\}$为$\sigma$的一个面,再如$\dim \sigma^\vee=n$.

\para[有理锥] 如果说一个$N_\rr$中锥是有理的,就是说他可以写成
$\operatorname{Cone}(S)$,其中$S\subset N$.

\para 一个强凸的有理锥$\sigma$可以由其边(一维的面)上的向量生成,
这被称为其极小生成集。此时,如果极小生成集构成了$N$的一组$\mathbb Z$-基,
则称$\sigma$是光滑(或正则的);如果极小生成元在$\mathbb R$-上线性独立,则称
$\sigma$是单纯的。

\para 给定有理锥$\sigma\subset N_\rr$,对偶锥的格点
$S_\sigma=\sigma^\vee\cap M\subset M$将是一个半群,进一步地,
这还是一个仿射半群(Gordan引理)。

\begin{pro}
令有理锥$\sigma\subset N_\rr$的半群为$S_\sigma$,则
\[
    U=\spec(\cc[S_\sigma])
\]
是一个仿射环簇,且$\dim U_\sigma=n$当且仅当$\sigma$是强凸的当且仅当
$T_N=N\otimes_\zz\cc^*$.
\end{pro}

证明的关键是考虑$U$的环簇的特征格,即$\mathbb ZS_\sigma$. 利用对偶锥的凸性,
我们可以证明,如果$km\in \mathbb ZS_\sigma$,则$m\in \mathbb ZS_\sigma$,其中$k$
为任意的正整数,所以$M/\zz S_\sigma$无挠。因此,$\mathbb ZS_\sigma$满秩当且仅当
他就是$M$,当且仅当$U$的环面就是$T_N$.