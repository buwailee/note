%!TEX program = xelatex
\documentclass[9pt]{extbook}
\usepackage{ctex}
	\CTEXoptions[today=old]
	\CTEXoptions[contentsname=Table of Contents]
\usepackage[zh,book]{noteheader}

\usepackage{titletoc}%使用目录
	\theoremstyle{plain}%定理环境样式
	\newtheorem{pro}{Proposition}[section]% 定义命题环境
	\newtheorem{theo}{Theorem}[section]% 定义定理环境
	\newtheorem*{rem}{Remark}% 定义注记环境
	\newtheorem{defi}{Definition}[section]% 定义定义环境
	\newtheorem{exa}{Example}[section]% 定义例子环境

\usepackage{hyperref}%使用xetex引擎
	\hypersetup %一些选项
{
	pdftoolbar=true,  % 显示Acrobat工具栏
	pdfmenubar=true,  % 展开Acrobat目录
	pdftitle={QFT},  % pdf题目,自己填z
	pdfauthor={Unsinn},  % pdf作者,自己填
	bookmarksnumbered=true,%书签中章节编号
	bookmarksopen=true,%目录层次打开
	bookmarksopenlevel=1,%目录层次打开的级数,可选数字或者 \maxdimen最大
	pdfsubject={Physics},  % 主题,自己填
	pdfkeywords={QFT}, % 关键字,自己填
}

%定义你的命令
\definecolor{shadecolor}{rgb}{0.92,0.92,0.92}

\newcommand{\no}[1]{{$(#1)$}}
% \renewcommand{\not}[1]{#1\!\!\!/}
\newcommand{\rr}{\mathbb{R}}
\newcommand{\zz}{\mathbb{Z}}
\newcommand{\aaa}{\mathfrak{a}}
\newcommand{\pp}{\mathfrak{p}}
\newcommand{\mm}{\mathfrak{m}}
\newcommand{\dd}{\mathrm{d}}
\newcommand{\oo}{\mathcal{O}}
\newcommand{\calf}{\mathcal{F}}
\newcommand{\calg}{\mathcal{G}}
\newcommand{\bbp}{\mathbb{P}}
\newcommand{\bba}{\mathbb{A}}
\newcommand{\osub}{\underset{\mathrm{open}}{\subset}}
\newcommand{\csub}{\underset{\mathrm{closed}}{\subset}}

\DeclareMathOperator{\im}{Im}
\DeclareMathOperator{\Hom}{Hom}
\DeclareMathOperator{\id}{id}
\DeclareMathOperator{\rank}{rank}
\DeclareMathOperator{\tr}{tr}
\DeclareMathOperator{\supp}{supp}
\DeclareMathOperator{\coker}{coker}
\DeclareMathOperator{\codim}{codim}
\DeclareMathOperator{\height}{height}
\DeclareMathOperator{\sign}{sign}

\DeclareMathOperator{\ann}{ann}
\DeclareMathOperator{\Ann}{Ann}
\DeclareMathOperator{\ev}{ev}
\newcommand{\re}[1]
	{\begin{center}
		\colorbox{shadecolor}{
		\begin{minipage}{135mm}
				\emph{``#1''}
			\end{minipage}}
	\end{center}}

\begin{document}
\frontmatter
\thispagestyle{empty}
\begin{flushright}
{\Huge\bfseries Quantum Field Theory}\\[\baselineskip]
{{\scshape translate\:}\Large {\itshape from} {\scshape Weinberg}} \par
{by DaDouBi@Shanghai Nanyang Model High School}\par
\today
\end{flushright}
\vfill
{\Large\itshape Just for fun}
\clearpage
这其实就是抄书(Steven Weinberg的The Quantum Theory of Fields),看不大懂所以抄起来了2333,@小平邦彦。
\clearpage
\tableofcontents
\mainmatter
\addtocounter{chapter}{1}
\chapter{Relaticistic Quantum Mechanics}
\re{Quantun field theory is the way it is because (with certain qualifications) this is the only way to reconcile quantum mechanics with special relativity.}
\section{Quantum Mechanics}
量子场论基于量子力学,所以共用着一些公理:

\no{1} 物理上的态被表征为Hilbert空间$\mathcal{H}$的矢量,更准确地说,是确定到一个等价类上的。在物理上,相差一个$e^{i\varphi}$因子并没有区别。因此,我们将所有这些等价的态矢量构成一个等价类,则实际上,态矢量应该指这一个等价类。但在具体处理和计算的时候,使用原空间还是方便的,所以我们还是使用态矢量来称呼$\mathcal{H}$的矢量,而对应的等价类称其为一个ray。

Hilbert空间是复数域上的矢量空间,且赋予了内积。更多的数学细节这里无视掉。假设以$\Phi$和$\Psi$表示两个态矢量,则他们的内积写作$(\Phi,\Psi)$,满足如下性质\footnote{用$a^*$表示$a$的复共轭。}
\[
(\Phi,\Psi)=(\Psi,\Phi)^*,
\]
以及对第二个变量线性
\[
(\Phi,\xi_1\Psi_1+\xi_2\Psi_2)=\xi_1(\Phi,\Psi_1)+\xi_2(\Phi,\Psi_2),
\]
结合上面两个性质,很容易证明内积对第一个变量反线性,即
\[
(\eta_1\Phi_1+\eta_2\Phi_2,\Psi)=\eta_1^*(\Phi_1,\Psi)+\eta_2^*(\Phi_2,\Psi).
\]
此外,对任意的$\Psi$,内积还满足$(\Psi,\Psi)\geq 0$,等号当且仅当$\Psi=0$的时候取到。

\no{2} 可观察量被表征为Hermite算子$A:\mathcal{H}\to\mathcal{H}$,即其为满足$A=A^{\dag}$的线性算子。可观测量的本征值即其所能取的值,对应的本征矢量即他取到值所在的态。

\no{3} 现在如果一个态被表征为$\mathscr{R}$,然后测试其是否处于一系列正交的ray中,即在$\{\mathscr{R}_i\}$中。发现他在$\mathscr{R}_n$中的概率为
\[
P(\mathscr{R}\to\mathscr{R}_n)=|(\Psi,\Psi_n)|^2,
\]
其中$\Psi\in \mathscr{R}$以及$\Psi_n\in \mathscr{R}_n$。虽然上面$\Psi$和$\Psi_n$是任取的,但很容易验证这里的概率是良定义的。这是因为相差的两个$e^{i\varphi}$因子由内积的线性性和外面的模$|e^{i\varphi}|=1$给消掉了。

此外,我们所有概率的和应该是1,所以
\[
\sum_n P(\mathscr{R}\to\mathscr{R}_n)=1.
\]

一个态矢量$\Psi$的波函数我们记作$\Psi(\sigma)$,其中$\sigma$是完全可以确定这个系统的物理量们,将其对应的正交归一本征矢记作$\Psi_\sigma$,波函数即态矢量对这个本征矢的展开系数,即$\Psi(\sigma)=(\Psi_\sigma,\Psi)$,
如果用Dirac符号,$\Psi_\sigma$记作$|\sigma\rangle$,则
$
	\Psi(\sigma)=\langle \sigma|\Psi\rangle.
$

\section{Symmetries}
\re{A symmetry transformation is a change in our point of view that does not change the results of possible experiments.}

假设现在有观测者$O$和$O'$和一个相同的系统,$O$他们观测到的ray是$\mathscr{R}$和$\{\mathscr{R}_i\}$,对$O'$则是$\mathscr{R}'$和$\{\mathscr{R}'_i\}$。那么至少应该有
\[
P(\mathscr{R}\to\mathscr{R}_n)=P(\mathscr{R}'\to\mathscr{R}'_n).
\]

所有变换的集合构成一个群,群乘法即复合。变换群一般不是交换群。

很容易看出,物理上的一个变换$\mathrm{P}:\mathscr{R} \mapsto \mathscr{R}'$应该可以引出$\mathcal{H}$上的一个变换$U(\mathrm{P})$。Wigner证明了,任何一个ray的变换,我们都可以定义一个$\mathcal{H}$上的一个变换$U$(我们称其为表示),他或者是线性且幺正的,或者是反线性反幺正的。在物理上,大部分的变换都是前者。

考虑平凡的情形,对于一个恒等变换$\mathrm{I}:\mathscr{R} \mapsto \mathscr{R}$,我们自然就(想要)对应于$\mathcal{H}$上的恒等变换换$U(\mathrm{I})=1:\Psi\mapsto \Psi$.此外,当我们考虑连续(默认为光滑的)变换的时候,我们就自然需要面对连通Lie群,连通Lie群的大部分性质只依赖于其单位元附近的元素,这就是说,我们可以考虑无穷小变换$U=1+i\epsilon t$来研究Lie群的性质,其中$\epsilon$为小的正实数,而$t$即$U$的Lie代数。

现在譬如有两个变换$\mathrm{T}_1:\mathscr{R}\mapsto \mathscr{R}'$和$\mathrm{T}_2:\mathscr{R}'\mapsto \mathscr{R}''$,他们的复合为$\mathrm{T}_2\mathrm{T}_1:\mathscr{R}\mapsto \mathscr{R}''$,那么对应的表示为$U(\mathrm{T}_1)$和$U(\mathrm{T}_2)$以及$U(\mathrm{T}_2\mathrm{T}_1)$之间的关系一般为
\[
U(\mathrm{T}_2)U(\mathrm{T}_1)\Psi=e^{i\psi(\mathrm{T}_2,\mathrm{T}_1)}U(\mathrm{T}_2\mathrm{T}_1)\Psi,
\]
$e^{i\psi(\mathrm{T}_2,\mathrm{T}_1)}$的出现是因为变换是ray之间的传递。

如果我们可以制备任意两个态的和,那么可以证明$e^{i\psi(\mathrm{T}_2,\mathrm{T}_1)}$是无关于具体的$\Psi$的(确实存在不能制备的任意两个态)。在这种情况下,表示的复合就可以写作
\[
U(\mathrm{T}_2)U(\mathrm{T}_1)=e^{i\psi(\mathrm{T}_2,\mathrm{T}_1)}U(\mathrm{T}_2\mathrm{T}_1),
\]
这种表示我们称为射影表示。

可以证明,适当扩大变换群,譬如从$\mathrm{SO}(3)$变到$\mathrm{SU}(2)$,我们总可以将射影表示中的因子$e^{i\psi(\mathrm{T}_2,\mathrm{T}_1)}$消去,这就是说我们有表示的复合关系如下:
\[
U(\mathrm{T}_2)U(\mathrm{T}_1)=U(\mathrm{T}_2\mathrm{T}_1),
\]
这就是数学书里面可以看到的一般的表示,即线性表示。以后除非特别声明,我们所谓的表示都是指线性表示,而不是更广义的射影表示。

对于连通Lie群,我们可以用有限的参数表示。譬如两族参数$\theta=\{\theta^a\}$和$\overline{\theta}=\{\bar{\theta}^b\}$,那么就有
\[
\mathrm{T}(\bar{\theta})\mathrm{T}(\theta)=\mathrm{T}(f(\bar{\theta},\theta)),
\]
这里$f(\theta,\bar{\theta})$为光滑函数。适当调整参数,我们可以使得参数都为0的时候,对应的$\mathrm{T}(0)$为恒等映射。那么就应该有
\[
f^a(\theta,0)=f^a(0,\theta)=\theta^a.
\]

现在我们在单位元附近展开$U(\mathrm{T}(\theta))$和$f^a$并保留到二阶(注意下面的$t_a$都是算子)
\[
U(\mathrm{T}(\theta))=1+i\theta^at_a+\frac{1}{2}\theta^b\theta^ct_{bc},
\]
\[
f^a(\overline{\theta},\theta)=\theta^a+\bar{\theta}^a+f^a_{\phantom{a}bc}\bar{\theta}^b\theta^c,
\]
将其应用到
\[
U(\mathrm{T}(\overline{\theta}))U(\mathrm{T}(\theta))=U(\mathrm{T}(f(\overline{\theta},\theta))),
\]
可以计算得
\[
t_{bc}=-t_bt_c-if^a_{\phantom{a}bc}t_a.
\]
所以一旦给定了群的结构,即$f$,那么我们就可以从一阶项系数求出二阶项系数。

因为在
\[
U(\mathrm{T}(\theta))=1+i\theta^at_a+\frac{1}{2}\theta^b\theta^ct_{bc}
\]
中$b,c$指标任意,所以$t_{bc}=t_{cb}$.因此,可以改写$t_{bc}=-t_bt_c-if^a_{\phantom{a}bc}t_a$为
\[
t_bt_c+if^a_{\phantom{a}bc}t_a=t_ct_b+if^a_{\phantom{a}cb}t_a,
\]
或者
\[
[t_b,t_c]=t_bt_c-t_ct_b=if^a_{\phantom{a}cb}t_a-if^a_{\phantom{a}bc}t_a=iC^a_{\phantom{a}bc}t_a,
\]
其中$C^a_{\phantom{a}bc}=f^a_{\phantom{a}cb}-f^a_{\phantom{a}bc}$,即所谓的结构常数。

考虑最简单的例子,$f^a(\theta,\overline{\theta})=\theta^a+\bar{\theta}^a$,那么$C^a_{\phantom{a}bc}=0$,因此$[t_b,t_c]=0$,此时
\[
U(\mathrm{T}(\theta))=\left(U\left(\mathrm{T}\left(\frac{\theta}{N}\right)\right)\right)^N=\lim_{N\to \infty}\left(1+\frac{i}{N}\theta^at_a\right)^N=e^{i\theta^at_a}.
\]
\section{Quantum Lorentz Transformations}
这里稍稍复习一下狭义相对论,其实主要是指标规则。

我们约定在使用坐标或者矩阵表示的时候,第四个分量才是第零分量,譬如$(1,1,1,-1)$是指第零分量为$-1$.此外$\eta_{\mu\nu}$和$\eta^{\mu\nu}$在数值上都是$\mathrm{diag}(1,1,1,-1)$,这俩还用作提升或者下降指标,譬如
\[
\eta_{\mu\nu}a^\nu=a_\mu,\quad \eta^{\mu\nu}A_\mu^{\phantom{\mu}\sigma}=A^{\nu\sigma}.
\]
从矩阵的数值上来看,提升或者下降第一个指标代表最后一行全部变号,提升或者下降第二个指标代表最后一列全部变号。

一般而言,我们并不能直接观测到数学上所谓的和坐标系(物理上就变成参考系了)选取无关的矢量,物理上可测的都是分量,然后判别矢量与否需要通过看这些分量是否在参考系变换下符合相应规则。所以狭义相对论里面记矢量一般为$x^\mu$,即其$x$的分量形式,他在不同参考系下的值是不同的。而Lorentz变换在这之上的表现就是矩阵,或者说坐标变换$\Lambda^\mu_{\phantom{\mu}\nu}$,就是说矢量在另一个参考系下$x$的坐标变成了$\Lambda^\mu_{\phantom{\mu}\nu}x^\nu$.注意$\Lambda$作为坐标变换和参考系变换的意义是不同的,这两者不是对应关系,坐标变换是作用在矢量的分量上面的,或者说反变矢量上面的,而参考系变换是作用在矢量的基上面的,或者说协变矢量上面的。所以当参考系变换为$\Lambda^{-1}$的时候,坐标变换为$\Lambda$.或者反过来,当参考系变换为$\Lambda$的时候,坐标变换为$\Lambda^{-1}$,在这个意义上,矢量$p^\mu$在参考系经过两次变换$\Lambda_1\Lambda_2$后就是
\[
(\Lambda_1^{-1})^\mu_{\phantom{\mu}\rho}(\Lambda_2^{-1})^\rho_{\phantom{\rho}\nu}p^\nu=((\Lambda_2\Lambda_1)^{-1})^\mu_{\phantom{\mu}\nu}p^\nu,
\]
这就是所谓的反变矢量。指标在下面的就是协变矢量,对应的是基,所以一般我们所说的矢量就是反变矢量。

从线性空间的知识可以得到,坐标和基之间存在对偶关系,对应的是空间和对偶空间的关系,从指标上来看就是指标的上下区别。通过空间和对偶空间的语言,我们可以形式化上面的陈述。但这里我们就略去了。

这里设选定的参考系为$\mathcal{O}$,我们记所有物理上的主动的操作$\mathcal{O}\to \mathcal{O}$的变换构成的空间为$G(\mathcal{O})$。那么参考系变换后的参考系为$\Lambda \mathcal{O}$,值得注意的是这里$\mathcal{O}$只是一个形式的记号,他表示变换后的参考系,此时坐标变换为$\Lambda^{-1}$。虽然可以理解成$\mathcal{O}$是一个标架,这种时候$\Lambda$就可以确实地表现为一个变换,但这里我们不采用这种做法。

此时$G$就是一个函子,它引入了变换空间间的映射$G(\Lambda):G(\mathcal{O})\to G(\Lambda\mathcal{O})$。作为例子,我们缩小到$L(\mathcal{O})$,即所有线性变换构成的空间。这种情况下,存在一个自然的反变函子,即满足$L(\Lambda_1)L(\Lambda_2)=L(\Lambda_2 \Lambda_1)$的函子:假设现在有一个线性变换$P$,相似变换$L(\Lambda)P=\Lambda^{-1} P \Lambda$就是一个反变函子。这在物理上来看就是相同的算子在不同的参考系内对应的不同矩阵,这就是说,如果算子$P$在$\mathcal{O}$下的矩阵是$P^\mu_{\phantom{\mu}\nu}$,那么算子在$\Lambda\mathcal{O}$下的矩阵是
\[
(\Lambda^{-1} P\Lambda)^\mu_{\phantom{\mu}\nu}=(\Lambda^{-1})^\mu_{\phantom{\mu}\rho}P^\rho_{\phantom{\rho}\sigma}\Lambda^\sigma_{\phantom{\sigma}\nu}.
\]
对应于反变函子的是协变函子,他符合$L(\Lambda_1)L(\Lambda_2)=L(\Lambda_1 \Lambda_2)$,譬如满足$L(\Lambda)P=\Lambda P \Lambda^{-1}$的伴随就是协变函子。

% 由于习惯上我们使用的情况是坐标变换为$\Lambda$,上面进行论证时候用的是$\Lambda^{-1}$,所以,对于矩阵的相似变换,我们通常用的是$\Lambda P\Lambda^{-1}$.以后谈及反变或者协变,我们所指的是对参考系变换而言的,如果是对坐标变换而言,$\Lambda P\Lambda^{-1}$是协变的。

线性空间的例子很重要,虽然在现实生活中线性操作并不常见,但是在态矢所在的空间,物理算符可都是线性变换。在这种情况下,我们不再处理$\Lambda$直接作用在动量或者其他矢量上,而是处理他的表示$U(\Lambda)$作用在态矢上。设$P$是Hilbert空间$\mathcal{H}$上面的一个线性变换,我们考察下面这样形式上是协变的伴随
\[
	L(\Lambda)P=U(\Lambda) P U(\Lambda^{-1})=U(\Lambda) P U^{-1}(\Lambda).
\]
他是坐标变换构成的群的一个线性表示,即对于变换群的群元$\Lambda_1,\Lambda_2$满足
\[
	L(\Lambda_1\Lambda_2)=L(\Lambda_1)L(\Lambda_2),
\]
这就是伴随表示\footnote{在数学上,Lie群$G$的伴随和他的Lie代数$\mathfrak{g}$的联系来自于伴随$\bm{Ad}(g):h\mapsto ghg^{-1}$在单位元上的导数$\mathrm{Ad}_g=\bm{Ad}(g)_*:T_eG\to T_eG$,但注意到Lie代数$\mathfrak{g}$就是Lie群在单位元的切空间$T_eG$,所以$\mathrm{Ad}_g\in \mathrm{Aut}(\mathfrak{g})$。我们将$\mathrm{Ad}:G\to \mathrm{Aut}(\mathfrak{g})$称为Lie群的伴随表示。}。

% 我们注意,$\mathcal{H}$是所有态矢量的空间,他不区分所选取的参考系,所以说,我们这里也不再把$L(\Lambda)P=U(\Lambda) P U(\Lambda^{-1})$当成同一个算子的不同矩阵,而直接当做物理上等价的不同算子。

下面我们可能会遇到一族线性算子$P^\mu$,他们被称为矢量算子,譬如动量的分量构成的一族算子,下面考察一下他们坐标变换下的表现。

首先,我们在$O$系里面看到$P^\mu$的本征矢$\Psi$,而这个时候我们在$O'$系中看到态矢量$U(\Lambda)\Psi$,如果$P^\mu$对于$\Psi$的本征值是$p^\mu$的话,那么我们自然希望$U(\Lambda)\Psi$也是$P^\mu$的本征矢,且本征值为$\Lambda^\mu_{\phantom{\mu}\nu}p^\nu$,即
\[
	P^\mu U(\Lambda)\Psi=\Lambda^\mu_{\phantom{\mu}\nu}p^\nu U(\Lambda)\Psi,
\]
或者写作
\[
	U(\Lambda)^{-1}P^\mu U(\Lambda)\Psi=\Lambda^\mu_{\phantom{\mu}\nu}p^\nu \Psi=\Lambda^\mu_{\phantom{\mu}\nu}P^\nu \Psi.
\]
同样的思路推而广之,我们如下定义:如果一族矢量$P^\mu$在伴随下满足
\[
	U(\Lambda) P^\mu U^{-1}(\Lambda)=(\Lambda^{-1})^\mu_{\phantom{\mu}\nu}P^\nu,
\]
那么我们就称呼这一族线性变换$P^\mu$是一个矢量(算子)。假如我们有两个变换$\Lambda_1$和$\Lambda_2$,那么有
\[
	U(\Lambda_2)U(\Lambda_1) P^\mu U^{-1}(\Lambda_1)U^{-1}(\Lambda_2)=(\Lambda_1^{-1})^\mu_{\phantom{\mu}\nu}(\Lambda_2^{-1})^\nu_{\phantom{\mu}\sigma}P^\sigma,
\]
左边等于$U(\Lambda_2\Lambda_1) P^\mu U^{-1}(\Lambda_2\Lambda_1)$,右边等于$\left((\Lambda_2\Lambda_1)^{-1}\right)^\mu_{\phantom{\mu}\nu}P^\nu$,于是
\[
U(\Lambda_2\Lambda_1) P^\mu U^{-1}(\Lambda_2\Lambda_1)=\left((\Lambda_2\Lambda_1)^{-1}\right)^\mu_{\phantom{\mu}\nu}P^\nu,
\]
可以看到这样的构造使得在Lorentz变换下,矢量算子的变换规则确实满足一种表示。

有了上面这个例子,我们可以反过来定义一般的变换规则,当一族量$\{A_l\}$满足
\[
	U(\Lambda) A_l U^{-1}(\Lambda)=\sum_{m}D_{lm}(\Lambda^{-1})A_m,
\]
的时候,称呼他是Lorentz不变的\footnote{或者说满足Lorentz不变性。},其中$D$是Lorentz群的一个表示,$D_{lm}$是他的矩阵。这个表示可以是标量表示$D(\Lambda)=1$,可以是上面演示的矢量表示,$D(\Lambda)^\mu_{\phantom{\mu}\nu}=\Lambda^\mu_{\phantom{\mu}\nu}$,还可以是其他表示,比如张量表示,此时二阶张量(算子)就是满足
\[
	U(\Lambda) J^{\mu \nu}U^{-1}(\Lambda)=(\Lambda^{-1})^\mu_{\phantom{\mu}\rho}(\Lambda^{-1})^\nu_{\phantom{\nu}\sigma}J^{\rho \sigma}.
\]

我们一般处理的Lorentz变换可能不是线性的,他将会多包含一个平移(注意,他们的表示一定是线性的),即一个一般的Lorentz变换$\mathrm{T}(\Lambda,a)$在坐标上表现为
\[
\bar{x}^\mu=\Lambda^\mu_{\phantom{\mu}\nu}x^\nu+a^\mu,
\]
$\Lambda$是不含平移的Lorentz变换,而$a$是平移。那么,Lorentz变换的复合规则可以写作
$\mathrm{T}(\Lambda',a')\mathrm{T}(\Lambda,a)=\mathrm{T}(\Lambda'\Lambda,a'+\Lambda'a)$.
这时候他们的表示为
\[U(\Lambda',a')U(\Lambda,a)=U(\Lambda'\Lambda,a'+\Lambda'a).\]

所有的Lorentz变换构成一个群,我们称为Poincar\'{e}群。下面我们来看看$\Lambda$的一些性质,或者说,他的矩阵的一些性质,这些性质直接来自于狭义相对论的基本假设,这里不详细描述。

在Lorentz变换下间隔是不变量,即$\eta_{\mu\nu}\dd x^\mu\dd x^\nu=\eta_{\mu\nu}\dd \bar{x}^\mu\dd \bar{x}^\nu$,那么我们就可以得到一个Lorentz变换应该满足
\[
\eta_{\mu\nu}\Lambda^\mu_{\phantom{\mu}\rho}\Lambda^\nu_{\phantom{\nu}\sigma}=\eta_{\rho\sigma}.
\]

所以说不含平移的Lorentz变换是一种正交群,我们称为Lorentz群,他构成Poincar\'{e}群的一个子群。可以直接验证$(\det \Lambda)^2=1$,这只要对上面求行列式就可以了。此外可以直接计算得$(\Lambda^{-1})^\rho_{\phantom{\rho}\nu}=\Lambda_\nu^{\phantom{\nu}\rho}$.

Lorentz群还能分为更小的子群,比如按照$\det \Lambda=1$和$\det \Lambda=-1$分类。或者$\Lambda^0_{\phantom{0}0}\geq 1$或$\Lambda^0_{\phantom{0}0}\leq -1$也构成Lorentz群的子群,再小一些,$\det \Lambda=1$且$\Lambda^0_{\phantom{0}0}\geq 1$也是一个子群。最后一个群具有很重要的意义,我们称为proper orthochronous Lorentz group.

任何一个Poincar\'{e}群的元素,要么他是属于proper orthochronous Lorentz group的,要么可以通过下面两个变换变到proper orthochronous Lorentz group,这两个变换是
\[
\mathscr{P}^\mu_{\phantom{\mu}\nu}=\begin{pmatrix}-1&&&\\&-1&&\\&&-1&\\&&&1 \end{pmatrix},\quad \mathscr{T}^\mu_{\phantom{\mu}\nu}=\begin{pmatrix}1&&&\\&\phantom{-}1&&\\&&\phantom{-}1&\\&&&-1 \end{pmatrix}.
\]
前者是空间反演,后者是时间反演。因此,只需要处理proper orthochronous Lorentz group就基本可以了。

最后我们来定义标量函数(算子)$\mathscr{H}(x)$如下:
\[
	U(\Lambda,a)\mathscr{H}(x) U^{-1}(\Lambda,a)=\mathscr{H}(\Lambda x+a),
\]
改写成
\[
	U(\Lambda,a)\mathscr{H}(x) =\mathscr{H}(\Lambda x+a)U(\Lambda,a),
\]
便可以看到这个定义是自然的。全时空积个分,就得到了标量(算子)$H$应该满足
\[
	U(\Lambda,a)H U^{-1}(\Lambda,a)=H,
\]
或者$[H,U(\Lambda,a)]=0$.

\section{The Poincar\'{e} Algebra}
Poincar\'{e}代数其实就是Poincar\'{e}群的(表示的)Lie代数。考虑无穷小Lorentz变换$\Lambda^\mu_{\phantom{\mu}\nu}=\delta^\mu_{\phantom{\mu}\nu}+\omega^{\mu}_{\phantom{\mu}\nu}$和$a^{\mu}=\epsilon^\mu$,利用$\Lambda$的正交性,有$\omega_{\mu\nu}=-\omega_{\nu\mu}$。很容易计算得到,4阶反对称的矩阵最多有6个独立元,而平移又有4个,因此Poincar\'{e}群的元素最多有10个独立元。

现在考虑上面的变换的表示$U(1+\omega,\epsilon)$,展开到一阶有
\[
U(1+\omega,\epsilon)=1+\frac{i}{2}\omega_{\rho\sigma}J^{\rho\sigma}-i\epsilon_\rho P^\rho,
\]
这里的$J^{\rho\sigma}$和$P^\sigma$就是Poincar\'{e}群的表示的Lie代数,其实就是角动量算子和动量算子。
由$U(1+\omega,\epsilon)$的幺正性,可以得到$J^{\rho\sigma}$和$P^\rho$是Hermite算子。由$\omega_{\mu\nu}=-\omega_{\nu\mu}$可以有$J^{\rho\sigma}=-J^{\sigma\rho}$。

为了求不同Lie代数之间的关系,我们展开伴随
\[
U(\Lambda,a)U(1+\omega,\epsilon)U^{-1}(\Lambda,a)=U(\Lambda(1+\omega)\Lambda^{-1},\Lambda\epsilon-\Lambda\omega\Lambda^{-1}a)
\]
的两侧到一阶,对比系数就可以得到
\[
\begin{split}
U(\Lambda,a)J^{\rho\sigma}U^{-1}(\Lambda,a)&=\Lambda_{\mu}^{\phantom{\mu}\rho}\Lambda_{\nu}^{\phantom{\nu}\sigma}
(J^{\mu\nu}-a^\mu P^\nu+a^\nu P^\mu),\\
U(\Lambda,a)P^{\rho}U^{-1}(\Lambda,a)&=\Lambda_{\mu}^{\phantom{\mu}\rho}P^\mu.
\end{split}
\]
在纯平移$\Lambda=1$下,可以看到$P^{\rho}$不变,但$J^{\rho\sigma}$却变了,所以$P^{\rho}$是平移不变的,而$J^{\rho\sigma}$不是。在无平移下$a=0$,可以看到$P^{\rho}$是一个矢量,而$J^{\rho\sigma}$是一个张量。

最后带入$U(\Lambda,a)=U(1+\omega,\epsilon)$就可以得到
\[
\begin{split}
	i\left[J^{\mu\nu},J^{\rho \sigma}\right]&=
	\eta^{\nu\rho}J^{\mu \sigma}-
	\eta^{\mu \rho}J^{\nu \sigma}-
	\eta^{\sigma\mu}J^{\rho \nu}+
	\eta^{\sigma\nu}J^{\rho \mu},\\
	i[P^\mu,J^{\rho\sigma}]&=\eta^{\mu \rho}P^\sigma-\eta^{\mu\sigma}P^{\rho},\\
	[P^\mu,P^\rho]&=0.
\end{split}
\]

记$\bm{P}=(P^1,P^2,P^3)$和$\bm{J}=(J^{23},J^{31},J^{12})$以及$\bm{K}=(J^{01},J^{02},J^{03})$,当然还有很重要的$H=P^0$,上面的那些符号不是空穴来风的,在物理上分别为三维动量,三维角动量,boost和Hamiltonian。结合一般的对易关系,就给出了那些我们熟知的关于$\bm{P}$等的对易关系,这里不再给出。但实际在逻辑上是反过来的,我们可以完全选$-J^{\mu\nu}$等作为生成元,但这和我们熟知在物理上的角动量的对易关系就不符合,因为$[-J^{\mu\nu},-J^{\rho\sigma}]=[J^{\mu\nu},J^{\rho\sigma}]$,但右边就多了一个负号,故$-J^{\mu\nu}$就不能叫做角动量。

如果我们使用同样的方法去确定动量算符前面的符号,可以发现,即使换$P^\mu$为$-P^\mu$,对易关系也不会发生改变。所以不能通过对易关系确定负号,而只能靠算子的物理意义,这其中,也只需确定$P^0$算子前面的符号即可。我们当然希望$P^0$就是Hamiltonian。众所周知,量子力学中有时间演化\footnote{这只是一个退化,我们没有必要认为这个时间演化是正确的。取而代之地,应该认为这是我们的理论在退化到量子力学的时候能够得到旧的理论,就像狭义相对论里面的动能在低速下除去一个常数项等于$mv^2/2$.}
\[\langle t+\tau,\sigma|\Psi\rangle=\langle t,\sigma|\exp (-iH\tau)|\Psi\rangle.\]
其中$\sigma$代表除了时间之外的其他自由度。

现在假设我们在$O$系看到了态矢$\Psi$,对应的波函数为$\langle t,\sigma|\Psi\rangle$,那么在$t'=t+\tau$的$O'$系看到了态矢$\exp (iP^0\tau)\Psi$,对应的波函数为$\langle t',\sigma|\exp (iP^0\tau)|\Psi\rangle$,对固定的$t$和$t'$,此二者的值应该相同$\langle t,\sigma|\Psi\rangle=\langle t',\sigma|\exp (iP^0\tau)|\Psi\rangle$,于是
\[
\langle t,\sigma|\Psi\rangle=\langle t+\tau,\sigma|\exp (iP^0\tau)|\Psi\rangle=\langle t,\sigma|\exp (-iH\tau)\exp (iP^0\tau)|\Psi\rangle
\]
于是$\exp (-iH\tau)\exp (iP^0\tau)=1$,也就是$P^0=H$.

推到其他动量分量,由于纯平移$\mathrm{T}(1,a')\mathrm{T}(1,a)=\mathrm{T}(1,a'+a)$是交换群,所以我们有表示
\[
U(1,a)=exp\left(-iP^\mu a_\mu\right),
\]
同样地,围绕一根固定轴的旋转的变换构成的群也是交换群,也有类似的表示。

\section{One-Particle States}
\re{We now consider the classification of one-particle states according to their transformation under the inhomogeneous Lorentz group.}

Poincar\'{e} 群的不可约表示给出了粒子分类的一种方式。

因为动量算子的四个分量相互对易,所以用动量的本征矢描述态矢量是自然的,剩余的自由度用$\sigma$来表示。考虑态矢量$\Psi_{p,\sigma}$,我们有
\[
P^\mu\Psi_{p,\sigma}=p^\mu\Psi_{p,\sigma},
\]
因此
\[
U(1,a)\Psi_{p,\sigma}=\exp\left(-ip^\mu a_\mu\right)\Psi_{p,\sigma}.
\]
至此,我们解决了平移的表示,以后考虑无平移的情况。

我们想要解决的是$U(\Lambda)\Psi$,因为要分解到动量的本征矢量上,则将$P^\mu$作用上去
\[
P^\mu U(\Lambda)\Psi_{p,\sigma}=U(\Lambda)U^{-1}(\Lambda)P^\mu U(\Lambda)\Psi_{p,\sigma}=U(\Lambda)\Lambda^\mu_{\phantom{\mu}\rho}P^\rho\Psi_{p,\sigma}=\Lambda^\mu_{\phantom{\mu}\rho}p^\rho U(\Lambda)\Psi_{p,\sigma},
\]
从这里可以看到$U(\Lambda)\Psi_{p,\sigma}$应该是$\Psi_{\Lambda p,\sigma'}$的线性组合,记做
\[
U(\Lambda)\Psi_{p,\sigma}=\sum_{\sigma'}C_{\sigma'\sigma}(\Lambda,p)\Psi_{\Lambda p,\sigma'}.
\]

上式的系数对应的矩阵$C_{\sigma'\sigma}(\Lambda,p)$就是矩阵表示,当然这没有做完,因为很难直接从其中看到物理意义,为此就要把矩阵表示分解为不可约表示。

首先,第一步就是要对动量分类,把动量使用Lorentz变换(不影响其他自由度)变到标准动量上,
\[
p^\mu=L^\mu_{\phantom{\mu}\nu}(p)k^\nu,
\]
现在定义
\[
\Psi_{p,\sigma}=N(p)U(L(p))\Psi_{k,\sigma},
\]
其中$N(p)$是一个数值因子,可以自行选取。

那么
\[
U(\Lambda)\Psi_{p,\sigma}=N(p)U(\Lambda L(p))\Psi_{k,\sigma}=N(p)U(L(\Lambda p))U(L^{-1}(\Lambda p)\Lambda L(p))\Psi_{k,\sigma},
\]
注意到
\[
k\xrightarrow{L(p)}p\xrightarrow{\Lambda}\Lambda p \xrightarrow{L^{-1}(\Lambda p)}k,
\]
所以这一系列复合,记做$W(\Lambda,p)=L^{-1}(\Lambda p)\Lambda L(p)$,满足$W^\mu_{\phantom{\mu}\nu}k^\nu=k^\mu$,使$k$不变。这些变换也构成一个群,称为小群。我们只要研究小群的表示就可以了,一旦知道小群的表示,代回上面的式子就知道了一般的表示。由于保$k$,所以表示的一般形式为
\[
U(W)\Psi_{k,\sigma}=\sum_{\sigma'}D_{\sigma'\sigma}(W)\Psi_{k,\sigma'}.
\]
可以验证,其满足
\[
D_{\sigma'\sigma}(W'W)=\sum_{\sigma''}D_{\sigma'\sigma''}(W')D_{\sigma''\sigma}(W).
\]

为了使得归一化变成我们喜闻乐见的形式
\[
(\Psi_{p',\sigma'},\Psi_{p,\sigma})=\delta_{\sigma'\sigma}\delta^3(\bm{p}'-\bm{p}),
\]
选取
\[
N(p)=\sqrt{\frac{k^0}{p^0}}.
\]

也许有些人是疑问的,为什么$\Psi_{p,\sigma}$的下标是四维动量,而归一化要使用三维动量的关系$\delta^3(\bm{p}'-\bm{p})$?

%下面都是个人扯淡:虽然Weinberg并没有指出态矢量内积的具体形式,但是看到这里大致可以猜出来还是默认使用了
%\[
%(\Phi,\Psi)=\int_{\rr^3}\dd^3 x\,(\Phi,\Psi_x)(\Psi_x,\Psi)=\int_{\rr^3}\dd^3 x\,\Phi^*(x)\Psi(x).
%\]
%在经典的量子力学中,动量的本征函数$e^{i\bm{p}\cdot \bm{x}}$的归一化关系就是$\delta^3(\bm{p}'-\bm{p})$.那么怎么才能出现$\delta^4(p'-p)$呢,我们只能修改内积定义为
%\[
%(\Phi,\Psi)=\int_{\rr^{3+1}}\dd^4 x\,\Phi^*(x)\Psi(x),
%\]
%这样才有归一化关系是$\delta^4(p'-p)$的可能。不过个人以为这里的定义只是习惯问题,此外不讨论内积的具体形式而把归一化关系只是做习惯约定也是有好处的,因为我们并不知道运动方程,所以也不能求得动量的本征函数,更不能说他就是$e^{ip\cdot x}$.总之,我们采取此约定,把态矢量只看做Hilbert空间内的矢量而不看做时间甚至空间的函数$\footnote{这很重要!这很重要!因为很重要所以说两遍!}$。

我认为$p$的三维分量通过相对论动量能量关系联系着第零分量
$(p^0)^2=\bm{p}^2+m^2$,所以一旦$\bm{p}$相等,则$p$也相等,所以只需要三个独立的变量就可以了。

现在来看各种情况下的小群,见Table \ref{little group}。但是,因为不可能出现低于真空的负能量,其次由$p^2=-m^2$,所以除非有虚质量所以不会出现(e)的情况。于是只有(a),(c),(f)三种情况是物理上有意义的,而这三种情况中最后一种情况是平凡的,因为其代表着真空。我们只需要分类考虑前两种情况。

\begin{table}[ht]
\centering
\begin{tabular}{l c c}
&Standard $k^\mu$ & Little Group\\
\hline
(a) $p^2=-M^2<0,p^0>0$ & $(0,0,0,M)$ & $\mathrm{SO}\left(3\right)$\\
\\
(b) $p^2=-M^2<0,p^0<0$ & $(0,0,0,-M)$ & $\mathrm{SO}\left(3\right)$\\
\\
(c) $p^2=0,p^0>0$ & $(0,0,\kappa,\kappa)$ & $\mathrm{ISO}\left(2\right)$\\
\\
(d) $p^2=0,p^0<0$ & $(0,0,\kappa,-\kappa)$ & $\mathrm{ISO}\left(2\right)$\\
\\
(e) $p^2=N^2>0$ & $(0,0,N,0)$ & $\mathrm{SO}\left(2,1\right)$\\
\\
(f) $p^\mu=0$ & $(0,0,0,0)$ & $\mathrm{SO}\left(3,1\right)$\\
\hline
\end{tabular}
\caption{Little Group}
\label{little group}
\end{table}

\subsection*{Mass Positive-Definite}
$\mathrm{SO}(3)$群的表示真是熟得不能再熟了。对于任意的$j=0,1/2,1,3/2,\dots$,都存在维度为$2j+1$的不可约表示。现设无穷小变换$R_{ik}=\delta_{ik}+\omega_{ik}$且$\omega_{ik}=-\omega_{ki}$,我们有
\[
\begin{split}
&D^{(j)}_{\sigma'\sigma}(1+\omega)=\delta_{\sigma'\sigma}+\frac{i}{2}\omega_{ik}(J^{(j)}_{ik})_{\sigma'\sigma},\\
&(J^{(j)}_{23}\pm iJ^{(j)}_{31})_{\sigma'\sigma}=\delta_{\sigma',\sigma\pm 1}\sqrt{(j\mp\sigma)(j\pm\sigma+1)},\\
&(J^{(j)}_{12})_{\sigma'\sigma}=\sigma\delta_{\sigma'\sigma},
\end{split}
\]
其中$\sigma$的范围是$-j,-j+1\dots,j-1,j$.

这里复习一下$\mathrm{SO}(3)$的一些更细的结构。$\mathrm{SO}(3)$是三维旋转群,不包括反射。可以证明,绕着固定单位矢量$\bm{a}$右手螺旋的逆时针方向旋转$\theta$后的$\bm{x}$表示为
\[
\mathrm{Rot}(\bm{a},\theta)\bm{x}=\bm{x}+(1-\cos\theta)\bm{a}\times(\bm{a}\times\bm{x})+
\sin\theta\,\bm{a}\times\bm{x}.
\]

为了证明他,首先注意到当$\bm{a}$和$\bm{x}$垂直的时候显然有
\[
\mathrm{Rot}(\bm{a},\theta)\bm{x}=\cos\theta\,\bm{x}+
\sin\theta\,\bm{a}\times\bm{x}.
\]
对于一般的$\bm{x}$,分解为平行于$\bm{a}$和垂直于$\bm{a}$(设其单位矢量为$\bm{b}$)的两个部分$\bm{x}=\mu\bm{a}+\nu\bm{b}$,而平行于$\bm{a}$的部分在旋转下不变$\mathrm{Rot}(\bm{a},\theta)\mu\bm{a}=\mu\bm{a}$。由于$\mathrm{Rot}(\bm{a},\theta)$的线性性,我们有
\[
\begin{split}
\mathrm{Rot}(\bm{a},\theta)\bm{x}&=\mu\bm{a}+\cos\theta\,(\nu\bm{b})+
\sin\theta\,\bm{a}\times(\nu\bm{b})\\
&=\bm{x}\cos\theta+(1-\cos\theta)\mu\bm{a}+
\sin\theta\,\bm{a}\times\bm{x}\\
&=\bm{x}\cos\theta+(1-\cos\theta)(\bm{a}\cdot\bm{x})\bm{a}+
\sin\theta\,\bm{a}\times \bm{x},
\end{split}
\]
但是$\bm{a}\times(\bm{a}\times\bm{x})=(\bm{a}\cdot\bm{x})\bm{a}-\bm{x}$.所以得证。

现在对$\mathrm{Rot}(\bm{a},\theta)$中的$\theta$求$\theta=0$处的导数即可得到其Lie代数
\[
-i\left.\frac{\dd}{\dd \theta}\right|_{\theta=0}\mathrm{Rot}(\bm{a},\theta)\bm{x}=-i\bm{a}\times\bm{x}
\]
选$\bm{a}=\bm{e}_i$,得到Lie代数的三个基为(选取得满足$[\hat{J}^1,\hat{J}^2]=i\hat{J}^3$等)
\[
-\hat{J}^1=\begin{pmatrix}
		&&\\
		&&i\\
		&\phantom{l}-i&
	 \end{pmatrix},\quad
-\hat{J}^2=\begin{pmatrix}
		&&-i\\
		&&\\
		i\phantom{-}&&
	 \end{pmatrix},\quad
-\hat{J}^3=\begin{pmatrix}
		&i\phantom{-}&\\
		-i&&\\
		&&
	 \end{pmatrix},
\]
(符号上,我们将$J^i$留给角动量算子,满足$U(\hat{J}^i)=J^i$)他们生成了
\[
e^{-i\theta \hat{J}^1},e^{-i\theta \hat{J}^2},e^{-i\theta \hat{J}^3}=
\begin{pmatrix}
 1 && \\
 & \cos \theta& -\sin \theta \\
 & \sin \theta & \cos \theta \\
\end{pmatrix}
,
\begin{pmatrix}
\cos \theta & &\sin \theta\\
 & 1 & \\
-\sin \theta & & \cos \theta \\
\end{pmatrix}
,
\begin{pmatrix}
\cos \theta & -\sin \theta &\\
\sin \theta&\cos \theta &\\
 & & 1 \\
\end{pmatrix}.
\]
这就是三个旋转矩阵,分别是对三根轴按右手螺旋方向做逆时针旋转。

记$\hat{\bm{p}}=\bm{p}/|\bm{p}|$为归一化矢量,而他的分量记做$\hat{p}^i$,而$R(\hat{\bm{p}})\in\mathrm{SO}(3)$表示将第三根轴的方向矢量$(0,0,1,0)$转到$\hat{\bm{p}}$方向的转动,他的逆$R^{-1}(\hat{\bm{p}})$就是$\hat{\bm{p}}$转到$(0,0,1,0)$.

$R(\hat{\bm{p}})$的具体表示如下,设$\hat{\bm{p}}$在球坐标下的表示为$(1,\theta,\varphi)$。我们先绕着第三根轴逆时针转$-\varphi$,这就是说作用一个$e^{i\varphi \hat{J}^3}$,现在$(1,\theta,\varphi)$就变成了$(1,\theta,0)$。然后对坐标系绕着第二根轴转$-\theta$,这就是说作用一个$e^{i\theta \hat{J}^2}$,此时$(1,\theta,0)$就变成了$(1,0,0)$,这就是第三根轴,所以$R^{-1}(\hat{\bm{p}})=e^{i\theta \hat{J}^2}e^{i\varphi \hat{J}^3}$,
或者
\[
	R(\hat{\bm{p}})=\exp(-i\varphi \hat{J}^3)\exp(-i\theta \hat{J}^2),
\]
他的表示就是
\[
	U(R(\hat{\bm{p}}))=\exp(-i\varphi J^3)\exp(-i\theta J^2).
\]

此外,还需要一个第三根轴和时间轴的Lorentz变换$B(|\bm{p}|)$,或者说一个boost的作用是:将质量$M$的静止粒子变成在动量方向为第三根轴且大小为$|\bm{p}|$.这个Lorentz变换的矩阵可以直接写出来
\[
	B(|\bm{p}|)k=
	\begin{pmatrix}
		1&&&\\
		&1&&\\
		&&\sqrt{1+(|\bm{p}|/M)^2}&|\bm{p}|/M\\
		&&|\bm{p}|/M&\sqrt{1+(|\bm{p}|/M)^2}
	 \end{pmatrix}
	 \begin{pmatrix}
		0\\
		0\\
		0\\
		M
	 \end{pmatrix}
	 =
	  \begin{pmatrix}
		0\\
		0\\
		|\bm{p}|\\
		\sqrt{|\bm{p}|^2+M^2}
	 \end{pmatrix}.
\]

现在使用他们就可以表示正质量粒子的标准Lorentz变换$L(p)$了,先将$p$的空间部分拉到第三根轴上面去$R^{-1}(\hat{\bm{p}})p$,然后反过来boost一下,就变成了标准动量$k=B^{-1}(|\bm{p}|)R^{-1}(\hat{\bm{p}})p$,虽然这其实就够了,但由于标准动量还可以容许一个空间旋转,我们可以为了某些理由适当选取一个空间旋转多加上去,比如取成伴随的形式
\[k=R(\hat{\bm{p}})B^{-1}(|\bm{p}|)R^{-1}(\hat{\bm{p}})p,\]此时
\[
L(p)=R(\hat{\bm{p}})B(|\bm{p}|)R^{-1}(\hat{\bm{p}}).
\]
取成伴随的形式一般来说表示着某种协变性,或者说对称性,体现在这里就是$L(p)$满足空间旋转对称性$\mathscr{R}L(p)=L(\mathscr{R}p)\mathscr{R}$,或者等价地,$W(\mathscr{R},p)=\mathscr{R}$。可以看到这样选取的$L(p)$使得小群的表示变得特别令人熟悉。这个等式在物理上就是说对于运动的有质量粒子,在三维旋转下有着和静止时相同的变换。证明如下:
\[
W(\mathscr{R},p)=L^{-1}(\mathscr{R}p)\mathscr{R}L(p)=R(\mathscr{R}\hat{\bm{p}})B^{-1}(|\mathscr{R}\bm{p}|)R^{-1}(\mathscr{R}\hat{\bm{p}})\mathscr{R}R(\hat{\bm{p}})B(|\bm{p}|)R^{-1}(\hat{\bm{p}}),
\]
对于$|\mathscr{R}\bm{p}|$,其实他就是$|\bm{p}|$。对于$R^{-1}(\mathscr{R}\hat{\bm{p}})\mathscr{R}R(\hat{\bm{p}})$,他将第三根轴转到$\hat{\bm{p}}$然后转到$\mathscr{R}\hat{\bm{p}}$,最后转回第三根轴,所以他是一个绕着第三根轴的旋转。而$B(|\bm{p}|)$是和绕第三根轴的旋转可交换的,于是
\[
\begin{split}
W(\mathscr{R},p)&=R(\mathscr{R}\hat{\bm{p}})B^{-1}(|\bm{p}|)B(|\bm{p}|)R^{-1}(\mathscr{R}\hat{\bm{p}})\mathscr{R}R(\hat{\bm{p}})R^{-1}(\hat{\bm{p}})\\
&=R(\mathscr{R}\hat{\bm{p}})R^{-1}(\mathscr{R}\hat{\bm{p}})\mathscr{R}R(\hat{\bm{p}})R^{-1}(\hat{\bm{p}})=\mathscr{R}.
\end{split}
\]

\subsection*{Mass Zero}
这里的小群是$\mathrm{ISO}(2)$,其中任意的元素对$(0,0,1,1)$进行变换后都不变。当然暴力计算可以得到$\mathrm{ISO}(2)$的全部元素的形状,但是书上的算法相当美妙,如下:对于$t^\mu=(0,0,0,1)$,Lorentz变换作为正交变换不改变内积,所以
\[
(W t)^\mu (W t)_\mu=t^\mu t_\mu=1,
\]
\[
(W t)^\mu k_\mu=(W t)^\mu (W k)_\mu=t^\mu k_\mu=1.
\]
因此$(Wt)^\mu=(\alpha,\beta,\xi,\xi+1)$,此外$\xi=(\alpha^2+\beta^2)/2$.

所以$W^\mu_{\phantom{\mu}\nu}$对$t^\nu$的作用和下面的$S^\mu_{\phantom{\mu}\nu}$对$t^\nu$的作用是相同的
\[
S^\mu_{\phantom{\mu}\nu}(\alpha,\beta)=
\begin{pmatrix}
1&0&-\alpha&\alpha\\
0&1&-\beta&\beta\\
\alpha&\beta&1-\xi&\xi\\
\alpha&\beta&-\xi&1+\xi
\end{pmatrix},
\]
这就是说,$S^\mu_{\phantom{\mu}\nu}$和$W^\mu_{\phantom{\mu}\nu}$至多差一个让$t^\mu=(0,0,0,1)$不变的算子,所以差一个空间旋转,此外$S^\mu_{\phantom{\mu}\nu}$也保持$(0,0,1,1)$不变,所以这个空间旋转要使得第三根轴不变,所以是绕第三根轴的旋转$R$:
\[
R^\mu_{\phantom{\mu}\nu}(\theta)=
\begin{pmatrix}
\cos\theta&\sin\theta&&\\
-\sin\theta&\cos\theta&&\\
&&1&\\
&&&1
\end{pmatrix},
\]
因此
\[
W(\theta,\alpha,\beta)=S(\alpha,\beta)R(\theta).
\]

此外下式可直接计算成立
\[
R(\theta)S(\alpha,\beta)R^{-1}(\theta)=S(\alpha\cos\theta+\beta\sin\theta,-\alpha\sin\theta+\beta\cos\theta).
\]

计算$W$的Lie代数只要求导就可以了,当$\beta=\theta=0$的时候,对$W$求$\alpha=0$的导数,同样对$\beta$也可以来一次,两者的Lie代数分别是
\[a^\mu_{\phantom{\mu}\nu}=
\begin{pmatrix}
0&0&i&-i\\
0&0&0&0\\
-i&0&0&0\\
-i&0&0&0
\end{pmatrix},\quad
b^\mu_{\phantom{\mu}\nu}=
\begin{pmatrix}
0&0&0&0\\
0&0&i&-i\\
0&-i&0&0\\
0&-i&0&0
\end{pmatrix}.
\]
提升第二个指标后,并令$A=U(a),B=U(b)$后可以得到
\[
A^{\mu\nu}=J_2+K_1,\quad B^{\mu\nu}=-J_1+K_2,
\]
对于第三个参数$\theta$对应的Lie代数的表示就是熟知的$J_3$.

使用对易关系,可以得到$[A,B]=0$,以及
\[
[J_3,A]=iB,\quad [J_3,B]=-iA.
\]
所以$A,B$有着相同的本征矢$\Psi_{k,a,b}$。

将最后一个参数$\theta$对应算子的本征矢分解到$A,B$的本征矢上面,这样自然就会产生一个关于$\theta$连续的本征值。(先将$U(R(\theta))$作用到$\Psi_{k,a,b}$上,再用$A$和$B$作用,检查本征值。)

但是,对于无质量的粒子,我们(在实验上)并没有发现存在这样的连续参数,于是$A\Psi_{k,\sigma}=B\Psi_{k,\sigma}=0$.

这样就只剩下\[J_3\Psi_{k,\sigma}=\sigma\Psi_{k,\sigma},\]
因为$k$是第三个方向的,所以$\sigma$是角动量在这个方向的分量(就是运动方向),称为helicity。

值得注意的是,这里并没有任何理由对helicity做出什么限制,原则上helicity可以取任何实数。但是因为拓扑上的要求,即$\mathrm{SU}(2)$是双连通的,所以helicity只能取整数和半整数。

现在对于任何的小群元素$W$,有
\[
U(W)\Psi_{k,\sigma}=e^{i\alpha A+i\beta B+i\theta J_3}\Psi_{k,\sigma}=e^{i\theta J_3}\Psi_{k,\sigma}=e^{i\theta \sigma}\Psi_{k,\sigma},
\]
所以
\[
D_{\sigma'\sigma}(W)=e^{i\theta \sigma}\delta_{\sigma'\sigma}.
\]

helicity是一个Lorentz不变量,就是说,不管是哪个观测者观测一个无质量的粒子,他们都可以得到同一个helicity。我们可以使用helicity来进行粒子分类。但下面可以看到,相反helicity的粒子可能是依靠空间反演对称性联系在一起的,因此,一旦空间反演对称性成立,他们对应的就是同一种粒子。所以,在这种情况下,一个一般的粒子态应该是由相反helicity的态叠加而成的:
\[
\Psi_{p;\alpha}=\alpha_+\Psi_{p,\sigma}+\alpha_-\Psi_{p,-\sigma},
\]
其中
\[
|\alpha_+|^2+|\alpha_-|^2=1.
\]

虽然这里直接算出了小群的元素,但是从$L(p)$出发,有时候也是方便的。$L(p)$作用在$k=(0,0,\kappa,\kappa)$使他变成动量$p$,这里选取$L(p)$为
\[
	L(p)=R(\hat{\bm{p}})B(|\bm{p}|/\kappa),
\]
其中
\[
B(u)=\begin{pmatrix}
1&&&\\
&1&&\\
&&(u^2+1)/2u&(u^2-1)/2u\\
&&(u^2-1)/2u&(u^2+1)/2u
\end{pmatrix}
\]
是一个boost.

\section{Space Inversion and Time-Reversal}
时间反演和空间反演分别是这两个算符
\[
\mathscr{P}^\mu_{\phantom{\mu}\nu}=\begin{pmatrix}-1&&&\\&-1&&\\&&-1&\\&&&\phantom{-}1 \end{pmatrix},
\quad \mathscr{T}^\mu_{\phantom{\mu}\nu}=\begin{pmatrix}1&&&\\&\phantom{-}1&&\\&&\phantom{-}1&\\&&&-1 \end{pmatrix}.
\]
现在考虑他们的表示
\[
\mathrm{P}=U(\mathscr{P},0),\quad \mathrm{T}=U(\mathscr{T},0),
\]
如果他们守恒,则满足
\[
\begin{split}
\mathrm{P}U(\Lambda,a)\mathrm{P}^{-1}&=U(\mathscr{P}\Lambda\mathscr{P}^{-1},\mathscr{P}a),\\
\mathrm{T}U(\Lambda,a)\mathrm{T}^{-1}&=U(\mathscr{T}\Lambda\mathscr{T}^{-1},\mathscr{T}a).
\end{split}
\]
\re{It should be kept in mind that this (i.e. $\mathrm{P}$ and $\mathrm{T}$ are conserved) is only an approximation.}

下面假设他们都守恒。研究$\mathrm{P}$和$\mathrm{T}$的性质,很重要的一点就是探究他是线性的还是反线性的,是幺正的还是反幺正的。应用上的守恒等式,对$H$有
\[
\mathrm{P}iH\mathrm{P}^{-1}=iH.
\]
如果$\mathrm{P}$是反线性的,则
\[
\mathrm{P}H\mathrm{P}^{-1}=-H.
\]
从而对于任意正能的态$\Psi$,$\mathrm{P}^{-1}\Psi$是负能态,因为没有比真空能0还要低的能量状态,所以$\mathrm{P}$是线性的。同样,可以对$\mathrm{T}$检查得知其是反线性的。因此
\[
\mathrm{P}H\mathrm{P}^{-1}=\mathrm{T}H\mathrm{T}^{-1}=H.
\]

知道$\mathrm{P}$和$\mathrm{T}$的性质后,就可以得知其和$H$外的其他Lie代数的对易关系:
\[
\begin{split}
\mathrm{P}\bm{J}\mathrm{P}^{-1}&=\bm{J},\\
\mathrm{P}\bm{K}\mathrm{P}^{-1}&=-\bm{K},\\
\mathrm{P}\bm{P}\mathrm{P}^{-1}&=-\bm{P},\\
\mathrm{T}\bm{J}\mathrm{T}^{-1}&=-\bm{J},\\
\mathrm{T}\bm{K}\mathrm{T}^{-1}&=\bm{K},\\
\mathrm{T}\bm{P}\mathrm{T}^{-1}&=-\bm{P},\\
\end{split}
\]

现在考虑$\mathrm{P}$和$\mathrm{T}$对单粒子态的作用。

\subsection*{$M>0$}
此即小群表中的第一行的情况,单粒子态完全由$\bm{P},H,J_3$在本征值为$0,M,\sigma$的本征矢量$\Psi_{k,\sigma}$确定。同样,考虑$\mathrm{P}$和Lie代数的对易关系后,可以直接得到$\mathrm{P}\Psi_{k,\sigma}$对应的还是本征值为$0,M,\sigma$的本征矢量,所以$\mathrm{P}\Psi_{k,\sigma}$和$\Psi_{k,\sigma}$只差一个可能依赖于自旋$\sigma$的$|\eta|=1$的相位因子$\eta_{\sigma}$:
\[
\mathrm{P}\Psi_{k,\sigma}=\eta_{\sigma}\Psi_{k,\sigma}.
\]

利用关系
\[
(J_1\pm iJ_2)\Psi_{k,\sigma}=\sqrt{(j\mp \sigma)(j\pm \sigma+1)}\Psi_{k,\sigma\pm 1}.
\]
将$P$作用到两端,可以发现$\eta_{\sigma}=\eta_{\sigma\pm 1}$,因此$\eta$独立于自旋。故而
\[
\mathrm{P}\Psi_{k,\sigma}=\eta\Psi_{k,\sigma}.
\]
我们称$\eta$为固有宇称,他只依赖于粒子种类。

对于更一般的态$\Psi_{p,\sigma}$,只要把小群的表示应用到
\[
\Psi_{p,\sigma}=\sqrt{\frac{k^0}{p^0}}U(L(p))\Psi_{k,\sigma},
\]
有
\[
\mathrm{P}\Psi_{p,\sigma}=\eta\Psi_{\mathscr{P}p,\sigma}.
\]

时间反演类似,但注意到$\mathrm{T}\bm{J}\mathrm{T}^{-1}=-\bm{J}$,所以应该是
\[
\mathrm{T}\Psi_{k,\sigma}=\zeta_{\sigma}\Psi_{k,-\sigma},
\]
同样的方法可以得到
\[
-\zeta_{\sigma}=\zeta_{\sigma\pm 1}.
\]
使用归纳法可以得到$\zeta_{\sigma}=\zeta(-1)^{j-\sigma}$,其中$\zeta$也是一个只依赖于粒子种类的因子。(但这个因子没有物理意义!因为我们可以通过调整态矢量消掉他。)

因此
\[
\mathrm{T}\Psi_{k,\sigma}=\zeta(-1)^{j-\sigma}\Psi_{k,\sigma},
\]
且
\[
\mathrm{T}\Psi_{p,\sigma}=\zeta(-1)^{j-\sigma}\Psi_{\mathscr{T}p,-\sigma}.
\]

\subsection*{$M=0$}
这里的手段比较奇妙,注意到$\mathscr{P}$把三维动量反了过来(由于只有第三分量,所以就是第三分量反向),而三维角动量不变,因此将helicity变成相反数,所以上面的思路不可以用了(我并没有理解书上去研究$\mathscr{P}$对四维动量的作用的原因。)。但是,如果我们再复合上一个绕第二根轴的$180^\circ$旋转$R_2^{-1}$,那么就把$k$的第三分量翻了回来,但helicity是Lorentz不变量,所以其不变依旧为$-\sigma$。因此
\[
U(R_2^{-1})\mathrm{P}\Psi_{k,\sigma}=\eta_{\sigma}\Psi_{k,-\sigma}.
\]

通过三维旋转的复合等手段,可以最后得到
\[
\mathrm{P}\Psi_{p,\sigma}=\eta_{\sigma}e^{\mp i\pi\sigma}\Psi_{\mathscr{P}p,-\sigma},
\]
其中$\mp$取决于$p$的第二分量是正是负。这里的正负号在$p$的第二分量为0附近不是连续的,这种不连续是本质的,因为旋转群$\mathrm{SO}(3)$并不是单连通的。

对于时间反演,可以看到他把动量和角动量全反了,但是这个时候角动量在动量方向的分量,即helicity却不变。因此
\[
U(R_2^{-1})\mathrm{T}\Psi_{k,\sigma}=\zeta_{\sigma}\Psi_{k,\sigma}.
\]
最后
\[
\mathrm{T}\Psi_{p,\sigma}=\zeta_{\sigma}e^{\pm i\pi\sigma}\Psi_{\mathscr{T}p,\sigma}.
\]

研究作用两次后的情况是有趣的,尤其是时间反演,对于有质量的粒子
\[
\mathrm{T}^2\Psi_{p,\sigma}=(-1)^{2j}\Psi_{p,\sigma},
\]
和无质量的粒子
\[
\mathrm{T}^2\Psi_{p,\sigma}=e^{\pm 2i\pi\sigma}\Psi_{p,\sigma}=(-1)^{2|\sigma|}\Psi_{p,\sigma}.
\]
对于无质量的粒子,我们称其helicity的绝对值为其自旋。那么任意半整数自旋的粒子都满足
\[
\mathrm{T}^2\Psi=-\Psi.
\]

现在开始考虑相互作用,但在相互作用下时间反演对称性成立。由于$\mathrm{T}$是对易于$H$的反线性算子,我们对任意能量的本征矢$\Psi$都有$\mathrm{T}\Psi$也是相同能量下的本征矢,但他们不可能属于相同的ray(这就是说存在简并)。假若属于相同的ray,则
\[
\mathrm{T}\Psi=\zeta\Psi,
\]
于是
\[
\mathrm{T}^2\Psi=\mathrm{T}\zeta\Psi=\zeta^*\mathrm{T}\Psi=|\zeta|^2\Psi=\Psi.
\]
和上面的矛盾了。因此满足$\mathrm{T}^2\Psi=-\Psi$的能量的本征矢都必须简并于另外的本征矢,这就是所谓的Kramers简并。
\section{Projective Representations}
这节都是数学,或者是证明,很多就略去了。
\re{It shows that there are just two (not exclusive) ways that intrinsically projective representations may arise: either algebraically, because the group is represented projectively even near the identity, or topologically, because the group is not simply connected, and hence a path from 1 to $T$ and then form $T$ to $T'$ may not be continuously deformable into some other path from 1 to $TT'$.}

一般来说,扩大群可以避开上面两个困难(譬如从$\mathrm{SO}(3)$扩充到他的万有覆盖$\mathrm{SU}(2)$),使得我们不用再处理射影表示。这里补充两个Lie群的定理:
\begin{theo}
对于一个任意的有限维Lie代数$\mathfrak{g}$,那么在同构意义下有唯一的单连通Lie群$G$,他的Lie代数就是$\mathfrak{g}$.
\end{theo}
\begin{theo}
一个连通Lie群$G$,记$G'$为其万有覆叠空间,覆叠映射为$\pi:G'\to G$.对于任意的选择$e'\in \pi^{-1}(e)$,总有唯一的$G'$上的Lie群结构使得$e'$是单位元且$\pi$是群同态。
\end{theo}
Poincar\'{e}群的拓扑值得一提,一些构造也是漂亮的。Poincar\'{e}群是平移群和Lorentz群的半只鸡,即
\[
\mathbb{R}^{3,1} \rtimes \mathrm{O}(3,1),
\]
其中不寻常的为后面的Lorentz群$\mathrm{O}(3,1)$,下面主要研究更小的子群$\mathrm{SO}(3,1)$。

每一个实四维矢量$(V^1,	V^2,V^3,V^0)$都可以构造一个Hermite矩阵
\[v=V^\mu\sigma_\mu=
\begin{pmatrix}
V^0+V^3&V^1-iV^2\\
V^1+iV^2&V^0-V^3
\end{pmatrix},
\]
其中$\sigma_\mu$是Pauli矩阵。这样构造的一大好处是
\[
V^\mu V_\mu=-\det v.
\]
接着我们考虑变换$\mathrm{Ad}(\lambda):v\mapsto \lambda^\dag v \lambda$,其中$ \lambda$是任意的$2\times2$的复矩阵。容易看出,$\mathrm{Ad}(\lambda)$线性且不改变Hermite性。现如果$|\det \lambda|=1$,则
\[
\det v=\det(\mathrm{Ad}(\lambda)v),
\]
适当调整系数,可以使得$\det \lambda=1$(这就是说$\lambda\in \mathrm{SL}(2,\mathbb{C})$)而不影响变换的一般性。

此外,$\mathrm{Ad}(\lambda)$将自然引出一个作用到$V^\mu$上的行列式为正的线性算子$\Lambda(\lambda)$通过
\[\mathrm{Ad}(\lambda)(V^\mu\sigma_\mu)=(\Lambda(\lambda)^\mu_{\phantom{\mu}\nu}V^\nu)\sigma_\mu\]
他满足
\[
\Lambda(\lambda\lambda')=\Lambda(\lambda)\Lambda(\lambda'),
\]
所以$\Lambda$是一个群同态。

因为$\mathrm{Ad}(\lambda)$不改变$\det v$,所以$\Lambda(\lambda)$也不改变$V^\mu V_\mu$,因此$\Lambda(\lambda)\in \mathrm{SO}(3,1)$。

由于$2\times 2$的复矩阵被$\det \lambda=1$限制,所以只有三个复独立参数,即六个实独立参数,数量和Lorentz群的参数个数相同,但是由于$\mathrm{Ad}(\pm 1)\mapsto 1$,所以根据同构基本定理,实际上应该有
\[
\mathrm{SO}(3,1)\cong \mathrm{SL}(2,\mathbb{C})/\mathbb{Z}_2.
\]

由于任何一个复可逆$2\times 2$矩阵都可以唯一分解(极分解)为
\[
\lambda=ue^{h}
\]
其中$u$幺正而$h$是Hermite矩阵。现在假如$\det \lambda=1$,则
\[
\det(u)e^{\mathrm{Tr}(h)}=1
\]
于是$\det(u)=1$而$\mathrm{Tr}(h)=0$.前者的一般形式为
\[
u=
\begin{pmatrix}
a+ib&c+id\\
-c+id&a-ib
\end{pmatrix},
\]
且满足$a^2+b^2+c^2+d^2=1$,因此其拓扑上等价为3-球面$\mathbb{S}^3$.而前者的一般形式为
\[
h=\begin{pmatrix}
e&f-ig\\
f+ig&-e
\end{pmatrix},
\]
拓扑上等价于$\rr^4$,因此$\mathrm{SL}(2,\mathbb{C})$在拓扑上等价于$\rr^4\times \mathbb{S}^3$,因此$\mathrm{SO}(3,1)$拓扑上等价于$\rr^4\times \mathbb{S}^3/\mathbb{Z}_2$,或者等价于$\rr^4\times \mathbb{PR}^3$,因此是双连通的。

现在来看$\mathrm{SL}(2,\mathbb{C})$的Lie代数$\mathfrak{sl}(2,\mathbb{C})$,在$\mathrm{SL}(2,\mathbb{C})$的极分解中,令$b=-ih$,则
\[
\mathrm{Tr}(b)=0,b^\dag+b=0,
\]
以及$u=e^a$有
\[
\mathrm{Tr}(a)=0,a^\dag+a=0,
\]
所以$a,b\in\mathfrak{su}(2)$,且$\lambda=e^{a+ib}$.

注意到任取一个实数$t$和$a\in\mathfrak{su}(2)$,还有$ta \in\mathfrak{su}(2)$,所以任意的一个$\lambda \in \mathrm{SL}(2,\mathbb{C})$都可以写成
\[
\lambda=e^{ta+itb}
\]
他在$t=0$的导数$a+ib$就构成$\mathfrak{sl}(2,\mathbb{C})$,那么任意的$c\in \mathfrak{sl}(2,\mathbb{C})$都可以写成
\[
c=a+ib,
\]
其中$a,b\in\mathfrak{su}(2)$,这就是说$\mathfrak{sl}(2,\mathbb{C})$可以看成$\mathfrak{su}(2)$的复化。

考虑任意两个Lorentz变换,由于Lorentz群是双连通的,所以一定有
\[
\left(U(\Lambda)U(\Lambda')U^{-1}(\Lambda\Lambda')\right)^2=1
\]
即
\[
U(\Lambda)U(\Lambda')=\pm U(\Lambda\Lambda'),
\]
将其应用到绕第三根轴半圈的变换$\Lambda=\Lambda'=\mathscr{R}$和他的表示$U(\mathscr{R})=e^{i\pi J_3}$和上有
\[
e^{2\pi iJ_3}=\pm U(\mathscr{R}\mathscr{R})=\pm 1,
\]
作用到无质量的粒子态上面
\[
e^{2\pi iJ_3}\Psi_{k,\sigma}=e^{2\pi i\sigma}\Psi_{k,\sigma}=\pm \Psi_{k,\sigma},
\]
即$e^{2\pi i\sigma}=\pm 1$,这就是helicity,或者说是无质量粒子的自旋,要取整数或者半整数的原因。因为无质量粒子的角动量不再是$\mathfrak{so}(3)=\mathfrak{su}(2)$的表示,所以这里失去了代数原因,可是大自然还是眷顾我们的,Lorentz群的拓扑性质让我们对无质量粒子也得到了一样的结论。

\chapter{Scattering Theory}
\re{Such one-particles states by themselves are not very exciting --- it is only when two or more particles interact with each other that anything interesting can happen.}

正如其名,散射理论处理的大致是这样的过程:两个(多个)量子客体从无穷远开始相互靠近,然后接触,然后再分开到无穷远。因为最初和最后他们相聚很远,所以可以看成没有相互作用,也就是自由粒子的情况,所以这个时候整个系统的态可以看成不同自由粒子的直积。而他们相互作用的那一段区域,我们能够知道的(测量的)是所谓的cross-sections,一种概率分布。

由于这章要处理多粒子,为了不使得公式过长,下面采用多重指标的记法。比如$p$代表的是一族$\{p_1,p_2,\dots,p_n\}$,特别当$p$不出现在指标的时候,一般首先理解成乘积,譬如$\dd p$被理解成$\dd p_1\dd p_2\cdots \dd p_n$。一种情况是惹人注意的,即指数的情况,虽然也是乘积但常常会被误导,譬如$e^{\Lambda^\mu_{\phantom{\mu}\nu}p^\nu}$应当被理解成
\[e^{\Lambda^\mu_{\phantom{\mu}\nu}p_1^\nu}e^{\Lambda^\mu_{\phantom{\mu}\nu}p_2^\nu}\cdots e^{\Lambda^\mu_{\phantom{\mu}\nu}p_n^\nu}=e^{\Lambda^\mu_{\phantom{\mu}\nu}(p_1^\nu+p_2^\nu+\cdots+p_n^\nu)},\]
形式上看上去可能是个加法,但实际上还是乘积。

当$p$出现在(变换作用对象,态矢等的)指标上的时候,就可能理解指标为$\{p_1,p_2,\dots,p_n\}$。比如$\Psi_p$就是$\Psi_{p_1,p_2,\dots,p_n}$或者简记为$\Psi_{p_1p_2\dots p_n}$。这个规则的出现主要是因为作用对象一般来说包含着多粒子的信息,但是却不能拆开成直积的形式。相反,比如对于相位因子$\xi_n$,一般还是会理解成乘积$\xi_{n_1}\xi_{n_2}\cdots$。再或者当出现在(主要是变换的表示的)矩阵元的指标的时候(这时候存在两个族或者多个族),一般也会理解成乘积,譬如$D_{\sigma n}$就被理解成$D_{\sigma_1 n_1}D_{\sigma_2 n_2}\cdots D_{\sigma_k n_k}$。

我们采取多重指标的做法,主要是为了简化公式。上面的规则也不是完全适用的,具体还是要看上下文。这种时候,虽然多粒子的公式基本和单粒子的公式长得一样,然而一旦具体推导,有些条件是多粒子特有的,一定不要被公式的样子所迷惑。

\section{In and Out States}
多粒子态如果相互之间没有作用,那么对其的Lorentz群的作用可以看成对一个直积的态的作用。现在约定记号,记$\alpha=\{p,\sigma,n\}$,代表着一众动量,角动量的第三分量(对于无质量粒子是helicity)和粒子种类,那么下面的符号就意味着
\[\int \dd \alpha \cdots=\sum_{\sigma,n}\int \dd^3 p\cdots,\]
此外,正交关系为
\[
(\Psi_\alpha,\Psi_{\alpha'})=\delta(\alpha-\alpha').
\]
$\delta(\alpha-\alpha')$其实是下列式子的缩写
\[
\delta^3(\bm{p}_1-\bm{p}_1')\delta_{\sigma_1 \sigma_1'}\delta_{n_1 n_1'}
\delta^3(\bm{p}_2-\bm{p}_2')\delta_{\sigma_2 \sigma_2'}\delta_{n_2 n_2'}
\cdots\pm \text{permutations},
\]
因此完全性关系写作
\[
\Psi=\int \dd \alpha\,\Psi_{\alpha}(\Psi_\alpha,\Psi).
\]

首先,在无相互作用的情况下,有
\[
H_0\Phi_\alpha=E_\alpha\Phi_\alpha,
\]
其中$E_\alpha=p^0_1+p^0_2+\cdots$.且对于一般的多粒子(无相互作用)态,我们有如下变换规则
\[
U(\Lambda,a)\Psi_{p,\sigma,n}=\exp(-ia_\mu (\Lambda p)^\mu)\sqrt{\frac{(\Lambda p)^0}{p^0}}\sum_{\sigma'}D^{(j)}_{\sigma' \sigma}\left( W(\Lambda,p)\right)\Psi_{\Lambda p,\sigma',n},
\]
当有相互作用的时候,Hamiltonian写作
\[
	H=H_0+V,
\]
当散射过程$t\to\pm\infty$的时候,各个粒子之间可以看做是没有相互作用的,即$V\to 0$,所以我们对应这些时候的态,在Lorentz变换下,应该也有上面自由粒子的变换规则。因此我们定义两个态,一个是未作用时的,称为in态,即$t\to -\infty$时,一个是作用好久后,称为out态,即$t \to \infty$时,前者记做$\Psi_\alpha^+$,后者记做$\Psi_\alpha^-$.

但是我们不把$\Psi_\alpha^\pm$看成$\Psi_\alpha$随着时间演化的结果,而把$\Psi_\alpha^\pm$看做单独的态,这样的话,把所有的信息放到态矢量里面,尤其是时空信息,是方便的,这就是所谓的Heisenberg绘景。

如果我们在$O$系中时间为$t$,态矢量为$\Psi$,那么在$t'=t-\tau$的$O'$系中观测到的态矢量为$\exp(-iH\tau)\Psi$,此时若$\tau\to-\infty$,我们将观测到in态\footnote{因为当$O'$中的时间为$t$的时候,$O$中的时间实际上是$t+\tau$,因此$\tau\to -\infty$就是作用好久以前,$\tau\to \infty$就是作用好久以后。} $\Psi_\alpha^+$,$\tau\to\infty$,我们将观测到out态$\Psi_\alpha^-$.

%现在,如果改变参考系,尤其是改变时间坐标,我们看到的虽然是不同的态矢量,但是却要是等价的态矢量,这就是说我们需要一个Lorentz变换作用到$\Psi$上,关于时间的生成元是$H$,所以,这个变换就是$e^{iHt}$这样的形式。假设,在$O$系我们的时间为$t$,在$O'$系我们的时间$t'=t+\tau$,所以当我们在$O$系看到的是$\Psi$,在$O'$系看到的就是$e^{iH\tau}\Psi$,固定$t$,让$\tau$改变,就是说我们不断改变参考系$O'$,我们看到了一系列等价于$\Psi$的态$e^{iH\tau}\Psi$,而当时间间隔$\tau\to\infty$的时候,我们就可以说$e^{iH\tau}\Psi$就是$\Psi_\alpha^+$。同样,$\Psi_\alpha^-$就对应着$\tau\to-\infty$。

由于时间能量的不确定性关系$\Delta E\Delta t\geq \hbar/2$,一旦一个态是能量的本征态,即$H\Psi_\alpha=E_\alpha \Psi_\alpha$,那么他就无法在时间上完全定位。因此,我们可以考虑一个波包
\[
\int \dd \alpha \, g(\alpha) \Psi_\alpha,
\]
其中$g(\alpha)$是一个光滑因子,他的支集位于一个有限区间$\Delta E$中。于是我们应该有
\[
	e^{-iH\tau}\int\dd \alpha\, g(\alpha) \Psi_\alpha^\pm=\int\dd \alpha\, e^{-iE_\alpha\tau}g(\alpha) \Psi_\alpha^\pm,
\]
和相应的自由粒子态在$\tau\ll -1/\Delta E$或者$\tau\gg 1/\Delta E$时的叠加表现地一样。

现在我们定义in和out态为,如果自由粒子满足方程
\[
H_0\Phi_\alpha=E_\alpha\Phi_\alpha,
\]
则in和out态满足方程
\[
H\Psi_\alpha^\pm=E_\alpha\Psi_\alpha^\pm,
\]
且当$\tau\to\mp\infty$时使得
\begin{equation}
\label{jianjinguanxi}
\int e^{-iE_\alpha\tau}g(\alpha) \Psi_\alpha^\pm \to \int e^{-iE_\alpha\tau}g(\alpha) \Phi_\alpha,
\end{equation}
或者写作
\[
e^{-iH\tau}\int\dd \alpha\, g(\alpha) \Psi_\alpha^\pm \to e^{-iH_0\tau}\int\dd \alpha\,g(\alpha) \Phi_\alpha,
\]
成立。

如果记$\Omega(\tau)=e^{iH\tau}e^{-iH_0\tau}$的话,上面的式子有时候会形式地写作
\[
\Psi_\alpha^\pm=\Omega(\mp \infty)\Phi_\alpha.
\]
可以用渐进过程证明$\Psi_\alpha^\pm$们也是正交的,即
\[
(\Psi_\alpha^\pm,\Psi_\beta^\pm)=(\Phi_\alpha,\Phi_\beta)=\delta(\alpha-\beta).
\]

现在改写关于$\Psi_\alpha^\pm$的能量的本征方程
\[
(E_\alpha-H_0)\Psi_\alpha^\pm=V\Psi_\alpha^\pm,
\]
其中$(E_\alpha-H_0)$并不是可逆的,因为对于相同能量的$E_\beta$也等于0。但是如果我们加或者减一个很小的$i\epsilon I$,其中$I$是单位算子而$\epsilon$是无穷小正数,$(E_\alpha-H_0\pm i\epsilon I)$或者写作$(E_\alpha-H_0\pm i\epsilon)$就是可逆的\footnote{这是比较魔性的解法,但并不严格,具体参考教材。}。然后考虑当$V\to 0$的时候$\Psi_\alpha^\pm \to \Phi_\alpha$,所以我们有解
\[
\Psi_\alpha^\pm=\Phi_\alpha+(E_\alpha-H_0\pm i\epsilon)^{-1}V\Psi_\alpha^\pm.
\]
或者写成对自由粒子的本征态展开
\[
\Psi_\alpha^\pm=\Phi_\alpha+\int \dd \beta \,\frac{T^\pm_{\beta\alpha}\Phi_\alpha}{E_\beta-E_\alpha\pm i\epsilon},
\]
其中$T^\pm_{\beta\alpha}=(\Phi_\beta,V\Psi_\alpha^\pm)$.这就是所谓的Lippmann-Schwinger方程。可以验证上面这个解确实满足渐进关系\eqref{jianjinguanxi},其中$i\epsilon$前正负号的选取也来自于此。

有一点也是有趣的,$(E\pm i\epsilon)^{-1}$的虚部其实正比于一个关于$\epsilon$的$\delta$函数序列$\delta_{\epsilon}(E)=\epsilon/(\pi(E^2+\epsilon^2))$,在渐进过程后,可以写作
\[
(E\pm i\epsilon)^{-1}=\frac{E}{E^2+\epsilon^2}\mp i \pi \delta_{\epsilon}(E)\to \frac{E}{E^2+\epsilon^2}\mp i \pi \delta(E)
\]

\section{The $S$-Matrix}
如果我们入射一个$\Psi_\alpha^+$的粒子,然后在作用后发现了$\Psi_\beta^-$的粒子,我们需要计算/测量的是$\alpha \to \beta$的概率,那么基本就是要看$(\Psi_\beta^-,\Psi_\alpha^+)$,不放定义这就是$S$矩阵的矩阵元
\[
	S_{\beta\alpha}=\left(\Psi_\beta^-,\Psi_\alpha^+\right),
\]
容易验证$S$矩阵是幺正的。

但是有时候处理算子而不是矩阵会变得更加方便,同时自由粒子的态也很方便,所以$S$算子被定义为
\[(\Phi_\beta,S\Phi_\alpha)=S_{\beta\alpha},
\]
使用\[
\Psi_\alpha^\pm=\Omega(\mp \infty)\Phi_\alpha,
\]我们可以得到
\[S=\Omega(\infty)^\dag \Omega(-\infty)=U(+\infty,-\infty),
\]
其中\[U(\tau,\tau_0)=\Omega(\tau)^\dag \Omega(\tau_0)=e^{iH_0\tau}e^{-iH(\tau-\tau_0)}e^{-iH_0\tau_0}.\]

当没有相互作用的时候$\Psi_\alpha^-=\Psi_\alpha^+$,此时$S_{\alpha\beta}=\delta(\alpha-\beta)$。当计算存在相互作用时候的概率的时候应该去掉这项,转而计算$|S_{\beta\alpha}-\delta(\beta-\alpha)|^2$.

实际上,借助Lippmann-Schwinger方程,我们可以计算$S$矩阵的矩阵元为如下形式
\[
S_{\beta\alpha}=\delta(\beta-\alpha)-2\pi i\delta(E_\beta-E_\alpha)T^+_{\beta\alpha}.
\]
当$V$很弱的时候,$T^+_{\beta\alpha}=(\Phi_\beta,V\Psi^+_\alpha)$中的$\Psi^+_\alpha$可以近似看做$\Phi_\alpha$,此时
\[
S_{\beta\alpha}\simeq\delta(\beta-\alpha)-2\pi i\delta(E_\beta-E_\alpha)(\Phi_\beta,V\Phi_\alpha),
\]
这就是Born近似。

同样借助Lippmann-Schwinger方程,我们也可以不使用时间的渐进过程而直接得到$\Psi_\alpha^\pm$的正交性和$S$矩阵的幺正性。

\section{Symmetries of the $S$-Matrix}
\subsection*{Lorentz Invariance}
不考虑时间、空间反演。

对于任何一个Lorentz变换,我们可能希望定义一个幺正算子$U(\Lambda,a)$使得其可以作用在in或out态上时,都有着相同的表示(即第二章描述的一般的表示)。当我们说一个理论是Lorentz不变的时候,我们是指同样的算子$U(\Lambda,a)$同时可以作用在in和out态上。因为$U(\Lambda,a)$是幺正算子,
\[
S_{\beta\alpha}=\left(\Psi_\beta^-,\Psi_\alpha^+\right)=\left(U(\Lambda,a)\Psi_\beta^-,U(\Lambda,a)\Psi_\alpha^+\right).
\]
这就应该被理解成$S$矩阵的Lorentz不变性,展开上式,则可以写作(参考第二章的表示,需要从小群的表示代回去)
\begin{equation}
\label{smatrixdui}
\begin{split}
	S_{p',\sigma',n';p,\sigma,n}&=e^{ia_\mu\Lambda^\mu_{\phantom{\mu}\nu}(p'^\nu-p^\nu)}\sqrt{\frac{(\Lambda p)^0 (\Lambda p')^0}{p^0p'^0}}\\
	&\times\sum_{\bar{\sigma}} D^{(j)}_{\bar{\sigma}\sigma}(W(\Lambda,p))\sum_{\bar{\sigma}'} D^{(j')*}_{\bar{\sigma}'\sigma'}(W(\Lambda,p'))\\
	&\times S_{\Lambda p',\bar{\sigma}',n';\Lambda p,\bar{\sigma},n}.
\end{split}
\end{equation}
由于左边不显含$a$,故除非动量守恒,则$S$矩阵为0.所以$S$矩阵有如下形式:
\[
S_{\beta\alpha}=\delta(\beta-\alpha)-2\pi iM_{\beta\alpha}\delta^4(p_\beta-p_\alpha).
\]

上面的式子\eqref{smatrixdui}应当被理解为$S$矩阵Lorentz不变性的定义。一般情况下并不能够满足,而是只对一些特定的Hamiltonian才有可能成立。下面需要探究怎么样的Hamiltonian能够使得$S$矩阵满足Lorentz不变性。

对于自由粒子态$\Phi$以及其相关的表示我们给予下标0,然后对任意的$U_0(\Lambda,a)$有
\[
U_0(\Lambda,a)\Phi_{p,\sigma,n}=e^{-ia_\mu\Lambda^\mu_{\phantom{\mu}\nu}p^\nu}\sqrt{\frac{(\Lambda p)^0}{p^0}}\sum_{\bar{\sigma}} D^{(j)}_{\bar{\sigma}\sigma}(W(\Lambda,p))\Phi_{\Lambda p,\bar{\sigma},n}.
\]
若$[U_0(\Lambda,a),S]=0$则式\eqref{smatrixdui}成立(注意$S_{\beta\alpha}=(\Phi_\beta,S\Phi_\alpha)$),特别地考虑那些生成元,$\bm{P}_0$和$\bm{J}_0$和$\bm{K}_0$以及$H_0$,我们有
\[
	[H_0,S]=[\bm{J}_0,S]=[\bm{P}_0,S]=[\bm{K}_0,S]=0.
\]
而由于这些生成元是自由粒子的,所以满足典型的对易关系。所以,下面我们希望做出适当的假设使得上面的对易关系成立,那$S$矩阵的Lorentz不变性也成立了。

在这之中$[H_0,S]$是任何情况下都可以保证的。考虑任意的自由粒子能量本征态$\Phi_\alpha$,那么
\[
	SH_0\Phi_\alpha=E_\alpha S\Phi_\alpha=E_\alpha\int \dd \beta\, \Phi_\beta(\Phi_\beta,S\Phi_\alpha)=\int \dd \beta\, E_\alpha\Phi_\beta S_{\beta\alpha},
\]
同理
\[
	H_0S\Phi_\alpha=H_0\int \dd \beta\, \Phi_\beta(\Phi_\beta,S\Phi_\alpha)=\int \dd \beta\, E_\beta \Phi_\beta S_{\beta\alpha},
\]
现在由于在
\[
S_{\beta\alpha}=\delta(\beta-\alpha)-2\pi iM_{\beta\alpha}\delta^4(p_\beta-p_\alpha)
\]
右边两项代表着能量守恒的$\delta$函数的出现,所以$S_{\beta\alpha}E_\beta=S_{\beta\alpha}E_\alpha$,进而可以推知$[H_0,S]=0$.为了导出剩下的对易关系,我们需要对相互作用所有了解。

\re{In virtually all know field theories, the effect of intersections is to add an interaction term $V$ to the Hamiltonian, while leaving the momentum and angular momentum unchanged. (The only known exceptions are theories with topologically twisted fields, such as those with magnetic monipoles, where the angular momentum of states depends on the interactions.)}
即对于“精确的”生成元来说,有如下形式:
\[
H=H_0+V,\qquad \bm{P}=\bm{P}_0,\qquad\bm{J}=\bm{J}_0.
\]

现在来考虑这些非自由粒子的生成元,或者说“精确的”生成元,我们当然希望可以定义出的生成元可以对in态的变换群有着一般的Lorentz变换到in态相同的形式(对于out态是否有着相同的变换形式这里并不是显然的),因此相同的群结构也要求他们满足相同的对易关系。

由于$\bm{P}=\bm{P}_0$和$\bm{J}=\bm{J}_0$,所以对于$\bm{J}$与本身,$\bm{P}$与本身,和$\bm{J}$与$\bm{P}$之间的对易关系来说没有任何改变。因为
\[
	[H,\bm{J}]=[H_0,\bm{J}_0]+[V,\bm{J}_0]=[V,\bm{J}_0],
\]
所以,我们还需要对相互作用加上假设$[V,\bm{J}_0]=0$,同理还有$[V,\bm{P}_0]=0$,就得到了$[H,\bm{J}]=[H,\bm{P}]=0$.

因为$\bm{J}=\bm{J}_0$和$H_0$与$H$都对易,所以$\bm{J}_0$和$U(t,t_0)$对易,取极限后可以得知$[\bm{J}_0,S]=0$,同理也可得$[\bm{P}_0,S]=0$.这样,剩下的只有$S$和$\bm{K}_0$的对易没有证明了,这需要其他假设。

首先的假设就是关于$\bm{K}$和其他“精确”生成元的对易关系。因为$\bm{K}$和$\bm{P}$的对易关系以及$[K^i_0,P^j_0]=-iH_0\delta^{ij}$,所以不可能有$\bm{K}=\bm{K}_0$除非$H=H_0$.因此必然有$\bm{K}=\bm{K}_0+\bm{W}$,靠着$[K^i,P^j]=-iH\delta^{ij}$,我们需要有$[\bm{K}_0,V]=-[\bm{W},H]$.然而,努力的他并没有什么物理意义,我们总可以找到平凡的$\bm{W}$使得上面的式子成立。
\re{The crucial point in the Lorentz invariance of a theory is not that there should exist a set of exact generators satisfying the commutation relations, but rather that these operators should act the same way on in and out states; merely finding an operators $\bm{K}$ that satisfies $[\bm{K}_0,V]=-[\bm{W},H]$ is not enough.}

为使他有物理意义,还需要做更多假定,比如我们假设$\bm{W}$的矩阵元还是光滑的,特别地,其不能存在类似于$(E_\alpha-E_\beta)^{-1}$这样的奇点,则我们可以证明最后的$[\bm{K}_0,S]=0$.

利用对易关系,可以得到下面的等式
\[
[\bm{K}_0,U(\tau,\tau_0)]=-\bm{W}(\tau)U(\tau,\tau_0)+U(\tau,\tau_0)\bm{W}(\tau_0),
\]
其中$\bm{W}(t)=e^{iH_0t}\bm{W}e^{-iH_0t}$.光滑性假设和$[\bm{K}_0,V]=-[\bm{W},H]$使得当$t\to \pm \infty$的时候$W(t)\to 0$,这个假设是自然的,其类似于$V(t)\to 0$的假设。所以$[\bm{K}_0,U(\infty,-\infty)]=[\bm{K}_0,S]=0$。

总结一下,为了$S$矩阵满足Lorentz不变性,我们找到了如下假定\footnote{这意味着可能有其他的假定依然能够使得$S$矩阵满足Lorentz不变性。}:首先,相互作用表现为Hamiltonian上加上一项势能项$V$;对于这项$V$,需要满足$[V,\bm{J}_0]=[V,\bm{P}_0]=0$;对于boost的精确生成元$\bm{K}=\bm{K}_0+\bm{W}$,其中$\bm{W}$的矩阵元应该足够光滑。

上述第一第二条假设导出了$[\bm{J}_0,S]=[\bm{P}_0,S]=0$,第一第三条假设导出了$[\bm{K}_0,S]=0$,加上自然满足的$[H_0,S]=0$,我们就导出了$[U_0(\Lambda,a),S]$,继而得出了$S$矩阵满足Lorentz不变性。

最后,在上述假设下,可以计算得对于任意的$A=\bm{P}$, $\bm{J}$,$ \bm{K}$, $H$我们都有
\[
A\Omega(\mp \infty)=\Omega(\mp \infty)A_0.
\]
利用这个式子可以证明我们强调的不变性的关键在于作用到in和out态的形式相同。因为
\[
A\Psi_\alpha^\pm=A\Omega(\mp \infty)\Phi_\alpha=\Omega(\mp \infty)A_0\Phi_\alpha,
\]
所以这些生成元生成的算子也满足
\[
U(\Lambda,a)\Psi_\alpha^\pm=\Omega(\mp \infty)U_0(\Lambda,a)\Phi_\alpha,
\]
右边计算出表示
\[
U_0(\Lambda,a)\Phi_\alpha=\sum_{\alpha'}C_{\alpha'\alpha}\Phi_{\alpha'},
\]
则得到了
\[
U(\Lambda,a)\Psi_\alpha^\pm=\sum_{\alpha'}C_{\alpha'\alpha}\Omega(\mp \infty)\Phi_{\alpha'}=\sum_{\alpha'}C_{\alpha'\alpha}\Psi^\pm_{\alpha'},
\]
所以$U(\Lambda,a)$对于$\Psi_\alpha^\pm$有相同的表示。

前面我们虽然没有计算$\bm{J}$和$\bm{K}$等的对易关系,但由于in和out态在Lorentz变换下有着和自由粒子态一样的表示,我们也有理由去相信剩余的对易关系是成立的。

最后,在数学上,如果
\[
	U(\Lambda,a)\Omega(\mp \infty)=\Omega(\mp \infty)U_0(\Lambda,a)
\]
中的$\Omega(\mp \infty)$是双射,则$U$和$U_0$这两个表示被称为是等价的。

\subsection*{Internal Symmetries}

毫无疑问,存在着与Lorentz对称独立的对称,譬如粒子和反粒子之间的荷共轭对称。这种变换$T$的表示为一个幺正算子$U(T)$满足
\[
U(T)\Psi_{p,\sigma,n}=\sum_{\bar{n}} \mathcal{D}_{\bar{n}n}(T)\Psi_{ p,\sigma,\bar{n}},
\]
对于矩阵$\mathcal{D}$,可以验证满足
\[
	\mathcal{D}(T)\mathcal{D}(T')=\mathcal{D}(TT').
\]
类似地,矩阵也是幺正的,即$\mathcal{D}^{-1}(T)=\mathcal{D}^\dag(T)$.同样,将这种表示同时引到in和out态上面,我们可以写出$S$矩阵在变换下需要满足的方程,并以此作为对称性的定义:
\[
	S_{p',\sigma',n';p,\sigma,n}=\sum_{N,N'}\mathcal{D}(T)^*_{N'n'}\mathcal{D}(T)_{Nn}S_{p',\sigma',N';p,\sigma,N}.
\]

考虑一个特殊的例子,如果对于自由粒子的表示$U_0(T)$和$H$与$V$都对易,那么我们就可以同时对in和out态引入$U_0(T)$作为其表示,那么这就是一个满足对称性的例子。

考虑单参Lie群,任意一个表示都可以写作$U(T(\theta))=\exp(iQq)$,其中$Q$是一个Hermite算子。此时他的矩阵就可以写作
\[
	\mathcal{D}_{n'n}(T(\theta))=\delta_{n'n}\exp(iq_n\theta),
\]
类似于上面对于$S$矩阵满足对称性蕴含动量守恒的做法,我们可以导出$S$矩阵满足对称性下有
\[
	q_{n'_1}+q_{n'_2}+\cdots=q_{n_1}+q_{n_2}+\cdots.
\]
上面这个等式一个重要的例子就是电荷守恒。

关于时间空间反演对称性,下面不再详细阐述,只给出一些$S$矩阵满足对称性的定义。

对于空间反演,我们有
\[
	S_{p',\sigma',n';p,\sigma,n}=\eta^*_{n'}\eta_{n}S_{\mathscr{P}p',\sigma',n';\mathscr{P}p,\sigma,n}.
\]

对于时间反演,我们有
\[
	S_{\beta;\alpha}=S_{\mathscr{T}\alpha;\mathscr{T}\beta},
\]
或者写开来
\[
	S_{p',\sigma',n';p,\sigma,n}=\zeta_{n'}(-1)^{j'-\sigma'}\zeta^*_{n}(-1)^{j-\sigma}S_{\mathscr{P}p,-\sigma,n;\mathscr{P}p',-\sigma',n'}.
\]

此外还有上面提到的荷共轭对称性,着对应的变换写作$\mathrm{C}$,他对态的作用表示为
\[
	\mathrm{C}\Psi^{\pm}_{p,\sigma,n}=\xi_n\Psi^{\pm}_{p,\sigma,n^c},
\]
其中$n^c$是$n$的反粒子。

$S$矩阵满足这种对称性应该写作
\[
	S_{p',\sigma',n';p,\sigma,n}=\xi^*_{n'}\xi_nS_{p',\sigma',n^{\prime c};p,\sigma,n^c}.
\]

上面三种变换可以组合,尤其重要的是CPT,在当今目前任何的场论中,他都是守恒的。$S$矩阵CPT对称性应该写作
\[
	S_{\beta;\alpha}=S_{\mathscr{CPT}\alpha;\mathscr{CPT}\beta}.
\]
\section{Rates and Cross-Sections}
上面可以看到,$S_{\alpha\beta}$矩阵显含着$\delta^4(p_\alpha-p_\beta)$,这保证能量和动量的守恒。可是当我们考虑概率$|S_{\alpha\beta}|^2$,那么我们应该如何处理$|\delta^4(p_\alpha-p_\beta)|^2$呢?

\re{The proper way to approach these problems is by studying the way that experiments are actually done, using wave packets to represent particles localized far from each other before a collision, and then following the time-history of these superpositions of multi-particle states.}

我们考虑一个被约束在体积为$V$的立方体区域(边长为$L
$)内的体系,相对侧上的点确定,因此空间波函数的单值性限制了动量必须是量子化的
\[
	\bm{p}=\frac{2\pi}{L}(n_1,n_2,n_3),
\]
其中$n_i$是整数。现在定义这种情况下的$\delta$函数,
\[
	\delta^3_V(\bm{p}'-\bm{p})=\frac{1}{(2\pi)^3}\int_V \dd^3 x \,e^{i(\bm{p}'-\bm{p})\cdot \bm{x}},
\]
很容易看到$V\to \infty$的时候$\delta^3_V(\bm{p}'-\bm{p})\to \delta^3(\bm{p}'-\bm{p})$.由于这是$\delta$型序列,我们换个$\delta$型序列经常是方便的,
\[
	\delta^3_V(\bm{p}'-\bm{p})=\frac{V}{(2\pi)^3}\delta_{\bm{p}',\bm{p}}.
\]
他们都是$\delta$型序列且当$\bm{p}'=\bm{p}$都等于$V/(2\pi)^3$.

使用正交化关系
\[
(\Psi_\alpha,\Psi_{\alpha'})=\delta(\alpha-\alpha'),
\]
我们可以得到,我们可以看到正交关系中不仅仅是$\delta(\alpha-\alpha')$们组合乘积的和,还含有一个因子$(V/(2\pi)^3)^N$,其中$N$为粒子个数。我们将这个因子纳入波函数中,定义
\[
	\Psi_\alpha^{\text{Box}}=\left(\frac{V}{(2\pi)^3}\right)^{N_\alpha/2}\Psi_\alpha,
\]
这样$\Psi_\alpha^{\text{Box}}$们的正交化关系就没有因子了,写作
\[
\left(\Psi_\alpha^{\text{Box}},\Psi_{\alpha'}^{\text{Box}}\right)=\delta_{\alpha,\alpha'}.
\]
同样可以改写一下$S$矩阵为
\[
S_{\beta\alpha}^{\text{Box}}=\left(\frac{V}{(2\pi)^3}\right)^{-(N_\alpha+N_\beta)/2}S_{\beta\alpha}.
\]

当然,如果放任粒子们永远在这个箱子里面,那么任意的可能转变都会无数次地发生(个人存疑),所以我们为了计算出有意义的转变概率,我们应该还要限制时间。我们假设相互作用只在时间$T$内存在,即是假设存在一个时间盒子,那么我们的能量守恒$\delta$函数也需要改写为
\[
	\delta_T(E_\alpha-E_\beta)=\frac{1}{2\pi}\int_{-T/2}^{T/2}\dd t \,e^{i(E_\alpha-E_\beta)t}.
\]

最后,我们可以对箱子内的多粒子系统,在打开作用开关之前是$\alpha$态,在关闭作用开关之后的态为$\beta$态,他们从$\alpha$态到$\beta$态的转变概率为
\[
	P(\alpha\to\beta)=\left|S_{\beta\alpha}^{\text{Box}}\right|^2=\left(\frac{V}{(2\pi)^3}\right)^{-(N_\alpha+N_\beta)}\left|S_{\beta\alpha}\right|^2.
\]

现在考虑动量空间中的体积元$\dd^3 p$内所含的盒单粒子态,由于$\bm{p}=2\pi/L\,(n_1,n_2,n_3)$,所以所含的个数应该为$\dd^3 p/(2\pi/L)^3$,这就是$(n_1,n_2,n_3)$三元组在$\bm{p}$附近的$\dd^3 p$中的个数。我们可以定义末态$\beta$的改变量$\dd \beta$是每一个单粒子末态$\dd^3 p$的乘积,所以态的总数为
\[
	\dd \mathcal{N}_\beta=\left(\frac{V}{(2\pi)^3}\right)^{N_\beta}\dd \beta.
\]
所以末态的小改变引发的概率的改变写作
\[
	(\dd P)(\alpha\to\beta)=P(\alpha\to\beta)\dd \mathcal{N}_\beta=\left(\frac{V}{(2\pi)^3}\right)^{-N_\alpha}\left|S_{\beta\alpha}\right|^2\dd \beta.
\]

我们以后只关心这样的情况:末态和初态不同,且任何末态中的粒子构成的子集不能和对应着的初态中的粒子子集有着相同的四动量(这意味着我们只关心$S$矩阵的连通分支)。这种情况下,使用
\[
S_{\beta\alpha}=\delta(\beta-\alpha)-2\pi iM_{\beta\alpha}\delta^4(p_\beta-p_\alpha),
\]
第一项消失,而后面的$\delta$函数应该使用盒子里的$\delta$函数代替。综上,我们可以将$S$矩阵写作
\[
S_{\beta\alpha}=-2\pi i \delta^3_V(\bm{p}_\beta-\bm{p}_\alpha)\delta_T(E_\beta-E_\alpha)M_{\beta\alpha}.
\]

我们对盒子的引入,使得我们可以对$\delta$函数的模方进行计算
\[
	\begin{split}
	\left(\delta^3_V(\bm{p}_\beta-\bm{p}_\alpha)\right)^2&=\delta^3_V(\bm{p}_\beta-\bm{p}_\alpha)\delta^3_V(0)=\frac{V}{(2\pi)^3}\delta^3_V(\bm{p}_\beta-\bm{p}_\alpha),\\
	\left(\delta_T(E_\beta-E_\alpha)\right)^2&=\delta_T(E_\beta-E_\alpha)\delta_T(0)=\frac{T}{2\pi}\delta_T(E_\beta-E_\alpha),
	\end{split}
\]
因此
\[
	\dd P(\alpha\to\beta)=4\pi^2\frac{T}{2\pi}\left(\frac{V}{(2\pi)^3}\right)^{-(N_\alpha-1)}\left|M_{\beta\alpha}\right|^2\delta^3_V(\bm{p}_\beta-\bm{p}_\alpha)\delta_T(E_\beta-E_\alpha)\dd \beta.
\]

然后考虑$T$和$V$都很大的时候,这个时候我们把$\delta$函数过度为标准的$\delta$函数,即
\[
	\dd P(\alpha\to\beta)=4\pi^2\frac{T}{2\pi}\left(\frac{V}{(2\pi)^3}\right)^{-(N_\alpha-1)}\left|M_{\beta\alpha}\right|^2\delta^4(p_\beta-p_\alpha)\dd \beta.
\]
可以看到在极限过程下,$\dd P(\alpha\to\beta)$正比于$T$,所以我们也会使用微分转变速率$\dd \Gamma(\alpha\to\beta)=\dd P(\alpha\to\beta)/T$来描述
\[
	\dd \Gamma (\alpha\to\beta)=(2\pi)^{3N_\alpha-2}V^{1-N_\alpha}\left|M_{\beta\alpha}\right|^2\delta^4(p_\beta-p_\alpha)\dd \beta,
\]
而此时的$S$矩阵就写作
\[
S_{\beta\alpha}=-2\pi iM_{\beta\alpha}\delta^4(p_\beta-p_\alpha).
\]
\subsection*{$N_\alpha=1$}
此即单粒子的情况
\[
	\dd \Gamma (\alpha\to\beta)=2\pi\left|M_{\beta\alpha}\right|^2\delta^4(p_\beta-p_\alpha)\dd \beta,
\]
当然这里的时间是有要求的,就是要比这个粒子的平均寿命$\tau_\alpha$要短很多,因此我们没有理由去取$T\to \infty$的极限,同时,对于能量,我们就应该要大于$1/\tau_\alpha$大小的间隔。
\subsection*{$N_\alpha=2$}
此时
\[
	\dd \Gamma (\alpha\to\beta)=(2\pi)^{4}V^{-1}\left|M_{\beta\alpha}\right|^2\delta^4(p_\beta-p_\alpha)\dd \beta,
\]
所以其反比于$V$,或者说正比于空间密度。但实验物理学家一般不处理每密度转变速率,而是去处理所谓的截面
\[
	\dd \sigma (\alpha\to\beta)=(2\pi)^{4}u_\alpha^{-1}\left|M_{\beta\alpha}\right|^2\delta^4(p_\beta-p_\alpha)\dd \beta,
\]
其中$u_\alpha$是两个粒子的相对速率,下面将会定义。

下面考虑转变速率和截面在Lorentz下的表现,这将帮助给出相对速率的定义。为了避免\eqref{smatrixdui}中依赖于动量的矩阵元的复杂,我们考虑\eqref{smatrixdui}两边的模方,并且带入$S$矩阵$S_{\beta\alpha}=-2\pi iM_{\beta\alpha}\delta^4(p_\beta-p_\alpha)$,可以得到
\[
	R_{\beta\alpha}=\sum_{\text{spins}}|M_{\beta\alpha}|^2\prod_\beta E\prod_\alpha E
\]
是关于末态初态$p$的Lorentz不变标量,其中$\prod_\beta E$代表着所有在$\beta$态的粒子的单粒子能量$p_0=\sqrt{\bm{p}^2+m^2}$的乘积,$\prod_\alpha E$同理。

将单粒子的$\dd \Gamma (\alpha\to\beta)$对所有的自旋求和,则有
\[
	\begin{split}
	\sum_{\text{spins}}\dd \Gamma (\alpha\to\beta)&=
	2\pi\sum_{\text{spins}}\left|M_{\beta\alpha}\right|^2\delta^4(p_\beta-p_\alpha)\dd \beta\\
	&=2\pi E^{-1}_\alpha R_{\beta\alpha}\delta^4(p_\beta-p_\alpha)\dd \beta\left/\prod_\beta E\right..
	\end{split}
\]
注意到$\dd^3 \bm{p}/\sqrt{\bm{p}^2+m^2}$是Lorentz不变的,因此$\dd \beta\left/\prod_\beta E\right.$也是Lorentz不变的。所以最后上式中只有$1/E_\alpha$不是Lorentz不变的。这就意味着衰变速率在Lorentz变换下表现得和$1/E_\alpha$相同,因此,香港记者跑得越快,他衰变得越慢。

类似地,对双粒子,截面可以写成
\[
	\begin{split}
	\sum_{\text{spins}}\dd \sigma (\alpha\to\beta)&=(2\pi)^{4}u_\alpha^{-1}\sum_{\text{spins}}\left|M_{\beta\alpha}\right|^2\delta^4(p_\beta-p_\alpha)\dd \beta,\\
	&=(2\pi)^{4}u_\alpha^{-1}E^{-1}_{1}E^{-1}_{2}R_{\beta\alpha}\delta^4(p_\beta-p_\alpha)\dd \beta\left/\prod_\beta E\right..
	\end{split}
\]
其中$E_{1}$和$E_{2}$即是双粒子的能量。把截面做成一个Lorentz不变量是方便的,因此这就要求$u_\alpha^{-1}E^{-1}_{1}E^{-1}_{2}$或者$u_\alpha E_{1}E_{2}$是一个Lorentz不变标量。此外,如果选取一个参考系使得粒子静止,他们的相对速率$u_\alpha$就应该是另一个粒子的速率。满足这些要求的$u_\alpha$为
\[
	u_\alpha=\frac{1}{E_{1}E_{2}} \sqrt{\left(\eta_{\mu\nu}p_1^\mu p_2^\nu\right)^2-m_1^2 m_2^2}.
\]

接着我们来看质心系的表现,$p_1=(\bm{p},E_1)$和$p_2=(-\bm{p},E_2)$,然后通过上式可以计算得相对速率为
\[
	u_\alpha=\frac{|\bm{p}|(E_1+E_2)}{E_1 E_2}=\left|\frac{\bm{p}_1}{E_1}-\frac{\bm{p}_2}{E_2}\right|
\]
这非常类似于我们在非相对论情况下熟知的相对速率公式(这个用Weinberg的说法就是bonus,福利→\_→)。值得一提的是,相对速率不是物理上的速率,他在速度接近光速的时候的值能够接近2.

在质心系$\bm{p}_\alpha=0$,而末态诸粒子的动量为$\bm{p}'_1,\bm{p}'_2,\cdots$,那么
\[
	\delta^4(p_\beta-p_\alpha)\dd \beta=\delta^3(\bm{p}'_1+\bm{p}'_2+\cdots)\delta(E'_1+E'_2+\cdots-E)\dd^3 p'_1 \dd^3 p'_2\cdots,
\]
其中$E$是$\alpha$态的总能量。对任意一个$\bm{p}'_k$积分都能去掉$\delta^3$,这里对$\bm{p}'_1$积分。
\[
	\delta^4(p_\beta-p_\alpha)\dd \beta \to\delta(E'_1+E'_2+\cdots-E)\dd^3 p'_2\cdots,
\]
且式中$\bm{p}'_1=-\bm{p}'_2-\bm{p}'_3-\cdots$.

若末态只有双粒子,则
\[
	\delta^4(p_\beta-p_\alpha)\dd \beta \to\delta(E'_1+E'_2-E)\dd^3 p'_2,
\]
或者展开然后用球坐标
\[
	\delta^4(p_\beta-p_\alpha)\dd \beta \to\delta\left(\sqrt{|\bm{p}'_1|^2+m^2}+\sqrt{|\bm{p}'_2|^2+m^2}-E\right)|\bm{p}'_2|^2 \dd |\bm{p}'_2| \dd \Omega.
\]
其中$\bm{p}'_1=-\bm{p}'_2$且$\dd \Omega=\sin\theta\dd\theta\dd\varphi$是立体角元。现在使用$\delta$函数的性质:对$f(x_0)=0$且$f'(x_0)\neq 0$有
\[
	\delta\left(f(x)\right)=\frac{\delta(x-x_0)}{|f'(x_0)|}.
\]
把$\delta$函数内的式子看成$|\bm{p}'_1|$的函数,在他的零点$k'$展开,然后再按照$\dd |\bm{p}'_1|$积分消去这个$\delta$函数。通过一大堆计算可以得到
\[
	\delta^4(p_\beta-p_\alpha)\dd \beta \to \frac{k'E'_1 E'_2}{E}\dd \Omega,
\]
其中
\[
	\begin{split}
		k'&=\frac{1}{2E}\sqrt{(E^2-m^{\prime 2}_1-m^{\prime 2}_2)^2-4m^{\prime 2}_1 m^{\prime 2}_2},\\
		E'_1&=\sqrt{k^{\prime 2}+m^{\prime 2}_1}=\frac{E^2+m^{\prime 2}_1-m^{\prime 2}_2}{2E},\\
		E'_2&=\sqrt{k^{\prime 2}+m^{\prime 2}_2}=\frac{E^2-m^{\prime 2}_1+m^{\prime 2}_2}{2E}.
	\end{split}
\]

对于单粒子衰变成双粒子,我们有
\[
	\frac{\dd \Gamma(\alpha\to \beta)}{\dd \Omega} =\frac{2\pi k'E'_1 E'_2}{E}|M_{\beta\alpha}|^2,
\]
对于双粒子到双粒子,我们有
\[
	\frac{\dd \sigma(\alpha\to \beta)}{\dd \Omega} =\frac{(2\pi)^4 k'E'_1 E'_2}{Eu_\alpha}|M_{\beta\alpha}|^2=\frac{(2\pi)^4 k'E'_1 E'_2 E_1 E_2}{E^2 u_\alpha}|M_{\beta\alpha}|.
\]
\section{Perturbation Theory}
历史上计算$S$矩阵最有用的技术是微扰理论,就是将$S$矩阵展开成相互作用项$V$的级数。下面先讨论含时微扰。

首先$S=U(\infty,-\infty)$,其中
\[
	U(\tau,\tau_0)=e^{iH_0\tau}e^{-iH(\tau-\tau_0)}e^{-iH_0\tau_0}.
\]
将上面的式子对$\tau$求导,可以得到
\begin{equation}
\label{weifenfc}
	i\frac{\dd}{\dd\tau}U(\tau,\tau_0)=V(\tau)U(\tau,\tau_0),
\end{equation}

其中
\[
	V(\tau)=e^{iH_0 \tau}Ve^{-iH_0 \tau}.
\]

现在我们把微分方程\eqref{weifenfc}变成积分方程,
\[
	U(\tau,\tau_0)=1-i\int_{\tau_0}^{\tau}\dd t \, V(t)U(t,\tau_0),
\]
不断将$U(\tau,\tau_0)$迭代进积分号有
\[
\begin{split}
	U(\tau,\tau_0)=&1-i\int_{\tau_0}^{\tau}\dd t_1 \, V(t_1)+(-i)^2\int_{\tau_0}^{\tau}\dd t_1\int_{\tau_0}^{t_1}\dd t_2\, V(t_1)V(t_2)\\
	&+(-i)^3\int_{\tau_0}^{\tau}\dd t_1\int_{\tau_0}^{t_1}\dd t_2\int_{\tau_0}^{t_2}\dd t_3\, V(t_1)V(t_2)V(t_3)+\cdots,
\end{split}
\]
然后置$\tau=\infty,\tau_0=-\infty$,得到了
\begin{equation}
\label{weirao}
\begin{split}
	S=&1-i\int_{-\infty}^{\infty}\dd t_1 \, V(t_1)+(-i)^2\int_{-\infty}^{\infty}\dd t_1\int_{-\infty}^{t_1}\dd t_2\, V(t_1)V(t_2)\\
	&+(-i)^3\int_{-\infty}^{\infty}\dd t_1\int_{-\infty}^{t_1}\dd t_2\int_{-\infty}^{t_2}\dd t_3\, V(t_1)V(t_2)V(t_3)+\cdots.
\end{split}
\end{equation}

除却上面的这个方法,借助Lippmann-Schwinger方程
\[
\Psi_\alpha^\pm=\Phi_\alpha+\int \dd \beta \,\frac{T^\pm_{\beta\alpha}\Phi_\alpha}{E_\beta-E_\alpha\pm i\epsilon},
\]
其中$T^\pm_{\beta\alpha}=(\Phi_\beta,V\Psi_\alpha^\pm)$.然后作用$V$并且对$\Phi_\beta$求内积,可以得到方程
\[
T^\pm_{\beta\alpha}=V_{\beta\alpha}+\int \dd \gamma\, \frac{V_{\beta\gamma}T^\pm_{\gamma\alpha}}{E_\alpha-E_\gamma\pm i\epsilon},
\]
其中$V_{\beta\gamma}=(\Phi_\beta,V\Phi_\gamma)$。反复迭代,我们也可以得到一个$T^\pm_{\beta\alpha}$的展开,然后也可以用此得到$S$矩阵的展开。

我们设阶梯函数$\theta(x)$当$x\geq 0$的时候为$1$,否则为$0$.而$\theta(t_1-t_2)$我们称为时序。

对于显然的等式
\[
	\int_{-\infty}^{\infty}\dd t_1\int_{-\infty}^{t_1}\dd t_2\, V(t_1)V(t_2)=\int_{-\infty}^{\infty}\dd t_1\int_{-\infty}^{\infty}\dd t_2\, \theta(t_1-t_2)V(t_1)V(t_2),
\]
右边做一个$t_1,t_2$的置换
\[
	\int_{-\infty}^{\infty}\dd t_1\int_{-\infty}^{\infty}\dd t_2\, \theta(t_1-t_2)V(t_1)V(t_2)=\int_{-\infty}^{\infty}\dd t_2\int_{-\infty}^{\infty}\dd t_1\, \theta(t_2-t_1)V(t_2)V(t_1),
\]
所以
\[
	\int_{-\infty}^{\infty}\dd t_1\int_{-\infty}^{t_1}\dd t_2\, V(t_1)V(t_2)=\frac{1}{2!}\int_{-\infty}^{\infty}\dd t_1\int_{-\infty}^{\infty}\dd t_2\, T\left\{V(t_1)V(t_2)\right\},
\]
其中$T\left\{V(t_1)V(t_2)\right\}=\theta(t_1-t_2)V(t_1)V(t_2)+\theta(t_2-t_1)V(t_2)V(t_1)$.类似地,对其他低阶高阶项,我们都可以做置换,然后重写成对$T\left\{*\right\}$在全时间的积分,即
\begin{equation}
\label{weirao2}
\begin{split}
	S=&1-i\int_{-\infty}^{\infty}\dd t_1 \, T\left\{V(t_1)\right\}+\frac{(-i)^2}{2!}\int_{-\infty}^{\infty}\dd t_1\int_{-\infty}^{\infty}\dd t_2\, T\left\{V(t_1)V(t_2)\right\}\\
	&+\frac{(-i)^3}{3!}\int_{-\infty}^{\infty}\dd t_1\int_{-\infty}^{\infty}\dd t_2\int_{-\infty}^{\infty}\dd t_3\, T\left\{V(t_1)V(t_2)V(t_3)\right\}+\cdots\\
	=&1+\sum_{n=1}^{\infty}\frac{(-1)^n}{n!}\int_{\rr^n}\dd t_1\dd t_2\cdots\dd t_n\, T\left\{V(t_1)V(t_2)\cdots V(t_n)\right\}
\end{split}
\end{equation}
其中
\[
	T\left\{V(t_1)V(t_2)\cdots V(t_n)\right\}=\sum_{\sigma\in S^n}\left(\prod_{k=1}^{n-1}\theta(t_{\sigma_k}-t_{\sigma_{k+1}})V(t_{\sigma_k})
	\right)V(t_{\sigma_n}),
\]
$S^n$是置换构成的对称群。

利用这和$\exp$的幂级数展开的相似性,我们可以将上面的级数形式地写成
\[
	S=T\exp\left(-i\int_{-\infty}^{\infty}V(t)\right),
\]
尽管他未必收敛。对此,一个很重要的情况就是当$V(t)$们可交换的时候
\[
	S=\exp\left(-i\int_{-\infty}^{\infty}V(t)\right).
\]

含时微扰的展开\eqref{weirao2}可以帮助我们建立一大类$S$矩阵的对称性理论。因为$S$矩阵元是$S$算子作用在自由粒子态上产生的,我们自然希望对于自由粒子的变换$U_0(\Lambda,a)$和$S$算子是可以交换的,或者等价地,$S$算子和那些生成元$\bm{P}_0,\bm{J}_0,\bm{K}_0,H_0$是可以交换的。为此,假设有一类作用,$V(t)$可以写成作用密度$\mathscr{H}(\bm{x},t)$对于空间的积分,即
\[
	V(t)=\int \dd^3 x\,\mathscr{H}(\bm{x},t),
\]
而$\mathscr{H}(\bm{x},t)$是满足Lorentz不变性的标量,即
\[
	U_0(\Lambda,a)\mathscr{H}(x)U_0^{-1}(\Lambda,a)=\mathscr{H}(\Lambda x+a).
\]
那么按照\eqref{weirao2}
\[
S=1+\sum_{n=1}^{\infty}\frac{(-1)^n}{n!}\int\dd^4 x_1\dd^4 x_2\cdots\dd
^4 x_n\, T\left\{\mathscr{H}(x_1)\mathscr{H}(x_2)\cdots \mathscr{H}(x_n)\right\},
\]
现在除了$\theta(t_1-t_2)$这样的函数是否也是满足Lorentz不变性还不知道之外,其他都已经是满足Lorentz不变性的了。

现在两个时空点的间隔$x_1-x_2$对应时序$\theta(t_1-t_2)$是满足Lorentz不变性,除非$x_1-x_2$是类空的,就是说除非$x_1-x_2$与自己的内积中空间部分大于时间部分。所以当$x_1-x_2$是类空的,甚至包括是类光的时候(包括是因为有时候类光情况下$\theta(t_1-t_2)$的Lorentz不变性的破坏),为了使得$S$矩阵满足Lorentz不变性,我们只好退一步让
\[
	[\mathscr{H}(x_1),\mathscr{H}(x_2)]=0,
\]
这个条件可以理解成因果律的要求,如果两个时空点是类空的,那么他们之间相互没有影响,故而是可以交换的。这样的假设下$S$算子就写成
\[
S=1+\sum_{n=1}^{\infty}\frac{(-1)^n}{n!}\int\dd^4 x_1\dd^4 x_2\cdots\dd
^4 x_n\, \mathscr{H}(x_1)\mathscr{H}(x_2)\cdots \mathscr{H}(x_n),
\]
自然是满足Lorentz不变性的。现在,在上面的假设下$S$算子是一定满足Lorentz不变性的。

\re{Theories of this class are not the only ones that are Lorentz invariant, but the most general Lorentz invariant theories are not very different. In particular, there is always a commutation condition has no counterpart for non-relativistic systems, for which time-orderding is always Galilean-invariant. It is the condition that makes the combination of Lorentz invariance and quantum mechanics so restrictive.}

现在来形式化非微扰地证明一下这个结论。由于
\[
	V=\int \dd^3 x\,\mathscr{H}(\bm{x},0),
\]
而$\mathscr{H}(\bm{x},t)$满足$U_0(\Lambda,a)\mathscr{H}(x)U_0^{-1}(\Lambda,a)=\mathscr{H}(\Lambda x+a)$,所以
\[
	\exp(\bm{a}\cdot\bm{P}_0)V=\int \dd^3 x\,\exp(\bm{a}\cdot\bm{P}_0)\mathscr{H}(\bm{x},0)=\int \dd^3 x\,\mathscr{H}(\bm{x}+\bm{a},0)\exp(\bm{a}\cdot\bm{P}_0)=V \exp(\bm{a}\cdot\bm{P}_0),
\]
于是$[V,\bm{P}_0]=0$,同理还有$[V,\bm{J}_0]=0$.

剩下的,做一个无穷小boost,可知
\[
	-i[\bm{K}_0,\mathscr{H}(\bm{x},t)]=t\nabla\mathscr{H}(\bm{x},t)+\bm{x}\partial_0\mathscr{H}(\bm{x},t),
\]
对$\bm{x}$积分,并让$t=0$,注意到$-i[H_0,\mathscr{H}]=\partial_0\mathscr{H}$
\[
	[\bm{K}_0,V]=i\int \dd^3 x\,\partial_0\mathscr{H}(\bm{x},0) \bm{x}=\int \dd^3 x[H_0,-\bm{x}\mathscr{H}(\bm{x},0)]=[H_0,\bm{W}],
\]
其中$\bm{W}=-\int \dd^3 x\,\bm{x}\mathscr{H}(\bm{x},0)$,一般来说这个$\bm{W}$的矩阵元是光滑的。

最后只要检验$[\bm{W},V]=0$就好了,这样我们就得到了$[\bm{K}_0,V]=[H,\bm{W}]$,直接计算有
\[
	[\bm{W},V]=-\int \dd^3 x\dd^3y\, \bm{x}\bigl[\mathscr{H}(\bm{x},0),\mathscr{H}(\bm{y},0)\bigr],
\]
但$(\bm{x}-\bm{y},0)$类空,所以$\bigl[\mathscr{H}(\bm{x},0),\mathscr{H}(\bm{y},0)\bigr]=0$给出了我们想要的式子。

\section{Implications of Unitarity}
现在来看看$S$矩阵作为幺正矩阵的结果。利用
\[
S_{\beta\alpha}=\delta(\beta-\alpha)-2\pi iM_{\beta\alpha}\delta^4(p_\beta-p_\alpha),
\]
以及幺正性,可以得到
\[
	\begin{split}
	\delta(\gamma-\alpha)=&\int \dd\beta \, S_{\gamma\beta}^\dag S_{\beta\alpha}=\delta(\gamma-\alpha)+2\pi i\delta^4(p_\gamma-p_\alpha)(M^\dag_{\gamma\alpha}-M_{\gamma\alpha})\\
	&+4\pi^2\int\dd\beta \,\delta^4(p_\gamma-p_\beta)\delta^4(p_\beta-p_\alpha)M_{\gamma\beta}^\dag M_{\beta\alpha},
	\end{split}
\]
稍稍整理一下有
\[
	iM^\dag_{\gamma\alpha}-iM_{\gamma\alpha}+2\pi\int\dd\beta \,\delta^4(p_\beta-p_\alpha)M_{\gamma\beta}^\dag M_{\beta\alpha}=0,
\]
当$\gamma=\alpha$的时候
\[
	\mathrm{Im}(M_{\alpha\alpha})=-\pi\int\dd\beta \,\delta^4(p_\beta-p_\alpha) |M_{\beta\alpha}|^2.
\]

利用他可以写出$\dd \Gamma(\alpha\to\beta)$的积分
\[
	\begin{split}
	\Gamma_\alpha=\int \dd\beta\, \frac{\dd \Gamma(\alpha\to\beta)}{\dd\beta}&=(2\pi)^{3N_\alpha-2}V^{1-N_\alpha}\int\dd\beta \,\delta^4(p_\beta-p_\alpha) |M_{\beta\alpha}|^2\\
	&=-\pi^{-1}(2\pi)^{3N_\alpha-2}V^{1-N_\alpha}\mathrm{Im}(M_{\alpha\alpha}).
	\end{split}
\]
这就是光学定理,但是习惯上,我们并不写成这样的形式,这里不再展开经典形式。

CPT不变性提供了一个粒子$\alpha\to\beta$与他的反粒子$\beta\to\alpha$的联系。而此时光学定理就给出了$\Gamma$和对应的$\Gamma$之间的关系:
\[
	\Gamma_{p,\sigma,n}=\Gamma_{p,-\sigma,n^c}.
\]

此外另一个$S$矩阵作为幺正矩阵的有趣结果是Boltzmann的$H$定理,这里不加阐述。值得一提的是,这个结果因为是幺正性的结论,所以并不是近似的。

\chapter{The Cluster Decomposition Principle}
这章要讨论Hamiltonian的具体结构,我们可以通过给出他所有的矩阵元来定义,或者等价的,任意的算子都可以表示为关于产生单粒子和湮灭单粒子的算符的函数(通过给出所有的矩阵元)。产生和湮灭算符历史上是在电磁场以及其他场的正则量子化时引入量子力学的,他们提供了一个对无质量粒子产生和湮灭的自然的理论框架。
\re{However, there is a deeper reason for constructing the Hamiltonian out of creation and annihilation operators, which goes beyond the need to quantize any pre-existing field theory like electrodynamics, and has nothing to do with whether particles can actually be produced or destroyed. The great advantage of this formalism is that if we express the Hamiltonian as a sum of products of creation and annihilation operators, with suitable non-singular coefficients, the the $S$-matrix will automatically satisfy a crucial physical requirement, the cluster decomposition principle, which says in effect that distant experiments yield uncorrelated results.}
\section{Bosons and Fermions}

\section{Creation and Annihilation Operators}
生成算子$a^\dag(q)$被定义为在态矢量$\Phi_{q_1 \cdots q_N}$前加上一个量子数的算子
\[
	a^\dag(q)\Phi_{q_1 \cdots q_N}=\Phi_{qq_1 \cdots q_N}.
\]
于是任何一个态矢量都可以写作
\[
	\Phi_{q_1 \cdots q_N}=a^\dag(q_1)\cdots a^\dag(q_N)\Phi_0,
\]
其中$\Phi_0$为真空态。湮灭算子$a(q)$是$a^\dag(q)$的共轭算子,当$q_1,\cdots,q_N$都是玻色子和费米子的时候,$a^\dag(q)$可以定义为
\[
	a(q)\Phi_{q_1\cdots q_N}=\sum_{r=1}^N(\pm)^{r+1}\delta(q-q_r)\Phi_{q_1\cdots q_{r-1}q_{r+1}\cdots q_N},
\]
其中$+$是对玻色子,$-$是对费米子。对于真空态,无论是费米的还是玻色的,我们都有
\[
	a(q)\Phi_0=\Phi_0.
\]
通过直接计算,很容易验证对易关系
\[
\begin{split}
	[a(q'),a^\dag(q)]_\mp=&a(q')a^\dag(q)\mp a^\dag(q)a(q')=\delta(q'-q),\\
	[a(q'),a(q)]_\mp=&a(q')a(q)\mp a(q)a(q')=0,\\
	[a^\dag(q'),a^\dag(q)]_\mp=&a^\dag(q')a^\dag(q)\mp a^\dag(q)a^\dag(q')=0,\\
\end{split}
\]
其中$-$是对玻色子,$+$是对费米子。
我们现在考虑这两个算子在	Lorentz下面的表现,首先
\[
U_0(\Lambda,\alpha)\Psi_{p,\sigma,n}=e^{-i(\Lambda p)\cdot \alpha}\sqrt{\frac{(\Lambda p)^0}{p^0}}\sum_{\bar{\sigma}} D^{(j_n)}_{\bar{\sigma}\sigma}(W(\Lambda,p))\Psi_{\Lambda p,\bar{\sigma},n},
\]
上式等价于
\[
U_0(\Lambda,\alpha)a^\dag(\mathbf{p},\sigma,n)\Psi_0=e^{-i(\Lambda p)\cdot \alpha}\sqrt{\frac{(\Lambda p)^0}{p^0}}\sum_{\bar{\sigma}} D^{(j_n)}_{\bar{\sigma}\sigma}(W(\Lambda,p))a^\dag(\mathbf{p}_\Lambda,\bar{\sigma},n)\Psi_0,
\]
其中$\mathbf{p}_\Lambda$是$\Lambda p$的空间部分,由于$U^{-1}_0(\Lambda,\alpha)\Psi_0=\Psi_0$,所以
\[
	U_0(\Lambda,\alpha)a^\dag(\mathbf{p},\sigma,n)U^{-1}_0(\Lambda,\alpha)\Psi_0=e^{-i(\Lambda p)\cdot \alpha}\sqrt{\frac{(\Lambda p)^0}{p^0}}\sum_{\bar{\sigma}} D^{(j_n)}_{\bar{\sigma}\sigma}(W(\Lambda,p))a^\dag(\mathbf{p}_\Lambda,\bar{\sigma},n)\Psi_0,
\]
最后我们可以得到
\[
\begin{split}
	U_0(\Lambda,\alpha)a^\dag(\mathbf{p},\sigma,n)U_0^{-1}(\Lambda,\alpha)&=e^{-i(\Lambda p)\cdot \alpha}\sqrt{\frac{(\Lambda p)^0}{p^0}}\sum_{\bar{\sigma}} D^{(j_n)}_{\bar{\sigma}\sigma}\left(W(\Lambda,p)\right)a^\dag(\mathbf{p}_\Lambda,\bar{\sigma},n),\\
	U_0(\Lambda,\alpha)a(\mathbf{p},\sigma,n)U_0^{-1}(\Lambda,\alpha)&=e^{i(\Lambda p)\cdot \alpha}\sqrt{\frac{(\Lambda p)^0}{p^0}}\sum_{\bar{\sigma}} D^{(j_n)*}_{\bar{\sigma}\sigma}\left(W(\Lambda,p)\right)a(\mathbf{p}_\Lambda,\bar{\sigma},n).
\end{split}
\]


\section{Cluster Decomposition and Connected Amplitudes}




\chapter{Quantum Fields and Antiparticles}
(这章的构造实在漂亮。)
\section{Free Fields}
我们已经在第三章看到$S$矩阵是Lorentz不变的,如果相互作用能够写作
\[
	V(t)=\int \dd^3 x\,\mathscr{H}(\mathbf{x},t),
\]
其中$\mathscr{H}(\mathbf{x},t)$是满足Lorentz不变性的标量,即
\[
	U_0(\Lambda,a)\mathscr{H}(x)U_0^{-1}(\Lambda,a)=\mathscr{H}(\Lambda x+a),
\]
以及当$x_1-x_2$不是类时的时候要满足
\[
	[\mathscr{H}(x_1),\mathscr{H}(x_2)]=0.
\]
从第四章,为了方便满足集团分解原理,我们将用生成和湮灭算子来构造$\mathscr{H}(x)$,此时,生成和湮灭算子在Lorentz变换下满足
\[
\begin{split}
	U_0(\Lambda,\alpha)a^\dag(\mathbf{p},\sigma,n)U_0^{-1}(\Lambda,\alpha)&=e^{-i(\Lambda p)\cdot \alpha}\sqrt{\frac{(\Lambda p)^0}{p^0}}\sum_{\bar{\sigma}} D^{(j_n)}_{\bar{\sigma}\sigma}\left(W(\Lambda,p)\right)a^\dag(\mathbf{p}_\Lambda,\bar{\sigma},n),\\
	U_0(\Lambda,\alpha)a(\mathbf{p},\sigma,n)U_0^{-1}(\Lambda,\alpha)&=e^{i(\Lambda p)\cdot \alpha}\sqrt{\frac{(\Lambda p)^0}{p^0}}\sum_{\bar{\sigma}} D^{(j_n)*}_{\bar{\sigma}\sigma}\left(W(\Lambda,p)\right)a(\mathbf{p}_\Lambda,\bar{\sigma},n),
\end{split}
\]
其中$\mathbf{p}_\Lambda$是$\Lambda p$的空间部分。可以看到,在Lorentz变换下,生成和湮灭算子都将乘以一个矩阵,依赖于算子本身携带的动量。但是,怎么从生成和湮灭算子来构造出一个标量呢?答案是,我们用积分以及求和来积掉动量以及自旋和粒子种类,即用场来构造$\mathscr{H}(x)$,更准确地说,用湮灭场和生成场
\begin{equation}
\begin{split}
	\psi_l^{+}(x)&=\sum_{\sigma,n}\int \dd^3 p\, u_l(x;\mathbf{p},\bar{\sigma},n)a(\mathbf{p},\bar{\sigma},n),\\
	\psi_l^{-}(x)&=\sum_{\sigma,n}\int \dd^3 p\, v_l(x;\mathbf{p},\bar{\sigma},n)a^\dag(\mathbf{p},\bar{\sigma},n)
\end{split}
\label{chang}
\end{equation}
来构造$\mathscr{H}(x)$,其中$u,v$是系数,他们需要被选得使得这两个场在Lorentz变换下满足:
\[
\begin{split}
	U_0(\Lambda,\alpha)\psi_l^{+}(x)U_0^{-1}(\Lambda,\alpha)&=\sum_{m}D_{lm}(\Lambda^{-1})\psi_m^{+}(\Lambda x+\alpha),\\
	U_0(\Lambda,\alpha)\psi_l^{-}(x)U_0^{-1}(\Lambda,\alpha)&=\sum_{m}D_{lm}(\Lambda^{-1})\psi_m^{-}(\Lambda x+\alpha),
\end{split}
\]
其中$D(\Lambda)$是Lorentz群的一个表示\footnote{一般地,对于湮灭场和生成场,对应的应该是两个不同的表示$D^\pm$.但是,Schur引理保证了当这两个为同维的不可约表示时,他们其实是相同的。而可约表示又可以可以分解到不可约表示,所以不妨就直接假设他们是相同的。},可以是标量表示$D(\Lambda)=1$,可以是矢量表示$D(\Lambda)^{\mu}_{\phantom{\mu}\nu}=\Lambda^{\mu}_{\phantom{\mu}\nu}$,可以是张量或者旋量表示。

最后我们就可以构造$\mathscr{H}(x)$如下:
\[
\mathscr{H}(x)=\sum_{N,M}\sum_{l'_1,\cdots,l'_N}
\sum_{l_1,\cdots,l_M}g_{l'_1,\cdots,l'_N;l_1,\cdots,l_M}\psi_{l'_1}^{-}(x)\cdots \psi_{l'_N}^{-}(x)\psi_{l_1}^{+}(x)\cdots \psi_{l_M}^{+}(x),
\]
为了让他称为满足Lorentz不变性的标量,则系数$g_{l'_1,\cdots,l'_N;l_1,\cdots,l_M}$应该选得满足(采用多重指标记法)
\[
	g_{\bar{l}';\bar{l}}=\sum_{l,l'}D_{l'\bar{l}'}(\Lambda^{-1})D_{l\bar{l}}(\Lambda^{-1})g_{l';l}.
\]

现在可以计算$U_0(\Lambda,b)\psi_l^{+}(x)U_0^{-1}(\Lambda,b)$和$U_0(\Lambda,b)\psi_l^{-}(x)U_0^{-1}(\Lambda,b)$来求得其系数$u,v$应该满足的关系:
\begin{equation}
	\begin{split}
		\sum_{\bar{\sigma}}u_{\bar{l}}(\Lambda x+b;\mathbf{p}_\Lambda,\bar{\sigma},n)D^{(j_n)}_{\bar{\sigma}\sigma}\left(W(\Lambda,p)\right)&=\sqrt{\frac{p^0}{(\Lambda p)^0}}\sum_{l} D_{\bar{l}l}(\Lambda)e^{i(\Lambda p)\cdot b}u_{l}(x;\mathbf{p},\sigma,n),\\
		\sum_{\bar{\sigma}}v_{\bar{l}}(\Lambda x+b;\mathbf{p}_\Lambda,\bar{\sigma},n)D^{(j_n)*}_{\bar{\sigma}\sigma}\left(W(\Lambda,p)\right)&=\sqrt{\frac{p^0}{(\Lambda p)^0}}\sum_{l} D_{\bar{l}l}(\Lambda)e^{-i(\Lambda p)\cdot b}v_{l}(x;\mathbf{p},\sigma,n),
	\end{split}
\label{xishu}
\end{equation}
下面分三步来简化这个关系。
\subsection*{Translations}
首先考察$\Lambda=1$的时候,即对应$b$任意的纯平移情况,此时$D^{(j_n)}_{\bar{\sigma}\sigma}\left(W(\Lambda,p)\right)=\delta_{\bar{\sigma}\sigma}$以及$D_{\bar{l}l}(\Lambda)=\delta_{\bar{l}l}$,所以\eqref{xishu}就变成了
\[
	\begin{split}
		u_{l}(x+b;\mathbf{p},\sigma,n)&=e^{i p\cdot b}u_{l}(x;\mathbf{p},\sigma,n),\\
		v_{l}(x+b;\mathbf{p},\sigma,n)&=e^{-i p\cdot b}v_{l}(x;\mathbf{p},\sigma,n),
	\end{split}
\]
令$u_{l}(\mathbf{p},\sigma,n)=(2\pi)^{3/2}u_{l}(0;\mathbf{p},\sigma,n)$和$v_{l}(\mathbf{p},\sigma,n)=(2\pi)^{3/2}v_{l}(0;\mathbf{p},\sigma,n)$,此时
\[
	\begin{split}
		u_{l}(x;\mathbf{p},\sigma,n)&=(2\pi)^{-3/2}e^{i p\cdot x}u_{l}(\mathbf{p},\sigma,n),\\
		v_{l}(x;\mathbf{p},\sigma,n)&=(2\pi)^{-3/2}e^{-ip\cdot x}v_{l}(\mathbf{p},\sigma,n),
	\end{split}
\]
引入$(2\pi)^{-3/2}$这个因子是因为此时
\[
\begin{split}
	\psi_l^{+}(x)&=\sum_{\sigma,n}\left(\frac{1}{\sqrt{2\pi}}\right)^{3}\int \dd^3 p\, e^{ip\cdot x}u_{l}(\mathbf{p},\sigma,n)a(\mathbf{p},\bar{\sigma},n),\\
	\psi_l^{-}(x)&=\sum_{\sigma,n}\left(\frac{1}{\sqrt{2\pi}}\right)^{3}\int \dd^3 p\, e^{-ip\cdot x}v_{l}(\mathbf{p},\sigma,n)a^\dag(\mathbf{p},\bar{\sigma},n)
\end{split}
\]
就是Fourier变换。去掉平移后,可以得到和\eqref{xishu}等价的
\begin{equation}
	\begin{split}
		\sum_{\bar{\sigma}}u_{\bar{l}}(\mathbf{p}_\Lambda,\bar{\sigma},n)D^{(j_n)}_{\bar{\sigma}\sigma}\left(W(\Lambda,p)\right)&=\sqrt{\frac{p^0}{(\Lambda p)^0}}\sum_{l} D_{\bar{l}l}(\Lambda)u_{l}(\mathbf{p},\sigma,n),\\
		\sum_{\bar{\sigma}}v_{\bar{l}}(\mathbf{p}_\Lambda,\bar{\sigma},n)D^{(j_n)*}_{\bar{\sigma}\sigma}\left(W(\Lambda,p)\right)&=\sqrt{\frac{p^0}{(\Lambda p)^0}}\sum_{l} D_{\bar{l}l}(\Lambda)v_{l}(\mathbf{p},\sigma,n).
	\end{split}
	\label{xishu2}
\end{equation}
\subsection*{Boosts}
其次,令$\mathbf{p}=0$,让$\Lambda$为把静止质量为$m$的粒子变到四动量为$q^\mu$的标准Lorentz变换$L(q)$,此时$L(p)=1$,于是
\[
	W(\Lambda,p)=L^{-1}(\Lambda p)\Lambda L(p)=L^{-1}(q)L(q)=1.
\]
所以我们就可以把\eqref{xishu2}变成
\[
	\begin{split}
		u_{\bar{l}}(\mathbf{q},\sigma,n)&=\sqrt{\frac{m}{q^0}}\sum_{l} D_{\bar{l}l}(L(q))u_{l}(0,\sigma,n),\\
		v_{\bar{l}}(\mathbf{q},\sigma,n)&=\sqrt{\frac{m}{q^0}}\sum_{l} D_{\bar{l}l}(L(q))v_{l}(0,\sigma,n).
	\end{split}
\]
因此,只要知道了表示$D$和静止时的$u_{l}(0,\sigma,n)$以及$v_{l}(0,\sigma,n)$,我们就可以得到一般的系数。
\subsection*{Rotations}
最后,让$\mathbf{p}=0$,而$\Lambda$使得$\mathbf{p}_\Lambda=0$,就是说$\Lambda$是一个旋转$\mathscr{R}$,于是$W(\mathscr{R},p)=\mathscr{R}$.此时我们就可以把\eqref{xishu2}变成
\[
	\begin{split}
		\sum_{\bar{\sigma}}u_{\bar{l}}(0,\bar{\sigma},n)D^{(j_n)}_{\bar{\sigma}\sigma}\left(\mathscr{R}\right) & =\sum_{l} D_{\bar{l}l}(\mathscr{R})u_{l}(0,\sigma,n),\\
		\sum_{\bar{\sigma}}v_{\bar{l}}(0,\bar{\sigma},n)D^{(j_n)*}_{\bar{\sigma}\sigma}\left(\mathscr{R}\right) & =\sum_{l} D_{\bar{l}l}(\mathscr{R})v_{l}(0,\sigma,n),
	\end{split}
\]
这样,我们就得到$u_{l}(0,\sigma,n)$以及$v_{l}(0,\sigma,n)$需要满足的条件。

从$D^{(j)}_{\bar{\sigma}\sigma}\left(\mathscr{R}\right)$来看,这是标准的自旋$j$的不可约表示。任何Lorentz群的表示$D(\Lambda)$限制在旋转群就得到了旋转群的表示,上面两个式子告诉我们,如果场$\psi^\pm_l(x)$描述了自旋$j$的粒子,那么表示$D(\mathscr{R})$必须包含自旋$j$的不可约表示$D^{(j)}\left(\mathscr{R}\right)$,其中系数$u_{l}(0,\sigma,n)$和$v_{l}(0,\sigma,n)$描述了自旋$j$表示是如何嵌入到$D(\mathscr{R})$中的。

把所有的一切都往$V(t)$里面套,得到了(使用多重指标记法)
\[
	V=\sum_{NM}\int \dd^3 p'\dd^3 p \sum_{\sigma',\sigma,n',n}a^\dag(\mathbf{p}',\sigma',n')a(\mathbf{p},\sigma,n)\mathscr{V}_{NM}(\mathbf{p}',\sigma',n';\mathbf{p},\sigma,n),
\]
其中,$\sigma'=\{\sigma'_1,\cdots,\sigma'_N\}$, $n'=\{n'_1,\cdots,n'_N\}$, $\sigma=\{\sigma_1,\cdots,\sigma_M\}$, $n=\{n_1,\cdots,n_M\}$.以及
\[
	\mathscr{V}_{NM}(\mathbf{p}',\sigma',n';\mathbf{p},\sigma,n)=\delta(\mathbf{p}'-\mathbf{p})\bar{\mathscr{V}}_{NM}(\mathbf{p}',\sigma',n';\mathbf{p},\sigma,n),
\]
其中
\[
	\bar{\mathscr{V}}_{NM}(\mathbf{p}',\sigma',n';\mathbf{p},\sigma,n)=(2\pi)^{3-3(N+M)/2}\sum_{l',l}g_{l';l}v_{l'}(\mathbf{p}',\sigma',n')u_l(\mathbf{p},\sigma,n).
\]
这自然满足集团分解原理。
\re{The cluster decomposition principle together with Lorentz invariance thus makes it natural that the interaction density should be constructed out of the annihilation and creation fields.}

最后,我们需要检查作用密度之间的对易关系,使得他们可以在$x-y$是非类时的时候消失,作用密度是生成场和湮灭场的多项式,这就是说,我们需要检查
\[
	[\psi_l^+(x),\psi_{l'}^-(x)]_{\mp}=\frac{1}{(2\pi)^3}\sum_{\sigma,n}\int\dd^3 p\, u_{l}(\mathbf{p},\sigma,n)v_{l'}(\mathbf{p},\sigma,n)e^{ip\cdot (x-y)}
\]
是否在$x-y$非类时的时候为零。可是,事与愿违,一般来说,上式右边的积分并不能满足要求。这就是说,一般来说不能只用生成和湮灭场来构造场,我们只能去考虑他们的线性组合,如果线性组合满足对易关系,那么我们用他们的线性组合来构造多项式的时候也自然是满足的。设
$\psi_l(x)=\kappa_l\psi_l^+(x)+\lambda_l\psi_l^-(x)$,我们需要的关系就是
\[
	[\psi_l(x),\psi_{l'}(x)]_{\mp}=[\psi_l(x),\psi^\dag_{l'}(x)]_{\mp}=0
\]
在$x-y$非类时的时候成立。

上述关系往往被称为因果性关系,因为这是作用密度在类空的时候不能相互影响的假设下所希望满足的关系,但一般来说,存在无关因果性的情况,我们依然需要这个关系。所以我们只将其视作希望$S$矩阵满足Lorentz不变性的需要。

但是,在上述构造中,我们还是可能遇到问题。当粒子被生成或者被湮灭的时候,他带有一个或多个非零的守恒量子数比如荷(电荷什么的)。比如,如果种类为$n$的粒子带有$q(n)$的荷,荷算符为$Q$,那么
\[
	a(\mathbf{p},\sigma,n)Q\Psi=Qa(\mathbf{p},\sigma,n)\Psi+q(n)a(\mathbf{p},\sigma,n)\Psi,
\]
这是因为湮灭掉了一个$q(n)$的粒子,态的总荷数减少了$q(n)$,生成类似。用对易关系表示即
\[
\begin{split}
	[Q,a(\mathbf{p},\sigma,n)]&=-q(n)a(\mathbf{p},\sigma,n),\\
	[Q,a^\dag(\mathbf{p},\sigma,n)]&=+q(n)a^\dag(\mathbf{p},\sigma,n).
\end{split}
\]
为了使$\mathscr{H}$和$Q$对易,也就是为了荷守恒,我们可以把场构造地满足
\[
\begin{split}
	[Q,\psi_l(x)]=-q_l\psi_l(x),\\
	[Q,\psi^\dag_l(x)]=+q_l\psi^\dag_l(x),
\end{split}
\]
这样,作用密度作为$\psi_l(x)$和$\psi^\dag_l(x)$的多项式$\psi^\dag_l(x)\psi_{l'}(x)$(使用了多重指标记法),自然需要让其满足
\[
	q_{l}-q_{l'}=0.
\]

现在假设湮灭场和生成场\eqref{chang}中关于粒子种类的下标只有一个,分别为$n$和$\bar{n}$,此时
\[
\begin{split}
	[Q,\psi^+_l(x)]=-q(n)\psi^+_l(x),\\
	[Q,\psi^-_l(x)]=+q(\bar{n})\psi^-_l(x),
\end{split}
\]
对其线性组合,有
\[
	[Q,\psi_l(x)]=-q(n)\kappa_l\psi_l^+(x)+q(\bar{n})\lambda_l\psi_l^-(x),
\]
显然当且仅当$q(\bar{n})=-q(n)$的时候才能成立
\[
	[Q,\psi_l(x)]=-q(n)\psi_l(x),
\]

% 上面的对易关系对于湮灭场$\psi^+_l(x)$成立当且仅当所有种类为$n$的粒子被携带着相同荷的$q(n)=q_l$的场所湮灭,反之对于生成场$\psi^-_l(x)$成立当且仅当所有种类为$\bar{n}$的粒子被携带着相同荷的$q(\bar{n})=-q_l$的场所生成。

已经看到,为了构造一个量子数守恒(比如电荷守恒)的理论,必须有两种种类的粒子带有这种量子数:如果湮灭场湮灭掉了一个种类为$n$的粒子,那么就要用生成场生成一个有着相反的量子数的种类为$\bar{n}$的粒子(被称为反粒子)。这就是反粒子存在的原因。

因为一个可约表示可以分成几个不可约表示,因此,我们只要考虑不可约表示就可以了,此外,在Lorentz变换下,粒子种类不变,因此,我们不妨考虑单粒子种类的情况(去掉下标$n$),我们下面将具体构造几个场,确定系数$u,v$,以及求出满足条件的场算符。

稍稍谈及一下场方程,将$\psi_l(x)=\kappa_l\psi_l^+(x)+\lambda_l\psi_l^-(x)$和方程
\[
\begin{split}
	\psi_l^{+}(x)&=\sum_{\sigma,n}\left(\frac{1}{\sqrt{2\pi}}\right)^{3}\int \dd^3 p\, e^{ip\cdot x}u_{l}(\mathbf{p},\sigma,n)a(\mathbf{p},\bar{\sigma},n),\\
	\psi_l^{-}(x)&=\sum_{\sigma,n}\left(\frac{1}{\sqrt{2\pi}}\right)^{3}\int \dd^3 p\, e^{-ip\cdot x}v_{l}(\mathbf{p},\sigma,n)a^\dag(\mathbf{p},\bar{\sigma},n)
\end{split}
\]
一起考虑,可以得到
\[
\begin{split}
\square\psi_l^{+}=\partial_i\partial^i\psi_l^{+}&=\sum_{\sigma,n}\left(\frac{1}{\sqrt{2\pi}}\right)^{3}\int \dd^3 p\,(\partial_i\partial^i e^{ip\cdot x})u_{l}(\mathbf{p},\sigma,n)a(\mathbf{p},\bar{\sigma},n)\\
&=-\sum_{\sigma,n}\left(\frac{1}{\sqrt{2\pi}}\right)^{3}\int \dd^3 p\,p^ip_i e^{ip\cdot x}u_{l}(\mathbf{p},\sigma,n)a(\mathbf{p},\bar{\sigma},n),
\end{split}
\]
但是$p^ip_i=p^2=-m^2$,于是
\[
\square\psi_l^{+}=m^2\psi_l^{+},
\]
同理$\square\psi_l^{-}=m^2\psi_l^{-}$,所以我们的场$\psi_l(x)$就满足所谓的Kelin-Gordon场方程
\[
	(\square-m^2)\psi_l(x)=0.
\]
可以看到这个场方程只不过是我们构造的副产品。

\section{Causal Scalar Fields}
我们首先考虑最简单的标量表示$D(\Lambda)=1$,此时
\[
	\begin{split}
		\sum_{\bar{\sigma}}u_{l}(0,\bar{\sigma})D^{(j)}_{\bar{\sigma}\sigma}\left(\mathscr{R}\right)&=u_{l}(0,\sigma),\\
		\sum_{\bar{\sigma}}v_{l}(0,\bar{\sigma})D^{(j)*}_{\bar{\sigma}\sigma}\left(\mathscr{R}\right)&=v_{l}(0,\sigma),
	\end{split}
\]
这种情况下,只能有$D^{(j)}_{\bar{\sigma}\sigma}=\delta_{\bar{\sigma}\sigma}$,因此,这对应着$j=0$的不可约表示,于是$\sigma=0$,我们下面去掉指标$\sigma$和$l$,取
\[
	u(0)=v(0)=\frac{1}{\sqrt{2m}},
\]
因此
\[
	u(\mathbf{p})=v(\mathbf{p})=\frac{1}{\sqrt{2p^0}},
\]
湮灭场此时就是
\[
	\phi_l^{+}(x)=\left(\frac{1}{\sqrt{2\pi}}\right)^{3}\int \frac{\dd^3 p}{\sqrt{2p^0}}\, e^{ip\cdot x}a(\mathbf{p}),
\]
对应的生产场满足$\phi_l^{-}(x)=(\phi_l^{+}(x))^\dag$.

为了满足因果性条件,我们做计算
\[
	[\phi^{+}(x),\phi^{-}(y)]_{\mp}=\frac{1}{(2\pi)^{3}}\int\frac{\dd^3 p\,\dd^3 p'}{\sqrt{2p^0}\sqrt{2p'^0}}\,e^{ip\cdot x-p'\cdot y}\delta(\mathbf{p}-\mathbf{p'})=\Delta_+(x-y),
\]
其中
\[
\Delta_+(x)=\frac{1}{(2\pi)^{3}}\int\frac{\dd^3 p}{2p^0}\,e^{ip\cdot x}=\frac{m}{4\pi^2\sqrt{x^2}}\int_0^\infty\frac{u\dd u}{\sqrt{u^2+1}}\sin (m\sqrt{x^2} u).
\]
$\Delta_+(x)$显然是Lorentz不变的,但他不为0,因此
$[\phi^{+}(x),\phi^{-}(y)]_{\mp}=0$并不能满足需求。

按照一般的方法,我们做线性组合$\phi(x)=\kappa\phi^+(x)+\lambda\phi^-(x)$,注意到当$x^2>0$(即类空的时候),$\Delta_+(x)$是一个偶函数,然后当$x^2>0$时
\[
\begin{split}
	[\phi(x),\phi^\dag(y)]_{\mp}&=(|\kappa|^2\mp |\lambda|^2)\Delta_+(x-y),\\
	[\phi(x),\phi(y)]_{\mp}&=\kappa\lambda(1\mp 1)\Delta_+(x-y).
\end{split}	
\]
于是从后一个等式可以看到,零自旋的粒子只能是玻色子,应该取负号,再看第一个等式就可以得到$|\kappa|^2-|\lambda|^2=0$,适当调整生成和湮灭算子的相位\footnote{这是可以做到的,虽然定义相位同时改变生成和湮灭算子的相位,但是因为他们是共轭的,所以一个加一个减。},我们可以得到$\kappa=\lambda=k$,然后把常数$k$并入场算符,那么可以让$\kappa=\lambda=1$,于是
\[
	\phi(x)=\phi^+(x)+\phi^-(x)=\phi^\dag(x),
\]
就是我们要求的场算符,具体写开来就是
\[
	\phi(x)=\left(\frac{1}{\sqrt{2\pi}}\right)^{3}\int \frac{\dd^3 p}{\sqrt{2p^0}}\, (e^{ip\cdot x}a(\mathbf{p})+e^{-ip\cdot x}a^\dag(\mathbf{p})).
\]

如果有荷守恒的要求,那么,直接构造
\[
	\phi(x)=\kappa\phi^+(x)+\lambda\phi^{c-}(x),
\]
且$[Q,\phi^+]=-q\phi^+$和$[Q,\phi^{c+}]=q\phi^{c+}$,那么自然满足$[Q,\phi]=-q\phi$。其中$\phi^{c-}$是我们为了保证荷守恒而不得不新引入的生成场,他生成的粒子具有$-q$的荷,即这个场其实是反粒子的生成场。剩下的就是重复上面的推导,一样可以知道,粒子是玻色子,而系数可以取作$\kappa=\lambda=1$.
\section{Causal Vector Fields}
随后是矢量表示,$D(\Lambda)^\mu_{\phantom{\mu}\nu}=\Lambda^\mu_{\phantom{\mu}\nu}$.
\section{The Dirac Formalism}

\section{Causal Dirac Fields}

\section{General Irreducible Representation of the Homogeneous Lorentz Group}

\section{General Causal Fields}

\section{The CPT Theorem}

\section{Massless Particle Fields}
\addtocounter{chapter}{1}
\chapter{The Canonical Formalism}
Lagrange形式化使得我们可以比较容易分析对称性(不仅仅是Lorentz不变性):一个有着Lorentz不变性的Lagrange密度通过正则量子化将导致一个Lorentz不变的量子理论。就是说,这个理论允许合适的量子力学算子构造使得满足Poincar\'{e}代数,而这将导出Lorentz不变的$S$矩阵。

当然这不是那么平凡的,我们常常要给作用密度加上一个修正项(为了消去传播子的非协变项),而有着标量Lagrange密度的正则量子化手段将自动提供这一修正项。

\section{Cononical Variables}
\re{The purpose of the present section is to identity the canonical fields and their conjugates in various field theories, to tell us how to separate the free--field terms in the Lagrangian, and incidentally to reassure us that the conanical formalism is indeed applicable to physically realistic theories.}

我们将看到第五章里面构造的自由场将自动提供量子算符$q^n(\mathbf{x},t)$和$p_n(\mathbf{x},t)$满足熟悉的正则对易关系
\[
	\begin{split}
		[q^n(\mathbf{x},t),q^m(\mathbf{x},t)]_{\mp}&=0,\\
		[p_n(\mathbf{x},t),p_m(\mathbf{x},t)]_{\mp}&=0,\\
		[q^n(\mathbf{x},t),p_m(\mathbf{x},t)]_{\mp}&=i\delta^3(\mathbf{x}-\mathbf{y})\delta^n_m.\\
	\end{split}
\]
其中$-$对应Bose子,而$+$对应Fermi子,可以看到$q$和$p$调换也是一组正则变量。

对于零自旋的自共轭(反粒子是自己)标量场$\phi(x)$,则
\[
	[\phi(x),\phi(y)]_-=\Delta(x-y),
\]
其中
\[
	\Delta(x)=\int\frac{\dd^3 k}{2k^0 (2\pi)^3}(e^{ik\cdot x}-e^{-ik\cdot x}),
\]
那么,$q(\mathbf{x},t)=\phi(\mathbf{x},t)$和$p=\dot{\phi}(\mathbf{x},t)$将给正则对易关系。

对于自旋为1的实矢量场$v^\mu(x)$,我们有
\[
	[v^\mu(x),v^\nu(y)]_-=\left(\eta^{\mu\nu}-\frac{\partial^\mu\partial^\nu}{m^2}\right)\Delta(x-y),
\]
那么
\[
	\begin{split}
	q^i(\mathbf{x},t)&=v^i(\mathbf{x},t),\\
	p_i(\mathbf{x},t)&=\dot{v}^i(\mathbf{x},t)+\frac{\partial v^0(\mathbf{x},t)}{\partial x^i}
	\end{split}
\]
将给正则对易关系。

对于关于非Majorana的自旋$1/2$粒子的Dirac场$\psi_n(x)$,那么
\[
	\begin{split}
	q^n(x)&=\psi_n(x),\\
	p_n(x)&=i\psi^\dag_n(x)
	\end{split}
\]
将给出正则对易关系。

我们下面定义量子力学的变分算符,对于任意bosonic泛函$F[q(t),p(t)]$,在固定的$t$定义对于$q^n(\mathbf{x},t)$和$p_n(\mathbf{x},t)$的变分如下:
\[
	\begin{split}
	\frac{\delta F[q(t),p(t)]}{\delta q^n(\mathbf{x},t)}&=i\left[p_n(\mathbf{x},t),F[q(t),p(t)]\right],\\
	\frac{\delta F[q(t),p(t)]}{\delta p_n(\mathbf{x},t)}&=i\left[F[q(t),p(t)],q^n(\mathbf{x},t)\right],
	\end{split}
\]
所以对于全变分,我们有
\[
	\delta F[q(t),p(t)]=\int \dd^3 x\sum_n\left(\delta q^n(\mathbf{x},t)\frac{\delta F[q(t),p(t)]}{\delta q^n(\mathbf{x},t)}+\frac{\delta F[q(t),p(t)]}{\delta p_n(\mathbf{x},t)}\delta p_n(\mathbf{x},t)\right).
\]

特别地,我们考虑伴随作用,$H_0$作为自由粒子的时间生成元,我们有
\[
\begin{split}
	q^n(\mathbf{x},t)&=e^{iH_0t}q^n(\mathbf{x},0)e^{-iH_0t},\\
	p_n(\mathbf{x},t)&=e^{iH_0t}p_n(\mathbf{x},0)e^{-iH_0t},
\end{split}
\]
求一下导数就有
\[
\begin{split}
	\dot{q}^n(\mathbf{x},t)&=i[H_0,q^n(\mathbf{x},t)]=\frac{\delta H_0}{\delta p_n(\mathbf{x},t)},\\
	\dot{p}_n(\mathbf{x},t)&=-i[p_n(\mathbf{x},t),H_0]=-\frac{\delta H_0}{\delta q^n(\mathbf{x},t)},
\end{split}
\]
这就是让人熟悉的正则方程。

自由粒子的Hamilton量为
\[
	H_0=\sum_{n,\sigma}\int \dd^3 k\,\,a^\dag(\mathbf{k},\sigma,n)a(\mathbf{k},\sigma,n)\sqrt{\mathbf{k}^2+m^2},
\]
对于实的标量场,相差一个常数项,我们可以将其写成
\[
	H_0=\int \dd^3 x\,\,\left(\frac{1}{2}p^2+\frac{m^2}{2}q^2+\frac{1}{2}(\nabla q)^2\right),
\]
一般的教科书会将前者看成后者的推论。但对于我们来说是反过来的,如果Hamilton量不和前者差一个常数项,则将不会给出正确的Lagrange量。

使用Legendre变换,我们可以得到自由粒子的Lagrange量
\[
	L_0[q(t),\dot{q}(t)]=\sum_n \int \dd^3 x\,\,p_n(\mathbf{x},t) \dot{q}^n(\mathbf{x},t)-H_0,
\]
对于实的标量场,我们可以将其改写成
\[
	L_0=-\int \dd^3 x\,\,\left(\frac{1}{2}\partial_\mu\phi\,\partial^\mu\phi+\frac{m^2}{2}\phi^2\right).
\]

现在,我们考虑带有相互作用的情况,此时$H$将作为时间生成元,定义
\[
	\begin{split}
	Q^n(\mathbf{x},t)&=e^{iHt}q^n(\mathbf{x},0)e^{-iHt},\\
	P_n(\mathbf{x},t)&=e^{iHt}p_n(\mathbf{x},0)e^{-iHt},
	\end{split}
\]
由于这是一个相似变换,那么对易关系依旧成立
\[
	\begin{split}
		[Q^n(\mathbf{x},t),Q^m(\mathbf{x},t)]_{\mp}&=0,\\
		[P_n(\mathbf{x},t),P_m(\mathbf{x},t)]_{\mp}&=0,\\
		[Q^n(\mathbf{x},t),P_m(\mathbf{x},t)]_{\mp}&=i\delta^3(\mathbf{x}-\mathbf{y})\delta^n_m.\\
	\end{split}
\]
此外$H$和$e^{-iHt}$对易所以
\[
H[Q,P]=e^{iHt}H[q,p]e^{-iHt}=H[q,p].
\]
此时的正则方程写作
\[
\begin{split}
	\dot{Q}^n(\mathbf{x},t)&=i[H,Q^n(\mathbf{x},t)]=\frac{\delta H}{\delta P_n(\mathbf{x},t)},\\
	\dot{P}_n(\mathbf{x},t)&=-i[P_n(\mathbf{x},t),H]=-\frac{\delta H}{\delta Q^n(\mathbf{x},t)}.
\end{split}
\]
\section{The Lagrangian Formalism}
从经典场论开始,一般来说,Lagrange量$L[\Psi(t),\dot{\Psi}(t)]$是一个关于$\Psi(\mathbf{x},t)$和$\dot{\Psi}(\mathbf{x},t)$的泛函,那么他的共轭场$\Pi(\mathbf{x},t)$被定义为Lagrange量的变分
\[
	\Pi(\mathbf{x},t)=\frac{\delta L[\Psi(t),\dot{\Psi}(t)]}{\delta \dot{\Psi}(\mathbf{x},t)},
\]
运动方程写成
\[
	\dot{\Pi}(\mathbf{x},t)=\frac{\dd}{\dd t}\frac{\delta L[\Psi(t),\dot{\Psi}(t)]}{\delta \dot{\Psi}(\mathbf{x},t)}=\frac{\delta L[\Psi(t),\dot{\Psi}(t)]}{\delta \Psi(\mathbf{x},t)}.
\]

我们常常使用作用量原理将其改写,我们称
\[
	I[\Psi]=\int_{-\infty}^\infty \dd t\,\, L[\Psi(t),\dot{\Psi}(t)]
\]
为作用量,作用量原理即作用量变分为0,也就是
\[
	0=\delta I[\Psi]=\int \dd^4x\,\left(\frac{\delta L}{\delta \Psi(\mathbf{x},t)}\delta \Psi(t)+\frac{\delta L}{\delta \dot{\Psi}(\mathbf{x},t)}\delta\dot{\Psi}(t)\right),
\]
假设在无穷远的边界为0,此时分部积分得到
\[
	0=\delta I[\Psi]=\int \dd^4x\,\left(\frac{\delta L}{\delta \Psi(\mathbf{x},t)}-\frac{\dd}{\dd t}\frac{\delta L}{\delta \dot{\Psi}(\mathbf{x},t)}\right)\delta \Psi(t),
\]
这就是我们的运动方程了。

从作用量原理出发,我们可以看到,如果Lagrange量存在一个全导数的增量$\partial_\mu \mathscr{F}^\mu$或者关于时间的全导数项$\dd F/\dd t$,我们的方程并没有什么影响。前者是因为无穷远为0的自然边界条件,后者并不那么显然,可以参考原书以及Landau第一卷。

由于场方程完全由泛函$I[\Psi]$确定,那么我们自然尝试去构造一个Lorentz不变理论使得$I[\Psi]$是一个标量,特别地,因为$I[\Psi]$是$L[\Psi(t),\dot{\Psi}(t)]$对时间的积分,我们自然尝试让$L$本身就是一个关于$\Psi$和$\Psi$一阶导数的标量函数$\mathscr{L}$对三维坐标的积分,
\[
	I[\Psi]=\int \dd^4x\,\, \mathscr{L}\left(\Psi(x),\{\partial_\mu \Psi(x)\}\right),
\]
其中$\{\partial_\mu \Psi(x)\}$代表一族由一阶偏导数构成的矢量,这样的$\mathscr{L}$被称为Lagrange密度。直接像上面一样,求一下变分然后分部积分就得到了场方程的形式
\[
	\frac{\partial}{\partial x^\mu}\frac{\partial \mathscr{L}}{\partial \left(\partial_\mu \Psi\right)}=\frac{\partial \mathscr{L}}{\partial \Psi},
\]
左边是标量积,右边又是和参考系无关的标量,所以这就是一个Lorentz不变的方程。

由于复标量总可以拆成实部和虚部的实标量,所以下面我们只考虑作用量是实的情况。使用Legendre变换,我们就可以从Lagrange量得到Hamilton量
\[
	H=\sum_l \int \dd^3 x \,\, \Pi_l(\mathbf{x},t)\dot{\Psi}^l(\mathbf{x},t)-L[\Psi(t),\dot{\Psi}(t)],
\]
这样独立变量就变成了$\Pi_l$和$\Psi^l$,很容易计算得到正则方程
\[
\begin{split}
	\dot{\Psi}^l(\mathbf{x},t)&=\frac{\delta H}{\delta \Pi_l(\mathbf{x},t)},\\
	\dot{\Pi}_l(\mathbf{x},t)&=-\frac{\delta H}{\delta \Psi^l(\mathbf{x},t)}.
\end{split}
\]

现在考虑Lagrange密度
\[
	\mathscr{L}=-\frac{1}{2}\partial_\mu\Psi\,\partial^\mu\Psi-\frac{m^2}{2}\Psi^2-\mathscr{H}(\Psi),
\]
带入场方程,我们有
\[
	\frac{\partial}{\partial x^\mu}\frac{\partial \mathscr{L}}{\partial \left(\partial_\mu \Psi\right)}=-\partial_\mu\partial^\mu\Psi=\frac{\partial \mathscr{L}}{\partial \Psi}=-m^2\Psi-\mathscr{H}'(\Psi),
\]
或者
\[
	(\square-m^2)\Psi=(\partial_\mu\partial^\mu-m^2)\Psi=\mathscr{H}'(\Psi).
\]
此时的对偶场为
\[
	\Pi=\frac{\partial \mathscr{L}}{\partial \dot{\Psi}}=\dot{\Psi},
\]
所以Hamilton量写作
\[
	H_0=\int \dd^3 x\,\,(\Pi\dot{\Psi}-\mathscr{L})=\int \dd^3 x\,\,\left(\frac{1}{2}\Pi^2+\frac{m^2}{2}\Psi^2+\frac{1}{2}(\nabla \Psi)^2+\mathscr{H}(\Psi)\right),
\]
量子化为标量场的方程,只要直接替换$\Pi$为$Q$,$\Psi$为$P$就是了。

但事情不是总像标量场那么简单,比如对于关于非Majorana的自旋$1/2$粒子的Dirac场$\psi_n(x)$,我们已经知道
\[
	\begin{split}
	q^n(x)&=i\psi^\dag_n(x),\\
	p_n(x)&=\psi_n(x),
	\end{split}
\]
我们并不能直接找到$q^n$的共轭变量,因为Lagrange量并不显含$\psi^\dag_n(x)$对时间的导数,对他求变分只能为0。

回到Lagrange形式化,我们要假设部分Lagrange量并不显含部分变量$\dot{\Psi}^l$,我们将这些变量记做$C^r$,其他的记做$Q^n$,那么$Q^n$有共轭变量
\[
	P_n(\mathbf{x},t)=\frac{\delta L[Q(t),\dot{Q}(t),C(t)]}{\delta \dot{Q}^n(\mathbf{x},t)},
\]
使用Legendre变换,我们就可以从Lagrange量得到Hamilton量
\[
	H=\sum_n \int \dd^3 x \,\, P_n\dot{Q}^n-L[Q(t),\dot{Q}(t),C(t)],
\]
但因为我们的独立变量是$P$和$Q$,所以我们还要把$C^r$和$\dot{Q}^n$用$P$和$Q$来表示,此时关于$C^r$的运动方程写作
\[
	\frac{\delta}{\delta C^r(\mathbf{x},t)}L[Q(t),\dot{Q}(t),C(t)]=0.
\]
\end{document}