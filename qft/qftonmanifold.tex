\documentclass[12pt]{article}
\usepackage{../noteheader}
\definecolor{shadecolor}{rgb}{0.92,0.92,0.92}

\newcommand{\no}[1]{{$(#1)$}}
% \renewcommand{\not}[1]{#1\!\!\!/}
\newcommand{\rr}{\mathbb{R}}
\newcommand{\zz}{\mathbb{Z}}
\newcommand{\aaa}{\mathfrak{a}}
\newcommand{\pp}{\mathfrak{p}}
\newcommand{\mm}{\mathfrak{m}}
\newcommand{\dd}{\mathrm{d}}
\newcommand{\oo}{\mathcal{O}}
\newcommand{\calf}{\mathcal{F}}
\newcommand{\calg}{\mathcal{G}}
\newcommand{\bbp}{\mathbb{P}}
\newcommand{\bba}{\mathbb{A}}
\newcommand{\osub}{\underset{\mathrm{open}}{\subset}}
\newcommand{\csub}{\underset{\mathrm{closed}}{\subset}}

\DeclareMathOperator{\im}{Im}
\DeclareMathOperator{\Hom}{Hom}
\DeclareMathOperator{\id}{id}
\DeclareMathOperator{\rank}{rank}
\DeclareMathOperator{\tr}{tr}
\DeclareMathOperator{\supp}{supp}
\DeclareMathOperator{\coker}{coker}
\DeclareMathOperator{\codim}{codim}
\DeclareMathOperator{\height}{height}
\DeclareMathOperator{\sign}{sign}

\DeclareMathOperator{\ann}{ann}
\DeclareMathOperator{\Ann}{Ann}
\DeclareMathOperator{\ev}{ev}

\theoremstyle{definition}
	\newtheorem{para}{}[part]
		\renewcommand{\thepara}{\arabic{para}}
\theoremstyle{plain}
	\newtheorem{thm}[para]{Theorem}
	\newtheorem{pro}[para]{Proposition}
	\newtheorem{lem}[para]{Lemma}

\title{Quantum Mechanics on Manifold}
\author{hhh}

\begin{document}

\maketitle

In special relativity, we use the symmetry group of spacetime to classify particles. However, when the spacetime is no longer \textit{flat}, the symmetry group of spacetime is not the Poincar\'e group. What's more, we should be careful to deal with translation on general spacetime that depends on the connection we selected. Therefore, we cannot directly conclude that the Hamiltonian or momentum operators are the infinitesimal translations which belong to Poincar\'e group.

In this article, we will give a method to generalize the symmetry method in quantum mechanics and develop a kind of theory of quantum mechanics on manifold.

\section{Background}

Suppose the spacetime is a semi-Riemannian manifold $M$ with dimension $N$, $\mathcal H$ is a Hilbert space to describe our quantum system. As we all konw, there's a Levi-Civita connection on $M$ which is denoted by $\nabla$.

In special relativity, the choose of the origin of coordinates does not bother us because we could remove this trouble by considering the action of Poincar\'e group. However, on the general spacetime $M$, transitions may depend on the choosen origin and path. Thus, it is convenient to denote  the origin of coordinates on each state vector or the whole Hilbert space $\mathcal H$, or formally, we will consider the a vector bundle $\pi:\mathsf H\to M$ whose fibers are $\mathcal H_x:=\pi^{-1}(x)\cong \mathcal H$. The construction of $\pi:\mathsf H\to M$ is not clear now.

On the vector bundle $\pi:\mathsf H\to M$, we need a connection $D$ to tell us how to translate a state vector along a path on $M$. Suppose $\psi$ is a local trivialization of $\mathsf H$ near $x\in M$ and $\gamma:[0,1]\to M$ is a smooth path on $M$ such that $\gamma(0)=x$, $\dot\gamma(0)=v$ and $\operatorname{im}(\gamma)$ is near $x$. Locally, suppose $\Psi$ is a section of $\mathsf H$ near $x$ and $\psi(\Psi_y)=(y,\psi_y\Psi_y)\in U\times \mathcal H$, then
\[
	\psi_{\gamma(t)}\left(D_{\dot\gamma(t)}\Psi(\gamma(t))\right)=\dot c^n(t)\Psi_n+\Gamma_{im}^n\dot\gamma^i(t)c^m\Psi_n,
\]
where $\{\Psi_n\}$ is a basis of $\mathcal H$. If $\Psi(\gamma(t))$ is parallel along $\gamma$, i.e. $D_{\dot\gamma(t)}\Psi(\gamma(t))=0$, then
\[
	\dot c^n+\Gamma_{im}^n\dot\gamma^i(t)c^m=0.
\]
For short time $t$, we have
\[
	c^n(t)=c^n(0)+t\dot c^n(0)+O(t^2)=c^n(0)-t\Gamma_{im}^n(x)v^i c^m(0)+O(t^2).
\]
Thus the infinitesimal translation is
\[
	\psi_{\gamma(t)}\Psi(\gamma(t))-\psi_{x}\Psi(x)=-t\Gamma_{im}^n(x)v^i c^m(0)\Psi_n+O(t^2)=-t\Gamma_{m}^n(v)c^m(0)\Psi_n+O(t^2),
\]
where $\Gamma^n_m$ is the connection $1$-form. If we introduce a new operator $\Gamma(v)$ on $\mathcal H$ whose matrix is $\Gamma_{m}^n(\dot\gamma(0))$ such that
\[
	\Gamma(v)\Psi_m:=\Gamma_{m}^n(v)\Psi_n,
\]
we could get the infinitesimal translation of state vector $\Psi$ that
\[
	\psi_{\gamma(t)}\Psi(\gamma(t))-\psi_{x}\Psi(x)=-t\Gamma(v)c^m(0)\Psi_m+O(t^2)=-t\Gamma(v) \psi_{x}\Psi(x)+O(t^2),
\]
or
\[
	\left. \frac{\mathrm d}{\mathrm dt}\right|_{t=0}\psi_{\gamma(t)}\Psi(\gamma(t))=-\Gamma(v) \psi_{x}\Psi(x).
\]

It's usually more convenient to work on the space on sections. Suppose $\{\Psi_n\}$ is a basis of $\mathcal H$, and $\psi$ is a local trivialization near $x$, then $y\mapsto \psi_y^{-1}\Psi_n$ is a local section of $\pi:\mathsf H\to M$ near $x$, we will denote it by $\widetilde \Psi_n$. Define $\widetilde\Gamma(v):\mathcal H_x\to \mathcal H_x$ by
\[
	\widetilde\Gamma(v)\widetilde \Psi_m(x)=\Gamma^n_m(v)\widetilde \Psi_n(x),
\]
it relates to $\Gamma(v)$ by $\widetilde\Gamma(v)=\psi_x^{-1}\Gamma(v)\psi_x$. Therefore, the infinitesimal translation of state vector can be written as
\[
	\left. \frac{\mathrm d}{\mathrm dt}\right|_{t=0}\psi_{\gamma(t)}\Psi(\gamma(t))=- \psi_{x}\widetilde\Gamma(v)\Psi(x).
\]

\begin{pro}\label{pro:1}
	The infinitesimal translation is constant on a geodesic if and only if 1-forms $\Gamma^n_m$ satisfy the Killing equation.
\end{pro}

\begin{proof}
For geodesic $\gamma(t)$, we have
\[
	0=\frac{\mathrm d}{\mathrm dt}\Gamma(\dot\gamma(t))=\Gamma_{i;j}\dot\gamma^i\dot\gamma^j+\Gamma_{i}\ddot\gamma_{j}^i=\Gamma_{i;j}\dot\gamma^i\dot\gamma^j.
\]
The above equation is valid for all $\gamma(t)$, then $\Gamma_{i;j}+\Gamma_{j;i}=0$. This is just the Killing equation.
\end{proof}

Define the momentum operator $P_\mu=i\hbar \widetilde{\Gamma}(\partial_\mu)$, then
\[
	\hbar^2 R_D(\partial_\mu,\partial_\nu)^m_n=([P_\mu,P_\nu])^m_n+\hbar^2(\partial_\mu\Gamma_\nu - \partial_\mu\Gamma_\nu)^m_n,
\]
or
\begin{equation}
	\hbar^2 R_D(\partial_\mu,\partial_\nu)^m_n=([P_\mu,P_\nu])^m_n-\hbar^2(\Gamma_{\mu;\nu} - \Gamma_{\nu;\mu})^m_n.
\end{equation}

Equation (\theequation) suggests us that
\[
	\Gamma_{\mu;\nu}=\Gamma_{\nu;\mu}.
\]
What's more, if we want $P_\mu(\gamma(t))$ is constant along $\gamma(t)$, Proposition \ref{pro:1} suggests us that
\[
	\Gamma_{\mu;\nu}+\Gamma_{\nu;\mu}=0.
\]
With these two equation, we should assume that $\Gamma_{\mu;\nu}=0$, and then
\[
	[P_\mu,P_\nu]=\hbar^2 R_D(\partial_\mu,\partial_\nu)
\]
is the commutator between momentum operators, where $R_D$ depends on the connection $D$. We want to derive $R_D$ from the Riemannian curvature $R_\nabla$ of the Levi-Civita connection of the spacetime $M$ by using quantumization. In this case, if the curvature of spacetime is non-zero, so is $[P_\mu,P_\nu]$.

Suppose $S$ is a section of $\mathsf{H}^*\otimes \mathsf{H}$ and $D_\mu(S)=0$ for some $\mu$, then
\[
	0=D_\mu S= \partial_\mu S+[\widetilde{\Gamma}(\partial_\mu),S]
\]
or
\[
	\partial_\mu S=\frac{i}{\hbar}[P_\mu,S],
\]
it is the Heisenberg equation of motion. From now on, we will set $\hbar=1$.

The above observation suggests us that the right equation of evolution of a operator $S\in \Gamma (\mathsf{H}^*\otimes \mathsf{H})$ is $D_{\dot\gamma(t)}S=0$, where $\gamma(t)$ is a geodesic.

% For fixed $A:\mathbb R^N\to\mathbb R^N$, suppose $U_x(A)$ satisfies the equation of motion $D_{\dot\gamma(t)}U_{\gamma(t)}(A)$ for all geodesic $\gamma$. Then the equation
% \[
% 	\partial_\mu U_{x}(A)=-[\widetilde{\Gamma}_\mu,U_{x}(A)]
% \]
% holds for all $\mu$. 

% Now suppose the transition map $g_x=\varphi'_{*x}\varphi_{*x}^{-1}$ is independent of $x$, then that
% \begin{align*}
% 	U_x(g_x)\widetilde{\Gamma}_\mu&=\partial_\mu U_x(g_x)+\widetilde{\Gamma'}_\mu U_x(g_x)\\
% 	&=-[\widetilde{\Gamma}_\mu,U_{x}(g_x)]+\widetilde{\Gamma'}_\mu U_x(g_x)\\
% 	&=U_x(g_x)\widetilde{\Gamma}_\mu+(\widetilde{\Gamma'}_\mu-\widetilde{\Gamma}_\mu)U_x(g_x)
% \end{align*}
% tells us that $\widetilde{\Gamma'}_\mu=\widetilde{\Gamma}_\mu$ for this change!

\section{Gauge}

Suppose $\psi$ and $\psi'$ are trivialization near $x$ and then induce two linear map $\psi_x$, $\psi'_x:\mathcal H_x\to \mathcal H$. The different trivialization should not affect the physics because of the principle of relativity. Especially, probability from this quantum system would be invariant under the map $\psi'_x\psi^{-1}_x:\mathcal H\to \mathcal H$, i.e. for any state vectors $\Phi$, $\Psi\in \mathcal H$,
\[
	|(\Phi,\Psi)|^2=\bigl|\bigl(\psi'_x\psi^{-1}_x\Phi,\psi'_x\psi^{-1}_x\Psi\bigr)\bigr|^2.
\]
Thus, the well-known Wigner theorem tells us that $\psi'_x\psi^{-1}_x:\mathcal H\to \mathcal H$ is linear (anti-linear) and unitary (anti-unitary). 

For this reason, now we can define an `metric' on $\mathcal H_x$ by
\[
	(\Psi_x,\Phi_x)_x:=\left(\psi_x\Psi_x,\psi_x\Phi_x\right)_{\mathcal H},
\]
where $\psi$ is a trivialization, and it's not changed by taking another trivialization $\psi'$ because $\psi'_x\psi^{-1}_x$ is unitary. For locally contant sections $\widetilde \Psi:x\to \psi_x^{-1}\Psi$ and $\widetilde \Phi:x\to \psi_x^{-1}\Phi$, the definition above tells us that
\[
	(\widetilde \Psi_x,\widetilde \Phi_x)_x=(\Psi,\Phi)
\]
is independent of $x$.

Suppose $(V,\varphi)$ is a chart on $M$ containing $x$, where $\varphi$ is an isomorphism to an open subset of $\mathbb R^{N}$. If there's another chart $(V,\varphi')$, where $\varphi'$ is a diffeomorphism from $V$ to an open subset of $\mathbb R^{N}$, then we have a map from $\varphi(x)\in \mathbb R^N$ to $\varphi'(x)\in \mathbb R^N$, which induces a map $U_x(\varphi'\varphi^{-1}):\mathcal H_x\to \mathcal H_x$. It's from two different ways to view the same quantum system.

The map $U_x(\varphi'\varphi^{-1}):\mathcal H_x\to \mathcal H_x$ should not affect the physics because of the principle of relativity. Therefore, the Wigner theorem tells us that $U_x(\varphi'\varphi^{-1}):\mathcal H_x\to \mathcal H_x$ is linear (anti-linear) and unitary (anti-unitary). It's convenient but not necessary to assume that 
\[
	U_x(\varphi''\varphi'^{-1})U_x(\varphi'\varphi^{-1})=U_x(\varphi''\varphi^{-1}),
\] 
because there should be one phase factor in the exact relation, just like the flat case. This additional phase factor may be vary important in physics, but let's forget it at first. It's also convenient (but not necessary) to assume $U_x(\varphi'\varphi^{-1})$ is linear in our disscussion.

It's natural to guess that $U_x(\varphi'\varphi^{-1})$ should only depend on the local imformation of $\varphi'\varphi^{-1}$ at $\varphi(x)\in \mathbb R^N$, so let's assume $U_x$ only dependent $a=\varphi'(x)-\varphi(x)$ and $\Lambda=(\varphi'\varphi^{-1})_{*\varphi(x)}$.

\section{Construction}

The above discussion is quite general since we don't need any detail about the bundle. If we want to relate the bundle trivialization $\psi:\pi^{-1}(V)\to V\times \mathcal H$ and the chart $(V,\varphi)$ on $M$, it's straightforward and naive to define a `function' between $\varphi(x)$ and $\psi_x:\mathcal H_x\to \mathcal H$ such that $\psi_x=T[\varphi(x)]:\mathcal H_x\to \mathcal H$. However, there exist two different charts $\varphi$ and $\varphi'$ such that $\varphi(x)=\varphi'(x)\in \mathbb R^N$, so the desired `function' should also depend on another imformation of $\varphi'$. 
From experience, one could guess that $\psi_x$ is a function of $\varphi(x)$ and $\varphi_{*x}$, or more formally, the 1-jet of $\varphi$. We will use $[\varphi](x)$ or $[\varphi]_x$ to denote the pair $(\varphi(x),\varphi_{*x})$.

Suppose $(V,\varphi)$ is a chart on $M$, we will define a bundle trivialization $T([\varphi]):\pi^{-1}(V)\to V\times \mathcal H$ by 
\[
	T([\varphi]):\Psi\mapsto (x,T_x(\varphi(x),\varphi_{*x})\Psi_x)=(x,T_x([\varphi])\Psi_x),
\]
where $x=\pi(\Psi)$ and $T$ is a unknown function determined by our quantum system. In this case, the whole structure of the bundle $\mathsf H$ is given by $T$, just like the tangent bundle, and one could guess the linear map
\begin{equation}
	U_x(\varphi'\varphi^{-1})=T_{x}([\varphi'])^{-1}T_{x}([\varphi]):\mathcal H_x\to \mathcal H_x
\end{equation}
and the unitary map
\[
	\psi_x U_x(\varphi'\varphi^{-1})\psi_x^{-1}=T_{x}([\varphi])T_{x}([\varphi'])^{-1}:\mathcal H\to \mathcal H.
\]
% $U_x$ only depends on $[\varphi'\varphi^{-1}]$ now, it's a translation plus a linear map, and then naturally written as $U_x(\Lambda,a)$. 

% Suppose $A:\mathbb R^N\to \mathbb R^N$ is a inversable function, then as a local section of $\mathsf H^*\otimes\mathsf H$,
% \[
% 	D_v(U_x(A))=\langle \dd U_x(A),v\rangle+[\widetilde{\Gamma}_x(v),U_x(A)]
% \]

One usually want his connection compatible with the `metric', i.e.
\[
	X(\Psi,\Phi)=(D_X\Psi,\Phi)+(\Psi,D_X\Phi),
\]
where $X$ is a vector field on $M$ and $D$ is our connection.

Suppose $\psi'$ is another local trivialization near $p$, then we will get
\[
	\Gamma_i(x)=G_x^{-1}\partial_i G_x+G^{-1}_x\Gamma'_i(x)G_x,
\]
where $G_x=\psi'_x\psi^{-1}_x$, or 
\[
	\Gamma'_i(x)=G_x\partial_i G_x^{-1}+G_x\Gamma_i(x)G_x^{-1}.
\]
If we want to construct a connection on $\mathsf H$, we could constant it locally to meet the relation above.

Recall that if $\psi_x=T([\varphi]_x)$ and $\psi'_x=T([\varphi']_x)$, then the above equation can be written as
\[
	\Gamma_i(x)=U(g_x)^{-1}\partial_iU(g_x)+U(g_x)^{-1}\Gamma'_iU(g_x)=U(g_x^{-1}\partial_ig_x)+U(g_x)^{-1}\Gamma'_i(x)U(g_x),
\]
where $g_x=\varphi'_{*x}\varphi_{*x}^{-1}$ is the transition map of tangent bundle $TM$ and $U$ is defined by equation (\theequation).

So if there's a `connection quantumization' method to construct connection from $\Gamma'_i$ from the Levi-Civita connection from $\Gamma^L$, i.e. $\Gamma'_i=U(\Gamma^L)$, then the relation of Levi-Civita connection that
\[
	\Gamma^L_i(x)=g_x^{-1}\partial_ig_x+g_x^{-1}{\Gamma'}^L_ig_x
\]
reduce that
\[
	\Gamma_i(x)=U(g_x^{-1}\partial_ig_x)+U(g_x)^{-1}\Gamma'_i(x)U(g_x),
\]
as we want.

\section{something}

Suppose $\hat p^\mu$ is momentum operators on $\mathcal H$, then we can define $\hat p^\mu_x:=\psi_x^{-1}\hat p^\mu\psi_x:\mathcal H_x\to \mathcal H_x$. Let $\Psi_{p}$ is a eigenstate of $\hat p^\mu_x$ with eigenvalue $p^\mu$, i.e.
\[
	\hat p^\mu_x\Psi_{p}=p^\mu\Psi_{p}.
\]
Given a geodesic $\gamma(\tau)$, $p^\mu(\tau):=m\dot\gamma^\mu(\gamma(\tau))$ satisfies
\[
	\dot p^\mu(\tau)+\frac{1}{m}A^\mu_{\alpha\beta}(\gamma(\tau))p^\alpha(\tau)p^\beta(\tau)=0,
\]
where $m$ is the mass and $A$ is the connection form of Levi-Civita connection. Then for a short `time' $\tau$,
\[
	p^\mu(\tau)=p^\mu-\frac{\tau}{m}A^\mu_{\alpha\beta}(x)p^\alpha p^\beta +O(t^2),
\]
where $x=\gamma(0)$ and $p^\mu=p^\mu(0)$.

Suppose $\Psi_{p}(\tau)$ is the state coming from the parallel transporting of $\Psi_{p}$ along the $\gamma(\tau)$, it should has the momentum $p^\mu(\tau)$, i.e.
\[
	\hat p^\mu_{\gamma(\tau)}\Psi_p(\gamma(\tau))=p^\mu(\tau)\Psi_p(\gamma(\tau)).
\]
For small $\tau$, we have that
\[
	\hat p^\mu
	\left(
		\psi_{x}\Psi_p-\frac \tau m p^\nu\Gamma_{\nu}\psi_{x}\Psi_p
	\right)
	=p^\mu(\tau)
	\left(
		\psi_{x}\Psi_p-\frac \tau m p^\nu\Gamma_{\nu}\psi_{x}\Psi_p
	\right)
	+O(\tau^2)
\]
and
\[
	\frac \tau m (\hat p^\mu-p^\mu )p^\nu\Gamma_{\nu}\psi_{x}\Psi_p=(p^\mu(\tau)-p^\mu )\psi_{x}\Psi_p+O(\tau^2),
\]
thus
\[
	(\hat p^\mu-p^\mu )p^\nu\Gamma_{\nu}\psi_{x}\Psi_p=-A^\mu_{\alpha\beta}(x)p^\alpha p^\beta \psi_{x}\Psi_p,
\]
or in another form that
\[
	[\hat p^\mu_x,\widetilde{\Gamma}_{\nu}]\hat p^\nu_x\Psi_p=-A^\mu_{\alpha\beta}(x)\hat p_x^\alpha \hat p_x^\beta \Psi_p.
\]
Then it's natural to guess that
\[
	[\widetilde{\Gamma}_{\nu},\hat p^\mu_x]=A^\mu_{\rho\nu}(x)\hat p_x^\rho.
\]

\textit{momentum operator $p^\mu$ should depend on the choice of chart!}

% we can assume that $\hat p^\mu$ satisfy the quation $D_\nu \hat p^\mu=0$, thus
% \[
% 	(\hat p^\nu(\gamma(t)))^m_n=\hat p^\nu(\gamma(0))-t\dot\gamma^\mu [{\Gamma}_\mu,\hat p^\nu]^m_n
% \]

% \[
% 	0=D_\mu p^\nu= \partial_\mu p^\nu+[\widetilde{\Gamma}_\mu,p^\nu]
% \]

for every geodesic $\gamma$, we give a state $\Psi_\gamma\in \mathcal H$. every information is in the $\Psi_\gamma\in \mathcal H$!
\end{document}