\chapter{基本构造}

\section{极限与余极限}

\para 设$\mathcal{J}$是一个小范畴(即范畴对象的全体能够构成一个集合),我们称函子$D:\mathcal{J}\to \mathcal{C}$为$\mathcal{C}$中的一个$\mathcal{J}$-图,或者简略叫做图。对于$j\in \mathcal{J}$,$D(j)$称为该图的一个顶点,而对任意的态射$\alpha:j_1\to j_2$,态射$D(\alpha)$称为该图的一条边。

\para 称$(A,\lambda)$为$J$-图的一个{锥形},如果$A$是$\mathcal{C}$的一个对象,而$\lambda$为一族态射$\lambda_{j}:A\to D(j)$使得如下交换图对所有的顶点和边都成立
\[
	\xymatrix{
		&A \ar[dl]_{\lambda_{j}}\ar[dr]^{\lambda_{j'}}&\\
		D(j)\ar[rr]^{D(\alpha)}&&D(j')
	}
\]

称呼一个锥形$(A,\lambda)$是$J$-图的一个{极限},如果对于任意的锥形$(B,\mu)$,都有唯一的态射$f:B\to A$使得如下分解$\mu_j:B\xrightarrow{f}A\xrightarrow{\lambda_j}D(j)$成立。

在实际过程中,如果极限$(A,\lambda)$中的态射族是明确的(乃至于是由对象$A$确定的),那么我们会用$A$来表示极限。并且一般将$J$-图$D$的极限写作$\varprojlim_{j\in J} D(j)$.

作为例子,模范畴的直积就是一种极限,他对应的小范畴,两个不同的元素之间没有态射。

\para 锥形和极限都有对偶的概念,在交换图中,无外乎就是将箭头完全反过来。称$(A,\lambda)$为$J$-图的一个{余锥形},如果$A$是$\mathcal{C}$的一个对象,而$\lambda$为一族态射$\lambda_{j}:D(j)\to A$使得如下交换图对所有的顶点和边都成立
\[
	\xymatrix{
		&A&\\
		D(j)\ar[rr]^{D(\alpha)} \ar[ur]^{\lambda_{j }}&&D(j')\ar[ul]_{\lambda_{j'}}
	}
\]

称呼一个余锥形$(A,\lambda)$是$J$-图的一个{余极限},如果对于任意的余锥形$(B,\mu)$,都有唯一的态射$f:A\to B$使得如下分解$\mu_j:D(j)\xrightarrow{\lambda_j}A\xrightarrow{f}B$成立。一般将$J$-图$D$的余极限写作$\varinjlim_{j\in J} D(j)$.

同样,作为例子,模范畴的直和就是一种余极限,他对应的小范畴,两个不同的元素之间没有态射。

\para 设$I$是一个偏序集,偏序关系为$\leq$,那么族$\{\{x\}\,:\, x\in I\}$构成一个(小)范畴,其中的态射定义为
\[
	\Hom_{I}\left(x,y\right)=\begin{cases}
	(x,y)&\text{, if }x\leq y\text{;}\\
	\varnothing&\text{, otherwise}.
	\end{cases}
\]

如果我们定义$x\geq y$当且仅当$y\leq x$,那么新的偏序集$(I,\geq)$就是偏序集$(I,\leq)$的对偶范畴,略去偏序符号,有时候会记作$I^{\mathrm{op}}$.

$D:I\to \mathcal{C}$是一个协变函子,就是对$i\leq j\leq k$成立$D(j\leq k)\circ D(i\leq j)=D(i\leq k)$,就是说从小到大有映射。反之,反变函子$D:I\to \mathcal{C}$对$i\leq j\leq k$成立$D(i\leq j)\circ D(j\leq k)=D(i\leq k)$,就是说从大到小有映射。

对于一个反变函子$D:I\to \mathcal{C}$,我们可以定义一个协变函子$D^{\mathrm{op}} :I^{\mathrm{op}}\to \mathcal{C}$通过$D^{\mathrm{op}}(j\geq i)=D(i\leq j)$,此时对$k\geq j\geq i$成立$D^{\mathrm{op}}(j\geq i)\circ D^{\mathrm{op}}(k\geq j)=D^{\mathrm{op}}(k\geq i)$,可见这确实是一个协变函子。

\para 称$I$是一个滤相的偏序集,即对于任意的$i$, $j\in I$都存在$k\in I$使得$k\leq i$和$k\leq j$同时成立。称$I$是一个定向的偏序集,即对于任意的$i$, $j\in I$都存在$k\in I$使得$i\leq k$和$j\leq k$同时成立。

很容易看到,如果偏序集$I$是滤相的(定向的),那么$I^{\mathrm{op}}$就是定向的(滤相的),反之亦然。

作为例子,考虑拓扑空间中所有非空开集按照包含构成的偏序集,即$U\leq V$当且仅当$U\subset V$,那么这个偏序集是定向的,但不是滤相的,因为对于任意的$U$和$V$总有$U\leq U\cup V$和$V\leq U\cup V$成立,但对于不交的$U$和$V$,并不存在非空开集同时包含于他们其中。

同样是拓扑的例子,考虑所有包含$p$的非空开集按照包含构成的偏序集,那么这个偏序集既是滤相的又是定向的。

\para 将滤相的(定向的)偏序集$I$看作一个范畴,对任意的范畴$\mathcal{C}$,如果一个$I$-图$D$,$D:I\to \mathcal{C}$是协变函子,则这称为一个$C$上的一个{逆系统}({定向系统})。逆系统一般谈论极限$\varprojlim_{i\in I} D(i)$,而定向系统一般谈论$\varinjlim_{i\in I} D(i)$,为了思考这个原因,我们考虑如下交换图
\[
	\xymatrix{
		&A \ar[dl]_{\lambda_{i}}\ar[dr]^{\lambda_{j}}\ar[dd]^{\lambda_{k}}&\\
		D(i)&&D(j)\\
		&D(k)\ar[ul]^{D(k\leq i)}\ar[ur]_{D(k\leq j)}&
	}
	\quad
% \xymatrix{
% 	&A \ar[dl]_{\lambda_{i}}\ar[dr]^{\lambda_{j}}\ar[dd]^{\lambda_{k}}&\\
% 	D(i)\ar[dr]_{D(i\leq k)}&&D(j)\ar[dl]^{D(j\leq k)}\\
% 	&D(k)&
% }
	\xymatrix{
		&A &\\
		\ar[ur]^{\mu_{i}}D(i)\ar[dr]_{D(i\leq k)}&&D(j)\ar[ul]_{\mu_{j}}\ar[dl]^{D(j\leq k)}\\
		&D(k)\ar[uu]_{\mu_{k}}&
	}
\]
左边是对逆系统考虑锥形,右边是对定向系统考虑余锥形。从左边来看,如果$i\leq j\leq k$成立,则$D(k)$构成了一个锥形的顶点,当我们考虑极限的时候,这时候极限就应该表现得像那些“极小”的元素$D(k)$一样。而且如果$A$是极限,那么箭头$A\to D(k)$也构成了唯一分解。类似地考虑右边的图,那些“极大”的元素$D(k)$构成了余锥形的顶点,如果$A$是余极限,那么箭头$D(k)\to A$也构成了唯一分解。

\para 考虑$I$是滤相的(定向的),而$D$是协变函子,那么$I$-图$D$是逆系统(定向系统)。考虑$I$是滤相的(定向的),而$D$是反变函子,那么$I^{\mathrm{op}}$-图$D^{\mathrm{op}}$是定向系统(逆系统)。

\para 设$I$是一个偏序集,如果$i\in I$有,$i>j$对所有$j\in I$不成立,或者说,与可以比较的元素$j\in I$都有$i\leq j$,则称$i$是$I$的一个极小元。如果$i\leq j$对所有$j\in I$都成立,则称$i\in I$是$I$的最小元。

在滤相的偏序集中,极小元和最小元等价。最小元是极小的,这是显然的。反之,因为如果存在极小元$i$,那么任取一个$j\in I$都存在一个$k\in I$使得$k\leq i$和$k\leq j$都成立,然而$i$是极小元,所以$i=k$,所以$i\leq j$.

同理我们可以定义极大元与最大元,在定向的偏序集中,此二者等价。

更广义地,对于一个偏序集$I$的子集$J$,如果对于任意的$i\in I$,都存在$j\in J$使得$j\leq i$,则称$J$和$I$是共尾的。显然,滤相的偏序集中的极小元构成的单点集就和原来的偏序集共尾。

\pro 设$I$-图$D$是一个逆系统,如果$I$存在一个与其共尾的子集$J$,则$J$-图$D$是一个逆系统,且$\varprojlim_{i\in I} D(i)=\varprojlim_{i\in J} D(i)$. 对偶地,设$I$-图$D$是一个定向系统,如果$I^{\mathrm{op}} $存在一个与其共尾的子集$J^{\mathrm{op}} $,则$J$-图$D$是一个定向系统,且$\varinjlim_{i\in I} D(i)=\varinjlim_{i\in J} D(i)$. \rule{2mm}{2mm}

所以对于逆系统,如果存在极小元,那么极小元就是它的极限,对偶地,对于定向系统,如果存在极大元,那么极大元就是它的对偶极限。这符合我们上面的直观。

\para 一个拓扑空间$X$能被看成一个范畴,对象取作他的所有开集,而态射取作
\[
	\Hom_{X}(U,V)=\begin{cases}
	\bigl\{i^U_V:U\hookrightarrow V\bigr\}&\text{, if }U\subset V\text{;}\\
	\varnothing&\text{, otherwise}.
	\end{cases}
\]

考虑拓扑空间$X$中一个非空开集族$\mathfrak{B}$,对于任意的两个$U$, $V\in \mathfrak{B}$,都存在一个$W\in \mathfrak{B}$使得$U\cup V\subset W$成立,这样的$\mathfrak{B}$按照包含构成了定向的偏序集。现在定义函子$i:\mathfrak{B}\to X$通过$i(U)=U$以及$i(U\leq V)=i^U_V:U\hookrightarrow V$,它显然成立复合$i(U\leq W)=i(V\leq W)\circ i(U\leq V)$,所以这个$\mathfrak{B}$-图$i$是一个定向系统。可以看到它的余极限$\varinjlim_{U\in \mathfrak{B}} U$实际上就是$\bigcup_{U\in \mathfrak{B}} U$,而态射族就是$i_U:U\hookrightarrow \bigcup_{U\in I}$.

同样考虑拓扑空间$X$中一个包含一个点$p\in X$的所有非空开集构成的族$\mathfrak{B}$,它按照包含构成一个滤相的偏序集。同样可以定义函子$i:\mathfrak{B}\to X$通过$i(U)=U$以及$i(U\leq V)=i^U_V:U\hookrightarrow V$,同样显然成立复合$i(U\leq W)=i(V\leq W)\circ i(U\leq V)$,所以这个$\mathfrak{B}$-图$i$是一个逆系统。

但它的极限$\varprojlim_{U\in \mathfrak{B}} U$就不一定存在,因为类比于上一个例子,它的极限应该类似于所有$\mathfrak{B}$中元素的交,但这不一定是一个开集。

\para 设我们有一个$X$上的$\mathcal{K}$-预层$\calf$,他是$X\to \mathcal{K}$的一个反变函子。赋予$X$一个偏序,此时函数$U\leq V$即$U\hookrightarrow V$. 那么$X$的那些包含$p\in X$的非空开集构成的子集族$\mathfrak{B}$继承了偏序结构,也是一个范畴,而预层$\calf$限制在$\mathfrak{B}$给出了反变函子$\mathfrak{B}\to \mathcal{K}$.

由于$\mathfrak{B}$是滤相的偏序集,而$\calf$是反变函子,因此$\mathfrak{B}^{\mathrm{op}}$-图$\calf^{\mathrm{op}}$是定向系统,进而我们会考虑余极限
\[
	\varinjlim_{U\in \mathfrak{B}^{\mathrm{op}}} \calf^{\mathrm{op}}(U)=\varinjlim_{U\in \mathfrak{B}^{\mathrm{op}}} \calf(U),
\]
如果这个余极限存在,那么就称为预层$\calf$在点$p$处的纤维,记作$\calf_p$.

\section{直和与直积}

直积和直和都属于用一些小的模来构造出大的模的手段。

\para 设$\{M_i\,:\, i\in I\}$是一族$R$-模,则存在一个$R$-模$M$以及一族同态$\pi_i:M\to M_i$使得,如果存在另一个模$N$和一族同态$\rho_i:N\to M_i$,那么就唯一存在同态$\rho:N\to M$使得分解$\rho_i:N\xrightarrow{\rho} M \xrightarrow{\pi_i} M_i$成立对任意的$i\in I$都成立。这样的一个模$M$被称为$\{M_i\,:\, i\in I\}$的直积,而$\pi_i$被称为典范投影,通常将$M$记作$\prod_{i\in I}M_i$. 直积对应于范畴里product的概念。

存在性的证明是简单的,考虑$\{M_i\,:\, i\in I\}$作为集合的直积,很容易检验它有一个$R$-模结构,且投影是同态。更详细的检查这里就略去了。

其他满足这个性质的模及同态,都可以唯一分解到这个典型的模和同态(这里是直积和投影)的性质,被称为泛性质。

\para 设$\{M_i\,:\, i\in I\}$是一族$R$-模,则存在一个$R$-模$M$以及一族同态$\pi_i:M_i\to M$使得,如果存在另一个模$N$和一族同态$\rho_i:M_i\to N$,那么就唯一存在同态$\rho:M\to N$使得分解$\rho_i:M_i\xrightarrow{\pi_i} M \xrightarrow{\rho} N$成立对任意的$i\in I$都成立。这样的一个模$M$被称为$\{M_i\,:\, i\in I\}$的直和,而$\pi_i$被称为典范内射,通常将$M$记作$\bigoplus_{i\in I}M_i$. 直和对应于范畴里coproduct的概念。

直和存在性的构造并不如同直积那么容易。我们考虑$\{M_i\,:\, i\in I\}$的形式和$M$,即
\[
	M=\left\{\sum_{i\in I} a_i m_i\,:\, m_i\in M_i\right\},
\]
其中的求和只有有限项,或者说系数$a_i$只有有限项非零。可以看到这是一个模。定义$\pi_i:M_i\to M$为$\pi_i(m_i)=m_i$,可以检验这是一个模同态。

现在任取$N$和一族同态$\rho_i:M_i\to N$,我们将其扩展为$\rho_i:M\to N$通过补充定义$\rho_i(m)=0$,如果$m\notin M_i$. 随后我们定义$\rho = \sum_{i\in I} \rho_i$,需要检验这个求和对任意的$m\in M$是有限的。为此,任取$m\in M$,由于他可以分解成$m=\sum_i a_i m_i$,所以
\[
	\rho(m)=\sum_i a_i \rho_i(m_i),
\]
求和是有限求和因为$a_i$只有有限项非零。有了构造,泛性质的检验就是直接的了。

注意到交换群为$\zz$-模,所以我们也证明了交换群有直和与直积。

\para 如果满足泛性质,则他在同构意义上唯一。

以直积为例,如果$(M,\pi_i)$和$(M',\pi'_i)$都满足泛性质,则利用$M$的泛性质,我们有唯一的同态$\rho:M'\to M$,利用$M'$的泛性质,我们有唯一的同态$\rho':M\to M'$. 然后$\rho \rho':M\to M$和$\id_M$是同时满足泛性质的分解,由唯一性,$\rho\rho'=\id_M$. 同理,$\rho'\rho=\id_{M'}$. 所以他们同构。

\para 由构造,有限直和与有限直积在模范畴里面是等价的。考虑两个$R$-模$M_1$和$M_2$,那么$M_1$可以看成$M_1\oplus M_2$的子模,通过$m\to (m,0)$. 反过来,设$M_1$和$M_2$是一个模$M$的两个子模,则我们有如下短正合列
\[
	0\to M_1\cap M_2 \xrightarrow{\mu} M_1\oplus M_2\xrightarrow{\nu} M_1+M_2\to 0,
\]
其中$\mu:m\mapsto (m,m)$,$\nu:(m_1,m_2)\mapsto m_1-m_2$. 所以在$M_1\cap M_2=\{0\}$的时候,我们有同构$\nu: M_1\oplus M_2\to M_1+M_2$. 如果$M_1+M_2=M$,那么就是有同构$M\cong M_1\oplus M_2$.

\theo 在左$R$-模范畴,任意的极限与余极限存在。 \notprove

\para 如果$M$和$N$都是双边$R$-模(比如我们一直讨论的交换环的情况),则


\section{自由对象}

简单来说,自由对象是遗忘函子的伴随函子。遗忘函子以前我们已经谈过,他将结构比较多的对象变成了结构比较少的对象,而自由对象就是反过来,从一个结构比较少的对象来构造一个结构比较多的对象。

\para 

\section{多项式环}

多项式环的重要性无需赘述。

\para 设$R$是一个环,而$a=(a_0,a_1,\cdots,a_n,\cdots)$是一列$R$中的元素,记只有有限个元素非零的$a$构成的集合为$R[x]$. 在$R[x]$上,我们可以定义加法
\[
	a+b=(a_0+b_0,a_1+b_1,\cdots),
\]
由于$a+b$也是只有有限个元素非零,所以$a+b \in R[x]$. 对于加法$0=(0,0,\cdots)$显然是零元,而$a$的逆元$-a=(-a_0,-a_1,\cdots)$.

然后是乘法,定义
\[
	(ab)_i=\sum_{k+j=i}a_kb_j,
\]
可以看出$ab$也只有有限个元素非零,所以$ab\in R[x]$.

分配律的检验是直接的,
\[
	((a+b)c)_i=\sum_{k+j=i}(a+b)_kc_j=\sum_{k+j=i}a_kc_j+\sum_{k+j=i}b_kc_j=(ac)_i+(bc)_i,
\]
所有的一切都可以直接检验。因此$R[x]$确实是一个环,称为多项式环。

一般而言,我们将$R[x]$中的元素$f=(a_0,a_1,\cdots)$写成
\[
	f=\sum_{n}a_n x^n
\]
的形式,这就是我们熟悉的多项式,一切的运算都和我们熟悉的那样。$x$称之为不定元,不定元只是一个符号,在不发生歧义的情况下可以任意选取。

\para 定义多元多项式环$R[x_1,\cdots,x_n]$为$R[x_1,\cdots,x_{n-1}][x_n]$,这是一个递归定义。

\para 一个多项式$f=(a_0,a_1,\cdots)$中的最高项即使得$a_n$不为零的最大的$n$,这个$n$记作$\deg(f)$,称为多项式$f$的幂次。最高幂次的系数我们称之为最高次系数,或者首项系数。其余的项目通常我们会指出具体的指标,比如说$k$-次项,就是指$a_kx^k$,而$k$-此项系数就是指$a_k$. 如果最高次系数为$1$,这个多项式称为首一多项式。

对于首一多项式而言$f$,$\deg(fg)=\deg(f)+\deg(g)$. 但是一般的多项式并不如此,比如在$\zz/4\zz$上的多项式$2x$,有$(2x)(2x)=0$.

对于$\deg(f+g)$,我们有估计$\deg(f+g)\leq \max\{\deg(f),\deg(g)\}$,不等号是可以取严格的,同样比如在$\zz/4\zz$上的多项式$2x$,有$2x+2x=0$.

\theo 多项式除法算法:设$f$, $g\in R[x]$,而且$g$的首项系数可逆,则存在唯一的多项式$p$, $r\in R[x]$使得$f=pg+r$,且$\deg(r)< \deg(f)$.

\proof
	可以假设$g$是首一多项式,因为如果设$g$的首项系数为$a$,则$g/a$是一个首一多项式,如果命题对首一多项式成立,即存在$q$, $r\in R[x]$使得$f=q(g/a)+r$,则$f=(q/a)g+r=pg+r$,其中$p=q/a$和$r$都是多项式。这样就得到了我们的命题。

	首先假设存在,证明唯一性。设$f=p'g+r'$以及$f=pg+r$成立,则$(p'-p)g=r'-r$,由于$g$是首一的,且$p'-p$非零,所以$\deg((p'-p)g)\geq \deg(g)$,但是$\deg(r'-r)< \deg(g)$,这就造成了矛盾。下面证明存在性。

	设$\deg(f)=n$, $\deg(g)=m$,如果$n<m$,则取$p=0$, $r=f$. 考虑$n\leq m$的情况,设$f$的首项系数为$a_0$,考虑多项式$f_1=f-a_0x^{n-m}g$,由于$f$和$a_0x^{n-m}g$的最高次项相同,所以$\deg(f_1)<\deg(f)$,如果$\deg(f_1)<\deg(g)$,那么$f=a_0x^{n-m}g+f_1$就给出了分解。

	否则继续对$f_1$进行这样的操作,得到$f_2=f_1-a_1x^{\deg(f_1)-m}g$,再比较$\deg(f_2)$与$\deg(g)$. 不断如是进行下去,由于$f$的幂次有限,而每次操作,幂次都至少减一,所以该过程在进行至多$n-m+1$次后就会停止。设该过程在第$k$次后停止,则我们就得到了
	\[
	f_{k}=f-a_0x^{n-m}g-a_1x^{\deg(f_1)-m}g-\cdots-a_{k-1}x^{\deg(f_{k-1})-m}g
	\]
	使得$\deg(f_k)< \deg(g)$. 即$f=f_0$,则我们就得到了分解
	\[
	f=f_k+a_0x^{n-m}g+a_1x^{\deg(f_1)-m}g+\cdots+a_{k-1}x^{\deg(f_{k-1})-m}g=\left(\sum_{i=0}^{k-1}a_{i}x^{\deg(f_{i})-m}\right)g+f_k.
	\]
	因此$p=\sum_{i=0}^{k-1}a_{i}x^{\deg(f_{i})-m}$以及$r=f_k$就是我们需要的多项式。
\qed

存在性的证明就是整个算法,从算法来看,这个命题即使是对非交换环上的多项式环也是成立的。

\section{矩阵与行列式}

\para 设$M$是一个有限生成$R$-模,他的生成元可以取作$\{m_1$, $\cdots$, $m_n\}$,则一个$R$中元素的$n$-元组$(r_1$, $\cdots$, $r_n)$可以确定一个线性组合$\sum_{i=1}^nr_im_i$,继而是一个模中的元素。

\para 设$M$和$N$有限生成$R$-模,前者的生成元是$\{m_1$, $\cdots$, $m_p\}$,后者的是$\{n_1$, $\cdots$, $n_q\}$,然后考虑一个模同态$\varphi:M\to N$,由于
\[
	\varphi\left(\sum_{i} r_im_i\right)=\sum_{i} r_i\varphi\left(m_i\right),
\]
所以它由所有$\varphi\left(m_i\right)$描述,由于他在$R^n$中,将其写作
\[
	\varphi\left(m_i\right)=\sum_{j}\varphi_{ji}n_j.
\]
于是$\{\varphi_{ij}\,:\, 1\leq i \leq p,\,\,1\leq j \leq q\}$这$p\times q$个数构成的整体称为$\varphi$的矩阵,记作$(\varphi)$,他完全描述了整个同态$\varphi$.

设$\psi:R^n\to R^l$的矩阵是$(\psi)_{ij}$,我们下面计算$\psi\circ\varphi$的矩阵:
\[
	\psi\circ\varphi(1_i)=\psi\left(\sum_{j=1}^n\varphi_{ji}1_j\right)=\sum_{j=1}^n\varphi_{ji}\psi(1_j)=\sum_{j=1}^n\sum_{k=1}^l\psi_{kj}\varphi_{ji}1_k,
\]
所以$\psi\circ\varphi$的矩阵$(\psi\circ\varphi)$为
\[
	(\psi\circ\varphi)_{ki}=\sum_{j=1}^n\psi_{kj}\varphi_{ji},
\]
这被称为矩阵乘法法则。

\para 通常会使用一张表来表示矩阵
\[
(\varphi)=
\begin{pmatrix}
	\varphi_{11} & \varphi_{12} & \cdots & \varphi_{1m}\\
	\varphi_{21} & \varphi_{22} & \cdots & \varphi_{2m}\\
	\vdots & \vdots & \ddots & \vdots \\
	\varphi_{n1} & \varphi_{n2} & \cdots & \varphi_{nm}\\
\end{pmatrix}
\]
可以看到$\varphi(1_i)$关于$\{1_j\in R^n\}$中展开的系数出现在$(\varphi)$的第$i$列。

\para 矩阵乘法法则此时则写作:
\begin{equation}
\begin{pmatrix}
	\psi_{11} & \psi_{12} & \cdots & \psi_{1n}\\
	\psi_{21} & \psi_{22} & \cdots & \psi_{2n}\\
	\vdots & \vdots & \ddots & \vdots \\
	\psi_{l1} & \psi_{l2} & \cdots & \psi_{ln}
\end{pmatrix}
\begin{pmatrix}
	\varphi_{11} & \varphi_{12} & \cdots & \varphi_{1m}\\
	\varphi_{21} & \varphi_{22} & \cdots & \varphi_{2m}\\
	\vdots & \vdots & \ddots & \vdots \\
	\varphi_{n1} & \varphi_{n2} & \cdots & \varphi_{nm}\\
\end{pmatrix}
=
\begin{pmatrix}
	\sum_{i=1}^n \psi_{1i}\varphi_{i1} & \sum_{i=1}^n \psi_{1i}\varphi_{i2} & \cdots & \sum_{i=1}^n \psi_{1i}\varphi_{im}\\
	\sum_{i=1}^n \psi_{2i}\varphi_{i1} & \sum_{i=1}^n \psi_{2i}\varphi_{i2}& \cdots & \sum_{i=1}^n \psi_{2i}\varphi_{im}\\
	\vdots & \vdots & \ddots & \vdots \\
	\sum_{i=1}^n \psi_{li}\varphi_{i1} & \sum_{i=1}^n \psi_{li}\varphi_{i2}& \cdots & \sum_{i=1}^n \psi_{li}\varphi_{im}
\end{pmatrix}
\end{equation}
很显然,$(\psi\circ\varphi)_{ij}$由第一个矩阵的第$i$行和第二个矩阵的第$j$列逐个相乘然后相加而得。矩阵乘法把$l\times n$和$n \times m$的矩阵变成了$l \times m$的矩阵。

尤其重要的情况是$m=1$的时候,此时,$n\times 1$的矩阵称为一个列,为了省下书写空间,通常写成行的形式,即$(\varphi_{1i},\cdots,\varphi_{ni})^T$的形式,上标$T$代表将一行转成一列。设$\varphi_{i}=(\varphi_{1i},\cdots,\varphi_{ni})^T$,则一个矩阵可成看出一些列,
\[
	\begin{pmatrix}
	\varphi_{11} & \varphi_{12} & \cdots & \varphi_{1m}\\
	\varphi_{21} & \varphi_{22} & \cdots & \varphi_{2m}\\
	\vdots & \vdots & \ddots & \vdots \\
	\varphi_{n1} & \varphi_{n2} & \cdots & \varphi_{nm}\\
	\end{pmatrix}
	=
	\begin{pmatrix}
	\varphi_{1} & \varphi_{2} & \cdots & \varphi_{m}
	\end{pmatrix},
\]
其中$\varphi_i=(\varphi_{1i},\cdots,\varphi_{ni})^T$. 

关于列的矩阵乘法为
\begin{equation}
\begin{pmatrix}
	\psi_{11} & \psi_{12} & \cdots & \psi_{1n}\\
	\psi_{21} & \psi_{22} & \cdots & \psi_{2n}\\
	\vdots & \vdots & \ddots & \vdots \\
	\psi_{l1} & \psi_{l2} & \cdots & \psi_{ln}
\end{pmatrix}
\begin{pmatrix}
	\varphi_{11} \\
	\varphi_{21}  \\
	\vdots \\
	\varphi_{n1} \\
\end{pmatrix}
=
\begin{pmatrix}
	\sum_{i=1}^n \psi_{1i}\varphi_{i1}\\
	\sum_{i=1}^n \psi_{2i}\varphi_{i1} \\
	\vdots \\
	\sum_{i=1}^n \psi_{li}\varphi_{i1}
\end{pmatrix}
\end{equation}
通常将其右侧写作$(\psi)\varphi_1$. 那么一般的矩阵乘法则写作:
\[
	(\psi)
	\begin{pmatrix}
	\varphi_{1} & \varphi_{2} & \cdots & \varphi_{m}
	\end{pmatrix}
	=
	\begin{pmatrix}
	(\psi)\varphi_{1} & (\psi)\varphi_{2} & \cdots & (\psi)\varphi_{m}
	\end{pmatrix}.
\]
