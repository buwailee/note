\chapter{基本构造}
\ThisULCornerWallPaper{1}{../Pictures/4.png}

\section{极限与余极限}

\para 设$I$是一个小范畴(即范畴对象的全体能够构成一个集合),我们称函子$D:I\to \mathcal{C}$为$\mathcal{C}$中的一个$I$-图,或者简略叫做图。对于$i\in I$,$D(i)$称为该图的一个顶点(经常改用下标写作$D_i$),而对任意的态射$\alpha_{ij}:i\to j$,态射$D(\alpha_{ij})$称为该图的一条边。

作为例子,当$I$是一个指标集,即不同的$i$, $j\in I$之间不存在态射时,一个$I$-图就是一族以$I$为指标集的对象。

\para 称$(A,\lambda)$为$I$-图的一个{锥形},如果$A$是$\mathcal{C}$的一个对象,而$\lambda$为一族态射$\lambda_{j}:A\to D(i)$使得如下交换图对所有的顶点和边都成立
\[
	\xymatrix{
		&D(i)\ar[dd]^{D(\alpha_{ij})}\\
		A\ar[ru]^{\lambda_i}\ar[rd]_{\lambda_j}&\\
		&D(j)
	}
\]
称呼一个锥形$(A,\lambda)$是$I$-图的一个\idx{极限}(\idx{limit}),如果对于任意的锥形$(B,\mu)$,都存在唯一的态射$f:B\to A$使得如下分解$\mu_j:B\xrightarrow{f}A\xrightarrow{\lambda_j}D(i)$成立。这里极限满足的性质被称为泛性质,是范畴论中经常出现的一种性质。

如果极限$(A,\lambda)$中的态射族是明确的(乃至于是由对象$A$确定的),那么我们会用$A$来表示极限。并且一般将$I$-图$D$的极限写作${\varprojlim}_{i\in I} D(i)$. 用交换图来表述上面的分解,那么就是说
\[
	\xymatrix{
		&&D(i)\ar[dd]^{D(\alpha_{ij})}\\
		B\ar@{.>}[r]|-{f}\ar@/^/[rru]^-{\mu_i}\ar@/_/[rrd]_{\mu_j}&{\varprojlim}_{i\in I} D(i)\ar[ru]\ar[rd]&\\
		&&D(j)
	}
\]
途中略去的箭头是极限自带的态射族,虚线代表唯一存在。

作为例子,一族集合的直积是一个极限。设$\{X_i\,:\, i\in I\}$是一族集合,它的直积记作$\prod_{i\in I}X_i$,且到每个分量都有一个投影函数$\pi_i:\prod_{i\in I}X_i\to X_i$. 这两个数据构成了这一族集合的极限。

\pro 如果一个$I$-图有两个极限$(A,\lambda)$和$(B,\mu)$,则$A$与$B$同构。因此,满足泛性质的对象,这里是极限,在同构意义上唯一。

\proof 假设$(A,\lambda)$和$(B,\mu)$都满足泛性质。利用$A$的泛性质,有同态$\rho:B\to A$,利用$B$的泛性质,有同态$\rho':A\to B$. 接着考虑$\rho \rho':A\to A$和$\id_A:A\to A$,他们同时都满足两个锥形$(A,\lambda)$到$(A,\lambda)$的分解,由分解的唯一性,$\rho\rho'=\id_A$. 同理,$\rho'\rho=\id_{B}$. 所以$A$与$B$同构。\qed

\para 锥形和极限都有对偶的概念,在交换图中,无外乎就是将箭头完全反过来。称$(A,\lambda)$为$I$-图的一个{余锥形},如果$A$是$\mathcal{C}$的一个对象,而$\lambda$为一族态射$\lambda_{j}:D(i)\to A$使得如下交换图对所有的顶点和边都成立
\[
	\xymatrix{
		&D(i)\ar[ld]_{\lambda_i}\ar[dd]^{D(\alpha_{ij})}\\
		A&\\
		&D(j)\ar[lu]^{\lambda_j}
	}
\]

称呼一个余锥形$(A,\lambda)$是$I$-图的一个{余极限},如果对于任意的余锥形$(B,\mu)$,都存在唯一的态射$f:A\to B$使得如下分解$\mu_j:D(i)\xrightarrow{\lambda_j}A\xrightarrow{f}B$成立。这是另一种泛性质。一般将$I$-图$D$的余极限写作${\varinjlim}_{i\in I} D(i)$. 类似极限,如果余极限存在,则它在同构意义上唯一。

\pro \label{homlimit}设$\mathcal{C}$是一个范畴,其中的对象都是集合,态射都是映射。而$I$是一个小范畴,再设$X:i\mapsto X_i$是一个$I$-图,如果极限与余极限存在,则有自然同构
\[
	{\varprojlim} \mathcal{C}(-,X_i)\cong \mathcal{C}\left(-,{\varprojlim} X_i\right),\quad {\varprojlim} \mathcal{C}(X_i,-)\cong  \mathcal{C}\left({\varinjlim} X_i,-\right)
\]
其中极限与余极限都是对$I$-图进行的。

\proof
	仅证第一点,第二点是类似的。给定$Y$,由于对每一个$i\in I$都有态射$\pi_i: {\varprojlim} X_i\to X_i$,所以$\pi_{i*}$给出了映射$\mathcal{C}\left(Y,{\varprojlim} X_i\right)\to \mathcal{C}(Y,X_i)$,于是$\left(\mathcal{C}\left(Y,{\varprojlim} X_i\right),\{\pi_{i*}\,:\, i\in I\}\right)$是一个锥形。下面验证这就是$I$-图$i\mapsto \mathcal{C}(Y,X_i)$的极限,然后由极限在同构意义上的唯一性就得到了同构。

	任取$I$-图$i\mapsto \mathcal{C}(Y,X_i)$的一个锥形$(A,\lambda)$以及$a\in A$,则$\lambda_i(a):Y\to X_i$成立如下交换图
	\[
		\xymatrix{
			&Y \ar[dl]_{\lambda_{i}(a)}\ar[dr]^{\lambda_{j}(a)}&\\
			X_i\ar[rr]^{X(\alpha_{ij})}&&X_j
		}
	\]
	这是一个锥形,由于存在极限${\varprojlim} X_i$,所以存在态射$\mu_\lambda(a):Y\to {\varprojlim} X_i$使得分解
	\[
		\lambda_i(a):Y\xrightarrow{\mu_\lambda(a)}{\varprojlim} X_i\xrightarrow{\pi_i}X_i
	\]
	成立,遍历$a\in A$将给出态射$\mu_\lambda:A\to \mathcal{C}\left(Y,{\varprojlim} X_i\right)$使得分解
	\[
	\lambda_i: A\xrightarrow{\mu_\lambda}\mathcal{C}\left(Y,{\varprojlim} X_i\right)\xrightarrow{\pi_{i*}}\mathcal{C}(Y,X_i)
	\]
	成立,所以$\mathcal{C}\left(Y,{\varprojlim} X_i\right)$就是$I$-图$i\mapsto \mathcal{C}(Y,X_i)$的极限。由极限的泛性质,于是有同构
	\[
		{\varprojlim} \mathcal{C}(Y,X_i)\cong \mathcal{C}\left(Y,{\varprojlim} X_i\right),
	\]
	函子性的检查是简单的。
\qed

\para 回忆一下偏序集,设$I$是一个偏序集,偏序关系为$\leq$,那么通过定义态射
\[
	{I}\left(x,y\right)=\begin{cases}
	(x,y)&\text{, if }x\leq y\text{;}\\
	\varnothing&\text{, otherwise}.
	\end{cases}
\]
则$I$是一个小范畴,如果我们定义$x\geq y$当且仅当$y\leq x$,那么新的偏序集$(I,\geq)$就是偏序集$(I,\leq)$的对偶范畴,略去偏序符号,有时候会记作$I^{\mathrm{op}}$.

$D:I\to \mathcal{C}$是一个协变函子,就是对$i\leq j\leq k$成立$D(j\leq k)\circ D(i\leq j)=D(i\leq k)$,就是说从小到大有映射。反之,反变函子$D:I\to \mathcal{C}$对$i\leq j\leq k$成立$D(i\leq j)\circ D(j\leq k)=D(i\leq k)$,就是说从大到小有映射。

对于一个反变函子$D:I\to \mathcal{C}$,我们可以定义一个协变函子$D^{\mathrm{op}} :I^{\mathrm{op}}\to \mathcal{C}$通过$D^{\mathrm{op}}(j\geq i)=D(i\leq j)$,此时对$k\geq j\geq i$成立$D^{\mathrm{op}}(j\geq i)\circ D^{\mathrm{op}}(k\geq j)=D^{\mathrm{op}}(k\geq i)$,可见这确实是一个协变函子。

\para 设$I$是一个指标集,即只存在关系$x\leq x$. 此时构成的$I$-图的极限称为\idx{积}(\idx{product}),余极限称为\idx{余积}(\idx{coproduct})。前面提到过的集合的直积就是一种积。

\para 称$I$是一个滤相的偏序集,即对于任意的$i$, $j\in I$都存在$k\in I$使得$k\leq i$和$k\leq j$同时成立。称$I$是一个定向的偏序集,即对于任意的$i$, $j\in I$都存在$k\in I$使得$i\leq k$和$j\leq k$同时成立。很容易看到,如果偏序集$I$是滤相的(定向的),那么$I^{\mathrm{op}}$就是定向的(滤相的),反之亦然。所以下面的讨论,将固定$I$是一个滤相偏序集,而$I^{\mathrm{op}}$则是一个定向偏序集。

作为例子,考虑拓扑空间中所有非空开集按照包含构成的偏序集,即$U\leq V$当且仅当$U\subset V$,那么这个偏序集是定向的,但不是滤相的,因为对于任意的$U$和$V$总有$U\leq U\cup V$和$V\leq U\cup V$成立,但对于不交的$U$和$V$,并不存在非空开集同时包含于他们其中。同样是拓扑的例子,考虑所有包含$p$的非空开集按照包含构成的偏序集,那么这个偏序集既是滤相的又是定向的。

\para 将滤相的(定向的)偏序集$I$看作一个范畴,对任意的范畴$\mathcal{C}$,如果一个$I$-图$D$,$D:I\to \mathcal{C}$是协变函子,则这称为一个$C$上的一个{逆系统}({定向系统})。对偶地,如果$D$是反变函子,那么$I^{\mathrm{op}}$-图$D^{\mathrm{op}}$是定向系统(逆系统)。

逆系统一般谈论极限${\varprojlim}_{i\in I} D(i)$,而定向系统一般谈论${\varinjlim}_{i\in I} D(i)$,为了思考这个原因,我们考虑如下交换图
\[
	\xymatrix{
		&A \ar[dl]_{\lambda_{i}}\ar[dr]^{\lambda_{j}}\ar[dd]^{\lambda_{k}}&\\
		D(i)&&D(i)\\
		&D(k)\ar[ul]^{D(k\leq i)}\ar[ur]_{D(k\leq j)}&
	}
	\quad
% \xymatrix{
% 	&A \ar[dl]_{\lambda_{i}}\ar[dr]^{\lambda_{j}}\ar[dd]^{\lambda_{k}}&\\
% 	D(i)\ar[dr]_{D(i\leq k)}&&D(i)\ar[dl]^{D(j\leq k)}\\
% 	&D(k)&
% }
	\xymatrix{
		&A &\\
		\ar[ur]^{\mu_{i}}D(i)\ar[dr]_{D(i\leq k)}&&D(i)\ar[ul]_{\mu_{j}}\ar[dl]^{D(j\leq k)}\\
		&D(k)\ar[uu]_{\mu_{k}}&
	}
\]
左边是对逆系统考虑锥形,右边是对定向系统考虑余锥形。从左边来看,如果$i\leq j\leq k$成立,则$D(k)$构成了一个锥形的顶点,当我们考虑极限的时候,这时候极限就应该表现得像那些“极小”的元素$D(k)$一样。而且如果$A$是极限,那么箭头$A\to D(k)$也构成了唯一分解。类似地考虑右边的图,那些“极大”的元素$D(k)$构成了余锥形的顶点,如果$A$是余极限,那么箭头$D(k)\to A$也构成了唯一分解。

\para 设$I$是一个偏序集,如果$i\in I$有,$i>j$对所有$j\in I$不成立,或者说,与可以比较的元素$j\in I$都有$i\leq j$,则称$i$是$I$的一个极小元。如果$i\leq j$对所有$j\in I$都成立,则称$i\in I$是$I$的最小元。

在滤相的偏序集中,极小元和最小元等价。最小元是极小的,这是显然的。反之,因为如果存在极小元$i$,那么任取一个$j\in I$都存在一个$k\in I$使得$k\leq i$和$k\leq j$都成立,然而$i$是极小元,所以$i=k$,所以$i\leq j$. 同理我们可以定义极大元与最大元,在定向的偏序集中,此二者等价。

更广义地,对于一个偏序集$I$的子集$J$,如果对于任意的$i\in I$,都存在$j\in J$使得$j\leq i$,则称$J$和$I$是共尾的。显然,滤相的偏序集中的极小元构成的单点集就和原来的偏序集共尾。

\pro 设$I$-图$D$是一个逆系统,如果$I$存在一个与其共尾的子集$J$,则$J$-图$D$是一个逆系统,且${\varprojlim}_{i\in I} D(i)={\varprojlim}_{i\in J} D(i)$. 对偶地,设$I$-图$D$是一个定向系统,如果$I^{\mathrm{op}} $存在一个与其共尾的子集$J^{\mathrm{op}} $,则$J$-图$D$是一个定向系统,且${\varinjlim}_{i\in I} D(i)={\varinjlim}_{i\in J} D(i)$. \rule{2mm}{2mm}

所以对于逆系统,如果存在极小元,那么极小元就是它的极限,对偶地,对于定向系统,如果存在极大元,那么极大元就是它的余极限。这符合我们上面的直观。

\para 一个拓扑空间$X$能被看成一个范畴,对象取作他的所有开集,而态射取作
\[
	{X}(U,V)=\begin{cases}
	\bigl\{i^U_V:U\hookrightarrow V\bigr\}&\text{, if }U\subset V\text{;}\\
	\varnothing&\text{, otherwise}.
	\end{cases}
\]

考虑拓扑空间$X$中一个非空开集族$\mathfrak{B}$,对于任意的两个$U$, $V\in \mathfrak{B}$,都存在一个$W\in \mathfrak{B}$使得$U\cup V\subset W$成立,这样的$\mathfrak{B}$按照包含构成了定向的偏序集。现在定义函子$i:\mathfrak{B}\to X$通过$i(U)=U$以及$i(U\leq V)=i^U_V:U\hookrightarrow V$,它显然成立复合$i(U\leq W)=i(V\leq W)\circ i(U\leq V)$,所以这个$\mathfrak{B}$-图$i$是一个定向系统。可以看到它的余极限${\varinjlim}_{U\in \mathfrak{B}} U$实际上就是$\bigcup_{U\in \mathfrak{B}} U$,而态射族就是$i_U:U\hookrightarrow \bigcup_{U\in \mathfrak{B}} U$.

同样考虑拓扑空间$X$中一个包含一个点$p\in X$的所有非空开集构成的族$\mathfrak{B}$,它按照包含构成一个滤相的偏序集。同样可以定义函子$i:\mathfrak{B}\to X$通过$i(U)=U$以及$i(U\leq V)=i^U_V:U\hookrightarrow V$,同样显然成立复合$i(U\leq W)=i(V\leq W)\circ i(U\leq V)$,所以这个$\mathfrak{B}$-图$i$是一个逆系统。但它的极限${\varprojlim}_{U\in \mathfrak{B}} U$就不一定存在,因为类比于上一个例子,它的极限应该类似于所有$\mathfrak{B}$中元素的交,但这不一定是一个开集。

\para 设我们有一个$X$上的$\mathcal{K}$-预层$\calf$,他是$X\to \mathcal{K}$的一个反变函子。赋予$X$一个偏序,此时函数$U\leq V$即$U\hookrightarrow V$. 那么$X$的那些包含$p\in X$的非空开集构成的子集族$\mathfrak{B}$继承了偏序结构,也是一个范畴,而预层$\calf$限制在$\mathfrak{B}$给出了反变函子$\mathfrak{B}\to \mathcal{K}$.

由于$\mathfrak{B}$是滤相的偏序集,而$\calf$是反变函子,因此$\mathfrak{B}^{\mathrm{op}}$-图$\calf^{\mathrm{op}}$是定向系统,进而我们会考虑余极限
\[
	{\varinjlim}_{U\in \mathfrak{B}^{\mathrm{op}}} \calf^{\mathrm{op}}(U)={\varinjlim}_{U\in \mathfrak{B}^{\mathrm{op}}} \calf(U),
\]
如果这个余极限存在,那么就称为预层$\calf$在点$p$处的纤维,记作$\calf_p$.

\para 极限很好地抽象了许多有用的概念,比如等值子。等值子的一个特例就是模范畴的核。这就意味着,我们可以把核看成一种极限,这在某些地方是实用的,比如在模范畴里去证明右伴随函子是左正和的时候。

考虑$\mathcal{J}$是这样一个范畴,它有两个对象,除了自己到自己的恒等态射外,两个对象之间还有同向的两个态射,我们记作$\xymatrix{\cdot\ar@<0.3ex>[r] \ar@<-0.3ex>[r]&\cdot}$. 设$D:\mathcal{J}\to\mathcal{C}$是一个协变函子,他将两个态射变为$f$, $g:X\to Y$,于是在$\mathcal{C}$中关于$J$-图$D$的极限就是对象$\mathrm{eq}(f,g)$以及态射$i:\mathrm{eq}(f,g)\to X$,满足:$fi=gi$,任取对象$A\in \mathcal{C}$以及态射$p:A\to X$使得$fp=gp$,则存在唯一的态射$h:A\to \mathrm{eq}(f,g)$使得下面的交换图成立:
\[
\begin{xy}
	\xymatrix{
		\mathrm{eq}(f,g)\ar[r]^-{i}&X\ar@<0.3ex>[r]^-{f} \ar@<-0.3ex>[r]_-{g}&Y\\
		A\ar@{.>}[u]^-{h}\ar[ur]_-{p}
	}
\end{xy}
\]

集合范畴的等值子很简单,设$f$, $g:X\to Y$是两个映射,则$\mathrm{eq}(f,g)=\{x\in X\,:\, f(x)=g(x)\}$. 不难检验,对于模范畴,$0$可以看成任意两个模之间的同态(比如看作左乘$0$),此时$\ker(f)=\mathrm{eq}(f,0)$. 

对偶地,可以定义余等值子,它使得如下交换图成立
\[
\begin{xy}
	\xymatrix{
		X\ar@<0.3ex>[r]^-{f} \ar@<-0.3ex>[r]_-{g}&Y\ar[r]^-{\pi}\ar[dr]_-{q}&\mathrm{coeq}(f,g)\ar@{.>}[d]^-{h}\\
		&&A
	}
\end{xy}
\]
类似地,在模范畴里面,$\coker(f)=\mathrm{coeq}(f,0)$.

如果有一个函子$R:\mathcal{C}\to \mathcal{D}$,对任意极限满足$R{\varprojlim}_i Y_i\cong {\varprojlim}_i (RY_i)$,那么应用到等值子上面,就有
\[
	R\left(\mathrm{eq}(f,g)\right)\cong \mathrm{eq}(Rf,Rg).
\]
设$R$是一个模范畴到模范畴的一个函子,如果$R$对任意的零态射$0:X\to Y$都有$R0=0$,则这个等式给出了核的一个关系$R\left(\ker f\right)\cong \ker(Rf)$. 如果$R$对零模$0$也有$R(0)=0$,则如果$f$是单射,核的那个关系将给出$\ker(Rf)\cong R(0)=0$. 于是$Rf$也是一个单射。

\para 另一个极限的例子如下。考虑范畴$\mathcal{J}=\cdot \to \cdot \leftarrow \cdot$,即$\mathcal{J}$只有三个对象,除去恒等态射外,剩下的还有两边指向中间的态射。那么$\mathcal{J}$-图$X\xrightarrow{f} Z \xleftarrow{g} Y$的极限$X\times_Z Y$用泛性质表述就是如下交换图
\[
\begin{xy}
	\xymatrix{
		A\ar@/_/[rdd]\ar@/^/[drr]\ar@{.>}[dr]&&\\
		&X\times_Z Y\ar[d]_{\pi_2}\ar[r]_{\pi_1}&X\ar[d]^f\\
		&Y\ar[r]_g&Z
	}
\end{xy}
\]
这样的极限$X\times_Z Y$,被称为$X$, $Y$关于$Z$的\idx{纤维积},或者叫\idx{拉回}。如果$X=Y$,则我们回到了等值子的情况。假设$X$, $Y$和$Z$都是集合,不难检验,$X\times_Z Y=\{(x,y)\in X \times Y\,:\,f(x)=g(y)\in Z\}$. 于是若$Z$是单点集,则$X\times_Z Y=X\times Y$,若$X$和$Y$的子集,它们到$Z$都是标准的含入映射,则$X\times_Z Y=X\cap Y$.

\section{模范畴中的极限与余极限}

\para 首先介绍模范畴里的积,他被称为直积。下面重新描述一下泛性质:设$\{M_i\,:\, i\in I\}$是一族左$R$-模,则存在一个左$R$-模$M$以及一族同态$\pi_i:M\to M_i$使得,如果存在另一个模$N$和一族同态$\rho_i:N\to M_i$,那么就唯一存在同态$\rho:N\to M$使得分解$\rho_i:N\xrightarrow{\rho} M \xrightarrow{\pi_i} M_i$成立对任意的$i\in I$都成立。这样的一个模$M$被称为$\{M_i\,:\, i\in I\}$的直积,而$\pi_i$被称为典范投影,通常将$M$记作$\prod_{i\in I}M_i$. 

存在性的证明是简单的,考虑$\{M_i\,:\, i\in I\}$作为集合的直积,很容易检验它有一个$R$-模结构,且投影是同态。更详细的检查这里就略去了。

\para 对偶地,这里描述一下模范畴里的余积,他就是直和。设$\{M_i\,:\, i\in I\}$是一族$R$-模,则存在一个$R$-模$M$以及一族同态$\pi_i:M_i\to M$使得,如果存在另一个模$N$和一族同态$\rho_i:M_i\to N$,那么就唯一存在同态$\rho:M\to N$使得分解$\rho_i:M_i\xrightarrow{\pi_i} M \xrightarrow{\rho} N$成立对任意的$i\in I$都成立。这样的一个模$M$被称为$\{M_i\,:\, i\in I\}$的直和,而$\pi_i$被称为典范内射,通常将$M$记作$\bigoplus_{i\in I}M_i$. 

直和存在性的构造并不如同直积那么容易。我们考虑$\{M_i\,:\, i\in I\}$的形式和$M$,即
\[
	M=\left\{\sum_{i\in I} a_i m_i\,:\, m_i\in M_i\right\},
\]
其中的求和只有有限项,或者说系数$a_i$只有有限项非零(这被称为几乎处处为零,注意到,如果没有拓扑,求和总应该是有限的)。不难看到这是一个模。定义$\pi_i:M_i\to M$为$\pi_i(m_i)=m_i$,可以检验这是一个模同态。

现在任取$N$和一族同态$\rho_i:M_i\to N$,我们将其扩展为$\rho_i:M\to N$通过补充定义$\rho_i(m)=0$,如果$m\notin M_i$. 随后我们定义$\rho = \sum_{i\in I} \rho_i$,需要检验这个求和对任意的$m\in M$是有限的。为此,任取$m\in M$,由于他可以分解成$m=\sum_i a_i m_i$,所以
\[
	\rho(m)=\sum_i a_i \rho_i(m_i),
\]
求和是有限求和因为$a_i$只有有限项非零。有了构造,泛性质的检验就是直接的了。

由构造,有限直和与有限直积在模范畴里面是等价的。考虑两个$R$-模$M_1$和$M_2$,那么$M_1$可以看成$M_1\oplus M_2$的子模,通过$m\to (m,0)$. 反过来,设$M_1$和$M_2$是一个模$M$的两个子模,则我们有如下短正合列
\[
	0\to M_1\cap M_2 \xrightarrow{\mu} M_1\oplus M_2\xrightarrow{\nu} M_1+M_2\to 0,
\]
其中$\mu:m\mapsto (m,m)$,$\nu:(m_1,m_2)\mapsto m_1-m_2$. 所以在$M_1\cap M_2=\{0\}$的时候,我们有同构$\nu: M_1\oplus M_2\to M_1+M_2$. 如果$M_1+M_2=M$,那么就是有同构$M\cong M_1\oplus M_2$.

\para 注意到交换群为$\zz$-模,所以我们也证明了交换群有直和与直积。

\para 前面已经看到,左$R$-模范畴内,$\ker(f)=\mathrm{eq}(f,0)$以及$\coker(f)=\mathrm{coeq}(f,0)$,存在性都是清楚的。更一般地,设$f$, $g:M\to N$是两个模同态,则
\[
	\mathrm{eq}(f,g)=\ker(f-g)
\]
就是等值子,而
\[
	\mathrm{coeq}(f,g)=\coker (f-g)
\]
就是余等值子。

\theo 在左$R$-模范畴,任意的极限与余极限存在。 \notprove

\section{自由对象}

一句话来概括自由对象,它是遗忘函子的左伴随函子的像,遗忘函子的左伴随函子称为自由函子,所以自由对象是自由函子作用在一个对象上得到的。何谓伴随函子?给定一个函子$f:\mathcal{C}\to \mathcal{D}$,它的伴随函子从某种程度上来说就是它的逆$g:\mathcal{D}\to \mathcal{C}$. 但是远不如逆那么强,准确来说,就是说存在函子同态(自然变换)$\sigma:fg\to \id_{\mathcal{D}}$以及$\tau:\id_{\mathcal{C}}\to gf$使得函子复合
\[
	f\xrightarrow{f\tau}fgf\xrightarrow{\sigma f}f,\quad g\xrightarrow{\tau g}gfg\xrightarrow{g\sigma}g
\]
都得到恒等自然变换。此时$f$被称为$g$的左伴随函子,$g$被称为$f$的右伴随函子。

以遗忘函子举例,如果一个函子$g:\mathcal{C}\to \mathsf{Set}$使得$\mathcal{C}$中的对象$X$忘掉了某代数结构,变成了集合$|X|$。那么怎么样的函子才是它的伴随函子?必然是一个赋予$\mathsf{Set}$中对象代数结构的函子$f$,但是从一个结构少的对象到结构多的对象,我们不一定能还原出$X$,比如我们忘掉了$2+4=6$,那么在从$|X|$赋予代数结构得到的$f(|X|)$中,只能把$2$, $4$, $6$单独处理,引入一个新的元素`$2+4$'来表示$f(|X|)$中$2+4$的结果,而不与$6$建立联系。所以$f(|X|)=f(g(X))$作为集合,即$g(f(g(X)))$,一般来说应该比$g(X)$大。那么如何使得$f(|X|)$回忆其原本的代数结构?最方便的,就是找一个态射$f(|X|)\to X$,这样`$2+4$'就映射到了$6$. 改变$X$,就得到了自然变换$\sigma:fg\to \id_\mathcal{C}$.

类似地,给定一个集合$S\in \mathsf{Set}$,生成一个$f(S)\in \mathcal{C}$. 为了满足代数结构,必须得添进去一些元素,比如要生成一个群,得添进去逆元。所以,作为集合,$S$是$g(f(S))$的子集,或者说存在映射$S\to g(f(S))$,改变$S$就得到了自然变换$\tau:\id_{\mathsf{Set}}\to gf$.

现在来审查条件
\[
	f\xrightarrow{f\tau}fgf\xrightarrow{\sigma f}f,\quad g\xrightarrow{\tau g}gfg\xrightarrow{g\sigma}g.
\]
设$S$是一个集合,则
\[
	f(S)\xrightarrow{f(\tau(S))}f(g(f(S)))\xrightarrow{\sigma(f(S))}f(S),
\]
其中$f(\tau(S))$意味着$f(S)$看成$f(g(f(S)))$的子对象,而$\sigma(f(S))$意味着从$f(g(f(S)))$回忆起$f(S)$的代数结构,他们的复合将得到$f(S)$. 另一个条件也是类似的,这里就不说了。

但是这个定义并不是最常用的,一般而言,我们会使用下面这个定义。

\para 设$f:\mathcal{C}\to \mathcal{D}$和$g:\mathcal{D}\to \mathcal{C}$是一对函子,如果对任意的$T\in\mathcal{C}$和$Y\in\mathcal{D}$存在自然同构
\[
	\alpha(T,Y):\mathcal{C}(T,g(Y))\to \mathcal{D}(f(T),Y),
\]
则$(f,g)$被称为一对\idx{伴随函子}(\idx{adjoint functor})。$f$被称为$g$的左伴随函子,$g$被称为$f$的右伴随函子。

固定$Y$,对$T$的那个自然同构说明$g(Y)$是反变函子$\mathcal{D}(f(\star),Y)$的表示对象。反过来其实也对,如果对每一个$Y$,反变函子$T\mapsto {\mathcal{D}}(f(T),Y)$都是可表的,则我们可以构造出他的一个右伴随函子。所以实际上,双函子同构可以减弱为对$T$的。下面我们证明这一点。

\proof 
	设$\mathcal{C}$和$\mathcal{D}$是两个范畴,而$f:\mathcal{C}\to \mathcal{D}$是一个函子。对于给定的$Y\in\mathcal{D}$,反变函子$T\mapsto {\mathcal{D}}(f(T),Y)$属于$\hat{\mathcal{C}}$. 如果他是可表函子,被$g(Y)$所表示,于是就有函子同构$\alpha(Y):h_{g(Y)}\to {\mathcal{D}}(f(\star),Y)$:给出$S$, $T\in \mathcal{C}$以及同态$\psi:S\to T$,有交换图
	\[
	\begin{xy}
		\xymatrix{
			h_{g(Y)}(T)\ar[rr]^-{\alpha(Y,T)} \ar[d]_{\psi^*}&&{\mathcal{D}}(f(T),Y) \ar[d]^{f(\psi)^*}\\
			h_{g(Y)}(S)\ar[rr]^-{\alpha(Y,S)}&&{\mathcal{D}}(f(S),Y)
		}
	\end{xy}
	\]
	其中$\alpha(Y,T)$用来简记$\alpha(Y)(T)$. 任取$u:T\to {g(Y)}$,有$\alpha(Y,S)(u\psi)=\alpha(Y,T)(u)f(\psi)$. 特别地,当$T=g(Y)$时,取$u=\id_{g(Y)}$,记$\alpha(Y,g(Y))\bigl(\id_{g(Y)}\bigr)=\sigma_Y:f(g(Y))\to Y$,则$\alpha(Y,S)(\psi)=\sigma_Y f(\psi)$,其中$\psi:S\to g(Y)$.

	改变$Y$,即给出一个态射$v:Y\to Y'$,$g(v)=\alpha^{-1}{Y}\left(v\sigma_Y\right):g(Y)\to g(Y')$是一个态射。于是$g:Y\mapsto g(Y)$,以及$g:v\mapsto \alpha^{-1}_{g(Y)}\left(v\sigma_Y\right)$就构成了一个函子$g:\mathcal{D}\to \mathcal{C}$. 再记$\beta_T(Y)=\alpha(Y,T)$,对每一个$T$,$\beta_T$将给出了函子同构
	\[
		\beta_T:\mathcal{C}(T,g(\star))\to \mathcal{D}(f(T),\star).
	\]
	自然变换的检验是直接的。于是$\alpha(Y,T)$就是一个双函子同构。
\qed

\pro 设$L:\mathcal{C}\to \mathcal{D}$是一个左伴随函子,而$R:\mathcal{D}\to \mathcal{C}$是一个右伴随函子。设有一个$\mathcal{C}$中的$I$-图$X:I\to \mathcal{C}$以及一个$\mathcal{D}$中的$J$-图$Y:J\to \mathcal{D}$,于是,$LX$是$\mathcal{D}$中的$I$-图,$RY$是$\mathcal{C}$中的一个$J$图,并且存在同构
\[
	L{\varinjlim}_{i\in I} X_i\cong {\varinjlim}_{i\in I} (LX_i),\quad
	R{\varprojlim}_{j\in J} Y_j\cong {\varprojlim}_{j\in J} (RY_j).
\]
粗略来说,就是左伴随函子与余极限可交换,而右伴随函子与极限可交换。

\proof
	命题的证明是利用Yoneda引理,他有协变和反变两个版本:协变的是,如果函子$\mathcal{C}(X,-)$与函子$\mathcal{C}(X',-)$同构,则$X\cong X'$. 反变的是,如果函子$\mathcal{C}(-,X)$与函子$\mathcal{C}(-,X')$同构,则$X\cong X'$.

	由Proposition \eqref{homlimit},我们知道如下函子同构
	\[
	\mathcal{C}\left(-,{\varprojlim} RX_i\right)\cong {\varprojlim} \mathcal{C}(-,RX_i),
	\]
	连同伴随函子的函子同构,有如下同构链
	\[
	\mathcal{C}\left(-,{\varprojlim} RX_i\right)\cong {\varprojlim} \mathcal{C}(-,RX_i)\cong {\varprojlim} \mathcal{D}(L-,X_i)\cong \mathcal{D}\left(L-,{\varprojlim} X_i\right)\cong \mathcal{C}\left(-,R{\varprojlim} X_i\right),
	\]
	所以${\varprojlim} RX_i\cong R{\varprojlim} X_i$.

	同样由Proposition \eqref{homlimit},有如下函子同构
	\[
	\mathcal{D}\left({\varinjlim} LX_i,-\right)\cong {\varprojlim} \mathcal{D}(LX_i,-),
	\]
	连同伴随函子的函子同构,有如下同构链
	\[
	\mathcal{D}\left({\varinjlim} LX_i,-\right)\cong {\varprojlim} \mathcal{D}(LX_i,-)\cong {\varprojlim} \mathcal{C}(X_i,R-)\cong \mathcal{C}\left({\varinjlim} X_i,R-\right)\cong \mathcal{D}\left(L{\varinjlim} X_i,-\right),
	\]
	所以${\varinjlim} LX_i\cong L{\varinjlim} X_i$.
\qed

\para 首先介绍$R$-模范畴的自由对象,遗忘函子是$R$-模到集合。设$I$是一个指标集,我们定义$R^I=\bigoplus_{i\in I}R$,他被称为自由$R$-模,或者当$R$是清楚地时候,称为自由模。自由模是$R$-模范畴的自由对象。从构造,检查是清楚的。

如果$I$是有限集,$n=|I|$,则我们通常将自由模$R^I$记作$R^n$. 记$1_i$为第$i$个指标的$R$中的$1$.

\para 设$M$是一个$R$-模,$S$是他的一个子集,称呼$M$被$S$生成,就是说$M$中的元素可以写成$S$中元素的有限线性组合。即任取$m\in M$,存在一个系数集$\{a_s\in R\,:\, s\in S\}$,其中只有有限个系数非零,使得
\[
	m=\sum_{s\in S}a_s s.
\]
称呼一个$R$-模$M$是有限生成的,就是说存在一个$M$的有限子集生成他。

显然$R^I$由$\{1_i\,:\,i\in I\}$生成,其中$1_i$代表的是第$i$个指标的$R$中的$1$,所以$R^n$是有限生成模。

\para 设$U$是一个集合,上面有乘法运算,满足结合律,这样的代数结构被称为\idx{半群}(\idx{semigroup}). 如果半群含有乘法单位元,则称他为\idx{幺半群}(\idx{monoid}). 群是特殊的幺半群,一个环除去加法结构也是一个幺半群。半群之间的同态依然定义为满足$f(ab)=f(a)f(b)$的映射$f$. 所以所有半群构成一个范畴,所有幺半群也构成一个范畴。

从幺半群到集合有自然的遗忘函子。反过来,下面从一个集合构造一个幺半群,这就构成了幺半群范畴的自由对象。

设$S$是一个集合,设$U(S)$作为集合是$S$中所有有限长序列(或者说有序组)的集合,其中空序列也看成有限长序列。对于序列$(a_1$, $\cdots$, $a_n)$,我们通常记作$a_1\cdots a_n$,将其称为一个字符串,而空字符串记作$1$. 两个字符串之间的乘法就定义成字符串的连接,即
\[
	(a_1\cdots a_n)(b_1\cdots b_m)=a_1\cdots a_nb_1\cdots b_m,
\]
空字符串因此自然地成为该乘法的单位元,结合律由构造也是显然的。因此$U(S)$就构成了一个幺半群,他被称为被$S$生成的自由幺半群。泛性质的检验是直接的。

自由交换幺半群的构造也是类似的,可以理解成所有有限长无序组,乘法定义为无序组的拼接。

\para 类似于幺半群,从集合$S$出发,可以构造自由群$G(S)$,他是群范畴的自由对象。首先从每个$s\in S$出发,新建一个符号$s^{-1}$,这样就有了一个新的集合$S^{-1}=\{s^{-1}\,:\, s\in S\}$. 然后考虑自由幺半群$U(S\cup S^{-1})$. 通过$\mu: ss^{-1}\mapsto 1$以及$\mu: s^{-1}s\mapsto 1$定义一个半群自同态$\mu:U(S\cup S^{-1})\to U(S\cup S^{-1})$,则我们需要的自由群$G(S)$就是$\mu$在$U(S\cup S^{-1})$中的像。$G(S)$确确实实是一个群的检验是直接的,这里略去。

上面这个构造也可以看成从一个幺半群构造一个群的过程,这点直接令$S$是一个幺半群即可。

\para 设$S$是群$G$的一个子集,考虑自由群$G(S)$,通过任取$s\in S$,$\mu(s)=s$将给给出一个群同态$\mu:S\mapsto G$,定义子群$\langle S\rangle=\mu(G(S))$,他被称为集合$S$生成的子群。

\para 当然也有不是到集合范畴的忘却函子,比如将一个交换群看成一个群的忘却函子。那么反过来,我们要从群构造一个交换群,这就构成了群的交换化的过程。这是上面那个忘却函子的自由对象。

设$G$是一个群,记$[G,G]$是$G$中所有形如$aba^{-1}b^{-1}$的元素构成的集合生成的子群。这是一个正规子群,任取$b\in G$,存在形如$aba^{-1}b^{-1}\in [G,G]$使得$(aba^{-1}b^{-1})b=aba^{-1}=b(b^{-1}aba^{-1})\in b[G,G]$,所以$[G,G]b=b[G,G]$. 考虑商群$G/[G,G]$,可以看到,这是一个交换群,因为在这个商群中,$aba^{-1}b^{-1}$和$e$看成是一样的,即$ab$和$ba$看成是一样的。

从一个集合$S$,可以构造自由群$G(S)$,然后将其交换化得到了一个交换群$A(S)=G(S)/[G(S),G(S)]$. 这被称为自由交换群,他是交换群范畴到集合范畴的遗忘函子伴随的自由对象。而这又可以通过看成自由$\mathbb{Z}$-模。

\section{张量积}

\para 张量抽象了多线性函数,尤其是双线性函数。设$A$是右$R$-模,$B$是左$R$-模,那么我们称$f:A\times B\to G$,其中$G$是一个交换群,为一个双线性函数,如果他满足
\[
	f(a+b,a')=f(a,a')+f(b,a'),\quad f(a,a'+b')=f(a,a')+f(a,b'),
\]
以及对$r\in R$满足$f(ar,b)=f(a,rb)$.

特别地,如果我们存在一个双线性函数$\varphi$,以及一个交换群$G$,使得每一个$A\times B$上的双线性函数$f:A\times B\to H$都可以唯一分解为
\[
	f:A\times B\xrightarrow{\varphi} G\xrightarrow{h_f}H,
\]
其中$h_f$对每一个双线性函数$f$存在且唯一,则称呼$G$为$A$与$B$的张量积,记做$A\otimes_R B$,而$\varphi(a,b)$记做$a\otimes_R b$,如果下标$R$不重要,那么我们可以省略他。

设$R$是交换环,$A$和$B$都是$R$-模,那么$A\times B$具有$R$-模结构,即$r(a,b)=(ra,b)$. 通过将$r(a\otimes b)$可以定义为$(ra)\otimes b$,$A\otimes_R B$显然有一个$R$-模结构,这样,$\varphi$将是一个$R$-模同态。而上面的泛性质,也可以改成相应的$R$-模和$R$-模同态。

\lem 在模范畴内,张量积存在。而且由上面的泛性质,他确定到一个同构。

\proof 对于唯一性,我们考虑两个张量积$(G,\varphi)$和$(G',\varphi')$,那么根据张量积的性质,有分解
\[
	\varphi:A\times B\xrightarrow{\varphi'} G'\xrightarrow{h_{\varphi}}G,\quad \varphi':A\times B\xrightarrow{\varphi} G\xrightarrow{h_{\varphi'}}G',
\]
所以只要验证$h_\varphi\circ h_{\varphi'}=\id_{G}$和$h_{\varphi'}\circ h_{\varphi}=\id_{G'}$就好了,这样我们就得到了$G$和$G'$之间的同构。而上述等式来自于分解的唯一性,显然,我们有分解
\[
	A\times B\xrightarrow{\varphi'} G'\xrightarrow{h_{\varphi}}G\xrightarrow{h_{\varphi'}}G',\quad A\times B\xrightarrow{\varphi'} G'\xrightarrow{\id_{G'}}G',
\]
显然,所以由唯一性得到了$h_{\varphi'}\circ h_{\varphi}=\id_{G'}$,同理有另一个等式。

对于存在性,我们可以直接构造。首先,我们知道在交换群范畴(作为$\zz$-模)有直和存在,那么我们可以构造自由交换群
\[
	F=\bigoplus_{(a,b)\in A\times B} \zz=\sum_{(a,b)\in A\times B}n_{(a,b)}1_{a,b}.
\]
这是一个形式和,系数只有有限项非零,其中$1_{a,b}$是对应指标$(a,b)$的那个$\zz$中的$1$。我们令$H$是由
\[
	1_{a,b}+1_{a',b}-1_{a+a',b},\quad 1_{a,b}+1_{a,b'}-1_{a,b+b'},\quad 1_{ar,b}-1_{a,rb}
\]
生成的子群,那么我们$A\otimes B$就可以构造为$F/H$,令$\varphi(a,b)=a\otimes b=[1_{a,b}]$,即$1_{a,b}$的陪集。这确实是张量积,具体的检验这里就不进行了。\qed

\para 有了两个模的张量积,我们自然也可以拓展为三个模的张量积,我们可以通过模仿两个模的张量积的泛性质\footnote{即三线性的函数可以唯一分解。},定义一个新的三个模之间的张量积$A\otimes B\otimes C$,然后可以检验$(A\otimes B)\otimes C$和$A\otimes (B\otimes C)$同时也满足泛性质,所以他们之间是同构的。在这层意义上,我们可以认为张量积满足结合律,因此我们自然也有了有限个模的张量积。

\para 下面讨论张量积在同态下的表现。设$\varphi:M\to M'$是右$R$-模同态和$\psi:N\to N'$是左$R$-模同态,则它诱导了一个交换群同态
\[
	\alpha(\varphi,\psi):M\otimes N\to M'\otimes N',
\]
使得复合公式
\[
	\alpha(\varphi'\varphi,\psi'\psi)=\alpha(\varphi',\psi')\alpha(\varphi,\psi)
\]
成立。大家经常将$\alpha(\varphi,\psi)$记作$\varphi\otimes \psi$,两个模同态的张量积的意义后面会阐述,这里看成形式的符号即可。

\proof
	映射
	\[
	\begin{array}{ccc}
		M\times N&\to& M'\otimes N'\\
		(m,n)&\mapsto& \varphi(m)\otimes \psi(n)
	\end{array}
	\]
	显然是双线性的,所以它诱导了映射$M\otimes N\to M'\otimes N'$,而张量积的泛性质中的唯一性给出了复合公式,于是这就是我们想要的$\alpha(\varphi,\psi)$.
\qed

注意到复合公式成立意味着成立交换图
\[
\begin{xy}
	\xymatrix
	{
		M\otimes N\ar[rr]^{\alpha(\id_M,\psi)}\ar[d]_{\alpha(\varphi,\id_{N})}&&M\otimes N'\ar[d]^{\alpha(\varphi,\id_{N'})}\\
		M'\otimes N\ar[rr]^{\alpha(\id_{M'},\psi)}&&M'\otimes N'
	}
\end{xy}
\]
成立。

上面的交换图用范畴论的语言来表述就是,$-\otimes N$以及$-\otimes \id_N:=\alpha(-,\id_N)$一起构成了一个函子,而同态$N\to N'$将诱导出一个函子间的态射(即自然变换)。这样的同态我们称为自然同态或者函子式同态,描述同态其实就是建立范畴论的初衷,为了抽象出自然变换才定义了范畴等概念。然后到了教科书上,逻辑就反了过来。同理,$(N\otimes -,\id_N\otimes -)$也是一个函子。

\para 上一个命题给出了如下映射:
\[
	\alpha:\Hom(M,M')\times \Hom(N,N')\to \Hom(M\otimes N,M'\otimes N').
\]
由于$\Hom(M,M')$和$\Hom(N,N')$都是$R$-模,并且上述映射是双线性的,所以他自然诱导了一个映射
\[
	\Hom(M,M')\otimes \Hom(N,N')\to \Hom(M\otimes N,M'\otimes N').
\]
这也就是为什么大家将$\alpha(\varphi,\psi)$记作$\varphi\otimes \psi$. 

\pro 记左$R$-模范畴之间的态射集为$\Hom_R(-,*)$. 则成立双函子同构
\[
	\Hom_{\mathbb{Z}}(M\otimes_R N,G)\cong \Hom_R(M,\Hom_\mathbb{Z}(N,G)),
\]
其中$M$和$N$是可以变动的。因此,$-\otimes_R N$就是一个左伴随函子,而$\Hom_\mathbb{Z}(N,*)=h^N$是一个右伴随函子。\notprove

根据左伴随函子与逆极限可交换,而直和是模范畴的一种逆极限,我们可以得知同构
\[
	\left(\bigoplus_i M_i\right)\otimes N\cong \bigoplus_i \left(M_i\otimes N\right).
\]

\pro 成立交换群的同构$R\otimes M\cong M$以及$(R/\mathfrak{a})\otimes M\cong M/\mathfrak{a}M$,其中$\mathfrak{a}$是$R$的一个理想。

\proof
	记$(r,m)\mapsto rm$为映射$\varphi$,毫无疑问这是一个双线性映射,所以诱导了一个交换群同态$\psi:R\otimes M\to M$,使得分解$\varphi:R\times M\xrightarrow{h} R\otimes M\xrightarrow{\psi} M$成立。毫无疑问,$\psi$是一个满射,因为对任意的$m$都有$m=\varphi(1,m)$,所以$m=\psi(h(1,m))$. 注意到$R\otimes M$里面的元素可以写成这样的有限和
	\[
	\sum_i r_i\otimes m_i=\sum_i 1\otimes r_im_i=1\otimes \left(\sum_i r_i m_i\right),
	\]
	因此,$\psi\left(\sum_i r_i\otimes m_i\right)=0$推出$\sum_i r_i m_i=0$,再利用上面的等式推出$\sum_i r_i\otimes m_i$. 所以$\psi$是一个单射。

	第二个同构,依然是记$(\bar{r},m)\mapsto \overline{rm}\in M/\mathfrak{a}M$为映射$\varphi$. 这是一个定义良好的映射,选取$\bar{r}$的不同代表元$r'=r+a$,其中$a\in\mathfrak{a}$,所以$r'm=rm+am$,由于$am\in \mathfrak{a}M$,所以$\overline{r'm}=\overline{rm}$. 双线性性从定义看是显然的,所以诱导了一个交换群同态$\psi:(R/\mathfrak{a})\otimes M\to M/\mathfrak{a}M$,使得分解$\varphi:R/\mathfrak{a}\times M\xrightarrow{h} (R/\mathfrak{a})\otimes M\xrightarrow{\psi} M/\mathfrak{a}M$成立。

	先检验$\psi$是一个满射,这只要检验$\varphi$是一个满射。任取$\bar{m}\in M/\mathfrak{a}M$,则有$\varphi(\bar{1},m)=\bar{m}$. 最后要检验$\psi$是一个单射,
	如果$\psi(\sum_i \bar{r}_i\otimes m_i)=0$,可以推出$\sum_i \overline{r_im_i}=0$,于是$\sum_i r_im_i \in \mathfrak{a}M$,这意味着$r_i\in \mathfrak{a}$,所以$\sum_i \bar{r}_i\otimes m_i=\sum_i 0\otimes m_i=0$.
\qed

当$R$是一个交换环,$M$是一个$R$-模,此时上面证明的同态实际上都是$R$-模同态,所以也自然是$R$-模同构。这两个同构不仅仅是同构,改变$M$的时候,还是一个自然同构。这里就略去检验了。很多时候,这不是检验不检验的问题,这是一个相信不相信的问题。

\para 上面我们已经看到了,$-\otimes N$构成一个函子,而实际上,它是右正和函子。

\para 从这里开始,以后讨论的张量积都是$R$-模范畴的张量积,所以两个$R$-模的张量积依然是$R$-模,所有出现的同态都是$R$-模同态。并且,我们有函子性同构$-\otimes N\cong N\otimes -$.


\section{代数与多项式环}

设有两个环$A$, $B$和一个环同态$f:A\to B$,再设$a\in A$, $b\in B$,我们可以通过$f$定义他们的乘法为$a\cdot b=f(a)b\in B$,这样环$B$就被赋予了一个$A$-模结构。环$B$若被赋予一个$A$-模结构,则称$B$是一个$A$-\idx{代数}(\idx{algebra})。

设$B$是$A$-代数,有$f:A\to B$,设$C$也是$A$-代数,有$g:A\to C$,那么$A$-代数之间的同态$h:B\to C$,首先是$B$和$C$之间的环同态,还要和$A$-模结构相容,即$g=h\circ f$. 所有$A$-代数构成一个范畴,记作$\mathsf{Alg}_A$,其中的对象由一个环$B$和一个环同态$\mu:A\to B$构成。

多项式环是幺半群环的一种,所以直接从幺半群环开始。

\para 设$U$是一个幺半群,而$R$是一个环,将其看成一个集合,那么从$U$出发可以构造一个自由$R$-模$R(U)$. 任取$a\in R(U)$,他可以写成$a=\sum_{i\in I} r_i u_i$,其中$I$是一个有限指标集。类似地,同样任取$b\in R(U)$可以写成$b=\sum_{i\in J} s_i v_i$,而$J$同样是一个有限指标集。定义乘法
\[
	ab=\sum_{i\in I,\, j\in J}r_is_j u_i v_j,
\]
结合律与分配律是自然成立的,于是$R(U)$是一个环。

如果$U$是一个交换幺半群,则$R(U)$是一个交换环。此时$R(U)$是一个$R$-代数。

\para 设$S$是一个集合,他生成的自由交换幺半群为$U(S)$,记$R[S]=R(U(S))$,称其为$R$上由集合$S$生成的多项式环。特别地,如果$S$是有限集$\{x_1$, $\cdots$, $x_n\}$,则$R[S]$记作$R[x_1$, $\cdots$, $x_n]$. 

当$S=\{x\}$,$R[x]$中的元素写成$f=\sum_{i=0}^na_ix^i$的形式,这就是熟悉的一元多项式。$x$称之为不定元,不定元只是一个符号,在不发生歧义的情况下可以任意选取。这即如下事实的体现:如果两个集合存在双射,则他们生成的多项式环同构。

从定义,不难发现作为$R$-代数,$R[S\cup T]$与$R[S][T]$同构,所以完全可以把这两个$R$-代数等同。

与从$R$-代数范畴到集合范畴的忘却函子伴随的,正是从一个集合构造多项式。所以多项式是$R$-代数范畴的自由对象。

\para 设$A$是$R$-代数,如果存在满同态$R[x_1,\cdots ,x_n]\to B$,或者说$B$同构于$R[x_1$, $\cdots$, $x_n]/\mathfrak{a}$,其中$\mathfrak{a}$是$R[x_1,\cdots ,x_n]$的一个理想,则我们称呼$B$是有限生成$R$-代数。一个环称为有限生成的就是指他作为$\zz$-代数是有限生成的。如果$B$作为$R$-模是有限生成的,则称$B$作为$A$-代数是有限的。

\para 一个一元多项式$f=\sum_{i=0}^na_ix^i$使得$a_i$不为零的最大的$i$记作$\deg(f)$,称为多项式$f$的幂次,$0$的幂次补充定义为$0$. 最高幂次的系数我们称之为最高次系数,或者首项系数。其余的项目通常我们会指出具体的指标,比如说$k$-次项,就是指$a_kx^k$,而$k$-此项系数就是指$a_k$. 如果最高次系数为$1$,这个多项式称为首一多项式。

对于首一多项式而言$f$,$\deg(fg)=\deg(f)+\deg(g)$. 但是一般的多项式并不如此,比如在$\zz/4\zz$上的多项式$2x$,有$(2x)(2x)=0$.

对于$\deg(f+g)$,我们有估计$\deg(f+g)\leq \max\{\deg(f),\deg(g)\}$,不等号是可以取严格的,同样比如在$\zz/4\zz$上的多项式$2x$,有$2x+2x=0$.

\theo 多项式除法算法:设$f$, $g\in R[x]$,而且$g$的首项系数可逆,则存在唯一的多项式$p$, $r\in R[x]$使得$f=pg+r$,且$\deg(r)<\deg(f)$.

\proof
	可以假设$g$是首一多项式,因为如果设$g$的首项系数为$a$,则$g/a$是一个首一多项式,如果命题对首一多项式成立,即存在$q$, $r\in R[x]$使得$f=q(g/a)+r$,则$f=(q/a)g+r=pg+r$,其中$p=q/a$和$r$都是多项式。这样就得到了我们的命题。

	首先假设存在,证明唯一性。设$f=p'g+r'$以及$f=pg+r$成立,则$(p'-p)g=r'-r$,由于$g$是首一的,且$p'-p$非零,所以$\deg((p'-p)g)\geq \deg(g)$,但是$\deg(r'-r)< \deg(g)$,这就造成了矛盾。下面证明存在性。

	设$\deg(f)=n$, $\deg(g)=m$,如果$n<m$,则取$p=0$, $r=f$. 考虑$n\leq m$的情况,设$f$的首项系数为$a_0$,考虑多项式$f_1=f-a_0x^{n-m}g$,由于$f$和$a_0x^{n-m}g$的最高次项相同,所以$\deg(f_1)<\deg(f)$,如果$\deg(f_1)<\deg(g)$,那么$f=a_0x^{n-m}g+f_1$就给出了分解。

	否则继续对$f_1$进行这样的操作,得到$f_2=f_1-a_1x^{\deg(f_1)-m}g$,再比较$\deg(f_2)$与$\deg(g)$. 不断如是进行下去,由于$f$的幂次有限,而每次操作,幂次都至少减一,所以该过程在进行至多$n-m+1$次后就会停止。设该过程在第$k$次后停止,则我们就得到了
	\[
	f_{k}=f-a_0x^{n-m}g-a_1x^{\deg(f_1)-m}g-\cdots-a_{k-1}x^{\deg(f_{k-1})-m}g
	\]
	使得$\deg(f_k)< \deg(g)$. 即$f=f_0$,则我们就得到了分解
	\[
	f=f_k+a_0x^{n-m}g+a_1x^{\deg(f_1)-m}g+\cdots+a_{k-1}x^{\deg(f_{k-1})-m}g=\left(\sum_{i=0}^{k-1}a_{i}x^{\deg(f_{i})-m}\right)g+f_k.
	\]
	因此$p=\sum_{i=0}^{k-1}a_{i}x^{\deg(f_{i})-m}$以及$r=f_k$就是我们需要的多项式。
\qed

存在性的证明就是整个算法,从算法来看,这个命题即使是对非交换环上的多项式环也是成立的。

\section{矩阵与行列式}

\para 记$1_i=(0,\,\cdots\!,\,1,\,\cdots\!,\,0)\in R^m$,即只有第$i$个位置是一,其他都是零的那个元素。考虑一个模同态$\varphi:R^m\to R^n$,由于
\[
	\varphi\left(\sum_{i} r_i1_i\right)=\sum_{i} r_i\varphi\left(1_i\right),
\]
所以它由所有$\varphi\left(1_i\right)$描述,由于他在$R^n$中,将其写作
\[
	\varphi\left(1_i\right)=\sum_{j}\varphi_{ji}1_j.
\]
于是$\{\varphi_{ij}\,:\, 1\leq i \leq p,\,\,1\leq j \leq q\}$这$p\times q$个数构成的整体称为$\varphi$的矩阵,记作$(\varphi)$,他完全描述了整个同态$\varphi$.

设$\psi:R^n\to R^l$的矩阵是$(\psi)_{ij}$,我们下面计算$\psi\circ\varphi$的矩阵:
\[
	\psi\circ\varphi(1_i)=\psi\left(\sum_{j=1}^n\varphi_{ji}1_j\right)=\sum_{j=1}^n\varphi_{ji}\psi(1_j)=\sum_{j=1}^n\sum_{k=1}^l\psi_{kj}\varphi_{ji}1_k,
\]
所以$\psi\circ\varphi$的矩阵$(\psi\circ\varphi)$为
\[
	(\psi\circ\varphi)_{ki}=\sum_{j=1}^n\psi_{kj}\varphi_{ji},
\]
这被称为矩阵乘法法则。

\para 通常会使用一张表来表示矩阵
\[
(\varphi)=
\begin{pmatrix}
	\varphi_{11} & \varphi_{12} & \cdots & \varphi_{1m}\\
	\varphi_{21} & \varphi_{22} & \cdots & \varphi_{2m}\\
	\vdots & \vdots & \ddots & \vdots \\
	\varphi_{n1} & \varphi_{n2} & \cdots & \varphi_{nm}\\
\end{pmatrix}
\]
可以看到$\varphi(1_i)$关于$\{1_j\in R^n\}$中展开的系数出现在$(\varphi)$的第$i$列。

下面将单独研究矩阵,而不将其看成某个自由模的矩阵。之所以这样,因为有时候我们处理的矩阵并不能很好地表为自由模之间的同态。记号上,表示矩阵时可加或不加括号,比如$\varphi$或者$(\varphi)$. 相应的,表示矩阵元时写作$\varphi_{ij}$或者$(\varphi)_{ij}$. 如果需要以矩阵元来表述矩阵,写作$(\varphi_{ij})$. 

按照习惯,我们下面依旧假设$R$是一个交换环,虽然在非交换环上谈论矩阵也是可能的。称呼一个矩阵是$R$上的矩阵,如果他的矩阵元都属于$R$.

所有环$R$的$m\times n$矩阵构成一个$R$-模,实际上,任取$r\in R$以及矩阵$\varphi$,定义矩阵$r\varphi= (r\varphi_{ij})$. 另取矩阵$\psi$,定义矩阵加法$\psi+\varphi=(\psi_{ij}+\varphi_{ij})$.

矩阵乘法就按照自由模里面推出的法则来定义:
\begin{equation}
\begin{pmatrix}
	\psi_{11} & \psi_{12} & \cdots & \psi_{1n}\\
	\psi_{21} & \psi_{22} & \cdots & \psi_{2n}\\
	\vdots & \vdots & \ddots & \vdots \\
	\psi_{l1} & \psi_{l2} & \cdots & \psi_{ln}
\end{pmatrix}
\begin{pmatrix}
	\varphi_{11} & \varphi_{12} & \cdots & \varphi_{1m}\\
	\varphi_{21} & \varphi_{22} & \cdots & \varphi_{2m}\\
	\vdots & \vdots & \ddots & \vdots \\
	\varphi_{n1} & \varphi_{n2} & \cdots & \varphi_{nm}\\
\end{pmatrix}
=
\begin{pmatrix}
	\sum_{i=1}^n \psi_{1i}\varphi_{i1} & \cdots & \sum_{i=1}^n \psi_{1i}\varphi_{im}\\
	\sum_{i=1}^n \psi_{2i}\varphi_{i1} & \cdots & \sum_{i=1}^n \psi_{2i}\varphi_{im}\\
	\vdots & \ddots & \vdots \\
	\sum_{i=1}^n \psi_{li}\varphi_{i1} & \cdots & \sum_{i=1}^n \psi_{li}\varphi_{im}
\end{pmatrix}
\end{equation}
即$(\psi\varphi)_{ij}$由第一个矩阵的第$i$行和第二个矩阵的第$j$列逐个相乘然后相加而得,矩阵乘法把$l\times n$和$n \times m$的矩阵变成了$l \times m$的矩阵。矩阵乘法满足结合律是直接的计算。

尤其重要的情况是$m=1$的时候,此时,$n\times 1$的矩阵称为一个列,一个列我们直接记成一个$n$-元组$\varphi=(\varphi_1$, $\cdots$, $\varphi_n)$. 其中,记$1_i=(0$, $\cdots$, $1$, $\cdots$, $0)$,即只有第$i$个位置为$1$,其他位置都为零的那个列。所有的列都可以看成$\{1_i\,:\, 1\leq i \leq n\}$的线性组合,
\[
	(\varphi_1,\,\cdots\!,\,\varphi_n)=\sum_{i=1}^n\varphi_i1_i.
\]
特别地,记矩阵$I_n=\begin{pmatrix}1_{1} & 1_{2} & \cdots & 1_{n}\end{pmatrix}$,称之为单位矩阵。由矩阵乘法,对任意的$m\times n$矩阵$\varphi$有$\varphi I_n=\varphi$以及$I_m\varphi =\varphi$成立。单位矩阵具体写出来就是
\[
	I_n=
		\begin{pmatrix}
			1 & & &\\
			& 1 & &\\
			& & \ddots &\\
			& & & 1
		\end{pmatrix}
	=(\delta_{ij}),
\]
其中$\delta_{ij}$被称为Kronecker符号,或者叫Kronecker delta,定义为
\[
	\delta_{{ij}}=
	\begin{cases}
	1,&\text{if } i=j,\\
	0,&\text{if } i\neq j.
	\end{cases}
\]

设$\varphi_{i}=(\varphi_{1i}$, $\cdots$, $\varphi_{ni})$,则一个矩阵可以看成一些列的并列,
\[
	\begin{pmatrix}
	\varphi_{11} & \varphi_{12} & \cdots & \varphi_{1m}\\
	\varphi_{21} & \varphi_{22} & \cdots & \varphi_{2m}\\
	\vdots & \vdots & \ddots & \vdots \\
	\varphi_{n1} & \varphi_{n2} & \cdots & \varphi_{nm}\\
	\end{pmatrix}
	=
	\begin{pmatrix}
	\varphi_{1} & \varphi_{2} & \cdots & \varphi_{m}
	\end{pmatrix},
\]
其中$\varphi_i=(\varphi_{1i}$, $\cdots$, $\varphi_{ni})$. 

关于列的矩阵乘法为
\begin{equation}
\begin{pmatrix}
	\psi_{11} & \psi_{12} & \cdots & \psi_{1n}\\
	\psi_{21} & \psi_{22} & \cdots & \psi_{2n}\\
	\vdots & \vdots & \ddots & \vdots \\
	\psi_{l1} & \psi_{l2} & \cdots & \psi_{ln}
\end{pmatrix}
\begin{pmatrix}
	\varphi_{1} \\
	\varphi_{2}  \\
	\vdots \\
	\varphi_{n} \\
\end{pmatrix}
=
\begin{pmatrix}
	\sum_{i=1}^n \psi_{1i}\varphi_{i}\\
	\sum_{i=1}^n \psi_{2i}\varphi_{i} \\
	\vdots \\
	\sum_{i=1}^n \psi_{li}\varphi_{i}
\end{pmatrix}
\end{equation}
通常将其右侧写作$\psi\varphi$. 那么一般的矩阵乘法则写作:
\[
	\psi
	\begin{pmatrix}
	\varphi_{1} & \varphi_{2} & \cdots & \varphi_{m}
	\end{pmatrix}
	=
	\begin{pmatrix}
	\psi\varphi_{1} & \psi\varphi_{2} & \cdots & \psi\varphi_{m}
	\end{pmatrix}.
\]

\para 下面要讨论行列式函数,行列式函数是对方阵定义的函数,所谓方阵就是$n\times n$矩阵。将所有矩阵元属于环$R$的$n\times n$矩阵构成的集合记作$M_n(R)$. $M_n(R)$通过矩阵加法和乘法构成一个环,但不是交换环,单位元是$I_n$. 

对$M(R,1)$可以通过$(r)\mapsto r$定义一个函数$f_1:M_n(R)\to R$. 然后对$M_n(R)$里面的矩阵$\varphi$,记$\varphi$除去第$i$行第$j$列得到的$(n-1)\times (n-1)$矩阵为$\Phi_{ij}$,定义
\[
	f_n(\varphi)=\sum_{i=1}^n (-1)^{i+j}\varphi_{ij}f_{n-1}(\Phi_{ij}).
\]
这个展开被称为按第$j$列展开,这个归纳定义的目前的缺陷在于,似乎按照不同的两列展开会有不同的结果。其实并不会这样,首先考虑$n=2$的情况,设$\varphi=\begin{pmatrix}a&b\\c&d\end{pmatrix}$,按第一列展开会得到$ad-bc$,按第二列展开也会得到$ad-bc$. 然后假设当$i<n$的时候,$f_i$都已经定义良好了,即并不依赖于按列展开的选取。那么考虑$f_n(\varphi)$的按第$i$列和按第$j$列的两个展开,可以假设$i<j$。对按照第$i$列展开的式子,对所有$f_{n-1}(\Phi_{kl})$按第$(j-1)$列展开。对按照第$j$列展开的式子,对所有$f_{n-1}(\Phi_{kl})$按第$i$列展开。这两个展开将会得到相同的结果。所以$f_n(\varphi)$的定义并不依赖于对列的展开。

这样,对每一个$n$,我们归纳定义了$f_n:M_n(R)\to R$,这个函数对每一个$n$统一记作$\det:M_n(R)\to R$,称之为行列式。设$n\times n$矩阵为$\varphi=\begin{pmatrix}\varphi_{1} & \varphi_{2} & \cdots & \varphi_{n}\end{pmatrix}$,他的行列式记作$\det(\varphi)=\det(\varphi_1$, $\cdots$, $\varphi_n)$,其中每一个$\varphi_i$都是一个$n\times 1$的列。如果矩阵要把他所有的元素写出来,则行列式会记作
\[
\det(\psi)=\begin{vmatrix}
	\psi_{11} & \psi_{12} & \cdots & \psi_{1n}\\
	\psi_{21} & \psi_{22} & \cdots & \psi_{2n}\\
	\vdots & \vdots & \ddots & \vdots \\
	\psi_{l1} & \psi_{l2} & \cdots & \psi_{ln}
\end{vmatrix}.
\]

\pro 行列式具有以下性质:

1. 行列式对每一个列线性。

2. 行列式中如果有两列相同,则行列式为零。

3. $\det(I_n)=1$对每一个$n$都成立。

\proof 
	第一点由按列展开的定义可得。对于第二点,我们采用对$n$归纳,当$n=2$的时候,
	\[
		\begin{vmatrix}a&a\\b&b\end{vmatrix}=ab-ab=0.
	\]
	当$n>3$的时候,选不同于该两列的第三列展开,就可以归结到$n-1$的情况,随后由归纳法就得到了结论。

	第三点是直接的计算,对第$n$列展开,我们就得到了
	\[
		\det(I_n)=\det(1_1,\,\cdots\!,\,1_{n-1})=\det(I_{n-1}),
	\]
	最后由$\det(I_1)=\det(1_1)=1$就得到了结论。
\qed

应用第一、第二点,我们有
\[
	0=\det(\cdots\!,\psi+\varphi,\,\cdots\!,\,\psi+\varphi,\cdots)=\det(\cdots\!,\psi,\,\cdots\!,\,\varphi,\cdots)+\det(\cdots\!,\varphi,\,\cdots\!,\,\psi,\cdots),
\]
所以,调换行列式的两列,行列式的值变成相反数。

满足上列性质第一、第二点的函数$M_n(R)\to R$被称为反对称多线性函数。反对称的意思就是调换任意两列将得到相反的结果。

\para 给定一个给定矩阵$\psi=\begin{pmatrix}\psi_{1} & \psi_{2} & \cdots & \psi_{n}\end{pmatrix}$,由于每一个$\psi_i$都可以写成$\psi_i=\sum_{j=1}^n\psi_{ji}1_j$,所以任取一个反对称多线性函数$F$,则
\[
	F(\psi_1,\,\cdots\!,\,\psi_n)=\sum_{j_1=1}^n\cdots \sum_{j_n=1}^n \psi_{j_11}\cdots \psi_{j_n n} F(1_{j_1},\,\cdots\!,\,1_{j_n}),
\]
所以对反对称多线性函数的计算,只要计算所有形如$F(1_{i_1}$, $\cdots$, $1_{i_n})$的式子即可。同时,上面的式子也说明了,多线性函数是矩阵元的多项式函数。

\pro 设$F:M_n(R)\to R$是一个反对称多线性函数,则$F=F(I_n)D$. 作为推论,行列式的三点性质将完全决定行列式函数。

\proof 
	从上面看到的,只要证明
	\[
		F(1_{i_1},\,\cdots\!,\,1_{i_n})=F(I_n)D(1_{i_1},\,\cdots\!,\,1_{i_n}).
	\]
	如果$\{i_1$, $\cdots$, $i_n\}$中有重复指标,两边都为零。如果没有重复指标,采用冒泡排序\footnote{先比较相邻的元素,如果第一个比第二个大,就交换他们两个。对每一对相邻元素作同样的工作,从开始第一对到结尾的最后一对。在这一点,最后的元素应该会是最大的数。针对所有的元素重复以上的步骤,除了最后一个。持续每次对越来越少的元素重复上面的步骤,直到没有任何一对数字需要比较。},可以通过不断两两调换两列,将$(i_1$, $\cdots$, $i_n)$变成$(1$, $\cdots$, $n)$,对等式两边进行同样的调换两列的操作,等式依旧成立,所以最后只要检验$F(I_n)=F(I_n)D(I_n)$,而这来自于$D(I_n)=1$.
\qed

\pro 行列式乘法公式:$\det(\varphi \psi)=\det(\varphi)\det(\psi)$.

\proof 
	由矩阵乘法,
	\[
	\varphi
	\begin{pmatrix}
	\psi_{1} & \psi_{2} & \cdots & \psi_{n}\\
	\end{pmatrix}
	=
	\begin{pmatrix}
	\varphi\psi_{1} & \varphi\psi_{2} & \cdots & \varphi\psi_{n}\\
	\end{pmatrix}
	\]
	因此$\det (\varphi \psi)=\det (\varphi\psi_{1}$, $\cdots$, $\varphi\psi_{n})$.

	将$\det(\varphi\psi_{1}$, $\cdots$, $\varphi\psi_{n})$记作$F(\psi_{1}$, $\cdots$, $\psi_{n})$,可以看到这是一个反对称的多重线性映射,所以
	\[
	F(\psi_{1},\,\cdots\!,\,\psi_{n})=F(I_n)\det (\psi_{1},\,\cdots\!,\,\psi_{n})
	\]
	也就是说
	\[
	\det (\varphi\psi_{1},\,\cdots\!,\,\varphi\psi_{n})=\det(\varphi 1_1,\,\cdots\!,\,\varphi 1_n)\det(\psi_{1},\,\cdots\!,\,\psi_{n}).
	\]

	最后,只需要计算$\varphi 1_i=\begin{pmatrix}\varphi_{1} & \varphi_{2} & \cdots & \varphi_{n}\end{pmatrix}1_i$,由矩阵乘法$\varphi 1_i=\varphi_i$,所以
	\[
	\det(\varphi\psi)=\det(\varphi\psi_{1},\,\cdots\!,\,\varphi\psi_{n})=\det(\varphi_1,\,\cdots\!,\,\varphi_n)\det(\psi_{1},\,\cdots\!,\,\psi_{n})=\det(\varphi)\det(\psi).
	\]
\qed

\para 矩阵的一个起源是线性方程组,这里暂时只考虑$n$个方程$n$个未知元的情况。所谓的线性方程组即如下的方程
\[
	\begin{cases}
	a_{11}x_1+\cdots+a_{1n}x_n=b_1,\\
	\qquad\qquad\,\vdots\\
	a_{n1}x_1+\cdots+a_{nn}x_n=b_n,
	\end{cases}
\]
其中$a_{ij}$, $b_i\in R$是已知的,而解线性方程组就是找到$x_i\in R$满足上述方程。

采用矩阵和矩阵乘法,上述的线性方程组可以写作$ax=b$,其中$a=(a_{ij})=\begin{pmatrix}a_{1} & a_{2} & \cdots & a_{n}\end{pmatrix}$, $x=(x_1$, $\cdots$, $x_n)$和$b=(b_1$, $\cdots$, $b_n)$. 下面将利用行列式给出一种可能的求解方式,其实也就是著名的Cramer法则。陈述如下:

将矩阵的第$i$列换作$b$,计算这个新矩阵的行列式,有
\[
	\det(a_1,\,\cdots\!,\,b,\,\cdots\!,\,a_n)=\det\left(a_1,\,\cdots\!,\,\sum_{j=1}^na_j x_j,\,\cdots\!,\,a_n\right)=\sum_{j=1}^n x_j \det(a_1,\,\cdots\!,\,a_j,\,\cdots\!,\,a_n),
\]
在求和中,除了$j=i$可能行列式$\det(a_1$, $\cdots$, $a_j$, $\cdots$, $a_n)$不为零,其他的行列式,由于有相同的两列都为零,于是
\[
	x_i \det(a)=\det(a_1,\,\cdots\!,\,b,\,\cdots\!,\,a_n).
\]
这就是Cramer法则。如果$\det(a)$可逆,则$x_i=\det(a_1$, $\cdots$, $b$, $\cdots$, $a_n)\det(a)^{-1}$.

\para 利用Cramer法则,可以谈论一个方阵的右逆。设$\varphi$是一个$n\times n$方阵,而$\psi$又是另一个$n\times n$方阵,如果$\varphi\psi=I_n$,则称$\psi$是$\varphi$的右逆,同样,也称$\varphi$是$\psi$的左逆。

设$\psi=\begin{pmatrix}\psi_{1} & \psi_{2} & \cdots & \psi_{n}\\\end{pmatrix}$,则
\[
	\varphi\psi=\begin{pmatrix}\varphi\psi_{1} & \varphi\psi_{2} & \cdots & \varphi\psi_{n}\end{pmatrix}=\begin{pmatrix}1_{1} & 1_{2} & \cdots & 1_{n}\end{pmatrix}=I_n,
\]
所以求右逆等价于求解$n$个线性方程组$\varphi\psi_i=1_i$. Cramer法则告诉我们,如果$\det(\varphi)$可逆,则$\varphi$存在右逆。

\para 设$\varphi$是一个方阵,将$\varphi$除去第$i$行第$j$列得到的矩阵记作$\Phi_{ij}$,则矩阵元$\varphi_{ij}$对应的代数余子式$\varphi_{ji}^*$定义为$(-1)^{i+j}\det(\Phi_{ij})$,定义矩阵$\varphi^*=(\varphi_{ji}^*)$,称作方阵$\varphi$的伴随矩阵。注意到上面$i$和$j$的顺序,伴随矩阵第$i$行第$j$列的矩阵元是原矩阵第$j$行第$i$列的代数余子式,即$(\varphi^*)_{ij}=\varphi_{ji}^*$. 将行列式的按列展开用伴随矩阵重写的话,就写做
\[
	\det(\varphi)=\sum_{j=1}^n\varphi_{ji}\varphi_{ji}^*=\sum_{j=1}^n(\varphi^*)_{ij}\varphi_{ji}.
\]

伴随矩阵给出了方阵左逆的存在性的一些判据,直接计算
\[
	(\varphi^*\varphi)_{ij}=\sum_{k=1}^n(\varphi^*)_{ik}\varphi_{kj}=\sum_{k=1}^n\varphi^*_{ki}\varphi_{kj},
\]
当$i=j$时,由按列展开有$(\varphi^*\varphi)_{ii}=\det(\varphi)$. 当$i\neq j$时,依然利用按列展开,将右边的和式还原成行列式,可以发现第$i$列和第$j$列都是$(\varphi_{1j}$, $\cdots$, $\varphi_{nj})$,所以当$i\neq j$时有$(\varphi^*\varphi)_{ij}=0$. 于是
\[
	\varphi^*\varphi=\det(\varphi)I_n.
\]
所以如果$\det(\varphi)$,他的左逆写作$\det(\varphi)^{-1}\varphi^*$.

\para 如果$\varphi$同时存在左逆和右逆,设$\psi$是$\varphi$的左逆,$\pi$是$\varphi$的右逆,于是
\[
\begin{aligned}
	\psi\varphi\pi&=(\psi\varphi)\pi=I_n \pi=\pi,\\
	\psi\varphi\pi&=\psi(\varphi\pi)=\psi I_n=\psi,
\end{aligned}
\]
所以$\pi=\psi$,此时$\pi=\psi$就被称为矩阵$\varphi$的逆,记作$\varphi^{-1}$. 因此,从Cramer法则与伴随矩阵,我们已经给出了方阵逆存在的一个充分条件:$\det(\varphi)$可逆。

\pro 方阵$\varphi$左(右)逆存在的充分必要条件是$\det(\varphi)$可逆。如果存在左(右)逆,则也存在右(左)逆,且左逆等于右逆。于是方阵$\varphi$逆存在的充分必要条件是$\det(\varphi)$可逆。

\proof 上面已经证明了,如果$\det(\varphi)$可逆,则$\varphi$同时存在左逆以及右逆,且左逆等于右逆。反过来,如果$\varphi$有左逆$\psi$,则$\det(\psi\varphi)=\det(I_n)=1$,由行列式乘法公式,$1=\det(\psi\varphi)=\det(\psi)\det(\varphi)$,于是$\det(\varphi)$可逆。存在右逆同理可以推出$\det(\varphi)$可逆。从$\det(\varphi)$可逆就得出了结论。\qed

方阵存在逆时,我们会称呼该方阵是可逆的。上面的命题就是在说,环$R$上的方阵$\varphi$可逆当且仅当$\det(\varphi)$可逆。举个例子,整数环$\mathbb{Z}$上的方阵$\varphi$可逆当且仅当$\det(\varphi)=\pm 1$. 而域$k$上的方阵$\varphi$可逆当且仅当$\det(\varphi)\neq 0$. 随后如果可逆,伴随矩阵给出了逆的一个计算公式。

对于交换环,由于乘法可以交换,我们无需担心左逆不能推出右逆。但一般而言,在任意的非交换环里面,左逆存在不一定能推出右逆存在,反之亦然。但是交换环上的方阵构成的环$M_n(R)$中,行列式函数将逆的存在性等价于$R$中逆的存在性,继而保证了$M_n(R)$上左逆存在可以推出右逆存在。