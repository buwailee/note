\chapter{基本构造}

\section{极限与余极限}

\para 设$\mathcal{J}$是一个小范畴(即范畴对象的全体能够构成一个集合),我们称函子$D:\mathcal{J}\to \mathcal{C}$为$\mathcal{C}$中的一个$\mathcal{J}$-图,或者简略叫做图。对于$j\in \mathcal{J}$,$D(j)$称为该图的一个顶点,而对任意的态射$\alpha:j_1\to j_2$,态射$D(\alpha)$称为该图的一条边。

\para 称$(A,\lambda)$为$J$-图的一个{锥形},如果$A$是$\mathcal{C}$的一个对象,而$\lambda$为一族态射$\lambda_{j}:A\to D(j)$使得如下交换图对所有的顶点和边都成立
\[
	\xymatrix{
		&A \ar[dl]_{\lambda_{j}}\ar[dr]^{\lambda_{j'}}&\\
		D(j)\ar[rr]^{D(\alpha)}&&D(j')
	}
\]

称呼一个锥形$(A,\lambda)$是$J$-图的一个{极限},如果对于任意的锥形$(B,\mu)$,都有唯一的态射$f:B\to A$使得如下分解$\mu_j:B\xrightarrow{f}A\xrightarrow{\lambda_j}D(j)$成立。

在实际过程中,如果极限$(A,\lambda)$中的态射族是明确的(乃至于是由对象$A$确定的),那么我们会用$A$来表示极限。并且一般将$J$-图$D$的极限写作$\varprojlim_{j\in J} D(j)$.

作为例子,模范畴的直积就是一种极限,他对应的小范畴,两个不同的元素之间没有态射。

\para 锥形和极限都有对偶的概念,在交换图中,无外乎就是将箭头完全反过来。称$(A,\lambda)$为$J$-图的一个{余锥形},如果$A$是$\mathcal{C}$的一个对象,而$\lambda$为一族态射$\lambda_{j}:D(j)\to A$使得如下交换图对所有的顶点和边都成立
\[
	\xymatrix{
		&A&\\
		D(j)\ar[rr]^{D(\alpha)} \ar[ur]^{\lambda_{j }}&&D(j')\ar[ul]_{\lambda_{j'}}
	}
\]

称呼一个余锥形$(A,\lambda)$是$J$-图的一个{余极限},如果对于任意的余锥形$(B,\mu)$,都有唯一的态射$f:A\to B$使得如下分解$\mu_j:D(j)\xrightarrow{\lambda_j}A\xrightarrow{f}B$成立。一般将$J$-图$D$的余极限写作$\varinjlim_{j\in J} D(j)$.

同样,作为例子,模范畴的直和就是一种余极限,他对应的小范畴,两个不同的元素之间没有态射。

\para 设$I$是一个偏序集,偏序关系为$\leq$,那么族$\{\{x\}\,:\, x\in I\}$构成一个(小)范畴,其中的态射定义为
\[
	\Hom_{I}\left(x,y\right)=\begin{cases}
	(x,y)&\text{, if }x\leq y\text{;}\\
	\varnothing&\text{, otherwise}.
	\end{cases}
\]

如果我们定义$x\geq y$当且仅当$y\leq x$,那么新的偏序集$(I,\geq)$就是偏序集$(I,\leq)$的对偶范畴,略去偏序符号,有时候会记作$I^{\mathrm{op}}$.

$D:I\to \mathcal{C}$是一个协变函子,就是对$i\leq j\leq k$成立$D(j\leq k)\circ D(i\leq j)=D(i\leq k)$,就是说从小到大有映射。反之,反变函子$D:I\to \mathcal{C}$对$i\leq j\leq k$成立$D(i\leq j)\circ D(j\leq k)=D(i\leq k)$,就是说从大到小有映射。

对于一个反变函子$D:I\to \mathcal{C}$,我们可以定义一个协变函子$D^{\mathrm{op}} :I^{\mathrm{op}}\to \mathcal{C}$通过$D^{\mathrm{op}}(j\geq i)=D(i\leq j)$,此时对$k\geq j\geq i$成立$D^{\mathrm{op}}(j\geq i)\circ D^{\mathrm{op}}(k\geq j)=D^{\mathrm{op}}(k\geq i)$,可见这确实是一个协变函子。

\para 称$I$是一个滤相的偏序集,即对于任意的$i$, $j\in I$都存在$k\in I$使得$k\leq i$和$k\leq j$同时成立。称$I$是一个定向的偏序集,即对于任意的$i$, $j\in I$都存在$k\in I$使得$i\leq k$和$j\leq k$同时成立。

很容易看到,如果偏序集$I$是滤相的(定向的),那么$I^{\mathrm{op}}$就是定向的(滤相的),反之亦然。

作为例子,考虑拓扑空间中所有非空开集按照包含构成的偏序集,即$U\leq V$当且仅当$U\subset V$,那么这个偏序集是定向的,但不是滤相的,因为对于任意的$U$和$V$总有$U\leq U\cup V$和$V\leq U\cup V$成立,但对于不交的$U$和$V$,并不存在非空开集同时包含于他们其中。

同样是拓扑的例子,考虑所有包含$p$的非空开集按照包含构成的偏序集,那么这个偏序集既是滤相的又是定向的。

\para 将滤相的(定向的)偏序集$I$看作一个范畴,对任意的范畴$\mathcal{C}$,如果一个$I$-图$D$,$D:I\to \mathcal{C}$是协变函子,则这称为一个$C$上的一个{逆系统}({定向系统})。逆系统一般谈论极限$\varprojlim_{i\in I} D(i)$,而定向系统一般谈论$\varinjlim_{i\in I} D(i)$,为了思考这个原因,我们考虑如下交换图
\[
	\xymatrix{
		&A \ar[dl]_{\lambda_{i}}\ar[dr]^{\lambda_{j}}\ar[dd]^{\lambda_{k}}&\\
		D(i)&&D(j)\\
		&D(k)\ar[ul]^{D(k\leq i)}\ar[ur]_{D(k\leq j)}&
	}
	\quad
% \xymatrix{
% 	&A \ar[dl]_{\lambda_{i}}\ar[dr]^{\lambda_{j}}\ar[dd]^{\lambda_{k}}&\\
% 	D(i)\ar[dr]_{D(i\leq k)}&&D(j)\ar[dl]^{D(j\leq k)}\\
% 	&D(k)&
% }
	\xymatrix{
		&A &\\
		\ar[ur]^{\mu_{i}}D(i)\ar[dr]_{D(i\leq k)}&&D(j)\ar[ul]_{\mu_{j}}\ar[dl]^{D(j\leq k)}\\
		&D(k)\ar[uu]_{\mu_{k}}&
	}
\]
左边是对逆系统考虑锥形,右边是对定向系统考虑余锥形。从左边来看,如果$i\leq j\leq k$成立,则$D(k)$构成了一个锥形的顶点,当我们考虑极限的时候,这时候极限就应该表现得像那些“极小”的元素$D(k)$一样。而且如果$A$是极限,那么箭头$A\to D(k)$也构成了唯一分解。类似地考虑右边的图,那些“极大”的元素$D(k)$构成了余锥形的顶点,如果$A$是余极限,那么箭头$D(k)\to A$也构成了唯一分解。

\para 考虑$I$是滤相的(定向的),而$D$是协变函子,那么$I$-图$D$是逆系统(定向系统)。考虑$I$是滤相的(定向的),而$D$是反变函子,那么$I^{\mathrm{op}}$-图$D^{\mathrm{op}}$是定向系统(逆系统)。

\para 设$I$是一个偏序集,如果$i\in I$有,$i>j$对所有$j\in I$不成立,或者说,与可以比较的元素$j\in I$都有$i\leq j$,则称$i$是$I$的一个极小元。如果$i\leq j$对所有$j\in I$都成立,则称$i\in I$是$I$的最小元。

在滤相的偏序集中,极小元和最小元等价。最小元是极小的,这是显然的。反之,因为如果存在极小元$i$,那么任取一个$j\in I$都存在一个$k\in I$使得$k\leq i$和$k\leq j$都成立,然而$i$是极小元,所以$i=k$,所以$i\leq j$.

同理我们可以定义极大元与最大元,在定向的偏序集中,此二者等价。

更广义地,对于一个偏序集$I$的子集$J$,如果对于任意的$i\in I$,都存在$j\in J$使得$j\leq i$,则称$J$和$I$是共尾的。显然,滤相的偏序集中的极小元构成的单点集就和原来的偏序集共尾。

\pro 设$I$-图$D$是一个逆系统,如果$I$存在一个与其共尾的子集$J$,则$J$-图$D$是一个逆系统,且$\varprojlim_{i\in I} D(i)=\varprojlim_{i\in J} D(i)$. 对偶地,设$I$-图$D$是一个定向系统,如果$I^{\mathrm{op}} $存在一个与其共尾的子集$J^{\mathrm{op}} $,则$J$-图$D$是一个定向系统,且$\varinjlim_{i\in I} D(i)=\varinjlim_{i\in J} D(i)$. \rule{2mm}{2mm}

所以对于逆系统,如果存在极小元,那么极小元就是它的极限,对偶地,对于定向系统,如果存在极大元,那么极大元就是它的对偶极限。这符合我们上面的直观。

\para 一个拓扑空间$X$能被看成一个范畴,对象取作他的所有开集,而态射取作
\[
	\Hom_{X}(U,V)=\begin{cases}
	\bigl\{i^U_V:U\hookrightarrow V\bigr\}&\text{, if }U\subset V\text{;}\\
	\varnothing&\text{, otherwise}.
	\end{cases}
\]

考虑拓扑空间$X$中一个非空开集族$\mathfrak{B}$,对于任意的两个$U$, $V\in \mathfrak{B}$,都存在一个$W\in \mathfrak{B}$使得$U\cup V\subset W$成立,这样的$\mathfrak{B}$按照包含构成了定向的偏序集。现在定义函子$i:\mathfrak{B}\to X$通过$i(U)=U$以及$i(U\leq V)=i^U_V:U\hookrightarrow V$,它显然成立复合$i(U\leq W)=i(V\leq W)\circ i(U\leq V)$,所以这个$\mathfrak{B}$-图$i$是一个定向系统。可以看到它的余极限$\varinjlim_{U\in \mathfrak{B}} U$实际上就是$\bigcup_{U\in \mathfrak{B}} U$,而态射族就是$i_U:U\hookrightarrow \bigcup_{U\in I}$.

同样考虑拓扑空间$X$中一个包含一个点$p\in X$的所有非空开集构成的族$\mathfrak{B}$,它按照包含构成一个滤相的偏序集。同样可以定义函子$i:\mathfrak{B}\to X$通过$i(U)=U$以及$i(U\leq V)=i^U_V:U\hookrightarrow V$,同样显然成立复合$i(U\leq W)=i(V\leq W)\circ i(U\leq V)$,所以这个$\mathfrak{B}$-图$i$是一个逆系统。

但它的极限$\varprojlim_{U\in \mathfrak{B}} U$就不一定存在,因为类比于上一个例子,它的极限应该类似于所有$\mathfrak{B}$中元素的交,但这不一定是一个开集。

\para 设我们有一个$X$上的$\mathcal{K}$-预层$\calf$,他是$X\to \mathcal{K}$的一个反变函子。赋予$X$一个偏序,此时函数$U\leq V$即$U\hookrightarrow V$. 那么$X$的那些包含$p\in X$的非空开集构成的子集族$\mathfrak{B}$继承了偏序结构,也是一个范畴,而预层$\calf$限制在$\mathfrak{B}$给出了反变函子$\mathfrak{B}\to \mathcal{K}$.

由于$\mathfrak{B}$是滤相的偏序集,而$\calf$是反变函子,因此$\mathfrak{B}^{\mathrm{op}}$-图$\calf^{\mathrm{op}}$是定向系统,进而我们会考虑余极限
\[
	\varinjlim_{U\in \mathfrak{B}^{\mathrm{op}}} \calf^{\mathrm{op}}(U)=\varinjlim_{U\in \mathfrak{B}^{\mathrm{op}}} \calf(U),
\]
如果这个余极限存在,那么就称为预层$\calf$在点$p$处的纤维,记作$\calf_p$.

\section{直和与直积}

直积和直和都属于用一些小的模来构造出大的模的手段。

\para 设$\{M_i\,:\, i\in I\}$是一族$R$-模,则存在一个$R$-模$M$以及一族同态$\pi_i:M\to M_i$使得,如果存在另一个模$N$和一族同态$\rho_i:N\to M_i$,那么就唯一存在同态$\rho:N\to M$使得分解$\rho_i:N\xrightarrow{\rho} M \xrightarrow{\pi_i} M_i$成立对任意的$i\in I$都成立。这样的一个模$M$被称为$\{M_i\,:\, i\in I\}$的直积,而$\pi_i$被称为典范投影,通常将$M$记作$\prod_{i\in I}M_i$. 直积对应于范畴里product的概念。

存在性的证明是简单的,考虑$\{M_i\,:\, i\in I\}$作为集合的直积,很容易检验它有一个$R$-模结构,且投影是同态。更详细的检查这里就略去了。

其他满足这个性质的模及同态,都可以唯一分解到这个典型的模和同态(这里是直积和投影)的性质,被称为泛性质。

\para 设$\{M_i\,:\, i\in I\}$是一族$R$-模,则存在一个$R$-模$M$以及一族同态$\pi_i:M_i\to M$使得,如果存在另一个模$N$和一族同态$\rho_i:M_i\to N$,那么就唯一存在同态$\rho:M\to N$使得分解$\rho_i:M_i\xrightarrow{\pi_i} M \xrightarrow{\rho} N$成立对任意的$i\in I$都成立。这样的一个模$M$被称为$\{M_i\,:\, i\in I\}$的直和,而$\pi_i$被称为典范内射,通常将$M$记作$\bigoplus_{i\in I}M_i$. 直和对应于范畴里coproduct的概念。

直和存在性的构造并不如同直积那么容易。我们考虑$\{M_i\,:\, i\in I\}$的形式和$M$,即
\[
	M=\left\{\sum_{i\in I} a_i m_i\,:\, m_i\in M_i\right\},
\]
其中的求和只有有限项,或者说系数$a_i$只有有限项非零。可以看到这是一个模。定义$\pi_i:M_i\to M$为$\pi_i(m_i)=m_i$,可以检验这是一个模同态。

现在任取$N$和一族同态$\rho_i:M_i\to N$,我们将其扩展为$\rho_i:M\to N$通过补充定义$\rho_i(m)=0$,如果$m\notin M_i$. 随后我们定义$\rho = \sum_{i\in I} \rho_i$,需要检验这个求和对任意的$m\in M$是有限的。为此,任取$m\in M$,由于他可以分解成$m=\sum_i a_i m_i$,所以
\[
	\rho(m)=\sum_i a_i \rho_i(m_i),
\]
求和是有限求和因为$a_i$只有有限项非零。有了构造,泛性质的检验就是直接的了。

注意到交换群为$\zz$-模,所以我们也证明了交换群有直和与直积。

\para 如果满足泛性质,则他在同构意义上唯一。

以直积为例,如果$(M,\pi_i)$和$(M',\pi'_i)$都满足泛性质,则利用$M$的泛性质,我们有唯一的同态$\rho:M'\to M$,利用$M'$的泛性质,我们有唯一的同态$\rho':M\to M'$. 然后$\rho \rho':M\to M$和$\id_M$是同时满足泛性质的分解,由唯一性,$\rho\rho'=\id_M$. 同理,$\rho'\rho=\id_{M'}$. 所以他们同构。

\para 由构造,有限直和与有限直积在模范畴里面是等价的。考虑两个$R$-模$M_1$和$M_2$,那么$M_1$可以看成$M_1\oplus M_2$的子模,通过$m\to (m,0)$. 反过来,设$M_1$和$M_2$是一个模$M$的两个子模,则我们有如下短正合列
\[
	0\to M_1\cap M_2 \xrightarrow{\mu} M_1\oplus M_2\xrightarrow{\nu} M_1+M_2\to 0,
\]
其中$\mu:m\mapsto (m,m)$,$\nu:(m_1,m_2)\mapsto m_1-m_2$. 所以在$M_1\cap M_2=\{0\}$的时候,我们有同构$\nu: M_1\oplus M_2\to M_1+M_2$. 如果$M_1+M_2=M$,那么就是有同构$M\cong M_1\oplus M_2$.

\theo 在左$R$-模范畴,任意的极限与余极限存在。 \notprove

\para 设$I$是一个指标集,我们定义$R^I=\bigoplus_{i\in I}R$,他被称为自由模。如果$I$是有限集,$n=|I|$,则我们通常将自由模$R^I$记作$R^n$. 记$1_i$为第$i$个指标的$R$中的$1$.

\para 设$M$是一个$R$-模,$S$是他的一个子集,称呼$M$被$S$生成,就是说$M$中的元素可以写成$S$中元素的有限线性组合。即任取$m\in M$,存在一个系数集$\{a_s\in R\,:\, s\in S\}$,其中只有有限个系数非零,使得
\[
	m=\sum_{s\in S}a_s s.
\]
称呼一个$R$-模$M$是有限生成的,就是说存在一个$M$的有限子集生成他。

显然$R^I$由$\{1_i\,:\,i\in I\}$生成,其中$1_i$代表的是第$i$个指标的$R$中的$1$,所以$R^n$是有限生成模。

\para 域上的自由模是矢量空间。

\para 如果$M$和$N$都是双边$R$-模(比如我们一直讨论的交换环的情况),则


\section{自由对象}

简单来说,自由对象是遗忘函子的左伴随函子。所以这里先给出伴随函子的定义。

\para 设$\mathcal{C}$和$\mathcal{D}$是两个范畴,而$f:\mathcal{C}\to \mathcal{D}$是一个函子。对于给定的$Y\in\mathcal{D}$,反变函子$T\mapsto \Hom_{\mathcal{D}}(f(T),Y)$属于$\hat{\mathcal{C}}$. 如果他是可表函子,被$X$所表示,于是就有函子同构$\alpha:h_X\to \Hom_{\mathcal{D}}(f(\star),Y)$:给出$S$, $T\in \mathcal{C}$以及同态$\psi:S\to T$,有交换图
\[
\begin{xy}
	\xymatrix{
		h_X(T)\ar[rr]^-{\alpha_T} \ar[d]_{\psi^*}&&\Hom_{\mathcal{D}}(f(T),Y) \ar[d]^{f(\psi)^*}\\
		h_X(S)\ar[rr]^-{\alpha_S}&&\Hom_{\mathcal{D}}(f(S),Y)
	}
\end{xy}
\]
任取$u:T\to X$,有$\alpha_S(u\psi)=\alpha_T(u)f(\psi)$. 特别地,当$T=X$时,取$u=\id_X$,记$\alpha_X(\id_X)=\sigma_Y:f(X)\to Y$,$\alpha_S(\psi)=\sigma_Y f(\psi)$,其中$\psi:S\to X$.

\para 设$f:\mathcal{C}\to \mathcal{D}$和$g:\mathcal{D}\to \mathcal{C}$是一对函子,如果同构
\[
	\alpha_T(Y):\Hom_\mathcal{C}(T,g(Y))\to \Hom_\mathcal{D}(f(T),Y)
\]
不管对$T$还是对$Y$都是自然同构,则$(f,g)$被称为一对伴随函子。$f$被称为$g$的左伴随函子,$g$被称为$f$的右伴随函子。

固定$Y$,对$T$的那个自然同构说明$g(Y)$是反变函子$\Hom_\mathcal{D}(f(\star),Y)$的表示对象。反过来其实也对,如果对每一个$Y$,反变函子$T\mapsto \Hom_{\mathcal{D}}(f(T),Y)$都是可表的,则我们可以构造出他的一个左伴随函子。

设$g(Y)$是$\Hom_\mathcal{D}(f(\star),Y)$的表示对象。改变$Y$,即给出一个态射$v:Y\to Y'$,$g(v)=\alpha^{-1}_{g(Y)}\left(v\sigma_Y\right):g(Y)\to g(Y')$是一个态射。于是$g:Y\mapsto g(Y)$,以及$g:v\mapsto \alpha^{-1}_{g(Y)}\left(v\sigma_Y\right)$就构成了一个函子$g:\mathcal{D}\to \mathcal{C}$. 再记$\beta_T(Y)=\alpha_{g(Y)}(T)$,对每一个$T$,$\beta_T$将给出了函子同构
\[
	\beta_T:\Hom_\mathcal{C}(T,g(\star))\to \Hom_\mathcal{D}(f(T),\star).
\]
自然变换的检验是直接的。

\para 以忘却函子为例,如果$g$是忘却函子,再具体一点,如果他是将群$G$变成集合的那个忘却函子。再设$T=\{e\}$是一个单点集,如果他存在左伴随函子$f$,则有同构
\[
	\Hom_{\text{Group}}(f(\{e\}),G)\cong \Hom_{\text{Set}}(\{e\},g(G))
\]

\section{多项式环}

多项式环的重要性无需赘述。

\para 设$R$是一个环,而$a=(a_0,a_1,\,\cdots\!,\,a_n,\cdots)$是一列$R$中的元素,记只有有限个元素非零的$a$构成的集合为$R[x]$. 在$R[x]$上,我们可以定义加法
\[
	a+b=(a_0+b_0,a_1+b_1,\cdots),
\]
由于$a+b$也是只有有限个元素非零,所以$a+b \in R[x]$. 对于加法$0=(0,0,\cdots)$显然是零元,而$a$的逆元$-a=(-a_0,-a_1,\cdots)$.

然后是乘法,定义
\[
	(ab)_i=\sum_{k+j=i}a_kb_j,
\]
可以看出$ab$也只有有限个元素非零,所以$ab\in R[x]$.

分配律的检验是直接的,
\[
	((a+b)c)_i=\sum_{k+j=i}(a+b)_kc_j=\sum_{k+j=i}a_kc_j+\sum_{k+j=i}b_kc_j=(ac)_i+(bc)_i,
\]
所有的一切都可以直接检验。因此$R[x]$确实是一个环,称为多项式环。

一般而言,我们将$R[x]$中的元素$f=(a_0,a_1,\cdots)$写成
\[
	f=\sum_{n}a_n x^n
\]
的形式,这就是我们熟悉的多项式,一切的运算都和我们熟悉的那样。$x$称之为不定元,不定元只是一个符号,在不发生歧义的情况下可以任意选取。

\para 定义多元多项式环$R[x_1,\,\cdots\!,\,x_n]$为$R[x_1,\,\cdots\!,\,x_{n-1}][x_n]$,这是一个递归定义。

\para 一个多项式$f=(a_0,a_1,\cdots)$中的最高项即使得$a_n$不为零的最大的$n$,这个$n$记作$\deg(f)$,称为多项式$f$的幂次。最高幂次的系数我们称之为最高次系数,或者首项系数。其余的项目通常我们会指出具体的指标,比如说$k$-次项,就是指$a_kx^k$,而$k$-此项系数就是指$a_k$. 如果最高次系数为$1$,这个多项式称为首一多项式。

对于首一多项式而言$f$,$\deg(fg)=\deg(f)+\deg(g)$. 但是一般的多项式并不如此,比如在$\zz/4\zz$上的多项式$2x$,有$(2x)(2x)=0$.

对于$\deg(f+g)$,我们有估计$\deg(f+g)\leq \max\{\deg(f),\deg(g)\}$,不等号是可以取严格的,同样比如在$\zz/4\zz$上的多项式$2x$,有$2x+2x=0$.

\theo 多项式除法算法:设$f$, $g\in R[x]$,而且$g$的首项系数可逆,则存在唯一的多项式$p$, $r\in R[x]$使得$f=pg+r$,且$\deg(r)< \deg(f)$.

\proof
	可以假设$g$是首一多项式,因为如果设$g$的首项系数为$a$,则$g/a$是一个首一多项式,如果命题对首一多项式成立,即存在$q$, $r\in R[x]$使得$f=q(g/a)+r$,则$f=(q/a)g+r=pg+r$,其中$p=q/a$和$r$都是多项式。这样就得到了我们的命题。

	首先假设存在,证明唯一性。设$f=p'g+r'$以及$f=pg+r$成立,则$(p'-p)g=r'-r$,由于$g$是首一的,且$p'-p$非零,所以$\deg((p'-p)g)\geq \deg(g)$,但是$\deg(r'-r)< \deg(g)$,这就造成了矛盾。下面证明存在性。

	设$\deg(f)=n$, $\deg(g)=m$,如果$n<m$,则取$p=0$, $r=f$. 考虑$n\leq m$的情况,设$f$的首项系数为$a_0$,考虑多项式$f_1=f-a_0x^{n-m}g$,由于$f$和$a_0x^{n-m}g$的最高次项相同,所以$\deg(f_1)<\deg(f)$,如果$\deg(f_1)<\deg(g)$,那么$f=a_0x^{n-m}g+f_1$就给出了分解。

	否则继续对$f_1$进行这样的操作,得到$f_2=f_1-a_1x^{\deg(f_1)-m}g$,再比较$\deg(f_2)$与$\deg(g)$. 不断如是进行下去,由于$f$的幂次有限,而每次操作,幂次都至少减一,所以该过程在进行至多$n-m+1$次后就会停止。设该过程在第$k$次后停止,则我们就得到了
	\[
	f_{k}=f-a_0x^{n-m}g-a_1x^{\deg(f_1)-m}g-\cdots-a_{k-1}x^{\deg(f_{k-1})-m}g
	\]
	使得$\deg(f_k)< \deg(g)$. 即$f=f_0$,则我们就得到了分解
	\[
	f=f_k+a_0x^{n-m}g+a_1x^{\deg(f_1)-m}g+\cdots+a_{k-1}x^{\deg(f_{k-1})-m}g=\left(\sum_{i=0}^{k-1}a_{i}x^{\deg(f_{i})-m}\right)g+f_k.
	\]
	因此$p=\sum_{i=0}^{k-1}a_{i}x^{\deg(f_{i})-m}$以及$r=f_k$就是我们需要的多项式。
\qed

存在性的证明就是整个算法,从算法来看,这个命题即使是对非交换环上的多项式环也是成立的。

\section{矩阵与行列式}

\para 记$1_i=(0,\,\cdots\!,\,1,\,\cdots\!,\,0)\in R^m$,即只有第$i$个位置是一,其他都是零的那个元素。考虑一个模同态$\varphi:R^m\to R^n$,由于
\[
	\varphi\left(\sum_{i} r_i1_i\right)=\sum_{i} r_i\varphi\left(1_i\right),
\]
所以它由所有$\varphi\left(1_i\right)$描述,由于他在$R^n$中,将其写作
\[
	\varphi\left(1_i\right)=\sum_{j}\varphi_{ji}1_j.
\]
于是$\{\varphi_{ij}\,:\, 1\leq i \leq p,\,\,1\leq j \leq q\}$这$p\times q$个数构成的整体称为$\varphi$的矩阵,记作$(\varphi)$,他完全描述了整个同态$\varphi$.

设$\psi:R^n\to R^l$的矩阵是$(\psi)_{ij}$,我们下面计算$\psi\circ\varphi$的矩阵:
\[
	\psi\circ\varphi(1_i)=\psi\left(\sum_{j=1}^n\varphi_{ji}1_j\right)=\sum_{j=1}^n\varphi_{ji}\psi(1_j)=\sum_{j=1}^n\sum_{k=1}^l\psi_{kj}\varphi_{ji}1_k,
\]
所以$\psi\circ\varphi$的矩阵$(\psi\circ\varphi)$为
\[
	(\psi\circ\varphi)_{ki}=\sum_{j=1}^n\psi_{kj}\varphi_{ji},
\]
这被称为矩阵乘法法则。

\para 通常会使用一张表来表示矩阵
\[
(\varphi)=
\begin{pmatrix}
	\varphi_{11} & \varphi_{12} & \cdots & \varphi_{1m}\\
	\varphi_{21} & \varphi_{22} & \cdots & \varphi_{2m}\\
	\vdots & \vdots & \ddots & \vdots \\
	\varphi_{n1} & \varphi_{n2} & \cdots & \varphi_{nm}\\
\end{pmatrix}
\]
可以看到$\varphi(1_i)$关于$\{1_j\in R^n\}$中展开的系数出现在$(\varphi)$的第$i$列。

下面将单独研究矩阵,而不将其看成某个自由模的矩阵。之所以这样,因为有时候我们处理的矩阵并不能很好地表为自由模之间的同态。记号上,表示矩阵时可加或不加括号,比如$\varphi$或者$(\varphi)$. 相应的,表示矩阵元时写作$\varphi_{ij}$或者$(\varphi)_{ij}$. 如果需要以矩阵元来表述矩阵,写作$(\varphi_{ij})$. 

按照习惯,我们下面依旧假设$R$是一个交换环,虽然在非交换环上谈论矩阵也是可能的。称呼一个矩阵是$R$上的矩阵,如果他的矩阵元都属于$R$.

所有环$R$的$m\times n$矩阵构成一个$R$-模,实际上,任取$r\in R$以及矩阵$\varphi$,定义矩阵$r\varphi= (r\varphi_{ij})$. 另取矩阵$\psi$,定义矩阵加法$\psi+\varphi=(\psi_{ij}+\varphi_{ij})$.

矩阵乘法就按照自由模里面推出的法则来定义:
\begin{equation}
\begin{pmatrix}
	\psi_{11} & \psi_{12} & \cdots & \psi_{1n}\\
	\psi_{21} & \psi_{22} & \cdots & \psi_{2n}\\
	\vdots & \vdots & \ddots & \vdots \\
	\psi_{l1} & \psi_{l2} & \cdots & \psi_{ln}
\end{pmatrix}
\begin{pmatrix}
	\varphi_{11} & \varphi_{12} & \cdots & \varphi_{1m}\\
	\varphi_{21} & \varphi_{22} & \cdots & \varphi_{2m}\\
	\vdots & \vdots & \ddots & \vdots \\
	\varphi_{n1} & \varphi_{n2} & \cdots & \varphi_{nm}\\
\end{pmatrix}
=
\begin{pmatrix}
	\sum_{i=1}^n \psi_{1i}\varphi_{i1} & \sum_{i=1}^n \psi_{1i}\varphi_{i2} & \cdots & \sum_{i=1}^n \psi_{1i}\varphi_{im}\\
	\sum_{i=1}^n \psi_{2i}\varphi_{i1} & \sum_{i=1}^n \psi_{2i}\varphi_{i2}& \cdots & \sum_{i=1}^n \psi_{2i}\varphi_{im}\\
	\vdots & \vdots & \ddots & \vdots \\
	\sum_{i=1}^n \psi_{li}\varphi_{i1} & \sum_{i=1}^n \psi_{li}\varphi_{i2}& \cdots & \sum_{i=1}^n \psi_{li}\varphi_{im}
\end{pmatrix}
\end{equation}
即$(\psi\varphi)_{ij}$由第一个矩阵的第$i$行和第二个矩阵的第$j$列逐个相乘然后相加而得,矩阵乘法把$l\times n$和$n \times m$的矩阵变成了$l \times m$的矩阵。矩阵乘法满足结合律是直接的计算。

尤其重要的情况是$m=1$的时候,此时,$n\times 1$的矩阵称为一个列,一个列我们直接记成一个$n$-元组$\varphi=(\varphi_1$, $\cdots$, $\varphi_n)$. 其中,记$1_i=(0$, $\cdots$, $1$, $\cdots$, $0)$,即只有第$i$个位置为$1$,其他位置都为零的那个列。所有的列都可以看成$\{1_i\,:\, 1\leq i \leq n\}$的线性组合,
\[
	(\varphi_1,\,\cdots\!,\,\varphi_n)=\sum_{i=1}^n\varphi_i1_i.
\]
特别地,记矩阵$I_n=\begin{pmatrix}1_{1} & 1_{2} & \cdots & 1_{n}\end{pmatrix}$,称之为单位矩阵。由矩阵乘法,对任意的$m\times n$矩阵$\varphi$有$\varphi I_n=\varphi$以及$I_m\varphi =\varphi$成立。单位矩阵具体写出来就是
\[
	I_n=
		\begin{pmatrix}
			1 & & &\\
			& 1 & &\\
			& & \ddots &\\
			& & & 1
		\end{pmatrix}
	=(\delta_{ij}),
\]
其中$\delta_{ij}$被称为Kronecker符号,或者叫Kronecker delta,定义为
\[
	\delta_{{ij}}=
	\begin{cases}
	1,&\text{if} i=j,\\
	0,&\text{if} i\neq j.
	\end{cases}
\]

设$\varphi_{i}=(\varphi_{1i}$, $\cdots$, $\varphi_{ni})$,则一个矩阵可以看成一些列的并列,
\[
	\begin{pmatrix}
	\varphi_{11} & \varphi_{12} & \cdots & \varphi_{1m}\\
	\varphi_{21} & \varphi_{22} & \cdots & \varphi_{2m}\\
	\vdots & \vdots & \ddots & \vdots \\
	\varphi_{n1} & \varphi_{n2} & \cdots & \varphi_{nm}\\
	\end{pmatrix}
	=
	\begin{pmatrix}
	\varphi_{1} & \varphi_{2} & \cdots & \varphi_{m}
	\end{pmatrix},
\]
其中$\varphi_i=(\varphi_{1i}$, $\cdots$, $\varphi_{ni})$. 

关于列的矩阵乘法为
\begin{equation}
\begin{pmatrix}
	\psi_{11} & \psi_{12} & \cdots & \psi_{1n}\\
	\psi_{21} & \psi_{22} & \cdots & \psi_{2n}\\
	\vdots & \vdots & \ddots & \vdots \\
	\psi_{l1} & \psi_{l2} & \cdots & \psi_{ln}
\end{pmatrix}
\begin{pmatrix}
	\varphi_{1} \\
	\varphi_{2}  \\
	\vdots \\
	\varphi_{n} \\
\end{pmatrix}
=
\begin{pmatrix}
	\sum_{i=1}^n \psi_{1i}\varphi_{i}\\
	\sum_{i=1}^n \psi_{2i}\varphi_{i} \\
	\vdots \\
	\sum_{i=1}^n \psi_{li}\varphi_{i}
\end{pmatrix}
\end{equation}
通常将其右侧写作$\psi\varphi$. 那么一般的矩阵乘法则写作:
\[
	\psi
	\begin{pmatrix}
	\varphi_{1} & \varphi_{2} & \cdots & \varphi_{m}
	\end{pmatrix}
	=
	\begin{pmatrix}
	\psi\varphi_{1} & \psi\varphi_{2} & \cdots & \psi\varphi_{m}
	\end{pmatrix}.
\]

\para 下面要讨论行列式函数,行列式函数是对方阵定义的函数,所谓方阵就是$n\times n$矩阵。将所有矩阵元属于环$R$的$n\times n$矩阵构成的集合记作$M_n(R)$. $M_n(R)$通过矩阵加法和乘法构成一个环,但不是交换环,单位元是$I_n$. 

对$M(R,1)$可以通过$(r)\mapsto r$定义一个函数$f_1:M_n(R)\to R$. 然后对$M_n(R)$里面的矩阵$\varphi$,记$\varphi$除去第$i$行第$j$列得到的$(n-1)\times (n-1)$矩阵为$\Phi_{ij}$,定义
\[
	f_n(\varphi)=\sum_{i=1}^n (-1)^{i+j}\varphi_{ij}f_{n-1}(\Phi_{ij}).
\]
这个展开被称为按第$j$列展开,这个归纳定义的目前的缺陷在于,似乎按照不同的两列展开会有不同的结果。其实并不会这样,首先考虑$n=2$的情况,设$\varphi=\begin{pmatrix}a&b\\c&d\end{pmatrix}$,按第一列展开会得到$ad-bc$,按第二列展开也会得到$ad-bc$. 然后假设当$i<n$的时候,$f_i$都已经定义良好了,即并不依赖于按列展开的选取。那么考虑$f_n(\varphi)$的按第$i$列和按第$j$列的两个展开,可以假设$i<j$。对按照第$i$列展开的式子,对所有$f_{n-1}(\Phi_{kl})$按第$(j-1)$列展开。对按照第$j$列展开的式子,对所有$f_{n-1}(\Phi_{kl})$按第$i$列展开。这两个展开将会得到相同的结果。所以$f_n(\varphi)$的定义并不依赖于对列的展开。

这样,对每一个$n$,我们归纳定义了$f_n:M_n(R)\to R$,这个函数对每一个$n$统一记作$\det:M_n(R)\to R$,称之为行列式。设$n\times n$矩阵为$\varphi=\begin{pmatrix}\varphi_{1} & \varphi_{2} & \cdots & \varphi_{n}\end{pmatrix}$,他的行列式记作$\det(\varphi)=\det(\varphi_1$, $\cdots$, $\varphi_n)$,其中每一个$\varphi_i$都是一个$n\times 1$的列。如果矩阵要把他所有的元素写出来,则行列式会记作
\[
\det(\psi)=\begin{vmatrix}
	\psi_{11} & \psi_{12} & \cdots & \psi_{1n}\\
	\psi_{21} & \psi_{22} & \cdots & \psi_{2n}\\
	\vdots & \vdots & \ddots & \vdots \\
	\psi_{l1} & \psi_{l2} & \cdots & \psi_{ln}
\end{vmatrix}.
\]

\pro 行列式具有以下性质:

1. 行列式对每一个列线性。

2. 行列式中如果有两列相同,则行列式为零。

3. $\det(I_n)=1$对每一个$n$都成立。

\proof 
	第一点由按列展开的定义可得。对于第二点,我们采用对$n$归纳,当$n=2$的时候,
	\[
		\begin{vmatrix}a&a\\b&b\end{vmatrix}=ab-ab=0.
	\]
	当$n>3$的时候,选不同于该两列的第三列展开,就可以归结到$n-1$的情况,随后由归纳法就得到了结论。

	第三点是直接的计算,对第$n$列展开,我们就得到了
	\[
		\det(I_n)=\det(1_1,\,\cdots\!,\,1_{n-1})=\det(I_{n-1}),
	\]
	最后由$\det(I_1)=\det(1_1)=1$就得到了结论。
\qed

应用第一、第二点,我们有
\[
	0=\det(\cdots\!,\psi+\varphi,\,\cdots\!,\,\psi+\varphi,\cdots)=\det(\cdots\!,\psi,\,\cdots\!,\,\varphi,\cdots)+\det(\cdots\!,\varphi,\,\cdots\!,\,\psi,\cdots),
\]
所以,调换行列式的两列,行列式的值变成相反数。

满足上列性质第一、第二点的函数$M_n(R)\to R$被称为反对称多线性函数。反对称的意思就是调换任意两列将得到相反的结果。

\para 给定一个给定矩阵$\psi=\begin{pmatrix}\psi_{1} & \psi_{2} & \cdots & \psi_{n}\end{pmatrix}$,由于每一个$\psi_i$都可以写成$\psi_i=\sum_{j=1}^n\psi_{ji}1_j$,所以任取一个反对称多线性函数$F$,则
\[
	F(\psi_1,\,\cdots\!,\,\psi_n)=\sum_{j_1=1}^n\cdots \sum_{j_n=1}^n \psi_{j_11}\cdots \psi_{j_n n} F(1_{j_1},\,\cdots\!,\,1_{j_n}),
\]
所以对反对称多线性函数的计算,只要计算所有形如$F(1_{i_1},\,\cdots\!,\,1_{i_n})$的式子即可。同时,上面的式子也说明了,多线性函数是矩阵元的多项式函数。

\pro 设$F:M_n(R)\to R$是一个反对称多线性函数,则$F=F(I_n)D$. 作为推论,行列式的三点性质将完全决定行列式函数。

\proof 
	从上面看到的,只要证明\[F(1_{i_1},\,\cdots\!,\,1_{i_n})=F(I_n)D(1_{i_1},\,\cdots\!,\,1_{i_n}).\]如果$\{i_1$, $\cdots$, $i_n\}$中有重复指标,两边都为零。如果没有重复指标,采用冒泡排序\footnote{先比较相邻的元素,如果第一个比第二个大,就交换他们两个。对每一对相邻元素作同样的工作,从开始第一对到结尾的最后一对。在这一点,最后的元素应该会是最大的数。针对所有的元素重复以上的步骤,除了最后一个。持续每次对越来越少的元素重复上面的步骤,直到没有任何一对数字需要比较。},可以通过不断两两调换两列,将$(i_1$, $\cdots$, $i_n)$变成$(1$, $\cdots$, $n)$,对等式两边进行同样的调换两列的操作,等式依旧成立,所以最后只要检验$F(I_n)=F(I_n)D(I_n)$,而这来自于$D(I_n)=1$.
\qed

\pro 行列式乘法公式:$\det(\varphi \psi)=\det(\varphi)\det(\psi)$.

\proof 
	由矩阵乘法,
	\[
	\varphi
	\begin{pmatrix}
	\psi_{1} & \psi_{2} & \cdots & \psi_{n}\\
	\end{pmatrix}
	=
	\begin{pmatrix}
	\varphi\psi_{1} & \varphi\psi_{2} & \cdots & \varphi\psi_{n}\\
	\end{pmatrix}
	\]
	因此$\det (\varphi \psi)=\det (\varphi\psi_{1}$, $\cdots$, $\varphi\psi_{n})$.

	将$\det(\varphi\psi_{1}$, $\cdots$, $\varphi\psi_{n})$记作$F(\psi_{1}$, $\cdots$, $\psi_{n})$,可以看到这是一个反对称的多重线性映射,所以
	\[
	F(\psi_{1},\,\cdots\!,\,\psi_{n})=F(I_n)\det (\psi_{1},\,\cdots\!,\,\psi_{n})
	\]
	也就是说
	\[
	\det (\varphi\psi_{1},\,\cdots\!,\,\varphi\psi_{n})=\det(\varphi 1_1,\,\cdots\!,\,\varphi 1_n)\det(\psi_{1},\,\cdots\!,\,\psi_{n}).
	\]

	最后,只需要计算$\varphi 1_i=\begin{pmatrix}\varphi_{1} & \varphi_{2} & \cdots & \varphi_{n}\end{pmatrix}1_i$,由矩阵乘法$\varphi 1_i=\varphi_i$,所以
	\[
	\det(\varphi\psi)=\det(\varphi\psi_{1},\,\cdots\!,\,\varphi\psi_{n})=\det(\varphi_1,\,\cdots\!,\,\varphi_n)\det(\psi_{1},\,\cdots\!,\,\psi_{n})=\det(\varphi)\det(\psi).
	\]
\qed

\para 矩阵的一个起源是线性方程组,这里暂时只考虑$n$个方程$n$个未知元的情况。所谓的线性方程组即如下的方程
\[
	\begin{cases}
	a_{11}x_1+\cdots+a_{1n}x_n=b_1,\\
	\qquad\qquad\vdots\\
	a_{n1}x_1+\cdots+a_{nn}x_n=b_n,
	\end{cases}
\]
其中$a_{ij}$, $b_i\in R$是已知的,而解线性方程组就是找到$x_i\in R$满足上述方程。

采用矩阵和矩阵乘法,上述的线性方程组可以写作$ax=b$,其中$a=(a_{ij})=\begin{pmatrix}a_{1} & a_{2} & \cdots & a_{n}\end{pmatrix}$, $x=(x_1$, $\cdots$, $x_n)$和$b=(b_1$, $\cdots$, $b_n)$. 下面将利用行列式给出一种可能的求解方式,其实也就是著名的Cramer法则。陈述如下:

将矩阵的第$i$列换作$b$,计算这个新矩阵的行列式,有
\[
	\det(a_1,\,\cdots\!,\,b,\,\cdots\!,\,a_n)=\det\left(a_1,\,\cdots\!,\,\sum_{j=1}^na_j x_j,\,\cdots\!,\,a_n\right)=\sum_{j=1}^n x_j \det(a_1,\,\cdots\!,\,a_j,\,\cdots\!,\,a_n),
\]
在求和中,除了$j=i$可能行列式$\det(a_1$, $\cdots$, $a_j$, $\cdots$, $a_n)$不为零,其他的行列式,由于有相同的两列都为零,于是
\[
	x_i \det(a)=\det(a_1,\,\cdots\!,\,b,\,\cdots\!,\,a_n).
\]
这就是Cramer法则。如果$\det(a)$可逆,则$x_i=\det(a_1$, $\cdots$, $b$, $\cdots$, $a_n)\det(a)^{-1}$.

\para 利用Cramer法则,可以谈论一个方阵的右逆。设$\varphi$是一个$n\times n$方阵,而$\psi$又是另一个$n\times n$方阵,如果$\varphi\psi=I_n$,则称$\psi$是$\varphi$的右逆,同样,也称$\varphi$是$\psi$的左逆。

设$\psi=\begin{pmatrix}\psi_{1} & \psi_{2} & \cdots & \psi_{n}\\\end{pmatrix}$,则
\[
	\varphi\psi=\begin{pmatrix}\varphi\psi_{1} & \varphi\psi_{2} & \cdots & \varphi\psi_{n}\end{pmatrix}=\begin{pmatrix}1_{1} & 1_{2} & \cdots & 1_{n}\end{pmatrix}=I_n,
\]
所以求右逆等价于求解$n$个线性方程组$\varphi\psi_i=1_i$. Cramer法则告诉我们,如果$\det(\varphi)$可逆,则$\varphi$存在右逆。

\para 设$\varphi$是一个方阵,将$\varphi$除去第$i$行第$j$列得到的矩阵记作$\Phi_{ij}$,则矩阵元$\varphi_{ij}$对应的代数余子式$\varphi_{ji}^*$定义为$(-1)^{i+j}\det(\Phi_{ij})$,定义矩阵$\varphi^*=(\varphi_{ji}^*)$,称作方阵$\varphi$的伴随矩阵。注意到上面$i$和$j$的顺序,伴随矩阵第$i$行第$j$列的矩阵元是原矩阵第$j$行第$i$列的代数余子式,即$(\varphi^*)_{ij}=\varphi_{ji}^*$. 将行列式的按列展开用伴随矩阵重写的话,就写做
\[
	\det(\varphi)=\sum_{j=1}^n\varphi_{ji}\varphi_{ji}^*=\sum_{j=1}^n(\varphi^*)_{ij}\varphi_{ji}.
\]

伴随矩阵给出了方阵左逆的存在性的一些判据,直接计算
\[
	(\varphi^*\varphi)_{ij}=\sum_{k=1}^n(\varphi^*)_{ik}\varphi_{kj}=\sum_{k=1}^n\varphi^*_{ki}\varphi_{kj},
\]
当$i=j$时,由按列展开有$(\varphi^*\varphi)_{ii}=\det(\varphi)$. 当$i\neq j$时,依然利用按列展开,将右边的和式还原成行列式,可以发现第$i$列和第$j$列都是$(\varphi_{1j}$, $\cdots$, $\varphi_{nj})$,所以当$i\neq j$时有$(\varphi^*\varphi)_{ij}=0$. 于是
\[
	\varphi^*\varphi=\det(\varphi)I_n.
\]
所以如果$\det(\varphi)$,他的左逆写作$\det(\varphi)^{-1}\varphi^*$.

\para 如果$\varphi$同时存在左逆和右逆,设$\psi$是$\varphi$的左逆,$\pi$是$\varphi$的右逆,于是
\[
\begin{aligned}
	\psi\varphi\pi&=(\psi\varphi)\pi=I_n \pi=\pi,\\
	\psi\varphi\pi&=\psi(\varphi\pi)=\psi I_n=\psi,
\end{aligned}
\]
所以$\pi=\psi$,此时$\pi=\psi$就被称为矩阵$\varphi$的逆,记作$\varphi^{-1}$. 因此,从Cramer法则与伴随矩阵,我们已经给出了方阵逆存在的一个充分条件:$\det(\varphi)$可逆。

\pro 方阵$\varphi$左(右)逆存在的充分必要条件是$\det(\varphi)$可逆。如果存在左(右)逆,则也存在右(左)逆,且左逆等于右逆。于是方阵$\varphi$逆存在的充分必要条件是$\det(\varphi)$可逆。

\proof 上面已经证明了,如果$\det(\varphi)$可逆,则$\varphi$同时存在左逆以及右逆,且左逆等于右逆。反过来,如果$\varphi$有左逆$\psi$,则$\det(\psi\varphi)=\det(I_n)=1$,由行列式乘法公式,$1=\det(\psi\varphi)=\det(\psi)\det(\varphi)$,于是$\det(\varphi)$可逆。存在右逆同理可以推出$\det(\varphi)$可逆。从$\det(\varphi)$可逆就得出了结论。\qed

方阵存在逆时,我们会称呼该方阵是可逆的。上面的命题就是在说,环$R$上的方阵$\varphi$可逆当且仅当$\det(\varphi)$可逆。举个例子,整数环$\mathbb{Z}$上的方阵$\varphi$可逆当且仅当$\det(\varphi)=1$. 而域$k$上的方阵$\varphi$可逆当且仅当$\det(\varphi)\neq 0$. 随后如果可逆,伴随矩阵给出了逆的一个计算公式。

对于交换环,由于乘法可以交换,我们无需担心左逆不能推出右逆。但一般而言,在任意的非交换环里面,左逆存在不一定能推出右逆存在,反之亦然。但是交换环上的方阵构成的环$M_n(R)$中,行列式函数将逆的存在性等价于$R$中逆的存在性,继而保证了$M_n(R)$上左逆存在可以推出右逆存在。