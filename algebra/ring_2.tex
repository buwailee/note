\chapter{环(二)}
\ThisULCornerWallPaper{1}{../Pictures/14.png}

\section{完备化}

设$R$是一个环,而$R=\mm_0\supset \mm_1\supset \mm_2\supset \cdots$是一族理想。我们定义$R$关于这族理想的降链的完备化$\hat R$为$R/\mm_i$的极限
\[
\begin{aligned}
	\hat R&:=\varprojlim R/\mm_i\\
	&:=\left\{
	g=(g_1,g_2,\dots)\in \prod_i R/\mm_i\,:\, \text{给定$i$,对所有的$j>i$成立$g_j\equiv g_i$ (mod $\mm_i$)}
	\right\}.
\end{aligned}
\]
显然,$\hat R$是一个环,加法和乘法的定义都是自然的。这个环有自然的理想$\hat \mm_i$
\[
	\hat \mm_i:=\{g=(g_1,g_2,\dots)\in\hat R\,:\, \text{对所有$j\leq i$都有$g_j=0$}\}.
\]
由定义,不难看到$\hat R/\hat \mm_i\cong R/\mm_i$. 进而$\hat{\hat R}\cong \hat R$.

注意到局部化的等式$S^{-1}(R/\mm_i)=S^{-1}R/S^{-1}\mm_i$,因此局部化与完备化可交换。特别地,$\widehat{R_\mm}\cong \hat{R}_{\hat{\mm}}$.

直观来看,$R=\mm_0\supset \mm_1\supset \mm_2\supset \cdots$是降链,它们将$R$分成了一个一个等价类,而$\mm_i$愈小,则等价类越多,将$R$分得更细。对$g=(g_1,g_2,\dots)$,给定$i$,所有的$g_j$在$R/\mm_i$中看起来都是相同的,其中$j\geq i$,或者说,它们在同一个等价类中。所以$\hat R$中的一个元素,即从大的等价类中再取一个小的等价类,然后越来越细。

完备化的一个典型应用是取$\mm_i=\mm^i$. 比如整数环$\zz$关于$(p)^i$的完备化$\hat\zz_{(p)}$就是所谓的$p$进制数环。再比如,$R[x]$关于$(x)^i$的完备化是幂级数环$R[\![x]\!]$. 这时候,我们记这类完备化为$\hat R_\mm$.

\para 完备化的名字来自于,我们可以给$R$一个拓扑空间结构使得$\hat R$是$R$作为拓扑空间的完备化。实际上,$R$如果是拓扑空间而且是交换群,则自然我们希望是拓扑群,在拓扑群中,加法是$R\to R$的同胚。这时候取$\{\mm_i\}$为$R$原点处的拓扑基,进而我们就得到了$R$上的一个拓扑结构。这个拓扑结构的直观是与上面说的相同的,我们把更细的等价类看成了$R$中更小的开集。

在拓扑群$R$中,一个Cauchy列是$G$中的一个序列$\{g_1$, $g_2$, $\dots\}$,对每一个$R$中$0$的开邻域$U$,都存在一个正整数$n(U)$使得当$s$, $t>n(U)$时都有$g_s-g_t\in U$. 两个Cauchy列$\{x_i\}$和$\{y_i\}$是等价的,如果$x_i-y_i\to 0$. 这显然是一个等价关系。所有Cauchy列的等价类的集合就构成了$R$的完备化,记作$\bar{R}$. 

不难直接在$\bar{R}$上给出一个环结构,Cauchy列$\{p_i\}$和$\{q_i\}$的加法定义为序列$\{p_i+q_i\}$,乘法定义为序列$\{p_iq_i\}$. 这两个序列是Cauchy列,对加法,如果$p_j-p_k\in \mm_i$以及$q_j-q_k\in \mm_i$,则$(p_j+q_j)-(p_k+q_k)=p_j-p_k+q_j-q_k \in \mm_i$. 对乘法,如果$p_j-p_k\in \mm_i$以及$q_j-q_k\in \mm_i$,则$p_jq_j-p_kq_k=p_j(q_j-q_k)+q_k(p_j-p_k)\in \mm_i$. 

\lem Cauchy列的无穷子列也是Cauchy列,且与原列等价。

\proof
	设$\{x_i\}$是一个Cauchy列,而$\{x_{n_i}\}$是一个无穷子列,其中$n_i$随着$i$严格递增。于是,给定$0$的邻域$U$,则存在$n(U)$使得$i$, $j>n(U)$时候一定有$x_i-x_j\in U$,于是当$n_i$, $n_j>n(U)$时$x_{n_i}-x_{n_j}\in U$. 因此他也是Cauchy列。

	下面我们证明$x_{n_i}-x_i\to 0$,这等价于说,任取$0$的一个邻域$U$,存在一个$N$使得$i>N$时候一定有$x_{n_i}-x_{i}\in U$. 事实上,由于$n_i$是选成递增的,所以$n_i\geq i$. 由于$\{x_i\}$是一个Cauchy列,所以存在一个$n(U)$使得$i$, $j>n(U)$时候一定有$x_i-x_j\in U$,取$N=n(U)$,则$i>N$时也有$n_i>N$,于是我们就得到了结论。
\qed

\pro 我们两种方法定义的局部化是同构的,即$\bar{R}\cong \hat R$.

\proof
	首先指出$R$中的元素$g$将给出一族等价的Cauchy列。实际上,选取$g_i$在$R$中的原像$p_i$. 给定一个$R$中零的邻域$U$,我们总可以找一个$\mm_i$使得$\mm_i\subset U$. 由$\hat R$的定义,对$j$, $k>i$,都有$p_j-p_i\in \mm_i$以及$p_k-p_i \in \mm_i$,由于$\mm_i$是一个理想,所以$p_j-p_k=(p_j-p_i)-(p_k-p_i)\in \mm_i\subset U$. 这时候取$n(U)=i$即可。于是,$\{p_1$, $p_2$, $\dots\}$是一个Cauchy列。注意到,这只是对一族选定的原像得到的Cauchy列. 还需要证明,如果选取不同原像,将会得到等价的Cauchy列。实际上,如果$q_i$是$g_i$在$R$中选取的其他原像,则$p_i-q_i\in \mm_i$,因此$p_i-q_i\to 0$. 此即所需。这样就得到了映射$\varphi:\hat{R}\to \bar{R}$,它显然是一个环同态。

	反过来,我们证明$R$中每一条Cauchy列可以等价于$\hat R$中某个元素给出的Cauchy列。给定一条$R$中的Cauchy列$\{p_1$, $p_2$, $\dots\}$. 对原点的开邻域$\mm_i$,存在一个正整数$n(\mm_i)$使得$j$, $k>n(\mm_i)$时有$p_j-p_k\in \mm_i$. 为方便起见,选取$n(\mm_i)$的过程中将其选为严格递增的,这样$n(\mm_i)\geq i$. 记$n_i=n(\mm_i)+1$,考虑子列$q_i=p_{n_i}$,它显然是Cauchy列,并且,对每一个$\mm_i$,以及$j$, $k>i$,成立$q_j-q_k=p_{n_j}-p_{n_k}$,由于$n_j$, $n_k>n_i=n(\mm_i)+1>n(\mm_i)$,所以
	\[
		q_j-q_k=p_{n_j}-p_{n_k}\in \mm_i,
	\]
	这样我们就可以定义出一个$q\in \hat R$通过$q=(\bar{q}_1$, $\bar{q}_2$, $\dots)$,其中$\bar{q}_i$是$q_i$在$R/\mm_i$中的像。于是$\{q_i\}$就被称为$\{p_i\}$由$\{n_i\}$选取的子序列,他给出了$\hat R$里面的一个元素。由上一个引理,这条子序列与原序列等价。

	可以看到,改变选取的$\{n_i\}$或者等价类中不同的序列,似乎会给出$\hat R$里面不同的元素。我们下面依次说明,首先,$q\in \hat R$不依赖于$n(\mm_i)$的选取,其次,如果给出两条等价的Cauchy列,它们会得到$\hat R$中相同的元素。

	第一点,如果选取不同的严格递增的$n(\mm_i)$和$n'(\mm_i)$,得到了$n_i$和$n_i'$,继而有了不同的$q_i$和$q_i'$. 如果$n_i=n_i'$,则$q_i=q_i'$自不必说。如果$n_i\neq n_i'$,由于$q_i-q_i'=p_{n_i}-p_{n_i'}$且$n_i'$, $n_i$或者大于$n(\mm_i)$或者大于$n'(\mm_i)$,则$q_i-q_i'\in \mm_i$,所以它们在$R/\mm_i$有着相同的像,继而得到了同一个$q\in\hat R$.

	第二点,假设有两条等价的Cauchy列$\{p_1$, $p_2$, $\dots\}$和$\{p'_1$, $p'_2$, $\dots\}$,我们证明可以选出一列严格递增的正整数$\{n_i\}$使得子序列$q_i=p_{n_i}$和$q'_i=p'_{n_i}$可以定义$\hat R$中的一个元素,且满足$q_i-q_i'\in \mm_i$. 于是它们在$R/\mm_i$的像就相同了,继而得到了同一个$q\in\hat R$.

	对每一个$\mm_i$,由于$p$和$p'$等价,所以存在一个$m(\mm_i)$使得$j>m(\mm_i)$时一定有$p_j-p'_j\in \mm_i$,然后对$p$和$q'$,可以选出$n(\mm_i)$和$n'(\mm_i)$使得其满足Cauchy列的要求。现在定义$N(\mm_i)=\max\{m(\mm_i)$, $n(\mm_i)$, $n'(\mm_i)\}$以及$n_i=N(\mm_i)+1$. 于是子列$q_i=p_{n_i}$和$q'_i=p'_{n_i}$就是我们需要的子列。实际上,只需要证明$q_i-q_i'\in \mm_i$即可。由于,$n_i>N(\mm_i)\geq m(\mm_i)$,所以
	\[
		q_i-q'_i=p_{n_i}-p'_{n_i}\in \mm_i.
	\]
	这样,我们就得到了映射$\psi:\bar{R}\to \hat{R}$,它显然是一个环同态。

	最后,由构造,不难检验$\psi\varphi(g)=g$以及$\varphi\psi(p)=p$. 所以$\hat R$与$\bar R$等价。
\qed

这个同构提供了许多事情的不同观点。比如,从$R\to \hat R$有一个自然的映射$r\mapsto (r$, $r$, $r$, $\dots)$,很容易看到,这不一定是单的:如果$\bigcap_i \mm_i\neq 0$且$r-s\in \bigcap_i \mm_i$,则$r$与$s$映向$\hat R$中同一个元素。而以拓扑的观点来看,这就是在说$R$是不是一个Hausdorff空间。只有当具有足够可分性时,$\bar R$中的常序列才可能一一对应于$R$中的元素。这并不奇怪,因为众所周知,非Hausdorff空间中同一个序列可能有不同的极限。

因此,如果$R\to \hat R$的典范同态是同构,则我们称$R$是(关于理想族$\{\mm_i\}$)\idx{完备}的。如果理想族是$\mm_i=\mm^i$,则称关于理想$\mm$是完备的。显然,$\hat R$是完备的,因为它与$\hat{\hat R}$同构。

由于$R\to \hat R$的自然同态$r\mapsto (r$, $r$, $r$, $\dots)$,我们可以将$\hat R$看成一个$R$-模。

\para 我们下面将形式地处理级数,尽管上面给出了一个拓扑。在$\prod_{i=0}^\infty \mm_i$上,考虑如下等价关系:$(x_0$, $x_1$, $\dots)\sim (y_0$, $y_1$, $\dots)$,如果对每一个$n\geq 1$都有
\[
	\sum_{i=0}^{n-1} (x_i-y_i)\in \mm_n.
\]
不难看到这是一个等价关系。将$(x_0$, $x_1$, $\dots)$在$\prod_{i=0}^\infty \mm_i\,\big/\!\sim$中的像(即等价类)记作(形式)级数$\sum_{i=0}^\infty x_i$. 同样,可以定义级数对应的$R$中的部分和序列$\{X_n=\sum_{i=0}^{n-1} x_i\,:\, n\geq 1\}$,并且约定$X_0=0$.  虽然级数不同代表元的选取会对应有不同的部分和,但是对$X_n$而言,不过至多相差一个$\mm_n$中的元素,因为等价关系
\[
	\sum_{i=0}^{n-1} (x_i-y_i)\in \mm_n
\]
不过是在说$X_n-Y_n\in \mm_n$. 

给定一个级数$\sum_{i=0}^\infty x_i$,其部分和序列虽不唯一,但却给出了唯一的$(\bar{X}_1$, $\bar{X}_2$, $\dots)\in \hat{R}$,其中$\bar{X}_n$为$X_n$在$R/\mm_n$中的像。反过来,取顶$\hat R$中的元素$g$,对每一个$g_i\in R/\mm_i$找一个$R$中的原像$x_i$,定义$x'_0=x_1\in R=\mm_{0}$以及定义$x'_i=x_{i+1}-x_{i}\in \mm_{i}$,不难看出,级数$\sum_{i=0}^\infty x'_i$的部分和$\sum_{i=0}^{n-1} x'_i$就是$\{x_i\,:\,i\geq 1\}$. 所以$\hat R$都确定了一个级数$\sum_{i=0}^\infty x'_i$,其中$x'_i\in \mm_{i}$. 而且这个级数还是唯一的,因为选取不同的原像,部分和$X'_n=x_n$不过就差一个$\mm_n$中的元素而已。

因此,我们建立了$\hat R$中的元素与级数$\sum_{i=0}^\infty x_i$的一一对应,其中$x_i\in\mm_{i}$. 所以我们将就级数的和取作$\hat R$中的元素。作为这个等同的应用,此时$\hat\mm_n$是所有形如$\sum_{i=n}^\infty x_i$的级数,其中$x_i\in\mm_i$. 再比如,$\hat R$此时的$R$-模乘法就是
\[
	r\sum_{i=0}^nx_i=\sum_{i=0}^n rx_i.
\]

\para 两个级数的和是$\sum_i x_i+\sum_i y_i =\sum_i(x_i+y_i)$。乘法会复杂些,但还是挺自然的,记$\sum_i x_i$和$\sum_i y_i$的部分和分别是$\{X_i\}$与$\{Y_i\}$,则$\hat R$上的乘法翻译作
\[
	\left(\sum_{i=0}^\infty x_i\right)\left(\sum_{i=0}^\infty y_i\right)=\sum_{n=0}^\infty (X_{i+1}Y_{i+1}-X_{i}Y_i)=\sum_{i=0}^\infty (x_iy_i+x_iY_i+y_iX_i).
\]
一般而言,这个乘法并不方便,但是当$\mm_i\mm_j\subset \mm_{i+j}$时,由于$x_i\in\mm_{i}$以及$y_j\in\mm_{j}$,所以$x_iy_i\in \mm_{i+j}$,此时$\sum_{i+j=n}x_iy_j\in \mm_n$,不难证明此时的乘法就可以写作
\[
	\left(\sum_{i=0}^\infty x_i\right)\left(\sum_{i=0}^\infty y_i\right)=\sum_{n=0}^\infty \sum_{i+j=n}x_iy_j.
\]
实际上,我们只要证明
\[
	\sum_{i=0}^{n-1} (x_iy_i+x_iY_i+y_iX_i)-\sum_{k=0}^{n-1} \sum_{i+j=k}x_iy_j\in \mm_n
\]
即可,这是直接的计算。

这种情况的例子是很常见的,比如$\mm_i=\mm^i$. 作为上述乘法的例子,我们验证,如果$h\in\mm$,则$1-h$在$\hat{R}_\mm$中有逆$\sum_{i=0}^\infty h^i$,因为
\[
	(1-h)\sum_{i=0}^\infty h^i=1+\sum_{n=1}^\infty(1-1)=1.
\]
如果$R$是关于$\mm$是完备的,则$1-h$在$R$中也可逆的。

\para 这里来谈谈级数的重排。首先引入一个说法,如果$a\in \mm_{n}$但$a\not\in\mm_{n+1}$,则称$a$的阶数为$n$. $R$中每一个元素都有一个固定的阶数。

现在考虑这样一种情况,$x_i$的阶数为$i+d_i$,其中$d_i\geq 0$,而$y_i$是所有阶数为$i$的$x_j$的和,由于$x_j$的阶数总不小于$j$,所以$y_i$的求和项中$x_j$的$j$总不大于$i$,因此这个求和总是有限的。此时对每一个$n$,我们总有
\[
	\sum_{i=0}^{n-1} (x_i-y_i)=\sum_{i=0}^{n-1} x_i-\sum_{i=0}^{n-1} y_i\in \mm_n,
\]
这也是直接的计算,不过可以直接注意到$\sum_{i=0}^{n-1} y_i$即是所有阶数小于$n$的$x_j$的和,因此$\sum_{i=0}^{n-1} x_i$减去它后剩下的项阶数都应该不小于$n$.

这就意味着,在级数计算中,“高阶项”可以出现在低阶位置(指第$n$个位置处出现的加项阶数高于$n$),如果我们把他移到其他合理的位置(即不高于其阶数的位置),级数的值依然不变,都等于那种最合理的方式(即$k$-阶项就出现在第$k$个位置求和)。这种意义上,我们允许级数的部分重排。

举个例子,对级数$\sum_{i=0}^\infty x_i$,如果$x_i\in \mm_{i+d}$其中$d$是一个正整数,则
\[
	\sum_{i=0}^\infty x_i=\sum_{i=d}^\infty x_{i-d}.
\]

\para 利用级数记号,我们来观察典范同构$\hat{\hat R}\cong \hat{R}$. 首先,$\hat R/\hat \mm_n$中的元素都可以写成$\sum_{i=0}^{n-1}x_i$在$R/\mm_n$中的像。这就是典范同构$\hat R/\hat \mm_n \to R/\mm_n$在级数记号下的等同。

利用级数,$\hat{\hat R}$中的元素可以写成$\sum_{i=0}^\infty x_i$,而$x_i\in \hat \mm_i$,因此$x_i=\sum_{j\geq i}y_{ij}$,其中$y_{ij}\in \mm_j$. 将其还原成
\[
	\sum_{i=0}^nx_i=(\bar{x}_0,\bar{x}_0+\bar{x}_1,\bar{x}_0+\bar{x}_1+\bar{x}_2,\dots),
\]
利用等同,此时第$n$项$\sum_{i=0}^{n-1}\bar{x}_i$是$\sum_{i=0}^{n-1}\sum_{j=i}^{n-1} \bar{y}_{ij}$,其中$\bar{y}_{ij}\in R/\mm_n$. 换而言之,典范同构$\hat{\hat R}\cong \hat{R}$在这里即
\[
	(\bar{x}_0,\bar{x}_0+\bar{x}_1,\bar{x}_0+\bar{x}_1+\bar{x}_2,\dots)\leftrightarrow\left(\bar{y}_{00},\bar{y}_{00}+\bar{y}_{01}+\bar{y}_{11},\dots,\sum_{i=0}^{n-1}\sum_{j=i}^{n-1} \bar{y}_{ij},\dots\right),
\]
在这里,符号上,对应于第$R/\mm_n$位置处的$\bar{y}_{ij}$是$y_{ij}$在$R/\mm_n$中的像。注意到恒等式
\[
	\sum_{j=0}^{n-1}\sum_{i=0}^{j} \bar{y}_{ij}=\sum_{i=0}^{n-1}\sum_{j=i}^{n-1} \bar{y}_{ij},
\]
所以如果用级数来表述典范同构给出的等同,他就写作
\[
	\sum_{i=0}^\infty \sum_{j=i}^\infty y_{ij}=\sum_{j=0}^\infty \sum_{i=0}^jy_{ij},
\]
其中$y_{ij}\in \mm_j$. 注意到,等式左边是在$\hat{\hat R}$中,而右边是在$\hat R$中。

反过来,任取$\sum_{i=0}^\infty x_i\in \hat{R}$,其中$x_i\in \mm_0$,则我们可以将其看成$\sum_{j=0}^\infty y_j$,其中$y_0=\sum_{i=0}^\infty x_i$,而对$j\neq 0$,$y_j=0$. 这显然是一个单射,因为很清楚$\bigcap_{i=0}^\infty \hat{\mm}_i =\{0\}$. 用级数来看,$\sum_{i=0}^\infty x_i \in \hat{\mm}_n$意味着它前$n$项部分和处于$\mm_n$中,而现在$n$是任意的,所以$\sum_{i=0}^\infty x_i=0$.

\pro 如果$\mm$是$R$的一个极大理想,则$\hat R_\mm$是一个局部环,极大理想为$\hat\mm$.

\proof
	首先由$\hat R_\mm/\hat\mm\cong R/\mm$,可以得到$\mm$是一个极大理想。任取$g\not\in \hat \mm$,特别地,则$g_1\neq 0 \in R/\mm$,所以有逆$g_1^{-1}\in R/\mm$,选一个$a\in R$使得它在$R/\mm$中的像为$g_1^{-1}$,于是如果$ag$是可逆的,则$g$是可逆的。此时$ag$具有形式$(1$, $ag_2$, $ag_3$, $\dots)$.

	设$ag=1+\sum_{i=1}^\infty x_i$,其中$x_i\in \mm^i$. 记$h=-\sum_{i=1}^\infty x_i\in \hat{\mm}$,我们有$1-h\in \hat{\hat R}$,利用典范同构$\hat{\hat R}\cong \hat{R}$,只要它在$\hat{\hat R}$中有逆,则同构回来就得到了$\hat{R}$中$ag$的逆,而$1-h$的逆即熟知的几何级数$\sum_{i=0}^\infty h^i\in \hat{\hat R}$,此即所证。
\qed


\section{赋值环}

\para 令$R$是一个整环,$k$是它的分式域。称$R$是$k$的一个赋值环,如果任取非零$x\in k$,有$x\in R$或者$x^{-1}\in R$,或者都在。如果$R$还是一个Noether环,则$R$被称为一个离散赋值环,有时候会简单记作DVR.

首先指出,赋值环是一个局部环。

\proof 
	实际上,考虑所有$R$中不可逆元构成的集合$\mathfrak{m}$. 我们可以证明这是一个理想,任取$r\in R$,以及$a\in\mathfrak{m}$,如果$a=0$,则$ra=0\in \mathfrak{m}$,如果$ra\not\in R$或者说$(ra)^{-1}\in R$,则$a^{-1}=r(ra)^{-1}\in R$,这与$a$不可逆矛盾,所以$ra\in R$. 但如果$ra$可逆,则依然是$a^{-1}=r(ra)^{-1}\in R$给出$a$可逆,矛盾,所以$ra\in \mathfrak{m}$. 最后只要检验加法即可。如果$x$, $y\in\mathfrak{m}$,不妨设$y\neq 0$,所以$y^{-1}\in R$以及$xy^{-1}\in R$. 此时$x+y=(xy^{-1}+1)y$,由于$(xy^{-1}+1)\in R$以及$y\in\mathfrak{m}$,所以$x+y=(xy^{-1}+1)y\in \mathfrak{m}$.
\qed

\para 一旦给出一个$k$的赋值环,那么我们可以给出一族赋值环:设环$R'$满足$R\subset R'\subset k$,则$R'$也是域$k$的一个赋值环。这是简单的,作为$k$的子环,$R'$是整环,在$R\subset R'\subset k$上取三个整环的分式域,可知$k$是$R'$的分式域。。任取非零$x\in k$,或者$x\in R'$,如果不,则$x\not\in R$,于是$x^{-1}\in R\subset R'$. $k$自然是$k$的一个赋值环,这样的赋值环我们称为平凡赋值环。

\pro 设$R$是$k$的一个赋值环,则他是$k$中的整闭整环。

\proof
	设$x\in k$在$R$上整,于是我们有一个首一多项式方程
	\[
		x^n+r_1x^{n-1}+\cdots+r_n=0,
	\]
	其中$r_i\in R$. 如果$x\in R$,自然无需证明,否则$x^{-1}\in R$,于是
	\[
		x=-(b_1+b_2x^{-1}+\cdots+b_n x^{1-n})\in R.
	\]
\qed

\pro 对任意的一个域$k$,存在一个非平凡的赋值环$R$.

\proof
	设$K$是$k$的一个代数闭域。设$R$是$k$的一个真子环,而$f:R\to K$是一个环同态。偶对$(R,f)$之间可以定义如下偏序,即$(R,f)\leq (R',f')$当且仅当$R\subset R'$且$f'|_R=f$.

	考虑任意一条链$(R_1,f_1)\leq (R_2,f_2)\leq \cdots$,则$\bigcup_i R_i$是一个$k$的子环。包含于$k$是简单的,现在任取$x$, $y\in \bigcup_i R_i$,则必然存在一个正整数$n$使得$x$, $y\in R_n$,所以$x+y$, $xy\in R_n\subset \bigcup_i R_i$. 同样,我们可以定义一个同态$f:\bigcup_i R_i\to \Omega$:如果$x\in R_n$,则$f(x)=f_n(x)$. 可以看到,这个定义与$n$无关,所以是良定的,此时$(\bigcup_i R_i,f)$就是链$(R_1,f_1)\leq (R_2,f_2)\leq \cdots$的一个上界。同时,由于$(0)$是一个真子环,以及一个自然的包含,所以由Zorn引理,存在一个极大元$(R,g)$.

	下面证明$R$就是一个赋值环。

	首先$g(R)$作为域的子环是整环,由同构基本定义$g(R)\cong R/\ker(g)$,所以$\mathfrak{m}=\ker (g)$是一个素理想。将$R$对$\mathfrak{m}$进行局部化得到$R_\mathfrak{m}$. 我们可以通过$g'(r/s)=g(r)/g(s)$定义出一个同态$g':R_\mathfrak{m}\to K$,满足$g'|_R=g$,由极大性,$R=R_\mathfrak{m}$,所以$R$是一个局部环,$\mathfrak{m}$是它唯一的极大理想。\notprove
\qed

\pro 设$R$是一个整环,$k$是他的分式域,这里给出一个$R$是$k$的赋值环的等价条件:任取$\mathfrak{a}$, $\mathfrak{b}$为两个$R$的理想,则$\mathfrak{a}\subset \mathfrak{b}$或者$\mathfrak{b}\subset \mathfrak{a}$. 

这个命题给出了赋值环的一个理想结构的刻画,如果将理想们按照包含给一个偏序,这个命题就是说,整环$R$是一个赋值环,当且仅当它的所有理想构成的偏序集是全序的。

\proof
	如果存在两个理想$\mathfrak{a}$和$\mathfrak{b}$,但$\mathfrak{a}\subset \mathfrak{b}$和$\mathfrak{b}\subset \mathfrak{a}$都不成立,即存在$x\in \mathfrak{a}$和$y\in \mathfrak{b}$满足$x\not\in \mathfrak{b}$且$y\not\in \mathfrak{a}$. 假设$R$是一个赋值环,如果$x/y\in R$,则$x=y(x/y)\in \mathfrak{b}$,矛盾,所以$y/x=(x/y)^{-1}\in R$,则$y=x(y/x)\in\mathfrak{a}$,矛盾。所以$R$不是赋值环。逆否就得到了,如果$R$是一个赋值环,则$\mathfrak{a}\subset \mathfrak{b}$或者$\mathfrak{b}\subset \mathfrak{a}$.

	反过来,假设任取两个理想$\mathfrak{a}$和$\mathfrak{b}$,成立$\mathfrak{a}\subset \mathfrak{b}$或者$\mathfrak{b}\subset \mathfrak{a}$. 设$x/y\in k$,其中$x$, $y\in R$. 记$\mathfrak{a}=(x)$以及$\mathfrak{b}=(y)$,于是$x\in (y)$或者$y\in (x)$. 对于前者,$x=ay$,所以$x/y=ay/y=a\in R$. 对于后者,$y=bx$,所以$x/y=x/(bx)=1/b$,或者说$(x/y)^{-1}=b\in R$. 因此$R$是一个赋值环。
\qed

作为推论,如果$R$是一个赋值环,而$\mathfrak{p}$是它的一个素理想。则$R/\mathfrak{p}$以及$R_{\mathfrak{p}}$都是赋值环,整环是清楚的,而赋值环的判断只要看理想的结构即可,$R/\mathfrak{p}$的理想一一对应着$R$中包含$\mathfrak{p}$的理想,而$R_{\mathfrak{p}}$的理想一一对应着$R$的包含于$\mathfrak{p}$的理想。

\para 这里解释赋值环的名字。设$R$是一个赋值环,分式域为$k$,记$k^*=k-\{0\}$,而$U$是$R$中所有可逆元构成的集合。显然$R$是$k^*$的一个子群,进而有商群$\Gamma=k^*/U$. 这被称为赋值群。

在商群$\Gamma$上可以定义偏序,记$\bar{x}$是$x\in k^*$在$\Gamma$中的像,定义$\bar{x}\leq \bar{y}$当且仅当$x^{-1}y\in R$,这自然是良定的,选取不同的代表元,不过就是在$x^{-1}y$上乘以一个$R$中的可逆元而已。这是一个全序,实际上,如果$x^{-1}y\not\in R$,则$yx^{-1}\in R$,反之亦然。更进一步,这个全序还与群结构相容,即$\bar{x}\leq \bar{y}$可以推出,对任意的$\bar{z}$都有$\bar{x}\bar{z}\leq \bar{y}\bar{z}$. 这就意味着,两边是可以消去同一个因子却不改变偏序。正如在$\zz$上,$m\leq p$可以推出$m+p\leq n+p$. 一个具有全序的交换群,如果全序与群结构相容,则称这是全序交换群。称商同态$v:k^*\to\Gamma$为赋值映射,赋的值就是一个全序交换群里面的元素。按照一般习惯,交换群$\Gamma$中的乘法最好写成加法,于是$v(xy)=v(x)+v(y)$. 按照这个记号,任取$x\in R$,我们有$v(x)=v(1x)=v(x)+v(1)$,所以$v(1)=0$. 同样,由于$1^{-1}x\in R$,所以$v(x)\geq v(1)=0$. 这意味着$R$中的元素赋值都是非负的。

此外,$v(x+y)\geq \min\left(v(x),v(y)\right)$. 不妨设$v(x)\leq v(y)$,那就是要证明$v(x+y)\geq v(x)$. 由于$v(x)\leq v(y)$,于是$x^{-1}y\in R$以及$1+x^{-1}y\in R$,所以
\[
	v(x+y)=v(x(1+x^{-1}y))=v(x)+v(1+x^{-1}y)\geq v(x)+0=v(x).
\]

\para 反过来,设$G$是一个全序交换群,群同态$v:k^*\to G$如果$v(x+y)\geq \min\left(v(x),v(y)\right)$对任意的$x$, $y\in k^*$都成立,则它被称为一个赋值映射。一个赋值映射将如下定义$k$的一个赋值环:记$R=\{x\in k^* \,:\, v(x)\geq 0\}\cup \{0\}$,则$R$是$k$的一个赋值环。实际上,由于$v(1)=v(1\cdot 1)=v(1)+v(1)$,所以$v(1)=0$. 类似地,$0=v(1)=v(xx^{-1})=v(x)+v(x^{-1})$,所以,如果$v(x)\leq 0$,则$v(x^{-1})=-v(x)\geq 0$. 赋值环的名字就来自于赋值映射。

由于赋值环一定是局部环,从赋值映射的角度来看,极大理想就是$R-v^{-1}(0)$. 理由是简单的,如果$x\in R$是可逆的,则$v(x)\geq 0$和$v(x^{-1})\geq 0$同时成立,这就意味着$v(x)=0$. 所以$R-v^{-1}(0)$就是$R$中所有不可逆元的集合,理想结构的验证从赋值映射的性质来看是简单的。

\pro 在赋值环中,Noether的那种是很特殊的,他们是离散赋值环。一个赋值环是离散赋值环,当且仅当它的赋值群是整数群$\zz$. 

在离散赋值环上,添加$v(0)=+\infty$往往是方便的,记$(n,+\infty]=\{m\in \zz\,:\, m>n\}\cup \{+\infty\}$,则离散赋值环的极大理想就写作$v^{-1}\left((0,+\infty]\right)$. 

\proof
	假设一个赋值环$R$是离散赋值环,即是Noether环。考虑$R$的所有非平凡主理想,它们也是全序的,由于$R$是Noether环,所以存在一个极大非平凡主理想$(a)$使得其他非平凡主理想$(x)$都满足$(x)\subset (a)$. 这也意味着,所有不可逆元$x$都属于$(a)$,或者说$(a)$就是赋值环的那个极大理想。

	取定不可逆元$x$,他一定可以写成$x=ra^n$的形式。实际上,如果对每一个正整数$n$,都存在不可逆元$r_n\in R$使得$x=r_na^n$成立,则$r_{n+1}a^{n+1}=r_n a^n$和$R$是一个整环可以推出,$r_n = r_{n+1}a$,所以$(r_n)\subset r_{n+1}$. 于是我们就得到了一个升链
	\[
	(r_1)\subset (r_2)\subset \cdots \subset (r_n)\subset \cdots,
	\]
	由于$R$是Noether环,所以一定存在一个正整数$N$使得$(r_N)=(r_{N+1})$,因此存在$b\in R$使得$br_N=r_{N+1}$,但是由于$r_N=ar_{N+1}$,所以$r_N=abr_N$,由整环的消去律,就得到了$ab=1$,这与$a$不可逆矛盾。

	记$x=ra^n$,其中$r$可逆。在$\Gamma$中$\bar{x}=\bar{r}\bar{a}^n=\bar{a}^n$,或者$\bar{x}=1/(\bar{r}\bar{a}^n)=\bar{a}^{-n}$,这样我们就得到了$\Gamma=\{\bar{a}^n\,:\, n\in \zz\}\cong \zz$.

	反过来,设有一个赋值映射$v:k^*\to \zz$. 如果$x$, $y\in R$满足$v(x)=v(y)$,于是$v(xy^{-1})=v(x)-v(y)=0$,所以$xy^{-1}$可逆。由于$x=(xy^{-1})y$,所以$(x)=(y)$. 这也意味着,如果有一个真理想$\mathfrak{a}$以及$x\in \mathfrak{a}$,则所有与$x$具有相同赋值的$y\in R$也都在$\mathfrak{a}$中,因为$(y)=(x)\subset \mathfrak{a}$.

	设$\mathfrak{a}$是$R$的一个真理想,记$n(\mathfrak{a})=\min_{x\in \mathfrak{a}}v(x)$. 由于是真理想,所以$\mathfrak{a}$中元素都是不可逆的,所以$n(\mathfrak{a})>0$. 由上面的推理,$\mathfrak{a}=v^{-1}\left((n(\mathfrak{a})-1,\infty]\right)$. 因此$\mathfrak{a}$中所有的非平凡理想都具有$\mathfrak{a}_n=v^{-1}\left((n,\infty]\right)$的形式,其中$n$是一个非负整数。因为在$v^{-1}\left((n,\infty]\right)$中必然有一个$x$使得$v(x)=n+1$,此时$v^{-1}\left((n,\infty]\right)=(x)$. 所以每一个$v^{-1}\left((n,\infty]\right)$都是主理想,记作$(a_n)$. 所以将$R$中所有理想排出来得到如下严格降链:
	\[
		(a_0)\supset (a_1) \supset (a_2)\supset \cdots,
	\]
	进而,环$R$是Noether环。
\qed

从证明中,我们还可以看到,离散赋值环是一个PID,所以也自然是UFD.

\section{离散赋值环}
\section{主理想定理}