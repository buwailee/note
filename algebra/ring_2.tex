\renewcommand\chapterimg{../Pictures/14.png}
\chapter{环(二)}
\section{赋值环}

\para 令$R$是一个整环,$k$是它的分式域。称$R$是$k$的一个赋值环,如果任取非零$x\in k$,有$x\in R$或者$x^{-1}\in R$,或者都在。如果$R$还是一个Noether环,则$R$被称为一个离散赋值环,有时候会简单记作DVR.

首先指出,赋值环是一个局部环。

\proof 
	实际上,考虑所有$R$中不可逆元构成的集合$\mathfrak{m}$. 我们可以证明这是一个理想,任取$r\in R$,以及$a\in\mathfrak{m}$,如果$a=0$,则$ra=0\in \mathfrak{m}$,如果$ra\not\in R$或者说$(ra)^{-1}\in R$,则$a^{-1}=r(ra)^{-1}\in R$,这与$a$不可逆矛盾,所以$ra\in R$. 但如果$ra$可逆,则依然是$a^{-1}=r(ra)^{-1}\in R$给出$a$可逆,矛盾,所以$ra\in \mathfrak{m}$. 最后只要检验加法即可。如果$x$, $y\in\mathfrak{m}$,不妨设$y\neq 0$,所以$y^{-1}\in R$以及$xy^{-1}\in R$. 此时$x+y=(xy^{-1}+1)y$,由于$(xy^{-1}+1)\in R$以及$y\in\mathfrak{m}$,所以$x+y=(xy^{-1}+1)y\in \mathfrak{m}$.
\qed

\para 一旦给出一个$k$的赋值环,那么我们可以给出一族赋值环:设环$R'$满足$R\subset R'\subset k$,则$R'$也是域$k$的一个赋值环。这是简单的,作为$k$的子环,$R'$是整环,在$R\subset R'\subset k$上取三个整环的分式域,可知$k$是$R'$的分式域。。任取非零$x\in k$,或者$x\in R'$,如果不,则$x\not\in R$,于是$x^{-1}\in R\subset R'$. $k$自然是$k$的一个赋值环,这样的赋值环我们称为平凡赋值环。

\pro 设$R$是$k$的一个赋值环,则他是$k$中的整闭整环。

\proof
	设$x\in k$在$R$上整,于是我们有一个首一多项式方程
	\[
		x^n+r_1x^{n-1}+\cdots+r_n=0,
	\]
	其中$r_i\in R$. 如果$x\in R$,自然无需证明,否则$x^{-1}\in R$,于是
	\[
		x=-(b_1+b_2x^{-1}+\cdots+b_n x^{1-n})\in R.
	\]
\qed

\pro 对任意的一个域$k$,存在一个非平凡的赋值环$R$.

\proof
	设$K$是$k$的一个代数闭域。设$R$是$k$的一个真子环,而$f:R\to K$是一个环同态。偶对$(R,f)$之间可以定义如下偏序,即$(R,f)\leq (R',f')$当且仅当$R\subset R'$且$f'|_R=f$.

	考虑任意一条链$(R_1,f_1)\leq (R_2,f_2)\leq \cdots$,则$\bigcup_i R_i$是一个$k$的子环。包含于$k$是简单的,现在任取$x$, $y\in \bigcup_i R_i$,则必然存在一个正整数$n$使得$x$, $y\in R_n$,所以$x+y$, $xy\in R_n\subset \bigcup_i R_i$. 同样,我们可以定义一个同态$f:\bigcup_i R_i\to \Omega$:如果$x\in R_n$,则$f(x)=f_n(x)$. 可以看到,这个定义与$n$无关,所以是良定的,此时$(\bigcup_i R_i,f)$就是链$(R_1,f_1)\leq (R_2,f_2)\leq \cdots$的一个上界。同时,由于$(0)$是一个真子环,以及一个自然的包含,所以由Zorn引理,存在一个极大元$(R,g)$.

	下面证明$R$就是一个赋值环。

	首先$g(R)$作为域的子环是整环,由同构基本定义$g(R)\cong R/\ker(g)$,所以$\mathfrak{m}=\ker (g)$是一个素理想。将$R$对$\mathfrak{m}$进行局部化得到$R_\mathfrak{m}$. 我们可以通过$g'(r/s)=g(r)/g(s)$定义出一个同态$g':R_\mathfrak{m}\to K$,满足$g'|_R=g$,由极大性,$R=R_\mathfrak{m}$,所以$R$是一个局部环,$\mathfrak{m}$是它唯一的极大理想。\notprove
\qed

\pro 设$R$是一个整环,$k$是他的分式域,这里给出一个$R$是$k$的赋值环的等价条件:任取$\mathfrak{a}$, $\mathfrak{b}$为两个$R$的理想,则$\mathfrak{a}\subset \mathfrak{b}$或者$\mathfrak{b}\subset \mathfrak{a}$. 

这个命题给出了赋值环的一个理想结构的刻画,如果将理想们按照包含给一个偏序,这个命题就是说,整环$R$是一个赋值环,当且仅当它的所有理想构成的偏序集是全序的。

\proof
	如果存在两个理想$\mathfrak{a}$和$\mathfrak{b}$,但$\mathfrak{a}\subset \mathfrak{b}$和$\mathfrak{b}\subset \mathfrak{a}$都不成立,即存在$x\in \mathfrak{a}$和$y\in \mathfrak{b}$满足$x\not\in \mathfrak{b}$且$y\not\in \mathfrak{a}$. 假设$R$是一个赋值环,如果$x/y\in R$,则$x=y(x/y)\in \mathfrak{b}$,矛盾,所以$y/x=(x/y)^{-1}\in R$,则$y=x(y/x)\in\mathfrak{a}$,矛盾。所以$R$不是赋值环。逆否就得到了,如果$R$是一个赋值环,则$\mathfrak{a}\subset \mathfrak{b}$或者$\mathfrak{b}\subset \mathfrak{a}$.

	反过来,假设任取两个理想$\mathfrak{a}$和$\mathfrak{b}$,成立$\mathfrak{a}\subset \mathfrak{b}$或者$\mathfrak{b}\subset \mathfrak{a}$. 设$x/y\in k$,其中$x$, $y\in R$. 记$\mathfrak{a}=(x)$以及$\mathfrak{b}=(y)$,于是$x\in (y)$或者$y\in (x)$. 对于前者,$x=ay$,所以$x/y=ay/y=a\in R$. 对于后者,$y=bx$,所以$x/y=x/(bx)=1/b$,或者说$(x/y)^{-1}=b\in R$. 因此$R$是一个赋值环。
\qed

作为推论,如果$R$是一个赋值环,而$\mathfrak{p}$是它的一个素理想。则$R/\mathfrak{p}$以及$R_{\mathfrak{p}}$都是赋值环,整环是清楚的,而赋值环的判断只要看理想的结构即可,$R/\mathfrak{p}$的理想一一对应着$R$中包含$\mathfrak{p}$的理想,而$R_{\mathfrak{p}}$的理想一一对应着$R$的包含于$\mathfrak{p}$的理想。

\para 这里解释赋值环的名字。设$R$是一个赋值环,分式域为$k$,记$k^*=k-\{0\}$,而$U$是$R$中所有可逆元构成的集合。显然$R$是$k^*$的一个子群,进而有商群$\Gamma=k^*/U$. 这被称为赋值群。

在商群$\Gamma$上可以定义偏序,记$\bar{x}$是$x\in k^*$在$\Gamma$中的像,定义$\bar{x}\leq \bar{y}$当且仅当$x^{-1}y\in R$,这自然是良定的,选取不同的代表元,不过就是在$x^{-1}y$上乘以一个$R$中的可逆元而已。这是一个全序,实际上,如果$x^{-1}y\not\in R$,则$yx^{-1}\in R$,反之亦然。更进一步,这个全序还与群结构相容,即$\bar{x}\leq \bar{y}$可以推出,对任意的$\bar{z}$都有$\bar{x}\bar{z}\leq \bar{y}\bar{z}$. 这就意味着,两边是可以消去同一个因子却不改变偏序。正如在$\zz$上,$m\leq p$可以推出$m+p\leq n+p$. 一个具有全序的交换群,如果全序与群结构相容,则称这是全序交换群。称商同态$v:k^*\to\Gamma$为赋值映射,赋的值就是一个全序交换群里面的元素。按照一般习惯,交换群$\Gamma$中的乘法最好写成加法,于是$v(xy)=v(x)+v(y)$. 按照这个记号,任取$x\in R$,我们有$v(x)=v(1x)=v(x)+v(1)$,所以$v(1)=0$. 同样,由于$1^{-1}x\in R$,所以$v(x)\geq v(1)=0$. 这意味着$R$中的元素赋值都是非负的。

此外,$v(x+y)\geq \min\left(v(x),v(y)\right)$. 不妨设$v(x)\leq v(y)$,那就是要证明$v(x+y)\geq v(x)$. 由于$v(x)\leq v(y)$,于是$x^{-1}y\in R$以及$1+x^{-1}y\in R$,所以
\[
	v(x+y)=v(x(1+x^{-1}y))=v(x)+v(1+x^{-1}y)\geq v(x)+0=v(x).
\]

\para 反过来,设$G$是一个全序交换群,群同态$v:k^*\to G$如果$v(x+y)\geq \min\left(v(x),v(y)\right)$对任意的$x$, $y\in k^*$都成立,则它被称为一个赋值映射。一个赋值映射将如下定义$k$的一个赋值环:记$R=\{x\in k^* \,:\, v(x)\geq 0\}\cup \{0\}$,则$R$是$k$的一个赋值环。实际上,由于$v(1)=v(1\cdot 1)=v(1)+v(1)$,所以$v(1)=0$. 类似地,$0=v(1)=v(xx^{-1})=v(x)+v(x^{-1})$,所以,如果$v(x)\leq 0$,则$v(x^{-1})=-v(x)\geq 0$. 赋值环的名字就来自于赋值映射。

由于赋值环一定是局部环,从赋值映射的角度来看,极大理想就是$R-v^{-1}(0)$. 理由是简单的,如果$x\in R$是可逆的,则$v(x)\geq 0$和$v(x^{-1})\geq 0$同时成立,这就意味着$v(x)=0$. 所以$R-v^{-1}(0)$就是$R$中所有不可逆元的集合,理想结构的验证从赋值映射的性质来看是简单的。

\pro 在赋值环中,Noether的那种是很特殊的,他们是离散赋值环。一个赋值环是离散赋值环,当且仅当它的赋值群是整数群$\zz$. 

在离散赋值环上,添加$v(0)=+\infty$往往是方便的,记$(n,+\infty]=\{m\in \zz\,:\, m>n\}\cup \{+\infty\}$,则离散赋值环的极大理想就写作$v^{-1}\left((0,+\infty]\right)$. 

\proof
	假设一个赋值环$R$是离散赋值环,即是Noether环。考虑$R$的所有非平凡主理想,它们也是全序的,由于$R$是Noether环,所以存在一个极大非平凡主理想$(a)$使得其他非平凡主理想$(x)$都满足$(x)\subset (a)$. 这也意味着,所有不可逆元$x$都属于$(a)$,或者说$(a)$就是赋值环的那个极大理想。

	取定不可逆元$x$,他一定可以写成$x=ra^n$的形式。实际上,如果对每一个正整数$n$,都存在不可逆元$r_n\in R$使得$x=r_na^n$成立,则$r_{n+1}a^{n+1}=r_n a^n$和$R$是一个整环可以推出,$r_n = r_{n+1}a$,所以$(r_n)\subset r_{n+1}$. 于是我们就得到了一个升链
	\[
	(r_1)\subset (r_2)\subset \cdots \subset (r_n)\subset \cdots,
	\]
	由于$R$是Noether环,所以一定存在一个正整数$N$使得$(r_N)=(r_{N+1})$,因此存在$b\in R$使得$br_N=r_{N+1}$,但是由于$r_N=ar_{N+1}$,所以$r_N=abr_N$,由整环的消去律,就得到了$ab=1$,这与$a$不可逆矛盾。

	记$x=ra^n$,其中$r$可逆。在$\Gamma$中$\bar{x}=\bar{r}\bar{a}^n=\bar{a}^n$,或者$\bar{x}=1/(\bar{r}\bar{a}^n)=\bar{a}^{-n}$,这样我们就得到了$\Gamma=\{\bar{a}^n\,:\, n\in \zz\}\cong \zz$.

	反过来,设有一个赋值映射$v:k^*\to \zz$. 如果$x$, $y\in R$满足$v(x)=v(y)$,于是$v(xy^{-1})=v(x)-v(y)=0$,所以$xy^{-1}$可逆。由于$x=(xy^{-1})y$,所以$(x)=(y)$. 这也意味着,如果有一个真理想$\mathfrak{a}$以及$x\in \mathfrak{a}$,则所有与$x$具有相同赋值的$y\in R$也都在$\mathfrak{a}$中,因为$(y)=(x)\subset \mathfrak{a}$.

	设$\mathfrak{a}$是$R$的一个真理想,记$n(\mathfrak{a})=\min_{x\in \mathfrak{a}}v(x)$. 由于是真理想,所以$\mathfrak{a}$中元素都是不可逆的,所以$n(\mathfrak{a})>0$. 由上面的推理,$\mathfrak{a}=v^{-1}\left((n(\mathfrak{a})-1,\infty]\right)$. 因此$\mathfrak{a}$中所有的非平凡理想都具有$\mathfrak{a}_n=v^{-1}\left((n,\infty]\right)$的形式,其中$n$是一个非负整数。因为在$v^{-1}\left((n,\infty]\right)$中必然有一个$x$使得$v(x)=n+1$,此时$v^{-1}\left((n,\infty]\right)=(x)$. 所以每一个$v^{-1}\left((n,\infty]\right)$都是主理想,记作$(a_n)$. 所以将$R$中所有理想排出来得到如下严格降链:
	\[
		(a_0)\supset (a_1) \supset (a_2)\supset \cdots,
	\]
	进而,环$R$是Noether环。
\qed

从证明中,我们还可以看到,离散赋值环是一个PID,所以也自然是UFD.

\section{离散赋值环}
\section{主理想定理}