\chapter{环(一)}
依旧是我们的假设:下面出现的环都是交换含幺环。
\section{理想}

\para 理想是环$R$看成$R$-模时候的子模。一个理想被集合$S\subset R$生成是指在$R$-模意义上生成的子模,即$S$中元素的任意有限线性组合,我们将其记作$(S)$,当$S$是单元素集的时候,这就是主理想,当$S$是有限集的时候,时常就写成$(S)=(f_1,\,\,\cdots,f_n)$. 一个理想生成的理想自然就是他本身。

\para 在不同的理想之间可以定义运算,设$\mathfrak{a}$和$\mathfrak{b}$是$R$的两个理想,则
\[\mathfrak{a}+\mathfrak{b}=\{a+b\,:\,a\in\mathfrak{a},\,b\in\mathfrak{b}\}
\]
是一个理想。

两个理想的交显然还是一个理想(这可以类比两个子群的交还是子群),两个理想$\mathfrak{a}$和$\mathfrak{b}$的乘积$\mathfrak{a}\mathfrak{b}$被定义为集合$\{ab\,:\,a\in\mathfrak{a},\,b\in\mathfrak{b}\}$生成的理想。

\para 称一个环$R$是一个整环,如果任取非零的$a$, $b\in R$,可以推出$ab\neq 0$. 一般而言,一个环不是整环。如果$ab=0$但$a$和$b$都不等于零,则这样的$a$或者$b$被称为一个零因子。零显然是一个零因子。而整环就是一个没有非零零因子的环。换而言之,在整环里面,消去律是成立的。

考虑一个单同态$f:R\to S$,如果$S$是一个整环,则$R$也是一个整环。为了证明他,考虑到$ab=0$等价于$f(ab)=f(a)f(b)=0$,由于$S$是整环,所以$f(a)=0$或者$f(b)=0$,再由单同态就得到了结论。作为推论,整环的子环也一定是整环。

整环上的多项式环也是整环:设多项式$f$以及$g$都不为零,则它们的首项系数$a$与$b$不为零,因此$fg$的首项系数$ab$不为零,也就是说$fg\neq 0$.

\para 一个理想$\mathfrak{p}$被称为素理想,如果$R/\mathfrak{p}$是一个整环。一个理想$\mathfrak{m}$被称为极大理想,如果$R/\mathfrak{p}$是一个域。由于域是整环,所以极大理想是素理想,反之不然。由于$R=R/(0)$,所以只有在整环里面,零理想才是素理想。

\para 设$R$是一个环,$I(R)$是$R$所有真理想的集合,上面按照包含构成了一个偏序,即$\mathfrak{a}\leq \mathfrak{b}$当且仅当$\mathfrak{a}\subset \mathfrak{b}$.

极大理想的命名就来自于这个偏序。对于任意的理想链,他们的并所生成的理想要比他们大。并且由于$R$肯定有一个理想$(0)$,所以由Zorn引理,$R$中存在在上述偏序下极大的理想,下面检验这就是上面说的极大理想。设$\mathfrak{m}$是这样一个理想,$a\notin \mathfrak{m}$,则$(a)+\mathfrak{m}$是一个严格比$\mathfrak{m}$大的理想,由于$\mathfrak{m}$的极大性,没有比他大的真理想了,所以$(a)+\mathfrak{m}=R$. 因此,$1$可以写成$ra+m=1$的形式,在$R/\mathfrak{m}$中即$\bar{r}\bar{a}=1$,所以$\bar{a}$有逆。又因为$a$是在$R-\mathfrak{m}$中任取的,所以$R/\mathfrak{m}$是一个域。

反过来,如果$R/\mathfrak{m}$是一个域,则不存在$\mathfrak{m}$更大的真理想。假设如果存在$\mathfrak{a}$比$\mathfrak{m}$严格大,则有自然的商同态$\pi:R/\mathfrak{m}\to R/\mathfrak{a}$,由于$R/\mathfrak{m}$是一个域且$\pi$是满射,则$\ker \pi$作为域的理想只能是零理想,这样也就推出了$\pi$是一个同构,这与$\mathfrak{a}$比$\mathfrak{m}$严格大矛盾。

从可操作性来看,一开始的定义比这个极大理想的等价定义要方便不少。

\para 我们看到,极大理想是素理想。实际上,满足一些条件的极大的理想也会是素理想。
\begin{itemize}
\item 设$\mathcal{P}$是$R$的非有限生成理想构成的集合,则$\mathcal{P}$的极大元是素理想。
\item 设$\mathcal{P}$是$R$的非主理想的理想构成的集合,则$\mathcal{P}$的极大元是素理想。
\item 设$S$是一个乘性子集,设$\mathcal{P}$是$R$中与$S$不想交的理想构成的集合,则$\mathcal{P}$的极大元是素理想。
\end{itemize}

\para 设$f:R\to S$是一个环同态,如果$\mathfrak{p}$是$S$中的一个(素)理想,则$f^{-1}(\mathfrak{p})$是一个(素)理想。

\proof
	任取$a\in f^{-1}(\mathfrak{p})$以及$r\in R$,由于$f(a)\in \mathfrak{p}$,所以$f(r)f(a)=f(ra)\in \mathfrak{p}$,这也就推出了$ra\in f^{-1}(\mathfrak{p})$. 所以$f^{-1}(\mathfrak{p})$是一个理想。

	设$\pi:S\to S/\mathfrak{p}$是商同态,我们考虑复合映射$\pi\circ f:R\to S/\mathfrak{p}$,由于$f^{-1}(\mathfrak{p})\subset \ker(\pi\circ f)$,所以由商环的泛性质,$\pi\circ f$诱导出了单同态\[R/f^{-1}(\mathfrak{p})\to S/\mathfrak{p},\]
	单性从这里看出:如果$f(r_1)-f(r_2)=\mathfrak{p}$,则$r_1-r_2\in f^{-1}(\mathfrak{p})$. 

	当$\mathfrak{p}$是一个素理想的时候,$S/\mathfrak{p}$是整环,单同态$R/f^{-1}(\mathfrak{p})\to S/\mathfrak{p}$告诉我们$R/f^{-1}(\mathfrak{p})$也是整环,所以$f^{-1}(\mathfrak{p})$也是素理想。
\qed

\para 上面看到了理想的原像一定是一个理想,反过来,一般来说,一个理想的像不一定是一个理想。比如含入同态$\mathbb{Z}\hookrightarrow \mathbb{Q}$下,理想$(2)$的像不是理想。

但是,对于商映射,情况会好很多。设$\pi:R\to R/\mathfrak{a}$是一个商映射,而$\mathfrak{b}$是$R$中的一个理想,则$\bar{\mathfrak{b}}=\pi(\mathfrak{b})$是$R/\mathfrak{a}$中的一个理想。如果$\mathfrak{p}$是包含$\mathfrak{a}$的素理想,则$\bar{\mathfrak{p}}$也是一个素理想。

证明是朴实的,任取$a\in \mathfrak{b}$,以及$r \in R$,由于$ra\in \mathfrak{b}$,我们也就推出了$\bar{r}\bar{a}=\overline{ra}\in \bar{\mathfrak{b}}$. 所以我们可以考虑这样的商映射$\psi: R/\mathfrak{a}\to (R/\mathfrak{a})/\bar{\mathfrak{b}}$,他与商映射$\pi$复合可以得到满同态
\[
	\psi\pi:R\to (R/\mathfrak{a})/\bar{\mathfrak{b}},
\]
注意到$\psi\pi(r)=0$当且仅当$\bar{r}\in \bar{\mathfrak{b}}$,所以$\ker(\psi\pi)=\pi^{-1}(\bar{\mathfrak{b}})=\mathfrak{a}+\mathfrak{b}$. 由同构基本定理,我们有同构
\[
	R/(\mathfrak{a}+\mathfrak{b})\cong (R/\mathfrak{a})/\bar{\mathfrak{b}}.
\]

\para 利用上面这个观察,我们可以对商映射下的理想做出如下断言:$R/\mathfrak{a}$中的(素)理想一一对应着包含$\mathfrak{a}$的(素)理想,通过$\bar{\mathfrak{b}}\to \pi^{-1}(\bar{\mathfrak{b}})$.

\proof 
	由于$\pi$是一个满射,所以有等式$\pi(\pi^{-1}(\bar{\mathfrak{b}}))=\bar{\mathfrak{b}}$. 剩下我们要证明,如果$\mathfrak{b}\supset \mathfrak{a}$,则$\pi^{-1}(\pi(\mathfrak{b}))=\mathfrak{b}$,而这来自于$\pi^{-1}(\pi(\mathfrak{b}))=\mathfrak{a}+\mathfrak{b}$. 如果$\mathfrak{p}$是包含$\mathfrak{a}$的素理想,则$\mathfrak{a}+\mathfrak{p}=\mathfrak{p}$,上述同构写成$R/\mathfrak{p}\cong (R/\mathfrak{a})/\bar{\mathfrak{p}}$,因此$\bar{\mathfrak{p}}$也是素理想。
\qed

\para 任取环同态$f:R\to S$,我们可以做出如下分解$f:R\to f(R)\hookrightarrow S$,其中满同态$R\to f(R)$的结构我们是清楚的,因为我们可以利用同构$f(R)\cong R/\ker(f)$将它变成商同态$R\to R/\ker(f)$的情况。所以一般而言,含入同态才是造成理想的像不是理想的障碍,正如前面我们举的例子,含入同态$\mathbb{Z}\hookrightarrow \mathbb{Q}$下,理想$(2)$的像不是理想。

\section{Euclide环和主理想整环}

\para 域$k$上的多项式环$k[x]$是主理想整环。

\proof
	设$\mathfrak{a}$是$k[x]$的一个理想,由于单位理想和零理想都是主理想,所以假设$\mathfrak{a}$不是单位理想也不是零理想,这样$k$中的非零元素都不在$\mathfrak{a}$中。取一个$\mathfrak{a}$中幂次最小的非零多项式$f$,由构造,$f$存在且不可能是一个常数。显然,我们有包含关系$(f)\subset \mathfrak{a}$.

	任取$g\in \mathfrak{a}$,由多项式除法算法$g=pf+r$,其中$\deg(r)<\deg(f)$. 由于$r=g-pf\in \mathfrak{a}$,而$f$是$\mathfrak{a}$中幂次最小的非零多项式,所以$r=0$. 于是$g=pf\in (f)$给出了反向的包含关系$\mathfrak{a}\subset (f)$.

	综上,$\mathfrak{a}=(f)$. 所以$k[x]$是主理想整环。
\qed

\pro 一个整环$R$是唯一分解整环,当且仅当,他的不可约元生成的理想是素理想且$R$中的任意主理想升链稳定。

\para 因此主理想整环是唯一分解整环。

\section{域扩张与单扩张}

如果$k$是一个域,则任意的$k$-代数都是$k$-矢量空间。与此同时,赋予$k$-代数的$k$-模结构的那个同态的核$\ker(f)$作为域$k$中的理想,他只能是$k$本身或者$\{0\}$,前者太平凡,我们不考虑,后者得出了$f:k\to f(k)$是一个域同构。

\para 如果域$K$是一个$k$-代数,称$K$是$k$的一个域扩张,记作$K/k$。如果域$K$还是一个有限生成$k$-代数,称$K$是$k$的一个有限生成扩张。

因为$K$是$k$-矢量空间,我们可以定义域扩张的大小为$[K:k]=\dim_k(K)$.若$[K:k]$有限,称这个扩张是有限扩张。

前面说了,对于任意的$k$-代数$K$,$k$都同构于$K$中的一个子域,所以通常也将域扩张定义为包含$k$的更大的域。为了行文的简练,我们就假设域扩张为包含我们的域更大的域。

\para 设$B$是环,$A$是他的子环,如果对$a\in B$,存在$f\in A[x]$使得$f(a)=0$,称$a$在$A$上代数。如果$B$中任意的元素都在$A$上代数,则称$B$在$A$上代数。特别地,设$K/k$是一个扩张,若$K$在$k$上代数,则$K$被称为$k$的一个代数扩张。

每一个$k$中元素当然在$k$上代数,因为他是线性多项式的根。如果$\alpha$有逆且在$k$上代数,那么他的逆$1/\alpha$也在$k$上代数。实际上,因为$\alpha$在$k$上代数,所以存在多项式$f=\sum_{i=0}^na_ix^i$使得$f(\alpha)=0$。很容易检验,多项式$g=\sum_{i=0}^na_{n-i}x^i$使得$g(1/\alpha)=0$成立,所以$1/\alpha$在$k$上代数。

作为域扩张的例子,考虑多项式环$k[x_1,\,\,\cdots,x_n]$是一个$k$-代数,他的商域$F(k[x_1,\,\,\cdots,x_n])$就是$k$的一个扩张,并且$\dim_k(F(k[x_1,\,\,\cdots,x_n]))=\infty$,实际上,比如$\{x_1,\,\,\cdots,x_1^n,\cdots\}$是线性无关的。或者,如果$\mm$是$k[x_1,\,\,\cdots,x_n]$的一个极大理想,则$k[x_1,\,\,\cdots,x_n]/\mm$也是$k$的一个扩张,后面我们会看到这个扩张是一个有限扩张。

\pro 设$K/k$是$L/K$都是扩张,则$[L:k]\leq[K:k]$. 特别地,如果$K/k$和$L/K$都是有限扩张且$[K:k]=m$以及$[L:K]=n$,则$L/k$是有限扩张且$[L:k]=mn$。这就是说,有限扩张的有限扩张还是有限扩张。

\proof 
	设$\{a_1,\,\,\cdots,a_r\}$是$K$中的任意$k$-代数无关组,而$\{b_1,\,\,\cdots,b_s\}$是$L$中的任意$K$-代数无关组,我们来证明$\{a_ib_j\}$是$k$-线性无关组。设$\alpha=\sum_{i,j}c_{ij}a_ib_j$,其中$c_{ij}\in k$,因为$\sum_i c_{ij}a_i\in K$,所以如果$\alpha=0$,那么由$\{b_1,\,\,\cdots,b_s\}$的$K$-线性无关性,所以$\sum_i c_{ij}a_i=0$,然后再应用一次$\{a_1,\,\,\cdots,a_r\}$的$k$-线性无关性,就得到了对于任意的$i$, $j$都成立$c_{ij}=0$,于是$\{a_ib_j\}$是$k$-线性无关组。由此,维度的结论显然。
\qed

利用这个结论,比如$[K:k]$是一个素数,任意的域$L$如果满足$k\subset L\subset K$(他被称为扩张$K/k$的中间域)一定满足$[K:k]=[K:L][L:k]$,所以要么$[K:L]=[K:k]$要么$[K:k]=[L:k]$,分别对应$L=k$以及$L=K$. 

\para 设$K/k$是一个域扩张,$X$是$K$的一个子集,记$k[X]$为所有包含$k$和$X$的$K$的子环的交,令$k(X)$为所有包含$k$和$X$的$K$的子域的交。换而言之,$k[X]$即$K$中包含$k$和$X$的极小子环,而$k(X)$则是极小子域。由于$k[X]$是整环,所以由商域$F(k[X])$,因此$k(X)\subset F(k[X])$. 但是$F(k[X])$包含于任意包含$k$和$X$的子域,所以$F(k[X])\subset k(X)$. 所以$k(X)$就是$k[X]$的商域。所以下面我们研究$k[X]$.

我们特别关注$X$是单点集$\{\alpha\}$的情况,此时$k(\alpha)$被称为单扩张。下面两个命题把任意的$X$的情况约化到单扩张上面。

\pro 设$X$是$K$的任意子集,考虑指标集$I$为$X$的任意有限子集,则
\[
	k[X]=\bigcup_{\{a_1,\,\,\cdots,a_n\}\in I} k[a_1,\,\,\cdots,a_n].
\]

\proof
	设右边为$A$. 显然,任取有限子集$\{a_1,\,\,\cdots,a_n\}$,$k(a_1,\,\,\cdots,a_n)\subset k[X]$,因此$A\subset k[X]$. 反过来要证$k[X]\subset A$,只要证明$A$是一个环即可,则极小性将得到结论。任取$x$, $y\in A$,则存在有限集使得
	\[
		x\in k[a_1,\,\,\cdots,a_m],\quad y\in k[b_1,\,\,\cdots,b_n],
	\]
	因此$x+y$, $xy\in k[a_1,\,\,\cdots,a_m,b_1,\,\,\cdots,b_n]\subset A$,所以这是一个环。
\qed

类似地,还可以证明
\[
	k(X)=\bigcup_{\{a_1,\,\,\cdots,a_n\}\in I} k(a_1,\,\,\cdots,a_n).
\]
这就说明了,$k(X)$中的任意一个元素$\alpha$都可以找到有限个$X$中的元素$\{a_1,\,\,\cdots,a_n\}$使得$\alpha\in k(a_1,\,\,\cdots,a_n)$.

\pro 对$k$单扩张$\alpha_1$得到了$k(\alpha_1)$,再对其单扩张$\alpha_2$就得到了$k(\alpha_1)(\alpha_2)$,他也是$k$的一个扩张,则$k(\alpha_1)(\alpha_2)= k(\alpha_2)(\alpha_1)=k(\alpha_1,\alpha_2)$.

\proof 显然$k(\alpha_1,\alpha_2)\subset k(\alpha_1)(\alpha_2)$,因为$k(\alpha_1)(\alpha_2)$是一个域且同时包含$\alpha_1$和$\alpha_2$. 反过来,因为$k(\alpha_1,\alpha_2)$是一个域,而$\alpha_1$是他的元素,所以$k(\alpha_1)$是$k(\alpha_1,\alpha_2)$的子域,然后再对$k(\alpha_1)$做$\alpha_2$的单扩张,因为单扩张$k(\alpha_1)(\alpha_2)$是$k(\alpha_1,\alpha_2)$中包含$\alpha_2$的最小的域,所以$k(\alpha_1)(\alpha_2)\subset k(\alpha_1,\alpha_2)$.\qed

这个简单的命题顺便还告诉我们,有限次单扩张的顺序无关紧要。经过有限归纳,则$X$是有限集的情况就变成了有限次单扩张。

\para 现在考虑单扩张$k(\alpha)$. 显然
\[
	\{f(\alpha)\,:\, f\in k[x]\}\subset k[\alpha],
\]
但又由$k[\alpha]$的极小性,也就得到了$\{f(\alpha)\,:\, f\in k[x]\}=k[\alpha]$. 通过$\ev_{\alpha}:f\mapsto f(\alpha)$定义映射$\ev_{\alpha}:k[x]\to K$,不难检验这是一个$k$-线性映射以及一个环同态,他被称为赋值同态。利用赋值同态我们就可以将上面的结论总结为$k[\alpha]=\ev_{\alpha}(k[x])$.

$X$是有限集的情况是类似的,可以定义出多重赋值同态
\[
	\ev_{a_1,\,\,\cdots,a_n}:k[x_1,\,\,\cdots,x_n]\to K,
\]
然后可以检验$k[a_1,\,\,\cdots,a_n]=\ev_{a_1,\,\,\cdots,a_n}(k[x_1,\,\,\cdots,x_n])$.

在单扩张上,下面要证明,如果$k[\alpha]$在$k$上代数,则$k(\alpha)=k[\alpha]$.

\para 设$K/k$是一个扩张,$\alpha\in K$在$k$上代数,则存在多项式$f\in k[x]$使得$f(\alpha)=0$,取$g$是$k[x]$中以$\alpha$为零点的次数最低的首一多项式,称为$\alpha$的极小多项式。

\lem 极小多项式不可约。如果$f$也以$\alpha$为零点,则存在$h\in k[x]$使得$f=gh$.

\proof 
	假设可约,设$g=g_1g_2$,其中$g_1$和$g_2$都是次数比$g$低的多项式。那么在$\alpha$处,我们有$g_1(\alpha)g_2(\alpha)=0$,所以$g_1(\alpha)$和$g_2(\alpha)$中至少有一个为零,而他们都是次数比$g$低的在$\alpha$处为零的多项式,和极小多项式的选取矛盾。

	辗转相除,我们有分解$f=gh+r$,其中$r$是比$g$次数更低的多项式或者$r=0$。如果是前者,在$\alpha$处$r(\alpha)=f(\alpha)-g(\alpha)h(\alpha)=0$,所以$r$和极小多项式的选取矛盾。
\qed

我们已经知道了环同态$\ev_\alpha:k[x]\to k[\alpha]$,他当然是满的,他的核是那些在$\alpha$为零的多项式所构成的理想,从引理可以知道,他就是极小多项式生成的极大理想\footnote{若$k$是一个域,则多项式环$k[x]$是一个主理想整环,他的极大理想被不可约多项式生成。}$(g)$,所以$k[\alpha]\cong k[x]/(g)$是一个域。因而$k[\alpha]$就是我们想要的域$k(\alpha)$,他同构于$k[x]/(g)$,其中$g$是$\alpha$的极小多项式。

\para 如果$\alpha$不在$k$上代数,则称$\alpha$在$k$上超越。而单扩张$k(\alpha)$此时称为超越扩张。因为不存在多项式$f\in k[x]$以$\alpha$为跟,所以$\ker \ev_\alpha=\{0\}$,这就意味着$k[\alpha]\cong k[x]$,所以$k(\alpha)\cong F(k[x])=k(x)$.

至此,对单扩张,我们已经分成两种情况得到了$k[\alpha]$和$k(\alpha)$. 当$\alpha$在$k$上代数的时候,而他的极小多项式为$f$,则$k(\alpha)=k[\alpha]=k[x]/(f)$,当$\alpha$在$k$上超越的时候,$k[\alpha]\cong k[x]$.

% \proof 上面我们预设了一个比$k$大的域$K$的存在,免去了担心单扩张存在性的烦恼。现在有了上面单扩张的知识,我们也就可以直接构造单扩张来证明存在性。

% 	如果$\alpha$关于$k$超越,那么我们取$k(\alpha)$为$k[\alpha]$的商域$F(k[\alpha])$,这显然是一个$k$-代数,因此是一个$k$的扩张。如果$\alpha$关于$k$代数,就取$k[\alpha]$,他显然是一个$k$-代数,并且同构于域$k[x]/\mm$,其中$\mm$是$\alpha$的极小多项式生成的极大理想。\qed

\pro 两个单扩张同构,即$k(\alpha)\cong k(\beta)$,当且仅当$\alpha$和$\beta$都在$k$上代数且极小多项式相同,或者同为超越扩张。

\proof 
	如果$k(\alpha)$是超越扩张,而$k(\beta)$是代数扩张。前面已经知道$\dim_k(k(\alpha))=\infty$。如果$\beta$的极小多项式是$n$次的,那么$k[x]/\mm$中$\{1,x,\,\,\cdots,x^{n-1},x^n\}$是线性相关的,即$\dim_k(k(\beta))\leq n<\infty$,所以$k(\alpha)\not\cong k(\beta)$.

	现在,如果两者都是超越扩张,则$k(\alpha)\cong F(k[x])\cong k(\beta)$.如果两者都是代数扩张,则$k[x]/\mm_\alpha\cong k[x]/\mm_\beta$,即可推出$\mm_\alpha=\mm_\beta$,继而拥有相同的极小多项式。反过来,如果有相同的极小多项式,则$k(\alpha)\cong k[x]/\mm\cong k(\beta)$.
\qed
% 于是,对于扩张$k(\alpha_1,\,\,\cdots,\alpha_n)/k$,如果我们乐意,可以先超越扩张,然后再代数扩张。

% 下一节将关注代数扩张,对于单扩张,会有如下现在看起来应该是自然的命题:代数元的单扩张是代数扩张。

\section{整扩张与代数扩张}

\para 设$A$和$B$是环,且$A$是$B$的子环。称$x\in B$在$A$上整,如果他是某个$A[x]$中的首一多项式的根。如果$B$中任意的元素都在$A$上整,则称$K$在$k$上整。

设$k$是一个域,如果$\alpha$在$k$上整,则他在$k$上代数。反之,如果$\alpha$在$k$上代数,则存在一个多项式$f=\sum_{i=0}^na_ix^i$使得$f(\alpha)=0$,此时首一多项式$g=f/a_n$也满足$g(\alpha)$,所以$\alpha$在$k$上整。通过上面的讨论,我们发现在域上代数和在域上整等价。下面我们先证明几个关于整的结论,因为在域上面的等价性,他们也可以自然应用到域扩张上。

\pro 设$A$和$B$是环,且$A$是$B$的子环。以下命题等价:

	\no{1} $\,\alpha$在$A$上整。

	\no{2} $A[\alpha]$是一个有限生成$A$-模。

	\no{3} $A[\alpha]$包含在$B$的一个子环$C$中,$C$是一个有限生成$A$-模。

	\no{4} 存在忠实$A[\alpha]$-模$M$,他作为$A$-模时是有限生成的。
	\label{p2:1}

\proof \no{1} $\Rightarrow$ \no{2} :由于$\alpha$在$A$上代数,他的满足方程$\alpha^n+a_1\alpha^{n-1}+\cdots+a_n=0$,那么通过$\alpha^{n+r}=-(a_1\alpha^{n+r-1}+\cdots+a_n\alpha^r)$即可知$A[\alpha]$是一个有限生成$A$-模。

	\no{2} $\Rightarrow$ \no{3} :取$C=A[\alpha]$.

	\no{3} $\Rightarrow$ \no{4} :取$M=C$,这是一个忠实$A[\alpha]$-模,因为如果$aC=0$,由$C$有单位元,所以$a\cdot 1=a=0$.

	\no{4} $\Rightarrow$ \no{1} :因为$M$是$A[\alpha]$-模,所以$\alpha M\subset M$。因为$M$是有限生成$A$-模,设$M$被$\{x_1,\,\,\cdots,x_m\}$生成,则$\alpha M\subset M$告诉我们对任意的$i$都成立$\alpha x_i=\sum_{j=1}^m a_{ij} x_j$,其中$a_{ij}\in A$。所以
	\[
		\sum_{j=1}^m (\alpha\delta_{ij} -a_{ij})x_j=0,
	\]
	左乘$(\alpha\delta_{ij} -a_{ij})$的伴随矩阵,则$\det(\alpha\delta_{ij} -a_{ij})x_j=0$对任意的$1\leq j \leq m$都成立,也即$\det(\alpha\delta_{ij} -a_{ij})M=0$。由$M$的忠实性,$\det(\alpha\delta_{ij} -a_{ij})=0$,将行列式展开就是我们需要的首一多项式。\qed

如果$\{\alpha_1,\,\,\cdots,\alpha_n\}\subset B$都在$A$上整,那么$k[\alpha_1,\,\,\cdots,\alpha_n]$也是一个有限生成$k$-模,这只要利用$k[\alpha_1,\,\,\cdots,\alpha_n]=k[\alpha_1,\,\,\cdots,\alpha_{n-1}][\alpha_n]$经过有限次归纳即可。

\pro 设$A$和$B$是环,且$A$是$B$的子环。则所有在$A$上整的元素构成$B$的一个子环。如果这个子环就是$A$,那么称呼$A$在$B$中是整闭的。

\proof 如果$\alpha$和$\beta$在$A$上面整,$A[\alpha,\beta]$有限生成。因为$A[\alpha\pm\beta]\subset A[\alpha,\beta]$和$A[\alpha\beta]\subset A[\alpha,\beta]$,由上一个命题的\no{3},$\alpha\pm\beta$和$\alpha\beta$在$A$上面整。\qed

设$K/k$是一个扩张,这个命题告诉我们$K$中在$k$上代数的元素构成$K$中的子环。并且,因为如果$\alpha$代数,那么$1/\alpha$也代数,所以$K$中在$k$上代数的元素构成$K$中的子域。特别地,如果$\{\alpha_1,\,\,\cdots,\alpha_n\}$都在$k$上代数,则单扩张$k(\alpha_1,\,\,\cdots,\alpha_n)$是代数扩张。

\pro 唯一分解整环在他的商域中是整闭的。

一般而言,如果整环$R$在他的商域$F(R)$中是整闭的,则$R$就直接被称为是整闭的。

\proof 
	设$R$是唯一分解整环,$F(R)$是$R$是他的商域,再设$x\in F(R)$在$R$上整,对于唯一分解整环有分解$x=r/s$,其中$r$和$s$互素,那么就有方程
	\[
		r^n+a_1r^{n-1}s+\cdots+a_n s^n=0,
	\]
	其中$a_i\in R$,因此$s$需要整除$r^n$,而$r$和$s$互素,所以只能有$s=\pm 1$.这就说明了$x=\pm r\in R$.
\qed

\pro 设$A\subset B\subset C$是环,且$B$在$A$上整,$C$在$B$上整,则$C$在$A$上整。这就是整的传递性。

\proof 
	设$x\in C$,因为$x$在$B$上整,所以存在方程$x^n+\cdots+b_{n-1}x+b_n=0$,因为$b_i\in B$都在$A$上整,所以$B'=A[b_1,\,\,\cdots,b_n]$是有一个有限生成$A$-模。由同一个首一多项式,$x$也在$B'$上整,于是$B'[x]$是一个有限生成$B'$-模,由模的有限生成的传递性,则$B'[x]$是一个有限生成$A$-模,所以$x$在$A$上整。
\qed

回到域的情况,如果$K/k$是$L/K$都是扩张,则$L/k$是一个扩张。由整的传递性,如果$K/k$和$L/K$都是代数扩张,则$L/k$是代数扩张。

\pro 设环$A\subset B\subset C$,且$A$是Northerian的,$C$是有限生成$A$-代数,以及$C$或者是一个有限生成$B$-模,或者$C$在$B$上整,那么,$B$是一个有限生成$A$-代数。
	\label{p:2.4}

\proof 
	在题目的条件下,由Proposition \eqref{p2:1},$C$是一个有限生成$B$-模与$C$在$B$上整等价。所以只对$C$是一个有限生成$B$-模的情况证明。

	令$C=A[\bar{x}_1,\,\,\cdots,\bar{x}_m]\cong A[x_1,\,\,\cdots,x_m]/\mathfrak{a}$,以及令$y_1$, $y_2$, $\cdots$, $y_n$是$C$作为有限生成$B$-模的生成元,那么存在
	\begin{equation}
		\bar{x}_i=\sum_i\alpha_{ij}y_j,\quad y_iy_j=\sum_{k}\beta_{ijk}y_k,
	\end{equation}
	令$B_0$是由$\alpha_{ij}\in B$和$\beta_{ijk}\in B$生成的$A$-代数,由于$A$是Northerian的,所以$B_0$是Northerian的\footnote{这是因为有限生成$A$-代数同构于$A[x_1,\,\,\cdots,x_{N_B}]/\mathfrak{a}_B$,而他是Northerian的。},以及$A\subset B_0 \subset B$.

	由于$C$中的元素都是关于$\{\bar{x}_i\}$的、系数处于$A$中的多项式,那么式(\theequation)告诉我们,这个元素可以写成$\sum_i b_i y_i$,其中$b_i\in B_0$,所以$C$是一个有限生成$B_0$-模。而$B_0$是Northerian的就保证了$C$是一个Northerian的$B_0$-模。因为$B$又是$C$的子模,所以$B$是一个有限生成$B_0$-模。又$B_0$是一个有限生成$A$-代数,所以$B$是一个有限生成$A$-代数。
\qed

\pro 有限扩张等价于有限次单代数扩张。

\proof 注意到单代数扩张是有限扩张,这是因为,如果他的极小多项式为$n$次的,那么$\{1,x,\,\,\cdots,x^n\}$线性相关,而有限次有限扩张是有限扩张。

反之,设$K/k$不是代数扩张,那么存在一个元素$\alpha\in K$是超越的。因为$k\subset k(\alpha)\subset K$,所以$[K:k]\geq [k(\alpha):k]=\infty$. 如果$K/k$是代数扩张,但不是有限次单代数扩张,则对于任何的$n\in \zz^+$,一定存在一组$n$个元素的线性无关组,这和$\dim_k K$有限矛盾。所以一个有限扩张由有限次单代数扩张而成。\qed

如果加强条件,比如中间域有限,我们甚至可以得到有限扩张是单代数扩张的情况,这就是所谓的本原元定理,以后我们会遇到。

\lem Zariski引理:有限生成扩张是代数扩张。这还可以表述为,设$\mm$是$A=k[x_1,\,\,\cdots,x_n]$的一个极大理想,则$A(\mm)=A/\mm$是$k$的一个代数扩张。

在证明之前,我们可以注意到如下事实:设$x_i$在$A(\mm)$中的像为$\alpha_i$,则$A(\mm)=k(\alpha_1,\,\,\cdots,\alpha_n)$.

\proof 
	归纳证明这个命题。$n=1$的时候是简单的,$k[x]$中任意的极大理想$\mm$都是由一个不可约多项式$f$生成的,所以$\mm=(f)$,而单扩张的知识告诉我们,$k[x]/(f)$是一个代数扩张,他给$k$添加上了$f$的一个根。

	对$n$个变元的情况,假设对任意的$\mm$有$A(\mm)=k[x_1,\,\,\cdots,x_n]/\mm$是一个代数扩张。对$n+1$个变元的情况,设$k(\alpha_0,\,\,\cdots,\alpha_{n})=k[x_0,\,\,\cdots,x_n]/\mm$是$k$的一个有限生成扩张,考虑如下同态
	\[
		\ev_{(\alpha_1,\,\,\cdots,\alpha_n)}:k(\alpha_0)[x_1,\,\,\cdots,x_{n}]\to k(\alpha_0,\,\,\cdots,\alpha_{n})=k[x_0,\,\,\cdots,x_n]/\mm,
	\]
	其中$k(\alpha_0)$是一个$k$的单扩张。这是一个满同态,而右侧是一个域,所以$\ker(\ev_{(\alpha_1,\,\,\cdots,\alpha_n)})$是$k(\alpha_0)[x_1,\,\,\cdots,x_{n}]$的极大理想,由归纳假设$k(\alpha_0)(\alpha_1,\,\,\cdots,\alpha_{n})$是$k(\alpha_0)$的代数扩张。如果$k(\alpha_0)$在$k$上是代数的,则$k(\alpha_0,\,\,\cdots,\alpha_{n})$是$k$的代数扩张了。

	假设$k(\alpha_0)$是超越扩张,即$k(\alpha_0)=F(k[\alpha_0])$,$k(\alpha_0)$是$k[\alpha_0]$的商域。因为$\alpha_i$在$k(\alpha_0)$上是代数的,所以存在多项式
	\[
		a_{i0}\alpha_i^{N_i}+a_{i1}\alpha_i^{N_i-1}+\cdots +a_{i,N_i+1}=0,
	\]
	其中$a_{ij}\in k(\alpha_0)=F(k[\alpha_0])$。将其通分,可以得到一个新的等式,系数属于$k[\alpha_0]$,为了符号上的简单,不妨直接设$a_{ij}\in k[\alpha_0]$.

	将等式两边乘以$a_0^{N_i-1}$后可以看到$a_{i0}\alpha_i$在$k[\alpha_0]$上是整的,实际上,对所有的$i>0$和$\alpha_0$都可以找到这么一个$a_{i0}$。由于在$k[\alpha_0]$上整的元素构成一个环,而且$k[\alpha_0]$是他的一个子环,特别地,所有的$a_{i0}\in k[\alpha_0]$以及$\alpha_0\in k[\alpha_0]$都是整的,所以我们可以说存在一个元素$a=\prod_{i>0}a_{i0}\in k[\alpha_0]$,对每一个$\alpha_i$都成立$a\alpha_i$在$k[\alpha_0]$上是整的。

	现在任取一个$y\in k[\alpha_0,\,\,\cdots,\alpha_n]$,写作$y=\sum y_{i_0 \cdots i_n}\alpha_0^{N_{i_0}}\cdots\alpha_{n}^{N_{i_n}}$.
	因为在$k[\alpha_0]$上整的元素构成一个环,两边乘以$a^N$后可以得到$a^Ny$在$k[\alpha_0]$上是整的,其中$N$足够大,因为所有的求和都是有限的,所以$N$总是可以选出来的。

	我们已经证明了,随便取一个$y\in k(\alpha_0)$,则存在$N\in \mathbb{Z}^+$使得$a^Ny\in k(\alpha_0)$在$k[\alpha_0]$上整。由于$k[\alpha_0]$作为域上的多项式环是唯一分解整环,是整闭的,所以$a^Ny=f\in k[\alpha_0]$,因此
	\[
		y=\frac{f}{a^N}\in k(\alpha_0),
	\]
	其中$f\in k[\alpha_0]$. 而$y$是任意的,考虑$y=1/b$,其中$b\not\in \{1,a,a^2,\cdots\}$,他不能写成$f/a^N$的形式,这就完成了矛盾。故$k(\alpha_0)$不可能是超越扩张,$k(\alpha_0)$是代数扩张。所以$k(\alpha_0,\,\,\cdots,\alpha_{n})$是$k$的代数扩张。\qed

下面我们再提供一个Zariski引理的证明,他比上面的证明要短一些,用到了Propostion \eqref{p:2.4}.

\proof 设$A(\mm)=k(\alpha_1,\,\,\cdots,\alpha_n)$,如果$A(\mm)$关于$k$不是代数扩张,假设$\alpha_1,\cdots ,\alpha_r$关于$k$超越。我们可以先单扩张这些超越元,至于剩下的则关于域$B=k(\alpha_1,\cdots ,\alpha_r)$代数。

现在因为$A(\mm)$是$B$的有限扩张,根据包含关系$k\subset B\subset A(\mm)$和Propostion \eqref{p:2.4},我们可以得知$B$是一个有限生成$k$-代数,设$B=k[\beta_1,\cdots ,\beta_s]$,其中每一个$\beta_i$都有着形式$f_i/g_i$,而$f_i$, $g_i\in k[\alpha_1,\cdots \alpha_r]$。但是,$k[\alpha_1,\cdots \alpha_r]$中有多项式$h=g_1g_2\cdots g_{s}+1$使得$h^{-1}$不能写成$\beta_1,\cdots ,\beta_s$的多项式,矛盾。\qed

\pro  有限扩张、有限次单代数扩张和有限生成扩张相互等价。

\proof 前二者的等价我们已经知道了。对有限次单代数扩张而成的域$k(\alpha_1,\,\,\cdots,\alpha_m)$,由$x_i\mapsto \alpha_i$可以构造一个满的$k$-代数同态$\varphi:k[x_1,\,\,\cdots,x_m]\to k(\alpha_1,\,\,\cdots,\alpha_m)$,根据同构基本定理,
\[k(\alpha_1,\,\,\cdots,\alpha_m)\cong k[x_1,\,\,\cdots,x_m]/\ker \varphi,\]所以$k(\alpha_1,\,\,\cdots,\alpha_m)$是一个有限生成扩张。

反之,因为$A(\mm)=k[x_1,\,\,\cdots,x_n]/\mm=k(\alpha_1,\,\,\cdots,\alpha_n)$,由Zariski引理,$A(\mm)$是一个代数扩张,即$\alpha_i$都在$k$上代数,所以$A(\mm)=k(\alpha_1)\cdots(\alpha_m)$,其中每一次单扩张都是代数的。\qed

在某些书上,有限生成扩张被定义为有限次单代数扩张,通过上面的命题,我们知道了这两个定义是等价的。至此,我们完全分类了有限扩张。对于无限扩张,我们以后再研究。