%!TEX program = xelatex
\documentclass[9pt]{extarticle}
\usepackage[aptex]{noteheader}
	\theoremstyle{plain}
	\newtheorem{pro}{Proposition}
	\newtheorem{theo}{Theorem}
	\newtheorem{defi}{Definition}
	\newtheorem{lem}{Lemmma}
	\newtheorem{cor}{Corollary}
\pagestyle{plain}
\definecolor{shadecolor}{rgb}{0.92,0.92,0.92}

\newcommand{\no}[1]{{$(#1)$}}
% \renewcommand{\not}[1]{#1\!\!\!/}
\newcommand{\rr}{\mathbb{R}}
\newcommand{\zz}{\mathbb{Z}}
\newcommand{\aaa}{\mathfrak{a}}
\newcommand{\pp}{\mathfrak{p}}
\newcommand{\mm}{\mathfrak{m}}
\newcommand{\dd}{\mathrm{d}}
\newcommand{\oo}{\mathcal{O}}
\newcommand{\calf}{\mathcal{F}}
\newcommand{\calg}{\mathcal{G}}
\newcommand{\bbp}{\mathbb{P}}
\newcommand{\bba}{\mathbb{A}}
\newcommand{\osub}{\underset{\mathrm{open}}{\subset}}
\newcommand{\csub}{\underset{\mathrm{closed}}{\subset}}

\DeclareMathOperator{\im}{Im}
\DeclareMathOperator{\Hom}{Hom}
\DeclareMathOperator{\id}{id}
\DeclareMathOperator{\rank}{rank}
\DeclareMathOperator{\tr}{tr}
\DeclareMathOperator{\supp}{supp}
\DeclareMathOperator{\coker}{coker}
\DeclareMathOperator{\codim}{codim}
\DeclareMathOperator{\height}{height}
\DeclareMathOperator{\sign}{sign}

\DeclareMathOperator{\Gal}{Gal}
\DeclareMathOperator{\ann}{ann}
\DeclareMathOperator{\Ann}{Ann}
\DeclareMathOperator{\ev}{ev}
\newcommand{\cc}{\mathcal{C}}

\begin{document}

\no{1}. Suppose $\cc$ is a category, $\cc^\circ$ is its dual category. Thus any contravariant functor $\varphi:\cc\to D$ is a (covariant) functor $\varphi:\cc^\circ\to \mathcal{D}$.

\no{2}. Suppose $\cc$ and $\mathcal{D}$ are categories. We can construct a category of functors form $\cc$ to $\mathcal{D}$ by the follow steps. Firstly, the objects in \textit{Funct}$(\cc,\mathcal{D})$ are functors form $\cc$ to $\mathcal{D}$. Then, a morphism $f$ of functors from $F$ to $G$ (notation $f:F\to G$) is a family of morphisms in $\mathcal{D}$ that
\[
	f(X):F(X)\to G(X),
\]
one for each $X\in \cc$, satisfying the following condition: for all morphism $\varphi$ in $\cc$ the following diagram
\[
\begin{xy}
	\xymatrix
	{
		F(X)\ar[rr]^{f(X)} \ar[d]_{F(\varphi)}&&G(X) \ar[d]^{G(\varphi)}\\
		F(Y)\ar[rr]^{f(Y)}&&G(Y)
	}
\end{xy}
\]
is commutative.

If $F$ is isomorphic to $G$ (notation $F\cong G$), we should have morphisms $f$ and $g$ such that $f(X)\circ g(X)=\id_{G(X)}$ and $g(X)\circ f(X)=\id_{F(X)}$ for all $X\in \cc$.

\no{3}. Suppose $\hat{\cc}=\textit{Funct}(\cc^\circ,\textit{Set})$, $\forall X\in \cc$, we can associate a $\hat{X}\in \hat{\cc}$ by $\hat{X}(Y)=\Hom_\cc(Y,X)$ and $\hat{Z}(\varphi)f=f\circ \varphi$ for $\varphi \in \Hom_\cc(Y,X)$ and $f\in \Hom_\cc(X,Z)$. Then $\hat{Z}(\varphi):\hat{Z}(X)\to \hat{Z}(Y)$.

\begin{pro}
	Suppose $X$ is a set in the category \textit{Set}, it can be identified with the set $\hat{X}(e)=\Hom_{\textit{Set}}(e,X)$, where $e$ is a one-point set.
\end{pro}
\begin{proof}
	Obviously.
\end{proof}

Indeed, in an arbitrary category $\cc$ an analogue of one-point set does not necessarily exist. However, by considering $\Hom_{\cc}(Y,X)$ for all $Y\in \cc$ simultaneously, we can recover complete information about an object $X\in \cc$. This is the idea of representation theory.

\begin{defi}
	A functor $F\in \hat{\cc}$ is said to be representable if $F\cong \hat{X}$ for some $X\in \cc$. One says also that the object $X$ represents the functor $F$.
\end{defi}

\begin{theo}
$\Hom_\cc (X,Y)\cong \Hom_{\hat{\cc}} (\hat{X},\hat{Y})$.
\end{theo}
\begin{proof}
	Let $\varphi:X\to Y$ be a morphism in $\cc$. We associate with $\varphi$ the morphism of functors $\hat{\varphi}:\hat{X}\to \hat{Y}$ by
	\[
		\hat{\varphi}(Z):\theta\mapsto\varphi\circ\theta\in \hat{Y}(Z),
	\]
	where $\theta\in \hat{X}(Z)$ and $X$, $Y$, $Z\in \cc$. It is clear that $\hat{\varphi}\circ \hat{\phi}=\widehat{\varphi\circ\phi}$.

	Conversely, suppose $f\in \Hom_{\hat{\cc}} (\hat{X},\hat{Y})$, define map $i:\Hom_{\hat{\cc}} (\hat{X},\hat{Y})\to \Hom_\cc (X,Y)$ by 
	\[
		i:f\mapsto f(X)(\id_X),
	\]
	where $\id_X\in \Hom_\cc(X,X)=\hat{X}(X)$ and $f(\id_X)\in \hat{Y}(X)$. Then 
	\[
		i(\hat{\varphi})=\hat{\varphi}(X)(\id_X)=\varphi\circ \id_X=\varphi.
	\]

	On the other hand, we should show $\widehat{i(f)}=f$ when $f\in \Hom_{\hat{\cc}} (\hat{X},\hat{Y})$, and it's equivalent to show $\widehat{i(f)}(Z)=f(Z)$ for all $Z\in\cc$.

	Now suppose a morphism $\varphi:Z\to X$,
	\[
		i(f)\circ\varphi=\widehat{i(f)}(Z)(\varphi)=f(Z)(\varphi).
	\]
	Using the commutativity of the diagram,
	\[
	\begin{xy}
		\xymatrix{
			\hat{X}(X)\ar[rr]^{f(X)} \ar[d]_{\hat{X}(\varphi)}&&\hat{Y}(X) \ar[d]^{\hat{Y}(\varphi)}\\
			\hat{X}(Z)\ar[rr]^{f(Z)}&&\hat{Y}(Z)
		}
	\end{xy}
	\]

	\noindent then
	\[
		f(Z)\circ\hat{X}(\varphi)(\id_X)=\hat{Y}(\varphi)\circ f(X)(\id_X)=\hat{Y}(\varphi)(i(f))=i(f)\circ \varphi
	\]
	and
	\[
		f(Z)\circ\hat{X}(\varphi)(\id_X)=f(Z)(\id_X\circ\varphi)=f(Z)(\varphi).
	\]
	Thus $i$ is an isomorphism that $i:\Hom_{\hat{\cc}} (\hat{X},\hat{Y})\cong \Hom_\cc (X,Y)$.
\end{proof}

If $X$ represents the functor $F$, $\Hom_{\hat{\cc}}(\hat{Y},F)\cong \Hom_{\hat{\cc}}(\hat{Y},\hat{X})\cong\Hom_\cc(Y,X)$. If $Y$ also represents $F$, then there exists an isomorphism $\varphi:\hat{Y}\cong F$, then $i(\varphi)$ is the according isomorphism between $X$ and $Y$. Thus, the representing object of a representable functor is defined uniquely up to a isomorphism.

\begin{defi}
	Suppose $X$, $Y\in\cc$, the direct product $X\times Y$ is (upto an isomorphism) the object $Z$ representing the functor (if such functor is representable, or if such $Z$ exists)
	\[
		W\mapsto \hat{X}(W)\times\hat{Y}(W),
	\]
	where $\hat{X}(W)\times\hat{Y}(W)$ is the direct product of \textit{Set} which has been constructed directly.
\end{defi}
\begin{defi}
	Suppose $X$, $Y$, $S\in\textit{Set}$, $f:X\to S$ and $g:Y\to S$, the pullback of $f$ and $g$ or the fibre product of $X$ and $Y$ is
	\[
		X\times_S Y=\{(x,y)\in X\times Y:f(x)=g(y)\}.
	\]
	In a general category $\cc$, $X\times_SY$ is defined as the object $Z$ representing the functor
	\[
		W\mapsto \hat{X}(W)\times_{\hat{S}(W)}\hat{Y}(W).
	\]
\end{defi}
When $f=g$ are constant morphisms, $X\times_S Y\cong X\times Y$. When $f$ and $g$ are embedding, $X\times_S Y\cong X\cap Y$.

The universal property can be directly vertified. Since $Z$ represents the functor $W\mapsto \hat{X}(W)\times\hat{Y}(W)$, then $\hat{Z}(W)=\hat{X}(W)\times\hat{Y}(W)$. Let $W=Z$, the image of $\id_Z\in \hat{Z}(Z)$ is that $\id_Z=(\pi_X,\pi_Y)$, where $\pi_X\in \hat{Y}(Z)$ and $\pi_Y\in \hat{Y}(Z)$. This can be writen as
$X\xleftarrow{\pi_X} Z \xrightarrow{\pi_Y}Y$.

Now suppose $p_X:W\to X$ and $p_Y:W\to Y$, then there exist $\eta=(p_X,p_Y)\in \hat{X}(W)\times\hat{Y}(W)= \hat{Z}(W)$. The last thing we should to show is that $p_X=\pi_X\circ \eta$ and $p_Y=\pi_Y\circ \eta$. Indeed, for all $z\in Z$,
\[
	(\pi_X,\pi_Y)((p_X,p_Y)(z))=\id_Z((p_X,p_Y)(z))=(p_X,p_Y)(z).
\]

The uniqueness of $X\times Y$ up to an isomorphism can be drived from the above theorem or universal property directly. 

Because of some psychological reasons, we usually use a commutative diagram to visualize the universal property described above:
\[
\begin{xy}
	\xymatrix
	{
		X &&\ar[ll]_{\pi_X}Z\ar[rr]^{\pi_Y}&&Y \\
		&&\ar[urr]_{p_X}W\ar@{-->}[u]^{\eta}\ar[ull]^{p_Y}&&
	}
\end{xy}
\]
Similarly, the universal property of fibre product can be written as a commutative diagram:
\[
\begin{xy}
	\xymatrix{
		W \ar@/_/[ddr]_y \ar@{.>}[dr]|{\varphi} \ar@/^/[drr]^x \\  
	 	& X \times_S Y \ar[d]^q \ar[r]_p & X \ar[d]_f \\
	 	& Y \ar[r]^g & S
	}
\end{xy}
\]

$(A_\alpha,A_{\alpha\beta},\rho_{\alpha})$
\[
\begin{xy}
	\xymatrix{
		A \ar@/_/[ddr]_{f_\alpha} \ar@{.>}[dr]|{f} \ar@/^/[drr]^{f_\beta} \\  
	 	& \calf(U) \ar[d]^{\rho_\alpha} \ar[r]_{\rho_\beta} & \calf(U_\beta) \ar[d]_{\rho_{\beta\alpha}} \\
	 	& \calf(U_\alpha) \ar[r]^{\rho_{\alpha\beta}} & \calf(U_{\alpha}\cap U_\beta)
	}
\end{xy}
\]

Given any categorical construction, we can create the dual construction by inverting all arrows in the original construction. In such a way one can obtain the coporduct $X\coprod_S Y$ by
\[
\begin{xy}
	\xymatrix{
		W\\  
	 	& X \coprod_S Y \ar@{.>}[lu]|{\varphi} & X \ar[l]^p\ar@/_/[ull]_x  \\
	 	& Y \ar@/^/[uul]^y  \ar[u]_q& S\ar[u]^f\ar[l]_g
	}
\end{xy}
\]


We fix a category $\mathcal{J}$, and define the diagonal functor $\Delta:\cc\to \textit{Funct}(\mathcal{J},\cc)$ as follow:
\begin{itemize}
\item On objects: $\Delta X$ is the set of constant functors with the value $X$. In other words, $\Delta X(j)=X$ for all $j\in\mathcal{J}$, $\Delta X(\varphi)=\id_X$ for all morphism $\varphi$ in $\mathcal{J}$.
\item On morphisms: Suppose $\psi:X\to Y$ is a morphism in $\cc$, $\Delta\psi:\Delta X\to \Delta Y$ is defined as follows: $\Delta\psi(j):X=\Delta X(j)\to Y=\Delta Y(j)$ for all $j\in \mathcal{J}$.
\end{itemize}
It is clear that $\Delta(\varphi\circ\psi)=\Delta\varphi\circ\Delta\psi$, so $\Delta$ is indeed a functor.

\begin{defi}
	Suppose $F:\mathcal{J}\to \cc$ is a functor, the (projective or inverse) limit of $F$ in the category $\cc$ according to $\mathcal{J}$ is an object $X\in\cc$ representing the functor
	\[
		Y\mapsto \Hom_{\textit{Funct}(\mathcal{J},\cc)}(\Delta Y,F):\cc^\circ\to \textit{Set}.
	\]
	The limit of $F$ is denoted by $X=\varprojlim F$.
\end{defi}
\end{document}