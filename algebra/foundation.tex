\renewcommand\chapterimg{../Pictures/8.png}
\chapter{基础}

\section{代数结构}

这节就是罗列定义。

\para 设有两个集合$X$和$Y$,则称映射$f:X\times Y \to X$为集合$X$上的一个右\idx{作用}。称映射$g:Y\times X \to X$为集合$X$上的一个左作用。称映射$h:X\times X \to X$为集合$X$上的一个二元\idx{运算}。同样地,我们可以定义多元运算。

可以看到$\mathbb{R}$上的加法$f:\mathbb{R}\times \mathbb{R} \to \mathbb{R}$定义为$f(a,b)=a+b$是一个二元运算,同样可以检验乘法。再比如设$Y$是集合$X$上所有双射$f:X\to X$的集合,那么映射复合构成$Y$上的一个运算。

对于一个未知的运算或者作用,我们通常称之为“乘法”。左作用称为“左乘”,右作用称为“右乘”。对于运算$f(a,b)$通常直接记作$a*b$,或者再干脆一些省略中间的符号$*$,记作$ab$,读作$a$左乘$b$、$b$右乘$a$或$a$乘以$b$.

我们定义多个元素$\{a_i\,:\, 1\leq i \leq n\}$的乘法
\[
	a_1a_2\cdots a_n=a_1(a_2a_3\cdots a_n)=a_1(a_2(a_3\cdots a_n))=\cdots,
\]
这是一个递归定义。

\para 设集合$X$上有一个运算$h:X\times X \to X$,我们将$h(a,b)$直接记作$ab$. 如果任取$a$, $b$, $c\in X$都成立
\[
	(ab)c=a(bc),
\]
则该运算被称为是满足\idx{结合律}(\idx{associative property})的。

\para 对于在字符串$a_1a_2\cdots a_n$中任意加括号构成的字符串,比如
\[
(a_1(a_2(a_3a_4)a_5))(a_6(a_7a_8a_9))a_{10},
\]
先预设一个$n=0$,我们我们从左往右开始计数,遇到`$($'则$n\to n+1$,并称该括号为第$n$层括号,遇到`$)$'则$n\to n-1$,直到最后一个字符。比如上式中,第一个和第四个`$($'都是第一层括号,而第二个和第五个`$($'是第二层括号。如果到达最右端字符的时候,得到了$n=0$,则上面的字符串称为$a_1a_2\cdots a_n$的一种加括号的方式。

\pro 设$\{a_i\in A\,:\, 1\leq i\leq n\}$是$A$中任意的一个有限集,且$A$上存在一种运算满足结合律。对于任意字符串$a_1a_2\cdots a_n$中任意加括号的方式,经过运算后,他们都等于$a_1a_2\cdots a_n$.

\proof 对正整数$n$,记$[n]=\{1,2,\cdots,n\}$。设$f:[m]\to [n]$是一个递增函数,则$f$被称为$[n]$的一个$m$-划分。由于是递增函数,所以$m\leq n$. 

考虑$\{a_i\, :\, 1\leq i\leq n\}$,对于$[n]$的任意一个$(2m)$-划分$k$,$k(i)=k_i$,我们都可以通过在在每一对$a_{k_{2i-1}}$和$a_{k_{2i}}$之间加一对括号定义出一个字符串
\[
	a_1a_2\cdots (a_{k_1}\cdots a_{k_2})a_{k_2+1}\cdots (a_{k_3}\cdots a_{k_4})\cdots (a_{k_i}\cdots a_{k_{i+1}})\cdots a_n,
\]
一个$m$对括号,这个字符串的最大层数是$1$。现在,当这个字符串看作元素之间相乘的时候,我们可以证明此时他等于$a_1a_2\cdots a_n$.

为此,我们先证明如下特殊情况:对任意的正整数$n$和$k$以及元素$\{a_i\,:\, 1\leq i \leq n\}$成立
\[(a_1a_2\cdots a_{k})(a_{k+1}\cdots a_{n})=a_1a_2\cdots a_n.\]
首先注意到
\[
(a_1a_2\cdots a_{k})(a_{k+1}\cdots a_{n})=(a_1(a_2\cdots a_{k}))(a_{k+1}\cdots a_{n})=a_1((a_2\cdots a_{k})(a_{k+1}\cdots a_{n})),
\]
第二个等号来自于结合律。然后对$(a_2\cdots a_{k})(a_{k+1}\cdots a_{n})$进行同样的操作,如是归纳下去,由于$n$是有限的,所以归纳也是有限的,我们就可以得到
\[
(a_1a_2\cdots a_{k})(a_{k+1}\cdots a_{n})=a_1(a_2(a_3(\cdots (a_{k}(a_{k+1}\cdots a_{n})))))=a_1a_1a_2\cdots a_n.
\]

现在,假设有$m$对括号,多重乘积可能出现三种情况:一是$a_1a_2\cdots a_{k-1}(a_{k}\cdots a_n)$,即最后一对括号出现在最后,那么按照定义,他就等于$a_1a_2\cdots a_{k-1}a_{k}\cdots a_n$,这样我们就消去了最后一对括号。二是$a_1a_2\cdots$ $a_{k-1}(a_{k}\cdots a_l)$ $a_{l+1}\cdots a_n$,其中$a_1a_2\cdots a_{k-1}$中有$(m-1)$对括号,或者说$(a_{k}\cdots a_l)$是最后一对括号。我们利用定义和上面的结论可以得到
\begin{align*}
	a_1a_2\cdots a_{k-1}(a_{k}\cdots a_l)a_{l+1}\cdots a_n&=a_1a_2\cdots ((a_{k}\cdots a_l)(a_{l+1}\cdots a_n))\\
	&=a_1a_2\cdots a_{k-1}(a_{k}\cdots a_n)\\
	&=a_1a_2\cdots a_{k-1}a_{k}\cdots a_n.
\end{align*}
这样我们就消去了最后一对括号。如是往复,我们就可以消去全部的括号,得到$a_1\cdots a_n$.

我们已经对任意字符串$a_1a_2\cdots a_n$最大括号层数为$1$的所有可能加括号方式证明了,在满足结合律的时候等于$a_1a_2\cdots a_n$.

先设最大的层数为$N$,对所有的$(N-1)$层括号,他里面的最大括号层数为1,所以利用上面的结论,我们可以将这层内所有的括号消去,则最大层数降为$(N-1)$,如是往复,经过有限次归纳,就得到了他最终将等于$a_1a_2\cdots a_n$.\qed 

这个结论就是说,对于满足结合律的运算,有限个元素相乘的结果不依赖于他加括号的方式。

\para 有一个非空集合$G$和其上的二元运算$*$,或者记作$(G,*)$,称为一个\idx{群}(\idx{group}),如果满足:

\no{1}结合律:对于任意$a$, $b$, $c\in G$,有$(a*b)*c=a*(b*c)$;

\no{2}单位元:对于任意的$a\in G$,存在一个元素$e\in G$,使得$e*a=a*e=a$;

\no{3}反元素:对于每一个$a\in G$,存在一个元素$b\in G$,使得$b*a=a*b=e$.

$a$的反元素通常记作$a^{-1}$.此外如果不产生歧义,我们可以直接称呼$G$为一个群。显然$(\mathbb{R},+)$是一个群,单位元是0,$a^{-1}=-a$.将$\mathbb{R}$去掉了$0$之后的集合记作$\mathbb{R}-\{0\}$,则$(\mathbb{R}-\{0\},\cdot)$构成一个群,单位元是$1$,$a^{-1}=1/a$.

如果有群$(G,*)$,对于任意的$a$, $b\in G$满足$a*b=b*a$,则称$(G,*)$是一个\textit{交换群}\index{群!交换群},或者称为\idxx{群}{Abel群}。交换群的运算一般是记作加法的。

\para 我们称呼一个三元组$(R,+,\cdot)$为一个\idx{环}(\idx{ring}),如果满足下列性质:

\no{1} $(R,+)$构成一个交换群,运算称为加法,其中的单位元记作$0$,$a$的反元素记作$-a$;

\no{2} $(R,\cdot)$中的运算满足结合律,称为乘法。并且,在$R-\{0\}$中含有乘法的单位元,记作$1$. 如果$a$关于$1$有逆,则逆写作$a^{-1}$.

\no{3} 乘法满足分配律:对任意$a$, $b$, $c \in R$,成立$a\cdot(b+c)=a\cdot b+a\cdot c$以及$(b+c)\cdot a=b\cdot a+c\cdot a$.

同样地,如果不产生歧义,我们可以直接称呼$R$为一个环,而且略去乘法的符号。任取$a\in R$,由于
\[
	0a=(0+0)a=0a+0a,
\]
所以$0a=0$,同理$a0=0$. 如果在$R$中有$1=0$,则我们有任意的$r\in R$成立$r=1r=0r=0$. 这样的环我们称为零环,零环的地位在集合里面大概就类似空集。下面我们所讨论的环,一般而言不会是零环,即我们会假设$0\neq 1$.

如果$R$中的乘法是可交换的,则称$R$是一个\idxx{环}{交换环}。有些作者在环的定义中会去掉乘法单位元,称我们这里定义的环为含幺环。

和群不同的是,环乘法不一定是可逆的(即对$a$存在一个$b$使得$ab=1$),比如$0$,对于任意的$a\in R$都有$a0=0\neq 1$,所以$0$一定不是可逆的。如果一个环除了零之外的元素全部可逆,则称他为一个除环。交换除环被称为\idx{域}(\idx{field})。

在后面,除非特殊申明,环全部假设为交换含幺环。

\para 设$(M,+)$是一个交换群,$R$是一个环,如果有一个左乘$\mu:R\times M\to M$,左乘符号下面省略,使得对任意的$r$, $s\in R$以及$m$, $n\in M$,成立
\[
	(rs)m=r(sm),\quad r(m+n)=rm+rn,\quad (r+s)m=rm+sm,\quad 1m=m,
\]
则称呼$M$是一个左$R$-\idx{模},同理可以定义右$R$-模。任意的含幺环$R$显然是一个$R$-模。

由于
\[
	0m=(0+0)m=0m+0m,
\]
所以$0m=0$,注意等号两边的零是不同的。类似地,可以证明$r0=0$. 

如果$R$是交换环,则我们可以不区分左$R$-模和右$R$-模,只要等同$rm$和$mr$即可。我们后面谈论的基本都是这种模。

\para $\zz$显然是一个交换群,当然也是一个交换环。不太平凡的一点是,任意的交换群$G$可以看成$\zz$-模。这是因为对任意的$n\in \zz$和任意的$a\in G$,我们做出如下定义:如果$n>0$,定义$na$为$n$个$a$相加,如果$n=0$,定义$na=0$,如果$n<0$,定义$na=-((-n)a)$. 不难检验这是确实是一个$\zz$-模。

除了交换群之外,还有一类重要的模,\idx{矢量空间}。设$k$是一个域,则任意$k$-模$V$被称为$k$-矢量空间,$V$中的元素被称为矢量。

\para 设$\mu:H\times G\to G$为一个$G$上的左作用(一般是乘法),$K$是$G$的一个子集,记
\[
	HK=\{ab\,:\,a\in H,b\in K\}\subset G.
\]
特别地,当$H$为单点集$\{f\}$的时候,$\{f\}K$通常缩略写作$fK$. 或者当$K$是单点集$\{a\}$的时候,也缩略写作$Ha$.

设$H'\subset H$,以及$K'\subset K$,则有显然的包含关系
\[
	H'K\subset HK,\quad HK'\subset HK.
\]

% 对于环$R$的两个子集$H$和$K$,他们的加法就是上述左作用的一个特例
% \[
% 	H+K=\{a+b\,:\,a\in H,b\in K\}\subset R.
% \]
% 然后考虑到$R$上也有乘法,所以我们定义
% \[
% 	HK=\left\{\sum_i a_i b_i\,:\,a_i\in H,b_i\in K\right\}\subset R,
% \]
% 其中求和是任意有限求和。

\para 设$G$是一个群,$H$是他的一个子集。所以群运算$\mu$限制在$H\times H$得到了运算$\mu|_{H\times H}:H\times H\to G$. 如果$\im\mu|_{H\times H}\subset H$,则$(H,\mu)$也构成了一个群,他被称为$G$的一个\idxx{群}{子群}。最后一句话就是说,$H$中的任意两个元素相乘依然在$H$中,再或者用我们上面的写法,$HH\subset H$.

类似地,如果$R$是一个环,$S$是他的一个子集。如果对$R$的交换群结构,$S$是他的子群,且$SS\subset S$,则$S$被称为$R$的一个\idxx{环}{子环}。子环一般而言不用含有单位元。

同样,如果$M$是一个左$R$-模,$N$是他的一个子集。如果对$M$的交换群结构,$N$是他的子群,且$RN\subset N$,则$N$被称为$M$的一个\idxx{模}{子模}。

一个群/环/模的子群/环/模如果作为子集是真子集,则称其为真子群/环/模。

\para 注意到环$R$本身可以看成一个$R$-模,他的子模被称为理想。

显然$\{0\}$和$R$本身是$R$的理想,前者被称为零理想,后者被称为单位理想,这两者往往被称为平凡理想,因为我们感兴趣的理想往往不是他们。

设$a\in R$,则$aR$也是$R$的一个\idx{理想},被称为\idxx{理想}{主理想},有时候也记作$(a)$. 由于$\{0\}=0R$以及$R=1R$,所以零理想和单位理想都是主理想。主理想$(a)$一般也称作被$a$生成的主理想。

单位理想名字中的单位来自于可逆元的英文unit:考虑一个可逆元$a\in R$,由于存在$a^{-1}$,所以任取$r\in R$,都有$r=a(a^{-1}r)\in (a)$,所以$(a)=R$. 这就是说可逆元生成的主理想都是单位理想。

设$k$是一个域,则他没有非平凡理想:考虑$k$的一个非零理想$\mathfrak{a}$,他包含一个$a\neq 0$,所以$(a)\subset k\mathfrak{a} \subset k$,但是由于$a$可逆,所以$(a)=k\subset k\mathfrak{a} \subset k$,这也就推出了$k\mathfrak{a}=k$,然后由理想的定义$k\mathfrak{a}\subset \mathfrak{a}$就推出了$\mathfrak{a}=k$.

\section{同态与同构基本定理}

\para 设$G$和$H$是两个群,若映射$f:G\to H$满足
\[
	f(e_G)=e_H,\quad f(ab)=f(a)(b),
\]
则称$f$是一个群\idx{同态}。可以看到,群同态将单位元映射到单位元,将乘法映成乘法,基本保持了群的结构。如果还有一个群同态$g:H\to G$使得$fg=\id_H$以及$gf=\id_G$,则$f$或者$g$被称为一个群同构,通常会把$g$记作$f^{-1}$. 显然,一个群同构是一个双射,既单又满。如果两个群之间存在群同构,则称呼这两个群是同构的。

后面我们不再区分两个不同群的单位元的符号,基本上属于哪个群从上下文中都是清楚的。

\para 设$R$和$S$是两个环,若映射$f:R\to S$是交换群同态,且$f(rs)=f(r)f(s)$以及$f(1)=1$,则称$f$是一个环同态。类似地,还有环同构。

设$M$和$N$是两个$R$-模,若映射$f:M\to N$是交换群同态,且任取$r\in R$以及$m\in M$成立$f(rm)=rf(m)$,则称$f$是一个$R$-模同态。类似地,还有同构。

\para 任取一个同态$f:A\to B$,这里$A$和$B$可以是群、环、$R$-模,则$f(A)$自然地有一个$B$的子群、子环、子$R$-模的结构,称作同态$f$的像,记作$\im(f)$或者$f(A)$。如果$\im(f)=B$,则这个同态被称为满同态。

对群的情况,考虑原像集$f^{-1}(e)$. 任取$a$, $b\in f^{-1}(e)$,由于$f(ab)=f(a)f(b)=e$,所以$ab\in f^{-1}(e)$. 因此$f^{-1}(e)$是$A$的一个子群,记作$\ker(f)$,称作同态$f$的核。如果$\ker(f)=\{e\}$,则这个同态被称为单同态。

对环的情况,先只考虑他的加法群结构,则$\ker(f)=f^{-1}(0)$. 任取$a\in \ker(f)$以及$r\in A$,由于$f(ra)=f(r)f(a)=0$,所以$ra\in \ker(f)$或者$A\ker(f)\subset \ker(f)$,这就意味着$\ker(f)$是$A$的一个理想。

对$R$-模的情况,先只考虑他的加法群结构,则$\ker(f)=f^{-1}(0)$. 任取$a\in \ker(f)$以及$r\in A$,由于$f(ra)=rf(a)=0$,所以$ra\in \ker(f)$或者$R\ker(f)\subset \ker(f)$,这就意味着$\ker(f)$是$A$的一个子模。

\para 有一个自然的单同态:考虑$A$是$B$的子群/环/$R$-模,则$\id_B|_A:A\to B$就是一个单同态。此时这个映射(同态)被称为含入映射(同态),一般而言我们会略去他的符号,用一个弯曲的箭头写作$A\hookrightarrow B$. 正如我们前面看到的,设$\rho:A\to C$是一个同态,则他的核是$A$的子群/环/$R$-模,此时有$\ker\rho\hookrightarrow A$.

设$\{e\}$是平凡群,则同态$f:\{e\}\to A$只可能是$f(e)=e$,因此平凡群$\{e\}$可以看成任意群的子群,从他到任意群的同态只有一个同态,即含入同态。而任意同态$g:A\to \{e\}$满足$\ker(g)=A$,他被称为零同态。 类似的,在环中,零环$\{0\}$也提供了同样的作用,在$R$-模中,零模$\{0\}$也类似。

\para 若有一族同态$f_i:A_i\to A_{i+1}$满足$\im f_i=\ker f_{i+1}$,则列
\[
	\cdots \xrightarrow{f_{i-1}}A_i \xrightarrow{f_i} A_{i+1} \xrightarrow{f_{i+1}} A_{i+1}\xrightarrow{f_{i+1}}\cdots
\]
被称为\idx{正合}的。在正合列中,平凡群一般写作$1$,零环和零模则写作$0$,从平凡群出发或者到平凡群的同态记作$1$,零环和零模的则记作$0$,通常略去他们的符号,因为这些同态的我们是清楚的。

正合列可以用来表示一个同态是否是单射或者满射。考虑正合列$0\to A\xrightarrow{f} B$,由于$\ker(f)=\im(0\to A)=\{0\}$,所以$f$是一个单射。考虑正合列$A\xrightarrow{f} B\to 0$,由于$\im f=\ker(B\to 0)=B$,所以$f$是一个满射。

在正合列中,短正合列尤其重要,他写作
\[
	0\to A \xrightarrow{f} B \xrightarrow{g} C\to 0,
\]
其中$f$是单射,而$C$是满射。如果有如下正和列
\[
	0\to A \xrightarrow{f} B \to 0,
\]
这就意味着同态$f$即是单射又是满射。下面的命题告诉我们,$f$此时是一个同构。

\pro 一个(群/环/$R$-模)同态是同构当且仅当他既是单的又是满的。

\proof 一个同构肯定既是单的又是满的。反过来,设$f: A\to B$既是单的又是满的,则对$b\in B$,我们唯一存在一个$a\in A$使得$f(a)=b$,定义
\[
	\begin{array}{ccccc}
		g &:&B &\to& A,\\
		g &:&b &\mapsto& a.
	\end{array}
\]
我们下面检验这是一个同态。

对于群而言,设$g(b_1)=a_1$和$g(b_2)=a_2$,则只要检验$g(b_1b_2)=a_1a_2$,由于$f(a_1a_2)=f(a_1)f(a_2)=b_1b_2$以及$f$是一个单射就得出了$g(b_1b_2)=a_1a_2=g(b_1)g(b_2)$. 另外,显然$g(e)=e$,所以$g$是一个同态。

对于环而言,加法群同态部分已经由群的结论得到了,只要检验乘法即可。而乘法部分和群的情况是一模一样的。

对于$R$-模的情况,加法群同态部分已经由群的结论得到了,只要检验标量乘法即可。设$g(b)=a$,以及$g(rb)=a'$. 由于$f(a')=rb$以及$rf(a)=f(ra)=rb$,所以由$f$是一个单射也就得到了$a'=ra$,故$g(rb)=rg(b)$,所以$g$是一个同态。

最后显然$f\circ g =\id_B$以及$g\circ f=\id_A$,所以$f$是一个同构,而他的逆$g$常常记作$f^{-1}$.
\qed

\para 设$f:G\to H$是一个群同态,前面说过$\ker(f)$是$G$的一个子群。我们问反过来的一个问题,什么时候子群可以称为一个同态的核。为此,先看看核有什么性质,设$h\in \ker(f)$,则任取$g\in G$有
\[
	f\left(ghg^{-1}\right)=f(g)f(h)f(g)^{-1}=f(g)ef(g)^{-1}=e,
\]
所以$ghg^{-1}\in \ker(f)$,或者$gh \in \ker(f)g$,再或者$g\ker(f)\subset \ker(f)g$. 然后再考虑$g\to g^{-1}$我们就得到了反向包含,合起来就得到了$g\ker(f)=\ker(f)g$对任意$g\in G$都成立。

我们抽象出这个性质:如果一个$G$的子群$H$满足对任意的$g\in G$都成立$gH=Hg$,则$H$被称为$G$的\idxx{群!子群}{正规子群}。

一般而言,子群不是正规子群。但是,对于交换群,$gh=hg$,所以任意子群都是正规子群。反过来也不对,任意子群是正规子群也不一定是交换群。

\para 在继续回答上面的问题之前,我们先看看陪集。设$H$是$G$的一个子群,我们可以自然地定义一个等价关系:子集族$\{gH\,:\, g\in G\}$构成了$G$的一个划分。

证明是简单的。显然他们的并等于$G$,只要证明不交性即可,如果$gH\cap hH\neq \varnothing$,则存在$k\subset gH\cap hH$,故$g^{-1}k$和$h^{-1}k$属于$H$,由于$H$是子群,则$g^{-1}k\left(h^{-1}k\right)^{-1}=g^{-1}h\in H$,因此$g^{-1}hH=H$也就堆出了$hH=gH$.

这样定义的等价类被称为一个左陪集(coset),当然还可以类似定义右陪集。对于陪集,有如下简单性质:$|gH|=|H|$. 这点只要注意到$gh_1=gh_2$可以推出$h_1=h_2$即可。所以对于有限群,我们就得到了Lagrange定理:子群的元素个数可以整除群的元素个数,除出来就是陪集的个数。

\para 对于正规子群$H$而言,由于$gH=Hg$,所以正规子群的左陪集和右陪集相同。将$gH$记作$\bar{g}$,我们证明$\{\bar{g}\,:\, g\in G\}$构成了一个群。

首先是乘法,他来自于$G$的乘法:
\[
	\bar{g}\bar{h}=gHhH=HghH=H(gh)H=(gh)HH=(gh)H=\overline{gh}.
\]
从乘法定义看出看出,单位元是$\bar{e}$,而$\bar{g}$的逆就是$\overline{g^{-1}}$. 因此$\{\bar{g}\,:\, g\in G\}$构成了一个群,他被记作$G/H$,称作$G$关于$H$的商群,或者说$G$模掉$H$.

注意到,在定义乘法的时候,我们利用了正规子群的性质。这是非常关键的。

\para 前面我们看到了,任意同态的核都是正规子群。反过来,我们问过什么时候子群是一个同态的核。下面我们做出回答,任何正规子群都可以实现为一个同态的核。这就是说,同态的核刚好就是正规子群,不多不少。

设$H$是$G$的一个子群,我们定义一个同态
\[
	\begin{array}{ccccc}
		\pi &:&G &\to& G/H,\\
		\pi &:&g &\mapsto& \bar{g}.
	\end{array}
\]
利用$\bar{g}\bar{h}=\overline{gh}$可以看出他是一个同态,而且是满同态,且$\ker(\pi)=\bar{e}=H$. 这就证明了我们的论断。

使用短正合列,我们可以将上面的结论总结为:
\[
	1\to H \hookrightarrow G \xrightarrow{\pi} G/H\to 1.
\]

\para 类似于商群,我们可以定义出商环。设$\mathfrak{a}$是$R$的任意一个理想,作为交换群,$\mathfrak{a}$是$R$的正规子群,所以可以定义出商群$R/\mathfrak{a}=\{\bar{r}=r+\mathfrak{a}\,:\, r\in R\}$,换而言之,$r\in \bar{s}$当且仅当$r=s+a$,其中$a\in \mathfrak{a}$. 我们下面在$R/\mathfrak{a}$上定义出一个环结构。

我们定义$R/\mathfrak{a}$中的乘法为$\bar{r}\bar{s}=\overline{rs}$. 注意到这并不是$R$上的两个子集的乘法。这个乘法的定义是合理的,设$\overline{r'}=\bar{r}$,即$r'+a=r$, 以及$\overline{s'}=\bar{s}$,即$s'+b =s$,则
\[
	rs=(r'+a)(s'+b)=r's'+(r'b+as'+ab),
\]
其中$r'b+as'+ab\in \mathfrak{a}$,因为$\mathfrak{a}$是一个理想。故$\overline{rs}=\overline{r's'}$. 剩下的环的性质的检验都是直接的。

\para 同样的,我们可以定义出商模。设$M$是一个$R$-模,而$N$是$M$的任意一个子模,作为交换群可以定义出商群$M/N=\{\bar{m}=m+N\,:\, m\in M\}$. 我们下面在$M/N$上定义出一个$R$-模结构。

我们定义$M/N$中的标量乘法为$r\bar{m}=\overline{rm}$.这个乘法的定义是合理的,设$m'=m+n$,其中$n\in N$,则
\[
	rm'=rm+rn
\]
其中$rn\in N$,因为$N$是一个理想。故$\overline{rm}=\overline{rm'}$. 剩下的模的性质的检验都是直接的。

\theo 同构基本定理:设$f:A\to B$是一个(群/环/$R$-模)同态,若存在同态$g: A\to B$使得$\ker f\subset \ker g$,则他诱导了一个同态$\bar{g}:A/\ker{f}\to B$. 如果$g$是满同态,则$\bar{g}$是满同态。特别地,如果$f$是满同态,则$\bar{f}: A/\ker{f}\to B$是一个同构。

\proof
	对于群的情况,$g(\ker f)=e$,$g(a\ker f)=g(a)$. 对于环和$R$-模的情况,$g(\ker f)=0$,$g(a+\ker f)=g(a)$. 所以我们可以定义$\bar{g}(\bar{a})=g(a)$,不难检验这是一个(群/环/$R$-模)同态。并且$g$是满的时候,$\bar{g}$是满的。

	现在假设$f$是满同态,则$\bar{f}$是满同态。我们剩下的只要证明$\bar{f}$是个单射,或者$\ker(\bar{f})=\{\bar{e}\}$即可。设$\bar{f}(\bar{a})=f(a)=e$,于是$a\in \ker(f)=\bar{e}$,因此$\bar{a}\cap \bar{e}\neq\varnothing$,于是我们也就得到$\bar{a}=\bar{e}$,所以$\ker(\bar{f})=\{\bar{e}\}$. 

	由于$\bar{f}$是满同态也是单同态,所以他是一个同构。
\qed

一般只有最后一点被称为同构基本定理,但是前面的一部分是商群/环/$R$-模的泛性质,所以我们特地写出来。下面我们写两个常用的同构,他们是同构基本定理的简单推论。由于会谈到子模,所以我们可以先做观察:子模的子模依然是子模。

\para 设$L\subset M \subset N$是有包含关系的三个$R$-模,则商映射$\rho:N\to N/M$诱导了一个满射$\bar{\rho}:N/L \to N/M$使得$\ker \bar{\rho}=M/L$. 使用同构基本定理,我们就得到了同构$N/M=(N/L)/(M/L)$.

考虑商映射$\rho:N\to N/M$,由于$L\subset \ker \rho=M$,所以由同构基本定理,我们诱导了一个满射$\bar{\rho}:N/L \to N/M$,接下来只要求他的核即可。任取$\bar{x}\in N/L$以及代表元$x\in N$,由于$\bar{\rho}(\bar{x})=\rho(x)$,所以$\bar{\rho} (\bar{x})=0$当且仅当$x\in \ker \rho =M$,这就告诉我们$\ker \bar{\rho}=M/L$.

\para 设$M_1$和$M_2$是$M$的子模,则复合$M_2\hookrightarrow M_1+M_2 \to (M_1+M_2)/M_1$是一个满射,他的核是$M_1\cap M_2$,因此我们有同构
\[
	(M_1+M_2)/M_1\cong M_2/(M_1\cap M_2).
\]

\para 可以将上面的命题翻译到群上去。不过由于只有正规子群才可以谈商群,所以我们必须更加谨慎。两个相应的同构被如下表述:

\begin{itemize}
\item 给定群$G$,$N$和$M$为$G$的正规子群,满足$M$包含于$N$,则$N/M$是$G/M$的正规子群,且有如下的群同构: $ (G/M)/(N/M)\cong G/N$.

\item 给定群$G$ 、其正规子群$N$ 、其子群$H$ ,则$N\cap H$是$H$ 的正规子群,且我们有群同构如下:$H/(H\cap N)\cong HN/N$
\end{itemize}

证明和模的情况是类似的,只不过必须更加小心而已。对于环,我们一般谈理想,所以可以应用模的定理(作为模,理想是子模)。

\section{范畴与函子}

前面看到了,在群/环/模上面,我们都谈论了什么叫同态,什么是同构,以及后来将可能遇到的一些在不同的代数结构上有类似性质的构造。下面我们要引入新的语言来抽象这些东西,他的名字叫范畴。

\para 一个范畴$\mathcal{C}$有如下数据:
\begin{enumerate}
\item 一类(class)对象(object),类在这里的意思并不是构成一个集合。一个范畴中所有的对象可以不构成一个集合。如果$X$是$\mathcal{C}$的一个对象,我们沿用集合论的符号,写成$X\in \mathcal{C}$.

对每两个对象$X$和$Y$,有一个集合$\Hom(X,Y)$,其中的元素被称为态射。设$\varphi\in \Hom(X,Y)$,他通常沿用映射的记号写成$\varphi:X\to Y$,但态射不必是映射。沿用叫法,$\varphi:X\to Y$中$X$被称为定义域,而$Y$被称为值域。态射还需要满足如下假设:

\begin{itemize}

\item 任取态射$\varphi$,我们都可以从他得到唯一的定义域和值域,即$\varphi$如果同时属于$\Hom(X,Y)$和$\Hom(Z,W)$,则$X=Z$以及$Y=W$.

\end{itemize}

\item 任取对象$X$, $Y$和$Z\in \mathcal{C}$,则有映射$\Hom(X,Y)\times \Hom(Y,Z)\to \Hom(X,Z)$,沿袭映射的说法,他被称为态射的复合。设$\varphi\in \Hom(X,Y)$和$\psi \in \Hom(Y,Z)$,则复合出来的态射被记作$\psi\circ \varphi$或者再简单一点$\psi\varphi$. 复合还需要满足:
\begin{itemize}

\item 态射的复合满足结合率,即$\varphi(\psi\theta)=(\varphi\psi)\theta$.

\item 在每一个$\Hom(X,X)$中有一个元素$\id_X$使得,任取$\rho:X\to Y$以及$\psi:Z\to X$成立复合$\rho\id_X=\rho$以及$\id_X\psi=\psi$.
\end{itemize}
\end{enumerate}

如果一个范畴的对象可以构成一个集合,那么这个范畴就被称为小范畴。在定义中,我们可以看到,所有对象与集合、态射与映射的相似性,所以我们可以直接断言,所有集合以及集合间的态射构成了一个范畴,他被称为集合范畴。

类似地,所有的群/环/$R$-模和他们两两之间的同态构成了一个范畴。下面是一个稍微不一般一点的例子。

\para 设$I$是一个偏序集,偏序关系为$\leq$,那么我们如下定义态射使得他构成了一个范畴:
\[
	\Hom_{I}\left(x,y\right)=\begin{cases}
	\{x,\{x,y\}\}=(x,y)&\text{, if }x\leq y\text{;}\\
	\varnothing&\text{, otherwise}.
	\end{cases}
\]
复合映射就写作$(y,z)(x,y)=(x,z)$,因此偏序集具有一个(小)范畴结构,其中的态射记作$x\leq y$.

一个拓扑空间$X$能被看成一个范畴,对象取作他的所有开集,而态射取作
\[
	\Hom_{X}(U,V)=\begin{cases}
	\bigl\{i^U_V:U\hookrightarrow V\bigr\}&\text{, if }U\subset V\text{;}\\
	\varnothing&\text{, otherwise}.
	\end{cases}
\]
当然这个范畴也可以通过在开集族上给定一个偏序$U\leq V$当且仅当$U\subset V$给出。

所有拓扑空间和其间的连续映射当然构成一个范畴。所有光滑流形和其间的光滑映射构成一个范畴。最后一个例子是关系构成了范畴,他的对象是集合,而$\Hom(X,Y)$是所有$X\times Y$的子集,设$\varphi:X\to Y$和$\psi:Y\to Z$是两个态射,则复合被定义为
\[
	\psi\varphi=\{(x,z)\in X\times Z\,:\,\exists y\in Y\,\text{ s.t. } (x,y)\in\varphi,\, (y,z)\in \psi\}.
\]

\para 从一个给定的范畴$\mathcal{C}$,可以构造出一个新的范畴$\mathcal{C}^{\mathrm{op}}$如下:对象不变,而$\Hom_{\mathcal{C}^{\mathrm{op}}}(X,Y)$被定义为$\Hom_{\mathcal{C}}(Y,X)$,复合不变。他被称为给定范畴的对偶范畴。

\para 设$\varphi:X\to Y$和$\psi:Y\to X$,如果$\varphi\psi=\id_Y$和$\psi\varphi=\id_X$,则称对象$X$和$Y$是同构的,$\varphi$或者$\psi$被称为同构(态射)。当然,现在由于暂时不能谈一个态射是单射还是满射,即使可以,也不一定能像模/群/环范畴那样有命题“既单又满的态射是同构”。

\para 类似于集合之间有映射,范畴之间也有个函子(functor). 由于范畴有对象以及对象之间的态射,所以函子也是由关于对象和范畴的数据构成的,设$\mathcal{C}$和$\mathcal{D}$是两个范畴,$F$被称为一个函子,如果
\begin{enumerate}

\item 对任意的$X\in \mathcal{C}$,有唯一的$F(X)\in \mathcal{D}$.

\item 对任意的$\varphi\in \Hom(X,Y)$,存在唯一的$F(\varphi)\in  \Hom(F(X),F(Y))$使得成立复合关系$F(\varphi\psi)=F(\varphi)F(\psi)$以及$F(\id_X)=\id_{F(X)}$.

\end{enumerate}

一般将“从$\mathcal{C}$到$\mathcal{D}$的函子$F$”记作$F:\mathcal{C}\to \mathcal{D}$. 虽然记号相同,但是我们可以通过语境来区分他与态射。

函子的实例是非常多的,比如从$R$-模范畴到交换群范畴就有一个函子,他将一个模看成一个交换群,模同态看成交换群同态。这样的函子,从某个结构丰富的范畴到结构没那么丰富的范畴,被称为遗忘函子。遗忘函子又比如从拓扑空间范畴到集合范畴。

\para 范畴与函子也构成一个范畴,即范畴的范畴。关键是我们要定义函子之间的态射。设$F$和$G$都是从$\mathcal{C}$到$\mathcal{D}$的函子,如果对每一个$X\in \mathcal{C}$都有一个态射$f(X):F(X)\to G(X)$使得如下交换图成立
\[
\begin{xy}
	\xymatrix
	{
		F(X)\ar[rr]^{f(X)} \ar[d]_{F(\varphi)}&&G(X) \ar[d]^{G(\varphi)}\\
		F(Y)\ar[rr]^{f(Y)}&&G(Y)
	}
\end{xy}
\]
则$f$被称为一个自然变换,或者叫函子间的态射。

\para 在有些作者那里,我们上面定义的函子被称为协变函子,而函子$F:\mathcal{C}^{\mathrm{op}} \to \mathcal{D}$被称为反变函子。后者明确写出来即$F(\varphi\psi)=F(\psi)F(\varphi)$,其中$\psi$和$\varphi$都是$\mathcal{C}$中的态射。

\section{极限与余极限}

\para 设$\mathcal{J}$是一个小范畴(即范畴对象的全体能够构成一个集合),我们称函子$D:\mathcal{J}\to \mathcal{C}$为$\mathcal{C}$中的一个$\mathcal{J}$-图,或者简略叫做图。对于$j\in \mathcal{J}$,$D(j)$称为该图的一个顶点,而对任意的态射$\alpha:j_1\to j_2$,态射$D(\alpha)$称为该图的一条边。

\para 称$(A,\lambda)$为$J$-图的一个{锥形},如果$A$是$\mathcal{C}$的一个对象,而$\lambda$为一族态射$\lambda_{j}:A\to D(j)$使得如下交换图对所有的顶点和边都成立
\[
	\xymatrix{
		&A \ar[dl]_{\lambda_{j}}\ar[dr]^{\lambda_{j'}}&\\
		D(j)\ar[rr]^{D(\alpha)}&&D(j')
	}
\]

称呼一个锥形$(A,\lambda)$是$J$-图的一个{极限},如果对于任意的锥形$(B,\mu)$,都有唯一的态射$f:B\to A$使得如下分解$\mu_j:B\xrightarrow{f}A\xrightarrow{\lambda_j}D(j)$成立。

在实际过程中,如果极限$(A,\lambda)$中的态射族是明确的(乃至于是由对象$A$确定的),那么我们会用$A$来表示极限。并且一般将$J$-图$D$的极限写作$\varprojlim_{j\in J} D(j)$.

作为例子,模范畴的直积就是一种极限,他对应的小范畴,两个不同的元素之间没有态射。

\para 锥形和极限都有对偶的概念,在交换图中,无外乎就是将箭头完全反过来。称$(A,\lambda)$为$J$-图的一个{余锥形},如果$A$是$\mathcal{C}$的一个对象,而$\lambda$为一族态射$\lambda_{j}:D(j)\to A$使得如下交换图对所有的顶点和边都成立
\[
	\xymatrix{
		&A&\\
		D(j)\ar[rr]^{D(\alpha)} \ar[ur]^{\lambda_{j }}&&D(j')\ar[ul]_{\lambda_{j'}}
	}
\]

称呼一个余锥形$(A,\lambda)$是$J$-图的一个{余极限},如果对于任意的余锥形$(B,\mu)$,都有唯一的态射$f:A\to B$使得如下分解$\mu_j:D(j)\xrightarrow{\lambda_j}A\xrightarrow{f}B$成立。一般将$J$-图$D$的余极限写作$\varinjlim_{j\in J} D(j)$.

同样,作为例子,模范畴的直和就是一种余极限,他对应的小范畴,两个不同的元素之间没有态射。

\para 设$I$是一个偏序集,偏序关系为$\leq$,那么族$\{\{x\}\,:\, x\in I\}$构成一个(小)范畴,其中的态射定义为
\[
	\Hom_{I}\left(x,y\right)=\begin{cases}
	(x,y)&\text{, if }x\leq y\text{;}\\
	\varnothing&\text{, otherwise}.
	\end{cases}
\]

如果我们定义$x\geq y$当且仅当$y\leq x$,那么新的偏序集$(I,\geq)$就是偏序集$(I,\leq)$的对偶范畴,略去偏序符号,有时候会记作$I^{\mathrm{op}}$.

$D:I\to \mathcal{C}$是一个协变函子,就是对$i\leq j\leq k$成立$D(j\leq k)\circ D(i\leq j)=D(i\leq k)$,就是说从小到大有映射。反之,反变函子$D:I\to \mathcal{C}$对$i\leq j\leq k$成立$D(i\leq j)\circ D(j\leq k)=D(i\leq k)$,就是说从大到小有映射。

对于一个反变函子$D:I\to \mathcal{C}$,我们可以定义一个协变函子$D^{\mathrm{op}} :I^{\mathrm{op}}\to \mathcal{C}$通过$D^{\mathrm{op}}(j\geq i)=D(i\leq j)$,此时对$k\geq j\geq i$成立$D^{\mathrm{op}}(j\geq i)\circ D^{\mathrm{op}}(k\geq j)=D^{\mathrm{op}}(k\geq i)$,可见这确实是一个协变函子。

\para 称$I$是一个滤相的偏序集,即对于任意的$i$, $j\in I$都存在$k\in I$使得$k\leq i$和$k\leq j$同时成立。称$I$是一个定向的偏序集,即对于任意的$i$, $j\in I$都存在$k\in I$使得$i\leq k$和$j\leq k$同时成立。

很容易看到,如果偏序集$I$是滤相的(定向的),那么$I^{\mathrm{op}}$就是定向的(滤相的),反之亦然。

作为例子,考虑拓扑空间中所有非空开集按照包含构成的偏序集,即$U\leq V$当且仅当$U\subset V$,那么这个偏序集是定向的,但不是滤相的,因为对于任意的$U$和$V$总有$U\leq U\cup V$和$V\leq U\cup V$成立,但对于不交的$U$和$V$,并不存在非空开集同时包含于他们其中。

同样是拓扑的例子,考虑所有包含$p$的非空开集按照包含构成的偏序集,那么这个偏序集既是滤相的又是定向的。

\para 将滤相的(定向的)偏序集$I$看作一个范畴,对任意的范畴$\mathcal{C}$,如果一个$I$-图$D$,$D:I\to \mathcal{C}$是协变函子,则这称为一个$C$上的一个{逆系统}({定向系统})。逆系统一般谈论极限$\varprojlim_{i\in I} D(i)$,而定向系统一般谈论$\varinjlim_{i\in I} D(i)$,为了思考这个原因,我们考虑如下交换图
\[
	\xymatrix{
		&A \ar[dl]_{\lambda_{i}}\ar[dr]^{\lambda_{j}}\ar[dd]^{\lambda_{k}}&\\
		D(i)&&D(j)\\
		&D(k)\ar[ul]^{D(k\leq i)}\ar[ur]_{D(k\leq j)}&
	}
	\quad
% \xymatrix{
% 	&A \ar[dl]_{\lambda_{i}}\ar[dr]^{\lambda_{j}}\ar[dd]^{\lambda_{k}}&\\
% 	D(i)\ar[dr]_{D(i\leq k)}&&D(j)\ar[dl]^{D(j\leq k)}\\
% 	&D(k)&
% }
	\xymatrix{
		&A &\\
		\ar[ur]^{\mu_{i}}D(i)\ar[dr]_{D(i\leq k)}&&D(j)\ar[ul]_{\mu_{j}}\ar[dl]^{D(j\leq k)}\\
		&D(k)\ar[uu]_{\mu_{k}}&
	}
\]
左边是对逆系统考虑锥形,右边是对定向系统考虑余锥形。从左边来看,如果$i\leq j\leq k$成立,则$D(k)$构成了一个锥形的顶点,当我们考虑极限的时候,这时候极限就应该表现得像那些“极小”的元素$D(k)$一样。而且如果$A$是极限,那么箭头$A\to D(k)$也构成了唯一分解。类似地考虑右边的图,那些“极大”的元素$D(k)$构成了余锥形的顶点,如果$A$是余极限,那么箭头$D(k)\to A$也构成了唯一分解。

\para 考虑$I$是滤相的(定向的),而$D$是协变函子,那么$I$-图$D$是逆系统(定向系统)。考虑$I$是滤相的(定向的),而$D$是反变函子,那么$I^{\mathrm{op}}$-图$D^{\mathrm{op}}$是定向系统(逆系统)。

\para 设$I$是一个偏序集,如果$i\in I$有,$i>j$对所有$j\in I$不成立,或者说,与可以比较的元素$j\in I$都有$i\leq j$,则称$i$是$I$的一个极小元。如果$i\leq j$对所有$j\in I$都成立,则称$i\in I$是$I$的最小元。

在滤相的偏序集中,极小元和最小元等价。最小元是极小的,这是显然的。反之,因为如果存在极小元$i$,那么任取一个$j\in I$都存在一个$k\in I$使得$k\leq i$和$k\leq j$都成立,然而$i$是极小元,所以$i=k$,所以$i\leq j$.

同理我们可以定义极大元与最大元,在定向的偏序集中,此二者等价。

更广义地,对于一个偏序集$I$的子集$J$,如果对于任意的$i\in I$,都存在$j\in J$使得$j\leq i$,则称$J$和$I$是共尾的。显然,滤相的偏序集中的极小元构成的单点集就和原来的偏序集共尾。

\pro 设$I$-图$D$是一个逆系统,如果$I$存在一个与其共尾的子集$J$,则$J$-图$D$是一个逆系统,且$\varprojlim_{i\in I} D(i)=\varprojlim_{i\in J} D(i)$. 对偶地,设$I$-图$D$是一个定向系统,如果$I^{\mathrm{op}} $存在一个与其共尾的子集$J^{\mathrm{op}} $,则$J$-图$D$是一个定向系统,且$\varinjlim_{i\in I} D(i)=\varinjlim_{i\in J} D(i)$. \rule{2mm}{2mm}

所以对于逆系统,如果存在极小元,那么极小元就是它的极限,对偶地,对于定向系统,如果存在极大元,那么极大元就是它的对偶极限。这符合我们上面的直观。

\para 一个拓扑空间$X$能被看成一个范畴,对象取作他的所有开集,而态射取作
\[
	\Hom_{X}(U,V)=\begin{cases}
	\bigl\{i^U_V:U\hookrightarrow V\bigr\}&\text{, if }U\subset V\text{;}\\
	\varnothing&\text{, otherwise}.
	\end{cases}
\]

考虑拓扑空间$X$中一个非空开集族$\mathfrak{B}$,对于任意的两个$U$, $V\in \mathfrak{B}$,都存在一个$W\in \mathfrak{B}$使得$U\cup V\subset W$成立,这样的$\mathfrak{B}$按照包含构成了定向的偏序集。现在定义函子$i:\mathfrak{B}\to X$通过$i(U)=U$以及$i(U\leq V)=i^U_V:U\hookrightarrow V$,它显然成立复合$i(U\leq W)=i(V\leq W)\circ i(U\leq V)$,所以这个$\mathfrak{B}$-图$i$是一个定向系统。可以看到它的余极限$\varinjlim_{U\in \mathfrak{B}} U$实际上就是$\bigcup_{U\in \mathfrak{B}} U$,而态射族就是$i_U:U\hookrightarrow \bigcup_{U\in I}$.

同样考虑拓扑空间$X$中一个包含一个点$p\in X$的所有非空开集构成的族$\mathfrak{B}$,它按照包含构成一个滤相的偏序集。同样可以定义函子$i:\mathfrak{B}\to X$通过$i(U)=U$以及$i(U\leq V)=i^U_V:U\hookrightarrow V$,同样显然成立复合$i(U\leq W)=i(V\leq W)\circ i(U\leq V)$,所以这个$\mathfrak{B}$-图$i$是一个逆系统。

但它的极限$\varprojlim_{U\in \mathfrak{B}} U$就不一定存在,因为类比于上一个例子,它的极限应该类似于所有$\mathfrak{B}$中元素的交,但这不一定是一个开集。

\para 设我们有一个$X$上的$\mathcal{K}$-预层$\calf$,他是$X\to \mathcal{K}$的一个反变函子。赋予$X$一个偏序,此时函数$U\leq V$即$U\hookrightarrow V$. 那么$X$的那些包含$p\in X$的非空开集构成的子集族$\mathfrak{B}$继承了偏序结构,也是一个范畴,而预层$\calf$限制在$\mathfrak{B}$给出了反变函子$\mathfrak{B}\to \mathcal{K}$.

由于$\mathfrak{B}$是滤相的偏序集,而$\calf$是反变函子,因此$\mathfrak{B}^{\mathrm{op}}$-图$\calf^{\mathrm{op}}$是定向系统,进而我们会考虑余极限
\[
	\varinjlim_{U\in \mathfrak{B}^{\mathrm{op}}} \calf^{\mathrm{op}}(U)=\varinjlim_{U\in \mathfrak{B}^{\mathrm{op}}} \calf(U),
\]
如果这个余极限存在,那么就称为预层$\calf$在点$p$处的纤维,记作$\calf_p$.

\section{自由对象}

简单来说,自由对象是遗忘函子的伴随函子。遗忘函子以前我们已经谈过,他将结构比较多的对象变成了结构比较少的对象,而自由对象就是反过来,从一个结构比较少的对象来构造一个结构比较多的对象。

\para 

\section{多项式环}

多项式环的重要性无需赘述。

\para 设$R$是一个环,而$a=(a_0,a_1,\cdots,a_n,\cdots)$是一列$R$中的元素,记只有有限个元素非零的$a$构成的集合为$R[x]$. 在$R[x]$上,我们可以定义加法
\[
	a+b=(a_0+b_0,a_1+b_1,\cdots),
\]
由于$a+b$也是只有有限个元素非零,所以$a+b \in R[x]$. 对于加法$0=(0,0,\cdots)$显然是零元,而$a$的逆元$-a=(-a_0,-a_1,\cdots)$.

然后是乘法,定义
\[
	(ab)_i=\sum_{k+j=i}a_kb_j,
\]
可以看出$ab$也只有有限个元素非零,所以$ab\in R[x]$.

分配律的检验是直接的,
\[
	((a+b)c)_i=\sum_{k+j=i}(a+b)_kc_j=\sum_{k+j=i}a_kc_j+\sum_{k+j=i}b_kc_j=(ac)_i+(bc)_i,
\]
所有的一切都可以直接检验。因此$R[x]$确实是一个环,称为多项式环。

一般而言,我们将$R[x]$中的元素$f=(a_0,a_1,\cdots)$写成
\[
	f=\sum_{n}a_n x^n
\]
的形式,这就是我们熟悉的多项式,一切的运算都和我们熟悉的那样。$x$称之为不定元,不定元只是一个符号,在不发生歧义的情况下可以任意选取。

\para 定义多元多项式环$R[x_1,\cdots,x_n]$为$R[x_1,\cdots,x_{n-1}][x_n]$,这是一个递归定义。

\para 一个多项式$f=(a_0,a_1,\cdots)$中的最高项即使得$a_n$不为零的最大的$n$,这个$n$记作$\deg(f)$,称为多项式$f$的幂次。最高幂次的系数我们称之为最高次系数,或者首项系数。其余的项目通常我们会指出具体的指标,比如说$k$-次项,就是指$a_kx^k$,而$k$-此项系数就是指$a_k$. 如果最高次系数为$1$,这个多项式称为首一多项式。

对于首一多项式而言$f$,$\deg(fg)=\deg(f)+\deg(g)$. 但是一般的多项式并不如此,比如在$\zz/4\zz$上的多项式$2x$,有$(2x)(2x)=0$.

对于$\deg(f+g)$,我们有估计$\deg(f+g)\leq \max\{\deg(f),\deg(g)\}$,不等号是可以取严格的,同样比如在$\zz/4\zz$上的多项式$2x$,有$2x+2x=0$.

\theo 多项式除法算法:设$f$, $g\in R[x]$,而且$g$的首项系数可逆,则存在唯一的多项式$p$, $r\in R[x]$使得$f=pg+r$,且$\deg(r)< \deg(f)$.

\proof
	可以假设$g$是首一多项式,因为如果设$g$的首项系数为$a$,则$g/a$是一个首一多项式,如果命题对首一多项式成立,即存在$q$, $r\in R[x]$使得$f=q(g/a)+r$,则$f=(q/a)g+r=pg+r$,其中$p=q/a$和$r$都是多项式。这样就得到了我们的命题。

	首先假设存在,证明唯一性。设$f=p'g+r'$以及$f=pg+r$成立,则$(p'-p)g=r'-r$,由于$g$是首一的,且$p'-p$非零,所以$\deg((p'-p)g)\geq \deg(g)$,但是$\deg(r'-r)< \deg(g)$,这就造成了矛盾。下面证明存在性。

	设$\deg(f)=n$, $\deg(g)=m$,如果$n<m$,则取$p=0$, $r=f$. 考虑$n\leq m$的情况,设$f$的首项系数为$a_0$,考虑多项式$f_1=f-a_0x^{n-m}g$,由于$f$和$a_0x^{n-m}g$的最高次项相同,所以$\deg(f_1)<\deg(f)$,如果$\deg(f_1)<\deg(g)$,那么$f=a_0x^{n-m}g+f_1$就给出了分解。

	否则继续对$f_1$进行这样的操作,得到$f_2=f_1-a_1x^{\deg(f_1)-m}g$,再比较$\deg(f_2)$与$\deg(g)$. 不断如是进行下去,由于$f$的幂次有限,而每次操作,幂次都至少减一,所以该过程在进行至多$n-m+1$次后就会停止。设该过程在第$k$次后停止,则我们就得到了
	\[
	f_{k}=f-a_0x^{n-m}g-a_1x^{\deg(f_1)-m}g-\cdots-a_{k-1}x^{\deg(f_{k-1})-m}g
	\]
	使得$\deg(f_k)< \deg(g)$. 即$f=f_0$,则我们就得到了分解
	\[
	f=f_k+a_0x^{n-m}g+a_1x^{\deg(f_1)-m}g+\cdots+a_{k-1}x^{\deg(f_{k-1})-m}g=\left(\sum_{i=0}^{k-1}a_{i}x^{\deg(f_{i})-m}\right)g+f_k.
	\]
	因此$p=\sum_{i=0}^{k-1}a_{i}x^{\deg(f_{i})-m}$以及$r=f_k$就是我们需要的多项式。
\qed

存在性的证明就是整个算法,从算法来看,这个命题即使是对非交换环上的多项式环也是成立的。

\section{理想}

\para 理想是环$R$看成$R$-模时候的子模。一个理想被集合$S\subset R$生成是指在$R$-模意义上生成的子模,即$S$中元素的任意有限线性组合,我们将其记作$(S)$,当$S$是单元素集的时候,这就是主理想,当$S$是有限集的时候,时常就写成$(S)=(f_1,\,\cdots,\,f_n)$. 一个理想生成的理想自然就是他本身。

\para 在不同的理想之间可以定义运算,设$\mathfrak{a}$和$\mathfrak{b}$是$R$的两个理想,则
\[\mathfrak{a}+\mathfrak{b}=\{a+b\,:\,a\in\mathfrak{a},\,b\in\mathfrak{b}\}
\]
是一个理想。

两个理想的交显然还是一个理想(这可以类比两个子群的交还是子群),两个理想$\mathfrak{a}$和$\mathfrak{b}$的乘积$\mathfrak{a}\mathfrak{b}$被定义为集合$\{ab\,:\,a\in\mathfrak{a},\,b\in\mathfrak{b}\}$生成的理想。

\para 称一个环$R$是一个整环,如果任取非零的$a$, $b\in R$,可以推出$ab\neq 0$. 一般而言,一个环不是整环。如果$ab=0$但$a$和$b$都不等于零,则这样的$a$或者$b$被称为一个零因子。零显然是一个零因子。而整环就是一个没有非零零因子的环。换而言之,在整环里面,消去律是成立的。

考虑一个单同态$f:R\to S$,如果$S$是一个整环,则$R$也是一个整环。为了证明他,考虑到$ab=0$等价于$f(ab)=f(a)f(b)=0$,由于$S$是整环,所以$f(a)=0$或者$f(b)=0$,再由单同态就得到了结论。作为推论,整环的子环也一定是整环。

整环上的多项式环也是整环:设多项式$f$以及$g$都不为零,则它们的首项系数$a$与$b$不为零,因此$fg$的首项系数$ab$不为零,也就是说$fg\neq 0$.

\para 一个理想$\mathfrak{p}$被称为素理想,如果$R/\mathfrak{p}$是一个整环。一个理想$\mathfrak{m}$被称为极大理想,如果$R/\mathfrak{p}$是一个域。由于域是整环,所以极大理想是素理想,反之不然。由于$R=R/(0)$,所以只有在整环里面,零理想才是素理想。

\para 设$R$是一个环,$I(R)$是$R$所有真理想的集合,上面按照包含构成了一个偏序,即$\mathfrak{a}\leq \mathfrak{b}$当且仅当$\mathfrak{a}\subset \mathfrak{b}$.

极大理想的命名就来自于这个偏序。对于任意的理想链,他们的并所生成的理想要比他们大。并且由于$R$肯定有一个理想$(0)$,所以由Zorn引理,$R$中存在在上述偏序下极大的理想,下面检验这就是上面说的极大理想。设$\mathfrak{m}$是这样一个理想,$a\notin \mathfrak{m}$,则$(a)+\mathfrak{m}$是一个严格比$\mathfrak{m}$大的理想,由于$\mathfrak{m}$的极大性,没有比他大的真理想了,所以$(a)+\mathfrak{m}=R$. 因此,$1$可以写成$ra+m=1$的形式,在$R/\mathfrak{m}$中即$\bar{r}\bar{a}=1$,所以$\bar{a}$有逆。又因为$a$是在$R-\mathfrak{m}$中任取的,所以$R/\mathfrak{m}$是一个域。

反过来,如果$R/\mathfrak{m}$是一个域,则不存在$\mathfrak{m}$更大的真理想。假设如果存在$\mathfrak{a}$比$\mathfrak{m}$严格大,则有自然的商同态$\pi:R/\mathfrak{m}\to R/\mathfrak{a}$,由于$R/\mathfrak{m}$是一个域且$\pi$是满射,则$\ker \pi$作为域的理想只能是零理想,这样也就推出了$\pi$是一个同构,这与$\mathfrak{a}$比$\mathfrak{m}$严格大矛盾。

从可操作性来看,一开始的定义比这个极大理想的等价定义要方便不少。

\para 我们看到,极大理想是素理想。实际上,满足一些条件的极大的理想也会是素理想。
\begin{itemize}
\item 设$\mathcal{P}$是$R$的非有限生成理想构成的集合,则$\mathcal{P}$的极大元是素理想。
\item 设$\mathcal{P}$是$R$的非主理想的理想构成的集合,则$\mathcal{P}$的极大元是素理想。
\item 设$S$是一个乘性子集,设$\mathcal{P}$是$R$中与$S$不相交的理想构成的集合,则$\mathcal{P}$的极大元是素理想。
\end{itemize}
\proof

\qed

\pro 设$R$是一个环,而$I_1$, $\cdots$, $I_n$是一族理想,还有一个理想$J$满足$J\subset \bigcup_i I_i$. 如果$I_1$, $\cdots$, $I_n$中至多只有两个不是素理想,则存在一个$i$使得$J\subset I_i$. 如果将包含改成等号,命题依然成立。\notprove

\para 设$f:R\to S$是一个环同态,如果$\mathfrak{p}$是$S$中的一个(素)理想,则$f^{-1}(\mathfrak{p})$是一个(素)理想。

\proof
	任取$a\in f^{-1}(\mathfrak{p})$以及$r\in R$,由于$f(a)\in \mathfrak{p}$,所以$f(r)f(a)=f(ra)\in \mathfrak{p}$,这也就推出了$ra\in f^{-1}(\mathfrak{p})$. 所以$f^{-1}(\mathfrak{p})$是一个理想。

	设$\pi:S\to S/\mathfrak{p}$是商同态,我们考虑复合映射$\pi\circ f:R\to S/\mathfrak{p}$,由于$f^{-1}(\mathfrak{p})\subset \ker(\pi\circ f)$,所以由商环的泛性质,$\pi\circ f$诱导出了单同态\[R/f^{-1}(\mathfrak{p})\to S/\mathfrak{p},\]
	单性从这里看出:如果$f(r_1)-f(r_2)=\mathfrak{p}$,则$r_1-r_2\in f^{-1}(\mathfrak{p})$. 

	当$\mathfrak{p}$是一个素理想的时候,$S/\mathfrak{p}$是整环,单同态$R/f^{-1}(\mathfrak{p})\to S/\mathfrak{p}$告诉我们$R/f^{-1}(\mathfrak{p})$也是整环,所以$f^{-1}(\mathfrak{p})$也是素理想。
\qed

\para 上面看到了理想的原像一定是一个理想,反过来,一般来说,一个理想的像不一定是一个理想。比如含入同态$\mathbb{Z}\hookrightarrow \mathbb{Q}$下,理想$(2)$的像不是理想。

但是,对于商映射,情况会好很多。设$\pi:R\to R/\mathfrak{a}$是一个商映射,而$\mathfrak{b}$是$R$中的一个理想,则$\bar{\mathfrak{b}}=\pi(\mathfrak{b})$是$R/\mathfrak{a}$中的一个理想。如果$\mathfrak{p}$是包含$\mathfrak{a}$的素理想,则$\bar{\mathfrak{p}}$也是一个素理想。

证明是朴实的,任取$a\in \mathfrak{b}$,以及$r \in R$,由于$ra\in \mathfrak{b}$,我们也就推出了$\bar{r}\bar{a}=\overline{ra}\in \bar{\mathfrak{b}}$. 所以我们可以考虑这样的商映射$\psi: R/\mathfrak{a}\to (R/\mathfrak{a})/\bar{\mathfrak{b}}$,他与商映射$\pi$复合可以得到满同态
\[
	\psi\pi:R\to (R/\mathfrak{a})/\bar{\mathfrak{b}},
\]
注意到$\psi\pi(r)=0$当且仅当$\bar{r}\in \bar{\mathfrak{b}}$,所以$\ker(\psi\pi)=\pi^{-1}(\bar{\mathfrak{b}})=\mathfrak{a}+\mathfrak{b}$. 由同构基本定理,我们有同构
\[
	R/(\mathfrak{a}+\mathfrak{b})\cong (R/\mathfrak{a})/\bar{\mathfrak{b}}.
\]

\para 利用上面这个观察,我们可以对商映射下的理想做出如下断言:$R/\mathfrak{a}$中的(素)理想一一对应着包含$\mathfrak{a}$的(素)理想,通过$\bar{\mathfrak{b}}\to \pi^{-1}(\bar{\mathfrak{b}})$.

\proof 
	由于$\pi$是一个满射,所以有等式$\pi(\pi^{-1}(\bar{\mathfrak{b}}))=\bar{\mathfrak{b}}$. 剩下我们要证明,如果$\mathfrak{b}\supset \mathfrak{a}$,则$\pi^{-1}(\pi(\mathfrak{b}))=\mathfrak{b}$,而这来自于$\pi^{-1}(\pi(\mathfrak{b}))=\mathfrak{a}+\mathfrak{b}$. 如果$\mathfrak{p}$是包含$\mathfrak{a}$的素理想,则$\mathfrak{a}+\mathfrak{p}=\mathfrak{p}$,上述同构写成$R/\mathfrak{p}\cong (R/\mathfrak{a})/\bar{\mathfrak{p}}$,因此$\bar{\mathfrak{p}}$也是素理想。
\qed

\para 任取环同态$f:R\to S$,我们可以做出如下分解$f:R\to f(R)\hookrightarrow S$,其中满同态$R\to f(R)$的结构我们是清楚的,因为我们可以利用同构$f(R)\cong R/\ker(f)$将它变成商同态$R\to R/\ker(f)$的情况。所以一般而言,含入同态才是造成理想的像不是理想的障碍,正如前面我们举的例子,含入同态$\mathbb{Z}\hookrightarrow \mathbb{Q}$下,理想$(2)$的像不是理想。

\para 定义一个理想$\mathfrak{a}$的根$\sqrt{\mathfrak{a}}$如下:
\[
	\sqrt{\mathfrak{a}}=\{r\in R\,:\,\exists n\in \mathbb{Z}^+\text{ s.t. }r^n\in \mathfrak{a}\}.
\]

一个理想的根依然是一个理想,检查中困难的是加法,设$a^n\in \aaa$和$b^m\in \aaa$,则$(a+b)^{m+n}$在二项式展开后可以发现,每一项都属于$\aaa$,所以$a+b\in \sqrt{\aaa}$.

一个理想的根其实是所有包含它的素理想的交。实际上,设$A$是所有包含$\mathfrak{a}$的素理想的交,设$\pp$是任意一个包含$\mathfrak{a}$的素理想,如果$f^n\in \mathfrak{a}$,则$f^n\in \pp$,由于$\pp$是素理想,所以$f^n=f\cdot f^{n-1}$给出$f\in\pp$或者$f^{n-1}\in\pp$,通过归纳法就有$f\in \pp$. 这就给出了$\sqrt{\mathfrak{a}}\subset A$. 反过来,如果$f\not\in \sqrt{a}$,考虑所有与$\{1,f,f^2,\cdots\}$不交的,但包含$\mathfrak{a}$的所有理想中极大的那个理想$\pp$,这是一个素理想,所以$f\not\in \pp$. 这就给出了$A\subset \sqrt{\mathfrak{a}}$.