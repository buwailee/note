\chapter{基础}
\ThisULCornerWallPaper{1}{../Pictures/8.png}

\section{代数结构}

这节就是罗列定义。

\para 设有两个集合$X$和$Y$,则称映射$f:X\times Y \to X$为集合$X$上的一个右\idx{作用}。称映射$g:Y\times X \to X$为集合$X$上的一个左作用。称映射$h:X\times X \to X$为集合$X$上的一个二元\idx{运算}。同样地,我们可以定义多元运算。

可以看到$\mathbb{R}$上的加法$f:\mathbb{R}\times \mathbb{R} \to \mathbb{R}$定义为$f(a,b)=a+b$是一个二元运算,同样可以检验乘法。再比如设$Y$是集合$X$上所有双射$f:X\to X$的集合,那么映射复合构成$Y$上的一个运算。

对于一个未知的运算或者作用,我们通常称之为“乘法”。左作用称为“左乘”,右作用称为“右乘”。对于运算$f(a,b)$通常直接记作$a*b$,或者再干脆一些省略中间的符号$*$,记作$ab$,读作$a$左乘$b$、$b$右乘$a$或$a$乘以$b$.

定义多个元素$\{a_i\,:\, 1\leq i \leq n\}$的乘法为
\[
	a_1a_2\cdots a_n=a_1(a_2a_3\cdots a_n)=a_1(a_2(a_3\cdots a_n))=\cdots,
\]
这是一个递归定义。

\para 设集合$X$上有一个运算$h:X\times X \to X$,将$h(a,b)$直接记作$ab$. 如果任取$a$, $b$, $c\in X$都成立
\[
	(ab)c=a(bc),
\]
则该运算被称为是满足\idx{结合律}(\idx{associative property})的。

\para 对于在字符串$a_1a_2\cdots a_n$中任意加括号构成的字符串,比如
\[
(a_1(a_2(a_3a_4)a_5))(a_6(a_7a_8a_9))a_{10},
\]
先预设一个$n=0$,我们从左往右开始计数,遇到`$($'则$n\to n+1$,并称该括号为第$n$层括号,遇到`$)$'则$n\to n-1$,直到最后一个字符。比如上式中,第一个和第四个`$($'都是第一层括号,而第二个和第五个`$($'是第二层括号。如果到达最右端字符的时候,得到了$n=0$,则上面的字符串称为$a_1a_2\cdots a_n$的一种加括号的方式。

\pro 设$\{a_i\in A\,:\, 1\leq i\leq n\}$是$A$中任意的一个有限集,且$A$上存在一种运算满足结合律。对于任意字符串$a_1a_2\cdots a_n$中任意加括号的方式,经过运算后,他们都等于$a_1a_2\cdots a_n$.

\proof 对正整数$n$,记$[n]=\{1,2,\cdots,n\}$。设$f:[m]\to [n]$是一个递增函数,则$f$被称为$[n]$的一个$m$-划分。由于是递增函数,所以$m\leq n$. 

考虑$\{a_i\, :\, 1\leq i\leq n\}$,对于$[n]$的任意一个$(2m)$-划分$k$,$k(i)=k_i$,我们都可以通过在在每一对$a_{k_{2i-1}}$和$a_{k_{2i}}$之间加一对括号定义出一个字符串
\[
	a_1a_2\cdots (a_{k_1}\cdots a_{k_2})a_{k_2+1}\cdots (a_{k_3}\cdots a_{k_4})\cdots (a_{k_i}\cdots a_{k_{i+1}})\cdots a_n,
\]
一个$m$对括号,这个字符串的最大层数是$1$。现在,当这个字符串看作元素之间相乘的时候,可以证明此时他等于$a_1a_2\cdots a_n$.

为此先证明如下特殊情况:对任意的正整数$n$和$k$以及元素$\{a_i\,:\, 1\leq i \leq n\}$成立
\[(a_1a_2\cdots a_{k})(a_{k+1}\cdots a_{n})=a_1a_2\cdots a_n.\]
首先注意到
\[
(a_1a_2\cdots a_{k})(a_{k+1}\cdots a_{n})=(a_1(a_2\cdots a_{k}))(a_{k+1}\cdots a_{n})=a_1((a_2\cdots a_{k})(a_{k+1}\cdots a_{n})),
\]
第二个等号来自于结合律。然后对$(a_2\cdots a_{k})(a_{k+1}\cdots a_{n})$进行同样的操作,如是归纳下去,由于$n$是有限的,归纳也是有限的,我们就可以得到
\[
(a_1a_2\cdots a_{k})(a_{k+1}\cdots a_{n})=a_1(a_2(a_3(\cdots (a_{k}(a_{k+1}\cdots a_{n})))))=a_1a_1a_2\cdots a_n.
\]

现在,假设有$m$对括号,多重乘积可能出现两种情况:一是$a_1a_2\cdots a_{k-1}(a_{k}\cdots a_n)$,即最后一对括号出现在最后,那么按照定义,他就等于$a_1a_2\cdots a_{k-1}a_{k}\cdots a_n$,这样就消去了最后一对括号。二是$a_1a_2\cdots$ $a_{k-1}(a_{k}\cdots a_l)$ $a_{l+1}\cdots a_n$,其中$a_1a_2\cdots a_{k-1}$中有$(m-1)$对括号,或者说$(a_{k}\cdots a_l)$是最后一对括号。利用定义和上面的结论可以得到
\begin{align*}
	a_1a_2\cdots a_{k-1}(a_{k}\cdots a_l)a_{l+1}\cdots a_n&=a_1a_2\cdots ((a_{k}\cdots a_l)(a_{l+1}\cdots a_n))\\
	&=a_1a_2\cdots a_{k-1}(a_{k}\cdots a_n)\\
	&=a_1a_2\cdots a_{k-1}a_{k}\cdots a_n.
\end{align*}
这样我们就消去了最后一对括号。如是往复,对$m$进行有限归纳可以消去全部的括号,得到$a_1\cdots a_n$.

我们已经对任意字符串$a_1a_2\cdots a_n$最大括号层数为$1$的所有可能加括号方式证明了,在满足结合律的时候等于$a_1a_2\cdots a_n$.

先设最大的层数为$N$,对所有的$(N-1)$层括号,他里面的最大括号层数为1,利用上面的结论可以将这层内所有的括号消去,则最大层数降为$(N-1)$,如是往复,经过有限次归纳,就得到了他最终将等于$a_1a_2\cdots a_n$.\qed 

这个结论就是说,对于满足结合律的运算,有限个元素相乘的结果不依赖于他加括号的方式。

\para 有一个非空集合$G$和其上的二元运算$*$,或者记作$(G,*)$,称为一个\idx{群}(\idx{group}),如果满足:

\no{1}结合律:对于任意$a$, $b$, $c\in G$,有$(a*b)*c=a*(b*c)$;

\no{2}单位元:对于任意的$a\in G$,存在一个元素$e\in G$,使得$e*a=a*e=a$;

\no{3}反元素:对于每一个$a\in G$,存在一个元素$b\in G$,使得$b*a=a*b=e$.

$a$的反元素通常记作$a^{-1}$.此外如果不产生歧义,我们可以直接称呼$G$为一个群。显然$(\mathbb{R},+)$是一个群,单位元是0,$a^{-1}=-a$.将$\mathbb{R}$去掉了$0$之后的集合记作$\mathbb{R}-\{0\}$,则$(\mathbb{R}-\{0\},\cdot)$构成一个群,单位元是$1$,$a^{-1}=1/a$.

如果有群$(G,*)$,对于任意的$a$, $b\in G$满足$a*b=b*a$,则称$(G,*)$是一个\textit{交换群}\index{群!交换群},或者称为\idxx{群}{Abel群}。交换群的运算一般是记作加法的。

\para 我们称呼一个三元组$(R,+,\cdot)$为一个\idx{环}(\idx{ring}),如果满足下列性质:

\no{1} $(R,+)$构成一个交换群,运算称为加法,其中的单位元记作$0$,$a$的反元素记作$-a$;

\no{2} $(R,\cdot)$中的运算满足结合律,称为乘法。并且,在$R-\{0\}$中含有乘法的单位元,记作$1$. 如果$a$关于$1$有逆,则逆写作$a^{-1}$.

\no{3} 乘法满足分配律:对任意$a$, $b$, $c \in R$,成立$a\cdot(b+c)=a\cdot b+a\cdot c$以及$(b+c)\cdot a=b\cdot a+c\cdot a$.

同样地,如果不产生歧义,我们可以直接称呼$R$为一个环,而且略去乘法的符号。任取$a\in R$,由于
\[
	0a=(0+0)a=0a+0a,
\]
所以$0a=0$,同理$a0=0$. 如果在$R$中有$1=0$,则对任意的$r\in R$成立$r=1r=0r=0$. 这样的环称为零环,零环的地位在集合里面大概就类似空集。下面我们所讨论的环,一般而言不会是零环,即我们会假设$0\neq 1$.

如果$R$中的乘法是可交换的,则称$R$是一个\idxx{环}{交换环}。有些作者在环的定义中会去掉乘法单位元,称我们这里定义的环为含幺环。

和群不同的是,环乘法不一定是可逆的(即对$a$存在一个$b$使得$ab=1$),比如$0$,对于任意的$a\in R$都有$a0=0\neq 1$,所以$0$一定不是可逆的。如果一个环除了零之外的元素全部可逆,则称他为一个除环。交换除环被称为\idx{域}(\idx{field})。

在后面,除非特殊申明,环全部假设为交换含幺环。

\para 设$(M,+)$是一个交换群,$R$是一个环,如果有一个左乘$\mu:R\times M\to M$,左乘符号下面省略,使得对任意的$r$, $s\in R$以及$m$, $n\in M$,成立
\[
	(rs)m=r(sm),\quad r(m+n)=rm+rn,\quad (r+s)m=rm+sm,\quad 1m=m,
\]
则称呼$M$是一个左$R$-\idx{模},同理可以定义右$R$-模。任意的含幺环$R$显然是一个$R$-模。

由于
\[
	0m=(0+0)m=0m+0m,
\]
所以$0m=0$,注意等号两边的零是不同的。类似地,可以证明$r0=0$. 

如果$R$是交换环,则我们可以不区分左$R$-模和右$R$-模,只要等同$rm$和$mr$即可。我们后面谈论的基本都是这种模。

\para $\zz$显然是一个交换群,当然也是一个交换环。不太平凡的一点是,任意的交换群$G$可以看成$\zz$-模。这是因为对任意的$n\in \zz$和任意的$a\in G$,我们做出如下定义:如果$n>0$,定义$na$为$n$个$a$相加,如果$n=0$,定义$na=0$,如果$n<0$,定义$na=-((-n)a)$. 不难检验这是确实是一个$\zz$-模。

除了交换群之外,还有一类重要的模,\idx{矢量空间}。设$k$是一个域,则任意$k$-模$V$被称为$k$-矢量空间,$V$中的元素被称为矢量。

\para 设$\mu:H\times G\to G$为一个$G$上的左作用(一般是乘法),$K$是$G$的一个子集,记
\[
	HK=\{ab\,:\,a\in H,b\in K\}\subset G.
\]
特别地,当$H$为单点集$\{f\}$的时候,$\{f\}K$通常缩略写作$fK$. 或者当$K$是单点集$\{a\}$的时候,也缩略写作$Ha$.

设$H'\subset H$,以及$K'\subset K$,则有显然的包含关系
\[
	H'K\subset HK,\quad HK'\subset HK.
\]

% 对于环$R$的两个子集$H$和$K$,他们的加法就是上述左作用的一个特例
% \[
% 	H+K=\{a+b\,:\,a\in H,b\in K\}\subset R.
% \]
% 然后考虑到$R$上也有乘法,所以我们定义
% \[
% 	HK=\left\{\sum_i a_i b_i\,:\,a_i\in H,b_i\in K\right\}\subset R,
% \]
% 其中求和是任意有限求和。

\para 设$G$是一个群,$H$是他的一个子集。群运算$\mu$限制在$H\times H$得到了运算$\mu|_{H\times H}:H\times H\to G$. 如果$\im\mu|_{H\times H}\subset H$,则$(H,\mu)$也构成了一个群,他被称为$G$的一个\idxx{群}{子群}。最后一句话就是说,$H$中的任意两个元素相乘依然在$H$中,再或者用上面的写法,$HH\subset H$.

类似地,如果$R$是一个环,$S$是他的一个子集。如果对$R$的交换群结构,$S$是他的子群,且$SS\subset S$,则$S$被称为$R$的一个\idxx{环}{子环}。子环一般而言不用含有单位元。

同样,如果$M$是一个左$R$-模,$N$是他的一个子集。如果对$M$的交换群结构,$N$是他的子群,且$RN\subset N$,则$N$被称为$M$的一个\idxx{模}{子模}。

一个群/环/模的子群/环/模如果作为子集是真子集,则称其为真子群/环/模。

\para 注意到环$R$本身可以看成一个$R$-模,他的子模被称为理想。

显然$\{0\}$和$R$本身是$R$的理想,前者被称为零理想,后者被称为单位理想,这两者往往被称为平凡理想,因为我们感兴趣的理想往往不是他们。

设$a\in R$,则$aR$也是$R$的一个\idx{理想},被称为\idxx{理想}{主理想},有时候也记作$(a)$. 由于$\{0\}=0R$以及$R=1R$,零理想和单位理想都是主理想。主理想$(a)$一般也称作被$a$生成的主理想。

单位理想名字中的单位来自于可逆元的英文unit:考虑一个可逆元$a\in R$,由于存在$a^{-1}$,所以任取$r\in R$,都有$r=a(a^{-1}r)\in (a)$,所以$(a)=R$. 这就是说可逆元生成的主理想都是单位理想。

设$k$是一个域,则他没有非平凡理想:考虑$k$的一个非零理想$\mathfrak{a}$,他包含一个$a\neq 0$,所以$(a)\subset k\mathfrak{a} \subset k$,但是由于$a$可逆,所以$(a)=k\subset k\mathfrak{a} \subset k$,这也就推出了$k\mathfrak{a}=k$,然后由理想的定义$k\mathfrak{a}\subset \mathfrak{a}$就推出了$\mathfrak{a}=k$.

\section{同态与同构基本定理}

\para 设$G$和$H$是两个群,若映射$f:G\to H$满足
\[
	f(e_G)=e_H,\quad f(ab)=f(a)(b),
\]
则称$f$是一个群\idx{同态}(\idx{homomorphism})。可以看到,群同态将单位元映射到单位元,将乘法映成乘法,基本保持了群的结构。如果还有一个群同态$g:H\to G$使得$fg=\id_H$以及$gf=\id_G$,则$f$或者$g$被称为一个群同构,通常会把$g$记作$f^{-1}$. 显然,一个群同构是一个双射,既单又满。如果两个群之间存在群同构,则称呼这两个群是同构的。

后面我们不再区分两个不同群的单位元的符号,基本上属于哪个群从上下文中都是清楚的。

\para 设$R$和$S$是两个环,若映射$f:R\to S$是交换群同态,且$f(rs)=f(r)f(s)$以及$f(1)=1$,则称$f$是一个环同态。类似地,还有环同构。

设$M$和$N$是两个$R$-模,若映射$f:M\to N$是交换群同态,且任取$r\in R$以及$m\in M$成立$f(rm)=rf(m)$,则称$f$是一个$R$-模同态。类似地,还有同构。

\para 任取一个同态$f:A\to B$,这里$A$和$B$可以是群、环、$R$-模,则$f(A)$自然地有一个$B$的子群、子环、子$R$-模的结构,称作同态$f$的像,记作$\im(f)$或者$f(A)$。如果$\im(f)=B$,则这个同态被称为满同态。

对群的情况,考虑原像集$f^{-1}(e)$. 任取$a$, $b\in f^{-1}(e)$,由于$f(ab)=f(a)f(b)=e$,所以$ab\in f^{-1}(e)$. 因此$f^{-1}(e)$是$A$的一个子群,记作$\ker(f)$,称作同态$f$的核。如果$\ker(f)=\{e\}$,则这个同态被称为单同态。

对环的情况,先只考虑他的加法群结构,则$\ker(f)=f^{-1}(0)$. 任取$a\in \ker(f)$以及$r\in A$,由于$f(ra)=f(r)f(a)=0$,所以$ra\in \ker(f)$或者$A\ker(f)\subset \ker(f)$,这就意味着$\ker(f)$是$A$的一个理想。

对$R$-模的情况,先只考虑他的加法群结构,则$\ker(f)=f^{-1}(0)$. 任取$a\in \ker(f)$以及$r\in A$,由于$f(ra)=rf(a)=0$,所以$ra\in \ker(f)$或者$R\ker(f)\subset \ker(f)$,这就意味着$\ker(f)$是$A$的一个子模。

\para 有一个自然的单同态:考虑$A$是$B$的子群/环/$R$-模,则$\id_B|_A:A\to B$就是一个单同态。此时这个映射(同态)被称为含入映射(同态),一般而言我们会略去他的符号,用一个弯曲的箭头写作$A\hookrightarrow B$. 正如前面看到的,设$\rho:A\to C$是一个同态,则他的核是$A$的子群/环/$R$-模,此时有$\ker\rho\hookrightarrow A$.

设$\{e\}$是平凡群,则同态$f:\{e\}\to A$只可能是$f(e)=e$,因此平凡群$\{e\}$可以看成任意群的子群,从他到任意群的同态只有一个同态,即含入同态。而任意同态$g:A\to \{e\}$满足$\ker(g)=A$,他被称为零同态。 类似的,在环中,零环$\{0\}$也提供了同样的作用,在$R$-模中,零模$\{0\}$也类似。

\para 若有一族同态$f_i:A_i\to A_{i+1}$满足$\im f_i=\ker f_{i+1}$,则列
\[
	\cdots \xrightarrow{f_{i-1}}A_i \xrightarrow{f_i} A_{i+1} \xrightarrow{f_{i+1}} A_{i+1}\xrightarrow{f_{i+1}}\cdots
\]
被称为\idx{正合}的。在正合列中,平凡群一般写作$1$,零环和零模则写作$0$,从平凡群出发或者到平凡群的同态记作$1$,零环和零模的则记作$0$,通常略去他们的符号,因为这些同态我们是清楚的。

正合列可以用来表示一个同态是否是单射或者满射。考虑正合列$0\to A\xrightarrow{f} B$,由于$\ker(f)=\im(0\to A)=\{0\}$,所以$f$是一个单射。考虑正合列$A\xrightarrow{f} B\to 0$,由于$\im f=\ker(B\to 0)=B$,所以$f$是一个满射。

在正合列中,短正合列尤其重要,他写作
\[
	0\to A \xrightarrow{f} B \xrightarrow{g} C\to 0,
\]
其中$f$是单射,而$C$是满射。如果有如下正和列
\[
	0\to A \xrightarrow{f} B \to 0,
\]
这就意味着同态$f$即是单射又是满射。下面的命题告诉我们,$f$此时是一个同构。

\pro 一个(群/环/$R$-模)同态是同构当且仅当他既是单的又是满的。

\proof 一个同构肯定既是单的又是满的。反过来,设$f: A\to B$既是单的又是满的,则对$b\in B$,唯一存在$a\in A$使得$f(a)=b$,定义
\[
	\begin{array}{ccccc}
		g &:&B &\to& A,\\
		g &:&b &\mapsto& a.
	\end{array}
\]
下面检验这是一个同态。

对于群而言,设$g(b_1)=a_1$和$g(b_2)=a_2$,则只要检验$g(b_1b_2)=a_1a_2$,由于$f(a_1a_2)=f(a_1)f(a_2)=b_1b_2$以及$f$是一个单射就得出了$g(b_1b_2)=a_1a_2=g(b_1)g(b_2)$. 另外,显然$g(e)=e$,所以$g$是一个同态。

对于环而言,加法群同态部分已经由群的结论得到了,只要检验乘法即可。而乘法部分和群的情况是一模一样的。

对于$R$-模的情况,加法群同态部分已经由群的结论得到了,只要检验标量乘法即可。设$g(b)=a$,以及$g(rb)=a'$. 由于$f(a')=rb$以及$rf(a)=f(ra)=rb$,所以由$f$是一个单射也就得到了$a'=ra$,故$g(rb)=rg(b)$,所以$g$是一个同态。

最后显然$f\circ g =\id_B$以及$g\circ f=\id_A$,所以$f$是一个同构,而他的逆$g$常常记作$f^{-1}$.
\qed

\para 设$f:G\to H$是一个群同态,前面说过$\ker(f)$是$G$的一个子群。我们问反过来的一个问题,什么时候子群可以称为一个同态的核。为此,先看看核有什么性质,设$h\in \ker(f)$,则任取$g\in G$有
\[
	f\left(ghg^{-1}\right)=f(g)f(h)f(g)^{-1}=f(g)ef(g)^{-1}=e,
\]
所以$ghg^{-1}\in \ker(f)$,或者$gh \in \ker(f)g$,再或者$g\ker(f)\subset \ker(f)g$. 然后再考虑$g\to g^{-1}$我们就得到了反向包含,合起来就得到了$g\ker(f)=\ker(f)g$对任意$g\in G$都成立。

我们抽象出这个性质:如果一个$G$的子群$H$满足对任意的$g\in G$都成立$gH=Hg$,则$H$被称为$G$的\idxx{群!子群}{正规子群}。

一般而言,子群不是正规子群。但是,对于交换群,$gh=hg$,所以任意子群都是正规子群。反过来也不对,任意子群是正规子群也不一定是交换群。

\para 在继续回答上面的问题之前,我们先看看陪集。设$H$是$G$的一个子群,可以自然地定义一个等价关系:子集族$\{gH\,:\, g\in G\}$构成了$G$的一个划分。

证明是简单的。显然他们的并等于$G$,只要证明不交性即可,如果$gH\cap hH\neq \varnothing$,则存在$k\subset gH\cap hH$,故$g^{-1}k$和$h^{-1}k$属于$H$,由于$H$是子群,则$g^{-1}k\left(h^{-1}k\right)^{-1}=g^{-1}h\in H$,因此$g^{-1}hH=H$也就堆出了$hH=gH$.

这样定义的等价类被称为一个左陪集(coset),当然还可以类似定义右陪集。对于陪集,有如下简单性质:$|gH|=|H|$. 这点只要注意到$gh_1=gh_2$可以推出$h_1=h_2$即可。对于有限群就得到了Lagrange定理:子群的元素个数可以整除群的元素个数,除出来就是陪集的个数。

\para 对于正规子群$H$而言,由于$gH=Hg$,所以正规子群的左陪集和右陪集相同。将$gH$记作$\bar{g}$,我们证明$\{\bar{g}\,:\, g\in G\}$构成了一个群。

首先是乘法,他来自于$G$的乘法:
\[
	\bar{g}\bar{h}=gHhH=HghH=H(gh)H=(gh)HH=(gh)H=\overline{gh}.
\]
从乘法定义看出看出,单位元是$\bar{e}$,而$\bar{g}$的逆就是$\overline{g^{-1}}$. 因此$\{\bar{g}\,:\, g\in G\}$构成了一个群,他被记作$G/H$,称作$G$关于$H$的商群,或者说$G$模掉$H$.

注意到,在定义乘法的时候,我们利用了正规子群的性质。这是非常关键的。

\para 前面我们看到了,任意同态的核都是正规子群。反过来,我们问过什么时候子群是一个同态的核。下面我们做出回答,任何正规子群都可以实现为一个同态的核。这就是说,同态的核刚好就是正规子群,不多不少。

设$H$是$G$的一个子群,我们定义一个同态
\[
	\begin{array}{ccccc}
		\pi &:&G &\to& G/H,\\
		\pi &:&g &\mapsto& \bar{g}.
	\end{array}
\]
利用$\bar{g}\bar{h}=\overline{gh}$可以看出他是一个同态,而且是满同态,且$\ker(\pi)=\bar{e}=H$. 这就证明了我们的论断。

使用短正合列,我们可以将上面的结论总结为:
\[
	1\to H \hookrightarrow G \xrightarrow{\pi} G/H\to 1.
\]

\para 类似于商群,我们可以定义出商环。设$\mathfrak{a}$是$R$的任意一个理想,作为交换群,$\mathfrak{a}$是$R$的正规子群,所以可以定义出商群$R/\mathfrak{a}=\{\bar{r}=r+\mathfrak{a}\,:\, r\in R\}$,换而言之,$r\in \bar{s}$当且仅当$r=s+a$,其中$a\in \mathfrak{a}$. 我们下面在$R/\mathfrak{a}$上定义出一个环结构。

我们定义$R/\mathfrak{a}$中的乘法为$\bar{r}\bar{s}=\overline{rs}$. 注意到这并不是$R$上的两个子集的乘法。这个乘法的定义是合理的,设$\overline{r'}=\bar{r}$,即$r'+a=r$, 以及$\overline{s'}=\bar{s}$,即$s'+b =s$,则
\[
	rs=(r'+a)(s'+b)=r's'+(r'b+as'+ab),
\]
其中$r'b+as'+ab\in \mathfrak{a}$,因为$\mathfrak{a}$是一个理想。故$\overline{rs}=\overline{r's'}$. 剩下的环的性质的检验都是直接的。

\para 同样的,我们可以定义出商模。设$M$是一个$R$-模,而$N$是$M$的任意一个子模,作为交换群可以定义出商群$M/N=\{\bar{m}=m+N\,:\, m\in M\}$. 我们下面在$M/N$上定义出一个$R$-模结构。

定义$M/N$中的标量乘法为$r\bar{m}=\overline{rm}$.这个乘法的定义是合理的,设$m'=m+n$,其中$n\in N$,则
\[
	rm'=rm+rn
\]
其中$rn\in N$,因为$N$是一个理想。故$\overline{rm}=\overline{rm'}$. 剩下的模的性质的检验都是直接的。

\theo 同构基本定理:设$f:A\to B$是一个(群/环/$R$-模)同态,若存在同态$g: A\to B$使得$\ker f\subset \ker g$,则他诱导了一个同态$\bar{g}:A/\ker{f}\to B$. 如果$g$是满同态,则$\bar{g}$是满同态。特别地,如果$f$是满同态,则$\bar{f}: A/\ker{f}\to B$是一个同构。

\proof
	对于群的情况,$g(\ker f)=e$,$g(a\ker f)=g(a)$. 对于环和$R$-模的情况,$g(\ker f)=0$,$g(a+\ker f)=g(a)$. 所以我们可以定义$\bar{g}(\bar{a})=g(a)$,不难检验这是一个(群/环/$R$-模)同态。并且$g$是满的时候,$\bar{g}$是满的。

	现在假设$f$是满同态,则$\bar{f}$是满同态。我们剩下的只要证明$\bar{f}$是个单射,或者$\ker(\bar{f})=\{\bar{e}\}$即可。设$\bar{f}(\bar{a})=f(a)=e$,于是$a\in \ker(f)=\bar{e}$,因此$\bar{a}\cap \bar{e}\neq\varnothing$,于是我们也就得到$\bar{a}=\bar{e}$,所以$\ker(\bar{f})=\{\bar{e}\}$. 

	由于$\bar{f}$是满同态也是单同态,所以他是一个同构。
\qed

一般只有最后一点被称为同构基本定理,但是前面的一部分是商群/环/$R$-模的泛性质,所以我们特地写出来。下面我们写两个常用的同构,他们是同构基本定理的简单推论。由于会谈到子模,先做观察:子模的子模依然是子模。

\para 设$L\subset M \subset N$是有包含关系的三个$R$-模,则商映射$\rho:N\to N/M$诱导了一个满射$\bar{\rho}:N/L \to N/M$使得$\ker \bar{\rho}=M/L$. 使用同构基本定理,我们就得到了同构$N/M=(N/L)/(M/L)$.

考虑商映射$\rho:N\to N/M$,由于$L\subset \ker \rho=M$,由同构基本定理,诱导了一个满射$\bar{\rho}:N/L \to N/M$,接下来只要求他的核即可。任取$\bar{x}\in N/L$以及代表元$x\in N$,由于$\bar{\rho}(\bar{x})=\rho(x)$,所以$\bar{\rho} (\bar{x})=0$当且仅当$x\in \ker \rho =M$,这就告诉我们$\ker \bar{\rho}=M/L$.

\para \label{modiso1}设$M_1$和$M_2$是$M$的子模,则复合$M_2\hookrightarrow M_1+M_2 \to (M_1+M_2)/M_1$是一个满射,他的核是$M_1\cap M_2$,因此有同构
\[
	(M_1+M_2)/M_1\cong M_2/(M_1\cap M_2).
\]

\para 可以将上面的命题翻译到群上去。不过由于只有正规子群才可以谈商群,所以我们必须更加谨慎。两个相应的同构被如下表述:

\begin{itemize}
\item 给定群$G$,$N$和$M$为$G$的正规子群,满足$M$包含于$N$,则$N/M$是$G/M$的正规子群,且有如下的群同构: $ (G/M)/(N/M)\cong G/N$.

\item 给定群$G$ 、其正规子群$N$ 、其子群$H$ ,则$N\cap H$是$H$ 的正规子群,且有群同构如下:$H/(H\cap N)\cong HN/N$
\end{itemize}

证明和模的情况是类似的,只不过必须更加小心而已。对于环,我们一般谈理想,所以可以应用模的定理(作为模,理想是子模)。

\para 设$G$和$G'$是两个群,$f:G\to G'$是一个群同态,设$H'\subset G'$是一个子群,则$f^{-1}(H')$是一个$G$的子群。事实上,任取$g$, $h\in f^{-1}(H')$,于是$f(gh)=f(g)f(h)\in H'$也就推出了$gh\in f^{-1}(H')$. 

类似地,设$M$和$M'$都是$R$-模,而$f:M\to M'$是一个$R$-模同态,考虑一个子模$N'\subset M'$,则$f^{-1}(N')$是$M$的一个子模。事实上,加法群结构来自于群的结论,任取$r\in R$以及$x\in f^{-1}(N')$,于是$f(rx)=rf(x)\in N'$推出了$rx\in f^{-1}(N')$. 

\section{范畴与函子}

前面看到了,在群/环/模上面,我们都谈论了什么叫同态,什么是同构,以及后来将可能遇到的一些在不同的代数结构上有类似性质的构造。下面我们要引入新的语言来抽象这些东西,他的名字叫范畴。

\para 一个范畴$\mathcal{C}$有如下数据:
\begin{enumerate}
\item 一类(class)对象(object),类在这里的意思并不是构成一个集合。一个范畴中所有的对象可以不构成一个集合。如果$X$是$\mathcal{C}$的一个对象,我们沿用集合论的符号,写成$X\in \mathcal{C}$.

对每两个对象$X$和$Y$,有一个集合$\mathcal{C}(X,Y)$,有时候也写做$\mathrm{Mor}_{\mathcal{C}}(X,Y)$乃至$\Hom_{\mathcal{C}}(X,Y)$,其中的元素被称为\idx{态射}(\idx{morphism}\footnote{从英文来看,态射构成的集合写作$\mathrm{Mor}_{\mathcal{C}}(X,Y)$比$\Hom_{\mathcal{C}}(X,Y)$准确,因为morphism不一定是homomorphism(同态)。}
). 设$\varphi\in \mathcal{C}(X,Y)$,他通常沿用映射的记号写成$\varphi:X\to Y$,但态射不必是映射。沿用叫法,$\varphi:X\to Y$中$X$被称为定义域,而$Y$被称为值域。态射还需要满足如下假设:

\begin{itemize}

\item 任取态射$\varphi$,我们可以从他得到唯一的定义域和值域,即$\varphi$如果同时属于$\mathcal{C}(X,Y)$和$\mathcal{C}(Z,W)$,则$X=Z$以及$Y=W$.

\end{itemize}

\item 任取对象$X$, $Y$和$Z\in \mathcal{C}$,则有映射$\mathcal{C}(X,Y)\times \mathcal{C}(Y,Z)\to \mathcal{C}(X,Z)$,沿袭映射的说法,他被称为态射的复合。设$\varphi\in \mathcal{C}(X,Y)$和$\psi \in \mathcal{C}(Y,Z)$,则复合出来的态射被记作$\psi\circ \varphi$或者再简单一点$\psi\varphi$. 复合还需要满足:
\begin{itemize}

\item 态射的复合满足结合率,即$\varphi(\psi\theta)=(\varphi\psi)\theta$.

\item 在每一个$\mathcal{C}(X,X)$中有一个元素$\id_X$使得,任取$\rho:X\to Y$以及$\psi:Z\to X$成立复合$\rho\id_X=\rho$以及$\id_X\psi=\psi$.
\end{itemize}
\end{enumerate}

如果一个范畴的对象可以构成一个集合,那么这个范畴就被称为小范畴。在定义中,我们可以看到,所有对象与集合、态射与映射的相似性,所以我们可以直接断言,所有集合以及集合间的映射构成了一个范畴,他被称为集合范畴,记作$\mathsf{Set}$.

类似地,所有的群/环/$R$-模和他们两两之间的同态构成了一个范畴。左$R$-模范畴我们记作${}_R\mathsf{Mod}$,右$R$-模范畴我们记作$\mathsf{Mod}_R$. 既是左$R$-模,又是右$S$-模的范畴我们记作${}_R\mathsf{Mod}_S$. 下面是一个稍微不一般一点的例子。

从一个范畴也可以构造一些范畴,比如子范畴:一个范畴$\mathcal{C}$的子范畴是一个范畴$\mathcal{S}$,其对象为$\mathcal{C}$内的对象,态射为$\mathcal{C}$内的态射,且有相同的单位态射与态射复合。直观上来看,$\mathcal{C}$的子范畴是一个从C中“移去”部分物件和态射的范畴。比如,模范畴是交换群范畴的子范畴。

\para 设$I$是一个偏序集,偏序关系为$\leq$,那么我们如下定义态射使得他构成了一个范畴:
\[
	\Hom_{I}\left(x,y\right)=\begin{cases}
	\{x,\{x,y\}\}=(x,y)&\text{, if }x\leq y\text{;}\\
	\varnothing&\text{, otherwise}.
	\end{cases}
\]
复合映射就写作$(y,z)(x,y)=(x,z)$,因此偏序集具有一个(小)范畴结构,其中的态射记作$x\leq y$.

一个拓扑空间$X$能被看成一个范畴,对象取作他的所有开集,而态射取作
\[
	\Hom_{X}(U,V)=\begin{cases}
	\bigl\{i^U_V:U\hookrightarrow V\bigr\}&\text{, if }U\subset V\text{;}\\
	\varnothing&\text{, otherwise}.
	\end{cases}
\]
当然这个范畴也可以通过在开集族上给定一个偏序$U\leq V$当且仅当$U\subset V$给出。

所有拓扑空间和其间的连续映射当然构成一个范畴。所有光滑流形和其间的光滑映射构成一个范畴。另外一个例子是关系构成了范畴,他的对象是集合,而$\text{Mor}(X,Y)$是所有$X\times Y$的子集,设$\varphi:X\to Y$和$\psi:Y\to Z$是两个态射,则复合被定义为
\[
	\psi\varphi=\bigl\{(x,z)\in X\times Z\,:\,\exists y\in Y\,\text{ s.t. } (x,y)\in\varphi,\, (y,z)\in \psi\bigr\}.
\]

\para 从一个给定的范畴$\mathcal{C}$,可以构造出一个新的范畴$\mathcal{C}^{\mathrm{op}}$如下:对象不变,定义
\[\mathcal{C}^{\mathrm{op}}(X,Y)=\mathcal{C}(Y,X),\]复合不变。他被称为给定范畴的对偶范畴。凡是纯范畴的命题与证明,将所有的箭头(即态射)全部反过来,可以得到对应的命题与证明,这样的命题被称为对偶命题。

对偶范畴的一个实例就是偏序集,他将小于等于改成大于等于。设$(X,\leq)$是一个偏序集,那么$(X,\leq)^\text{op}$与$(X,\leq)$具有相同的对象,但有着相反的态射,即$(X,\leq)^\text{op}$中的偏序$x\leq_{\text{op}} y$当且仅当在$(X,\leq)$中成立$y\leq x$. 对于符号$\leq_{\text{op}}$,我们通常以$\geq$代之,因此,$(X,\leq)$的对偶范畴就是$(X,\geq)$. 如果以$X$代之$(X,\geq)$,则以$X^{\text{op}}$代之$(X,\geq)$.

\para 在范畴论中,我们经常要讨论所谓的交换图。从一个对象$X_0$开始,我们可以通过多次复合不同态射得到一个对象$X_n$,我们可以通过图来表示这个过程,
\[
	X_0\xrightarrow{f_0} X_1\xrightarrow{f_1} X_2\cdots X_{n-1}\xrightarrow{f_{n-1}} X_n.
\]
但一般而言,从$X_0$到$X_n$我们有不同的路径,比如考虑最简单的图
\[
\begin{xy}
	\xymatrix{
		X\ar[r]^{f} \ar[dr]_{h}&Y \ar[d]^{g}\\
		&Z
	}
\end{xy}
\]
从$X$到$Z$一共是两条路径,其一是先通过$Y$然后再到$Z$,另一个是直接到$Z$,前者对应于态射$gf:X\to Z$,后者对应于态射$h:X\to Z$. 称一个图是交换的,就是说沿着不同路径却可以得到相同的结果。比如在上例中就是指$h=gf$. 另一类很常见的图是方的,
\[
\begin{xy}
	\xymatrix{
		A\ar[rr]^{f} \ar[d]_{h}&&B \ar[d]^{g}\\
		C\ar[rr]^{l}&&D
	}
\end{xy}
\]
从$A$到$D$有两条路径,此图交换就是在说$gf=lh$.

\para 设$\varphi:X\to Y$和$\psi:Y\to X$,如果$\varphi\psi=\id_Y$和$\psi\varphi=\id_X$,则称对象$X$和$Y$是同构的,$\varphi$或者$\psi$被称为同构(态射)。当然,现在由于暂时不能谈一个态射是单射还是满射,即使可以,也不一定能像模/群/环范畴那样有命题“既单又满的态射是同构”。$X$和$Y$同构记作$X\cong Y$. 集合范畴的同构即双射。

\para 类似于集合之间有映射,范畴之间也有个\idx{函子}(\idx{functor}). 由于范畴有对象以及对象之间的态射,所以函子也是由关于对象和范畴的数据构成的,设$\mathcal{C}$和$\mathcal{D}$是两个范畴,$F$被称为一个函子,如果
\begin{enumerate}
	\item 对任意的$X\in \mathcal{C}$,有唯一的$F(X)\in \mathcal{D}$.
	
	\item 对任意的$\varphi\in \mathcal{C}(X,Y)$,存在唯一的$F(\varphi)\in  \mathcal{D}(F(X),F(Y))$使得成立复合关系$F(\varphi\psi)=F(\varphi)F(\psi)$以及$F(\id_X)=\id_{F(X)}$.
\end{enumerate}

一般将“从$\mathcal{C}$到$\mathcal{D}$的函子$F$”记作$F:\mathcal{C}\to \mathcal{D}$. 虽然记号相同,但是我们可以通过语境来区分他与态射。

与态射类似,函子也能谈复合,设$f:\mathcal{C}\to \mathcal{D}$和$g:\mathcal{D}\to \mathcal{E}$是两个函子,则他们的复合函子$gf:\mathcal{C}\to \mathcal{E}$表现在对象上是$X\mapsto f(X)\mapsto g(f(X))$,表现在态射上是$\varphi\mapsto f(\varphi)\mapsto g(f(\varphi))$. 

函子的实例是非常多的,比如从$R$-模范畴到交换群范畴就有一个函子,他将一个模看成一个交换群,模同态看成交换群同态。这样的函子,从某个结构丰富的范畴到结构没那么丰富的范畴,被称为遗忘函子。遗忘函子又比如从拓扑空间范畴到集合范畴。

再比如,设$X$和$Y$是两个偏序集,则函子$f:X\to Y$是一个映射,且满足对$x_1\leq x_2$成立$f(x_1)\leq f(x_2)$. 这就是偏序集之间的保序映射。

范畴与函子在数学中的出现是如此地频繁的原因在于:对一个数学结构的认识,不仅仅需要描述对象本身,也需要描述这个对象与其他对象之间的作用,即还需要描述态射,这就构成范畴的内容。而函子的出现在于联系两个数学结构,比如我们经常从一些对象开始构造得到新的对象。同样,描述两个合适的数学结构之间的联系,一方面描述对象之间的联系,另一方面也要描述态射之间的联系,而这正是函子所做的。基于这些理由,一个合适的分类理论,不仅仅要分类对象,也要分类态射。

在有些作者那里,上面定义的函子被称为协变函子,而函子$f:\mathcal{C}^{\mathrm{op}} \to \mathcal{D}$被称为反变函子。后者明确写出来即$f(\varphi\psi)=f(\psi)f(\varphi)$,其中$\psi$和$\varphi$都是$\mathcal{C}$中的态射。于是,偏序集之间的反变函子就是逆序映射,对$x_1\leq x_2$成立$f(x_2)\leq f(x_1)$.

\para (反变)态射函子是(反变)函子的重要例子。设$\mathcal{C}$是一个范畴,$A\in \mathcal{C}$是一个对象。定义一个函子$\mathcal{C}(A,\star):\mathcal{C}\to \mathsf{Set}$如下:对于对象,$B\mapsto \mathcal{C}(A,B)$,对于态射$f:B\to C$,它得到了态射
\[
	\mathcal{C}(A,f)=f_*: g\mapsto f\circ g.
\]
此时$\mathcal{C}(A,\star)$被称为关于$A$的\idx{态射函子}。经常,我们将$\mathcal{C}(A,\star)$简单记$h^A$.

同样,我们可以定义\idx{反变态射函子}$\mathcal{C}(\star,A):\mathcal{C}^{\text{op}}\to \mathsf{Set}$:对于对象,$B\mapsto \mathcal{C}(B,A)$,对于态射$f:B\to C$,它通过
\[
	\mathcal{C}(f,A)=f^*: g\mapsto g\circ f
\]
得到了态射$\mathcal{C}(f,A):\mathcal{C}(C,A)\to \mathcal{C}(B,A)$. 经常,我们将$\mathcal{C}(\star,A)$简单记$h_A$,而且有时候会将$h_A(B)$就写做$A(B)$.

\para 一个范畴到另一个范畴的全部函子也构成一个范畴,即函子的范畴。对象是函子,所以关键点在于定义函子之间的态射。设$f$和$g$都是从$\mathcal{C}$到$\mathcal{D}$的函子,如果对每一个$X\in \mathcal{C}$都有一个态射$\varphi(X):f(X)\to g(X)$使得如下交换图成立
\[
\begin{xy}
	\xymatrix{
		f(X)\ar[rr]^{\varphi(X)} \ar[d]_{f(\varphi)}&&g(X) \ar[d]^{g(\varphi)}\\
		f(Y)\ar[rr]^{\varphi(Y)}&&g(Y)
	}
\end{xy}
\]
则$\varphi$被称为一个自然变换,或者叫函子间的态射。继而,谈论函子的同构也有了意义。将$\mathcal{C}$与$\mathcal{D}$之间的函子构成的范畴记作$[\mathcal{C},\mathcal{D}]$,特别地,将$[\mathcal{C}^\text{op},\mathsf{Set}]$记作$\hat{\mathcal{C}}$.

描述自然变换是范畴论的初衷,自然变换在数学中的频繁出现才使得人们想要抽象出这个概念,为了描述这个概念,人们发明了范畴与函子。自然变换频繁出现的原因在于一些构造:设有一个范畴$\mathcal{C}$,对每一个对象$X\in \mathcal{C}$构造了一个函子$f_X$,比如前面定义出的反变态射函子$h_X$. 当我们改变对象的时候,即给出一个态射$\varphi:X\to Y$,往往具体的构造将诱导出对每一个$V\in \mathcal{C}$都有态射$f_\varphi(V):f_X(V)\to f_Y(V)$. 由于是函子,对态射$\rho:V\to W$也诱导了态射,汇总起来就是有图
\[
\begin{xy}
	\xymatrix{
		f_X(V)\ar[rr]^{f_\varphi(V)} \ar[d]_{f_X(\rho)}&&f_Y(V) \ar[d]^{f_Y(\rho)}\\
		f_X(W)\ar[rr]^{f_\varphi(W)}&&f_Y(W)
	}
\end{xy}
\]
既然有图,要求交换当然是自然的。到了具体例子,图的交换性需要检验。

以反变态射函子为例。给定态射$\varphi:X\to Y$,则$\varphi_*$将给出一个自然变换$h_X\to h_Y$,实际上给定态射$\psi:W\to V$,需要检查图
\[
\begin{xy}
	\xymatrix{
		h_X(V)\ar[rr]^{\varphi_*} \ar[d]_{\psi^*}&&h_Y(V) \ar[d]^{\psi^*}\\
		h_X(W)\ar[rr]^{\varphi_*}&&h_Y(W)
	}
\end{xy}
\]
任取$\pi\in h_X(V)$即$\pi:V\to X$,有$\psi^*(\varphi_*(\pi))=\varphi\pi\psi$以及$\varphi_*(\psi^*(\pi))=\varphi\pi\psi$,所以图是交换的。类似例子给出如下命题:如果$X\cong Y$,则$h_X\cong h_Y$. 反命题实际上也是正确的,后面会予以证明。

\para 由于经常一个构造会涉及两个对象或多个对象,而分别改变对象就会产生不同的函子,为了抽象这个过程,就产生了下面积范畴的概念,正如讨论多元函数引入了集合的直积一样。

设$\mathcal{C}$和$\mathcal{D}$是范畴,他们的积范畴记作$\mathcal{C}\times\mathcal{D}$:对象是这样的二元组$(X,Y)$或$X\times Y$,其中$X\in \mathcal{C}$, $Y\in\mathcal{D}$. 而态射定义为$(\varphi,\psi)$或$\varphi\times \psi$,其中$\varphi$是$\mathcal{C}$中的态射,而$\psi$是$\mathcal{D}$中的态射,这样就建立了集合的等同
\[
	{\mathcal{C}}(X,X')\times {\mathcal{D}}(Y,Y')=(\mathcal{C}\times \mathcal{D})(X\times Y,X'\times Y').
\]
复合定义为$(\varphi,\psi)(\varphi',\psi')=(\varphi\varphi',\psi\psi')$. 在这个复合下,对象$X\times Y$到自己的恒等态射就是$(\id_X,\id_Y)$. 所以$\mathcal{C}\times\mathcal{D}$确确实实是一个范畴。同理,可以定义有限个范畴的积。

设$\mathcal{F}$是另一个范畴,函子$f:\mathcal{C}\times \mathcal{D}\to \mathcal{F}$被称为一个双函子。比如$\mathcal{C}(\star,\star)$是一个$\mathcal{C}^\text{op}\times \mathcal{C}\to \mathsf{Set}$的双函子。如果固定双函子的一个变元,那么双函子会退化成一个函子。比如$\mathcal{C}(\star,\star)$固定第一个变元为$X$的时候得到$h^X$.

反过来,如果$f$对每一个$X\in \mathcal{C}$是函子$f(X,\star):\mathcal{D}\to \mathcal{F}$,以及对每一个$Y\in \mathcal{C}$是函子$f(\star,Y):\mathcal{C}\to \mathcal{F}$,那么何时$f$才是一个双函子。考虑如下图
\[
\begin{xy}
	\xymatrix{
		f(X,Y)\ar[rr]^{f(\varphi,Y)} \ar[d]_{f(X,\psi)}&&f(X',Y) \ar[d]^{f(X',\psi)}\\
		f(X,Y')\ar[rr]^{f(\varphi,Y')}&&f(X',Y')
	}
\end{xy}
\]
如果这个图交换,那么从左上到右下的映射,即复合$f(X,\psi)f(\varphi,Y)$将定义出一个态射$f(\varphi,\psi)$,此时$f$构成一个双函子。具体的检验交由读者。

\para 设$f$, $g:\mathcal{C}\to \mathcal{D}$是函子,$X\in \mathcal{C}$是一个对象,通过$X:f\to f(X)$以及任取自然变换$\varphi:f\to g$,定义$X(\varphi)=\varphi(X):f(X)\to g(X)$,可以看到此时$X$此时构成一个$[\mathcal{C},\mathcal{D}] \to \mathcal{D}$的函子。可见,$f(X)$一方面可以将$f$理解成函子,另一方面也可以将$X$理解成函子。只不过前者是范畴$\mathcal{C}$到范畴$\mathcal{D}$的函子,后者是范畴$[\mathcal{C},\mathcal{D}]$到范畴$\mathcal{D}$的函子。将这点抽象出来,如果把$f(X)$写作$\Gamma(X,f)$,则
\[
	\Gamma(\star,\star):\mathcal{C}\times [\mathcal{C},\mathcal{D}]\to \mathcal{D}
\]
是一个双函子。这点的证明只要按定义应用上面的判据就可以了。

\para 有了二元积范畴,类似地,可以定义多元(有限)积范畴。类似于对角函数,可以定义对角函子:设$\mathcal{C}$是一个范畴,而$\mathcal{C}\times \cdots \times \mathcal{C}$是积范畴。那么可以定义函子$\Delta:\mathcal{C}\to \mathcal{C}\times \cdots \times \mathcal{C}$,对于对象$X$,定义$\Delta:X\mapsto (X$, $\cdots$, $X)$,对于态射$\varphi\in \mathcal{C}(X,Y)$,定义$\Delta: \varphi \mapsto (\varphi$, $\cdots$, $\varphi)$.

甚至是考虑关于一个集合的积范畴(即指标构成一个集合的情况),积范畴依然可以定义。但是如果比如考虑指标并不是一个集合,有些东西就会自然地出现问题。这个问题有相当深刻的背景,涉及了现代数学的逻辑基础,即公理化体系(主要是ZFC公理体系)上面。众所周知,在朴素的集合论里面有Russell悖论,这就限制了一些集合的构造,比如所有集合构成的一个整体并不是集合。到了范畴论里面,我们自然也会面临集合论的问题,比如当考虑所有范畴的构成的整体是不是范畴。

为了避免谈及ZFC公理体系的问题,Grothendieck给出了一个新的假设,即Grothendieck universe的存在。简单来说,就是给出了一个足够大的集合$\mathscr{U}$,而我们在具体问题中谈论的集合,都是$\mathscr{U}$的子集与元素。并且在$\mathscr{U}$内,他的元素与子集在ZFC体系下进行运算都会落入$\mathscr{U}$内。于是,在Grothendieck universe内,我们就模拟了一个集合论公理体系,在这里面讨论,就可以避免许多集合论的问题。所有可能的问题都会集中到Grothendieck universe的存在上。

\para 双函子之间的态射,表现为如下自然变换
\[
\begin{xy}
	\xymatrix{
		f(X,Y)\ar[rr]^{\varphi(X,Y)} \ar[d]_{f(\psi,\varphi)}&&g(X,Y) \ar[d]^{g(\psi,\varphi)}\\
		f(X',Y')\ar[rr]^{\varphi(X,Y)}&&g(X',Y')
	}
\end{xy}
\]
很清楚,当固定$X$的时候,$\varphi(X,\star)$是$f(X,\star)\to g(X,\star)$的一个自然变换。固定$Y$也一样。

反过来,设$f$, $g:\mathcal{C}\times \mathcal{D}\to \mathcal{F}$是两个双函子,如果任取$X\in \mathcal{C}$,存在自然变换$\alpha(X):f(X,\star)\to g(X,\star)$,任取$Y\in \mathcal{D}$,存在自然变换使得$\beta(Y):f(\star,Y)\to g(\star,Y)$,且$\alpha(X)(Y)=\beta(Y)(X)$,则存在自然变换$\gamma:f\to g$使得$\gamma(X,Y)=\alpha(X)(Y)=\beta(Y)(X)$.

\proof
	考虑态射$(\psi,\varphi):(X,Y)\to (X',Y')$,他可以拆成
	\[
		(\psi,\varphi): (X,Y) \xrightarrow{(\psi,\id_Y)} (X',Y)\xrightarrow{(\id_{X'},\varphi)} (X',Y'),
	\]
	分别应用自由变换
	\[
	\begin{xy}
		\xymatrix{
			f(X,Y)\ar[rr]^{\alpha(X)(Y)} \ar[d]_{f(\psi,\id_Y)}&&g(X,Y) \ar[d]^{g(\psi,\id_Y)}\\
			f(X',Y)\ar[rr]^{\alpha(X')(Y)}_{\beta(Y)(X')} \ar[d]_{f(\id_{X'},\varphi)}&&g(X',Y) \ar[d]^{g(\id_{X'},\varphi)}\\
			f(X',Y')\ar[rr]^{\beta(Y')(X')}&&g(X',Y')
		}
	\end{xy}
	\]
	因此,从左上到右下两条路给出$g(\psi,\varphi)\alpha(X)(Y)=\beta(Y')(X')f(\psi,\varphi)$,由于$\gamma(X,Y)=\alpha(X)(Y)=\beta(Y)(X)$,所以也就是$g(\psi,\varphi)\gamma(X,Y)=\gamma(X',Y')f(\psi,\varphi)$,所以$\gamma:f\to g$是一个自然变换。
\qed

% \lem Yoneda引理:设$\mathcal{C}$是一个范畴,$A\in \mathcal{C}$是一个对象,$F:\mathcal{C}\to \mathsf{Set}$是一个函子,则从$\mathcal{C}(A,\star)$到函子$F$上的自然变换一一对应着$F(A)$里面的元素。

% \proof 对每个自然变换$\alpha:\Hom_\mathcal{C}(A,\star)\to F$定义$\theta(\alpha)=\alpha(A)(\id_A)\in F(A)$,现在定义一个逆映射即可。对$x\in F(A)$,定义自然变换$\psi(x):\Hom_\mathcal{C}(A,\star)\to F$通过$\psi(x)(B)(f)=F(f)(x)$,不难验证这是一个自然变换,而且$\psi$和$\theta$是互逆的。\qed

\para 下面我们要引入可表函子,在具体给出定义之前先做出如下观察:设$X\in\mathsf{Set}$是一个集合,那么它与集合$X(e)={\mathsf{Set}}(e,X)$之间存在一个双射,其中$e$是任意的一个单点集。但事实上,在任意范畴$\mathcal{C}$,类似于集合范畴中的单点集的东西是不一定存在的,所以我们转而同时考虑所有的${\mathcal{C}}(Y,X)$,其中$Y\in \mathcal{C}$,或者说考虑反变态射函子$h_X$,此时我们可以得到对象$X\in \mathcal{C}$足够多乃至于全部的信息,这就类似于物理中测量的思想,人类进行的一切测量都是通过相互作用完成的,并且通过测量得到了被测物的信息。事实上,如果$h_X\cong h_Y$,则$X\cong Y$.

设$f$是一个$\mathcal{C}$到Set的反变函子,如果$f\cong h_X$对某个$X\in \mathcal{C}$成立,那么我们不仅仅得到了函子$f$的信息,我们也同样得到了对象$X$的信息。而且,比起任意的函子,反变态射函子是我们更愿意看到的,并且有着许多好的性质,因此我们可以做出如下定义:如果反变函子$f:\mathcal{C}^{\text{op}}\to \mathsf{Set}$同构于一个反变态射函子$h_X$,则$f$被称为可表函子,以及此时$f$被$X$表示,$X$是$f$的表示对象。

\lem 对任意的$X\in \mathcal{C}$以及$f\in \hat{\mathcal{C}}$,存在同构$i_{X,f}:\Gamma(X,f)=f(X)\to {\hat{\mathcal{C}}}(h_X,f)$.

\proof
	固定$f$,任取自然变换$\varphi\in {\hat{\mathcal{C}}}(h_X,f)$,对象$Y\in\mathcal{C}$以及态射$g:Y\to X$,有交换图
	\[
	\begin{xy}
		\xymatrix{
			h_X(X)\ar[rr]^{\varphi(X)} \ar[d]_{g^*}&&f(X) \ar[d]^{f(g)}\\
			h_X(Y)\ar[rr]^{\varphi(Y)}&&f(Y)
		}
	\end{xy}
	\]
	取$\id_X\in h_X(X)=\mathcal{C}(X,X)$,有$e_{X,f}(\varphi)=\varphi(X)(\id_X)\in f(X)$,因此有映射$e_{X,f}:{\hat{\mathcal{C}}}(h_X,f)\to f(X)$. 由交换图,$\varphi(Y)(g)=f(g)\left(e_{X,f}(\varphi)\right)$.

	反过来任取$a\in f(X)$,对每一个$Y\in\mathcal{C}$,我们可以定义映射
	\[
	\begin{array}{ccccc}
	i(a)(Y)&:&h_X(Y)&\to &f(Y)\\
	i(a)(Y)&:&g&\mapsto&f(g)(a)
	\end{array},
	\]
	这样也就定义了一个函子$i_{X,f}(a)\in {\hat{\mathcal{C}}}(h_X,f)$,继而有映射$i_{X,f}:f(X)\to {\hat{\mathcal{C}}}(h_X,f)$. 最后只要检验$e_{X,f}(i_{X,f}(a))=a$以及$i_{X,f}(e_{X,f}(\varphi))=\varphi$,他们都由$i_{X,f}(Y)(g)=f(g)\left(e_{X,f}(\varphi)\right)$保证。
\qed

对偶地,设$f$是一个$\mathcal{C}$到集合范畴的函子,则$\Gamma(X,f)\cong [\mathcal{C},\mathsf{Set}](h^X,f)$对任意的$X\in \mathcal{C}$都成立。

% \para 对于固定的$X$,他可以看成一个$\hat{\mathcal{C}}=[\mathcal{C}^\text{op},\mathsf{Set}]\to \mathsf{Set}$的函子。同样,对于固定的$X$,态射函子${\hat{\mathcal{C}}}(h_X,\star)$也是一个$\hat{\mathcal{C}}\to \mathsf{Set}$的函子。上面的引理就是在说,这两个函子作用在相同的对象$f\in \hat{\mathcal{C}}$上可以得到同构的两个对象。

% 对于固定的$f$,他本身是$\hat{\mathcal{C}}$中的函子,也可以写成$\Gamma(\star,f)$. 同时,函子${\hat{\mathcal{C}}}(h_\star,f):\mathcal{C}^{\text{op}}\to \mathsf{Set}$也是$\hat{\mathcal{C}}$中的函子。上面的引理就是在说,这两个函子作用在相同的对象$X\in \mathcal{C}$上可以得到同构的两个对象。

% 下面一个命题告诉我们,Lemma (\thepara)中的同构对于这两对函子都是自然变换,即实际上是函子间的同构,而不仅仅是作用在对象上的同构。

% 如果有一个自然变换$\varphi:f\to g$,他将诱导出态射$\varphi(X):f(X)\to g(X)$,以及态射$\varphi_*: {\hat{\mathcal{C}}}(h_X,f)\to {\hat{\mathcal{C}}}(h_X,g)$.

\para 固定$X$,${\hat{\mathcal{C}}}(h_X,\star)$是一个函子。固定$f$,${\hat{\mathcal{C}}}(h_\star,f)$也是一个函子。为了检查${\hat{\mathcal{C}}}(h_\star,\star)$是一个双函子,必须检查图
\[
\begin{xy}
	\xymatrix{
		\hat{\mathcal{C}}(h_{X},f)\ar[rr]^{\hat{\mathcal{C}}(\varphi^*,f)} \ar[d]_{\psi_*}&&\hat{\mathcal{C}}(h_{X'},f) \ar[d]^{\psi_*}\\
		\hat{\mathcal{C}}(h_{X},f')\ar[rr]^{\hat{\mathcal{C}}(\varphi^*,f')}&&\hat{\mathcal{C}}(h_{X'},f')
	}
\end{xy}
\]
其中$\varphi:X'\to X$是一个态射,而$\psi:f\to f'$是一个自然变换。任取$\pi\in \hat{\mathcal{C}}(h_{X},f)$,则先右再下得到$\psi_*(\varphi^*(\pi))=\psi\pi\varphi$. 先下再右得到$\varphi_*(\psi^*(\pi))=\psi\pi\varphi$. 所以图交换,因此${\hat{\mathcal{C}}}(h_\star,\star)$是一个双函子。

\pro 上面的引理可以加强到双函子同构:$i_{-,\star}:\Gamma(-,\star)\to {\hat{\mathcal{C}}}(h_-,\star)$,这被称为Yoneda引理。

\proof
	由于同构已经有了构造,所以只要检验是不是自然变换。而这点只要分别固定$X$和$f$去证明即可。

	固定$X$,给出$\varphi:f\to g$,检验交换图
	\[
	\begin{xy}
		\xymatrix{
			f(X)\ar@<0.3ex>[rr]^-{i_{X,f}} \ar[d]_{\varphi(X)}&&{\hat{\mathcal{C}}}(h_X,f) \ar[d]^{\varphi_*}\ar@<0.3ex>[ll]^-{e_{X,f}}\\
			g(X)\ar@<0.3ex>[rr]^-{i_{X,g}}&&{\hat{\mathcal{C}}}(h_X,g)\ar@<0.3ex>[ll]^-{e_{X,g}}
		}
	\end{xy}
	\]
	固定$f$,给出$\varphi:Y\to X$,检验交换图
	\[
	\begin{xy}
		\xymatrix{
			f(X)\ar@<0.3ex>[rr]^-{i_{X,f}} \ar[d]_{f(\varphi)}&&{\hat{\mathcal{C}}}(h_X,f) \ar[d]^{(\varphi^*)^*}\ar@<0.3ex>[ll]^-{e_{X,f}}\\
			f(Y)\ar@<0.3ex>[rr]^-{i_{Y,f}}&&{\hat{\mathcal{C}}}(h_Y,f)\ar@<0.3ex>[ll]^-{e_{Y,f}}
		}
	\end{xy}
	\]
	构造都是清楚的,检查都是直接的。
	
	仅仅以第一个交换图里面的一部分为例,任取$a\in f(X)$,检查$\varphi\circ (i_f(a))=i_g\bigl(\varphi(X)(a)\bigr)\in {\hat{\mathcal{C}}}(h_X,g)$. 给一个$Y\in \mathcal{C}$,$i_g\bigl(\varphi(X)(a)\bigr)(Y):p\mapsto g(p)\bigl (\varphi(X)(a)\bigr)$,其中$p:Y\to X$是一个态射。同样,$\varphi\circ (i_f(a))(Y)=\varphi\bigl(p\mapsto f(p)(a)\bigr):p\mapsto \varphi(Y)\bigl(f(p)(a)\bigr)$. 最后只要检验$g(p)\bigl (\varphi(X)(a)\bigr)=\varphi(Y)\bigl(f(p)(a)\bigr)$,注意到交换图
	\[
	\begin{xy}
		\xymatrix{
			f(X)\ar[rr]^{\varphi(X)} \ar[d]_{f(p)}&&g(X) \ar[d]^{g(p)}\\
			f(Y)\ar[rr]^{\varphi(Y)}&&g(Y)
		}
	\end{xy}
	\]
	所以得证,其他的检查也类似。
\qed 

\pro 对偶地,存在双函子同构$i_{-,\star}:\Gamma(-,\star)\to [\mathcal{C},\mathsf{Set}](h^-,\star)$,这也被称为Yoneda引理。

在Yoneda引理的证明中,已经看到,最不平凡的地方在于:如果函子$\mathcal{C}(-,X)$与函子$\mathcal{C}(-,X')$同构,则$X\cong X'$. 类似地,还有协变的版本,如果函子$\mathcal{C}(X,-)$与函子$\mathcal{C}(X',-)$同构,则$X\cong X'$. 所以,往往这两点被单独拎出来称为Yoneda引理。

\para 特别地,当$f=h_Y$,则$h_Y\cong {\hat{\mathcal{C}}}(h_\star,h_Y)$. 作用到$X$上得到$h_Y(X)={\mathcal{C}}(X,Y)\cong {\hat{\mathcal{C}}}(h_X,h_Y)$. 所以,$X\cong Y$当且仅当$h_X\cong h_Y$. 如果$f$是可表函子,$X$和$Y$同时表示了函子$f$,则同构$h_X\cong f \cong h_Y$诱导了同构$X\cong Y$,因此可表函子的表示对象确定到一个同构。 

\para 设$\mathcal{C}$是一个范畴,则$\id_\mathcal{C}:\mathcal{C}\to \mathcal{C}$是恒等函子,他使得$\id_\mathcal{C}(X)=X$,以及$\id_\mathcal{C}(\varphi)=\varphi$. 如果一个函子$f:\mathcal{C}\to\mathcal{C}$与恒等函子同构,即有如下交换图成立
\[
\begin{xy}
	\xymatrix{
		X\ar@<0.3ex>[rr]^-{p(X)} \ar[d]_{\varphi}&&f(X) \ar[d]^{f(\varphi)}\ar@<0.3ex>[ll]^-{q(X)}\\
		Y\ar@<0.3ex>[rr]^-{p(Y)}&&f(Y)\ar@<0.3ex>[ll]^-{q(Y)}
	}
\end{xy}
\]
那么在范畴论意义下,我们无法区分范畴$\mathcal{C}$和范畴$f(\mathcal{C})$的,因为这两个范畴有相同的结构。尽管,原本$X\in \mathcal{C}$在$f(\mathcal{C})$里面变成了$f(X)$,看上去变成了不同的对象,但这仅仅类似于重命名,对范畴的结构并没有任何影响。所以如果谈论范畴的等价,也应该止步于此,即有相同结构的范畴看成是等价的。

设$\mathcal{C}$和$\mathcal{D}$是两个范畴,而$f:\mathcal{C}\to \mathcal{D}$以及$g:\mathcal{D}\to \mathcal{C}$是两个函子,那么如果$gf\cong \id_\mathcal{C}$以及$fg\cong \id_\mathcal{D}$,则称这两个范畴等价。

这个定义并不好用,不过他有另一种表述。但是需要引入两个新的定义:称呼函子$f:\mathcal{C}\to \mathcal{D}$是忠实函子/完全函子,如果$f:\mathcal{C}(X,Y)\to \mathcal{D}\bigl(f(X),f(Y)\bigr)$是集合间的单射/满射。

\pro \label{equivcat}两个范畴$\mathcal{C}$和$\mathcal{D}$是等价范畴,当且仅当存在一个即忠实又完全的函子$f:\mathcal{C}\to \mathcal{D}$使得任取$Y\in\mathcal{D}$都有一个$X\in \mathcal{C}$使得$Y=f(X)$.\notprove

\section{理想}

\para 理想是环$R$看成$R$-模时候的子模。一个理想被集合$S\subset R$生成是指在$R$-模意义上生成的子模,即$S$中元素的任意有限线性组合,我们将其记作$(S)$或者$\langle S\rangle$,当$S$是单元素集的时候,这就是主理想,当$S$是有限集的时候,时常就写成$(S)=(f_1$, $\cdots$, $f_n)$或者$\langle f_1$, $\cdots$, $f_n\rangle$. 如果指标需要用一个指标集表示,那么就通常写作$\langle f_i\rangle_{i\in I}$或者$\langle f_i\,:\,i\in I\rangle$. 一个理想生成的理想自然就是他本身。

\para 在不同的理想之间可以定义运算,设$\mathfrak{a}$和$\mathfrak{b}$是$R$的两个理想,则
\[\mathfrak{a}+\mathfrak{b}=\{a+b\,:\,a\in\mathfrak{a},\,b\in\mathfrak{b}\}
\]
是一个理想。由于$\mathfrak{a}$, $\mathfrak{b}\subset \mathfrak{a}+\mathfrak{b}$,所以$\mathfrak{a}\cup\mathfrak{b}\subset \mathfrak{a}+\mathfrak{b}$. 一般而言,两个理想的并不是理想,但是两个理想的并生成的理想实际上就是两个理想的和。

两个理想的交显然还是一个理想(这可以类比两个子群的交还是子群),两个理想$\mathfrak{a}$和$\mathfrak{b}$的乘积$\mathfrak{a}\mathfrak{b}$被定义为集合$\{ab\,:\,a\in\mathfrak{a},\,b\in\mathfrak{b}\}$生成的理想。由于$\mathfrak{a}\mathfrak{b}\subset \mathfrak{a}$以及$\mathfrak{a}\mathfrak{b}\subset \mathfrak{b}$,所以有包含关系$\mathfrak{a}\mathfrak{b}\subset \mathfrak{a}\cap \mathfrak{b}$.

设$(a)$和$(b)$是环$R$的两个主理想,则$(a)(b)=(ab)$. 实际上,左手边的元素由所有$(ra)(sb)=rs(ab)$生成,其中$r$, $s$走遍$R$. 所以$(a)(b)\subset (ab)$. 反过来,因为$ab\in (a)(b)$,所以$(ab)\subset (a)(b)$.

最后,给定$R$的两个理想$\mathfrak{a}$和$\mathfrak{b}$,定义$(\mathfrak{a}:\mathfrak{b})$为$\{r\in R\,:\, r\mathfrak{b}\subset \mathfrak{a}\}$,这是一个理想,因为若$x\in (\mathfrak{a}:\mathfrak{b})$,由于$\mathfrak{a}$是一个理想,则$yx\mathfrak{b}\subset y\mathfrak{a}\subset \mathfrak{a}$或者$xy\in (\mathfrak{a}:\mathfrak{b})$. 当$\mathfrak{a}=(a)$是一个主理想的时候,通常会略去主理想的括号,写作$(a:\mathfrak{b})$,同样,如果$\mathfrak{b}=(b)$是主理想,会写做$(\mathfrak{a}:b)$. $(\mathfrak{a}:\mathfrak{b})$被称为理想$\mathfrak{a}$关于理想$\mathfrak{b}$的商,比如在整数环内,$(6:2)=(3)$.

\para 称一个环$R$是一个整环,如果任取非零的$a$, $b\in R$,可以推出$ab\neq 0$. 一般而言,一个环不是整环。如果$ab=0$但$a$和$b$都不等于零,则这样的$a$或者$b$被称为一个零因子。零显然是一个零因子。而整环就是一个没有非零零因子的环。换而言之,在整环里面,消去律是成立的。

考虑一个单同态$f:R\to S$,如果$S$是一个整环,则$R$也是一个整环。为了证明他,考虑到$ab=0$等价于$f(ab)=f(a)f(b)=0$,由于$S$是整环,所以$f(a)=0$或者$f(b)=0$,再由单同态就得到了结论。作为推论,整环的子环也一定是整环。

如果读者熟悉多项式环,这里可以看到另一大类例子,整环上的多项式环也是整环:设多项式$f$以及$g$都不为零,则它们的首项系数$a$与$b$不为零,因此$fg$的首项系数$ab$不为零,也就是说$fg\neq 0$.

\para 一个理想$\mathfrak{p}$被称为素理想,如果$R/\mathfrak{p}$是一个整环。一个理想$\mathfrak{m}$被称为极大理想,如果$R/\mathfrak{p}$是一个域。由于域是整环,所以极大理想是素理想,反之不然。由于$R=R/(0)$,所以只有在整环里面,零理想才是素理想。

\para 设$R$是一个环,$I(R)$是$R$所有理想的集合,上面按照包含构成了一个偏序,即$\mathfrak{a}\leq \mathfrak{b}$当且仅当$\mathfrak{a}\subset \mathfrak{b}$. 设$\mathcal{F}$是$I(R)$的一个子集,记$\mathcal{F}^c=I(R)-\mathcal{F}$.

极大理想的命名就来自于这个偏序,他是所有真理想构成的集合中的极大元素。对于任意的真理想链,他们的并所生成的理想是真理想。并且由于$R$肯定有一个理想$(0)$,所以由Zorn引理,$R$中存在在上述偏序下极大的理想,下面检验这就是上面说的极大理想。设$\mathfrak{m}$是这样一个理想,$a\notin \mathfrak{m}$,则$(a)+\mathfrak{m}$是一个严格比$\mathfrak{m}$大的理想,由于$\mathfrak{m}$的极大性,没有比他大的真理想了,所以$(a)+\mathfrak{m}=R$. 因此,$1$可以写成$ra+m=1$的形式,在$R/\mathfrak{m}$中即$\bar{r}\bar{a}=1$,所以$\bar{a}$有逆。又因为$a$是在$R-\mathfrak{m}$中任取的,所以$R/\mathfrak{m}$是一个域。

反过来,如果$R/\mathfrak{m}$是一个域,则不存在$\mathfrak{m}$更大的真理想。假设如果存在$\mathfrak{a}$比$\mathfrak{m}$严格大,则有自然的商同态$\pi:R/\mathfrak{m}\to R/\mathfrak{a}$,由于$R/\mathfrak{m}$是一个域且$\pi$是满射,则$\ker \pi$作为域的理想只能是零理想,这样也就推出了$\pi$是一个同构,这与$\mathfrak{a}$比$\mathfrak{m}$严格大矛盾。

从可操作性来看,一开始的定义比这个极大理想的等价定义要方便不少。

\para \label{oka}我们看到,极大理想是素理想。下面我们要推广这个结论。实际上,满足一些条件的极大的理想也会是素理想。而极大的存在性往往是其他条件保证的,比如Zorn引理,再比如极大性条件。

给定一族理想$\mathcal{F}$,对$a\in R$以及$R$的一个理想$\mathfrak{a}$,如果$\mathfrak{a}+(a)$, $(\mathfrak{a}:a)\in \mathcal{F}$可以推出$\mathfrak{a}\in \mathcal{F}$,则称$\mathcal{F}$是Oka理想族\footnote{\url{http://www.bowdoin.edu/~reyes/oka1.pdf}}。对$a$, $b\in R$以及$R$的一个理想$\mathfrak{a}$,如果$\mathfrak{a}+(a)$, $\mathfrak{a}+(b)\in \mathcal{F}$可以推出$\mathfrak{a}+(ab)\in \mathcal{F}$,则称$\mathcal{F}$是Ako理想族。

对于单位理想$(1)$,由于任取$a\in R$都有$a+(1)=(1)$以及$(1:a)=(1)$,所以对任意的Ako理想族或者Oka理想族$\mathcal{F}$,$\mathcal{F}$都包含单位理想。

\pro 如果$\mathcal{F}$是Oka理想族或者Ako理想族,那么$\mathcal{F}^c$中极大的理想是素理想。

\proof
	假设$\pp$是$\mathcal{F}^c$中极大的理想,但不是素理想。因为$\pp$不是单位理想,所以存在$a$, $b\not\in \pp$但$ab\in\pp$. 此时$(\pp:a)$和$\pp+(b)$都比$\pp$严格大,因为他们都包含$b$,同样$\pp+(a)$严格比$\pp$大,所以他们都属于$\mathcal{F}$. 但此时$\pp=\pp+(ab)\in \mathcal{F}^c$或者$\pp=\pp+(ab)\not\in \mathcal{F}$,所以$\mathcal{F}$既不是Ako理想族,也不是Oka理想族。矛盾。
\qed

作为应用,先证明如下理想族是Ako或者Oka理想族:

1. 只有单位理想的理想族。

2. $R$的所有有限生成理想构成的集合。

3. $R$的所有主理想的理想构成的集合。

4. 设$S\subset R$是一个对乘法封闭的子集,$R$中与$S$相交的理想构成的集合。

\proof
	1. 这是Ako理想族。如果$\mathfrak{a}+(a)=(1)$以及$\mathfrak{a}+(b)=(1)$,那么存在$x$, $y\in \mathfrak{a}$使得$x+a=1$以及$y+b=1$,于是$ab=(1-x)(1-y)$或者$(x+y-xy)+ab=1$,所以$\mathfrak{a}+(ab)=(1)$. 也是Oka理想族,如果$(\mathfrak{a}:a)=(1)$,那么$a\in \mathfrak{a}$,此时$\mathfrak{a}+(a)=\mathfrak{a}$. 因此$\mathfrak{a}+(a)=1$也推出了$\mathfrak{a}=(1)$.

	2. 这是Oka理想族。如果$\mathfrak{a}+(a)$和$(\mathfrak{a}:a)$是有限生成理想。设$\mathfrak{a}+(a)=\langle x_i+r_ia \rangle_{i\in I}$以及$(\mathfrak{a}:a)=\langle y_j\rangle_{j\in J}$,其中$I$, $J$都是有限的指标集。任取$x\in \mathfrak{a}$,由于$x\in \mathfrak{a}+(a)$,所以
	\[
		x= \sum_{i\in I}s_i(x_i+r_ia)=\sum_{i\in I}s_ix_i+\sum_{i\in I}s_ir_ia,
	\]
	由于$a\sum_{i\in I}s_ir_i=x-\sum_{i\in I}s_ix_i\in \mathfrak{a}$,因此$\sum_{i\in I}s_ir_i\in (\mathfrak{a}:a)$,可以被$\{y_j\}_{j\in J}$生成。所以$\mathfrak{a}$被$\{x_i$, $y_ja\}_{i\in I,j\in J}$生成。

	3. 这是Oka理想族。如果$\mathfrak{a}+(a)=(b)$是主理想且$(\mathfrak{a}:a)=(c)$也是主理想,如果$a\in \mathfrak{a}$,则$\mathfrak{a}=(b)$是主理想。假设$a\not\in\mathfrak{a}$,存在$x\in \mathfrak{a}$使得$x+a=b$,所以$bc=xc+ac\in \mathfrak{a}$,或者$(bc)\subset \mathfrak{a}$. 反过来,任取$x\in \mathfrak{a}$,由于$\mathfrak{a}+a=\mathfrak{b}$,所以存在$r$, $s\in R$使得$x=rb$以及$a=sb$. 因此$ra=rsb=sx\in\mathfrak{a}$给出了$r\in (\mathfrak{a}:a)=(c)$,于是$y=rb\in (c)(b)=(bc)$.

	4. 这是Ako理想族。设$\pp+(a)$与$S$交于$x$,$\pp+(b)$与$S$交于$y$,则$xy\in (\pp+(a))(\pp+(b))\subset \pp+(ab)$,但由于$S$对乘法封闭,$xy\in S$,所以$(\pp+(ab))\cap S\neq \varnothing$.
\qed

所以上述命题告诉我们:在保证存在性的前提下,

	1. 极大理想是素理想。

	2. 所有非有限生成的理想中极大的是素理想。

	3. 所有非主理想的理想中极大的是素理想。

	4. 所有与某乘性子集不交的理想中极大的是素理想。

\pro \label{primeav}设$R$是一个环,而$\mathfrak{a}_1$, $\cdots$, $\mathfrak{a}_n$是一族理想,还有一个理想$\mathfrak{b}$满足$\mathfrak{b}\subset \bigcup_i \mathfrak{a}_i$. 如果$\mathfrak{a}_1$, $\cdots$, $\mathfrak{a}_n$中至多只有两个不是素理想,则存在一个$i$使得$\mathfrak{b}\subset \mathfrak{a}_i$. 如果将包含改成等号,命题依然成立。

\proof
	如果$\mathfrak{b}=\bigcup_i \mathfrak{a}_i$,则由包含的命题可知,存在一个$i$使得$\mathfrak{b}\subset \mathfrak{a}_i$. 反过来,$\mathfrak{a}_i\subset \bigcup_i \mathfrak{a}_i=\mathfrak{b}$. 等号的命题得证。

	反证,如果$\mathfrak{b}\subset \bigcup_i \mathfrak{a}_i$但不存在一个$i$使得$\mathfrak{b}\subset \mathfrak{a}_i$. 适当重排顺序,可以假设前$n-2$个理想是素理想。我们采用有限归纳制造矛盾,素理想的假设将会自然地出现。为此,假设$\mathfrak{b}$不能包含于少于$n$个的$\mathfrak{a}_i$的并里面。这就是说,我们能找到$x_i\in \mathfrak{b}$使得$x_i\not\in \bigcup_{j\neq i}\mathfrak{a}_j$,由于$\mathfrak{b}\subset \bigcup_{i=1}^n \mathfrak{a}_i$,所以$x_i\in \mathfrak{a}_i$.

	考虑$x_1+x_2\cdots x_n\in \mathfrak{b}$,如果$\mathfrak{a}_1$是素理想,则$x_2\cdots x_n\not\in \mathfrak{a}_1$,同时由于$x_1\not\in \mathfrak{a}_2$, $\cdots$, $\mathfrak{a}_n$,所以对于任意的$1\leq i \leq n$有$x_1+x_2\cdots x_n\not\in \mathfrak{a}_i$,或者说$x_1+x_2\cdots x_n\not\in \bigcup_{i=1}^n \mathfrak{a}_i$,这与$x_1+x_2\cdots x_n\in \mathfrak{b}$矛盾。

	所以$\mathfrak{b}$必然处于更小的并里面,由反证假设$\mathfrak{b}\not\subset \mathfrak{a}_1$,所以$\mathfrak{b}\subset \bigcup_{i=2}^n \mathfrak{a}_i$. 重复上述论证,假设$\mathfrak{a}_2$是素理想,我们就可以推知$\mathfrak{b}\subset \bigcup_{i=3}^n \mathfrak{a}_i$直到$\mathfrak{b}\subset \mathfrak{a}_n$,这与反证假设矛盾,因此得证。 

	注意到,$x_{n-1}\not\in \mathfrak{a}_n$和$x_{n}\not\in \mathfrak{a}_{n-1}$就足以推出$x_{n-1}+x_{n}\not\in \mathfrak{a}_{n-1}\cup \mathfrak{a}_n$,并不需要素理想的假设,所以命题中才会有至多只有两个不是素理想。
\qed

\pro \label{primeau}设$\mathfrak{p}$是环$R$的一个素理想,而$\mathfrak{a}_1$, $\cdots$, $\mathfrak{a}_n$是一族理想,如果$\bigcap_i \mathfrak{a}_i\subset \pp$,则存在一个$i$使得$\mathfrak{a_i}\subset \mathfrak{p}$. 如果将包含改成等号,命题依然成立。

\proof
	如果$\mathfrak{p}=\bigcap_i \mathfrak{a}_i$,由包含的命题,存在一个$i$使得$\mathfrak{a_i}\subset \mathfrak{p}$. 反过来,$\mathfrak{p}=\bigcap_i \mathfrak{a}_i\subset \mathfrak{a}_i$. 所以我们只要证明包含的命题。假设任取$i$都有$\mathfrak{a}_i\not\subset \mathfrak{p}$,因此存在$x_i\in\mathfrak{a}_i$但$x_i\not\in \mathfrak{p}$. 于是对$x_1\cdots x_n\in \bigcap_{i} \mathfrak{a}_i$,因为$\pp$是素理想,所以$x_1\cdots x_n\not\in \pp$. 矛盾。
\qed

\para 如果对仿射簇理论(见Section \ref{variety})有所了解,则可以将他们翻译成几何的语言:将理想改成代数集、素理想改成仿射簇、包含改成包含于、并改成交。于是有:
\begin{itemize}
\item 如果一个代数集包含几个仿射簇的交,则在这些仿射簇中存在一个仿射簇包含于这个代数集中。
\item 如果一个仿射簇包含于几个代数集的并中,则在这些代数集中必然存在一个代数集包含这个仿射簇。
\end{itemize}

\para 设$f:R\to S$是一个环同态,如果$\mathfrak{p}$是$S$中的一个(素)理想,则$f^{-1}(\mathfrak{p})$是一个(素)理想。

\proof
	任取$a\in f^{-1}(\mathfrak{p})$以及$r\in R$,由于$f(a)\in \mathfrak{p}$,所以$f(r)f(a)=f(ra)\in \mathfrak{p}$,这也就推出了$ra\in f^{-1}(\mathfrak{p})$. 所以$f^{-1}(\mathfrak{p})$是一个理想。

	设$\pi:S\to S/\mathfrak{p}$是商同态,我们考虑复合映射$\pi\circ f:R\to S/\mathfrak{p}$,由于$f^{-1}(\mathfrak{p})\subset \ker(\pi\circ f)$,所以由商环的泛性质,$\pi\circ f$诱导出了单同态\[R/f^{-1}(\mathfrak{p})\to S/\mathfrak{p},\]
	单性从这里看出:如果$f(r_1)-f(r_2)=\mathfrak{p}$,则$r_1-r_2\in f^{-1}(\mathfrak{p})$. 

	当$\mathfrak{p}$是一个素理想的时候,$S/\mathfrak{p}$是整环,单同态$R/f^{-1}(\mathfrak{p})\to S/\mathfrak{p}$告诉我们$R/f^{-1}(\mathfrak{p})$也是整环,所以$f^{-1}(\mathfrak{p})$也是素理想。
\qed

\para 上面看到了理想的原像一定是一个理想,反过来,一般来说,一个理想的像不一定是一个理想。比如含入同态$\mathbb{Z}\hookrightarrow \mathbb{Q}$下,理想$(2)$的像不是理想。

但是,对于商映射,情况会好很多。设$\pi:R\to R/\mathfrak{a}$是一个商映射,而$\mathfrak{b}$是$R$中的一个理想,则$\bar{\mathfrak{b}}=\pi(\mathfrak{b})$是$R/\mathfrak{a}$中的一个理想。如果$\mathfrak{p}$是包含$\mathfrak{a}$的素理想,则$\bar{\mathfrak{p}}$也是一个素理想。

证明是朴实的,任取$a\in \mathfrak{b}$,以及$r \in R$,由于$ra\in \mathfrak{b}$,我们也就推出了$\bar{r}\bar{a}=\overline{ra}\in \bar{\mathfrak{b}}$. 所以我们可以考虑这样的商映射$\psi: R/\mathfrak{a}\to (R/\mathfrak{a})/\bar{\mathfrak{b}}$,他与商映射$\pi$复合可以得到满同态
\[
	\psi\pi:R\to (R/\mathfrak{a})/\bar{\mathfrak{b}},
\]
注意到$\psi\pi(r)=0$当且仅当$\bar{r}\in \bar{\mathfrak{b}}$,所以$\ker(\psi\pi)=\pi^{-1}(\bar{\mathfrak{b}})=\mathfrak{a}+\mathfrak{b}$. 由同构基本定理,有同构
\[
	R/(\mathfrak{a}+\mathfrak{b})\cong (R/\mathfrak{a})/\bar{\mathfrak{b}}.
\]

\para 利用上面这个观察,我们可以对商映射下的理想做出如下断言:$R/\mathfrak{a}$中的(素)理想一一对应着包含$\mathfrak{a}$的(素)理想,通过$\bar{\mathfrak{b}}\to \pi^{-1}(\bar{\mathfrak{b}})$.

\proof 
	由于$\pi$是一个满射,所以有等式$\pi(\pi^{-1}(\bar{\mathfrak{b}}))=\bar{\mathfrak{b}}$. 剩下我们要证明,如果$\mathfrak{b}\supset \mathfrak{a}$,则$\pi^{-1}(\pi(\mathfrak{b}))=\mathfrak{b}$,而这来自于$\pi^{-1}(\pi(\mathfrak{b}))=\mathfrak{a}+\mathfrak{b}$. 如果$\mathfrak{p}$是包含$\mathfrak{a}$的素理想,则$\mathfrak{a}+\mathfrak{p}=\mathfrak{p}$,上述同构写成$R/\mathfrak{p}\cong (R/\mathfrak{a})/\bar{\mathfrak{p}}$,因此$\bar{\mathfrak{p}}$也是素理想。
\qed

\para 任取环同态$f:R\to S$,我们可以做出如下分解$f:R\to f(R)\hookrightarrow S$,其中满同态$R\to f(R)$的结构我们是清楚的,因为我们可以利用同构$f(R)\cong R/\ker(f)$将它变成商同态$R\to R/\ker(f)$的情况。所以一般而言,含入同态才是造成理想的像不是理想的障碍,正如前面我们举的例子,含入同态$\mathbb{Z}\hookrightarrow \mathbb{Q}$下,理想$(2)$的像不是理想。

\para 定义一个理想$\mathfrak{a}$的根$\sqrt{\mathfrak{a}}$如下:
\[
	\sqrt{\mathfrak{a}}=\{r\in R\,:\,\exists n\in \mathbb{Z}^+\text{ s.t. }r^n\in \mathfrak{a}\}.
\]

一个理想的根依然是一个理想,检查中困难的是加法,设$a^n\in \aaa$和$b^m\in \aaa$,则$(a+b)^{m+n}$在二项式展开后可以发现,每一项都属于$\aaa$,所以$a+b\in \sqrt{\aaa}$.

一个理想的根其实是所有包含它的素理想的交。实际上,设$A$是所有包含$\mathfrak{a}$的素理想的交,设$\pp$是任意一个包含$\mathfrak{a}$的素理想,如果$f^n\in \mathfrak{a}$,则$f^n\in \pp$,由于$\pp$是素理想,所以$f^n=f\cdot f^{n-1}$给出$f\in\pp$或者$f^{n-1}\in\pp$,通过归纳法就有$f\in \pp$. 这就给出了$\sqrt{\mathfrak{a}}\subset A$. 反过来,如果$f\not\in \sqrt{a}$,考虑所有与$\{1,f,f^2,\cdots\}$不交的,但包含$\mathfrak{a}$的所有理想中极大的那个理想$\pp$,这是一个素理想,存在性来自于Zorn引理,所以$f\not\in \pp$. 这就给出了$A\subset \sqrt{\mathfrak{a}}$.