\chapter{Foundation}

\section{Basic Structures}

这节就是罗列定义。

\para 设有两个集合$X$和$Y$,则称映射$f:X\times Y \to X$为集合$X$上的一个右作用。称映射$g:Y\times X \to X$为集合$X$上的一个左作用。称映射$h:X\times X \to X$为集合$X$上的一个二元运算。同样地,我们可以定义多元运算。

可以看到$\mathbb{R}$上的加法$f:\mathbb{R}\times \mathbb{R} \to \mathbb{R}$定义为$f(a,b)=a+b$是一个二元运算,同样可以检验乘法。再比如设$Y$是集合$X$上所有双射$f:X\to X$的集合,那么映射复合构成$Y$上的一个运算。

对于一个未知的运算或者作用,我们通常称之为“乘法”。左作用称为“左乘”,右作用称为“右乘”。对于运算$f(a,b)$通常直接记作$a*b$,或者再干脆一些省略中间的符号$*$,记作$ab$,读作$a$左乘$b$、$b$右乘$a$或$a$乘以$b$.

我们定义多个元素$\{a_i\,:\, 1\leq i \leq n\}$的乘法
\[
	a_1a_2\cdots a_n=a_1(a_2a_3\cdots a_n)=a_1(a_2(a_3\cdots a_n))=\cdots,
\]
这是一个递归定义。

\para 设集合$X$上有一个运算$h:X\times X \to X$,我们将$h(a,b)$直接记作$ab$. 如果任取$a$, $b$, $c\in X$都成立
\[
	(ab)c=a(bc),
\]
则该运算被称为是满足结合律(associative property)的。

\para 对于在字符串$a_1a_2\cdots a_n$中任意加括号构成的字符串,比如
\[
(a_1(a_2(a_3a_4)a_5))(a_6(a_7a_8a_9))a_{10},
\]
先预设一个$n=0$,我们我们从左往右开始计数,遇到`$($'则$n\to n+1$,并称该括号为第$n$层括号,遇到`$)$'则$n\to n-1$,直到最后一个字符。比如上式中,第一个和第四个`$($'都是第一层括号,而第二个和第五个`$($'是第二层括号。如果到达最右端字符的时候,得到了$n=0$,则上面的字符串称为$a_1a_2\cdots a_n$的一种加括号的方式。

\pro 设$\{a_i\in A\,:\, 1\leq i\leq n\}$是$A$中任意的一个有限集,且$A$上存在一种运算满足结合律。对于任意字符串$a_1a_2\cdots a_n$中任意加括号的方式,经过运算后,他们都等于$a_1a_2\cdots a_n$.

\proof 对正整数$n$,记$[n]=\{1,2,\cdots,n\}$。设$f:[m]\to [n]$是一个递增函数,则$f$被称为$[n]$的一个$m$-划分。由于是递增函数,所以$m\leq n$. 

考虑$\{a_i\, :\, 1\leq i\leq n\}$,对于$[n]$的任意一个$(2m)$-划分$k$,$k(i)=k_i$,我们都可以通过在在每一对$a_{k_{2i-1}}$和$a_{k_{2i}}$之间加一对括号定义出一个字符串
\[
	a_1a_2\cdots (a_{k_1}\cdots a_{k_2})a_{k_2+1}\cdots (a_{k_3}\cdots a_{k_4})\cdots (a_{k_i}\cdots a_{k_{i+1}})\cdots a_n,
\]
一个$m$对括号,这个字符串的最大层数是$1$。现在,当这个字符串看作元素之间相乘的时候,我们可以证明此时他等于$a_1a_2\cdots a_n$.

为此,我们先证明如下特殊情况:对任意的正整数$n$和$k$以及元素$\{a_i\,:\, 1\leq i \leq n\}$成立
\[(a_1a_2\cdots a_{k})(a_{k+1}\cdots a_{n})=a_1a_2\cdots a_n.\]
首先注意到
\[
(a_1a_2\cdots a_{k})(a_{k+1}\cdots a_{n})=(a_1(a_2\cdots a_{k}))(a_{k+1}\cdots a_{n})=a_1((a_2\cdots a_{k})(a_{k+1}\cdots a_{n})),
\]
第二个等号来自于结合律。然后对$(a_2\cdots a_{k})(a_{k+1}\cdots a_{n})$进行同样的操作,如是归纳下去,由于$n$是有限的,所以归纳也是有限的,我们就可以得到
\[
(a_1a_2\cdots a_{k})(a_{k+1}\cdots a_{n})=a_1(a_2(a_3(\cdots (a_{k}(a_{k+1}\cdots a_{n})))))=a_1a_1a_2\cdots a_n.
\]

现在,假设有$m$对括号,多重乘积可能出现三种情况:一是$a_1a_2\cdots a_{k-1}(a_{k}\cdots a_n)$,即最后一对括号出现在最后,那么按照定义,他就等于$a_1a_2\cdots a_{k-1}a_{k}\cdots a_n$,这样我们就消去了最后一对括号。二是$a_1a_2\cdots a_{k-1}(a_{k}\cdots a_l)a_{l+1}\cdots a_n$,其中$a_1a_2\cdots a_{k-1}$中有$(m-1)$对括号,或者说$(a_{k}\cdots a_l)$是最后一对括号。我们利用定义和上面的结论可以得到
\begin{align*}
	a_1a_2\cdots a_{k-1}(a_{k}\cdots a_l)a_{l+1}\cdots a_n&=a_1a_2\cdots ((a_{k}\cdots a_l)(a_{l+1}\cdots a_n))\\
	&=a_1a_2\cdots a_{k-1}(a_{k}\cdots a_n)\\
	&=a_1a_2\cdots a_{k-1}a_{k}\cdots a_n.
\end{align*}
这样我们就消去了最后一对括号。如是往复,我们就可以消去全部的括号,得到$a_1\cdots a_n$.

我们已经对任意字符串$a_1a_2\cdots a_n$最大括号层数为$1$的所有可能加括号方式证明了,在满足结合律的时候等于$a_1a_2\cdots a_n$.

先设最大的层数为$N$,对所有的$(N-1)$层括号,他里面的最大括号层数为1,所以利用上面的结论,我们可以将这层内所有的括号消去,则最大层数降为$(N-1)$,如是往复,经过有限次归纳,就得到了他最终将等于$a_1a_2\cdots a_n$.\qed 

这个结论就是说,对于满足结合律的运算,有限个元素相乘的结果不依赖于他加括号的方式。

\para 有一个非空集合$G$和其上的二元运算$*$,或者记作$(G,*)$,称为一个群(group),如果满足:

\no{1}结合律:对于任意$a$, $b$, $c\in G$,有$(a*b)*c=a*(b*c)$;

\no{2}单位元:对于任意的$a\in G$,存在一个元素$e\in G$,使得$e*a=a*e=a$;

\no{3}反元素:对于每一个$a\in G$,存在一个元素$b\in G$,使得$b*a=a*b=e$.

$a$的反元素通常记作$a^{-1}$.此外如果不产生歧义,我们可以直接称呼$G$为一个群。显然$(\mathbb{R},+)$是一个群,单位元是0,$a^{-1}=-a$.将$\mathbb{R}$去掉了$0$之后的集合记作$\mathbb{R}-\{0\}$,则$(\mathbb{R}-\{0\},\cdot)$构成一个群,单位元是$1$,$a^{-1}=1/a$.

如果有群$(G,*)$,对于任意的$a$, $b\in G$满足$a*b=b*a$,则称$(G,*)$是一个交换群,或者称为Abel群。交换群的运算一般是记作加法的。

\para 我们称呼一个三元组$(R,+,\cdot)$为一个环(ring),如果满足下列性质:

\no{1} $(R,+)$构成一个交换群,运算称为加法,其中的单位元记作$0$,$a$的反元素记作$-a$;

\no{2} $(R,\cdot)$中的运算满足结合律,称为乘法。

\no{3} 乘法满足分配律:对任意$a$, $b$, $c \in R$,成立$a\cdot(b+c)=a\cdot b+a\cdot c$以及$(b+c)\cdot a=b\cdot a+c\cdot a$.

同样地,如果不产生歧义,我们可以直接称呼$R$为一个环,而且略去乘法的符号。任取$a\in R$,由于
\[
	0a=(0+0)a=0a+0a,
\]
所以$0a=0$,同理$a0=0$.

如果$R-\{0\}$中含有乘法的单位元,则称$R$是一个含幺环。如果$R$中的乘法是可交换的,则称$R$是一个交换环。

\para 设$(M,+)$是一个交换群,$R$是一个含幺环,如果有一个左乘$\mu:R\times M\to M$,左乘符号下面省略,使得对任意的$r$, $s\in R$以及$m$, $n\in M$,成立
\[
	(rs)m=r(sm),\quad r(m+n)=rm+rn,\quad (r+s)m=rm+sm,\quad 1m=m,
\]
则称呼$M$是一个左$R$-模,同理可以定义右$R$-模。任意的含幺环$R$显然是一个$R$-模。

由于
\[
	0m=(0+0)m=0m+0m,
\]
所以$0m=0$,注意等号两边的零是不同的。类似地,可以证明$r0=0$. 

\para $\zz$显然是一个交换群,当然也是一个含幺交换环。不太平凡的一点是,任意的交换群$G$可以看成$\zz$-模。这是因为对任意的$n\in \zz$和任意的$a\in G$,我们做出如下定义:如果$n>0$,定义$na$为$n$个$a$相加,如果$n=0$,定义$na=0$,如果$n<0$,定义$na=-((-n)a)$. 不难检验这是确实是一个$\zz$-模。

\para 对于环$(k,+,\cdot)$,如果$(k-\{0\},\cdot)$是一个交换群,则环$(k,+,\cdot)$被称为一个域。显然,域是一个含幺交换环。任意$k$-模$V$被称为$k$-矢量空间,$V$中的元素被称为矢量。