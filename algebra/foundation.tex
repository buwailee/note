\chapter{Foundation}

\section{Basic Structures}

\para 设有两个集合$X$和$Y$,则称映射$f:X\times Y \rightarrow X$为集合$X$上的一个右作用。称映射$g:Y\times X \rightarrow X$为集合$X$上的一个左作用。称映射$h:X\times X \rightarrow X$为集合$X$上的一个二元运算。同样地,我们可以定义多元运算。

可以看到$\mathbb{R}$上的加法$f:\mathbb{R}\times \mathbb{R} \rightarrow \mathbb{R}$定义为$f(a,b)=a+b$是一个二元运算,同样可以检验乘法。再比如设$Y$是集合$X$上所有双射$f:X\rightarrow X$的集合,那么映射复合构成$Y$上的一个运算。

对于一个未知的运算或者作用,我们通常称之为“乘法”。左作用称为“左乘”,右作用称为“右乘”。对于运算$f(a,b)$通常直接记作$a*b$,或者再干脆一些省略中间的符号$*$,记作$ab$,读作$a$左乘$b$、$b$右乘$a$或$a$乘以$b$.

可以清晰地看到,对于$X$上的二元运算的$a*b$其结果$ab$也是在$X$里面的。

\para 有一个非空集合$G$和其上的二元运算$*$,或者记作$(G,*)$,称为一个群,如果满足:

\no{1}结合律:对于任意$a$, $b$, $c\in G$,有$(a*b)*c=a*(b*c)$;

\no{2}单位元:对于任意的$a\in G$,存在一个元素$e\in G$,使得$e*a=a*e=a$;

\no{3}反元素:对于每一个$a\in G$,存在一个元素$b\in G$,使得$b*a=a*b=e$.

$a$的反元素通常记作$a^{-1}$.此外如果不产生歧义,我们可以直接称呼$G$为一个群。显然$(\mathbb{R},+)$是一个群,单位元是0,$a^{-1}=-a$.将$\mathbb{R}$去掉了$0$之后的集合记作$\mathbb{R}-\{0\}$,则$(\mathbb{R}-\{0\},\cdot)$构成一个群,单位元是$1$,$a^{-1}=1/a$.

\para 如果有群$(G,*)$,对于任意的$a,b\in G$满足$a*b=b*a$,则称$(G,*)$是一个交换群,或者称为Abel群。

\para 我们称呼一个三元组$(k,+,\cdot)$为一个域,如果满足下列性质:

\no{1} $(k,+)$构成一个交换群,运算称为加法,其中的单位元记作$0$,$a$的反元素记作$-a;$

\no{2} $(k-\{0\},\cdot)$构成一个交换群,运算称为乘法,其中的单位元记作$1$,$a$的反元素记作$a^{-1};$

\no{3}分配律:对所有$a,b,c \in k$,$a\cdot(b+c)=a\cdot b+a\cdot c$.

同样地,如果不产生歧义,我们可以直接称呼$k$为一个域。我们刚刚已经检查过了$\mathbb{R}$满足前两条性质,第三条分配律也是成立的,故而$\mathbb{R}$是一个域。所以我们通常称呼其为实数域十分合理。