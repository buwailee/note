\appendix
\renewcommand{\thepara}{\Alph{chapter}.\arabic{para}}

\chapter{Zorn引理与等价关系}
\ThisULCornerWallPaper{1}{../Pictures/19.png}
\section{Zorn引理}
\para 设$A$是一个集合,而$\leq$是$A$上的一个关系,如果满足

\begin{compactenum}[~~~(1)]
\item $\forall x\in A$, $x\leq x$;
\item 如果$x\leq y$以及$y\leq z$,则$x\leq z$;
\item 如果$x\leq y$以及$y \leq x$,则$x=y$.\\则此时我们称$(A,\leq)$是一个偏序集,称$\leq$是$A$的一个偏序。为方便,有时候会直接说$A$是一个偏序集。
\end{compactenum}

在记号上,如果$x\leq y$,则我们有时候也会记做$y\geq x$. 可以看到,如果$(A,\leq)$是一个偏序集,则$(A,\geq)$也是一个偏序集。

显而易见地,在实数域$\rr$上有一个自然的偏序,就是通常的小于等于,或者是大于等于。特别地一点是,实数域上任意两个数是可以比较的,即对$x$, $y\in \rr$,一定成立$x\leq y$或者$y\leq x$,但对于一般的偏序集而言,任意两个元素不一定可以比较。

如果一个偏序集的任意两个元素是可以相互比较的,则这个偏序集被称为全序的,或者叫全序集。实数域当然是一个全序集。再比如,赋值环的所有理想按照包含给出的偏序是全序的。

\para 设$(A,\leq)$是一个偏序集,而$T$是一个$A$的子集,则$(T,\leq)$自然也是一个偏序集。如果$T$中的任意两个元素可以比较,这就称呼$T$是一个链。特别地,如果$A$本身就是一个链,则$A$称为一个全序集。因此链又叫做全序子集。

\para 所谓链$T\subset A$的上界,就是某个$a\in A$,使得$x\leq a$对任意的$x\in T$都成立。下界对应的就是使得$a\leq x$对任意的$x\in T$都成立的$a\in A$。显然,上下界不一定唯一。

\para 设$B$是$A$的一个子集,设$x\in B$,如果对于$B$中任意可以与$x$比较的元素$a$,都有$a\leq x$,则称呼$x$是$B$的一个极大元。同理可以有极小元。

极大元可能不存在,也可能存在很多个。比如设除去$1$的正整数集合$\zz^+-\{1\}=\{2$, $3$, $\cdots\}$,我们如下赋予一个偏序:如果$a$整除$b$(通常记做$a|b$),则$b\leq a$。那么任意的素数$p$都是$\zz^+$中的极大元,因为任意可以与$p$比较的元素都是$np$的形式,因为$p|np$,所以$np\leq p$.

\para 设$B$是$A$的一个子集,设$x\in B$,如果对于$B$中任意的元素$a$,都有$a\leq x$,则称呼$x$是$B$的一个最大元。同理可以有最小元。

最大元不一定存在,但如果存在显然只能有一个,如果有两个,$a$或者$b$,则$a\leq b$以及$b\leq a$,所以$a=b$. 对于全序集来说,极大元和最大元等价。

\begin{thm}[Zorn引理]
设$(A,\leq)$是一个非空偏序集,且其中的每一条链都存在上界,则$A$中存在极大元。
\end{thm}

Zorn引理是选择公理的等价形式,这里不做证明了。

\section{等价关系}
\para 一个关系$\sim$被称为集合$S$上的等价关系,如果
\[
	a\sim a,
\]
\[
	a\sim b \Rightarrow b \sim a,
\]
\[
	a\sim b\text{ and } b\sim c \Rightarrow a \sim c,
\]
第一个被称为自反性,第二个被称为对称性,第三个被称为传递性。

\para 一旦给定集合$S$上的一个等价关系$\sim$,我们把与$a$等价的那些元素构成的子集记作$\bar{a}$. 从等价关系的三个性质可以看出:
\begin{compactitem}
\item $a$与$b$等价当且仅当$\bar{a}=\bar{b}$. 
\item $\bigcup_{a\in S}\bar{a}=S$.
\item 如果$\bar{a}\cap \bar{b}\neq \varnothing$,则$\bar{a}=\bar{b}$.
\end{compactitem}
这样的子集被称为等价类。从上面的第二点可以看出,不同的等价类不相交,所以通过等价关系,我们将$S$划分成了许多不相交的子集。满足上面性质的后两点的子集族被称为$S$的一个划分。

\para 反过来,如果我们给定了$S$的一个划分,则可以定义出一个等价关系:$a$和$b$等价当且仅当他们同处于一个划分$S_i$里面。而划分在这个等价关系里面就构成了等价类。所以等价关系与划分等价。
