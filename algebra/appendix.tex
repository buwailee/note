\appendix
\renewcommand{\thepara}{\Alph{chapter}.\arabic{para}}

\chapter{}
\ThisULCornerWallPaper{1}{../Pictures/19.png}

\section{Zorn引理}
\para 设$A$是一个集合,而$\leq$是$A$上的一个关系,如果满足

\begin{compactenum}[~~~(1)]
\item $\forall x\in A$, $x\leq x$;
\item 如果$x\leq y$以及$y\leq z$,则$x\leq z$;
\item 如果$x\leq y$以及$y \leq x$,则$x=y$.\\则此时我们称$(A,\leq)$是一个偏序集,称$\leq$是$A$的一个偏序。为方便,有时候会直接说$A$是一个偏序集。
\end{compactenum}

在记号上,如果$x\leq y$,则我们有时候也会记做$y\geq x$. 可以看到,如果$(A,\leq)$是一个偏序集,则$(A,\geq)$也是一个偏序集。

显而易见地,在实数域$\rr$上有一个自然的偏序,就是通常的小于等于,或者是大于等于。特别地一点是,实数域上任意两个数是可以比较的,即对$x$, $y\in \rr$,一定成立$x\leq y$或者$y\leq x$,但对于一般的偏序集而言,任意两个元素不一定可以比较。

如果一个偏序集的任意两个元素是可以相互比较的,则这个偏序集被称为全序的,或者叫全序集。实数域当然是一个全序集。再比如,赋值环的所有理想按照包含给出的偏序是全序的。

\para 设$(A,\leq)$是一个偏序集,而$T$是一个$A$的子集,则$(T,\leq)$自然也是一个偏序集。如果$T$中的任意两个元素可以比较,这就称呼$T$是一个链。特别地,如果$A$本身就是一个链,则$A$称为一个全序集。因此链又叫做全序子集。

\para 所谓链$T\subset A$的上界,就是某个$a\in A$,使得$x\leq a$对任意的$x\in T$都成立。下界对应的就是使得$a\leq x$对任意的$x\in T$都成立的$a\in A$。显然,上下界不一定唯一。

\para 设$B$是$A$的一个子集,设$x\in B$,如果对于$B$中任意可以与$x$比较的元素$a$,都有$a\leq x$,则称呼$x$是$B$的一个极大元。同理可以有极小元。

极大元可能不存在,也可能存在很多个。比如设除去$1$的正整数集合$\zz^+-\{1\}=\{2$, $3$, $\cdots\}$,我们如下赋予一个偏序:如果$a$整除$b$(通常记做$a|b$),则$b\leq a$。那么任意的素数$p$都是$\zz^+$中的极大元,因为任意可以与$p$比较的元素都是$np$的形式,因为$p|np$,所以$np\leq p$.

\para 设$B$是$A$的一个子集,设$x\in B$,如果对于$B$中任意的元素$a$,都有$a\leq x$,则称呼$x$是$B$的一个最大元。同理可以有最小元。

最大元不一定存在,但如果存在显然只能有一个,如果有两个,$a$或者$b$,则$a\leq b$以及$b\leq a$,所以$a=b$. 对于全序集来说,极大元和最大元等价。

\begin{thm}[Zorn引理]
设$(A,\leq)$是一个非空偏序集,且其中的每一条链都存在上界,则$A$中存在极大元。
\end{thm}

Zorn引理是选择公理的等价形式,这里不做证明了。

\section{Galois联络}

\begin{para}
设$S$和$T$是两个偏序集,并且存在两个映射$f:S\to T$以及$g:T\to S$满足
\begin{compactenum}[~~~~(1)]
\item 如果$s_1\leq s_2$,则$f(s_2)\leq f(s_1)$.
\item 如果$t_1\leq t_2$,则$g(t_2)\leq g(t_1)$.
\item $\,$$t\leq f(g(t))$以及$s\leq g(f(s))$.
\end{compactenum}
则称$S$和$T$之间存在一个Galois联络。
\end{para}

以范畴的观点来看,第一个条件就是说$f:S^{\text{op}}\to T^{\text{op}}$是一个函子,第二个条件就是在说,$g:T^{\text{op}}\to S^{\text{op}}$是一个函子。不仅如此,第三点还说明这是一对伴随函子。

\begin{pro}
设$S$和$T$之间存在一个Galois联络,则$g(T)$和$f(S)$之间存在一一映射,具体写出来即$s\mapsto f(s)$,以及$t \mapsto g(t)$.
\end{pro}

\begin{proof} 
	设$s=g(t)$,则$t\leq f(g(t))=f(s)$,所以$g(f(s))\leq g(t)=s$. 反过来,由于$s\leq g(f(s))$,所以$s=g(f(s))$. 所以$g|_{f(S)}:f(S)\to g(T)$是一个满射。同样(或者由对称性),对于$t=f(s)$,我们也有$t=f(g(t))$,所以$g|_{f(S)}:f(S)\to g(T)$是一个单射。故而$g|_{f(S)}$是一个双射,它的逆就是$f|_{g(T)}$.
\end{proof}

可以摘出证明中顺便得到的等式
\[
	g(t)=g(f(g(t))),\quad f(s)=f(g(f(s))).
\]


\section{等价关系}
\para 一个关系$\sim$被称为集合$S$上的等价关系,如果
\[
	a\sim a,
\]
\[
	a\sim b \Rightarrow b \sim a,
\]
\[
	a\sim b\text{ and } b\sim c \Rightarrow a \sim c,
\]
第一个被称为自反性,第二个被称为对称性,第三个被称为传递性。

\para 一旦给定集合$S$上的一个等价关系$\sim$,我们把与$a$等价的那些元素构成的子集记作$\bar{a}$. 从等价关系的三个性质可以看出:
\begin{compactitem}
\item $a$与$b$等价当且仅当$\bar{a}=\bar{b}$. 
\item $\bigcup_{a\in S}\bar{a}=S$.
\item 如果$\bar{a}\cap \bar{b}\neq \varnothing$,则$\bar{a}=\bar{b}$.
\end{compactitem}
这样的子集被称为等价类。从上面的第二点可以看出,不同的等价类不相交,所以通过等价关系,我们将$S$划分成了许多不相交的子集。满足上面性质的后两点的子集族被称为$S$的一个划分。

\para 反过来,如果我们给定了$S$的一个划分,则可以定义出一个等价关系:$a$和$b$等价当且仅当他们同处于一个划分$S_i$里面。而划分在这个等价关系里面就构成了等价类。所以等价关系与划分等价。

\section{偏序集上的极限}

\para 回忆一下偏序集,设$I$是一个偏序集,偏序关系为$\leq$,那么通过定义态射
\[
	{I}\left(x,y\right)=\begin{cases}
	(x,y)&\text{, if }x\leq y\text{;}\\
	\varnothing&\text{, otherwise}.
	\end{cases}
\]
则$I$是一个小范畴,如果我们定义$x\geq y$当且仅当$y\leq x$,那么新的偏序集$(I,\geq)$就是偏序集$(I,\leq)$的对偶范畴,略去偏序符号,有时候会记作$I^{\mathrm{op}}$.

$D:I\to \mathcal{C}$是一个协变函子,就是对$i\leq j\leq k$成立$D(j\leq k)\circ D(i\leq j)=D(i\leq k)$,就是说从小到大有映射。反之,反变函子$D:I\to \mathcal{C}$对$i\leq j\leq k$成立$D(i\leq j)\circ D(j\leq k)=D(i\leq k)$,就是说从大到小有映射。

对于一个反变函子$D:I\to \mathcal{C}$,我们可以定义一个协变函子$D^{\mathrm{op}} :I^{\mathrm{op}}\to \mathcal{C}$通过$D^{\mathrm{op}}(j\geq i)=D(i\leq j)$,此时对$k\geq j\geq i$成立$D^{\mathrm{op}}(j\geq i)\circ D^{\mathrm{op}}(k\geq j)=D^{\mathrm{op}}(k\geq i)$,可见这确实是一个协变函子。

\para 设$I$是一个指标集,即只存在关系$x\leq x$. 此时构成的$I$-图的极限称为\idx{积}(\idx{product}),余极限称为\idx{余积}(\idx{coproduct})。前面提到过的集合的直积就是一种积。

\para 称$I$是一个滤相的偏序集,即对于任意的$i$, $j\in I$都存在$k\in I$使得$k\leq i$和$k\leq j$同时成立。称$I$是一个定向的偏序集,即对于任意的$i$, $j\in I$都存在$k\in I$使得$i\leq k$和$j\leq k$同时成立。很容易看到,如果偏序集$I$是滤相的(定向的),那么$I^{\mathrm{op}}$就是定向的(滤相的),反之亦然。所以下面的讨论,将固定$I$是一个滤相偏序集,而$I^{\mathrm{op}}$则是一个定向偏序集。

作为例子,考虑拓扑空间中所有非空开集按照包含构成的偏序集,即$U\leq V$当且仅当$U\subset V$,那么这个偏序集是定向的,但不是滤相的,因为对于任意的$U$和$V$总有$U\leq U\cup V$和$V\leq U\cup V$成立,但对于不交的$U$和$V$,并不存在非空开集同时包含于他们其中。同样是拓扑的例子,考虑所有包含$p$的非空开集按照包含构成的偏序集,那么这个偏序集既是滤相的又是定向的。

\para 将滤相的(定向的)偏序集$I$看作一个范畴,对任意的范畴$\mathcal{C}$,如果一个$I$-图$D$,$D:I\to \mathcal{C}$是协变函子,则这称为一个$C$上的一个{逆系统}({定向系统})。对偶地,如果$D$是反变函子,那么$I^{\mathrm{op}}$-图$D^{\mathrm{op}}$是定向系统(逆系统)。

逆系统一般谈论极限${\varprojlim}_{i\in I} D(i)$,而定向系统一般谈论${\varinjlim}_{i\in I} D(i)$,为了思考这个原因,我们考虑如下交换图
\[
	\xymatrix{
		&A \ar[dl]_{\lambda_{i}}\ar[dr]^{\lambda_{j}}\ar[dd]^{\lambda_{k}}&\\
		D(i)&&D(i)\\
		&D(k)\ar[ul]^{D(k\leq i)}\ar[ur]_{D(k\leq j)}&
	}
	\quad
% \xymatrix{
% 	&A \ar[dl]_{\lambda_{i}}\ar[dr]^{\lambda_{j}}\ar[dd]^{\lambda_{k}}&\\
% 	D(i)\ar[dr]_{D(i\leq k)}&&D(i)\ar[dl]^{D(j\leq k)}\\
% 	&D(k)&
% }
	\xymatrix{
		&A &\\
		\ar[ur]^{\mu_{i}}D(i)\ar[dr]_{D(i\leq k)}&&D(i)\ar[ul]_{\mu_{j}}\ar[dl]^{D(j\leq k)}\\
		&D(k)\ar[uu]_{\mu_{k}}&
	}
\]
左边是对逆系统考虑锥形,右边是对定向系统考虑余锥形。从左边来看,如果$i\leq j\leq k$成立,则$D(k)$构成了一个锥形的顶点,当我们考虑极限的时候,这时候极限就应该表现得像那些“极小”的元素$D(k)$一样。而且如果$A$是极限,那么箭头$A\to D(k)$也构成了唯一分解。类似地考虑右边的图,那些“极大”的元素$D(k)$构成了余锥形的顶点,如果$A$是余极限,那么箭头$D(k)\to A$也构成了唯一分解。

\para 设$I$是一个偏序集,如果$i\in I$有,$i>j$对所有$j\in I$不成立,或者说,与可以比较的元素$j\in I$都有$i\leq j$,则称$i$是$I$的一个极小元。如果$i\leq j$对所有$j\in I$都成立,则称$i\in I$是$I$的最小元。

在滤相的偏序集中,极小元和最小元等价。最小元是极小的,这是显然的。反之,因为如果存在极小元$i$,那么任取一个$j\in I$都存在一个$k\in I$使得$k\leq i$和$k\leq j$都成立,然而$i$是极小元,所以$i=k$,所以$i\leq j$. 同理我们可以定义极大元与最大元,在定向的偏序集中,此二者等价。

更广义地,对于一个偏序集$I$的子集$J$,如果对于任意的$i\in I$,都存在$j\in J$使得$j\leq i$,则称$J$和$I$是共尾的。显然,滤相的偏序集中的极小元构成的单点集就和原来的偏序集共尾。

\begin{pro}
设$I$-图$D$是一个逆系统,如果$I$存在一个与其共尾的子集$J$,则$J$-图$D$是一个逆系统,且${\varprojlim}_{i\in I} D(i)={\varprojlim}_{i\in J} D(i)$. 对偶地,设$I$-图$D$是一个定向系统,如果$I^{\mathrm{op}} $存在一个与其共尾的子集$J^{\mathrm{op}} $,则$J$-图$D$是一个定向系统,且${\varinjlim}_{i\in I} D(i)={\varinjlim}_{i\in J} D(i)$. \notprove
\end{pro}

所以对于逆系统,如果存在极小元,那么极小元就是它的极限,对偶地,对于定向系统,如果存在极大元,那么极大元就是它的余极限。这符合我们上面的直观。

\para 一个拓扑空间$X$能被看成一个范畴,对象取作他的所有开集,而态射取作
\[
	{X}(U,V)=\begin{cases}
	\bigl\{i^U_V:U\hookrightarrow V\bigr\}&\text{, if }U\subset V\text{;}\\
	\varnothing&\text{, otherwise}.
	\end{cases}
\]

考虑拓扑空间$X$中一个非空开集族$\mathfrak{B}$,对于任意的两个$U$, $V\in \mathfrak{B}$,都存在一个$W\in \mathfrak{B}$使得$U\cup V\subset W$成立,这样的$\mathfrak{B}$按照包含构成了定向的偏序集。现在定义函子$i:\mathfrak{B}\to X$通过$i(U)=U$以及$i(U\leq V)=i^U_V:U\hookrightarrow V$,它显然成立复合$i(U\leq W)=i(V\leq W)\circ i(U\leq V)$,所以这个$\mathfrak{B}$-图$i$是一个定向系统。可以看到它的余极限${\varinjlim}_{U\in \mathfrak{B}} U$实际上就是$\bigcup_{U\in \mathfrak{B}} U$,而态射族就是$i_U:U\hookrightarrow \bigcup_{U\in \mathfrak{B}} U$.

同样考虑拓扑空间$X$中一个包含一个点$p\in X$的所有非空开集构成的族$\mathfrak{B}$,它按照包含构成一个滤相的偏序集。同样可以定义函子$i:\mathfrak{B}\to X$通过$i(U)=U$以及$i(U\leq V)=i^U_V:U\hookrightarrow V$,同样显然成立复合$i(U\leq W)=i(V\leq W)\circ i(U\leq V)$,所以这个$\mathfrak{B}$-图$i$是一个逆系统。但它的极限${\varprojlim}_{U\in \mathfrak{B}} U$就不一定存在,因为类比于上一个例子,它的极限应该类似于所有$\mathfrak{B}$中元素的交,但这不一定是一个开集。

\para 设我们有一个$X$上的$\mathcal{K}$-预层$\calf$,他是$X\to \mathcal{K}$的一个反变函子。赋予$X$一个偏序,此时函数$U\leq V$即$U\hookrightarrow V$. 那么$X$的那些包含$p\in X$的非空开集构成的子集族$\mathfrak{B}$继承了偏序结构,也是一个范畴,而预层$\calf$限制在$\mathfrak{B}$给出了反变函子$\mathfrak{B}\to \mathcal{K}$.

由于$\mathfrak{B}$是滤相的偏序集,而$\calf$是反变函子,因此$\mathfrak{B}^{\mathrm{op}}$-图$\calf^{\mathrm{op}}$是定向系统,进而我们会考虑余极限
\[
	{\varinjlim}_{U\in \mathfrak{B}^{\mathrm{op}}} \calf^{\mathrm{op}}(U)={\varinjlim}_{U\in \mathfrak{B}^{\mathrm{op}}} \calf(U),
\]
如果这个余极限存在,那么就称为预层$\calf$在点$p$处的纤维,记作$\calf_p$.