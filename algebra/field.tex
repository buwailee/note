\chapter{域与Galois理论}
\section{超越扩张}
在环的整扩张那里已经解决了域的有限扩张的分类问题,即有限扩张就是有限生成扩张。如果一个扩张不是有限扩张,则,要么这个扩张包含超越元,或者,他是代数扩张却不能由有限次单代数扩张而成。这节对前者进行讨论。

\para 设$K/k$是一个扩张,设$t_1$, $t_2$, $\cdots$, $t_n\in K$,如果任取$f\in k[x_1,\cdots,x_n]$都有$f(t_1,\cdots,t_n)$非零,则称呼$\{t_1$, $\cdots$, $t_n\}$是代数无关的,反之则称为相关的。很容易从定义看出,代数无关的集合的子集也是代数无关的。对一个$K$的无穷子集$S$,如果$S$的任意有限子集都在$k$上是代数无关的,则$S$被称作在$k$上代数无关。

从代数无关的定义来看,如果有一个$t_i$在$k$上代数,则$\{t_1$, $\cdots$, $t_n\}$是代数相关的。所以一个代数无关组必然都是超越元。

\para 设$K/k$是一个扩张,一个$K$中的元素$t$被称为在$\{u_1,\cdots ,u_n\}$上关于域$k(u_1,\cdots ,u_n)$代数相关的,就是说存在一个非零多项式$f\in k[u_1,\cdots ,u_n][x]$,使得$f(t)=0$。

他有如下性质:

\no{1} 因为存在$f(x)=u_i-x$,所以$u_i$是代数相关的。

\no{2} 如果$x$关于$\{u_1,\cdots ,u_n\}$相关,但是关于$\{u_1,\cdots ,u_{n-1}\}$无关,则$u_n$关于$\{u_1,\cdots ,u_{n-1},x\}$相关。

\no{3} 如果$\{v_i\}$相关于$\{w_j\}$,且$u$相关于$\{v_i\}$,则$u$相关于$\{w_j\}$.

然后可以类比线性代数中基的性质以及证明。类比线性无关,我们定义$\{u_i\}$代数无关如下:对任意的$i$,$u_i$不代数相关于其他$u_j$.

\pro $\{u_i\}$是代数无关的当且仅当,如果多项式$f$使得$f(u_1,\cdots ,u_n)=0$,那么$f=0$.

如果$\{u_i\}$代数无关,那么他们之间不存在代数方程相互联系,所以他们也被称为超越独立。

\para 一个域$k$被称为代数闭域,就是说$k[x]$中的每个多项式都可以分解为线性因子的乘积。等价地,任何多项式都在$k$中有至少一个根。

每一个域扩张都可以分解为先超越扩张,然后再代数扩张。分解不一定唯一,但是超越扩张的基数却是相同的,如果有限,那么就是次数相同。这个数就是所谓的超越次数。 从这里很容易看出$F(x_1,\cdots ,x_n)$作为单纯的超越扩张$n$次,那么他的超越次数为$n$。也可以这样定义,对于域扩张$E/F$, $E$中的极大代数无关集(超越基)的元素个数被称为超越次数。

\lem 设$E/F$和$E'/F'$是域扩张,且$\varphi:E\to E'$是域同态满足$\varphi(F)\subset F'$.现在设$f(x)\in F[x]$,若$\alpha\in E$是$f(x)$的根,则$\alpha'=\varphi(\alpha)$是$\varphi(f(x))$的根。

\proof 设$f(x)=\sum_i a_i x^i$,那么$\varphi(f(x))=\sum_i \varphi(a_i) x^i$,因此
\[
	\varphi(f(\alpha'))=\sum_i \varphi(a_i) \varphi(\alpha)^i=\varphi\left(\sum_i a_i\alpha^i\right)=\varphi\left(f(\alpha)\right)=0.
\]\qed

特别地,如果$\varphi$是$E$的自同态,且在$F$上的限制为恒等映射,那么如果$\alpha$是$f(x)$的一个根,则$\varphi(\alpha)$也是$f(x)$的一个根。

\para 设$E/F$是域扩张,称所有$E$在$F$上的限制为恒等映射的自同态构成的群为这个域扩张的Galois群,群运算为复合,记作$\mathrm{Gal}(E/F)$.

一般来说,如果$E/F$是有限扩张,那么他Galois群元的个数不多于$[E:F]$,如果$|\mathrm{Gal}(E/F)|=[E:F]$,则称扩张$E/F$是Galois扩张。

\theo Galois理论基本定理:设$E/F$是Galois扩张,

\no{1} 设$H$是$\mathrm{Gal}(E/F)$的子群,那么他和$E/F$的一个中间域$L=\{x\in E:h(x)=x,\forall h\in H\}$存在一一对应,他的逆为$L\mapsto \mathrm{Gal}(E/L)$.且
\[
[E:L]=|\mathrm{Gal}(E/L)|,\quad [L:F]=(G:\mathrm{Gal}(E/L)).
\]

\no{2} 上述对应诱导$G$的所有正规子群和$E/F$的Galois子扩张$L/F$之间的一一对应,此时
\[
	\mathrm{Gal}(L/F)\cong \mathrm{Gal}(E/F)/\mathrm{Gal}(E/L).
\]

如果一个域经过任意的代数扩张之后还是其本身,那么我们就称呼这个域是代数闭域。

\theo 对任意的域$k$,在同构意义上唯一存在包含他代数闭域$\bar{k}$。$\bar{k}$也被称为$k$的代数闭包。

\pro 任意多项式$f\in k[x]$都在$\bar{k}$中有根。

\para 一个域$k$被称为代数闭域,就是说$k[x]$中的每个多项式都可以分解为线性因子的乘积。等价地,任何多项式都在$k$中有至少一个根。

\para 假设$P$是$F$的一个扩张,一个$P$中的元素$v$被称为在$u_1,\cdots ,u_n$上关于域$F(u_1,\cdots ,u_n)$代数相关的,就是说存在一个非零多项式$f$,使得$f(v)=0$,这个多项式的系数是$F[u_1,\cdots ,u_n]$中的多项式。

% \section{Algebras and the proof of Hilbert's Nullstellensatz}

% \para 设$k$是代数闭域,如果$I$是$k[x_1,\cdots,x_n]$的一个理想,记$Z(I)$是这个理想的共同零点集,即$Z(I)$是使得理想$I$内所有多项式都为$0$的点的集合。反过来,对于一个集合$U\in k^n$,我们记$I(U)$为所有在$U$上为零的多项式所构成的理想。

% 可以从Zariski's lemma推出Hilbert's Nullstellensatz.这就是Atiyah\&Maconald第七章的习题14:

% \theo Hilbert's Nullstellensatz: 设$k$是代数闭域,假如我们有一个多项式$f\in k[x_1,\cdots,x_n]$在$Z(I)$为零,那么存在一个正整数$n$使得$f^n\in I$,这就是说$f\in r(I)$,其中$r(I)$是$I$的半径,常常也记做$\sqrt{I}$.

% \proof 让$A=k[x_1,\cdots ,x_n]$,$I$是他的一个理想,假设$f$在$Z(I)$上为0,即$f$属于$I(Z(I))$,但$f$不属于$r(I)$。因为$r(I)$是所有包含a的素理想之交,所以$f$必然不属于某个包含$T$的素理想$p$,让$f'$是$f$在$B=A/p$中的象,再设$C=B[1/f']$,$C$是一个有限生成的$k$代数,由${1'/f',x_1'/1'\cdots ,x_n'/1'}$生成。取$m$是$C$中的一个极大理想,$C/m$是一个域,也是一个有限生成$k$代数,由Zariski's lemma,所以是一个$k$的有限扩张,但是$k$是代数闭域,所以也就是$k$。

% 让$t_i$是$x_i$在映射
% \[
% 	\psi:A\xrightarrow{\pi_1}B\xrightarrow{\phi}C\xrightarrow{\pi_2}C/m\xrightarrow[\cong]{\pi_3} k
% \]
% 的象$t_i=\psi(x_i)$,我们记$t=(t_1,\cdots ,t_n)$,由于$\psi(x_i)=t_i=x_i(t)$对任意的$x_i$都成立,所以对任意的$g$属于$A$,我们有$\psi(g)=g(t)$.
% 现在假设$g$是$I$的元素,那么$\pi_1(g)=0$,故$g(t)=\psi(g)=0$,这就是说$t\in Z(I)$,此外,$\phi(\pi_1(f))=f'/1'$是$C$里面的一个单位,因此$\phi(\pi_1(f))=f'/1'$不在$m$里面(否则$m=C$),那么$\psi(f)$不等于$0$,所以$f(t)=\psi(f)$不等于零,矛盾,证毕。\qed

\section{域的自同构}

\para 域$K$的自同构就是作为环的自同构$\tau:K\to K$,它们全部构成一个群,记作$\mathrm{Aut}(K)$. 设$K/k$是一个扩张,如果$K$的自同构还是$k$-代数自同构,则称他为$k$-自同构,它们全部依然构成一个群,记作$\mathrm{Gal}(K/k)$,称为扩张$K/k$的Galois群,他是$\mathrm{Aut}(K)$的一个子群。设$\tau$是一个$k$-自同构,则任取$a\in k$都有$\tau(a)=a$.

设$\alpha\in K$是$k$上的一个代数元,而$f\in k[x]$是他的极小多项式。任取$\tau$是一个$k$-自同构,则$\tau(\alpha)$依然是$f$的一个根,即
\[
	f(\tau(\alpha))=\sum_i a_i (\tau(\alpha))^i=\sum a_i \tau(\alpha^i)=\tau(f(\alpha))=0.
\]
这就是为什么我们要讨论$k$-自同构的原因,他将一个极小多项式的根变成了它的另一个根。

\para 反过来,设$S$是$\mathrm{Aut}(K)$的一个子集,考虑集合
\[
	\mathcal{F}(S)=\{a\in K\,:\, \forall \tau\in S\,\text{s.t. }\tau(a)=a\}.
\]
不难看出这是一个域,只要考虑有逆元就行了。由$\tau(1)\tau(1)=\tau(1)$,以及域的消去律可以推出$\tau(1)=1$,所以$1\in \mathcal{F}(S)$. 
任取$a\in \mathcal{F}(S)$,由于$1=\tau(1)=\tau(a^{-1})\tau(a)=\tau(a^{-1})a$,所以$\tau(a^{-1})=a^{-1}$,因此$a^{-1}\in \mathcal{F}(S)$.

$\mathcal{F}$不一定是一个单射,比如$S=\{\tau\}$是单点集,但是$\tau^2\neq \tau$,此时我可以知道$\mathcal{F}(\{\tau,\tau^2\})=\mathcal{F}(\{\tau\})$.

\para 在$K$的子域间使用包含关系引入一个偏序,即$k_1\leq k_2$当且仅当$k_1\subset k_2$. 很容易看到如果$k_1\leq k_2$,则$\mathrm{Gal}(K/k_2)\subset \mathrm{Gal}(K/k_1)$. 

同样可以在$\mathrm{Aut}(K)$使用包含关系引入一个偏序,即$S\leq T$当且仅当$S\subset T$. 则$\mathcal{F}$是一个$\mathrm{Aut}(K)$到$K$子域的集合的映射,并且,如果$S\leq T$,则$\mathcal{F}(T)\leq \mathcal{F}(S)$.

以及在前面二者之间,我们有显然的包含关系
\[
	k\subset \mathcal{F}(\mathrm{Gal}(K/k))\text{ and } S\subset \mathrm{Gal}(K/\mathcal{F}(S)).
\]

将上述关系抽象出来,设$S$和$T$是两个偏序集,并且存在两个映射$f:S\to T$以及$g:T\to S$满足
\begin{itemize}
\item 如果$s_1\leq s_2$,则$f(s_2)\leq f(s_1)$.
\item 如果$t_1\leq t_2$,则$g(t_2)\leq g(t_1)$.
\item $t\leq f(g(t))$以及$s\leq g(f(s))$.
\end{itemize}
则称$S$和$T$之间存在一个Galois联络。

\para 设$S$和$T$之间存在一个Galois联络,则$g(T)$和$f(S)$之间存在一一映射,具体写出来即$s\mapsto f(s)$,以及$t \mapsto g(t)$.

\proof 
	设$s=g(t)$,则$t\leq f(g(t))=f(s)$,所以$g(f(s))\leq g(t)=s$. 反过来,由于$s\leq g(f(s))$,所以$s=g(f(s))$. 所以$g|_{f(S)}:f(S)\to g(T)$是一个满射。同样(或者由对称性),对于$t=f(s)$,我们也有$t=f(g(t))$,所以$g|_{f(S)}:f(S)\to g(T)$是一个单射。故而$g|_{f(S)}$是一个双射,它的逆就是$f|_{g(T)}$.
\qed

应用到域扩张的情况,即所有形如$\mathcal{F}(S)$的域和所有形如$\mathrm{Gal}(K/k)$的子群之间,有一个一一映射。

\para 设$K/k$是一个扩张,而$k(X)$是$X\subset K$生成的一个扩张,任取$\sigma$和$\tau\in \mathrm{Gal}(k(X)/k)$,如果$\sigma|_X=\tau|_X$,则$\sigma=\tau$. 这就说明了,对于生成的一个扩张,Galois群的作用只依赖于他在生产元上的集合的作用。

\proof 
	由于\[
		k(X)=\bigcup_{\{a_1,\,\,\cdots,a_n\}\in I} k(a_1,\,\,\cdots,a_n).
	\]
	或者说$k(X)$中的任意一个元素$\alpha$都可以找到有限个$X$中的元素$\{a_1,\,\,\cdots,a_n\}$使得$\alpha\in k(a_1,\,\,\cdots,a_n)$,即
	\[
		\alpha=\frac{f(a_1,\,\,\cdots,a_n)}{g(a_1,\,\,\cdots,a_n)},
	\]
	所以
	\[
		\tau(\alpha)=\frac{f(\tau(a_1),\,\,\cdots,\tau(a_n))}{g(\tau(a_1),\,\,\cdots,\tau(a_n))}=\frac{f(\sigma(a_1),\,\,\cdots,\sigma(a_n))}{g(\sigma(a_1),\,\,\cdots,\sigma(a_n))}=\sigma(\alpha).
	\]
\qed

考虑有限扩张的情况,由于有限扩张就是有限次单代数扩张$k(a_1,\,\,\cdots,a_n)$. 对每一个$a_i$,都有极小多项式$f_i$. 由于$\mathrm{Gal}(k(a_1,\,\,\cdots,a_n)/k)$里面的元素的作用就是将$a_i$变成$f_i$在$k(a_1,\,\,\cdots,a_n)$中的其他的根,所以如果每一个$f_i$的根的数目是有限的,则$\mathrm{Gal}(k(a_1,\,\,\cdots,a_n)/k)$限制在$\{a_1,\,\,\cdots,a_n\}$上的结果是有限的,进而上面证明的内容保证了$\mathrm{Gal}(k(a_1,\,\,\cdots,a_n)/k)$是一个有限群。

\pro 设$G$是$\mathfrak{Aut}(K)$的一个有限子群,则$|G|=[K:\mathcal{F}(G)]$以及$G=\mathrm{Gal}(K/\mathcal{F}(G))$.