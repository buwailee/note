\chapter{域与Galois理论}
\ThisULCornerWallPaper{1}{../Pictures/17.png}
\section{超越扩张}
在环的整扩张那里已经解决了域的有限扩张的分类问题,即有限扩张就是有限生成扩张。如果一个扩张不是有限扩张,则,要么这个扩张包含超越元,或者,他是代数扩张却不能由有限次单代数扩张而成。这节对前者进行讨论。

\begin{para}
设$K/k$是一个扩张,而$X$是一族不定元,如果$K$与$k[X]$的分式域$k(X)$同构,则称扩张$K/k$是纯超越扩张。

依然设$K/k$是一个扩张,再设$T$是$K$的一个子集,记$X_T$是所有形如$x_t$的不定元的集合,其中$t\in T$. 如果$\varphi:x_t\mapsto t$定义的环同态$\varphi:k[X_T]\to K$是单的,则称$T$在$k$上代数无关,否则代数相关。更进一步,如果$\varphi$还是一个代数扩张,则称$T$是扩张$K/k$的一组超越基。
\end{para}

如果$T=\{t_1$, $\dots$, $t_n\}$是一个有限集,则$T$代数无关当且仅当任取非零多项式$f\in k[x_1$, $\dots$, $x_n]$都有$f(t_1$, $\dots$, $t_n)\in K$非零。

注意到,如果令$\psi(f/g)=\varphi(f)/\varphi(g)$,则单同态$\varphi$将诱导域的同态$\psi:k(X_T)\to K$,以及同构$\psi:k(X_T)\to k(T)$,其中$k(T)$是$K$中包含$k$与$T$最小的子域。特别地,如果$T=\{t\}$是一个单点集,此时$T$在$k$上代数无关等价于$t$在$k$上超越。

显然,代数无关集的任意非空子集也必然是代数无关的。因为环的单同态限制在子环上依然是一个单同态。所以,代数无关集中的元素都在$k$上超越。

\begin{lem}
设$K/k$是一个扩张,$T$在$k$上代数无关,当且仅当,$T$的任意有限子集都在$k$上代数无关。
\end{lem}

当然,如果$T$是有限的,则这个引理就是废话。

\begin{proof}
因为代数无关集的非空子集也代数无关,所以我们只需证明若$T$的任意有限子集代数无关,则$T$代数无关即可。

显然$\varphi:x_t\mapsto t$可以定义一个环同态,剩下的只需证明这是单的。考虑$\varphi(f)=0$,由多项式环的构造,可以知道存在一个$T$的有限子集$S$使得$f\in k[S]\subset k[T]$,于是$\varphi|_{k[S]}(f)=0$. 但是因为$S$是$T$的有限子集,所以$S$是代数无关的,即$\varphi|_{k[S]}$是单的,故$f=0$.
\end{proof}

\begin{para}
设$K/k$是一个扩张,$T$是$K$的一个有限子集,如果$t$关于$k(T)$代数无关(相关),则称$t$关于$T$代数无关(相关)。
\end{para}

由于$k(T)$是$k[T]$的商域,所以上述定义不过是在说,如果对每一个非零多项式$f\in k[T][x]$,都有$f(t)\neq 0$,则$t$与$T$代数无关,否则相关。

他有如下性质:

\no{1} 设$t_i\in T$,因为存在$f(x)=t_i-x$使得$f(t_i)=0$,所以$t_i$关于$T$是代数相关的。

\no{2} 如果$x$关于$\{u_1$, $\dots$, $u_n\}$相关,但是关于$\{u_1$, $\dots$, $u_{n-1}\}$无关,则$u_n$关于$\{u_1$, $\dots$, $u_{n-1}$, $x\}$相关。这个性质被称为交换性。

\no{3} 如果$\{v_i\}$相关于$\{w_j\}$,且$u$相关于$\{v_i\}$,则$u$相关于$\{w_j\}$.

然后可以类比线性代数中基的性质以及证明。类比线性无关,我们定义$\{u_i\}$代数无关如下:对任意的$i$,$u_i$不代数相关于其他$u_j$.

\pro $\{u_i\}$是代数无关的当且仅当,如果多项式$f$使得$f(u_1,\dots ,u_n)=0$,那么$f=0$.

如果$\{u_i\}$代数无关,那么他们之间不存在代数方程相互联系,所以他们也被称为超越独立。

\para 一个域$k$被称为代数闭域,就是说$k[x]$中的每个多项式都可以分解为线性因子的乘积。等价地,任何多项式都在$k$中有至少一个根。

每一个域扩张都可以分解为先超越扩张,然后再代数扩张。分解不一定唯一,但是超越扩张的基数却是相同的,如果有限,那么就是次数相同。这个数就是所谓的超越次数。 从这里很容易看出$F(x_1,\dots ,x_n)$作为单纯的超越扩张$n$次,那么他的超越次数为$n$。也可以这样定义,对于域扩张$E/F$, $E$中的极大代数无关集(超越基)的元素个数被称为超越次数。

\lem 设$E/F$和$E'/F'$是域扩张,且$\varphi:E\to E'$是域同态满足$\varphi(F)\subset F'$.现在设$f(x)\in F[x]$,若$\alpha\in E$是$f(x)$的根,则$\alpha'=\varphi(\alpha)$是$\varphi(f(x))$的根。

\proof 设$f(x)=\sum_i a_i x^i$,那么$\varphi(f(x))=\sum_i \varphi(a_i) x^i$,因此
\[
	\varphi(f(\alpha'))=\sum_i \varphi(a_i) \varphi(\alpha)^i=\varphi\left(\sum_i a_i\alpha^i\right)=\varphi\left(f(\alpha)\right)=0.
\]\qed

特别地,如果$\varphi$是$E$的自同态,且在$F$上的限制为恒等映射,那么如果$\alpha$是$f(x)$的一个根,则$\varphi(\alpha)$也是$f(x)$的一个根。

\para 设$E/F$是域扩张,称所有$E$在$F$上的限制为恒等映射的自同态构成的群为这个域扩张的Galois群,群运算为复合,记作$\mathrm{Gal}(E/F)$.

一般来说,如果$E/F$是有限扩张,那么他Galois群元的个数不多于$[E:F]$,如果$|\mathrm{Gal}(E/F)|=[E:F]$,则称扩张$E/F$是Galois扩张。

\theo Galois理论基本定理:设$E/F$是Galois扩张,

\no{1} 设$H$是$\mathrm{Gal}(E/F)$的子群,那么他和$E/F$的一个中间域$L=\{x\in E:h(x)=x,\forall h\in H\}$存在一一对应,他的逆为$L\mapsto \mathrm{Gal}(E/L)$.且
\[
[E:L]=|\mathrm{Gal}(E/L)|,\quad [L:F]=(G:\mathrm{Gal}(E/L)).
\]

\no{2} 上述对应诱导$G$的所有正规子群和$E/F$的Galois子扩张$L/F$之间的一一对应,此时
\[
	\mathrm{Gal}(L/F)\cong \mathrm{Gal}(E/F)/\mathrm{Gal}(E/L).
\]

\para 假设$P$是$F$的一个扩张,一个$P$中的元素$v$被称为在$u_1,\dots ,u_n$上关于域$F(u_1,\dots ,u_n)$代数相关的,就是说存在一个非零多项式$f$,使得$f(v)=0$,这个多项式的系数是$F[u_1,\dots ,u_n]$中的多项式。