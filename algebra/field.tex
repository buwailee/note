% !TEX root = main.tex
\chapter{域与Galois理论}
\ThisULCornerWallPaper{1}{../Pictures/17.png}

\section{域扩张}

我们已经在上一章定义过域扩张、有限扩张、超越扩张、代数扩张等概念。这节,我们将进一步定义一些其他的域扩张的概念。

\begin{para}[分裂域]
    设$K/k$是一个域扩张,$f\in k[x]$是一个多项式,如果$f$在$K[x]$可以分解为线性因子的乘积$f=a\prod_{i=1}^n (x-\alpha_i)$,则称$f$在域$K$上分裂。若$S$是一族非常值的$k[x]$中的多项式,若任取$f$,它们都在$K$上分裂,且此时$K=k(X)$,其中$X$是$S$中元素在$K$中所有的根构成的集合,则称$K$是$S$的一个分裂域。换而言之,$S$的分裂域是最小使得$S$中多项式都分裂的域扩张。
\end{para}

由代数闭域的存在性,分裂域的存在是清楚的。反过来,$k[x]$中所有非常值多项式族的分裂域就是$k$的代数闭域。由于域嵌入方式的不同,所以显然可以存在很多分裂域,下面一个命题告诉我们如何去联系这些分裂域。

\begin{pro}
    若$\sigma:k\to k'$是一个域同构,$S$是一族非常值多项式,$K$是$S$的一个分裂域,而$K'$是$\sigma(S)$的一个分裂域,则存在一个同构$\tau:K\to K'$,使得如下交换图成立
    \[
        \xymatrix{
        K \ar@{->}[rr]^{\tau} &  & K' \\
        k \ar@{^{(}->}[u]^i \ar@{->}[rr]^{\sigma} &  & k' \ar@{^{(}->}[u]^{i'}
        }
    \]
\end{pro}

若$S=\{f\}$是一个单元素集,$f$是一个不可约多项式,则$\tau$是可以直接如下构造的:若$\alpha$是$f$的一个根,而$\alpha'$是$\sigma(f)$的一个根,注意到,我们有$k$-代数同构$k[x]/(f)\cong k(\alpha)$,当然,也有$k'$-代数同构$k'[x]/(\sigma(f))\cong k(\alpha')$. 同时,还有$\sigma$诱导的同构$k[x]/(f)\cong k'[x]/(\sigma(f)$,它们(以及其逆)三个适当复合就得到了我们想要的同构$\tau$,且还可以做到$\tau(\alpha)=\alpha'$.

\begin{proof}
    显然,若同态$\tau:K\to K'$存在且使得交换图成立,则像$\tau(K)\subset K'$必然是$\sigma(S)$的分裂域。但由于$K'$已经是$\sigma(S)$的分裂域了,此时$\tau(K)= K'$.所以我们只要说明$\tau$的存在性即可。
    
    考虑偶对$(L,\varphi)$的集合$A$,其中$L$是$K$的子域,而$\varphi:L\to K'$是一个同态使得交换图成立,即$\varphi i=i'\sigma$. 显然集合非空,因为$(k,i'\sigma)$在其中。定义,$(L,\varphi)\leq (L',\varphi')$当且仅当$L\subset L'$且$\varphi'|_L=\varphi$,这使得$A$是一个偏序集。对$A$的任意升链$\{(L_i,\varphi_i)\}$,$(L=\bigcup_i L_i,\varphi)$显然是一个上界,其中$\varphi(a_i)=\varphi_i(a_i)$若$a_i\in L_i$. 于是,由Zorn引理,$A$中存在一个极大元$(M,\tau)$.

    下面我们证明,$M=K$,于是$\tau:K\to K'$就是所求之同态。若$M\neq K$,则存在一个$f\in S$使得$f$在$M$上不分裂,此时,我们可以进一步对$M$做代数扩张,得到一个在$A$中比$M$大的元素,这与$M$极大性矛盾。
\end{proof}

作为推论,如果$k=k'$,则上面的命题告诉我们,给定非常值多项式族$S\subset k[x]$,任意两个$S$的分裂域作为$k$-代数都是同构的。

\begin{para}[正规扩张]
    设$K/k$是一个域扩张,若$K$是$k[x]$中某族多项式的分裂域,则称呼$K/k$是一个正规扩张。 
\end{para}

\begin{para}[可分扩张和完备域]
    若$k$是一个域,称不可约多项式$f\in k[x]$是可分的,若$f$没有重根。对任意多项式$g$,其可以分解为一族不可约多项式的乘积,若其中每个不可约多项式都是可分的,则称$g$是可分的。若$K/k$是一个扩张,元$\alpha\in K$被称为在$k$上可分当且仅当其极小多项式在$k$上可分,若$K$中元素都在$k$上可分,则称$K/k$是一个可分扩张。
    如果一个域的任何代数扩张都是可分的,则称这个域是完备域。
\end{para}

回忆,$f$有重根当且仅当$f$和其导数$f'$有公因子。

\begin{para}[交换环的特征]
    每个非零交换环$R$都有一个典范同态$\zz\to R$,其核必然是$\zz$的一个理想,由$0$或某个正素数$p$生成,此时,我们定义交换环的特征$\operatorname{char}(R)$为这个数。
\end{para}

若特征为$p$,则在交换环$R$中,$p=0$.

\begin{lem}
    若域$k$的特征为$0$,则其中任意不可约多项式$f$都是可分的,于是任何特征为零的域都是完备域。若域$k$的特征为$p>0$,则其中不可约多项式$f$是可分的当且仅当$f'\neq 0$.
\end{lem}

\begin{proof}
    因为$f$不可约,$f$和$f'$此时的公因子要么为$1$,要么为$f$. 因为域$k$的特征为$0$,所以$\deg f'=\deg f-1$,故$f'\neq 0$,对于非零特征的域,我们假设了这点。所以$f$和$f'$公因子只有可逆元,这意味着$f$没有重根。
\end{proof}

若域$k$中存在一个$a$使得其不能写作$b^p$的形式,则此时$x^p-a$是不可约的,但它是不可分的。

\begin{theo}[本原元定理]
如果$K/k$是一个有限可分扩张,则这是一个单扩张。扩张的那个元素叫做本原元。
\end{theo}

\section{域的自同构与Galois群}

\para 域$K$的自同构就是作为环的自同构$\tau:K\to K$,它们全部构成一个群,记作$\mathrm{Aut}(K)$. 设$K/k$是一个扩张,如果$K$的自同构还是$k$-代数自同构,则称他为$k$-自同构,它们全部依然构成一个群,记作$\Gal(K/k)$,称为扩张$K/k$的Galois群,他是$\mathrm{Aut}(K)$的一个子群。设$\tau$是一个$k$-自同构,则任取$a\in k$都有$\tau(a)=a$.

设$\alpha\in K$是$k$上的一个代数元,而$f\in k[x]$是他的极小多项式。任取$\tau$是一个$k$-自同构,则$\tau(\alpha)$依然是$f$的一个根,即
\[
    f(\tau(\alpha))=\sum_i a_i (\tau(\alpha))^i=\sum a_i \tau(\alpha^i)=\tau(f(\alpha))=0.
\]
这就是为什么我们要讨论$k$-自同构的原因,他将一个极小多项式的根变成了它的另一个根。

\para 反过来,设$S$是$\mathrm{Aut}(K)$的一个子集,考虑集合
\[
    \mathcal{F}(S)=\{a\in K\,:\, \forall \tau\in S\,\text{ s.t. }\tau(a)=a\}.
\]
不难看出这是一个域,只要考虑有逆元就行了。由$\tau(1)\tau(1)=\tau(1)$,以及域的消去律可以推出$\tau(1)=1$,所以$1\in \mathcal{F}(S)$.
任取$a\in \mathcal{F}(S)$,由于$1=\tau(1)=\tau(a^{-1})\tau(a)=\tau(a^{-1})a$,所以$\tau(a^{-1})=a^{-1}$,因此$a^{-1}\in \mathcal{F}(S)$.
\endpara

$\mathcal{F}$不一定是一个单射,比如$S=\{\tau\}$是单点集,但是$\tau^2\neq \tau$,此时我们可以知道$\mathcal{F}(\{\tau,\tau^2\})=\mathcal{F}(\{\tau\})$.

\para 在$K$的子域间使用包含关系引入一个偏序,即$k_1\leq k_2$当且仅当$k_1\subset k_2$. 容易看到,若$k_1\leq k_2$,则$\Gal(K/k_2)\subset \Gal(K/k_1)$.

同样可以在$\mathrm{Aut}(K)$使用包含关系引入一个偏序,即$S\leq T$当且仅当$S\subset T$. 则$\mathcal{F}$是一个$\mathrm{Aut}(K)$到$K$子域的集合的映射,并且,如果$S\leq T$,则$\mathcal{F}(T)\leq \mathcal{F}(S)$.

在前面二者之间,我们有显然的包含关系
\[
    k\subset \mathcal{F}(\Gal(K/k))\quad \text{ and }\quad S\subset \Gal(K/\mathcal{F}(S)).
\]
于是这构成了一个Galois联络(见附录),因此所有形如$\mathcal{F}(S)$的域和所有形如$\Gal(K/k)$的子群之间,有一个一一映射。

\begin{lem}
    设$K/k$是一个扩张,而$k(X)$是$X\subset K$生成的一个扩张,任取$\sigma$和$\tau\in \Gal(k(X)/k)$,如果$\sigma|_X=\tau|_X$,则$\sigma=\tau$.
\end{lem}

这就说明了,对于生成的一个扩张,Galois群的作用只依赖于他在生产元上的集合的作用。

\begin{proof}
    由于\[
        k(X)=\bigcup_{\{a_1,\dots,a_n\}\in I} k(a_1,\dots,a_n).
    \]
    或者说$k(X)$中的任意一个元素$\alpha$都可以找到有限个$X$中的元素$\{a_1$, $\dots$, $a_n\}$使得$\alpha\in k(a_1$, $\dots$, $a_n)$,即
    \[
        \alpha=\frac{f(a_1,\dots,a_n)}{g(a_1,\dots,a_n)},
    \]
    所以
    \[
        \tau(\alpha)=\frac{f(\tau(a_1),\dots,\tau(a_n))}{g(\tau(a_1),\dots,\tau(a_n))}=\frac{f(\sigma(a_1),\dots,\sigma(a_n))}{g(\sigma(a_1),\dots,\sigma(a_n))}=\sigma(\alpha).
    \]
\end{proof}

考虑有限扩张的情况,由于有限扩张就是有限次单代数扩张$k(a_1$, $\dots$, $a_n)$. 对每一个$a_i$,都有极小多项式$f_i$. 由于$\Gal(k(a_1$, $\dots$, $a_n)/k)$里面的元素的作用就是将$a_i$变成$f_i$在$k(a_1$, $\dots$, $a_n)$中的其他的根,所以如果每一个$f_i$的根的数目是有限的,则$\Gal(k(a_1$, $\dots$, $a_n)/k)$限制在$\{a_1$, $\dots$, $a_n\}$上的结果是有限的,进而上面证明的内容保证了$\Gal(k(a_1$, $\dots$, $a_n)/k)$是一个有限群。

\begin{pro}
    设$G$是$\mathrm{Aut}(K)$的一个有限子群,则$|G|=[K:\mathcal{F}(G)]$以及$G=\Gal(K/\mathcal{F}(G))$.
\end{pro}

\para[Galois扩张] 一个代数扩张$K/k$是Galois扩张,如果$\calf(\Gal(K/k))=k$.
\endpara

\begin{pro}
    代数扩张$K/k$是Galois扩张,当且仅当其是可分、正规扩张。
\end{pro}

\begin{pro}
    一个有限扩张$K/k$是Galois扩张,当且仅当$|\Gal(K/k)|=[K:k]$.
\end{pro}

\begin{pro}
    设$k(\alpha)$是$k$的一个单代数扩张,$f$是$\alpha$的极小多项式,则$k(\alpha)$是Galois扩张当且仅当$f$在$k(\alpha)$中有$\deg(f)$个不同的根。
\end{pro}

\section{仿射簇}
\label{variety}

代数簇粗略来说是一族多项式的零点集,他是古典代数几何的主要研究对象。有了它就可以谈论一些代数构造的几何直观。

\para 假设有一个域$k$,他的代数闭包为$\bar{k}$,定义$n$元有序组$\bba^n_k=\bar{k}\times \cdots \times \bar{k}$为($\bar{k}$上的)$n$维仿射空间,其中的元素被称为点,对一个点$p=(a_1,\dots ,a_n)$中的$a_i$被称为$p$的坐标。如果不是必须写出域$k$,则$n$维仿射空间通常直接写作$\bba^n$.

考虑域$k$上的多项式环$A=k[x_1,\dots ,x_n]$,其中的元素都是函数$f:\bba^n\to k$,于是可以定义他的零点集:对于$A$中的一众元素,即一个集合$T$,定义$Z(T)$为$T$的共同零点集,即那些使集合$T$里面元素都为$0$的点。注意到集合$T$可以生成一个多项式环的理想,这个理想的共同零点集和$Z(T)$是一样的。如果$T$是单点集,即只包含一个多项式$f$,此时我们会用$Z(f)$来简记$Z(\{f\})$.

由于$k$是一个域,所以多项式环$A$是Noether环,他的每个理想都是有限生成的,所以任意的$Z(T)$都可以表示为有限个多项式的共同零点。

\para $\bba^n$中的一个子集$Y$,当他是某一个$A$的子集$T$的零点集的时候,即$Z(T)=Y$,此时$Y$被称为一个(仿射)代数集。

考虑一个$\bba^n$中的子集$Y$,我们可以反过来去找$A$的元素$f$,使得$f$在$Y$中为零,于是定义:$I(Y)$是那些使得$f$在$Y$上为零的点的函数的集合。容易看出这是一个理想。

\para 令$Y$是一个代数集,定义仿射坐标环为$A(Y)=A/I(Y)$. 从定义来看,$A(Y)$是一个有限生成的$k$-代数。

\begin{pro}下面命题成立:
    \begin{compactenum}[~~~(1)]
        \item 如果$T_1\subset T_2 \subset A$,则$Z(T_2)\subset Z(T_1)$.
        \item 如果$Y_1\subset Y_2 \subset \bba^n$,则$I(Y_2)\subset I(Y_1)$.
        \item 如果$\{Y_\alpha\,:\alpha\in I\}$是一族$\bba^n$的子集,则$I\left(\bigcup_{\alpha\in I} Y_\alpha\right)=\bigcap_{\alpha\in I} I(Y_\alpha)$.
        \item 如果$\{T_\alpha\,:\alpha\in I\}$是一族$A$的子集,则$Z\left(\bigcup_{\alpha\in I} T_\alpha\right)=\bigcap_{\alpha\in I} Z(T_\alpha)$.
        \item $Z(T_1)\cup Z(T_2)=Z(T_1T_2)$.
        \item $Y\subset Z(I(Y))$以及$T\subset I(Z(T))$.
    \end{compactenum}
\end{pro}

\begin{proof}
    第一第二第六点从定义显然。第三点,由于$Y_\alpha\subset \bigcup_{\alpha\in I} Y_\alpha$,所以由(2)可知$I\left(\bigcup_{\alpha\in I} Y_\alpha\right)\subset\bigcap_{\alpha\in I} I(Y_\alpha)$. 反过来,任取$f\in \bigcap_{\alpha\in I} I(Y_\alpha)$,由于$f\in I(Y_\alpha)$,所以他在每个$Y_\alpha$上为零,故而在$\bigcup_{\alpha\in I} Y_\alpha$上为零。第四点的证明也类似。第五点,对于多项式$f$和$g$有显然的$Z(fg)=Z(f)\cup Z(g)$,然后从第四点得证。
\end{proof}

上面命题的第四第五点验证了代数集满足闭集公理,所以这也就赋予了$\bba^n_k$一个拓扑,称为Zariski拓扑。在这个拓扑下,闭集就是代数集。这个拓扑不一定是Hausdorff的,比如考虑$\bba^1_k$的情况,他的单点集不一定是闭集。但在$k=\bar{k}$的时候,考虑$P=(a_1,\dots ,a_n)$,那么存在理想$(x_1-a_1,\dots,x_n-a_n)$在$P$上为零,所以此时单点集是闭集。

结合上面命题的第一第二第六点,$A$的理想集和$\bba^n$的子集之间建立了一个Galois联络。而研究这个Galois联络下具体的对应,就是Hilbert's Nullstellensatz的内容,我们下面会慢慢阐述。首先指出一半的内容。

\begin{pro}
    如果$Y \subset \bba^n$,则$Z(I(Y))=\bar{Y}$,即$Y$的闭包。
\end{pro}

\begin{proof} 首先,由$Y\subset Z(I(Y))$和$Z(I(Y))$是闭的,所以$\bar{Y}\subset Z(I(Y))$. 反方向,让$\bar{Y}=Z(\mathfrak{a})$,所以$I(Z(\mathfrak{a}))\subset I(Y)$,即$\mathfrak{a}\subset I(Z(\mathfrak{a}))\subset I(Y)$,两边取零点集即$Z(I(Y))\subset Z(\mathfrak{a})=\bar{Y}$. \end{proof}

作为上述命题的推论。我们可以断言,如果一个多项式在$Y$上为$0$当且仅当他在$\bar{Y}$上为$0$.

\begin{proof} 即证明$I(\bar{Y})=I(Y)$. 将$Z(I(Y))=\bar{Y}$两边取理想,那么$I(\bar{Y})=I(Z(I(Y)))$,特别地,有包含关系$I(Y)\subset I(Z(I(Y)))=I(\bar{Y})$,但是$Y\subset \bar{Y}$又告诉我们$I(\bar{Y})\subset I(Y)$,所以$I(\bar{Y})=I(Y)$. \end{proof}

\para Galois联络保证了我们对于极大极小之间的对应。对于$A$的极大理想$\mm$,$Z(\mm)$是极小的零点集。如果有一个$Z(\aaa)$比$Z(\mm)$小,则$\mm\subset \aaa$就可以知道$\aaa=\mm$. 同时,由$\mm \subset I(Z(\mm))$也可以知道$I(Z(\mm))=\mm$. 反过来,对于一个极小的闭集$Y=Z(\aaa)$,它的理想集为$I(Z(\aaa))$. 取$\mathfrak{m}$为包含$I(Z(\aaa))$的任意极大理想,由Galois联络,$Z(\mathfrak{m})\subset Z(I(Y))=Y$,由极小性,$Y=Z(\mathfrak{m})$,同时这也保证了对应极大理想的唯一性。综上,每一个极大理想的零点集是极小的闭集,每一个极小的闭集都是某个极大理想的零点集。而由拓扑的知识我们知道,极小的闭集是单点集的闭包。

下面借助Zariski引理,我们来研究$Z(\mm)$和单点集闭包的结构。

\begin{lem}
    对任意$A=k[x_1,\dots,x_n]$中的极大理想$\mm$,$Z(\mm)\neq\varnothing$.
\end{lem}

\begin{proof} 
    由于$\mm$极大,那么$k[x_1,\dots,x_n]/\mm=k[\bar{x}_1,\dots,\bar{x}_n]$就是一个域,由Zariski引理,他是$k$的一个有限扩张。设$\bar{x}_i$是$x_i$在$k[x_1,\dots,x_n]/\mm$里面的像,由于是代数扩张,所以$\bar{x}_i\in \bar{k}$. 此时$\mm$上的多项式至少有公共零点$(\bar{x}_1,\dots,\bar{x}_n)\in \bba^n$,因为对任意$f\in \mm$都成立
    \[
        f(\bar{x}_1,\dots,\bar{x}_n)=\sum a_{i_1\cdots i_n} {\bar{x}}_1^{i_1}\cdots {\bar{x}}_n^{i_n}=\overline{\sum a_{i_1\cdots i_n} x_1^{i_1}\cdots x_n^{i_n}}=\bar{f}=0.
    \]
    所以这就推出了$Z(\mm)\neq\varnothing$.
\end{proof}

从引理,我们已经知道$P=(\bar{x}_1,\dots,\bar{x}_n)\in \bba^n_k$是$\mm$的一个公共零点。如果$(\bar{y}_1,\dots,\bar{y}_n)$也是$\mm$的一个公共零点,通过$\sigma_{\bar{y}}(\bar{x}_i)=\bar{y}_i$且如果$a\in k$则$\sigma_{\bar{y}}(a)=a$可以定义环同态
\[
    \sigma_{\bar{y}}:k[\bar{x}_1,\dots,\bar{x}_n]\to k[\bar{y}_1,\dots,\bar{y}_n].
\]

他是满的。任取$f(\bar{y}_1,\dots,\bar{y}_n)\in k[\bar{y}_1,\dots,\bar{y}_n]$,则$f(\bar{y}_1,\dots,\bar{y}_n)=\sigma_{\bar{y}}\bigl(f(\bar{x}_1,\dots,\bar{x}_n)\bigr)$.
他还是单的,因为$k[\bar{x}_1,\dots,\bar{x}_n]$是一个域,所以$\sigma_{\bar{y}}^{-1}(0)$作为域中的素理想只能是零理想或者全集,这样$\sigma_{\bar{y}}$就是一个同构,因此$k[\bar{y}_1,\dots,\bar{y}_n]$也是域。

对于对每一个$\bar{y}_i$都可以向上找到一个$k[x_1,\dots,x_n]$中的一个多项式$y_i$使得其模掉$\mm$后就是$\bar{y}_i$.
将所有$\bar{y}_i$的具体形式(作为$\{\bar{x}_i\}$的多项式)代入任意的$f(\bar{y}_1,\dots,\bar{y}_n) \in k[\bar{y}_1,\dots,\bar{y}_n]$,这样得到了
\[
    f\bigl(\bar{y}_1(\bar{x}_1,\dots,\bar{x}_n),\dots,\bar{y}_n(\bar{x}_1,\dots,\bar{x}_n)\bigr) \in k[\bar{x}_1,\dots,\bar{x}_n],
\]
所以$k[\bar{y}_1,\dots,\bar{y}_n]$是$k[\bar{x}_1,\dots,\bar{x}_n]$的一个子域,因为他们同构,所以$k[\bar{x}_1,\dots,\bar{x}_n]=k[\bar{y}_1,\dots,\bar{y}_n]$,这样$\sigma_{\bar{y}}$就是$A(\mm)=k[\bar{x}_1,\dots,\bar{x}_n]$的一个自同态且$\sigma_{\bar{y}}(k)=k$,这其实就是说$\sigma_{\bar{y}}\in \Gal(A(\mm)/k)$.

结合上面说的,我们对每一个$\mm$的公共零点$(\bar{y}_1,\dots,\bar{y}_n)$都找到了唯一的自同态$\sigma_{\bar{y}}$。反之,如果有一个自同态$\sigma$,那么$(\sigma(\bar{x}_1),\dots,\sigma(\bar{x}_n))$就是一个$\mm$的公共零点。这是因为对任意的$f\in\mm$,都成立
\[
    f(\sigma(\bar{x}_1),\dots,\sigma(\bar{x}_n))=\sum a_{i_1\cdots i_n} \sigma({\bar{x}}_1)^{i_1}\cdots \sigma({\bar{x}}_n)^{i_n}=\sigma\left(\sum a_{i_1\cdots i_n} \bar{x}_1^{i_1}\cdots \bar{x}_n^{i_n}\right)=0.
\]
这样就在$Z(\mm)$和$\Gal(A(\mm)/k)$建立起了一一对应关系。

\para 如果采用记号$\Gal(A(\mm)/k)\cdot P$来标记Galois群的元素作用上$P=(\bar{x}_1,\dots,\bar{x}_n)$得到的轨道
\[
    \Gal(A(\mm)/k)\cdot P=\bigl\{\sigma\cdot P=(\sigma(\bar{x}_1),\dots,\sigma(\bar{x}_n)):\forall \sigma\in \Gal(A(\mm)/k)\bigr\},
\]
上面的断言即$Z(\mm)=\Gal(A(\mm)/k)\cdot P$. 因为$A(\mm)$是有限扩张,Galois群$\Gal(A(\mm)/k)$的群元个数是有限的,所以$Z(\mm)$是有限集。

最后来谈谈反问题,前面已经看到,对于任意的极大理想,他的零点集是一个有限集。反过来,对于一个任意单点$P\in \bba^n_k$,我们想要寻找极大理想,使得$P$是他的公共零点。存在性是简单的,对于任意点$P\in \bba^n_k$,因为他的闭包就是极小的闭集,对应着一个极大理想$I(\overline{\{P\}})$,所以他的闭包就是一条Galois轨道$\overline{\{P\}}=\Gal(A(\overline{\{P\}})/k)\cdot P$. 下面的命题保证了唯一性。

\lem 代数集$Y$中的任意一点$P$都处于唯一一条Galois轨道$\overline{\{P\}}\subset Y$上。

\begin{proof} 设$P\in Y$,因为$Y$是闭集,所以$\overline{\{P\}}\subset Y$,因此$Y$中轨道的存在性来自于$\bba^n_k$中轨道的存在性,而这点在命题之前已经证明了。下面证明(在$\bba^n_k$上的)轨道的唯一性,即假设$P$在其他轨道上$P\in \overline{\{Q\}}$,我们要证明$\overline{\{Q\}}=\overline{\{P\}}$.

    因为$P\in \overline{\{Q\}}$,取闭包有$\overline{\{P\}}\subset \overline{\{Q\}}$。如果可以得到$Q\in \overline{\{P\}}$,取闭包有$\overline{\{Q\}}\subset \overline{\{P\}}$。结合两个包含第一点就证明完毕了。将$\overline{\{P\}}\subset \overline{\{Q\}}$取理想得到$I(\overline{\{Q\}})\subset I(\overline{\{P\}})$,因为这是两个极大理想,所以$I(\overline{\{Q\}})=I(\overline{\{P\}})$. 那么
    \[
        \Gal\bigl(A\bigl(\overline{\{P\}}\bigr)/k\bigr)=\Gal\bigl(A\bigl(\overline{\{Q\}}\bigr)/k\bigr)=G,
    \]
    因为已有$P\in \overline{\{Q\}}$,所以$P=\sigma \cdot Q$,其中$\sigma\in G$,两边作用上$\sigma^{-1}$就有$Q=\sigma^{-1}\cdot P$,所以$Q\in \overline{\{P\}}$. \end{proof}

\para 现在我们来看一般的$Z(\mathfrak{a})$. 对于一条轨道$\overline{\{P\}}\subset Z(\mathfrak{a})$,在$A$中可以找到唯一的极大理想$I(\overline{\{P\}})$. 反过来,任取一个包含$\mathfrak{a}$的极大理想$\mm$,由于$\mathfrak{a}\subset \mm$,所以$Z(\mm)\subset Z(\mathfrak{a})$. 而$Z(\mm)$正是一条Galois轨道。所以$Z(\aaa)$中的Galois轨道一一对应着包含$\aaa$的极大理想,这个观察给出了下面一个命题。

\begin{pro}
    如果$\mathfrak{a}$是$A$的一个理想,则$I(Z(\mathfrak{a}))=\sqrt{\mathfrak{a}}$.
\end{pro}

这个命题也被称为Hilbert's Nullstellensatz,即Hilbert零点定理。这个定理联系了代数集与(代数闭域上的)多项式环中的理想,可以说确立了几何和代数之间的基本关系,而代数几何正是建立在这一关联的基础之上的。

\begin{proof} 由于$Y=Z(\mathfrak{a})$是一个代数集,遍历其中所有的点的闭包,我们有
    \[
        I(Z(\mathfrak{a}))=I\left(\bigcup_{P\in Y}\overline{\{P\}}\right)=\bigcap_{P\in Y}I\left(\overline{\{P\}}\right),
    \]
    注意到Galois轨道一一对应着包含$\aaa$的极大理想,所以$I(Z(\mathfrak{a}))=\bigcap_{\mm\supset \aaa}\mm$. 因为$k[x_1,\dots,x_n]$是Jacobson环,所以$I(Z(\mathfrak{a}))=\sqrt{\mathfrak{a}}$. \end{proof}

对于任意满足$\mathfrak{a}=\sqrt{\mathfrak{a}}$的理想(称为根式理想),有$I(Z(\mathfrak{a}))=\mathfrak{a}$. 这个命题给出了明确的Galois联络两边的像,即根式理想与代数集一一对应。

\para 称拓扑空间$X$是不可约的,即是说$X$不能写作它的两个非空真闭子集的并。

在不可约空间内,一个开集必然是稠密的,否则他的闭包和他的补集的并构成了全集。可以证明,这样的开集自身也是不可约的。如果$Y$是$X$的不可约子集,那么$Y$的闭包也是$X$的不可约子集。这些都容易构造出两个闭集来证明。

\para 一个仿射(代数)簇(affine variety),是$\bba^n$中的不可约闭子集,或者说是$\bba^n$中的不可约代数集。一个仿射簇的开子集被称为是拟仿射(代数)簇。

下面这个命题可以使我们可以看到不可约代数集和素理想之间的联系:

\begin{pro}
    一个代数集$Y$是不可约的当且仅当$I(Y)$是一个素理想。
\end{pro}

\begin{proof} 令$\pp$是一个素理想,以及假设$Z(\pp)=Y_1\cup Y_2$,那么$\pp=I(Z(\pp))=I(Y_1)\cap I(Y_2)$,因此$\pp=I(Y_1)$或者$\pp=I(Y_2)$,因此$Z(\pp)$不可约。

    反之,假设$Y$不可约,那么如果$fg\in I(Y)$,此时$Y\subset Z(fg)=Z(f)\cup Z(g)$,因此
    \[
        Y=[Y\cap Z(f)]\cup [Y\cap Z(g)],
    \]
    所以$Y=Y\cap Z(f)$或$Y=Y\cap Z(g)$,即$Y\subset Z(f)$或者$Y\subset Z(g)$,即$f\in I(Y)$或者$g\in I(Y)$.因此$I(Y)$是素理想。\end{proof}

对于不可约代数集,$I(Y)=\pp$是一个素理想,所以也是根式理想,他与它的零点集一一对应,即$Y=Z(\pp)$. 所以上面的命题告诉我们,不可约代数集与$A=k[x_1,\dots,x_n]$的素理想一一对应。

前面说了,一个多项式在$Y$上为$0$当且仅当他在$\bar{Y}$上为$0$。特别地,如果$\bar{Y}$是一个仿射簇,那么而$Y$是其中的一个开集,那么由于仿射簇不可约,所以$Y$在$\bar{Y}$中稠密,此时,上面的那个论断现在似乎在暗示我们多项式函数是连续函数,即在Zariski拓扑下$\bba^n\to \bba^1=\bar{k}$的连续函数。

\para 设$Y$是一个代数集。一个函数$f:Y\to \bar{k}$在点$p\in Y$是正则的,就是说存在$p$的邻域$U$使得$p\in U\subset Y$,以及存在两个多项式$g$, $h$,其中$h$在$U$上处处不为$0$,使得$f=g/h$. 称一个函数在$Y$上正则,就是说他在$Y$上的每一点都正则。

多项式函数显然是正则函数。

\begin{pro}
    正则函数在Zariski拓扑下是连续函数。
\end{pro}

\begin{proof}
    按连续的定义,只要证闭集的逆象是闭的就好了。由于$\bba^1$中的闭集对应于$k[x]$中的理想的零点集,而$k[x]$中的理想是主理想,一个多项式的零点集总是有限的,所以$\bba^1$中的闭集都是有限集。如果我们证明了单点集的逆象是闭的,那也就证明了所有闭集的逆象是闭的。

    闭集可以局部检查,如果$Z$是拓扑空间$Y$的子集,那么他是闭集当且仅当对于任意一个$Y$的开覆盖$\{U_\alpha\}$,$Z\cap U_\alpha$是$U_\alpha$中的闭集。

    找个开覆盖,在每个$U_\alpha$中,$f$都可以写作$f=g_\alpha/h_\alpha$,此时
    \[
        f^{-1}(a)\cap U_\alpha=\bigl\{p\in U| g_\alpha(p)/h_\alpha(p)=a\bigr\},
    \]
    所以$g_\alpha(p)/h_\alpha(p)=a$又等价于$g_\alpha(p)-ah_\alpha(p)=0$,所以
    \[
        f^{-1}(a)\cap U_\alpha=Z(g_\alpha-ah_\alpha)\cap U_\alpha
    \]
    是一个闭集。
\end{proof}

\section{超越扩张}
在环的整扩张那里已经解决了域的有限扩张的分类问题,即有限扩张就是有限生成扩张。如果一个扩张不是有限扩张,则,要么这个扩张包含超越元,或者,他是代数扩张却不能由有限次单代数扩张而成。这节对前者进行讨论。

\begin{para}
设$K/k$是一个扩张,而$X$是一族不定元,如果$K$与$k[X]$的分式域$k(X)$同构,则称扩张$K/k$是纯超越扩张。

依然设$K/k$是一个扩张,再设$T$是$K$的一个子集,记$X_T$是所有形如$x_t$的不定元的集合,其中$t\in T$. 如果$\varphi:x_t\mapsto t$定义的环同态$\varphi:k[X_T]\to K$是单的,则称$T$在$k$上代数无关,否则代数相关。更进一步,如果$\varphi$还是一个代数扩张,则称$T$是扩张$K/k$的一组超越基。
\end{para}

如果$T=\{t_1$, $\dots$, $t_n\}$是一个有限集,则$T$代数无关当且仅当任取非零多项式$f\in k[x_1$, $\dots$, $x_n]$都有$f(t_1$, $\dots$, $t_n)\in K$非零。

注意到,如果令$\psi(f/g)=\varphi(f)/\varphi(g)$,则单同态$\varphi$将诱导域的同态$\psi:k(X_T)\to K$,以及同构$\psi:k(X_T)\to k(T)$,其中$k(T)$是$K$中包含$k$与$T$最小的子域。特别地,如果$T=\{t\}$是一个单点集,此时$T$在$k$上代数无关等价于$t$在$k$上超越。

显然,代数无关集的任意非空子集也必然是代数无关的。因为环的单同态限制在子环上依然是一个单同态。所以,代数无关集中的元素都在$k$上超越。

\begin{lem}
设$K/k$是一个扩张,$T$在$k$上代数无关,当且仅当,$T$的任意有限子集都在$k$上代数无关。
\end{lem}

当然,如果$T$是有限的,则这个引理就是废话。

\begin{proof}
因为代数无关集的非空子集也代数无关,所以我们只需证明若$T$的任意有限子集代数无关,则$T$代数无关即可。

显然$\varphi:x_t\mapsto t$可以定义一个环同态,剩下的只需证明这是单的。考虑$\varphi(f)=0$,由多项式环的构造,可以知道存在一个$T$的有限子集$S$使得$f\in k[S]\subset k[T]$,于是$\varphi|_{k[S]}(f)=0$. 但是因为$S$是$T$的有限子集,所以$S$是代数无关的,即$\varphi|_{k[S]}$是单的,故$f=0$.
\end{proof}

\begin{para}
设$K/k$是一个扩张,$T$是$K$的一个有限子集,如果$t$关于$k(T)$代数无关(相关),则称$t$关于$T$代数无关(相关)。
\end{para}

由于$k(T)$是$k[T]$的商域,所以上述定义不过是在说,如果对每一个非零多项式$f\in k[T][x]$,都有$f(t)\neq 0$,则$t$与$T$代数无关,否则相关。

他有如下性质:

\no{1} 设$t_i\in T$,因为存在$f(x)=t_i-x$使得$f(t_i)=0$,所以$t_i$关于$T$是代数相关的。

\no{2} 如果$x$关于$\{u_1$, $\dots$, $u_n\}$相关,但是关于$\{u_1$, $\dots$, $u_{n-1}\}$无关,则$u_n$关于$\{u_1$, $\dots$, $u_{n-1}$, $x\}$相关。这个性质被称为交换性。

\no{3} 如果$\{v_i\}$相关于$\{w_j\}$,且$u$相关于$\{v_i\}$,则$u$相关于$\{w_j\}$.

然后可以类比线性代数中基的性质以及证明。类比线性无关,我们定义$\{u_i\}$代数无关如下:对任意的$i$,$u_i$不代数相关于其他$u_j$.

\pro $\{u_i\}$是代数无关的当且仅当,如果多项式$f$使得$f(u_1,\dots ,u_n)=0$,那么$f=0$.

如果$\{u_i\}$代数无关,那么他们之间不存在代数方程相互联系,所以他们也被称为超越独立。

\para 一个域$k$被称为代数闭域,就是说$k[x]$中的每个多项式都可以分解为线性因子的乘积。等价地,任何多项式都在$k$中有至少一个根。

每一个域扩张都可以分解为先超越扩张,然后再代数扩张。分解不一定唯一,但是超越扩张的基数却是相同的,如果有限,那么就是次数相同。这个数就是所谓的超越次数。 从这里很容易看出$F(x_1,\dots ,x_n)$作为单纯的超越扩张$n$次,那么他的超越次数为$n$。也可以这样定义,对于域扩张$E/F$, $E$中的极大代数无关集(超越基)的元素个数被称为超越次数。

\lem 设$E/F$和$E'/F'$是域扩张,且$\varphi:E\to E'$是域同态满足$\varphi(F)\subset F'$.现在设$f(x)\in F[x]$,若$\alpha\in E$是$f(x)$的根,则$\alpha'=\varphi(\alpha)$是$\varphi(f(x))$的根。

\proof 设$f(x)=\sum_i a_i x^i$,那么$\varphi(f(x))=\sum_i \varphi(a_i) x^i$,因此
\[
	\varphi(f(\alpha'))=\sum_i \varphi(a_i) \varphi(\alpha)^i=\varphi\left(\sum_i a_i\alpha^i\right)=\varphi\left(f(\alpha)\right)=0.
\]\qed

特别地,如果$\varphi$是$E$的自同态,且在$F$上的限制为恒等映射,那么如果$\alpha$是$f(x)$的一个根,则$\varphi(\alpha)$也是$f(x)$的一个根。

\para 设$E/F$是域扩张,称所有$E$在$F$上的限制为恒等映射的自同态构成的群为这个域扩张的Galois群,群运算为复合,记作$\mathrm{Gal}(E/F)$.

一般来说,如果$E/F$是有限扩张,那么他Galois群元的个数不多于$[E:F]$,如果$|\mathrm{Gal}(E/F)|=[E:F]$,则称扩张$E/F$是Galois扩张。

\theo Galois理论基本定理:设$E/F$是Galois扩张,

\no{1} 设$H$是$\mathrm{Gal}(E/F)$的子群,那么他和$E/F$的一个中间域$L=\{x\in E:h(x)=x,\forall h\in H\}$存在一一对应,他的逆为$L\mapsto \mathrm{Gal}(E/L)$.且
\[
[E:L]=|\mathrm{Gal}(E/L)|,\quad [L:F]=(G:\mathrm{Gal}(E/L)).
\]

\no{2} 上述对应诱导$G$的所有正规子群和$E/F$的Galois子扩张$L/F$之间的一一对应,此时
\[
	\mathrm{Gal}(L/F)\cong \mathrm{Gal}(E/F)/\mathrm{Gal}(E/L).
\]

\para 假设$P$是$F$的一个扩张,一个$P$中的元素$v$被称为在$u_1,\dots ,u_n$上关于域$F(u_1,\dots ,u_n)$代数相关的,就是说存在一个非零多项式$f$,使得$f(v)=0$,这个多项式的系数是$F[u_1,\dots ,u_n]$中的多项式。