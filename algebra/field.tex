\chapter{域与Galois理论}
\section{trans deg}
上面一节解决了有限扩张的分类问题,即有限扩张就是有限生成扩张。如果一个扩张不是有限扩张,则,要么这个扩张包含超越元,或者,他是代数扩张却不能由有限次单代数扩张而成。这节要更细致地对非有限扩张进行分类。


\para 设$K/k$是一个扩张,一个$K$中的元素$t$被称为在$\{u_1,\cdots ,u_n\}$上关于域$k(u_1,\cdots ,u_n)$代数相关的,就是说存在一个非零多项式$f\in k[u_1,\cdots ,u_n][x]$,使得$f(t)=0$。

他有如下性质:

\no{1} 因为存在$f(x)=u_i-x$,所以$u_i$是代数相关的。

\no{2} 如果$x$关于$\{u_1,\cdots ,u_n\}$相关,但是关于$\{u_1,\cdots ,u_{n-1}\}$无关,则$u_n$关于$\{u_1,\cdots ,u_{n-1},x\}$相关。

\no{3} 如果$\{v_i\}$相关于$\{w_j\}$,且$u$相关于$\{v_i\}$,则$u$相关于$\{w_j\}$.

然后可以类比线性代数中基的性质以及证明。类比线性无关,我们定义$\{u_i\}$代数无关如下:对任意的$i$,$u_i$不代数相关于其他$u_j$.

\pro $\{u_i\}$是代数无关的当且仅当,如果多项式$f$使得$f(u_1,\cdots ,u_n)=0$,那么$f=0$.

如果$\{u_i\}$代数无关,那么他们之间不存在代数方程相互联系,所以他们也被称为超越独立。

\para 一个域$k$被称为代数闭域,就是说$k[x]$中的每个多项式都可以分解为线性因子的乘积。等价地,任何多项式都在$k$中有至少一个根。

每一个域扩张都可以分解为先超越扩张,然后再代数扩张。分解不一定唯一,但是超越扩张的基数却是相同的,如果有限,那么就是次数相同。这个数就是所谓的超越次数。 从这里很容易看出$F(x_1,\cdots ,x_n)$作为单纯的超越扩张$n$次,那么他的超越次数为$n$。也可以这样定义,对于域扩张$E/F$, $E$中的极大代数无关集(超越基)的元素个数被称为超越次数。

\lem 设$E/F$和$E'/F'$是域扩张,且$\varphi:E\to E'$是域同态满足$\varphi(F)\subset F'$.现在设$f(x)\in F[x]$,若$\alpha\in E$是$f(x)$的根,则$\alpha'=\varphi(\alpha)$是$\varphi(f(x))$的根。

\proof 设$f(x)=\sum_i a_i x^i$,那么$\varphi(f(x))=\sum_i \varphi(a_i) x^i$,因此
\[
	\varphi(f(\alpha'))=\sum_i \varphi(a_i) \varphi(\alpha)^i=\varphi\left(\sum_i a_i\alpha^i\right)=\varphi\left(f(\alpha)\right)=0.
\]\qed

特别地,如果$\varphi$是$E$的自同态,且在$F$上的限制为恒等映射,那么如果$\alpha$是$f(x)$的一个根,则$\varphi(\alpha)$也是$f(x)$的一个根。

\para 设$E/F$是域扩张,称所有$E$在$F$上的限制为恒等映射的自同态构成的群为这个域扩张的Galois群,群运算为复合,记作$\mathrm{Gal}(E/F)$.

一般来说,如果$E/F$是有限扩张,那么他Galois群元的个数不多于$[E:F]$,如果$|\mathrm{Gal}(E/F)|=[E:F]$,则称扩张$E/F$是Galois扩张。

\theo Galois理论基本定理:设$E/F$是Galois扩张,

\no{1} 设$H$是$\mathrm{Gal}(E/F)$的子群,那么他和$E/F$的一个中间域$L=\{x\in E:h(x)=x,\forall h\in H\}$存在一一对应,他的逆为$L\mapsto \mathrm{Gal}(E/L)$.且
\[
[E:L]=|\mathrm{Gal}(E/L)|,\quad [L:F]=(G:\mathrm{Gal}(E/L)).
\]

\no{2} 上述对应诱导$G$的所有正规子群和$E/F$的Galois子扩张$L/F$之间的一一对应,此时
\[
	\mathrm{Gal}(L/F)\cong \mathrm{Gal}(E/F)/\mathrm{Gal}(E/L).
\]

如果一个域经过任意的代数扩张之后还是其本身,那么我们就称呼这个域是代数闭域。

\theo 对任意的域$k$,在同构意义上唯一存在包含他代数闭域$\bar{k}$。$\bar{k}$也被称为$k$的代数闭包。

\pro 任意多项式$f\in k[x]$都在$\bar{k}$中有根。

\para 一个域$k$被称为代数闭域,就是说$k[x]$中的每个多项式都可以分解为线性因子的乘积。等价地,任何多项式都在$k$中有至少一个根。

\para 假设$P$是$F$的一个扩张,一个$P$中的元素$v$被称为在$u_1,\cdots ,u_n$上关于域$F(u_1,\cdots ,u_n)$代数相关的,就是说存在一个非零多项式$f$,使得$f(v)=0$,这个多项式的系数是$F[u_1,\cdots ,u_n]$中的多项式。

\section{Algebras and the proof of Hilbert's Nullstellensatz}

\para 设$k$是代数闭域,如果$I$是$k[x_1,\cdots,x_n]$的一个理想,记$Z(I)$是这个理想的共同零点集,即$Z(I)$是使得理想$I$内所有多项式都为$0$的点的集合。反过来,对于一个集合$U\in k^n$,我们记$I(U)$为所有在$U$上为零的多项式所构成的理想。

可以从Zariski's lemma推出Hilbert's Nullstellensatz.这就是Atiyah\&Maconald第七章的习题14:

\theo Hilbert's Nullstellensatz: 设$k$是代数闭域,假如我们有一个多项式$f\in k[x_1,\cdots,x_n]$在$Z(I)$为零,那么存在一个正整数$n$使得$f^n\in I$,这就是说$f\in r(I)$,其中$r(I)$是$I$的半径,常常也记做$\sqrt{I}$.

\proof 让$A=k[x_1,\cdots ,x_n]$,$I$是他的一个理想,假设$f$在$Z(I)$上为0,即$f$属于$I(Z(I))$,但$f$不属于$r(I)$。因为$r(I)$是所有包含a的素理想之交,所以$f$必然不属于某个包含$T$的素理想$p$,让$f'$是$f$在$B=A/p$中的象,再设$C=B[1/f']$,$C$是一个有限生成的$k$代数,由${1'/f',x_1'/1'\cdots ,x_n'/1'}$生成。取$m$是$C$中的一个极大理想,$C/m$是一个域,也是一个有限生成$k$代数,由Zariski's lemma,所以是一个$k$的有限扩张,但是$k$是代数闭域,所以也就是$k$。

让$t_i$是$x_i$在映射
\[
	\psi:A\xrightarrow{\pi_1}B\xrightarrow{\phi}C\xrightarrow{\pi_2}C/m\xrightarrow[\cong]{\pi_3} k
\]
的象$t_i=\psi(x_i)$,我们记$t=(t_1,\cdots ,t_n)$,由于$\psi(x_i)=t_i=x_i(t)$对任意的$x_i$都成立,所以对任意的$g$属于$A$,我们有$\psi(g)=g(t)$.
现在假设$g$是$I$的元素,那么$\pi_1(g)=0$,故$g(t)=\psi(g)=0$,这就是说$t\in Z(I)$,此外,$\phi(\pi_1(f))=f'/1'$是$C$里面的一个单位,因此$\phi(\pi_1(f))=f'/1'$不在$m$里面(否则$m=C$),那么$\psi(f)$不等于$0$,所以$f(t)=\psi(f)$不等于零,矛盾,证毕。\qed
