\renewcommand\chapterimg{../Pictures/7.png}
\chapter{模(一)}
假设下面出现的环都是交换环,所以我们不区分左右模。

\section{链条件}

\para 设$P$是一个偏序集,如果他的任意非空子集都有极大元,则$P$称为满足极大条件。对偶地,还有极小条件。

极大条件的一个有趣应用是Noether归纳原理:设$\mathcal{P}$是关于$P$中元素的一个命题,如果对固定的$a$,任意的$\mathcal{P}(x)$对任意的$x>a$成立可以推出$\mathcal{P}(a)$成立,则$\mathcal{P}$对所有$x\in P$都成立。

证明是简单的,假设$\mathcal{P}$至少对某个元素不成立,则所有使得$\mathcal{P}$不成立的元素的集合非空,由极大条件,这个集合中存在极大元$a$使得$\mathcal{P}(a)$不成立。任取$x>a$,由于$a$的极大性,$\mathcal{P}(x)$都成立,再由条件,可知$\mathcal{P}(a)$成立,矛盾。

\para 设$P$是一个偏序集,他的全序子集被称为一个链。如果给定一条链$S$,存在一个元素$a$,使得所有$S$中所有满足$a\leq x\in S$都成立$x=a$,则该链被称为是(向上)稳定的,简写成a.c.c. 类似的,还有(向下)稳定的,简写成d.c.c.

\pro 极大条件等价于所有链都是a.c.c.

\proof
	如果$P$满足极大条件,则任取一条链$S$,他都有极大元$a$,任意的$a\leq x\in S$都可以推出$x=a$. 反过来,如果$P$不满足极大条件,则存在$S\subset P$中没有极大元。则给定一个$x_0\in S$,可以找到一个$x_1\in S$满足$x_0<x_1$,再对$x_1$可以找到$x_2$使得$x_1<x_2$,如是进行下去,我们就找到了一条不稳定的链。
\qed

可见极大条件和a.c.c.是等价的,在应用中极大条件往往比较方便,而在证明/证伪Noether性的时候,构造一个具体的升链就比较简单。

\para 一个$R$-模$M$,如果他的子模集满足极大条件,则他称为Noether模,如果满足极小条件,则他称为Artin模。环$R$看成$R$-模时,如果他是Noether(Artin)模,则他称为Noether(Artin)环。

Noether模远比Artin模应用广泛,比如我们后面会断言Artin环一定是一个Noether环,反之不然。Noether性是有限性条件之一,下面一个命题展现了这一点。

\pro 如果三个$R$-模$P$, $M$和$Q$有一个短正合列$0\to P\to M\to Q\to 0$,则$M$是Noether模当且仅当$P$和$Q$都是Noether模。\notprove

\pro 一个模是Noether模当且仅当它的任意子模是有限生成的。\notprove

\theo Hilbert基定理:如果$R$是一个Noether环,则$R[x]$也是一个Noether环。\notprove

作为推论,如果$R$是一个Noether环,而$S$是一个有限生成$R$-代数,则$S$也是一个Noether环。

\para 下面的观察来自于Noether,他与拓扑上的不可约性对应。设$M$是一个Noether模,而$N$是它的一个子模,如果$N$不能写成两个$M$的子模$P$和$Q$的交,其中$N$是$P$和$Q$的真子模,则称$N$是一个不可约子模。

\pro \label{irrde}设$M$是一个Noether模,则每一个$M$的真子模都可以写成$M$的有限个不可约子模的交的形式,这被称为不可约分解。

\proof
	假设命题不对,考虑所有不能写成有限个不可约子模的交的那些真子模构成的集合,它按包含关系构成一个偏序集。由于这个集合非空,而且$M$是一个Noether模,所以存在极大元$N$。由于$N$是可约的,所以可以写成$P\cap Q$的形式。由于$P$和$Q$都严格比$N$大,所以它们可以写成有限个不可约子模的交,进而$N$可以写成有限个不可约子模的交。矛盾。
\qed

\section{有限生成模}

\para 一个$R$-模$M$是有限生成的当且仅当存在某个正整数$n$使得$M$是$R^n$的商模,或者用正合列写作
\[
	R^n \xrightarrow{\pi} M\to 0.
\]
考虑$M$的一组有限的生成元$\{m_i\,:\, 1\leq i\leq n\}$,他们生成了一个$R$模,满射通过将$\pi:1_i\mapsto m_i$来线性线性扩张得到,即定义
\[
	\pi\left(\sum_i r_i 1_i\right)=\sum_i r_i m_i.
\]
反过来,如果$M$可以写成某个$R^n$的商模,他自然被$\pi(1_i)$生成。

\pro Noether环上的有限生成模是Noether模。

\para 从自由模模去一些关系是构造一些模的常用手段。比如我们希望构造一个由所有满足$(a,b)=(b,a)$的$(a,b)$构成的一个模,那么最方便的方式就是从$R^2$出发,模去所有$(a,b)-(b,a)$生成的子模。后者被称为一个关系。

自由模模去一个关系得到一个模这个过程,抽象来说,就是存在正合列
\[
	R^I\to R^J \to M\to 0.
\]
这被称为模$M$的free presentation,如果$I$和$J$都是有限的,则$M$被称为finitely presented.

finitely presented模一定是有限生成模,但一般而言,有限生成模不是finitely presented,比如取一个非Noether环$R$以及它的一个非有限生成理想$I$使得$R/I$是一个有限生成模,那么有短正合列$0\to I\to R \to R/I\to 0$,此处由于$I$不是有限生成的,所以$R/I$就不是finitely presented.

但是到了Noether环上的模就不会如此,考虑任意的有限生成模$M$,他可以写成$R^n$的商模,即有短正合列$0\to N \hookrightarrow R^n \to M\to 0$,由于$N$是$R^n$的子模,而$R^n$是$R$上的有限生成模,所以$R^n$是Noether模,继而推知$N$是有限生成的,所以存在满同态$R^m\to N$,这样我们就得到了正合列$R^m\to R^n\to M\to 0$.

\theo Hamiltion-Cayley定理:设$M$是一个有限生成$R$-模,他能被$n$个元素生成. 再设$\mathfrak{a}$是$R$的一个理想,$\varphi:M\to M$是一个自同态。如果$\varphi(M)\subset \mathfrak{a}M$,则存在首一多项式
\[
	p(x)=x^n+p_1x^{n-1}+\cdots+p_n,
\]
其中$p_i\in I$,使得$p(\varphi)=\varphi^n+p_1\varphi^{n-1}+\cdots+p_n$是一个零映射,即任取$m\in M$,有$p(\varphi)(m)=0$. 一般写作$p(\varphi)=0$.

\proof
	设$\{m_1,\cdots,m_n\}$是$M$的生成元。考虑$\varphi(m_i)\in M$,因为$M$是有限生成的,所以
	\[
	\varphi(m_i)=\sum_{j}r_{ij}m_j,
	\]
	其中$r_{ij}\in I$. 将其写作$\sum_{j}(\delta_{ij}\varphi-r_{ij})m_i=0$,其中$\delta_{ij}$当$i=j$的时候为$1$,其他时候为$0$.

	将$M$看作$R[x]$-模,其中$xm=\varphi(m)$,则$\sum_{j}(\delta_{ij}x-r_{ij})m_i=0$. 左乘$x\delta_{ij}-r_{
	ij}$的伴随矩阵,就得到了
	\[
	\det\left(x\delta_{ij}-r_{ij}\right)m_i=0,
	\]
	对任意的$m_i$都成立。所以也对任意的$m\in M$都成立,而$\det\left(x\delta_{ij}-\sum_{j}r_{ij}\right)$就是我们需要的多项式。
\qed

\para 作为Hamiltion-Cayley定理的应用,我们可以断言,有限生成模的任意满自同态是同构。

\proof
	把$M$看出$R[x]$-模,其中$xm=\varphi(m)$,他是有限生成的,因为作为$R$-模是有限生成的。令$I=(x)$是$x$在$R[x]$中生成的主理想。由于$\varphi$是满射,则$IM=M$. 选则映射$\id_M=1:M\to M$,这显然是一个$R[x]$-模同态,由于$\id_M(M)\subset IM$,利用Hamiltion-Cayley定理,存在一个多项式$p$,系数属于$I$使得
	\[
	p(1)m=\left(1^n+xf_1(x)1^{n-1}+\cdots+xf_n(x)\right)m=(1-xq(x))m=0,
	\]
	其中$q\in R[x]$. 或者写作,$1-q(\varphi)\varphi=0$. 因此$q(\varphi)$是$\varphi$的一个逆,进而$\varphi$是一个同构。
\qed

\para 将上述命题应用到自由模。设$M\cong R^n$,而$I=\{x_1$, $\cdots$, $x_n\}$自由生成了$M$(即$\{x_1$, $\cdots$, $x_n\}$的非零$R$-系数线性组合都不为零),则$M$同构于$I=\{x_1$, $\cdots$, $x_n\}$生成的自由模$\bigoplus_{x\in I} R$.

\proof
	$\bigoplus_{x\in I} R$自然可以看成$R^n$,从$\bigoplus_{x\in I} R=R^n$到$M$有自然的满射$\alpha$,他将生成元映到生成元,由于$M\cong R^n$,所以有同构$\beta:M\to R^n$,因此就有了满同态$\alpha\beta:M\to M$,进而是同构。所以$\alpha =(\alpha\beta)\beta^{-1}$就是一个同构。
\qed

利用上面这个命题,可以证明当$m\neq n$时$R^m\not\cong R^n$. 设$m<n$,反证即可,如果$R^m\cong R^n$,则在$R^m$中可以取$n$个自由生成元,但这$n$个自由生成元都由$R^m$的自由生成元生成,所以他们不是自由的,矛盾。

\section{张量积}

\para 张量抽象了多线性函数,尤其是双线性函数。设$A$是右$R$-模,$B$是左$R$-模,那么我们称$f:A\times B\to G$,其中$G$是一个交换群,为一个双线性函数,如果他满足
\[
	f(a+b,a')=f(a,a')+f(b,a'),\quad f(a,a'+b')=f(a,a')+f(a,b'),
\]
以及对$r\in R$满足$f(ar,b)=f(a,rb)$.

特别地,如果我们存在一个双线性函数$\varphi$,以及一个交换群$G$,使得每一个$A\times B$上的双线性函数$f:A\times B\to H$都可以唯一分解为
\[
	f:A\times B\xrightarrow{\varphi} G\xrightarrow{h_f}H,
\]
其中$h_f$对每一个双线性函数$f$存在且唯一,则称呼$G$为$A$与$B$的张量积,记做$A\otimes_R B$,而$\varphi(a,b)$记做$a\otimes_R b$,如果下标$R$不重要,那么我们可以省略他。

如果$G$也有$R$-模结构,那么同态就是$R$-模同态。

\lem 在模范畴内,张量积存在。而且由上面的泛性质,他确定到一个同构。

\proof 对于唯一性,我们考虑两个张量积$(G,\varphi)$和$(G',\varphi')$,那么根据张量积的性质,有分解
\[
	\varphi:A\times B\xrightarrow{\varphi'} G'\xrightarrow{h_{\varphi}}G,\quad \varphi':A\times B\xrightarrow{\varphi} G\xrightarrow{h_{\varphi'}}G',
\]
所以只要验证$h_\varphi\circ h_{\varphi'}=\id_{G}$和$h_{\varphi'}\circ h_{\varphi}=\id_{G'}$就好了,这样我们就得到了$G$和$G'$之间的同构。而上述等式来自于分解的唯一性,显然,我们有分解
\[
	A\times B\xrightarrow{\varphi'} G'\xrightarrow{h_{\varphi}}G\xrightarrow{h_{\varphi'}}G',\quad A\times B\xrightarrow{\varphi'} G'\xrightarrow{\id_{G'}}G',
\]
显然,所以由唯一性得到了$h_{\varphi'}\circ h_{\varphi}=\id_{G'}$,同理有另一个等式。

对于存在性,我们可以直接构造,首先,我们知道在交换群范畴(作为$\zz$-模)有直和存在,那么我们可以构造自由交换群
\[
	F=\bigoplus_{(a,b)\in A\times B} \zz=\sum_{(a,b)\in A\times B}n_{(a,b)}1_{a,b}.
\]
这是一个形式和,系数只有有限项非零,其中$1_{a,b}$是对应指标$(a,b)$的那个$\zz$中的$1$。我们令$H$是由
\[
	1_{a,b}+1_{a',b}-1_{a+a',b},\quad 1_{a,b}+1_{a,b'}-1_{a,b+b'},\quad 1_{ar,b}-1_{a,rb}
\]
生成的子群,那么我们$A\otimes B$就可以构造为$F/H$,令$\varphi(a,b)=a\otimes b=[1_{a,b}]$,即$1_{a,b}$的陪集。这确实是张量积,具体的检验这里就不进行了。\qed

% 设$R$是交换环,$A$和$B$都是双边模,那么$A\otimes_R B$显然有一个$R$-模结构,比如$r(a\otimes b)$可以定义为$(ra)\otimes b$.特别地,如果交换环$T$是一个$R$-代数\footnote{设$f:R\to T$是一个环同态,那么我们可以通过$r\cdot t=f(r)t$在$T$上定义一个$R$-模结构,此时称$T$是一个$R$-代数。此外,$R$-代数同态是一个环同态,同时也是一个$R$-模同态。},设$A$是一个$R$-模,则$A\otimes_R T$就是一个$T$-模。这是因为,我们可以通过$t(a\otimes b)=a\otimes (tb)$来定义标量乘法。

\para 有了两个模的张量积,我们自然也可以拓展为三个模的张量积,我们可以通过模仿两个模的张量积的泛性质\footnote{即三线性的函数可以唯一分解。},定义一个新的三个模之间的张量积$A\otimes B\otimes C$,然后可以检验$(A\otimes B)\otimes C$和$A\otimes (B\otimes C)$同时也满足泛性质,所以他们之间是同构的。在这层意义上,我们可以认为张量积满足结合律,因此我们自然也有了有限个模的张量积。

\para 有同构$A\otimes\bigoplus_i B_i\cong \bigoplus_i (A\otimes B_i)$,$R\otimes A\cong A$以及$(R/I)\otimes B\cong B/IB$,其中$I$是$R$的一个理想。这俩的证明完全是利用泛性质,这里就不说了。

\para 下面讨论张量积在同态下的表现。设$\varphi:M\to M'$和$\psi:N\to N'$是$R$-模同态,则它诱导了一个同态
\[
	\alpha(\varphi,\psi):M\otimes N\to M'\otimes N',
\]
使得复合公式
\[
	\alpha(\varphi'\varphi,\psi'\psi)=\alpha(\varphi',\psi')\alpha(\varphi,\psi)
\]
成立。大家经常将$\alpha(\varphi,\psi)$记作$\varphi\otimes \psi$,两个模同态的张量积的意义后面会阐述,这里看成形式的符号即可。

\proof
	映射
	\[
	\begin{array}{ccc}
		M\times N&\to& M'\otimes N'\\
		(m,n)&\mapsto& \varphi(m)\otimes \psi(n)
	\end{array}
	\]
	显然是双线性的,所以它诱导了映射$M\otimes N\to M'\otimes N'$,而张量积的泛性质中的唯一性给出了复合公式,于是这就是我们想要的$\alpha(\varphi,\psi)$.
\qed

注意到复合公式成立意味着成立交换图
\[
\begin{xy}
	\xymatrix
	{
		M\otimes N\ar[rr]^{\alpha(\id_M,\psi)}\ar[d]_{\alpha(\varphi,\id_{N})}&&M\otimes N'\ar[d]^{\alpha(\varphi,\id_{N'})}\\
		M'\otimes N\ar[rr]^{\alpha(\id_{M'},\psi)}&&M'\otimes N'
	}
\end{xy}
\]
成立。

上面的交换图用范畴论的语言来表述就是,$\otimes N$以及$\otimes \id_N:=\alpha(*,\id_N)$一起构成了一个函子,而同态$N\to N'$将诱导出一个函子间的态射(即自然变换)。这样的同态我们称为自然同态或者函子式同态,描述同态其实就是建立范畴论的初衷,为了抽象出自然变换才定义了范畴等概念。然后到了教科书上,逻辑就反了过来。

\para 上一个命题给出了如下映射:
\[
	\alpha:\Hom(M,M')\times \Hom(N,N')\to \Hom(M\otimes N,M'\otimes N').
\]
由于$\Hom(M,M')$和$\Hom(N,N')$都是$R$-模,并且上述映射是双线性的,所以他自然诱导了一个映射
\[
	\Hom(M,M')\otimes \Hom(N,N')\to \Hom(M\otimes N,M'\otimes N').
\]
这也就是为什么大家将$\alpha(\varphi,\psi)$记作$\varphi\otimes \psi$. 

\para 上面我们已经看到了,$\otimes N$构成一个函子,而实际上,它是右正和函子。

\section{标量扩张与局部化}

\para 设$M$是$R$-模,$S$是$R$-代数,那么$M_S=S\otimes_R M$就是一个$S$-模,通过$s(t\otimes a)=(st)\otimes a$,我们称这个$S$-模$M_S$为$M$经过标量扩充而来的。

\lem 有以下同构:

\no{1} $M_S\otimes_S N_S\cong (M\otimes_R N)_S$.

\no{2} 设$N$是一个$S$-模,有同构$\Hom_R(M,N)\cong \Hom_S(M_S,N)$

\proof 第一个同构。实际上,$(s_1\otimes_R a,s_2\otimes_R b)\mapsto s_1s_2\otimes_R(a\otimes_R b)$是一个双线性映射,所以他唯一诱导了$\varphi:M_S\otimes_S N_S\to (M\otimes_R N)_S$. 反过来,映射$(s,a\otimes_R b)\mapsto (s\otimes_R a)\otimes_S (1\otimes_R b)=(1\otimes_R a)\otimes_S (s\otimes_R b)$也是一个双线性映射,所以他唯一诱导了$\psi:(M\otimes_R N)_S\to M_S\otimes_S N_S$.容易验证他们互逆。

第二个同构。设$\phi\in \Hom_R(M,N)$,我们可以通过$\Phi(s\otimes a)=s\phi(a)$定义$\Phi\in \Hom_S(M_S,N)$,反过来,已知$\Phi\in \Hom_S(M_S,N)$,我们可以通过$\phi(a)=\Phi(1\otimes a)$来定义$\phi\in \Hom_R(M,N)$.不难检验,这是一个同构。 \qed

对有限个$\{M_i\}_{i\in I}$,那么反复利用\no{1}我们就得到了同构
\[
	\bigotimes_{i\in I} \left(S\otimes_RM_i\right)\cong S\otimes_R\bigotimes_{i\in I} M_i.
\]

\para 设$S$是一个$R$-代数,我们有下面三个小命题:

\no{1}设$V$是一个$R$-模,$W$是一个$S$-模,$f:V\to W$是$R$-模同态,则他可以唯一扩张为一个$S$-模同态$V_S\to W$.

\no{2}设$V$和$W$都是$R$-模,$f:V\to W$是$R$-模同态,则他可以唯一扩张为一个$S$-模同态$V_S\to W_S$.

\no{3}设$V$和$W$都是$R$-模,$f:V\times\cdots\times V\to W$是$R$-多线性映射,则他可以唯一扩张为一个$S$-多线性映射$V_S\times\cdots\times V_S\to W_S$.

\proof 
	因为$\Hom_R(V,W)\cong \Hom_S(V_S,W)$,所以\no{1}得证。对于\no{2},我们有$i:W\hookrightarrow W_S$,那么$i\circ f:V\to W_S$就满足\no{1}题设,由\no{1}自然得证。对于\no{3},因为多线性映射他可以唯一分解出一个$V^k\to W$,然后由\no{2},可以唯一扩张为$(V^k)_S\to W_S$,因为$(V^k)_S=V_S\otimes_S \cdots\otimes_S V_S$,引理得证。
\qed

注意到,零映射扩张成的映射依然是零映射,这可以用来给出一些等式。比如在实Lie代数$V$中有Jacobi恒等式$B(u,v,w)=0$对任意的$u$, $v$, $w\in V$成立,其中$B$是一个多线性映射,所以经过复化后依然有$B_\cc (u,v,w)=0$对任意的$u$, $v$, $w\in V_\cc$成立。

\para 注意到$M$和$N$如果都是$R$-模,则$\Hom_R(M,N)$也可以看成一个$R$-模,通过定义$(rf)(m)=rf(m)$. 这样我们自然可以问$\Hom_R(M,N)$关于标量扩张的表现。

设$S$是一个$R$-代数,那么存在一个自然的$S$-模同态
\[
	\alpha:S\otimes\Hom_R(M,N)\to \Hom_S(S\otimes M,S\otimes N).
\]
由于$\alpha$是$S$-模同态,所以只要考察$\alpha(1\otimes \varphi)$即可,其中$\varphi\in \Hom_R(M,N)$. 所谓自然,就是说满足交换图
\[
\begin{xy}
	\xymatrix
	{
		S\otimes\Hom_R(M,N)\ar[rr]^{\alpha_S}\ar[d]&&\Hom_S(S\otimes M,S\otimes N)\ar[d]\\
		T\otimes\Hom_R(M,N)\ar[rr]^{\alpha_T}&&\Hom_T(T\otimes M,T\otimes N)
	}
\end{xy}
\]
如果$S$是一个平坦$R$-模且$M$具有有限表示,则$\alpha$还是一个同构。

\para 一个环$R$中的元素一般不是可逆的,那一个自然的问题就出现了,是否存在一个比这个环大一点的环$S$使得$R$中的某些元素(因为0自然不会是可逆的)在环$S$中是可逆的。一个经典的构造是从整数环$\mathbb{Z}$构造出有理数域$\mathbb{Q}$. 下面以整环的语境复述一遍。

设$R$是一个整环,而$R^{*}=R-\{0\}$是$R$中除去$0$得到的集合,考虑集合$R\times R^{*}$,即所有的二元组$(r,s)$的集合,其中$s\neq 0$. 其中$r$被称为分子,而$s$被称为分母。

分数里面很重要的是消去率,比如$8/4$应该等同于$2/1$. 于是上面构造的集合还太大,应该适当构造等价关系来完成上面的等同。实际上,这个等价关系就是:$(r,s)\sim (r',s')$当且仅当$rs'=r's$,即一个二元组消去同一因子或者乘以一个相同的元素都是等价的。此时,包含$(r,s)$的等价类记作$r/s$.

剩下的就是定义运算,
\[
	\frac{r}{s}+\frac{r'}{s'}=\frac{rs'+r's}{ss'},\quad \frac{r}{s}\frac{r'}{s'}=\frac{rr'}{ss'}.
\]
不难检验这个运算的定义这和代表元的选取无关。注意到整环的条件会用到乘法里面,因为如果不是整环,两个分母相乘可能得到零,而零并不能作为分母。这个环记作$F(R)$,乘法单位元是$1/1$,而零元是$0/1$,通过建立$r\in R$与$r/1\in F(R)$的等同,可以将$R$看成$F(R)$的子环。$F(R)$还是一个域,实际上,任取$r/s$,其中$r\neq 0$,则$s/r$是它的逆。称$F(R)$是整环$R$的分式域。

\para 这里我们来看另一种构造。设$R$是一个整环,记$R^*=R-\{0\}$,可以构造一个多项式环$R[R^*]$,对应于$r\in R^*$,$R[R^*]$中的元素我们记作$x_r$. 那么$F(R)$可以定义为
\[
	R[R^*]\big/\bigl\langle\{rx_r-1\,:\, r\in R^*\}\bigr\rangle.
\]
换而言之,$x_r$就是$r$的逆。在$F(R)$中的等号记作$\equiv$. 从构造,$rx_s$应该对应$r/s$.

下面我们来检验分子分母消去律,即计算$rx_{rs}$. 由于$s(rx_{rs})\equiv (sr)x_{rs}\equiv 1$,所以$rx_{rs}\equiv x_s$. 两边乘以$x_r$,就得到了$x_rx_s\equiv x_{rs}$. 这个环的加法就是
\[
	ax_{b}+cx_d\equiv a(dx_d)x_{b}+c(bx_{b})x_d\equiv (ad+bc)x_bx_d \equiv (ad+bc)x_{bd},
\]
乘法就是
\[
	(ax_{b})(cx_d)\equiv acx_bx_d\equiv acx_{bd}.
\]

\para 下面要推广分式域的概念,假设$R$是一个环,不一定是一个整环。考虑$S$是$R$的任意子集,我们可以类比定义一个环$S^{-1}R$或者$R[S^{-1}]$,称之为$R$关于$S$的分式环,
\[
	S^{-1}R=R\bigl[\{x_s\,:\, s\in S\}\bigr]\big/\bigl\langle\{sx_s-1\,:\, s\in S\}\bigr\rangle.
\]
这个环的加法就是
\[
	ax_{b}+cx_d\equiv a(dx_d)x_{b}+c(bx_{b})x_d\equiv (ad+bc)x_bx_d,
\]
乘法就是
\[
	(ax_{b})(cx_d)\equiv acx_bx_d.
\]
并且从
\[
	bd(ax_{b}-cx_d)\equiv ad-bc
\]
可知$ax_{b}\equiv cx_d$当且仅当$ad-bc\equiv 0$. 因此,如果$s$, $t$, $st\in S$,则分子分母消去律$sx_{st}\equiv x_t$顺道也是成立的。利用消去率,我们有$x_{st}\equiv sx_s x_{st}\equiv x_s x_t$.

观察上面构造的分式乘法。由于我们想要$ax_b$对应$a/b$,$cx_d$对应$c/d$,那么是否$(ax_{b})(cx_d)\equiv acx_bx_d$对应$(ac)/(bd)$?如果$bd\in S$,那么这是对的,因为$x_bx_d\equiv x_{bd}$. 但是一般而言这是不对的,所以对$S$我们希望是对乘法封闭的。此外,观察加法也是,只有在$bd\in S$的时候才有对应$a/b+c/d=(ad+bc)/(bd)$. 为了,将$T$看成一个$R$-代数,即建立环同态$r\mapsto r/1$,最好还需要假设$1\in S$.

如果$S$是$R$的一个子集,满足$1\in S$且$S$对乘法封闭,这样一个集合称之为乘性子集。

\lem 设$a\in R$,则$a\equiv 0$当且仅当存在$s \in S$使得$as=0$.

\proof 
	如果存在$s \in S$使得$as=0$,则$a\equiv asx_s\equiv 0$. 反过来,假设$a\equiv 0$但$as\neq 0$对任意$s\in S$都成立。由于$a\equiv 0$,则$a\in \bigl\langle\{sx_s-1\,:\, s\in S\}\bigr\rangle$. 逐次比较,唯一可能的选择就是关于$x$的一次项的线性组合,即
	\[
	a=\sum_{i=1}^n a_i(s_ix_{s_i}-1),
	\]
	其中求和是有限的。所以$a_is_i=0$对任意的$i$都成立,且$\sum_i a_i=-a$. 将$\sum_i a_i=-a$两边乘以$s_1\cdots s_n$,得到$as_1\cdots s_n=0$,由于$S$对乘法封闭,所以$s_1\cdots s_n\in S$,然后利用$as\neq 0$的假设就得到了矛盾。
\qed

\para 利用上面这个引理,可以给出$S^{-1}R$的另一个定义。它是在$R\times S$中引入等价关系得到的,这个等价关系写作:$(r,s)\sim (r',s')$当且仅当存在一个$t\in S$使得$t(rs'-r's)=0$. 将包含$(r,s)$的等价类记作$r/s$,加法和乘法定义的检验这里就略去了。一般教科书上,会直接使用这个定义而并不给出为什么这样定义等价关系,而这里给出了一个启发。

通过$r\mapsto r/1$,就给出了一个环同态$\varphi:R\to S^{-1}R$. 利用上面的引理,可以看到$R\to S^{-1}R$不一定是单同态,$a/1=0$并不在只在$a=0$的时候才成立,在存在$s\in S$使得$sa=0$的时候也成立。不管怎样,$S^{-1}R$是一个$R$-代数。设$M$是一个$R$-模,则$S^{-1}R\otimes_R M$是一个$S^{-1}R$-模,记作$S^{-1}M$. 这又是一个标量扩张的例子。

\pro 设$S$是$R$的一个乘性子集,记同态$R\to S^{-1}R$为$i_S$. 再设$T$是一个环,如果$\varphi:R\to T$是一个环同态,且使得$\varphi(S)=\{\varphi(s)\,:\, s\in S\}$中的元素都是$T$中的可逆元,则存在唯一的环同态$\pi:S^{-1}R\to T$使得分解$\varphi:R\xrightarrow{i_S}S^{-1}R\xrightarrow{\pi} T$成立。

换句话说,如果一个$R$-代数$T$中$S\cdot 1$都是可逆的,则$T$是一个$S^{-1}R$-代数。这被称为分式环的泛性质,这就指出分式环的实质在于将某个乘性子集搞成可逆的。

\proof
	令$\pi(r/s)=\varphi(r)\varphi(s)^{-1}$. 下面检验$\pi$是一个环同态,关键不在乘法上,而在于分式不同代表元的选取。令$r'/s'=r/s$,则存在一个$t\in S$使得$t(r's-rs')=0$,于是
	\[
		\pi(t)(\pi(r')\pi(s)-\pi(r)\pi(s'))=0,
	\]
	由于$\pi(t)$可逆,所以$\pi(r')\pi(s)=\pi(r)\pi(s')$或者$\pi(r)\pi(s)^{-1}=\pi(r')\pi(s')^{-1}$. 所以$\pi$是一个环同态。至于唯一性,由构造,$\pi(r/s)=\varphi(r)\varphi(s)^{-1}$,所以它由$\varphi$唯一决定。最后,$\pi i_S(r)=\pi(r/1)=\varphi(r)$,所以分解成立。
\qed

\para 固定环$R$,以及设$S$是$R$的一个子集。任取环同态$\varphi:R\to T$,如果$S$是可逆元构成的集合,则$\varphi(S)$也是可逆元构成的集合,如果$S$是乘性子集,则$\varphi(S)$也是乘性子集。

已知,$R$-代数构成了一个范畴,它的对象是二元组$(T,\varphi_T)$. 设$S$是$R$的一个乘性子集,如果$\varphi_T(S)$中的元素都是可逆的,这样的代数构成了一个范畴,我们记作$\mathrm{Alg}_{(R,S)}$,它的态射就是普通的$R$-代数的同态,因为同态自然将可逆元变成了可逆元。由于$T$是一个$R$-代数,它本身就是一个$R$-模,给定乘性子集$S\subset R$,通过标量扩张,$S^{-1}T=S^{-1}R\otimes_R T$就得到了一个$S^{-1}R$-代数,实际上$S^{-1}T$作为环同构于$\varphi_T(S)^{-1}T$. 记$i_T=i_S\otimes 1:T\to S^{-1}T$是自然诱导的同态。

任取$(T,\varphi_T)\in \mathrm{Alg}_R$,则局部化将给出一个函子$S^{-1}:T\mapsto S^{-1}T$,对于态射$\Psi:(T,\varphi_T)\to (U,\varphi_U)$,即满足$\varphi_U=\Psi\varphi_T$的环同态$\Psi:T\to U$. 考虑复合同态$i_{U}\Psi:T\to S^{-1}U$,由于$i_{U}\Psi(\varphi_T(s))=i_{U}\varphi_U(s)=s/1\in S^{-1}U$是可逆元,所以由局部环$S^{-1}T=\varphi_T(S)^{-1}T$的泛性质,存在同态
\[
	S^{-1}(\Psi):S^{-1}T\to S^{-1}U.
\]
使得交换图成立
\[
\begin{xy}
	\xymatrix{
		T\ar[rr]^-{i_T} \ar[d]_-{\Psi}&&S^{-1}T \ar@{.>}[d]^-{S^{-1}(\Psi)}\\
		U\ar[rr]^-{i_U} &&S^{-1}U
	}
\end{xy}
\]
因此,$S^{-1}$确实是一个函子。

反过来,从范畴$\mathrm{Alg}_{(R,S)}$到$\mathrm{Alg}_R$有自然的遗忘函子$f$,他忘掉了$S\cdot 1$在相应的$R$-代数中是可逆的。

\theo 存在双函子同构:
\[
	F(-,\star):\mathrm{Alg}_R\bigl(-,f(\star)\bigr)\to \mathrm{Alg}_{(R,S)}\bigl(S^{-1}(-),\star\bigr).
\]
因此$S^{-1}$是遗忘函子的左伴随函子,继而是一个自由函子。这也可以看成$S^{-1}R$的另一种定义,只要将$f$的左伴随函子作用在$R$上就得到了$S^{-1}R$.

\proof
	首先构造同构,任取同态$R$-代数同态$\Psi:X\to f(Y)$,其中$Y$使得$\varphi_Y(S)$在$Y$中是可逆的。由于是$R$-代数同态,所以$\varphi_Y=\Psi\varphi_X$. 任取$s\in S$,有$\varphi_Y(s)=\Psi\varphi_X(s)$可逆,通过分式环的泛性质,有唯一的同态$F(X,Y)(\Psi):\varphi_X(S)^{-1}X=S^{-1}X\to Y$. 于是$F(X,Y)$就是我们构造的同构。由构造,这是一个单射,反过来,任取$\rho:\mathrm{Alg}_{(R,S)}(S^{-1}(X),Y)$,我们有$F(X,Y)(i_X\circ \rho)=\rho$,因此这是一个满射。

	最后要验证函子性同构,分别对$X$和$Y$检验自然变换即可。这点过于琐碎,这里就略去了。
\qed

任取$X\in \mathrm{Alg}_{(R,S)}$,它可以自然地看成一个$S^{-1}R$-代数,这由分式环的泛性质保证,反过来,任意的$S^{-1}R$-代数都可以看成$\mathrm{Alg}_{(R,S)}$里面的元素,所以有理由认为$\mathrm{Alg}_{S^{-1}R}$和$\mathrm{Alg}_{(R,S)}$作为范畴等价。事实上,可以定义一个函子$\mathrm{Alg}_{(R,S)}\to \mathrm{Alg}_{S^{-1}R}$,对对象,将任意的$R$-代数$X$看作$S^{-1}R$-代数$X$,对态射,任取$\varphi\in \mathrm{Alg}_{(R,S)}(X,Y)$,它也可以看作$S^{-1}R$-代数同态(请检查)。在上面的等同下,由泛性质,$\mathrm{Alg}_{(R,S)}(X,Y)=\mathrm{Alg}_{S^{-1}R}(X,Y)$,所以这是一个完全且忠实的函子。同样,任取$S^{-1}R$-代数$X$,他也可以看成$\mathrm{Alg}_{(R,S)}$里面的对象。所以由Proposition \eqref{equivcat},$\mathrm{Alg}_{S^{-1}R}$和$\mathrm{Alg}_{(R,S)}$是等价范畴。

也正是如此,因此上面的双函子同构可以写成
\[
	F(-,\star):\mathrm{Alg}_R\bigl(-,f(\star)\bigr)\to \mathrm{Alg}_{S^{-1}R}\bigl(S^{-1}(-),\star\bigr).
\]
但是,范畴$\mathrm{Alg}_{(R,S)}$并不需要$S^{-1}R$的构造,所以我们才谈论的是这个范畴,然后才可以通过自由函子给出$S^{-1}R$的一个定义。当然,这里已经可以事后诸葛亮地说,函子$S^{-1}$是与$S^{-1}R$-代数到$R$-代数的遗忘函子左伴随的自由函子。

\para 上面讨论了分式环几个定义。下面讨论分式环的几个重要实例。从一个事实开始,设$S\subset R$是一个乘性子集,而$S'$是$S$中元素的全部因子构成的乘性子集,则$S^{-1}R=S'^{-1}R$.