\chapter{模(一)}
\ThisULCornerWallPaper{1}{../Pictures/7.png}

\section{正合列}

首先回顾正合列的相关知识,我们很久以前已经介绍过了左$R$-模范畴中的正合列。现在由于我们对Abel范畴有了一定的任意,所以这里直接将正合列定义到一个Abel范畴中,在许多操作上,他并不比模范畴麻烦多少。

\begin{para}
若有一族态射$f_i:A_i\to A_{i+1}$满足$\im f_i\approx \ker f_{i+1}$,则列
\[
	\cdots \xrightarrow{f_{i-1}}A_i \xrightarrow{f_i} A_{i+1} \xrightarrow{f_{i+1}} A_{i+1}\xrightarrow{f_{i+1}}\cdots
\]
被称为正合的。
\end{para}

对正合列来说,成立$f_{i+1}f_i=0$. 如果$\im f_i\approx \ker f_{i+1}$,则$\coker \ker f_{i+1} \approx \coker \im f_i$,换而言之,存在同构$t$使得$\coker \ker f_{i+1}=t\coker \im f_i$. 因此
\begin{align*}
f_{i+1}f_i&=\im(f_{i+1})\coim(f_{i+1})\im(f_{i})\coim(f_{i})\\
	&=\im(f_{i+1})(\coker \ker f_{i+1}) \im(f_{i})\coim(f_{i})\\
	&=\im(f_{i+1})t(\coker \im f_i) \im(f_{i})\coim(f_{i})\\
	&=\im(f_{i+1})t\bigl((\coker \im f_i) \im(f_{i})\bigr)\coim(f_{i})\\
	&=\im(f_{i+1})t0\coim(f_{i})\\
	&=0.
\end{align*}

\begin{para}[短正合列]
考虑正合列
\[
	0\to \cdot \xrightarrow{f} \cdot.
\]
为了满足正和性,我们需要$\ker f \approx \im 0=0$,即$\ker f = 0$,所以这个正合列无外乎是在说$f$是一个单态。对偶地,考虑正合列
\[
	\cdot\xrightarrow{f} X \to  0.
\]
为了满足正和性,我们需要$\im f \approx \ker 0$,两边作用一次$\coker$将得到
\[
	\coker f=\coker \ker \coker f=\coker \im f \approx \coker \ker 0.
\]
注意到,此时$\ker 0=\id_X$是一个满态,所以$\coker f \approx \coker(\ker 0)=0$,即$\coker f = 0$. 因此这个正合列无外乎是在说$f$是一个满态。

现在,我们来到正合列
\[
	0\to \cdot \xrightarrow{f} \cdot \to 0,
\]
由上面可知$f$即是单态又是满态,在Abel范畴中这等价于说$f$是一个同构。

最后,正合列
\[
	0\to \cdot \xrightarrow{f} \cdot \xrightarrow{g}\cdot\to 0,
\]
被称为\idx{短正合列}。它等价于说,$f$单$g$满且$\im f\approx \ker g$.
\end{para}

\begin{para}
设$T:\mathcal{C}\to \mathcal{D}$是准加性范畴之间的函子,如果对$\mathcal{C}$中的态射$f$, $g$成立$T(f+g)=Tf+Tg$. 则称$T$是一个加性函子。更形式地说,
\[
	T:\mathcal{C}(X,Y)\to \mathcal{D}(TX,TY)
\]
是一个交换群同态。
\end{para}

从定义,不难看到对零态射$0$,成立$T0=0$. 如果$0$是$\mathcal{C}$中的零对象,则$T:\mathcal{C}(0,0)\to \mathcal{D}(T0,T0)$告诉我们$\mathcal{D}(T0,T0)$是一个单点集,所以Propositon \ref{zeroobj}告诉我们$T0$是$\mathcal{D}$中的零对象,即$T0=0$.

\begin{pro}
$T:\mathcal{C}\to \mathcal{D}$是一个加性函子当且仅当$T$与双积可交换。所谓与双积可交换,就是把双积映成双积(包括其对应的相关态射)。
\end{pro}

\begin{proof}
设$T$是一个加性函子,而$p_1$, $p_2$, $i_1$和$i_2$是双积的四个典范态射,于是
\[
	p_ai_a = \id \quad \Rightarrow \quad T(p_a)T(i_a) = \id,
\]
\[
	p_ai_b = 0 \quad \Rightarrow \quad T(p_a)T(i_a) = T0=0,
\]
\[
	i_1p_1+i_2p_2 = \id \quad \Rightarrow \quad T(i_1)T(p_1)+T(i_2)T(p_2) = \id,
\]
告诉我们$T$与双积可交换。$Tp_1$, $Tp_2$, $Ti_1$和$Ti_2$是新双积的四个典范态射。

反过来,考虑态射$f$, $g:X\to Y$. 不难发现$f+g=c(f\oplus g)d$. 其中$d:X\to X\oplus X$满足唯一分解$p^X_1d=p^X_2d=\id_X$,这直接来自于积的泛性质,此外,$c:Y\oplus Y\to Y$满足唯一分解$ci^Y_1=ci^Y_2=\id_Y$,这直接来自于余积的泛性质。所以
\[
	c(f\oplus g)d=c(i_1^Y f p_1^X+i_2^Y g p_2^X)d=c i_1^Y f p_1^X d+ ci_2^Y g p_2^Xd=f+g.
\]

设$T$与双积可交换,双积的泛性质告诉我们$T(f\oplus g)=T(f)\oplus T(g)$. 于是,
\[
	T(f+g)=T(c(f\oplus g)d)=T(c)T(f\oplus g)T(d)=T(c)(T(f)\oplus T(g))T(d)=T(f)+T(g).
\]
所以$T$是加性函子。
\end{proof}

由于在Abel范畴中,双积等价于有限积等价于有限余积。所以只要函子与有限积或者有限余积与可交换,这就是加性函子。

\begin{para}
设$R:\mathcal{C}\to \mathcal{D}$与$L:\mathcal{D}\to \mathbb{C}$是一对伴随函子,其中$R$是右伴随函子,而$L$是左伴随函子。如果$\mathcal{C}$与$\mathcal{D}$都是Abel范畴,由于右伴随函子与极限可交换、左伴随函子与余极限可交换,所以不管左伴随函子还是右伴随函子,它们都是加性函子。

此外,由于核是一个极限,所以$R(\ker f)= \ker(Rf)$. 对偶地,$L(\coker g)= \coker(Lg)$.
\end{para}

\begin{lem}
假设$\mathcal{C}$与$\mathcal{D}$都是Abel范畴,$R:\mathcal{C}\to \mathcal{D}$是一个右伴随函子。如果$f$是$\mathcal{C}$中的单态,则$Rf$是$\mathcal{D}$中的单态。
\end{lem}

\begin{proof}
已知$f$是一个单态当且仅当$\ker f=0$. 由于右伴随函子与核可交换,所以$\ker(Rf)= R(\ker f)=0$,即$Rf$也是单态。
\end{proof}

在Abel范畴中,$f$是一个单态等价于$f=\im f$. 同时,由上面的引理,$Rf$也是单态,所以$Rf=\im(Rf)$,结合这两点就得到了
\[
	R(\im f)=Rf=\im(Rf).
\]
这意味着,对单态而言,$\im$与$R$也是可交换的。因此,如果$\im f\approx \ker g$,其中$f$是一个单态,则$\im(Rf)\approx \ker(Rg)$,其中$Rf$是一个单态。

\begin{pro}
如果$R$是一个从Abel范畴到Abel范畴的右伴随函子,则对正合列$0\to X_1\xrightarrow{f_1} X_2\xrightarrow{f_2} X_3$,他将诱导一个正合列
\[
	0\to R(X_1)\xrightarrow{R(f_1)} R(X_2)\xrightarrow{R(f_2)} R(X_3).
\]
\end{pro}

由上一个引理和其推论立即可知。对偶地,可以知道:

\begin{pro}
如果$L$是一个从Abel范畴到Abel范畴的左伴随函子,则对正合列$X_1\xrightarrow{f_1} X_2\xrightarrow{f_2} X_3 \to 0$,他将诱导一个正合列
\[
	L(X_1)\xrightarrow{L(f_1)} L(X_2)\xrightarrow{L(f_2)} L(X_3) \to 0.
\]
\end{pro}

将满足上面两个命题的加性函子抽象出来。

\begin{para}
设$T$是一个加性函子,如果他把正和列$0\to X_1\xrightarrow{f_1} X_2\xrightarrow{f_2} X_3$变成正合列
\[
	0\to T(X_1)\xrightarrow{T(f_1)} T(X_2)\xrightarrow{T(f_2)} T(X_3),
\]
则称$T$是一个左正和函子。对偶地,如果他把正和列$X_1\xrightarrow{f_1} X_2\xrightarrow{f_2} X_3\to 0$变成正合列
\[
	T(X_1)\xrightarrow{T(f_1)} T(X_2)\xrightarrow{T(f_2)} T(X_3)\to 0,
\]
则称$T$是一个右正和函子。

如果把短正合列变成短正合列,则称$T$是一个正和函子。所以,正和性等价于左正和又右正和。
\end{para}

综上,我们已经证明了:

\begin{thm}
设函子从Abel范畴到Abel范畴,则左伴随函子右正合,而右伴随函子左正和。
\end{thm}

\begin{pro}
设$\mathcal{C}$是一个Abel范畴,则函子$\mathcal{C}(X,-):\mathcal{C}\to \mathsf{Ab}$以及$\mathcal{C}(-,X):\mathcal{C}^\text{op} \to \mathsf{Ab}$都是左正合函子。
\end{pro}

\begin{proof}
	设
	\[
		0\to\cdot \xrightarrow{f}\cdot \xrightarrow{g}\cdot
	\]
	是一个正合列,我们要证明
	\[
	0\to \cdot \xrightarrow{f_*}\cdot \xrightarrow{g_*}\cdot
	\]
	也是一个正合列。首先,我们说明$\ker f_*=0$. 由于是在$\mathsf{Ab}$中,所以考虑$f_*(g)=fg=0$,由于$f$是单态,所以$g=0$,因此$\ker f_*=f_*^{-1}(0)=0$. 然后,由于$g_*f_*=(gf)_*$,所以$\im f_*\subset \ker g_*$. 最后只要检验$\ker g_*\subset \im f_*$. 任取$h$使得$g_*(h)=gh=0$,由$\ker g$的泛性质,存在分解$h=\ker(g)k=\im(f)k=fk=f_*(k)$,其中$\im(f)=f$来自于$f$是一个单态,所以$\ker g_* \subset \im f_*$.

	完全类似地,可以证明$\mathcal{C}(-,X):\mathcal{C}^\text{op} \to \mathsf{Ab}$是左正合函子。
\end{proof}

\section{追图}

我们将在Abel范畴中谈论追图。

\begin{lem}\label{zhuitu}
设$\mathcal{C}$是一个Abel范畴,而$X\in \mathcal{C}$是一个对象。考虑所有指向$X$的态射,在其中可以定义关系$\sim$如下:设$f:Y\to X$和$g:Z\to Y$,如果存在一个$W\in \mathcal{C}$以及两个满态$u:W\to Y$以及$v:W\to Z$使得$fu=gv$,则记$f\sim g$. 以交换图记,就是说,$f\sim g$当且仅当存在满态$u$和$v$使得如下交换图成立,
\[
\xymatrix{
	\cdot \ar[r]^u \ar[d]_v & \cdot\ar[d]^f\\
	\cdot \ar[r]^g & \cdot
}
\]
我们可以断言,$\sim$是一个等价关系,其中,自反性与对称性是显然的。
\end{lem}

为证明传递性,我们需要如下引理。

\begin{lem}\label{mt}
	在Abel范畴中,设$f:X\to Z$和$g:Y\to Z$是两个态射,则存在纤维积$X\times_Z Y$,投影分别记作$p_X$和$p_Y$. 此外,如果$f$是满态,则$p_Y$是满态.
\end{lem}

\begin{proof}[Proof of Lemma \ref{mt}]
	在Abel范畴中,我们已经知道积是存在的,即$X\times Y$存在,投影分别记作$\pi_X$和$\pi_Y$. 并且,核是存在的,则$X\times_Z Y$可以定义为$\ker(f\pi_X-g\pi_Y)$的定义域,而$p_X=\pi_X\ker(f\pi_X-g\pi_Y)$以及$p_Y=\pi_Y\ker(f\pi_X-g\pi_Y)$.

	现在,设$f$是一个满态,下面说明$f\pi_X-g\pi_Y$是满态。由于在Abel范畴中$X\times Y$也是双积$X\oplus Y$,所以存在映射$i_X:X\to X\oplus Y=X\times Y$使得$\pi_Xi_X=\id_X$还有$\pi_Yi_X=0$. 设$h(f\pi_X-g\pi_Y)=0$,由于$0=h(f\pi_X-g\pi_Y)i_X=hf\id_X=hf$以及$f$是一个满态,所以$h=0$. 此即所证。

	利用$f\pi_X-g\pi_Y$是满态,我们有$f\pi_X-g\pi_Y=\coim(f\pi_X-g\pi_Y)=\coker(\ker(f\pi_X-g\pi_Y))$. 最后,设$hp_Y=h\pi_Y\ker(f\pi_X-g\pi_Y)=0$,由$\coker$的泛性质,存在分解$h\pi_Y=k\coker(\ker(f\pi_X-g\pi_Y))=k(f\pi_X-g\pi_Y)$. 复合上$i_X$后,我们将得到
	\[
	0=h\pi_Yi_X=k(f\pi_X-g\pi_Y)i_X=kf,
	\]
	由于$f$是满态,所以$k=0$. 因此
	\[
	h=h\pi_Yi_Y=k(f\pi_X-g\pi_Y)i_Y=0.
	\]
	这就说明了$p_Y$是一个满态。
\end{proof}

\begin{proof}[Proof of Lemma \ref{zhuitu}]
	若$f\sim g$且$g\sim h$,考虑如下交换图
	\[
	\xymatrix{
	&Z_1\ar[r]^a\ar[rd]_b&X\ar[rd]^f&\\
	Z_1\times_{X'} Z_2\ar[ru]^{p_{Z_1}}\ar[rd]_{p_{Z_2}}&&X'\ar[r]^g&Y\\
	&Z_2\ar[ru]^c\ar[r]_d&X''\ar[ru]_h&
	}
	\]
	其中$a$, $b$, $c$, $d$都是满态。由上一个引理,$c$是满态给出$p_{Z_1}$是满态,$d$是满态给出$p_{Z_2}$是满态。最后,满态$ap_{Z_1}$和$dp_{Z_2}$给出了$f\sim h$.
\end{proof}

\begin{para}
从Lemma \ref{zhuitu},我们得到了指向$X$的态射的等价类,其中的元素称为对象$X$的伪元素,记作$[x]$,其中$x$是等价类中的代表元。而$[x]\in \mathcal{C}(-,X)/\sim$记作$[x]\in_m X$. 

设$[x]\in_m X$,我们记$-[x]=[-x]$,这是良定的,实际上,如果$x\sim x'$,则存在两个满态$u$, $v$使得$xu=x'v$,所以$-xu=-x'v$,即$-x\sim -x'$. 

值得注意的一点是,$[0]$中的元素都是$0$,即,如果$[x]=[0]$或$x\sim 0$,则$x=0$. 实际上,$x\sim 0$告诉我们存在满态$u$, $v$使得$xu=0v=0$,再由$u$是一个满态,所以$x=0$. 因此,$x\sim 0$等价于$x=0$. 为方便起见,我们将$[0]$也记作$0$.

设$f:X\to Y$是一个态射,$[x]\in_m X$,记$f(x)=[fx]$. 不难看到,$f:X\to Y$给出了一个映射$\mathcal{C}(-,X)/\sim \to \mathcal{C}(-,Y)/\sim $,实际上,如果$x\sim x'$,则$fx\sim fx'$. 我们依然将这个映射记作$f$. 特别地,$f(0)=[0]$,$f(x)=[0]$等价于$fx=0$.

再设$g:Y\to Z$是另一个态射,则$gf(x)=[gfx]=g(f(x))$. 这就是复合规则,和映射的情况是一摸一样的。
\end{para}

一般来说,所有指向$X$的态射的全体都不能构成一个集合。但是,当我们考虑等价类的全体,它们确实是可能构成一个集合的。比如在模范畴中就是如此,为此,我们需要下面的引理。

\begin{lem}
在模范畴中,$f\sim g$当且仅当$\im f = \im g$.
\end{lem}

\begin{proof}
如果$f\sim g$,则存在满态$u$, $v$使得$fu=gv$. 两边求$\coker$将得到$\coker(fu)=\coker(gv)$,因为$u$, $v$是满态,所以$\coker f=\coker(fu)=\coker(gv)=\coker g$. 求核之后就得到了$\im f=\im g$. 注意,这点在任意的Abel范畴中都成立。

反过来,设$f:X\to Z$而$g:Y\to Z$,考虑$X\times_Z Y=\{(x,y)\,:\,f(x)=g(y)\}$. 如果$\im f=\im g$,则$f(X)=g(Y)$. 所以任取$x_0\in X$都存在一个$y_0\in Y$使得$f(x_0)=g(y_0)$,此时$x_0=p_X(x_0,y_0)$,所以$p_X$是满同态。同理,$p_Y$是满同态。
\end{proof}

因此,在模范畴,指向$M$的箭头的等价类一一对应于$M$的子模,这当然构成一个集合。特别地,在$M$的子模的全体中,存在以单元素生成的子模,换而言之,明确了这些子模,也就明确了$M$的所有元素。这就是Abel范畴中伪元素的含义。这点还可以体现在符号$f(x)$上,实际上,如果我们把$[x]$理解成子模$\im x$,则$f(x)=[fx]=\im(f\circ x)=f(\im x)$. 这就还原了作为映射的$f$. 

下面的追图法告诉我们,在Abel范畴中,我们判断一个态射是单态还是满态可以应用模范畴中类似的手段来做。

\begin{pro}[追图法]
设$f:X\to Y$是一个态射,
\begin{compactenum}[~~~(1)]
\item $f$是单态,当且仅当$f(x)=0$可以推出$x=0$.
\item $f$是单态,当且仅当$f(x)=f(x')$可以推出$[x]=[x']$.
\item $f$是满态,当且仅当任取$[y]\in_m Y$,都存在$[x]\in_m X$使得$f(x)=[y]$.
\item $f=0$当且仅当任取$[x]\in_m X$都有$f(x)=0$.
\item $\cdot \xrightarrow{f} \cdot \xrightarrow{g}\cdot$是正和的,当且仅当$gf=0$且如果$g(y)=0$,则存在$[x]$使得$[y]=f(x)$.
\item 如果$f(x)=f(x')$,则存在$[x'']\in_m X$使得$f(x'')=0$. 且对任意满足$g(x)=0$的态射$g$,我们都有$g(x'')=-g(x')$. 对任意满足$h(x')=0$的态射,我们都有$h(x'')=h(x)$.
\end{compactenum}
\end{pro}

最后一点看着奇怪,实际上$x''$是模范畴中$x-x'$的类比。追图法的存在让我们可以像处理模范畴一样来处理Abel范畴中的交换图。

\begin{proof}
依次证明如下。
\begin{compactenum}[~~~(1)]
\item 如果$f$是单态,于是$f(x)=0$等价于$fx=0$等价于$x=0$等价于$x\sim 0$. 反过来,$f\ker f=0$等价于$f(\ker f)=0$,所以$\ker f\sim 0$即$\ker f=0$即$f$是单态。

\item 如果$f(x)=f(y)$,则$fx\sim fy$,所以存在满态$u$, $v$使得$fxu=fyv$,于是$xu=yu$给出了$x\sim y$. 反过来,如果$fx=0$,则$f(x)=0=f(0)$给出$x=0$. 所以$f$是一个单态。

\item 假设$f$是满态。设$f:X\to Y$和$y:X'\to Y$,我们考虑纤维积$X\times_Y X'$,由于$f$是满态,所以$p_{X'}:X\times_Y X'\to X'$是满态。此外,由$fp_X=yp_{X'}$,我们得到$f(p_X)=[y]$. 

反过来,设$f$不是满态,那么任取$[x]\in_m X$,都有$f(x)\neq [\id_Y]$. 实际上,如果$f(x)=[\id_Y]$,则存在满态$u$, $v$使得$fxu=v$,由于$v$是满态,所以$f$也必须是满态,矛盾。

\item 显然。

\item 考虑分解$f=\im(f)\coim(f)=\ker(g)\coim(f)$,因此$gf=0$是显然的。现在,如果$g(y)=0$,则$gy=0$. 由$\ker g$的泛性质,成立分解$y=\ker(g)h$. 考虑$h$和$\coim f$的纤维积,存在投影$p_h$和$p_{\coim(f)}$使得$hp_h=\coim(f)p_{\coim(f)}$,且从$\coim f$是一个满态可以得到$p_h$是一个满态. 所以$yp_h=\ker(g)hp_h=\ker(g)\coim(f)p_{\coim(f)}=fp_{\coim(f)}$给出了$f(p_{\coim(f)})=[y]$.

反过来,$gf=0$或等价的式子$g\im(f)=0$给出了$\im f \leq \ker g$. 我们下面只要证明$\ker g \leq \im f$. 由于$g\ker g=0$,所以存在一个$x\in_m X$使得$f(x)=[\ker g]$,此即存在满态$u$使得$\ker(g)u=fx$. 由Proposition \ref{uni},可以断言$\ker(g)\approx \im(fx)$. 然后考虑分解$fx=\im(f)\coim(f)x$,由Lemma \ref{lem1},存在唯一的单态$t$使得$\im(fx)=\im(f)t$,因此$\im(fx)\leq \im(f)$. 结合$\ker(g)\approx \im(fx)$,这就给出了我们需要的$\ker g\leq \im f$.

\item 如果$f(x)=f(x')$,则存在满态$u$, $v$使得$fxu=fx'v$,此时考虑$[xu-x'v]\in_m X$,它就满足所有的要求。
\end{compactenum}
\end{proof}

作为追图法的直接应用,我们来证明非常实用的五引理。

\begin{lem}[五引理]\label{5-lem}
在Abel范畴$\mathcal{C}$中,如果存在如下交换图
\[
	\xymatrix{
	A_1\ar[r]^{f_1}\ar[d]^{h_1}&A_2\ar[r]^{f_2}\ar[d]^{h_2}&A_3\ar[r]^{f_3}\ar[d]^{h_3}&A_4\ar[r]^{f_4}\ar[d]^{h_4}&A_5\ar[d]^{h_5}\\
	B_1\ar[r]^{g_1}&B_2\ar[r]^{g_2}&B_3\ar[r]^{g_3}&B_4\ar[r]^{g_4}&B_5\\
	}
\]
且横向的两行都是正合列。则,
\begin{compactenum}[~~~(1)]
\item 如果$h_2$, $h_4$是单态,$h_1$是满态,则$h_3$是单态。
\item 如果$h_2$, $h_4$是满态,$h_5$是单态,则$h_3$是满态。
\item 如果$h_2$, $h_4$是同构,$h_1$是满态,$h_5$是单态,则$h_3$是同构。
\end{compactenum}
\end{lem}

\begin{proof}
最后一点不过是前面两点的推论。第二点是第一点的对偶命题,所以我们只要证明第一点即可。

取$[x]\in_m A_3$,并且假设$h_3(x)=0$. 由于$h_4f_3(x)=g_3h_3(x)=0$,所以$h_4(f_3(x))=0$. 由于$h_4$是单态,所以$f_3(x)=0$. 由正合列条件,存在$[y]\in_m A_2$使得$f_2(y)=[x]$. 由于$g_2(h_2(y))=g_2h_2(y)=h_3f_2(y)=h_3(x)=0$. 所以由下面那条正合列条件,我们有$[z]\in_m B_1$使得$g_1(z)=h_2(y)$. 由于$h_1$是满态,所以存在$[w]\in_m A_1$使得$h_1(w)=[z]$. 利用第一个方块的交换性,我们有
\[
	g_1(z)=g_1h_1(w)=h_2f_1(w),
\]
从$g_1(z)=h_2(y)$以及$h_2$是一个单态,所以$f_1(w)=[y]$. 最后,$[x]=f_2(y)=f_2f_1(w)=0$. 此即所证。
\end{proof}

五引理最常见的应用是短五引理,他将用在短正合列上面。

\begin{lem}[短五引理]\label{short-5-lem}
在五引理中,令$A_1=B_1=A_5=B_5=0$,那么$h_1=h_5=\id_0$是自然的同构,我们得到了如下交换图(略去了恒等$\id_0:0\to 0$当然还有显然的$f_1=g_1=f_4=g_4=0$)
\[
	\xymatrix{
	0\ar[r]&A_2\ar[r]^{f_2}\ar[d]^{h_2}&A_3\ar[r]^{f_3}\ar[d]^{h_3}&A_4\ar[r]\ar[d]^{h_4}&0\\
	0\ar[r]&B_2\ar[r]^{g_2}&B_3\ar[r]^{g_3}&B_4\ar[r]&0\\
	}
\]
于是,
\begin{compactenum}[~~~(1)]
\item 如果$h_2$, $h_4$是单态,则$h_3$是单态。
\item 如果$h_2$, $h_4$是满态,则$h_3$是满态。
\item 如果$h_2$, $h_4$是同构,则$h_3$是同构。
\end{compactenum}
\end{lem}

\begin{lem}[蛇引理]\label{snake-lemma}
考虑如下交换图
\[
	\xymatrix{
	0\ar[r]&\cdot \ar[r]^{f}\ar[d]^{u}&\cdot \ar[r]^{g}\ar[d]^{v}&\cdot \ar[r]\ar[d]^{w}&0\\
	0\ar[r]&\cdot \ar[r]^{f'}&\cdot \ar[r]^{g'}&\cdot \ar[r]&0\\
	}
\]
其中两行是正合列。则我们可以拓展为交换图
\[
	\xymatrix{
	&0\ar[d]&0\ar[d]&0\ar[d]&\\
	0\ar@{.>}[r]&\cdot\ar@{.>}[r]^a\ar[d]^{\ker u}&\cdot\ar@{.>}[r]^b\ar[d]^{\ker v}&\cdot\ar[d]^{\ker w}&\\
	0\ar[r]&\cdot \ar[r]^{f}\ar[d]^{u}&\cdot \ar[r]^{g}\ar[d]^{v}&\cdot \ar[r]\ar[d]^{w}&0\\
	0\ar[r]&\cdot \ar[r]^{f'}\ar[d]^{\coker u}&\cdot \ar[r]^{g'}\ar[d]^{\coker v}&\cdot \ar[r]\ar[d]^{\coker w}&0\\
	&\cdot \ar@{.>}[r]_c\ar[d] &\cdot \ar@{.>}[r]_d\ar[d] &\cdot \ar@{.>}[r]\ar[d]& 0\\
	&0&0&0&\\
	}
\]
其中每行每列都是正合列,此外,还存在态射$\delta$使得如下正合列成立
\[
	0\to \ker u \xrightarrow[~~~]{a}\ker v \xrightarrow[~~~]{b}\ker w \xrightarrow[~~~]{\delta}\coker u \xrightarrow[~~~]{c}\coker v \xrightarrow[~~~]{d}\coker w \to 0.
\]
\end{lem}

\begin{proof}
$a$, $b$, $c$, $d$的存在唯一性直接来自于$\ker$与$\coker$的泛性质,正和性也不难检验。最难的在于构造$\delta$. 考虑交换图
\[
	\xymatrix{
	0\ar[r]&\cdot \ar[r]^-{\ker p_X}\ar@{.>}[d]^p&X\times_Z Y \ar[r]^-{p_X}\ar[d]^{p_Y}&X\ar[d]^{\ker w}\ar[r]&0\\
	0\ar[r]&\cdot \ar[r]^{f}\ar[d]^{u}&Y \ar[r]^{g}\ar[d]^{v}&Z \ar[d]^{w}\ar[r]&0\\
	0\ar[r]&U \ar[r]^{f'}\ar[d]^{\coker u}&W \ar[r]^{g'}\ar[d]^{i_W}&\cdot \ar@{.>}[d]^q\ar[r]&0\\
	0\ar[r]&V \ar[r]_-{i_V}&V\cup_U W \ar[r]_-{\coker i_V}&\cdot\ar[r]&0
	}
\]
其中$X\times_Z Y$是$\ker w$和$g$的纤维积,而$V\cup_U W$是$f'$和$\coker u$的余纤维积。虚线的态射可以如下得到:由于$gp_Y \ker p_X=\ker(w)p_X\ker p_X=0$,所以由$\ker g$的泛性质,我们得到了分解$p_Y\ker p_X=\ker(g)p$. 由于$f$是一个单态,所以$f=\im f=\ker g$,于是$p_Y\ker p_X=fp$. 类似地,可以得到$q$.

同时,上述交换图每行都是正合列。实际上只要考虑第一行就行,最后一行靠对偶原理即可。由于$g$是满态,所以由Lemma \ref{mt},$p_X$是满态。而$\im(\ker p_X)=\ker p_X$直接来自于$\ker p_X$是一个单态。

现在考虑$i_Wvp_Y\ker(p_X)=i_V\coker(u)up=0$. 由$\coker(\ker p_X)=p_X$的泛性质,我们有唯一的态射$j:X\to V\cup_U W$使得分解$i_Wvp_Y=jp_X$成立。考虑复合$\coker(i_V)jp_X=\coker(i_V)i_Wvp_Y=qw\ker(w)p_X=0$. 由于$p_X$是满态,所以$\coker(i_V)j=0$. 再由$\ker(\coker i_V)=i_V$的泛性质,可以得到唯一的态射$\delta:X\to V$使得分解$j=i_V \delta$成立。

最后,正和性的检验是直接的,追图法将解决一切。
\end{proof}

利用蛇引理,短五引理是直接的。如果$u$和$w$是单态,则我们在正合列中有
\[
	0\to 0 \xrightarrow[~~~]{a}\ker v \xrightarrow[~~~]{b}0,
\]
所以$\ker v=0$,这就是说$v$也是单态。满态的命题也类似。

\section{链条件}

\para 设$P$是一个偏序集,如果他的任意非空子集都有极大元,则$P$称为满足极大条件。对偶地,还有极小条件。

极大条件的一个有趣应用是Noether归纳原理:设$\mathcal{P}$是关于$P$中元素的一个命题,如果对固定的$a$,任意的$\mathcal{P}(x)$对任意的$x>a$成立可以推出$\mathcal{P}(a)$成立,则$\mathcal{P}$对所有$x\in P$都成立。

证明是简单的,假设$\mathcal{P}$至少对某个元素不成立,则所有使得$\mathcal{P}$不成立的元素的集合非空,由极大条件,这个集合中存在极大元$a$使得$\mathcal{P}(a)$不成立。任取$x>a$,由于$a$的极大性,$\mathcal{P}(x)$都成立,再由条件,可知$\mathcal{P}(a)$成立,矛盾。

\para 设$P$是一个偏序集,他的全序子集被称为一个链。如果给定一条链$S$,存在一个元素$a$,使得所有$S$中所有满足$a\leq x\in S$都成立$x=a$,则该链被称为是(向上)稳定的,简写成a.c.c. 类似的,还有(向下)稳定的,简写成d.c.c.

\begin{pro}
极大条件等价于所有链都是a.c.c. 对偶地,极小条件等价于所有链都是d.c.c.
\end{pro}

\begin{proof}
	如果$P$满足极大条件,则任取一条链$S$,他都有极大元$a$,任意的$a\leq x\in S$都可以推出$x=a$. 反过来,如果$P$不满足极大条件,则存在$S\subset P$中没有极大元。则给定一个$x_0\in S$,可以找到一个$x_1\in S$满足$x_0<x_1$,再对$x_1$可以找到$x_2$使得$x_1<x_2$,如是进行下去,我们就找到了一条不稳定的链。
\end{proof}

可见极大条件和a.c.c.是等价的,在应用中极大条件往往比较方便,而在证明/证伪Noether性的时候,构造一个具体的升链就比较简单。

\para 一个$R$-模$M$,如果他的子模集满足极大条件,则他称为Noether模,如果满足极小条件,则他称为Artin模。环$R$看成$R$-模时,如果他是Noether (Artin)模,则他称为Noether (Artin)环。

Noether模远比Artin模应用广泛,比如我们后面会断言Artin环一定是一个Noether环,反之不然。Noether性是有限性条件之一,下面一个命题展现了这一点。

\begin{pro}
如果三个$R$-模$P$, $M$和$Q$有一个短正合列$0\to P\xrightarrow{\alpha} M\xrightarrow{\beta} Q\to 0$,则$M$是Noether (Artin)模当且仅当$P$和$Q$都是Noether (Artin)模。
\end{pro}

\begin{proof}
只证明Noether的情况,Artin的情况是类似的。首先假设$P$和$Q$是Noether模。

考虑$M$的一个子模$M_i$,我们有$Q$的子模$Q_i=\beta(M_i)$,$P$的子模$P_i=\alpha^{-1}(M_i)$,还有一个短正合列
\[
	0\to P_i \xrightarrow{\alpha_i} M_i \xrightarrow{\beta_i} Q_i \to 0,
\]
其中$\alpha_i$是$\alpha$限制在$P_i$上得到的同态,还有$\beta_i$是$\beta$限制在$M_i$上得到的同态。

现在考虑$M$的一个子模升链$\cdots\hookrightarrow M_i\hookrightarrow M_{i+1}\hookrightarrow M_{i+2}\hookrightarrow \cdots$,所以我们有正合列的升链
\[
	\xymatrix{
	0\ar[r]&P_i\ar[r]\ar@{^{(}->}[d]&M_i\ar[r]\ar@{^{(}->}[d]&Q_i\ar@{^{(}->}[d]\ar[r]&0\\
	0\ar[r]&P_{i+1}\ar[r]&M_{i+1}\ar[r]&Q_{i+1}\ar[r]&0
	}
\]
由于$P$和$Q$是Noether模,所以当$i$足够大有
\[
	\xymatrix{
	0\ar[r]&P_i\ar[r]\ar@{=}[d]&M_i\ar[r]\ar@{^{(}->}[d]&Q_i\ar@{=}[d]\ar[r]&0\\
	0\ar[r]&P_{i+1}\ar[r]&M_{i+1}\ar[r]&Q_{i+1}\ar[r]&0
	}
\]
由短五引理,我们可以得到中间这个含入是同构,所以$M_i=M_{i+1}$. 因此$M$满足a.c.c.,是一个Noether模。

反过来,假设$M$是一个Noether模。由于$P$的任意子模升链可以通过$\alpha$上升到$M$的一条子模升链,由于$M$是Noether模,则该升链稳定。所以对足够大的$i$,我们有$\alpha(P_i)=\alpha(P_{i+1})=\cdots$. 由于$\alpha$是单射,所以$P_i=P_{i+1}=\cdots$. 因此$P$是Noether模。

类似地,$Q$任意的任意子模升链可以由$\beta$的原像得到$M$的一条子模升链,由于$M$是Noether模,则该升链稳定。所以对足够大的$i$,我们有$\beta^{-1}(Q_i)=\beta^{-1}(Q_{i+1})=\cdots$. 作用$\beta$之后就得到了$Q_i=Q_{i+1}=\cdots$. 因此$Q$是Noether模。
\end{proof}

作为推论,设$N$是$M$的一个子模,那么有自然的短正合列$0\to N\hookrightarrow M\to M/N\to 0$,因此$M$是Noether (Artin)模当且仅当$N$和$M/N$是Noether (Artin)模。

设$M=\bigoplus_{i=1}^nM_i$,因为有短正合列
\[
	0\to M_n\to \bigoplus_{i=1}^nM_i\to \bigoplus_{i=1}^{n-1}M_i\to 0,
\]
如果每一个$M_i$都是Noether (Artin)模,由归纳法,可知$M=\bigoplus_{i=1}^nM_i$是Noether (Artin)模。

\begin{pro}
一个模是Noether模当且仅当它的任意子模是有限生成的。
\end{pro}

\begin{proof}
	令$N$是一个Noether模$M$的一个子模,而$\mathscr{B}$是所有$N$的有限生成子模的集合。因为$0\in \mathscr{B}$,所以$\mathscr{B}$非空,于是由极大条件,$\mathscr{B}$中存在极大元$P$. 如果$P\neq N$,那么必然存在$x\in N-P$,于是$P+\langle x\rangle$严格包含$P$同时也有有限生成的,矛盾。于是$N=P$是有限生成的。

	反过来,假设$M$的任意子模是有限生成的。令$M_1\subset M_2\subset \cdots$是一个升链,于是$N=\bigcup_{n=1}^\infty M_n$是$M$的子模继而是有限生成的,记作$x_1$, $\cdots$, $x_r$,记$x_i\in M_{n_i}$,选一个最大的$n$,则$x_i\in M_n$,于是$M_n=N$,因此链是稳定的。进而$M$是Noether模。
\end{proof}

\begin{thm}
Hilbert基定理:如果$R$是一个Noether环,则$R[x]$也是一个Noether环。\notprove
\end{thm}

作为推论,如果$R$是一个Noether环,而$S$是一个有限生成$R$-代数,则$S$也是一个Noether环。

\para 下面的观察来自于Noether,他与拓扑上的不可约性对应。设$M$是一个Noether模,而$N$是它的一个子模,如果$N$不能写成两个$M$的子模$P$和$Q$的交,其中$N$是$P$和$Q$的真子模,则称$N$是一个不可约子模。

\begin{pro}
\label{irrde}设$M$是一个Noether模,则每一个$M$的真子模都可以写成$M$的有限个不可约子模的交的形式,这被称为不可约分解。
\end{pro}

\begin{proof}
	假设命题不对,考虑所有不能写成有限个不可约子模的交的那些真子模构成的集合,它按包含关系构成一个偏序集。由于这个集合非空,而且$M$是一个Noether模,所以存在极大元$N$。由于$N$是可约的,所以可以写成$P\cap Q$的形式。由于$P$和$Q$都严格比$N$大,所以它们可以写成有限个不可约子模的交,进而$N$可以写成有限个不可约子模的交。矛盾。
\end{proof}

\section{有限生成模}

\para 一个$R$-模$M$是有限生成的当且仅当存在某个正整数$n$使得$M$是$R^n$的商模,或者用正合列写作
\[
	R^n \xrightarrow{\pi} M\to 0.
\]
考虑$M$的一组有限的生成元$\{m_i\,:\, 1\leq i\leq n\}$,他们生成了一个$R$模,满射通过将$\pi:1_i\mapsto m_i$来线性扩张得到,即定义
\[
	\pi\left(\sum_i r_i 1_i\right)=\sum_i r_i m_i.
\]
反过来,如果$M$可以写成某个$R^n$的商模,他自然被$\pi(1_i)$生成。

\para Noether (Artin)环上的有限生成模是Noether (Artin)模。可以将其归结到有限生成自由模的情况即可,因为Noether (Artin)模的商模或者子模都是Noether (Artin)模。而这是因为在上节我们看到了,有限个Noether (Artin)模直和后还是Noether (Artin)模。

\para 从自由模模去一些关系是构造一些模的常用手段。比如我们希望构造一个由所有满足$(a,b)=(b,a)$的$(a,b)$构成的一个模,那么最方便的方式就是从$R^2$出发,模去所有$(a,b)-(b,a)$生成的子模。后者被称为一个关系。

自由模模去一个关系得到一个模这个过程,抽象来说,就是存在正合列
\[
	R^I\to R^J \to M\to 0.
\]
这被称为模$M$的free presentation,如果$I$和$J$都是有限的,则$M$被称为finitely presented.

finitely presented模一定是有限生成模,但一般而言,有限生成模不是finitely presented,比如取一个非Noether环$R$以及它的一个非有限生成理想$I$使得$R/I$是一个有限生成模,那么有短正合列$0\to I\to R \to R/I\to 0$,此处由于$I$不是有限生成的,所以$R/I$就不是finitely presented.

但是到了Noether环上的模就不会如此,考虑任意的有限生成模$M$,他可以写成$R^n$的商模,即有短正合列$0\to N \hookrightarrow R^n \to M\to 0$,由于$N$是$R^n$的子模,而$R^n$是$R$上的有限生成模,所以$R^n$是Noether模,继而推知$N$是有限生成的,所以存在满同态$R^m\to N$,这样我们就得到了正合列$R^m\to R^n\to M\to 0$.

\para 设$\mathfrak{a}$是$R$的一个理想,而$M$是一个$R$-模,记$\mathfrak{a}M$是所有形如$am$,其中$a\in \mathfrak{a}$以及$m\in M$的元素生成的。

作为例子,设$N$是$M$的一个子模,这里来计算$\mathfrak{a}(M/N)$. $\mathfrak{a}(M/N)$由$\sum_{i=1}^n a_i\bar{m}_i$生成,其中$\bar{m}_i\in M/N$. 观察$\sum_{i=1}^n a_i m_i$,为得到$\sum_{i=1}^n a_i\bar{m}_i$,需要模去$N$,但是$N$未必是$\mathfrak{a}M$的子模,所以可以先把$\mathfrak{a}M$扩张为$\mathfrak{a}M+N$,再模去$N$,添进去的多余的$N$里面的元素也一道模掉了,这样就得到了想要的$\mathfrak{a}(M/N)=(\mathfrak{a}M+N)/N$. 或者,可以通过先将$N$缩小到$\mathfrak{a}M$的一个子模$N\cap \mathfrak{a}M$,于是$\mathfrak{a}(M/N)=\mathfrak{a}M/(N\cap \mathfrak{a}M)$.

\begin{thm}[Hamilton-Cayley定理]
设$M$是一个有限生成$R$-模,他能被$n$个元素生成. 再设$\mathfrak{a}$是$R$的一个理想,$\varphi:M\to M$是一个自同态。如果$\varphi(M)\subset \mathfrak{a}M$,则存在首一多项式
\[
	p(x)=x^n+p_1x^{n-1}+\cdots+p_n,
\]
其中$p_i\in I$,使得$p(\varphi)=\varphi^n+p_1\varphi^{n-1}+\cdots+p_n$是一个零映射,即任取$m\in M$,有$p(\varphi)(m)=0$. 一般写作$p(\varphi)=0$.
\end{thm}

\begin{proof}
	设$\{m_1,\cdots,m_n\}$是$M$的生成元。考虑$\varphi(m_i)\in M$,因为$M$是有限生成的,所以
	\[
	\varphi(m_i)=\sum_{j}r_{ij}m_j,
	\]
	其中$r_{ij}\in I$. 将其写作$\sum_{j}(\delta_{ij}\varphi-r_{ij})m_i=0$,其中$\delta_{ij}$当$i=j$的时候为$1$,其他时候为$0$.

	将$M$看作$R[x]$-模,其中$xm=\varphi(m)$,则$\sum_{j}(\delta_{ij}x-r_{ij})m_i=0$. 左乘$x\delta_{ij}-r_{
	ij}$的伴随矩阵,就得到了
	\[
	\det\left(x\delta_{ij}-r_{ij}\right)m_i=0,
	\]
	对任意的$m_i$都成立。所以也对任意的$m\in M$都成立,而$\det\left(x\delta_{ij}-r_{ij}\right)$就是我们需要的多项式。
\end{proof}

\para 作为Hamilton-Cayley定理的应用,我们可以断言,有限生成模的任意满自同态是同构。

\begin{proof}
	把$M$看出$R[x]$-模,其中$xm=\varphi(m)$,他是有限生成的,因为作为$R$-模是有限生成的。令$I=(x)$是$x$在$R[x]$中生成的主理想。由于$\varphi$是满射,则$IM=M$. 选则映射$\id_M=1:M\to M$,这显然是一个$R[x]$-模同态,由于$\id_M(M)\subset IM$,利用Hamilton-Cayley定理,存在一个多项式$p$,系数属于$I$使得
	\[
	p(1)m=\left(1^n+xf_1(x)1^{n-1}+\cdots+xf_n(x)\right)m=(1-xq(x))m=0,
	\]
	其中$q\in R[x]$. 或者写作,$1-q(\varphi)\varphi=0$. 因此$q(\varphi)$是$\varphi$的一个逆,进而$\varphi$是一个同构。
\end{proof}

\para 设$M$是一个模,一组元素$\{x_1$, $\cdots$, $x_n$, $\cdots\}$被称为线性无关的,如果任取其中有限个元素$x_{i_1}$, $\cdots$, $x_{i_k}\in M$,以及对任意的$r_{i_1}$, $\cdots$, $r_{i_k}\in R$,使得如果$r_{i_1}x_{i_1}+\cdots+r_{i_k}x_{i_k}=0$可以推出$r_{i_1}=\cdots=r_{i_k}=0$. 如果有一组线性无关的元素可以生成$M$,则$M$同构于那族元素生成的自由模,此时$M$称为被$\{x_1$, $\cdots$, $x_n$, $\cdots\}$自由生成的,这组元素被称为自由生成元,或者叫做一组基。

\para 将上述命题应用到自由模。设$M\cong R^n$,而$I=\{x_1$, $\cdots$, $x_n\}$自由生成了$M$,则$M$同构于$I=\{x_1$, $\cdots$, $x_n\}$生成的自由模$\bigoplus_{x\in I} R$.

\begin{proof}
	$\bigoplus_{x\in I} R$自然可以看成$R^n$,从$\bigoplus_{x\in I} R=R^n$到$M$有自然的满射$\alpha$,他将生成元映到生成元,由于$M\cong R^n$,所以有同构$\beta:M\to R^n$,因此就有了满同态$\alpha\beta:M\to M$,进而是同构。所以$\alpha =(\alpha\beta)\beta^{-1}$就是一个同构。
\end{proof}

利用上面这个命题,可以证明当$m\neq n$时$R^m\not\cong R^n$. 设$m<n$,反证即可,如果$R^m\cong R^n$,则在$R^m$中可以取$n$个自由生成元,但这$n$个自由生成元都由$R^m$的自由生成元生成,所以他们不是自由的,矛盾。

因此,一个有限生成自由模的基的元素个数总是一定的。

\begin{pro}
域$k$上的有限生成模$M$都是自由模。
\end{pro}

\begin{proof}
	归纳法。如果$M$由一个非零元素$x$生成,由于$\ann(x)=(0)$,所以$M=kx$是自由模。设由$n$个元素生成的$k$-模是自由模,此时考虑由$n+1$个元素生成的$k$-模。设这$n+1$个非零元素为$\{x_1$, $\cdots$, $x_{n+1}\}$,如果这组生成元是自由生成元,则没什么要证的,如果不是,则存在一组系数$a_1$, $\cdots$, $a_{n+1}$使得等式$a_1x_1+\cdots+a_{n+1}x_{n+1}=0$成立,系数中至少有一个不是零,不妨设为$a_{n+1}$,于是
	\[
	a_{n+1}x_{n+1}=-a_1x_1-\cdots-a_nx_n,
	\]
	两边除以$a_{n+1}$得出$x_{n+1}$可以由$\{x_1$, $\cdots$, $x_{n}\}$生成,于是$M$也可以由$\{x_1$, $\cdots$, $x_{n}\}$生成,由归纳假设,$M$是一个自由模。
\end{proof}

利用选择公理,实际上,域上的模(矢量空间)都是自由模,但是这里暂时用不到这点。

\begin{thm}[Nakayama引理]
设$M$是一个有限生成$R$-模,$\mathfrak{a}$是$R$的一个理想,则$M=\mathfrak{a}M$可以推出存在一个$a\in \mathfrak{a}$使得$am=m$对任意的$m\in M$都成立。
\end{thm}

\begin{proof}
	在Hamilton-Cayley定理中,取$\varphi=\id_M$,因此,存在$a_1$, $\cdots$, $a_n\in \mathfrak{a}$使得$x=1+a_1+\cdots+a_n=1-a$左乘在$M$上是一个零映射,于是$(1-a)m=0$或者$am=m$对任意的$m\in M$都成立。
\end{proof}

特别地,当$\mathfrak{a}$包含于$R$任意的极大理想中时,$1-a$不处于任意的极大理想中,所以必然是可逆,因此$(1-a)m=0$推出了$m=0$. 因此Nakayama引理在局部环(只有一个极大理想的环)上会特别有用,因为此时包含于每一个极大理想中就变成了一个平凡的条件。

作为推论,令$M$是一个有限生成$R$-模,而$N$是$M$的一个子模,而$\mathfrak{a}$是包含于每一个极大理想中的一个理想,那么$M=\mathfrak{a}M+N$可以推出$M=N$. 实际上,将Nakayama引理应用到$M/N$上,注意到$\mathfrak{a}(M/N)=(\mathfrak{a}M+N)/N$即可。

\begin{pro}
将上面的推论稍微推广一下。设$M$是一个有限生成$R$-模,而$\mathfrak{a}$还是包含于每一个极大理想中的一个理想,设$\varphi:M\to M/\mathfrak{a}M$是商同态。如果$M$中的一组元素$\{m_1$, $\cdots$, $m_n\}$使得$\{\varphi(m_1)$, $\cdots$, $\varphi(m_n)\}$生成$M/\mathfrak{a}M$,则$\{m_1$, $\cdots$, $m_n\}$生成$M$.
\end{pro}

证明只需要将上面的推论应用到$N=\sum_{i=1}^n Rm_i$即可。

\para 作为Nakayama引理的一个应用,我们谈谈余切空间。将在流形$M$上一点$p$的某个邻域$U$光滑的函数$f$记作一个偶对$(f,U)$,并在两个偶对上定义如下等价关系:$(f,U)\sim (g,V)$,如果存在一个开集$W\subset U\cap V$使得$f|_W=g|_V$. 这样,所有等价类构成了一个环$\mathcal{F}_p$,这是一个局部环,唯一的极大理想由那些在点$p$上为零的光滑函数构成,因为如果$f$不在点$p$为零,由连续性,在足够小的一个邻域上,$f$存在逆。将$\mathcal{F}_p$的极大理想记作$\mathfrak{m}$. 流形上点$p$处的余切空间$T_p^*M$定义为$\mathfrak{m}/\mathfrak{m}^2$,这是一个$\mathcal{F}_p/\mathfrak{m}$-矢量空间。而商同态$d:\mathfrak{m}\to \mathfrak{m}/\mathfrak{m}^2$正是微分算子,或者说从$0$-阶到$1$-阶的外微分算子。为方便起见,设流形为$\mathbb{R}^n$,点$p$是原点$0$.

$\mathfrak{m}$是一个清楚的有限生成$\mathcal{F}_p$-模,生成元为$\{x_1$, $\cdots$, $x_n\}$. 现在设一族元素$\{f_1$, $\cdots$, $f_n\}$,使得$\{df_1$, $\cdots$, $df_n\}$生成了作为$\mathcal{F}_p/\mathfrak{m}=\mathbb{R}$-模的余切空间$\mathfrak{m}/\mathfrak{m}^2$.

应用Nakayama引理,我们可以知道$\{f_1$, $\cdots$, $f_n\}$生成了$\mathfrak{m}$. 所以此时,$\{f_1$, $\cdots$, $f_n\}$表现得就像局部坐标函数一样。这是在说什么呢?为此,将$\{df_1$, $\cdots$, $df_n\}$按照$\{dx_1$, $\cdots$, $dx_n\}$展开,有
\[
	df_i=\sum_{j=1}^n\partial_jf_i dx_j,
\]
其中展开系数构成的矩阵$(\partial_jf_i)$就是所谓的Jacobian. $\{df_1$, $\cdots$, $df_n\}$生成$\mathfrak{m}/\mathfrak{m}^2$,此时当且仅当$(\partial_jf_i)$可逆。因此,Nakayama引理告诉我们,如果$(\partial_jf_i)$可逆,则$\{f_1$, $\cdots$, $f_n\}$生成了$\mathfrak{m}$. 于是,对每一个$x_i$,我们有
\[
	x_i=\sum_{j=1}^n a_{ij}f_j.
\]
于是,我们就得到了$\{f_1$, $\cdots$, $f_n\}$的“反函数”。这并不是真正的反函数,是因为系数$a_{ij}\in \mathcal{F}_p$还没写成$\{f_1$, $\cdots$, $f_n\}$的函数形式。

注意到这与反函数定理的相似性,反函数定理就是在说:如果在一点处如果一族函数线性化后(微分后)性质足够好(Jacobian可逆),则非微分的函数在那点附近性质也非常好(可逆,有反函数)。因此,Nakayama引理此时和反函数定理是相近的。

\section{标量扩张与局部化}

\para 设$M$是$R$-模,$S$是$R$-代数,那么$M_S=S\otimes_R M$就是一个$S$-模,通过$s(t\otimes a)=(st)\otimes a$,我们称这个$S$-模$M_S$为$M$经过标量扩充而来的。

\begin{lem}
有以下同构:

\no{1} $M_S\otimes_S N_S\cong (M\otimes_R N)_S$.

\no{2} 设$N$是一个$S$-模,有同构$\Hom_R(M,N)\cong \Hom_S(M_S,N)$
\end{lem}

\begin{proof} 第一个同构。实际上,$(s_1\otimes_R a,s_2\otimes_R b)\mapsto s_1s_2\otimes_R(a\otimes_R b)$是一个双线性映射,所以他唯一诱导了$\varphi:M_S\otimes_S N_S\to (M\otimes_R N)_S$. 反过来,映射$(s,a\otimes_R b)\mapsto (s\otimes_R a)\otimes_S (1\otimes_R b)=(1\otimes_R a)\otimes_S (s\otimes_R b)$也是一个双线性映射,所以他唯一诱导了$\psi:(M\otimes_R N)_S\to M_S\otimes_S N_S$.容易验证他们互逆。

第二个同构。设$\phi\in \Hom_R(M,N)$,我们可以通过$\Phi(s\otimes a)=s\phi(a)$定义$\Phi\in \Hom_S(M_S,N)$,反过来,已知$\Phi\in \Hom_S(M_S,N)$,我们可以通过$\phi(a)=\Phi(1\otimes a)$来定义$\phi\in \Hom_R(M,N)$.不难检验,这是一个同构。 \end{proof}

对有限个$\{M_i\}_{i\in I}$,那么反复利用\no{1}我们就得到了同构
\[
	\bigotimes_{i\in I} \left(S\otimes_RM_i\right)\cong S\otimes_R\bigotimes_{i\in I} M_i.
\]

\para 设$S$是一个$R$-代数,我们有下面三个小命题:

\no{1}设$V$是一个$R$-模,$W$是一个$S$-模,$f:V\to W$是$R$-模同态,则他可以唯一扩张为一个$S$-模同态$V_S\to W$.

\no{2}设$V$和$W$都是$R$-模,$f:V\to W$是$R$-模同态,则他可以唯一扩张为一个$S$-模同态$V_S\to W_S$.

\no{3}设$V$和$W$都是$R$-模,$f:V\times\cdots\times V\to W$是$R$-多线性映射,则他可以唯一扩张为一个$S$-多线性映射$V_S\times\cdots\times V_S\to W_S$.

\begin{proof} 
	因为$\Hom_R(V,W)\cong \Hom_S(V_S,W)$,所以\no{1}得证。对于\no{2},我们有$i:W\hookrightarrow W_S$,那么$i\circ f:V\to W_S$就满足\no{1}题设,由\no{1}自然得证。对于\no{3},因为多线性映射他可以唯一分解出一个$V^k\to W$,然后由\no{2},可以唯一扩张为$(V^k)_S\to W_S$,因为$(V^k)_S=V_S\otimes_S \cdots\otimes_S V_S$,引理得证。
\end{proof}

注意到,零映射扩张成的映射依然是零映射,这可以用来给出一些等式。比如在实Lie代数$V$中有Jacobi恒等式$B(u,v,w)=0$对任意的$u$, $v$, $w\in V$成立,其中$B$是一个多线性映射,所以经过复化后依然有$B_\cc (u,v,w)=0$对任意的$u$, $v$, $w\in V_\cc$成立。

\para 注意到$M$和$N$如果都是$R$-模,则$\Hom_R(M,N)$也可以看成一个$R$-模,通过定义$(rf)(m)=rf(m)$. 这样我们自然可以问$\Hom_R(M,N)$关于标量扩张的表现。

设$S$是一个$R$-代数,那么存在一个自然的$S$-模同态
\[
	\alpha:S\otimes\Hom_R(M,N)\to \Hom_S(S\otimes M,S\otimes N).
\]
由于$\alpha$是$S$-模同态,所以只要考察$\alpha(1\otimes \varphi)$即可,其中$\varphi\in \Hom_R(M,N)$. 所谓自然,就是说满足交换图
\[
\begin{xy}
	\xymatrix
	{
		S\otimes\Hom_R(M,N)\ar[rr]^{\alpha_S}\ar[d]&&\Hom_S(S\otimes M,S\otimes N)\ar[d]\\
		T\otimes\Hom_R(M,N)\ar[rr]^{\alpha_T}&&\Hom_T(T\otimes M,T\otimes N)
	}
\end{xy}
\]
如果$S$是一个平坦$R$-模且$M$具有有限表示,则$\alpha$还是一个同构。

\para 一个环$R$中的元素一般不是可逆的,那一个自然的问题就出现了,是否存在一个比这个环大一点的环$S$使得$R$中的某些元素(因为0自然不会是可逆的)在环$S$中是可逆的。一个经典的构造是从整数环$\mathbb{Z}$构造出有理数域$\mathbb{Q}$. 下面以整环的语境复述一遍。

设$R$是一个整环,而$R^{*}=R-\{0\}$是$R$中除去$0$得到的集合,考虑集合$R\times R^{*}$,即所有的二元组$(r,s)$的集合,其中$s\neq 0$. 其中$r$被称为分子,而$s$被称为分母。

分数里面很重要的是消去率,比如$8/4$应该等同于$2/1$. 于是上面构造的集合还太大,应该适当构造等价关系来完成上面的等同。实际上,这个等价关系就是:$(r,s)\sim (r',s')$当且仅当$rs'=r's$,即一个二元组消去同一因子或者乘以一个相同的元素都是等价的。此时,包含$(r,s)$的等价类记作$r/s$.

剩下的就是定义运算,
\[
	\frac{r}{s}+\frac{r'}{s'}=\frac{rs'+r's}{ss'},\quad \frac{r}{s}\frac{r'}{s'}=\frac{rr'}{ss'}.
\]
不难检验这个运算的定义这和代表元的选取无关。注意到整环的条件会用到乘法里面,因为如果不是整环,两个分母相乘可能得到零,而零并不能作为分母。这个环记作$F(R)$,乘法单位元是$1/1$,而零元是$0/1$,通过建立$r\in R$与$r/1\in F(R)$的等同,可以将$R$看成$F(R)$的子环。$F(R)$还是一个域,实际上,任取$r/s$,其中$r\neq 0$,则$s/r$是它的逆。称$F(R)$是整环$R$的分式域。

\para 这里我们来看另一种构造。设$R$是一个整环,记$R^*=R-\{0\}$,可以构造一个多项式环$R[R^*]$,对应于$r\in R^*$,$R[R^*]$中的元素我们记作$x_r$. 那么$F(R)$可以定义为
\[
	R[R^*]\big/\bigl\langle\{rx_r-1\,:\, r\in R^*\}\bigr\rangle.
\]
换而言之,$x_r$就是$r$的逆。在$F(R)$中的等号记作$\equiv$. 从构造,$rx_s$应该对应$r/s$.

下面我们来检验分子分母消去律,即计算$rx_{rs}$. 由于$s(rx_{rs})\equiv (sr)x_{rs}\equiv 1$,所以$rx_{rs}\equiv x_s$. 两边乘以$x_r$,就得到了$x_rx_s\equiv x_{rs}$. 这个环的加法就是
\[
	ax_{b}+cx_d\equiv a(dx_d)x_{b}+c(bx_{b})x_d\equiv (ad+bc)x_bx_d \equiv (ad+bc)x_{bd},
\]
乘法就是
\[
	(ax_{b})(cx_d)\equiv acx_bx_d\equiv acx_{bd}.
\]

\para 下面要推广分式域的概念,假设$R$是一个环,不一定是一个整环。考虑$S$是$R$的任意子集,我们可以类比定义一个环$S^{-1}R$或者$R[S^{-1}]$,称之为$R$关于$S$的分式环,
\[
	S^{-1}R=R\bigl[\{x_s\,:\, s\in S\}\bigr]\big/\bigl\langle\{sx_s-1\,:\, s\in S\}\bigr\rangle.
\]
这个环的加法就是
\[
	ax_{b}+cx_d\equiv a(dx_d)x_{b}+c(bx_{b})x_d\equiv (ad+bc)x_bx_d,
\]
乘法就是
\[
	(ax_{b})(cx_d)\equiv acx_bx_d.
\]
并且从
\[
	bd(ax_{b}-cx_d)\equiv ad-bc
\]
可知$ax_{b}\equiv cx_d$当且仅当$ad-bc\equiv 0$. 因此,如果$s$, $t$, $st\in S$,则分子分母消去律$sx_{st}\equiv x_t$顺道也是成立的。利用消去率,我们有$x_{st}\equiv sx_s x_{st}\equiv x_s x_t$.

观察上面构造的分式乘法。由于我们想要$ax_b$对应$a/b$,$cx_d$对应$c/d$,那么是否$(ax_{b})(cx_d)\equiv acx_bx_d$对应$(ac)/(bd)$?如果$bd\in S$,那么这是对的,因为$x_bx_d\equiv x_{bd}$. 但是一般而言这是不对的,所以对$S$我们希望是对乘法封闭的。此外,观察加法也是,只有在$bd\in S$的时候才有对应$a/b+c/d=(ad+bc)/(bd)$. 为了,将$T$看成一个$R$-代数,即建立环同态$r\mapsto r/1$,最好还需要假设$1\in S$.

如果$S$是$R$的一个子集,满足$1\in S$且$S$对乘法封闭,这样一个集合称之为乘性子集。乘性子集有两个特别重要的例子,对应着两个重要的分式环。

首先,设有非零$x\in R$,则$S=\{1$, $x$, $x^2$, $\cdots\}$是一个乘性子集,这时候,$S^{-1}R$记作$R_x$.

其次,设$\mathfrak{p}$是$R$的一个素理想,此时$S=R-\mathfrak{p}$是一个乘性子集,此时$S^{-1}R$记作$R_\mathfrak{p}$. 这类分式环被称为环$R$关于素理想$\mathfrak{p}$的局部化。

\begin{lem}
设$a\in R$,则$a\equiv 0$当且仅当存在$s \in S$使得$as=0$.
\end{lem}

\begin{proof} 
	如果存在$s \in S$使得$as=0$,则$a\equiv asx_s\equiv 0$. 反过来,假设$a\equiv 0$但$as\neq 0$对任意$s\in S$都成立。由于$a\equiv 0$,则$a\in \bigl\langle\{sx_s-1\,:\, s\in S\}\bigr\rangle$. 逐次比较,唯一可能的选择就是关于$x$的一次项的线性组合,即
	\[
	a=\sum_{i=1}^n a_i(s_ix_{s_i}-1),
	\]
	其中求和是有限的。所以$a_is_i=0$对任意的$i$都成立,且$\sum_i a_i=-a$. 将$\sum_i a_i=-a$两边乘以$s_1\cdots s_n$,得到$as_1\cdots s_n=0$,由于$S$对乘法封闭,所以$s_1\cdots s_n\in S$,然后利用$as\neq 0$的假设就得到了矛盾。
\end{proof}

\para 利用上面这个引理,可以给出$S^{-1}R$的另一个定义。它是在$R\times S$中引入等价关系得到的,这个等价关系写作:$(r,s)\sim (r',s')$当且仅当存在一个$t\in S$使得$t(rs'-r's)=0$. 将包含$(r,s)$的等价类记作$r/s$,加法和乘法定义的检验这里就略去了。一般教科书上,会直接使用这个定义而并不给出为什么这样定义等价关系,而这里给出了一个启发。

通过$r\mapsto r/1$,就给出了一个环同态$\varphi:R\to S^{-1}R$. 利用上面的引理,可以看到$R\to S^{-1}R$不一定是单同态,$a/1=0$并不在只在$a=0$的时候才成立,在存在$s\in S$使得$sa=0$的时候也成立。不管怎样,$S^{-1}R$是一个$R$-代数。设$M$是一个$R$-模,则$S^{-1}R\otimes_R M$是一个$S^{-1}R$-模,记作$S^{-1}M$. 这又是一个标量扩张的例子。

$S^{-1}M$也可以通过在$M\times S$上加上等价关系定义:$(m,s)\sim (m',s')$当且仅当存在一个$t\in S$使得$t(s'm-sm')=0$. 将包含$(m,s)$的等价类记作$m/s$,加法和乘法定义的检验这里同样略去了。两种定义的一致性这里也略去证明了。

\para 设$f:M\to N$是一个$R$-模同态,则$\id_{S^{-1}R}\otimes f:S^{-1}M\to S^{-1}N$是一个$S^{-1}R$-模同态,不妨将$\id_{S^{-1}R}\otimes f$也记作$S^{-1}f$. 如果把$S^{-1}f$用元素来表示,就是$S^{-1}f(m/s)=f(m)/s$. 结合这两点,$S^{-1}f$构成了$R$-模范畴到$S^{-1}R$-模范畴的一个函子。

下面我们要说明,这还是一个正合函子。设$M_1\xrightarrow{f} M_2\xrightarrow{g} M_3$是$R$-模的正合列,则
\[
	S^{-1}M_1\xrightarrow{S^{-1}f} S^{-1}M_2\xrightarrow{S^{-1}g} S^{-1}M_3
\]
是$S^{-1}R$-模的正合列。

实际上,由于$gf=0$,所以$S^{-1}(gf)=S^{-1}g\circ S^{-1}f=0$,于是$\im S^{-1}f\subset \ker S^{-1}g$. 反过来,任取$m/s\in \ker S^{-1}g$,则$g(m)/s=0$,于是存在一个$t\in S$使得$g(m)t=g(mt)=0$,所以$mt\in \ker g=\im f$,于是存在$n$使得$mt=f(n)$,于是$m/s=mt/(st)=f(n)/(st)=S^{-1}f(n/(st))\in \im S^{-1}f$.

\begin{pro}
$S^{-1}$或者说$S^{-1}R\otimes -$是正合函子,等价于说$S^{-1}R$是平坦$R$-模。
\end{pro}

\begin{pro}
设$S$是$R$的一个乘性子集,记同态$R\to S^{-1}R$为$i_S$. 再设$T$是一个环,如果$\varphi:R\to T$是一个环同态,且使得$\varphi(S)=\{\varphi(s)\,:\, s\in S\}$中的元素都是$T$中的可逆元,则存在唯一的环同态$\pi:S^{-1}R\to T$使得分解$\varphi:R\xrightarrow{i_S}S^{-1}R\xrightarrow{\pi} T$成立。
\end{pro}

换句话说,如果一个$R$-代数$T$中$S\cdot 1$都是可逆的,则$T$是一个$S^{-1}R$-代数。这被称为分式环的泛性质,这就指出分式环的实质在于将某个乘性子集搞成可逆的。

\begin{proof}
	令$\pi(r/s)=\varphi(r)\varphi(s)^{-1}$. 下面检验$\pi$是一个环同态,关键不在乘法上,而在于分式不同代表元的选取。令$r'/s'=r/s$,则存在一个$t\in S$使得$t(r's-rs')=0$,于是
	\[
		\pi(t)(\pi(r')\pi(s)-\pi(r)\pi(s'))=0,
	\]
	由于$\pi(t)$可逆,所以$\pi(r')\pi(s)=\pi(r)\pi(s')$或者$\pi(r)\pi(s)^{-1}=\pi(r')\pi(s')^{-1}$. 所以$\pi$是一个环同态。至于唯一性,由构造,$\pi(r/s)=\varphi(r)\varphi(s)^{-1}$,所以它由$\varphi$唯一决定。最后,$\pi i_S(r)=\pi(r/1)=\varphi(r)$,所以分解成立。
\end{proof}

\para 固定环$R$,以及设$S$是$R$的一个子集。任取环同态$\varphi:R\to T$,如果$S$是可逆元构成的集合,则$\varphi(S)$也是可逆元构成的集合,如果$S$是乘性子集,则$\varphi(S)$也是乘性子集。

已知,$R$-代数构成了一个范畴,它的对象是二元组$(T,\varphi_T)$. 设$S$是$R$的一个乘性子集,如果$\varphi_T(S)$中的元素都是可逆的,这样的代数构成了一个范畴,我们记作$\mathsf{Alg}_{(R,S)}$,它的态射就是普通的$R$-代数的同态,因为同态自然将可逆元变成了可逆元。由于$T$是一个$R$-代数,它本身就是一个$R$-模,给定乘性子集$S\subset R$,通过标量扩张,$S^{-1}T=S^{-1}R\otimes_R T$就得到了一个$S^{-1}R$-代数,实际上$S^{-1}T$作为环同构于$\varphi_T(S)^{-1}T$. 记$i_T=i_S\otimes 1:T\to S^{-1}T$是自然诱导的同态。

任取$(T,\varphi_T)\in \mathsf{Alg}_R$,则分式环将给出一个函子$S^{-1}:T\mapsto S^{-1}T$,对于态射$\Psi:(T,\varphi_T)\to (U,\varphi_U)$,即满足$\varphi_U=\Psi\varphi_T$的环同态$\Psi:T\to U$. 考虑复合同态$i_{U}\Psi:T\to S^{-1}U$,由于$i_{U}\Psi(\varphi_T(s))=i_{U}\varphi_U(s)=s/1\in S^{-1}U$是可逆元,所以由分式环$S^{-1}T=\varphi_T(S)^{-1}T$的泛性质,存在同态
\[
	S^{-1}(\Psi):S^{-1}T\to S^{-1}U.
\]
使得交换图成立
\[
\begin{xy}
	\xymatrix{
		T\ar[rr]^-{i_T} \ar[d]_-{\Psi}&&S^{-1}T \ar@{.>}[d]^-{S^{-1}(\Psi)}\\
		U\ar[rr]^-{i_U} &&S^{-1}U
	}
\end{xy}
\]
因此,$S^{-1}$确实是一个函子。

反过来,从范畴$\mathsf{Alg}_{(R,S)}$到$\mathsf{Alg}_R$有自然的遗忘函子$f$,他忘掉了$S\cdot 1$在相应的$R$-代数中是可逆的。

\begin{thm}
存在双函子同构:
\[
	F(-,\star):\mathsf{Alg}_R\bigl(-,f(\star)\bigr)\to \mathsf{Alg}_{(R,S)}\bigl(S^{-1}(-),\star\bigr).
\]
\end{thm}

因此$S^{-1}$是遗忘函子的左伴随函子,继而是一个自由函子。这也可以看成$S^{-1}R$的另一种定义,只要将$f$的左伴随函子作用在$R$上就得到了$S^{-1}R$.

\begin{proof}
	首先构造同构,任取同态$R$-代数同态$\Psi:X\to f(Y)$,其中$Y$使得$\varphi_Y(S)$在$Y$中是可逆的。由于是$R$-代数同态,所以$\varphi_Y=\Psi\varphi_X$. 任取$s\in S$,有$\varphi_Y(s)=\Psi\varphi_X(s)$可逆,通过分式环的泛性质,有唯一的同态$F(X,Y)(\Psi):\varphi_X(S)^{-1}X=S^{-1}X\to Y$. 于是$F(X,Y)$就是我们构造的同构。由构造,这是一个单射,反过来,任取$\rho:\mathsf{Alg}_{(R,S)}(S^{-1}(X),Y)$,我们有$F(X,Y)(i_X\circ \rho)=\rho$,因此这是一个满射。

	最后要验证函子性同构,分别对$X$和$Y$检验自然变换即可。这点过于琐碎,这里就略去了。
\end{proof}

任取$X\in \mathsf{Alg}_{(R,S)}$,它可以自然地看成一个$S^{-1}R$-代数,这由分式环的泛性质保证,反过来,任意的$S^{-1}R$-代数都可以看成$\mathsf{Alg}_{(R,S)}$里面的元素,所以有理由认为$\mathsf{Alg}_{S^{-1}R}$和$\mathsf{Alg}_{(R,S)}$作为范畴等价。事实上,可以定义一个函子$\mathsf{Alg}_{(R,S)}\to \mathsf{Alg}_{S^{-1}R}$,对对象,将任意的$R$-代数$X$看作$S^{-1}R$-代数$X$,对态射,任取$\varphi\in \mathsf{Alg}_{(R,S)}(X,Y)$,它也可以看作$S^{-1}R$-代数同态(请检查)。在上面的等同下,由泛性质,$\mathsf{Alg}_{(R,S)}(X,Y)=\mathsf{Alg}_{S^{-1}R}(X,Y)$,所以这是一个完全且忠实的函子。同样,任取$S^{-1}R$-代数$X$,他也可以看成$\mathsf{Alg}_{(R,S)}$里面的对象。所以由Proposition \eqref{equivcat},$\mathsf{Alg}_{S^{-1}R}$和$\mathsf{Alg}_{(R,S)}$是等价范畴。

也正是如此,因此上面的双函子同构可以写成
\[
	F(-,\star):\mathsf{Alg}_R\bigl(-,f(\star)\bigr)\to \mathsf{Alg}_{S^{-1}R}\bigl(S^{-1}(-),\star\bigr).
\]
但是,范畴$\mathsf{Alg}_{(R,S)}$并不需要$S^{-1}R$的构造,所以我们才谈论的是这个范畴,然后才可以通过自由函子给出$S^{-1}R$的一个定义。当然,这里已经可以事后诸葛亮地说,函子$S^{-1}$是与$S^{-1}R$-代数到$R$-代数的遗忘函子左伴随的自由函子。

\para 上面讨论了分式环几个定义。下面讨论分式环的几个重要实例。从一个事实开始,设$S\subset R$是一个乘性子集,而$S'$是$S$中元素的全部因子构成的乘性子集,则$S^{-1}R=S'^{-1}R$.

\para 设$M$是一个$R$-模,$S$是$R$的一个乘性子集,记$\varphi:M\to S^{-1}M$是局部化自然的同态,满足$\varphi(m)=m/1$. 设$N\subset M$是一个子模,则$S^{-1}N$是$S^{-1}M$的一个子模。由于子模的原像是子模,所以如果$L$是$S^{-1}M$的子模,则$\varphi^{-1}(L)$是$M$的子模。如果$N$是$M$的一个子模,则$N\subset \varphi^{-1}(S^{-1}N)$. 特别地,如果$N=\varphi^{-1}(L)$,则$L=S^{-1}N$.

这些关系非常类似Galois联络,只不过这里都不是逆序的而是保持顺序的。

\begin{proof}
	除去最后一点都是简单的。令$N$是$M$的一个子模,任取$a\in N$,我们有$a=\varphi^{-1}(a/1)$,所以$N\subset \varphi^{-1}(S^{-1}N)$. 现在设$N=\varphi^{-1}(L)$,于是前面的包含关系给出$\varphi^{-1}(L)\subset \varphi^{-1}(S^{-1}N)$或者$L\subset S^{-1}N$. 反过来,令$a\in N$以及$s\in S$,则$a/s=(a/1)(1/s)=\varphi(a)/s$,因为$\varphi(a)\in L$,而$L$是一个$S^{-1}R$-模,所以$a/s\in L$,这就给出了$S^{-1}N=L$.
\end{proof}

利用这个结论,我们可以发现,$S^{-1}M$的每一个子模$L$,实际上都可以写成$M$中的子模$\varphi^{-1}(L)$的分式模$L=S^{-1}(\varphi^{-1}(L))$. 

\para 反过来,如果$M$的子模$N$满足$\ann(M/N)\cap S=\varnothing$,或者说,任取$s\in S$以及$m\in M-N$,都有$sm\not\in N$,则$N$可以写成$\varphi^{-1}(L)$的形式,其中$L=S^{-1}(\varphi^{-1}(L))=S^{-1}N$. 

$\ann(M/N)\cap S=\varnothing$这个条件可以写成:任取$s\in S$以及$m\in M-N$,都有$sm\not\in N$. 或者写成逆否的形式,设$s\in S$以及$m\in M$,如果$sm\in N$,则$m\in N$. 应用这个条件的理由是,在一般的模里面,$a/1\in S^{-1}N$并不能推出$a\in N$,因为如果存在一个$s\in S$使得$sa\in N$,就可以推出$a/1=sa/s\in S^{-1}N$. 而这又是因为$M-N$一般不是一个模。

\begin{proof}
	包含关系$N\subset \varphi^{-1}(S^{-1}N)$已经是清楚的,任取$a\in \varphi^{-1}(S^{-1}N)$或者说$\varphi(a)=a/1\in S^{-1}N$. 于是存在$s\in S$使得$(sa)/s$的分子$sa\in N$,继而根据条件$\ann(M/N)\cap S=\varnothing$可以推知$a\in N$. 
\end{proof}

\begin{pro}
结合上面的结论,这里做出断言:$S^{-1}M$的子模一一对应于$M$中满足$\ann(M/N)\cap S=\varnothing$的子模$N$,通过$N\mapsto S^{-1}N$以及$L\mapsto \varphi^{-1}(L)$.
\end{pro}

应用到环$R$的理想上面,$S^{-1}R$的理想一一对应着$R$中那些满足$\ann(R/\mathfrak{a})\cap S=\varnothing$的理想$\mathfrak{a}$. 由于$\ann(R/\mathfrak{a})=\mathfrak{a}$,所以$S^{-1}R$的理想一一对应着$R$中与$S$不相交的理想。由于素理想的原像是素理想,所以$S^{-1}R$中的素理想一一对应着$R$中与$S$不相交的素理想。

\begin{pro}
如果$M$是一个Noether或者Artin模,$S$是一个$R$的乘性子集,则$S^{-1}M$是一个Noether或者Artin模。
\end{pro}

\begin{proof}
	只要证明$S^{-1}M$中的非空子模族$\mathcal{F}$有极大/极小元即可。利用一一对应,$\varphi^{-1}(\mathcal{F})$是$M$的非空子模族,所以存在一个极大/极小元$N\in \varphi^{-1}(\mathcal{F})$,继而对应回$S^{-1}M$中就得到了$\mathcal{F}$中的极大/极小元。
\end{proof}

\begin{pro}
设$f:M\to N$是一个$R$-模同态,则$f$是单同态(满同态)的充分必要条件是对每一个素理想$\mathfrak{p}$,$f_{\mathfrak{p}}=(R-\mathfrak{p})^{-1}f$是单同态(满同态).
\end{pro}

\begin{proof}
	如果$f$是单的,则$\ker(f)$
\end{proof}

\section{有限长模}

\para 设$M$是一个$R$-模,如果$M$没有非平凡子模,则称$M$是一个单模。任取非零元$x\in M$,考虑子模$Rx\subset M$,如果$M$是单模,我们只有两个可能,$Rx=\{0\}$或者$Rx=M$. 但是由于$x=1x\in Rx$非零,所以$Rx=M$. 于是任意的一个单模都是由一个元素生成的。反过来,单元素生成的模是不是单模?答案是不一定的。

设$Rx$是一个$R$-单模,考虑自然的$R$-模同态$f:r\mapsto rx$,于是$f^{-1}(0)=\ann(x)$,由同构基本定理,$Rx\cong R/\ann(x)$. 由于$Rx$是单模,所以$Rx$没有非平凡子模,再由同构,$R/\ann(x)$没有非平凡理想,因此$R/\ann(x)$是一个域,于是$\ann(x)$是极大理想。反过来也对,如果$\ann(x)$是极大理想,则$Rx$是单模。

考虑$R=k$是一个域,设$x\in M$是一个非零元,此时$\ann(x)$作为$k$的理想只可能是$(0)$,否则$x=1x=0$. 域中唯一的极大理想就是$(0)$,于是$kx$一定是一个单模。

\para 设$M$是一个$R$-模,如果它同时是Artin模和Noether模,那他就被称为一个有限长模。对于严格递增链$0=M_0\subset M_1\subset \cdots\subset M_n=M$,$n$被称为这条链的长度。有限长模的任意链都是有限的,因为向上有Noether条件,向下有Artin条件。设有一条严格递增链$M_0\subset M_1\subset \cdots\subset M_n \subset \cdots $,任取$i\in \mathbb{Z}^+$,如果$M_{i}/M_{i-1}$都是单模,则这样一条链被称为$M$的一条合成列。有限长模的合成列一定是有限长的。同时,合成列当然是不一定唯一的。

给定任意一个$R$-模的包含$N\subset M$,考虑任意的中间模$N'$,即满足$N\subset N'\subset M$的$R$-模$N'$,则他在商模$M/N$中的像$N'/N$是$M/N$的一个子模。反过来,$M/N$的子模在商映射下的原像就是一个中间模。因此如果$N\subset M$中间不能再插入任意的一个子模等价于$M/N$中没有非平凡子模,即$M/N$是一个单模。于是合成列的定义就是说这是一个不能再插入一些子模让它变得更长的链,换而言之,极大链。

\begin{pro}
有限长模的合成列的长度是一定的,因此可以定义有限长模的长度为其中任意一条合成列的长度。如果一个模不是有限长,则定义它的长度为无穷。此外,如果一条严格递增链不是合成列,则可以再其中插入子模得到合成列。
\end{pro}

以长度的视角,单模等价于说模的长度为1.

\begin{proof}
	记$l(M)$是$M$所有的合成列中最短那条的长度。设$N$是$M$的一个真子模,我们先证明$l(N)<l(M)$.

	给定一个合成列$0=M_0\subset M_1\subset \cdots\subset M_n=M$,记$N$的子模$N\cap M_i$为$N_i$,进而有$N$的子模链$0=N_0\subset N_1\subset \cdots\subset N_n=N$. 由同构\eqref{modiso1}
	\[
	N_i/N_{i-1}=(N\cap M_i)/(N\cap M_{i-1})\cong (M_{i-1}+N\cap M_i)/M_{i-1},
	\]
	显然,$M_{i-1}+N\cap M_i$是$M_i$的一个子模,由于$M_i/M_{i-1}$是一个单模,于是$M_{i-1}+N\cap M_i=M_{i-1}$或者$M_i$,前者意味着$N\cap M_i\subset M_{i-1}$或者$N_i/N_{i-1}=\{0\}$再或者$N_i=N_{i-1}$,后者意味着$N_i/N_{i-1}\cong M_i/M_{i-1}$是一个单模。在$0=N_0\subset N_1\subset \cdots\subset N_n=N$中去掉那些相等的项,我们就得到了$N$的一条合成列,长度小于等于$l(M)$. 由于$l(N)$是$N$所有的合成列中最短那条的长度,所以$l(N)\leq l(M)$. 然后设$l(N)=l(M)$,这意味着对每一个$i$都成立$M_{i-1}+N_i=M_{i-1}+N\cap M_i=M_i$. 对$i=1$,可以推知$N_1=M_1$,然后对$i=2$有$M_2=M_1+N_2=N_1+N_2=N_2$,经过有限归纳,可以得知$N=N_n=M_n=M$,这和$N$是$M$的一个真子模矛盾。于是$l(N)<l(M)$.

	然后任取一条完全递增链$0=M_0\subset M_1\subset \cdots\subset M_k=M$,我们有
	\[
	l(0)<l(M_1)<\cdots<l(M_k)=l(M),
	\]
	由于$l(0)=0$,所以$l(M)\geq k$. 最后考虑任意一条合成列,我们有$l(M)$大于等于它的长度,又$l(M)$是$M$所有的合成列中最短那条的长度,因此$M$的任意合成列的长度都是$l(M)$.

	最后,考虑$M$的任意的严格递增链,如果他的长度等于$l(M)$,则这是一条合成列。否则,在其中插入子模,直到长度等于$l(M)$,我们就得到了一条合成列。
\end{proof}

\begin{pro}
一个$R$-模$M$是有限长模当且仅当它存在一个有限长合成列。
\end{pro}

\begin{proof}
	$(\Rightarrow)$部分是自然的。反过来,如果$M$不是一个有限长模,则它或者Artin模或者Noether模,于是存在向上不稳定的严格递增链或者向下不稳定的严格递减链,那一个都会使得$M$的合成列长度不有限,于是逆否得证。
\end{proof}

\begin{pro}
设有短正合列$0\to U\to V\to W\to 0$,则$V$是有限长的,当且仅当$U$和$W$是有限长的,且$l(V)=l(U)+l(W)$.
\end{pro}

\begin{proof}
	第一点来自于定义。不妨假设$U$是$V$的一个子模,而$W=V/U$. 实际上,因为同构的模有着相同的子模结构,所以同构的模的长度相同。考虑严格递增链$0\subset U\subset V$,然后由于$V$是有限长的,我们可以将其扩张为一条合成列,写作
	\[
	0\subset U_1\subset \cdots \subset U_k=U\subset V_{k+1}\subset \cdots \subset V_n=V,
	\]
	于是$0\subset U_1\subset \cdots \subset U_k=U$是$U$的一条合成列,以及
	\[
	0\subset V_{k+1}/U\subset \cdots \subset V_n/U=V/U
	\]
	是$V/U$的一条合成列,因为$(V_{i+1}/U)/(V_{i}/U)\cong V_{i+1}/V_i$. 所以等式$n=k+(n-k)$就给出了$l(V)=l(U)+l(V/U)$.
\end{proof}

设$f:M\to N$是有限长模之间的一个满同态,我们有短正合列$0\to \ker(f)\to M\to N\to 0$,于是上面的命题给出了$l(M)=l(\ker f)+l(N)$. 如果$f:M\to N$不是满同态,那么可以构造出$f:M\to \im(f)$,这是一个满同态,于是$l(M)=l(\ker f)+l(\im f)$. 

\para 域上的有限长模是最常用的一个有限长模。设$k$是一个域,定义$k$-矢量空间$V$的维度$\dim_k(V)$为$V$的长度。维度是线性代数理论中一个很重要的不变量,以后我们会看到维度的其他等价定义。

设$M$是域$k$上的有限生成模,它是自由模,不妨直接写作$k^n$,一组自由生成元写作$\{1_1$, $\cdots$, $1_n\}$,那么对小于$n$的正整数$m$,$\{1_1$, $\cdots$, $1_m\}$自由生成了它的一个子模$k^m$. 于是我们有链
\[
	0\subset k^1\subset  k^2 \subset \cdots \subset k^n,
\]
其中$k^m/k^{m-1}\cong k$,这是一个合成列。因此$\dim_k(k^n)=n$. 

由于自由模的自由生成元的个数与生成元的选取无关,所以有限维矢量空间的维度也就是自由生成元的个数。

如果$k$-模$M$不是有限生成的,则$\dim_k(M)=\infty$. 实际上,给定一个$x_1\in M$,他生成的子模$\langle x_1\rangle=kx_1\cong k^1$,然后再从$M-\langle x_1\rangle$选出一个非零元$x_2$,得到$\langle x_1,x_2\rangle\cong k^2$,接着再从$M-\langle x_1,x_2\rangle$选出一个非零元$x_3$,如是进行下去,我们就得到了一个严格递增升链
\[
	0\subset k^1\subset k^2\subset \cdots \subset k^n\subset \cdots,
\]
按构造$k^m/k^{m-1}\cong k$,于是这是一个无限长合成列。

所以,一个$k$-矢量空间是有限生成的当且仅当他是有限维的。

\begin{pro}[线性代数基本定理]
设$f:V\to W$是两个有限维$k$-矢量空间的同态,由于有限维$k$-矢量空间都是有限长$k$-模,则$l(M)=l(\ker f)+l(\im f)$给出维度的等式:
\[
	\dim_k(V)=\dim_k(\ker f)+\dim_k(\im f).
\]
\end{pro}

\begin{pro}
如果$k$-矢量空间$V$是Artin或者Noether模,则$V$是有限维的。即在矢量空间上,Artin性等价于Noether性等价于有限维。
\end{pro}

\begin{proof}
	假设$V$不是有限维的,则我们可以构造出一条合成列,
	\[
	0\subset \langle x_1\rangle\subset \langle x_1,x_2\rangle\subset \cdots \subset \langle x_1,\,\cdots\!,x_n\rangle\subset \cdots,
	\]
	因此这自然不是Noether模。然后考察
	\[
		\bigoplus_{i=1}^\infty kx_i \supset \bigoplus_{i=2}^\infty kx_i \supset \bigoplus_{i=3}^\infty kx_i\supset \cdots,
	\]
	因此这也不是Artin模。逆否即得证。
\end{proof}

\begin{thm}
设$M$是一个有限长$R$-模,长度为$n$. 对于$M$的任意合成列$0=M_0\subset M_1\subset \cdots\subset M_n=M$,$M_i/M_{i-1}\cong R/\mathfrak{m}_i$都是域,$\mathfrak{m}_i$都是极大理想,但集合$I=\{\mathfrak{m}_i\,:\,1\leq i\leq n\}$与选取的合成列无关,并且有同构
\[
	M\cong \bigoplus_{\mathfrak{m}\in I} M_\mathfrak{m}.
\]
\end{thm}

\begin{proof}
	我们局部化处理,因为如果在每一个极大理想局部同构,则整体也同构。任取$R$的一个极大理想$\mathfrak{n}$. 先考虑长度为1的情况,此时$M$是一个单模,于是$M\cong R/\mathfrak{m}$,其中$\mathfrak{m}$是一个极大理想,不妨就直接写成$M=R/\mathfrak{m}$. 如果$\mathfrak{n}=\mathfrak{m}$,此时$\mathfrak{n}$外的元素在$R/\mathfrak{m}=R/\mathfrak{n}$中都是可逆元,于是$M_{\mathfrak{n}}=(R/\mathfrak{m})_{\mathfrak{n}}=R/\mathfrak{m}=M$. 

	如果$\mathfrak{n}\neq \mathfrak{m}$,因此$\mathfrak{m}$是极大理想,所以$\mathfrak{m}\not\subset \mathfrak{n}$以及$\mathfrak{n}\not\subset \mathfrak{m}$,因此$\mathfrak{m}_\mathfrak{n}=R_\mathfrak{n}$. 于是$M_\mathfrak{n}=(R/\mathfrak{m})_\mathfrak{n}=R_\mathfrak{n}/\mathfrak{m}_\mathfrak{n}=0$. 由$(M_\mathfrak{m})_\mathfrak{n}=R_\mathfrak{n}\otimes R_\mathfrak{m} \otimes M$,其中张量积都是$R$-模的张量积. 于是有同构$(M_\mathfrak{m})_\mathfrak{n}\cong R_\mathfrak{n}\otimes M\otimes R_\mathfrak{m}$,任取一个$a\in \mathfrak{m}$但$a\not\in \mathfrak{n}$,于是任取$(r/p)\otimes m \in R_\mathfrak{n}\otimes M$有
	\[
		\frac{r}{p}\otimes a=\frac{ra}{pa}\otimes m =\frac{r}{pa}\otimes (am),
	\]
	因为$a\in \mathfrak{m}$,但$m\in M=R/\mathfrak{m}$,所以$am=0$. 所以$R_\mathfrak{n}\otimes M=0$也就推出了$(M_\mathfrak{m})_\mathfrak{n}=0=M_\mathfrak{n}$. 由于已经遍历了极大理想,都有$(M_\mathfrak{m})_\mathfrak{n}=M_\mathfrak{n}$,于是$M_\mathfrak{m}=M$.

	现在考虑一般的情况,设$0=M_0\subset M_1\subset \cdots\subset M_n=M$是一个合成列。对任意一个极大理想$\mathfrak{m}$进行局部化,我们有
	\[
	0=(M_0)_\mathfrak{m}\subset (M_1)_\mathfrak{m}\subset \cdots\subset (M_n)_\mathfrak{m}=M_\mathfrak{m}.
	\]
	由于$M_i/M_{i-1}$是单模,长度为1,所以当$\mathfrak{m}=\ann(M_i/M_{i-1})$时,$(M_i/M_{i-1})_\mathfrak{m}\cong M_i/M_{i-1}\cong R/\mathfrak{m}$,否则$(M_i/M_{i-1})_{\mathfrak{m}}=(M_i)_{\mathfrak{m}}/(M_{i-1})_{\mathfrak{m}}=0$给出了$(M_i)_{\mathfrak{m}}=(M_{i-1})_{\mathfrak{m}}$. 如果$\mathfrak{m}$在$I$中不出现,则上面的推理给出$M_\mathfrak{m}=(M_n)_\mathfrak{m}=(M_0)_\mathfrak{m}=0$. 反过来,只有当$\mathfrak{m}$在$I$中出现,在链中将重复项删去就得到了$M_\mathfrak{m}$的一个合成列
	\[
	0=(M_0)_\mathfrak{m}\subset (M_{i_1})_\mathfrak{m}\subset \cdots\subset (M_{i_k})_\mathfrak{m}=M_\mathfrak{m}.
	\]
	任取单模$(M_{i_k})_\mathfrak{m}/(M_{i_{k-1}})_\mathfrak{m}$都有
	\[
	(M_{i_k})_\mathfrak{m}/(M_{i_{k-1}})_\mathfrak{m}=(M_{i_k}/M_{i_{k-1}})_\mathfrak{m}=M_{i_k}/M_{i_{k-1}}\cong R/\mathfrak{m}.
	\]

	取不同于$\mathfrak{m}$的另一个极大理想$\mathfrak{n}$,然后再局部化,可以得到另一条链
	\[
	0=((M_0)_\mathfrak{m})_\mathfrak{n}\subset ((M_{i_1})_\mathfrak{m})_\mathfrak{n}\subset \cdots\subset ((M_{i_k})_\mathfrak{m})_\mathfrak{n}=(M_\mathfrak{m})_\mathfrak{n},
	\]
	由于$((M_{i_k})_\mathfrak{m})_\mathfrak{n}/((M_{i_{k-1}})_\mathfrak{m})_\mathfrak{n}=((M_{i_k})_\mathfrak{m}/(M_{i_{k-1}})_\mathfrak{m})_{\mathfrak{n}}\cong R/\mathfrak{m}=((M_{i_k}/M_{i_{k-1}})_\mathfrak{m})_\mathfrak{n}=((R/\mathfrak{m})_{\mathfrak{m}})_\mathfrak{n}=0$,所以推出了$(M_\mathfrak{m})_\mathfrak{n}=0$. 正如长度为1的情况一般。

	现在来看命题,有限长模$M$的一个极大理想$\mathfrak{m}$是否属于$I$,只取决于$M_\mathfrak{m}$是否为零,而与合成列的选取无关。现在只剩下那个同构了,设$i_\mathfrak{m}:M\to M_{\mathfrak{m}}$是局部化诱导的同态,进而定义$\alpha=\sum_{\mathfrak{m}\in I}i_{\mathfrak{m}}:M\to \bigoplus_{\mathfrak{m}\in I} M_{\mathfrak{m}}$. 这检验这是一个同构,只要对每一个极大理想$\mathfrak{n}$进行局部化即可,而这由已知的,对不同的两个极大理想$\mathfrak{m}$和$\mathfrak{n}$有$(M_\mathfrak{m})_\mathfrak{n}=0$,以及如果$\mathfrak{m}\not\in I$则$M_\mathfrak{m}=0$这两点是清楚的。
\end{proof}

\begin{pro}
设$M$是一个有限长模,而$\mathfrak{m}$是一个极大理想,则$M=M_{\mathfrak{m}}$当且仅当存在一个正整数$n$使得$\mathfrak{m}^nM=0$.
\end{pro}

\begin{proof}
	任取另一个极大理想$\mathfrak{n}$,那么$\mathfrak{m}$存在一个不属于$\mathfrak{n}$中的元素$a$,这个元素在$M_\mathfrak{n}$中是可逆的,但是$a^n\in \mathfrak{m}^n$给出$a^nM_\mathfrak{n}=0$,由可逆性就得到了$M_\mathfrak{n}=0$. 所以只有可能$M_\mathfrak{m}\neq 0$,由上面的命题,$M\cong M_\mathfrak{m}$.

	反过来,由于$M\cong M_\mathfrak{m}$,所以对于任意的合成列$0=M_0\subset M_1\subset \cdots\subset M_n=M$都有$M_i/M_{i-1}\cong R/\mathfrak{m}$或者$\ann(M_i/M_{i-1})=\mathfrak{m}$再或者$\mathfrak{m}M_{i}\subset M_{i-1}$,于是
	\[
	\mathfrak{m}^nM=\mathfrak{m}^nM_{n}\subset \mathfrak{m}^{n-1}M_{n-1}\subset \cdots \subset \mathfrak{m}M_1\subset M_0=0,
	\]
	因此$\mathfrak{m}^nM=0$.
\end{proof}

\para 由于$R$本身就是一个$R$-模,如果$R$是有限长$R$-模,已经知道,在$R$的任何合成列中出现的极大理想不依赖于合成列的选取,个数不大于$R$的长度。记这些极大理想的集合为$I$,作为$R$-模成立同构$R\cong\bigoplus_{\mathfrak{m}\in I}R_\mathfrak{m}$.

\begin{lem}
回忆\eqref{oka}里面的定义,这里再指出一个理想族是Oka理想族,这个理想族由$R$中所有使得$R/\mathfrak{a}$是有限长$R$-模的理想$\mathfrak{a}$组成。
\end{lem}

\begin{proof} 
	给定$\mathfrak{a}$和$a\in R$,如果$R/(\mathfrak{a}+(a))$和$R/(\mathfrak{a}:a)$都是有限长$R$-模,要证明$R/\mathfrak{a}$是有限长$R$-模。可以假设$a\not\in \mathfrak{a}$,否则$R/(\mathfrak{a}+(a))=R/\mathfrak{a}$. 首先注意到短正合列
	\[
		0\to R/(\mathfrak{a}:a)\xrightarrow{\times a} R/\mathfrak{a} \to R/(\mathfrak{a}+(a))\to 0,
	\]
	其中第二个箭头是乘以$a$,是单射,因为如果$(r+(\mathfrak{a}:a))a=ar+\mathfrak{a}\in R/\mathfrak{a}$为零,则$r\in (\mathfrak{a}:a)$,第三个箭头是商环的泛性质诱导的,显然是满射。

	检验正和性最好回到$R$中,由于商同态$R\to R/(\mathfrak{a}:a)$是一个满射,所以$R/(\mathfrak{a}:a)\xrightarrow{\times a} R/\mathfrak{a}$的像就是$R\xrightarrow{\times a} R/\mathfrak{a}$的像,也就是$(a)$在$R/\mathfrak{a}$的像。另一方面,考虑$R/\mathfrak{a} \to R/(\mathfrak{a}+(a))$的核,任取$r+\mathfrak{a}\in R/\mathfrak{a}$,如果他在$R/(\mathfrak{a}+(a))$中为零,则$r\in (a)$,因此$R/\mathfrak{a} \to R/(\mathfrak{a}+(a))$的核也是$(a)$在$R/\mathfrak{a}$的像,正和性得证。

	由于短正合列正中一个模是Noether/Artin模当且仅当两边的模是Noether/Artin模,所以从正合列两边都是有限长模,就得到了$R/\mathfrak{a}$是有限长模。因此$R$中所有使得$R/\mathfrak{a}$是有限长$R$-模的理想$\mathfrak{a}$构成的理想族是Oka理想族。
\end{proof}

\begin{pro}
一个Noether环$R$,如果所有的素理想都是极大理想,则他是一个有限长$R$-模。
\end{pro}

\begin{proof}
	由于是Noether环,非空理想族满足极大性条件,考虑$R$中所有使得$R/\mathfrak{a}$是有限长$R$-模的理想$\mathfrak{a}$构成的Oka理想族$\mathcal{F}$. 假设$R$不是有限长模,则$(0)$就在其中,因此$\mathcal{F}^c$非空,由极大性条件,$\mathcal{F}^c$中存在极大元$\mathfrak{p}$,并且由$\mathcal{F}$是一个Oka理想族,$\mathfrak{p}$是一个素理想,进而由条件是一个极大理想。由于$R/\mathfrak{p}$是一个单模,长度为1. 但是$\mathfrak{p}\not\in \mathcal{F}$,所以$R/\mathfrak{p}$不是有限长模,矛盾。
\end{proof}

\begin{pro}
一个环$R$如果是Artin环,则$R$的每一个素理想都是极大理想且$R$极大理想数目有限。

\end{pro}

\begin{proof}
	令$\mathfrak{p}$是$R$的一个素理想,则$R/\mathfrak{p}$是一个Artin整环。取非零$x\in R/\mathfrak{p}$,有降链$(x)\supset (x^2)\cdots$,由于是Artin整环,所以存在正整数$n$使得$(x^n)=(x^{n+1})$,因此$x^n=x^{n+1}y$,由整环的消去律得到$xy=1$. 所以$R/\mathfrak{p}$是一个域。

	剩下我们考虑所有有限个极大理想的交构成的理想族,由极小条件,这个理想族有极小元$\mathfrak{m}_1\cap \cdots \cap \mathfrak{m}_n$. 任取极大理想$\mathfrak{m}$,由于$\mathfrak{m}\cap\mathfrak{m}_1\cap \cdots \cap \mathfrak{m}_n\subset \mathfrak{m}_1\cap \cdots \cap \mathfrak{m}_n$,因此极小性给出$\mathfrak{m}\cap\mathfrak{m}_1\cap \cdots \cap \mathfrak{m}_n=\mathfrak{m}_1\cap \cdots \cap \mathfrak{m}_n$,故$\mathfrak{m}_1\cap \cdots \cap \mathfrak{m}_n\subset \mathfrak{m}$. 由Proposition \eqref{primeau},存在$i$使得$\mathfrak{m}_i\subset \mathfrak{m}$,由于是极大理想,所以$\mathfrak{m}=\mathfrak{m}_i$.
\end{proof}

\begin{pro}
一个环$R$是Artin环,当且仅当,$R$是一个Noether环且所有素理想都是极大理想。
\end{pro}

\begin{proof}
	一个环$R$是Noether环且所有素理想都是极大理想,我们已经看到这是一个有限长$R$-模,进而是一个Artin环。反过来,我们也已经看到,Artin环的所有素理想都是极大理想。

	Artin性就是极小性条件。考虑$R$所有极大理想的任意有限乘积构成的理想族,它存在极小元$\mathfrak{a}$. 由于$\mathfrak{a}^2\subset \mathfrak{a}$也是有限个极大理想的乘积,由极小性$\mathfrak{a}^2=\mathfrak{a}$. 类似地,任取极大理想$\mathfrak{m}$,有$\mathfrak{m}\mathfrak{a}=\mathfrak{a}$. 这就是说,$\mathfrak{a}$中的元素在任意的极大理想$\mathfrak{m}$中。我们下面证明$\mathfrak{a}$只能为零。

	如果$\mathfrak{a}$非零,在所有使得$\mathfrak{b}\mathfrak{a}\neq 0$的理想$\mathfrak{b}$中选择极小的那个,我们有$(\mathfrak{b}\mathfrak{a})\mathfrak{a}=\mathfrak{b}\mathfrak{a}^2=\mathfrak{b}\mathfrak{a}\neq 0$,于是$\mathfrak{b}\mathfrak{a}\subset \mathfrak{b}$,由极小性,$\mathfrak{b}\mathfrak{a}=\mathfrak{b}$. 必然存在一个$b\in\mathfrak{b}$使得$b\mathfrak{a}\neq 0$,于是由极小性$\mathfrak{b}=(b)$,因此$\mathfrak{b}\mathfrak{a}=\mathfrak{b}$,所以存在一个$a\in \mathfrak{a}$使得$ab=b$,或者$(a-1)b=0$. 由于$a$在任意的极大理想中,所以$a-1$是可逆的,故$b=0$. 矛盾。

	现在已经知道,$0$可以写成有限个极大理想的乘积,不妨写作$0=\mathfrak{m}_1\cdots \mathfrak{m}_n$. 考虑链
	\[
	0=\mathfrak{m}_1\cdots \mathfrak{m}_n\subset \mathfrak{m}_1\cdots \mathfrak{m}_{n-1}\subset \cdots\subset \mathfrak{m}_1\subset R,
	\]
	有$(\mathfrak{m}_1\cdots \mathfrak{m}_{k-1})/(\mathfrak{m}_1\cdots \mathfrak{m}_{k})$是一个$R/\mathfrak{m}_{k}$-矢量空间。由于$(\mathfrak{m}_1\cdots \mathfrak{m}_{k-1})/(\mathfrak{m}_1\cdots \mathfrak{m}_{k})$作为一个Artin模子模的商模也是一个Artin模。看作$R/\mathfrak{m}_{k}$-矢量空间时,Artin性与Noether性等价,所以这是一个Noether模。

	考虑正合列
	\[
	0\to \mathfrak{m}_1\cdots \mathfrak{m}_{k}\hookrightarrow\mathfrak{m}_1\cdots \mathfrak{m}_{k-1}\to (\mathfrak{m}_1\cdots \mathfrak{m}_{k-1})/(\mathfrak{m}_1\cdots \mathfrak{m}_{k})\to 0,
	\]
	正中间的Noether性来自于两边的Noether性,所以有限次归纳后,可以得到$R$是一个Noether模。
\end{proof}

\para 结合上面两个命题,一个环$R$是Artin环等价于他是一个有限长$R$-模,也当且仅当他是Noether环且所有的素理想都是极大理想。因此,与模的情况不一样,到了环这里,Artin性就是一个远比Noether性强的一个性质。附带的,还断言了,Artin环的素理想(极大理想)有限。

由于Artin环是一个有限长模,所以可以把有限长模的结构定理用到Artin环上,这里就不表了。
