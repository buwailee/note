\chapter{模(一)}
假设下面出现的环都是交换含幺环,所以我们不区分左右模。
\section{直和、直积与张量积}

张量抽象了多线性函数,尤其是双线性函数。

\para 设$A$是右$R$-模,$B$是左$R$-模,那么我们称$f:A\times B\to G$,其中$G$是一个交换群,为一个双线性函数,如果他满足
\[
	f(a+b,a')=f(a,a')+f(b,a'),\quad f(a,a'+b')=f(a,a')+f(a,b'),
\]
以及对$r\in R$满足$f(ar,b)=f(a,rb)$.

特别地,如果我们存在一个双线性函数$\varphi$,以及一个交换群$G$,使得每一个$A\times B$上的双线性函数$f:A\times B\to H$都可以唯一分解为
\[
	f:A\times B\xrightarrow{\varphi} G\xrightarrow{h_f}H,
\]
其中$h_f$对每一个双线性函数$f$存在且唯一,则称呼$G$为$A$与$B$的张量积,记做$A\otimes_R B$,而$\varphi(a,b)$记做$a\otimes_R b$,如果下标$R$不重要,那么我们可以省略他。

\lem 在模范畴内,张量积存在。而且由上面的泛性质,他确定到一个同构。

\proof 对于唯一性,我们考虑两个张量积$(G,\varphi)$和$(G',\varphi')$,那么根据张量积的性质,有分解
\[
	\varphi:A\times B\xrightarrow{\varphi'} G'\xrightarrow{h_{\varphi}}G,\quad \varphi':A\times B\xrightarrow{\varphi} G\xrightarrow{h_{\varphi'}}G',
\]
所以只要验证$h_\varphi\circ h_{\varphi'}=\id_{G}$和$h_{\varphi'}\circ h_{\varphi}=\id_{G'}$就好了,这样我们就得到了$G$和$G'$之间的同构。而上述等式来自于分解的唯一性,显然,我们有分解
\[
	A\times B\xrightarrow{\varphi'} G'\xrightarrow{h_{\varphi}}G\xrightarrow{h_{\varphi'}}G',\quad A\times B\xrightarrow{\varphi'} G'\xrightarrow{\id_{G'}}G',
\]
显然,所以由唯一性得到了$h_{\varphi'}\circ h_{\varphi}=\id_{G'}$,同理有另一个等式。

对于存在性,我们可以直接构造,首先,我们知道在交换群范畴有直和存在,那么我们可以构造自由交换群$F=\bigoplus_{(a,b)\in A\times B} \zz\langle (a,b)\rangle$.令$1_{a,b}$是$\zz\langle (a,b)\rangle$中的$1$,我们令$H$是由
\[
	1_{a,b}+1_{a',b}-1_{a+a',b},\quad 1_{a,b}+1_{a,b'}-1_{a,b+b'},\quad 1_{ar,b}-1_{a,rb}
\]
生成的子群,那么我们$A\otimes B$就可以构造为$F/H$,令$\varphi(a,b)=a\otimes b=[1_{a,b}]$,即$1_{a,b}$的陪集。这确实是张量积,具体的检验这里就不进行了。\qed

设$R$是交换环,$A$和$B$都是双边模,那么$A\otimes_R B$显然有一个$R$-模结构,比如$r(a\otimes b)$可以定义为$(ra)\otimes b$.特别地,如果交换环$T$是一个$R$-代数\footnote{设$f:R\to T$是一个环同态,那么我们可以通过$r\cdot t=f(r)t$在$T$上定义一个$R$-模结构,此时称$T$是一个$R$-代数。此外,$R$-代数同态是一个环同态,同时也是一个$R$-模同态。},设$A$是一个$R$-模,则$A\otimes_R T$就是一个$T$-模。这是因为,我们可以通过$t(a\otimes b)=a\otimes (tb)$来定义标量乘法。

\para 有同构$A\otimes\bigoplus_i B_i\cong \bigoplus_i (A\otimes B_i)$,$R\otimes A\cong A$以及$(R/I)\otimes B\cong B/IB$,其中$I$是$R$的一个理想。这俩的证明完全是利用泛性质,这里就不说了。

\para 有了两个模的张量积,我们自然也可以拓展为三个模的张量积,我们可以通过模仿两个模的张量积的泛性质\footnote{即三线性的函数可以唯一分解。},定义一个新的三个模之间的张量积$A\otimes B\otimes C$,然后可以检验$(A\otimes B)\otimes C$和$A\otimes (B\otimes C)$同时也满足泛性质,所以他们之间是同构的。在这层意义上,我们可以认为张量积满足结合律,因此我们自然也有了有限个模的张量积。

\section{有限生成模与可表示模}

\section{链条件}

\section{自由模与矢量空间}

\section{代数与标量扩张}

设有两个环$A$, $B$和一个环同态$f:A\to B$,再设$a\in A$, $b\in B$,我们可以通过$f$定义他们的乘法为$a\cdot b=f(a)b\in B$,这样环$B$就被赋予了一个$A$-模结构。

\para 环$B$若被赋予一个$A$-模结构,则称$B$是一个$A$-代数。

设$B$是$A$-代数,有$f:A\to B$,设$C$也是$A$-代数,有$g:A\to C$,那么$A$-代数之间的同态$h:B\to C$,首先是$B$和$C$之间的环同态,还要和$A$-模结构相容,即$g=h\circ f$.

\para 设$B$是$A$-代数,我们称呼$B$是有限生成$A$-代数,如果他同构于$A[x_1,\cdots ,x_n]/\mathfrak{a}$,其中$\mathfrak{a}$是$A[x_1,\cdots ,x_n]$的一个理想。一个环称为有限生成的就是指他作为$\zz$-代数是有限生成的。如果$B$作为$A$-模是有限生成的,则称$B$作为$A$-代数是有限的。

复化是所谓的标量扩充或者base change的特例,所以我们直接来谈模(交换环上的双边模)的标量扩充。

\para 关于模的张量基,可以参看附录。如果$M$是$R$-模,$S$是$R$-代数,那么$M_S=S\otimes_R M$就是一个$S$-模,通过$s(t\otimes a)=(st)\otimes a$,我们称这个$S$-模$M_S$为$M$经过标量扩充而来的。

\lem 有以下同构:

\no{1} $M_S\otimes_S N_S\cong (M\otimes_R N)_S$.

\no{2} 设$N$是一个$S$-模,有同构$\Hom_R(M,N)\cong \Hom_S(M_S,N)$

\proof 第一个同构。实际上,$(s_1\otimes_R a,s_2\otimes_R b)\mapsto s_1s_2\otimes_R(a\otimes_R b)$是一个双线性映射,所以他唯一诱导了$\varphi:M_S\otimes_S N_S\to (M\otimes_R N)_S$.

反过来,映射$(s,a\otimes_R b)\mapsto (s\otimes_R a)\otimes_S (1\otimes_R b)=(1\otimes_R a)\otimes_S (a\otimes_R b)$也是一个双线性映射,所以他唯一诱导了$\psi:(M\otimes_R N)_S\to M_S\otimes_S N_S$.容易验证他们互逆。

第二个同构。设$\phi\in \Hom_R(M,N)$,我们可以通过$\Phi(s\otimes a)=s\phi(a)$定义$\Phi\in \Hom_S(M_S,N)$,反过来,已知$\Phi\in \Hom_S(M_S,N)$,我们可以通过$\phi(a)=\Phi(1\otimes a)$来定义$\phi\in \Hom_R(M,N)$.不难检验,这是一个同构。 \qed

对有限个$\{M_i\}_{i\in I}$,那么反复利用\no{1}我们就得到了同构
\[
\bigotimes_{i\in I} \left(S\otimes_RM_i\right)\cong S\otimes_R\bigotimes_{i\in I} M_i.
\]

\para 设$V$是一个$\rr$上的$n$维矢量空间,那么$V\cong \rr\oplus \cdots \oplus \rr$,所以$V_{\cc}\cong \rr_{\cc}\oplus \cdots \oplus \rr_{\cc}$,由于$\rr_{\cc}=\cc\otimes \rr=\cc$,所以$V_{\cc}\cong \cc\oplus \cdots \oplus \cc$,因此$V_{\cc}$这就被称为$V$的复化,同时因为$v\mapsto 1\otimes v$是单射,所以我们把$V$看成$V_\cc$的实的子空间,即$1\otimes V=V$.

\lem 复化扩张引理:

\no{1}设$V$是一个实矢量空间,$W$是一个复矢量空间,$f:V\to W$是实线性映射,则他可以唯一扩张为一个复线性映射$V_\cc\to W$.

\no{2}设$V$和$W$都是实矢量空间,$f:V\to W$是实线性映射,则他可以唯一扩张为一个复线性映射$V_\cc\to W_\cc$.

\no{3}设$V$和$W$都是实矢量空间,$f:V\times\cdots\times V\to W$是实多线性映射,则他可以唯一扩张为一个复多线性映射$V_\cc\times\cdots\times V_\cc\to W_\cc$.

\proof 
	因为$\Hom_\rr(V,W)\cong \Hom_\cc(V_\cc,W)$,所以\no{1}得证。对于\no{2},我们有$i:W\hookrightarrow W_\cc$,那么$i\circ f:V\to W_\cc$就满足\no{1}题设,由\no{1}自然得证。对于\no{3},因为多线性映射他可以唯一分解出一个$V^k\to W$,然后由\no{2},可以唯一扩张为$(V^k)_\cc\to W_\cc$,因为$(V^k)_\cc=V_\cc\otimes_\cc \cdots\otimes_\cc V_\cc$,引理得证。
\qed