% author: buwailee@nmhs
\documentclass[11pt]{extarticle}
\usepackage{noteheader}
\usepackage{indentfirst}

\usepackage[section]{../egastyle}
\usepackage{titletoc}
% \usepackage[titletoc,title]{appendix}
	% \renewcommand{\appendixname}{Appendix}

\usepackage{hyperref}
	\hypersetup{bookmarksnumbered=true}

\definecolor{shadecolor}{rgb}{0.92,0.92,0.92}

\newcommand{\no}[1]{{$(#1)$}}
% \renewcommand{\not}[1]{#1\!\!\!/}
\newcommand{\rr}{\mathbb{R}}
\newcommand{\zz}{\mathbb{Z}}
\newcommand{\aaa}{\mathfrak{a}}
\newcommand{\pp}{\mathfrak{p}}
\newcommand{\mm}{\mathfrak{m}}
\newcommand{\dd}{\mathrm{d}}
\newcommand{\oo}{\mathcal{O}}
\newcommand{\calf}{\mathcal{F}}
\newcommand{\calg}{\mathcal{G}}
\newcommand{\bbp}{\mathbb{P}}
\newcommand{\bba}{\mathbb{A}}
\newcommand{\osub}{\underset{\mathrm{open}}{\subset}}
\newcommand{\csub}{\underset{\mathrm{closed}}{\subset}}

\DeclareMathOperator{\im}{Im}
\DeclareMathOperator{\Hom}{Hom}
\DeclareMathOperator{\id}{id}
\DeclareMathOperator{\rank}{rank}
\DeclareMathOperator{\tr}{tr}
\DeclareMathOperator{\supp}{supp}
\DeclareMathOperator{\coker}{coker}
\DeclareMathOperator{\codim}{codim}
\DeclareMathOperator{\height}{height}
\DeclareMathOperator{\sign}{sign}

\DeclareMathOperator{\ann}{ann}
\DeclareMathOperator{\Ann}{Ann}
\DeclareMathOperator{\ev}{ev}
	\newcommand{\cc}{\mathbb{C}}

\title{The note of GTM 150}
\author{munuxi}
\begin{document}
\maketitle
\tableofcontents

\section*{Elementary Definitions}

\pro A domain is a UFD iff every irreducible element is prime and every chain of principal ideals is a.c.c.

\para Intersection of two submodules can be realized as the kernel of homomorphism $\mu:M_1\otimes M_2\to M_1+M_2$ given by $\mu(m_1,m_2)=m_1-m_2$. So there is a short exact sequence
\[
	0\to M_1\cap M_2 \to M_1\oplus M_2 \to M_1+M_2 \to 0.
\]

\section{Roots of Commutative Algebra}

\section{The Basis Theorem}

\theo Hilbert Basis Theorem: If a ring $R$ is Noetherian, then $R[x]$ is Noetherian.

Then $R[x_1,\cdots,x_n]=R[x_1,\cdots,x_{n-1}][x_n]$ is also Noetherian after finite induction. Since ``f.g.'' is stable under a surjective homomorphism, as a quotient ring of $R[x_1,\cdots,x_n]$, a f.g. $R$-algebra is a Noetherian ring.

\pro If $R$ is a noetherian ring, M is a f.g. $R$-module, then M is a Noetherian $R$-module.

\Section{G}

\end{document}