\chapter{模(二)}

\section{准素分解}

\para 考虑一个$R$-模$M$,如果对$r\in R$存在一个非零的$m\in M$使得$rm=0$,则这个$r$称为$M$的一个零因子。对固定的$m\in M$,使得$rm=0$的$r\in R$的集合记作$\ann(m)$. 这是一个$R$中的理想,因为$ram=r0=0$对$a\in \ann(m)$成立。设$S$是$M$的一个子集,定义$\ann(S)=\bigcap_{m\in S}\ann(m)$.

设$M$是一个$R$-模,设$\mathfrak{a}\subset \ann(M)$是$R$中的一个理想,则$M$可以看成一个$R/\mathfrak{a}$-模,因为$(r+\mathfrak{a})m=rm$. 特别地,$M$是一个$R/\ann(M)$-模,此时,如果$rm=0$,则可以推出$r$和$m$中有一个零。

\para $\ann(m)$中的那些素理想,称之为$M$的associated primes,定义$\Ann_R(M)$为所有associated primes的集合,如果下标$R$是清楚的,我们会略去它。

当然有可能所有的$\ann(m)$都不是素理想,但是如果$R$是一个Noether环,则下面的引理保证了$\Ann(M)$非空。

\lem 设$\mathcal{P}$是所有形如$\ann(m)$的理想按照包含构成的一个偏序集,则$\mathcal{P}$中的极大元都是素理想。

\proof
	设$\ann(m)$是一个极大元,再设$r\notin \ann(m)$,即$rm\neq 0$. 很清楚,$\ann(m)\subset \ann(rm)$,所以由极大性可知$\ann(rm)=\ann(m)$. 现在如果$rs\in \ann(m)$但$r\notin \ann(m)$,则$srm=0$可知$s\in \ann(rm)=\ann(m)$,所以$\ann(m)$是一个素理想。
\qed

如果$R$是一个Noether环,则它的理想构成的非空子集一定存在极大元,这就保证了$\Ann(M)$非空。并且,如果$r$是$M$的一个零因子,含有$r$的那个极大的$\ann(m)$属于$\Ann(M)$,这就是说,任意的零因子必然属于$\Ann(M)$中的某个元素,或者如下包含关系
\[
	\bigcup_{0\neq m\in M}\ann(m)\subset \bigcup_{\mathfrak{a}\in \Ann(M)}\mathfrak{a}. 
\]

从这里开始,我们假设这节下面出现的环都是Noether环,除非特别申明。

\lem 设$M$是一个$R$-模,如果$\ann(m)$是素理想,则对$M$的子模$Rm$的任意非零子模$N$有$\ann(N)=\ann(m)$.

\proof
	显然,$\ann(m)\subset \ann(N)$. 由于有$R$-模同构$\psi: Rm\to R/\ann(m)$,所以$Rm$的任意非零子模$N$对应着$R/\ann(m)$的非零理想$\psi(N)$. 如果$r\in \ann(N)$,则$r\psi(N)=\psi(rN)=0$,因为$\psi(N)$非零且$R/\ann(m)$是整环,所以$r$在$R/\ann(m)$中的像为零,即是说$r\in \ann(M)$. 所以$\ann(N)\subset \ann(m)$.
\qed

\pro 如果我们有$R$-模的正合列$0\to M'\to M\to M''\to 0$,则我们有包含关系
\[
	\Ann(M')\subset \Ann(M)\subset \Ann(M')\cup \Ann(M'').
\]

\proof
	由于单同态并不会改变$\Ann$,所以可以将$M'$看成$M$的子模,所以第一个包含显然。对第二个包含,任取$\mathfrak{a}\in \Ann(M)-\Ann(M')$,我们要证明$\mathfrak{a}\in \Ann(M')$. 设$\mathfrak{a}=\ann(m)$,其中$m\in M$,由$m$生成的子模$Rm$同构于$R/\mathfrak{a}$. 考虑$N=Rm\cap M'$,他是$Rm$的子模,如果$N$非零,因为$\mathfrak{a}\notin \Ann(M')$,$\mathfrak{a}N\neq 0$,因此$\mathfrak{a}\not\subset \ann(N)=\ann(m)=\mathfrak{a}$,矛盾,所以只可能$Rm\cap M'=0$. 在这种情况下,$Rm$同构于他在$M''$中的像,所以$\mathfrak{a}\in \Ann(M'')$.
\qed

考虑$M=M'\oplus M''$的情况,由于$M'$和$M''$都有到$M$的单同态,所以$\Ann(M')\subset \Ann(M)$以及$\Ann(M'')\subset \Ann(M)$,继而他给出了$\Ann(M)=\Ann(M')\cup \Ann(M'')$.

\para 设$N$是$M$的一个子模,如果$\Ann(M/N)$只有一个元素,则称呼$N$是$M$的一个准素子模,如果$\Ann(M/N)=\{\mathfrak{p}\}$,则称呼$N$是$\mathfrak{p}$-准素的。

如果设$N_1$和$N_2$都是$\mathfrak{p}$-准素的,则$N_1\cap N_2$也是. 事实上,考虑单同态$M/(N_1\cap N_2)\to M/N_1\oplus M/N_2$,因此$\Ann(M/(N_1\cap N_2))\subset \Ann(M/N_1)\cup \Ann(M/N_2)=\{\mathfrak{p}\}\cup \{\mathfrak{p}\}=\{\mathfrak{p}\}$. 但是由于$\Ann(M/(N_1\cap N_2))$非空,所以$\Ann(M/(N_1\cap N_2))=\{\mathfrak{p}\}$.

经过有限次归纳可以得到:有限个$\mathfrak{p}$-准素子模的交依然是一个$\mathfrak{p}$-准素子模。

\lem 设$R$是Noether环,$M$是有限生成$R$-模,则$M$的不可约子模$N$是准素子模。

\proof
	如果$\Ann(M/N)$中至少存在两个素理想$\pp_1=\ann(\bar{m}_1)$和$\pp_2=\ann(\bar{m}_2)$,那么在$M/N$中有子模$R\bar{m}_1\cong R/\pp_1$和$R\bar{m}_2\cong R/\pp_2$. 如果$R\bar{m}_1\cap R\bar{m}_2$非零,则$R\bar{m}_1\cap R\bar{m}_2$作为$R\bar{m}_1$的非零子模,$\ann(R\bar{m}_1\cap R\bar{m}_2)=\pp_1$,作为$R\bar{m}_2$的非零子模,$\ann(R\bar{m}_1\cap R\bar{m}_2)=\pp_2$,矛盾,于是推出$R\bar{m}_1\cap R\bar{m}_2=\{\bar{0}\}$. 回到$M$中的原像,我们就可以知道$N$可以写成两个$M$的子模的交,这与$N$不可约矛盾。
\qed

准素子模的重要性来自于下面这个定理,他被称为准素分解。

\theo 设$R$是Noether环,$M$是有限生成$R$-模,如果$N$是$M$的一个真子模,那么他是有限个准素子模的交。

\proof
	由于Noether环上的有限生成模是Noether模,回忆Proposition \eqref{irrde},$N$可以写成$M$中的有限个不可约子模的交的形式,而每一个不可约子模都是准素子模。
\qed

\section{主理想整环上的有限生成模}

\section{微分模}

\para 设$S$是一个环,$M$是一个$S$-模,则一个加法群同态$d:S\to M$被称为一个导子,如果它满足Leibniz法则
\[
	d(fg)=fdg+gdf.
\]
如果$S$是一个$R$-代数,设$\varphi:R\to S$是定义代数的环同态,那么$r\cdot m=\varphi(r)m$定义出了$M$上的一个$R$-模结构。此时,导子$d$如果是$R$-模同态,则称呼$d$是$R$-线性的。$S$到$M$所有的$R$-线性导子构成了一个$R$-模。

由于$d(1\cdot 1)=d(1)+d(1)$,因此对任意的导子有$d(1)=0$. 如果$d$是$R$线性的,对$r\in R$,我们有$d(r)=rd(1)=0$. 设$s\in S$,则$d(s^n)=ns^{n-1}d(s)$,这个可以用Leibniz法则直接计算得到。

\pro 给定环$R$和$R$-代数$S$,则存在一个$S$-模$\Omega_{S/R}$和一个$R$-线性的导子$d:S\to R$使得任意的$S$-模$M$与$R$-线性导子$\delta: S\to M$都存在唯一的$R$-线性映射$\varphi:\Omega_{S/R}\to M$使得分解
\[
	\delta:S\xrightarrow{d}\Omega_{S/R}\xrightarrow{\varphi}M
\]
成立。

\proof
	考虑集合$\{d(f)\,:\, f\in S\}$生成的自由$S$-模$M$,再考虑其中由
	\[
	d(fg)-fd(g)-gd(f),\quad d(r_1f_1+r_2f_2)-r_1d(f_1)-r_2d(f_2)
	\]
	生成的子模$N$,于是$\Omega_{S/R}$就被构造为$M/N$,而$d:f\mapsto d(f)$很容易检验这是一个$R$-线性导子。

	泛性质的检验从构造是简单的。
\qed

今后,$d(f)$的括号不产生理解上的差异的时候会略去记成$df$. 上面定义的模就被称为$S$的微分模。由泛性质,微分模在同构意义下唯一。

\para 设$S$是一个有限生成$R$-代数,他被$\{f_1,\cdots,f_n\}$生成,那么任意的单项式$f_1^{k_1}\cdots f_n^{k_n}\in S$求导后有
\[
	d(f_1^{k_1}\cdots f_n^{k_n})=\sum_{i=1}^n f_1^{k_1}\cdots d(f_i^{k_i})\cdots f_n^{k_n},
\]
如果$k_i=0$则$d(f_i^{k_i})=0$,如果$k_i>0$则$d(f_i^{k_i})=f_i^{k_i-1}df_i$,所以这个单项式可以由$\{df_i\}$线性组合而言,继而$\Omega_{S/R}$就可以由$\{df_i\}$线性组合。

因此,我们求出了,如果$S=R[x_1,\cdots,x_n]$,则$\Omega_{S/R}=\bigoplus_{i=1}^n Sdx_i$,即由$\{dx_i\}$生成的自由模。

\para 回忆一个$R$-代数$S$实际上是一个三元组$(R,S,\varphi)$,其中$\varphi:R\to S$是一个环同态。考虑所有这样三元组构成的范畴,其中态射$\rho:(R,S,\varphi)\to (R',S',\varphi')$是两个环同态$\rho_R:R\to R'$以及$\rho_S:S\to S'$,它们需要满足交换图
\[
\begin{xy}
	\xymatrix
	{
		R\ar[rr]^{\rho_R} \ar[d]_{\varphi}&&R' \ar[d]^{\varphi'}\\
		S\ar[rr]^{\rho_S}&&S'
	}
\end{xy}
\]
不难检验复合的成立。

由于每个代数都可以考虑它的微分模,那么我们自然可以考虑微分模在上述态射下的变化,为此,我们记$\Omega(S,R,\varphi)$来代替$\Omega_{S/R}$,则上述态射诱导了如下的交换图
\[
\begin{xy}
	\xymatrix
	{
		R\ar[r]^{\varphi}\ar[d]_{\rho_R}&S\ar[rr]^{d} \ar[d]_{\rho_S}&&\Omega(S,R,\varphi) \ar[d]\\
		R'\ar[r]^{\varphi'}&S'\ar[rr]^{d'}&&\Omega(S',R',\varphi')
	}
\end{xy}
\]
没有标记的那个同态来自于微分模的泛性质,因此,我们可以将$\Omega$看成一个代数范畴到模范畴的一个函子。

\para 考虑$R=R'$的情况,此时代数的态射约化成了$S\to S'$的一个环同态,换而言之,即考虑
$R\to S\to T$的情况。

考虑$T$-模$\Omega_{T/R}$和$\Omega_{T/S}$,由于$S$-线性可以推出$R$-线性,所以$dc\mapsto dc$给出了满同态$\Omega_{T/R}\to\Omega_{T/S}$. 对$S$-模$\Omega_{S/R}$,我们可以标量扩张成一个$T$-模,通过$T\otimes_S \Omega_{S/R}$. 由于同样都是$R$-线性,所以通过$t\otimes ds\mapsto tds$可以给出同态$T\otimes_S \Omega_{S/R}\to \Omega_{T/R}$,并且,由于$ds$在$\Omega_{T/S}$中为零,所以上面两个同态复合为零。综上,我们有正合列
\[
	T\otimes_S \Omega_{S/R}\xrightarrow{\varphi} \Omega_{T/R}\xrightarrow{\psi} \Omega_{T/S}\to 0,
\]
其中$\varphi(t\otimes ds)\mapsto tds$以及$\psi(dt)\mapsto dt$.