\chapter{模(二)}

\section{准素分解}

\para 考虑一个$R$-模$M$,如果对$r\in R$存在一个非零的$m\in M$使得$rm=0$,则这个$r$称为$M$的一个零因子。对固定的$m\in M$,使得$rm=0$的$r\in R$的集合记作$\ann(m)$. 这是一个$R$中的理想,因为$ram=r0=0$对$a\in \ann(m)$成立。设$S$是$M$的一个子集,定义$\ann(S)=\bigcap_{m\in S}\ann(m)$.

$\ann(m)$中的那些素理想,称之为$M$的associated primes,定义$\Ann_R(M)$为所有associated primes的集合,如果下标$R$是清楚的,我们会略去它。

当然有可能所有的$\ann(m)$都不是素理想,但是如果$R$是一个Noether环,则下面的引理保证了$\Ann(M)$非空。

\lem 设$\mathcal{P}$是所有形如$\ann(m)$的理想按照包含构成的一个偏序集,则$\mathcal{P}$中的极大元都是素理想。

\proof
	设$\ann(m)$是一个极大元,再设$r\notin \ann(m)$,即$rm\neq 0$. 很清楚,$\ann(m)\subset \ann(rm)$,所以由极大性可知$\ann(rm)=\ann(m)$. 现在如果$rs\in \ann(m)$但$r\notin \ann(m)$,则$srm=0$可知$s\in \ann(rm)=\ann(m)$,所以$\ann(m)$是一个素理想。
\qed

如果$R$是一个Noether环,则它的理想构成的非空子集一定存在极大元,这就保证了$\Ann(M)$非空。并且,如果$r$是$M$的一个零因子,含有$r$的那个极大的$\ann(m)$属于$\Ann(M)$,这就是说,任意的零因子必然属于$\Ann(M)$中的某个元素,或者如下包含关系
\[
	\bigcup_{0\neq m\in M}\ann(m)\subset \bigcup_{\mathfrak{a}\in \Ann(M)}\mathfrak{a}. 
\]

从这里开始,我们假设这节下面出现的环都是Noether环,除非特别申明。

\lem 设$M$是一个$R$-模,如果$\ann(m)$是素理想,则对$M$的子模$Rm$的任意非零子模$N$有$\ann(N)=\ann(m)$.

\proof
	显然,$\ann(m)\subset \ann(N)$. 由于有$R$-模同构$\psi: Rm\to R/\ann(m)$,所以$Rm$的任意非零子模$N$对应着$R/\ann(m)$的非零理想$\psi(N)$. 如果$r\in \ann(N)$,则$r\psi(N)=\psi(rN)=0$,因为$\psi(N)$非零且$R/\ann(m)$是整环,所以$r$在$R/\ann(m)$中的像为零,即是说$r\in \ann(M)$. 所以$\ann(N)\subset \ann(m)$.
\qed

\pro 如果我们有$R$-模的正合列$0\to M'\to M\to M''\to 0$,则我们有包含关系
\[
	\Ann(M')\subset \Ann(M)\subset \Ann(M')\cup \Ann(M'').
\]

\proof
	由于单同态并不会改变$\Ann$,所以可以将$M'$看成$M$的子模,所以第一个包含显然。对第二个包含,任取$\mathfrak{a}\in \Ann(M)-\Ann(M')$,我们要证明$\mathfrak{a}\in \Ann(M')$. 设$\mathfrak{a}=\ann(m)$,其中$m\in M$,由$m$生成的子模$Rm$同构于$R/\mathfrak{a}$. 考虑$N=Rm\cap M'$,他是$Rm$的子模,如果$N$非零,因为$\mathfrak{a}\notin \Ann(M')$,$\mathfrak{a}N\neq 0$,因此$\mathfrak{a}\not\subset \ann(N)=\ann(m)=\mathfrak{a}$,矛盾,所以只可能$Rm\cap M'=0$. 在这种情况下,$Rm$同构于他在$M''$中的像,所以$\mathfrak{a}\in \Ann(M'')$.
\qed

考虑$M=M'\oplus M''$的情况,由于$M'$和$M''$都有到$M$的单同态,所以$\Ann(M')\subset \Ann(M)$以及$\Ann(M'')\subset \Ann(M)$,继而他给出了$\Ann(M)=\Ann(M')\cup \Ann(M'')$.

\para 设$N$是$M$的一个子模,如果$\Ann(M/N)$只有一个元素,则称呼$N$是$M$的一个准素子模,如果$\Ann(M/N)=\{\mathfrak{p}\}$,则称呼$N$是$\mathfrak{p}$-准素的。

如果设$N_1$和$N_2$都是$\mathfrak{p}$-准素的,则$N_1\cap N_2$也是. 事实上,考虑单同态$M/(N_1\cap N_2)\to M/N_1\oplus M/N_2$,因此$\Ann(M/(N_1\cap N_2))\subset \Ann(M/N_1)\cup \Ann(M/N_2)=\{\mathfrak{p}\}\cup \{\mathfrak{p}\}=\{\mathfrak{p}\}$. 但是由于$\Ann(M/(N_1\cap N_2))$非空,所以$\Ann(M/(N_1\cap N_2))=\{\mathfrak{p}\}$.

经过有限次归纳可以得到:有限个$\mathfrak{p}$-准素子模的交依然是一个$\mathfrak{p}$-准素子模。

\lem 设$R$是Noether环,$M$是有限生成$R$-模,则$M$的不可约子模$N$是准素子模。

\proof
	如果$\Ann(M/N)$中至少存在两个素理想$\pp_1=\ann(\bar{m}_1)$和$\pp_2=\ann(\bar{m}_2)$,那么在$M/N$中有子模$R\bar{m}_1\cong R/\pp_1$和$R\bar{m}_2\cong R/\pp_2$. 如果$R\bar{m}_1\cap R\bar{m}_2$非零,则$R\bar{m}_1\cap R\bar{m}_2$作为$R\bar{m}_1$的非零子模,$\ann(R\bar{m}_1\cap R\bar{m}_2)=\pp_1$,作为$R\bar{m}_2$的非零子模,$\ann(R\bar{m}_1\cap R\bar{m}_2)=\pp_2$,矛盾,于是推出$R\bar{m}_1\cap R\bar{m}_2=\{\bar{0}\}$. 回到$M$中的原像,我们就可以知道$N$可以写成两个$M$的子模的交,这与$N$不可约矛盾。
\qed

准素子模的重要性来自于下面这个定理,他被称为准素分解。

\theo 设$R$是Noether环,$M$是有限生成$R$-模,如果$N$是$M$的一个真子模,那么他是有限个准素子模的交。

\proof
	由于Noether环上的有限生成模是Noether模,回忆Proposition \eqref{irrde},$N$可以写成$M$中的有限个不可约子模的交的形式,而每一个不可约子模都是准素子模。
\qed

\section{主理想整环上的有限生成模}

\section{极限与余极限}

\para 设$\mathcal{J}$是一个小范畴(即范畴对象的全体能够构成一个集合),我们称函子$D:\mathcal{J}\to \mathcal{C}$为$\mathcal{C}$中的一个$\mathcal{J}$-图,或者简略叫做图。对于$j\in \mathcal{J}$,$D(j)$称为该图的一个顶点,而对任意的态射$\alpha:j_1\to j_2$,态射$D(\alpha)$称为该图的一条边。

\para 称$(A,\lambda)$为$J$-图的一个{锥形},如果$A$是$\mathcal{C}$的一个对象,而$\lambda$为一族态射$\lambda_{j}:A\to D(j)$使得如下交换图对所有的顶点和边都成立
\[
	\xymatrix{
		&A \ar[dl]_{\lambda_{j}}\ar[dr]^{\lambda_{j'}}&\\
		D(j)\ar[rr]^{D(\alpha)}&&D(j')
	}
\]

称呼一个锥形$(A,\lambda)$是$J$-图的一个{极限},如果对于任意的锥形$(B,\mu)$,都有唯一的态射$f:B\to A$使得如下分解$\mu_j:B\xrightarrow{f}A\xrightarrow{\lambda_j}D(j)$成立。

在实际过程中,如果极限$(A,\lambda)$中的态射族是明确的(乃至于是由对象$A$确定的),那么我们会用$A$来表示极限。并且一般将$J$-图$D$的极限写作$\varprojlim_{j\in J} D(j)$.

作为例子,模范畴的直积就是一种极限,他对应的小范畴,两个不同的元素之间没有态射。

\para 锥形和极限都有对偶的概念,在交换图中,无外乎就是将箭头完全反过来。称$(A,\lambda)$为$J$-图的一个{余锥形},如果$A$是$\mathcal{C}$的一个对象,而$\lambda$为一族态射$\lambda_{j}:D(j)\to A$使得如下交换图对所有的顶点和边都成立
\[
	\xymatrix{
		&A&\\
		D(j)\ar[rr]^{D(\alpha)} \ar[ur]^{\lambda_{j }}&&D(j')\ar[ul]_{\lambda_{j'}}
	}
\]

称呼一个余锥形$(A,\lambda)$是$J$-图的一个{余极限},如果对于任意的余锥形$(B,\mu)$,都有唯一的态射$f:A\to B$使得如下分解$\mu_j:D(j)\xrightarrow{\lambda_j}A\xrightarrow{f}B$成立。一般将$J$-图$D$的余极限写作$\varinjlim_{j\in J} D(j)$.

同样,作为例子,模范畴的直和就是一种余极限,他对应的小范畴,两个不同的元素之间没有态射。

\para 设$I$是一个偏序集,偏序关系为$\leq$,那么族$\{\{x\}\,:\, x\in I\}$构成一个(小)范畴,其中的态射定义为
\[
	\Hom_{I}\left(\{x\},\{y\}\right)=\begin{cases}
	\bigl\{x\mapsto y\bigr\}&\text{, if }x\leq y\text{;}\\
	\varnothing&\text{, otherwise}.
	\end{cases}
\]
当然,我们一般直接将$I$和$\{\{x\}\,:\, x\in I\}$等同,此时会说偏序集具有一个(小)范畴结构,其中的态射记作$x\leq y$.

如果我们定义$x\geq y$当且仅当$y\leq x$,那么新的偏序集$(I,\geq)$就是偏序集$(I,\leq)$的对偶范畴,略去偏序符号,有时候会记作$I^\circ$.

$D:I\to \mathcal{C}$是一个协变函子,就是对$i\leq j\leq k$成立$D(j\leq k)\circ D(i\leq j)=D(i\leq k)$,就是说从小到大有映射。反之,反变函子$D:I\to \mathcal{C}$对$i\leq j\leq k$成立$D(i\leq j)\circ D(j\leq k)=D(i\leq k)$,就是说从大到小有映射。

对于一个反变函子$D:I\to \mathcal{C}$,我们可以定义一个协变函子$D^\circ :I^\circ\to \mathcal{C}$通过$D^\circ(j\geq i)=D(i\leq j)$,此时对$k\geq j\geq i$成立$D^\circ(j\geq i)\circ D^\circ(k\geq j)=D^\circ(k\geq i)$,可见这确实是一个协变函子。

\para 称$I$是一个滤相的偏序集,即对于任意的$i$, $j\in I$都存在$k\in I$使得$k\leq i$和$k\leq j$同时成立。称$I$是一个定向的偏序集,即对于任意的$i$, $j\in I$都存在$k\in I$使得$i\leq k$和$j\leq k$同时成立。

很容易看到,如果偏序集$I$是滤相的(定向的),那么$I^\circ$就是定向的(滤相的),反之亦然。

作为例子,考虑拓扑空间中所有非空开集按照包含构成的偏序集,即$U\leq V$当且仅当$U\subset V$,那么这个偏序集是定向的,但不是滤相的,因为对于任意的$U$和$V$总有$U\leq U\cup V$和$V\leq U\cup V$成立,但对于不交的$U$和$V$,并不存在非空开集同时包含于他们其中。

同样是拓扑的例子,考虑所有包含$p$的非空开集按照包含构成的偏序集,那么这个偏序集既是滤相的又是定向的。

\para 将滤相的(定向的)偏序集$I$看作一个范畴,对任意的范畴$\mathcal{C}$,如果一个$I$-图$D$,$D:I\to \mathcal{C}$是协变函子,则这称为一个$C$上的一个{逆系统}({定向系统})。逆系统一般谈论极限$\varprojlim_{i\in I} D(i)$,而定向系统一般谈论$\varinjlim_{i\in I} D(i)$,为了思考这个原因,我们考虑如下交换图
\[
	\xymatrix{
		&A \ar[dl]_{\lambda_{i}}\ar[dr]^{\lambda_{j}}\ar[dd]^{\lambda_{k}}&\\
		D(i)&&D(j)\\
		&D(k)\ar[ul]^{D(k\leq i)}\ar[ur]_{D(k\leq j)}&
	}
	\quad
% \xymatrix{
% 	&A \ar[dl]_{\lambda_{i}}\ar[dr]^{\lambda_{j}}\ar[dd]^{\lambda_{k}}&\\
% 	D(i)\ar[dr]_{D(i\leq k)}&&D(j)\ar[dl]^{D(j\leq k)}\\
% 	&D(k)&
% }
	\xymatrix{
		&A &\\
		\ar[ur]^{\mu_{i}}D(i)\ar[dr]_{D(i\leq k)}&&D(j)\ar[ul]_{\mu_{j}}\ar[dl]^{D(j\leq k)}\\
		&D(k)\ar[uu]_{\mu_{k}}&
	}
\]
左边是对逆系统考虑锥形,右边是对定向系统考虑余锥形。从左边来看,如果$i\leq j\leq k$成立,则$D(k)$构成了一个锥形的顶点,当我们考虑极限的时候,这时候极限就应该表现得像那些“极小”的元素$D(k)$一样。而且如果$A$是极限,那么箭头$A\to D(k)$也构成了唯一分解。类似地考虑右边的图,那些“极大”的元素$D(k)$构成了余锥形的顶点,如果$A$是余极限,那么箭头$D(k)\to A$也构成了唯一分解。

\para 考虑$I$是滤相的(定向的),而$D$是协变函子,那么$I$-图$D$是逆系统(定向系统)。考虑$I$是滤相的(定向的),而$D$是反变函子,那么$I^\circ$-图$D^\circ$是定向系统(逆系统)。

\para 设$I$是一个偏序集,如果$i\in I$有,$i>j$对所有$j\in I$不成立,或者说,与可以比较的元素$j\in I$都有$i\leq j$,则称$i$是$I$的一个极小元。如果$i\leq j$对所有$j\in I$都成立,则称$i\in I$是$I$的最小元。

在滤相的偏序集中,极小元和最小元等价。最小元是极小的,这是显然的。反之,因为如果存在极小元$i$,那么任取一个$j\in I$都存在一个$k\in I$使得$k\leq i$和$k\leq j$都成立,然而$i$是极小元,所以$i=k$,所以$i\leq j$.

同理我们可以定义极大元与最大元,在定向的偏序集中,此二者等价。

更广义地,对于一个偏序集$I$的子集$J$,如果对于任意的$i\in I$,都存在$j\in J$使得$j\leq i$,则称$J$和$I$是共尾的。显然,滤相的偏序集中的极小元构成的单点集就和原来的偏序集共尾。

\pro 设$I$-图$D$是一个逆系统,如果$I$存在一个与其共尾的子集$J$,则$J$-图$D$是一个逆系统,且$\varprojlim_{i\in I} D(i)=\varprojlim_{i\in J} D(i)$. 对偶地,设$I$-图$D$是一个定向系统,如果$I^\circ $存在一个与其共尾的子集$J^\circ $,则$J$-图$D$是一个定向系统,且$\varinjlim_{i\in I} D(i)=\varinjlim_{i\in J} D(i)$. \rule{2mm}{2mm}

所以对于逆系统,如果存在极小元,那么极小元就是它的极限,对偶地,对于定向系统,如果存在极大元,那么极大元就是它的对偶极限。这符合我们上面的直观。

\para 一个拓扑空间$X$能被看成一个范畴,对象取作他的所有开集,而态射取作
\[
	\Hom_{X}(U,V)=\begin{cases}
	\bigl\{i^U_V:U\hookrightarrow V\bigr\}&\text{, if }U\subset V\text{;}\\
	\varnothing&\text{, otherwise}.
	\end{cases}
\]

考虑拓扑空间$X$中一个非空开集族$\mathfrak{B}$,对于任意的两个$U$, $V\in \mathfrak{B}$,都存在一个$W\in \mathfrak{B}$使得$U\cup V\subset W$成立,这样的$\mathfrak{B}$按照包含构成了定向的偏序集。现在定义函子$i:\mathfrak{B}\to X$通过$i(U)=U$以及$i(U\leq V)=i^U_V:U\hookrightarrow V$,它显然成立复合$i(U\leq W)=i(V\leq W)\circ i(U\leq V)$,所以这个$\mathfrak{B}$-图$i$是一个定向系统。可以看到它的余极限$\varinjlim_{U\in \mathfrak{B}} U$实际上就是$\bigcup_{U\in \mathfrak{B}} U$,而态射族就是$i_U:U\hookrightarrow \bigcup_{U\in I}$.

同样考虑拓扑空间$X$中一个包含一个点$p\in X$的所有非空开集构成的族$\mathfrak{B}$,它按照包含构成一个滤相的偏序集。同样可以定义函子$i:\mathfrak{B}\to X$通过$i(U)=U$以及$i(U\leq V)=i^U_V:U\hookrightarrow V$,同样显然成立复合$i(U\leq W)=i(V\leq W)\circ i(U\leq V)$,所以这个$\mathfrak{B}$-图$i$是一个逆系统。

但它的极限$\varprojlim_{U\in \mathfrak{B}} U$就不一定存在,因为类比于上一个例子,它的极限应该类似于所有$\mathfrak{B}$中元素的交,但这不一定是一个开集。

\para 设我们有一个$X$上的$\mathcal{K}$-预层$\calf$,他是$X\to \mathcal{K}$的一个反变函子。赋予$X$一个偏序,此时函数$U\leq V$即$U\hookrightarrow V$. 那么$X$的那些包含$p\in X$的非空开集构成的子集族$\mathfrak{B}$继承了偏序结构,也是一个范畴,而预层$\calf$限制在$\mathfrak{B}$给出了反变函子$\mathfrak{B}\to \mathcal{K}$.

由于$\mathfrak{B}$是滤相的偏序集,而$\calf$是反变函子,因此$\mathfrak{B}^\circ$-图$\calf^\circ$是定向系统,进而我们会考虑余极限
\[
	\varinjlim_{U\in \mathfrak{B}^\circ} \calf^\circ(U)=\varinjlim_{U\in \mathfrak{B}^\circ} \calf(U),
\]
如果这个余极限存在,那么就称为预层$\calf$在点$p$处的纤维,记作$\calf_p$.

% \para 考虑$X$的一个拓扑基,它按照包含也构成一个偏序集,考虑它的一个子族$\mathfrak{B}$,使得每一个元素$V\in \mathfrak{B}$都有$V\subset U$,他继承了来自于拓扑基的偏序结构。

% 设我们有一个$X$上的$\mathcal{K}$-预层$\calf$,他是$X\to \mathcal{K}$的一个反变函子。赋予$X$一个偏序,此时函数$U\leq V$即$U\hookrightarrow V$. 那么$X$的那些包含$p\in X$的非空开集构成的子集族$\mathfrak{B}$继承了偏序结构,也是一个范畴,而预层$\calf$限制在$\mathfrak{B}$给出了反变函子$\mathfrak{B}\to \mathcal{K}$.

\theo 在左$R$-模范畴,任意的极限与余极限存在。 \rule{2mm}{2mm}