\renewcommand\chapterimg{../Pictures/5.png}

\chapter{模(二)}

\section{准素分解}

\para 考虑一个$R$-模$M$,如果对$r\in R$存在一个非零的$m\in M$使得$rm=0$,则这个$r$称为$M$的一个零因子。对固定的$m\in M$,使得$rm=0$的$r\in R$的集合记作$\ann(m)$. 这是一个$R$中的理想,因为$ram=r0=0$对$a\in \ann(m)$成立。设$S$是$M$的一个子集,定义$\ann(S)=\bigcap_{m\in S}\ann(m)$.

设$M$是一个$R$-模,设$\mathfrak{a}\subset \ann(M)$是$R$中的一个理想,则$M$可以看成一个$R/\mathfrak{a}$-模,因为$(r+\mathfrak{a})m=rm$. 特别地,$M$是一个$R/\ann(M)$-模,此时,如果$rm=0$,则可以推出$r$和$m$中有一个零。

\para $\ann(m)$中的那些素理想,称之为$M$的associated primes,定义$\Ann_R(M)$为所有associated primes的集合,如果下标$R$是清楚的,我们会略去它。

当然有可能所有的$\ann(m)$都不是素理想,但是如果$R$是一个Noether环,则下面的引理保证了$\Ann(M)$非空。

\lem 设$\mathcal{P}$是所有形如$\ann(m)$的理想按照包含构成的一个偏序集,则$\mathcal{P}$中的极大元都是素理想。

\proof
	设$\ann(m)$是一个极大元,再设$r\notin \ann(m)$,即$rm\neq 0$. 很清楚,$\ann(m)\subset \ann(rm)$,所以由极大性可知$\ann(rm)=\ann(m)$. 现在如果$rs\in \ann(m)$但$r\notin \ann(m)$,则$srm=0$可知$s\in \ann(rm)=\ann(m)$,所以$\ann(m)$是一个素理想。
\qed

如果$R$是一个Noether环,则它的理想构成的非空子集一定存在极大元,这就保证了$\Ann(M)$非空。并且,如果$r$是$M$的一个零因子,含有$r$的那个极大的$\ann(m)$属于$\Ann(M)$,这就是说,任意的零因子必然属于$\Ann(M)$中的某个元素,或者如下包含关系
\[
	\bigcup_{0\neq m\in M}\ann(m)\subset \bigcup_{\mathfrak{a}\in \Ann(M)}\mathfrak{a}. 
\]

从这里开始,我们假设这节下面出现的环都是Noether环,除非特别申明。

\lem 设$M$是一个$R$-模,如果$\ann(m)$是素理想,则对$M$的子模$Rm$的任意非零子模$N$有$\ann(N)=\ann(m)$.

\proof
	显然,$\ann(m)\subset \ann(N)$. 由于有$R$-模同构$\psi: Rm\to R/\ann(m)$,所以$Rm$的任意非零子模$N$对应着$R/\ann(m)$的非零理想$\psi(N)$. 如果$r\in \ann(N)$,则$r\psi(N)=\psi(rN)=0$,因为$\psi(N)$非零且$R/\ann(m)$是整环,所以$r$在$R/\ann(m)$中的像为零,即是说$r\in \ann(M)$. 所以$\ann(N)\subset \ann(m)$.
\qed

\pro 如果我们有$R$-模的正合列$0\to M'\to M\to M''\to 0$,则我们有包含关系
\[
	\Ann(M')\subset \Ann(M)\subset \Ann(M')\cup \Ann(M'').
\]

\proof
	由于单同态并不会改变$\Ann$,所以可以将$M'$看成$M$的子模,所以第一个包含显然。对第二个包含,任取$\mathfrak{a}\in \Ann(M)-\Ann(M')$,我们要证明$\mathfrak{a}\in \Ann(M')$. 设$\mathfrak{a}=\ann(m)$,其中$m\in M$,由$m$生成的子模$Rm$同构于$R/\mathfrak{a}$. 考虑$N=Rm\cap M'$,他是$Rm$的子模,如果$N$非零,因为$\mathfrak{a}\notin \Ann(M')$,$\mathfrak{a}N\neq 0$,因此$\mathfrak{a}\not\subset \ann(N)=\ann(m)=\mathfrak{a}$,矛盾,所以只可能$Rm\cap M'=0$. 在这种情况下,$Rm$同构于他在$M''$中的像,所以$\mathfrak{a}\in \Ann(M'')$.
\qed

考虑$M=M'\oplus M''$的情况,由于$M'$和$M''$都有到$M$的单同态,所以$\Ann(M')\subset \Ann(M)$以及$\Ann(M'')\subset \Ann(M)$,继而他给出了$\Ann(M)=\Ann(M')\cup \Ann(M'')$.

\para 设$N$是$M$的一个子模,如果$\Ann(M/N)$只有一个元素,则称呼$N$是$M$的一个准素子模,如果$\Ann(M/N)=\{\mathfrak{p}\}$,则称呼$N$是$\mathfrak{p}$-准素的。

如果设$N_1$和$N_2$都是$\mathfrak{p}$-准素的,则$N_1\cap N_2$也是. 事实上,考虑单同态$M/(N_1\cap N_2)\to M/N_1\oplus M/N_2$,因此$\Ann(M/(N_1\cap N_2))\subset \Ann(M/N_1)\cup \Ann(M/N_2)=\{\mathfrak{p}\}\cup \{\mathfrak{p}\}=\{\mathfrak{p}\}$. 但是由于$\Ann(M/(N_1\cap N_2))$非空,所以$\Ann(M/(N_1\cap N_2))=\{\mathfrak{p}\}$.

经过有限次归纳可以得到:有限个$\mathfrak{p}$-准素子模的交依然是一个$\mathfrak{p}$-准素子模。

\lem 设$R$是Noether环,$M$是有限生成$R$-模,则$M$的不可约子模$N$是准素子模。

\proof
	如果$\Ann(M/N)$中至少存在两个素理想$\pp_1=\ann(\bar{m}_1)$和$\pp_2=\ann(\bar{m}_2)$,那么在$M/N$中有子模$R\bar{m}_1\cong R/\pp_1$和$R\bar{m}_2\cong R/\pp_2$. 如果$R\bar{m}_1\cap R\bar{m}_2$非零,则$R\bar{m}_1\cap R\bar{m}_2$作为$R\bar{m}_1$的非零子模,$\ann(R\bar{m}_1\cap R\bar{m}_2)=\pp_1$,作为$R\bar{m}_2$的非零子模,$\ann(R\bar{m}_1\cap R\bar{m}_2)=\pp_2$,矛盾,于是推出$R\bar{m}_1\cap R\bar{m}_2=\{\bar{0}\}$. 回到$M$中的原像,我们就可以知道$N$可以写成两个$M$的子模的交,这与$N$不可约矛盾。
\qed

准素子模的重要性来自于下面这个定理,他被称为准素分解。

\theo 设$R$是Noether环,$M$是有限生成$R$-模,如果$N$是$M$的一个真子模,那么他是有限个准素子模的交。

\proof
	由于Noether环上的有限生成模是Noether模,回忆Proposition \eqref{irrde},$N$可以写成$M$中的有限个不可约子模的交的形式,而每一个不可约子模都是准素子模。
\qed

\section{有限长模}

\para 设$M$是一个$R$-模,如果$M$没有非平凡子模,则称$M$是一个单模。任取非零元$x\in M$,考虑子模$Rx\subset M$,如果$M$是单模,我们只有两个可能,$Rx=\{0\}$或者$Rx=M$. 但是由于$x=1x\in Rx$非零,所以$Rx=M$. 于是任意的一个单模都是由一个元素生成的。反过来,单元素生成的模是不是单模?答案是不一定的。

设$Rx$是一个$R$-单模,考虑自然的$R$-模同态$f:r\mapsto rx$,于是$f^{-1}(0)=\ann(x)$,由同构基本定理,$Rx\cong R/\ann(x)$. 由于$Rx$是单模,所以$Rx$没有非平凡子模,再由同构,$R/\ann(x)$没有非平凡理想,因此$R/\ann(x)$是一个域,于是$\ann(x)$是极大理想。反过来也对,如果$\ann(x)$是极大理想,则$Rx$是单模。

考虑$R=k$是一个域,设$x\in M$是一个非零元,此时$\ann(x)$作为$k$的理想只可能是$(0)$,否则$x=1x=0$. 域中唯一的极大理想就是$(0)$,于是$kx$一定是一个单模。

\para 设$M$是一个$R$-模,如果它同时是Artin模和Noether模,那他就被称为一个有限长模。对于严格递增链$0=M_0\subset M_1\subset \cdots\subset M_n=M$,$n$被称为这条链的长度。有限长模的任意链都是有限的,因为向上有Noether条件,向下有Artin条件。设有一条严格递增链$M_0\subset M_1\subset \cdots\subset M_n \subset \cdots $,任取$i\in \mathbb{Z}^+$,如果$M_{i}/M_{i-1}$都是单模,则这样一条链被称为$M$的一条合成列。有限长模的合成列一定是有限长的。同时,合成列当然是不一定唯一的。

给定任意一个$R$-模的包含$N\subset M$,考虑任意的中间模$N'$,即满足$N\subset N'\subset M$的$R$-模$N'$,则他在商模$M/N$中的像$N'/N$是$M/N$的一个子模。反过来,$M/N$的子模在商映射下的原像就是一个中间模。因此如果$N\subset M$中间不能再插入任意的一个子模等价于$M/N$中没有非平凡子模,即$M/N$是一个单模。于是合成列的定义就是说这是一个不能再插入一些子模让它变得更长的链,换而言之,极大链。

\pro 有限长模的合成列的长度是一定的,因此可以定义有限长模的长度为其中任意一条合成列的长度。如果一个模不是有限长,则定义它的长度为无穷。

以长度的视角,单模等价于说模的长度为1.

\proof
	记$l(M)$是$M$所有的合成列中最短那条的长度。设$N$是$M$的一个真子模,我们先证明$l(N)<l(M)$.

	给定一个合成列$0=M_0\subset M_1\subset \cdots\subset M_n=M$,记$N$的子模$N\cap M_i$为$N_i$,进而有$N$的子模链$0=N_0\subset N_1\subset \cdots\subset N_n=N$. 由同构\eqref{modiso1}
	\[
	N_i/N_{i-1}=(N\cap M_i)/(N\cap M_{i-1})\cong (M_{i-1}+N\cap M_i)/M_{i-1},
	\]
	显然,$M_{i-1}+N\cap M_i$是$M_i$的一个子模,由于$M_i/M_{i-1}$是一个单模,于是$M_{i-1}+N\cap M_i=M_{i-1}$或者$M_i$,前者意味着$N\cap M_i\subset M_{i-1}$或者$N_i/N_{i-1}=\{0\}$再或者$N_i=N_{i-1}$,后者意味着$N_i/N_{i-1}\cong M_i/M_{i-1}$是一个单模。在$0=N_0\subset N_1\subset \cdots\subset N_n=N$中去掉那些相等的项,我们就得到了$N$的一条合成列,长度小于等于$l(M)$. 由于$l(N)$是$N$所有的合成列中最短那条的长度,所以$l(N)\leq l(M)$. 然后设$l(N)=l(M)$,这意味着对每一个$i$都成立$M_{i-1}+N_i=M_{i-1}+N\cap M_i=M_i$. 对$i=1$,可以推知$N_1=M_1$,然后对$i=2$有$M_2=M_1+N_2=N_1+N_2=N_2$,经过有限归纳,可以得知$N=N_n=M_n=M$,这和$N$是$M$的一个真子模矛盾。于是$l(N)<l(M)$.

	然后任取一条完全递增链$0=M_0\subset M_1\subset \cdots\subset M_k=M$,我们有
	\[
	l(0)<l(M_1)<\cdots<l(M_k)=l(M),
	\]
	由于$l(0)=0$,所以$l(M)\geq k$. 最后考虑任意一条合成列,我们有$l(M)$大于等于它的长度,又$l(M)$是$M$所有的合成列中最短那条的长度,因此$M$的任意合成列的长度都是$l(M)$.
\qed

\pro 一个$R$-模$M$是有限长模当且仅当它存在一个有限长合成列。

\proof
	$(\Rightarrow)$部分是自然的。反过来,如果$M$不是一个有限长模,则它或者Artin模或者Noether模,于是存在向上不稳定的严格递增链或者向下不稳定的严格递减链,那一个都会使得$M$的合成列长度不有限,于是逆否得证。
\qed

\para 设$k$是一个域,定义$k$-矢量空间$V$的维度$\dim_k(V)$为$V$的长度。维度是线性代数理论中一个很重要的不变量,以后我们会看到维度的其他等价定义。

设$M$是域$k$上的有限生成模,它是自由模,不妨直接写作$k^n$,一组自由生成元写作$\{1_1$, $\cdots$, $1_n\}$,那么对小于$n$的正整数$m$,$\{1_1$, $\cdots$, $1_m\}$自由生成了它的一个子模$k^m$. 于是我们有链
\[
	0\subset k^1\subset  k^2 \subset \cdots \subset k^n,
\]
其中$k^m/k^{m-1}\cong k$,这是一个合成列。因此$\dim_k(k^n)=n$. 

由于自由模的自由生成元的个数与生产元的选取无关,所以有限维矢量空间的维度也就是自由生成元的个数。

如果$k$-模$M$不是有限生成的,则$\dim_k(M)=\infty$. 实际上,给定一个$x_1\in M$,他生成的子模$\langle x_1\rangle=kx_1\cong k^1$,然后再从$M-\langle x_1\rangle$选出一个非零元$x_2$,得到$\langle x_1,x_2\rangle\cong k^2$,接着再从$M-\langle x_1,x_2\rangle$选出一个非零元$x_3$,如是进行下去,我们就得到了一个严格递增升链
\[
	0\subset k^1\subset k^2\subset \cdots \subset k^n\subset \cdots,
\]
按构造$k^m/k^{m-1}\cong k$,于是这是一个无限长合成列。

所以,一个$k$-矢量空间是有限生成的当且仅当他是有限维的。

\pro 如果$k$-矢量空间$V$是Artin或者Noether模,则$V$是有限维的。即在矢量空间上,Artin性等价于Noether性等价于有限维。

\proof
	假设$V$不是有限维的,则我们可以构造出一条合成列,
	\[
	0\subset \langle x_1\rangle\subset \langle x_1,x_2\rangle\subset \cdots \subset \langle x_1,\,\cdots\!,x_n\rangle\subset \cdots,
	\]
	因此这自然不是Noether模。然后考察
	\[
		\bigoplus_{i=1}^\infty kx_i \supset \bigoplus_{i=2}^\infty kx_i \supset \bigoplus_{i=3}^\infty kx_i\supset \cdots,
	\]
	因此这也不是Artin模。逆否即得证。
\qed


\theo 设$M$是一个有限长$R$-模,长度为$n$. 对于$M$的任意合成列$0=M_0\subset M_1\subset \cdots\subset M_n=M$,$M_i/M_{i-1}=\mathfrak{m}_i$都是极大理想,但集合$I=\{\mathfrak{m}_i\,:\,1\leq i\leq n\}$与选取的合成列无关,并且有同构
\[
	M\cong \bigoplus_{\mathfrak{m}\in I} M_\mathfrak{m}.
\]

\proof
	我们局部化处理,因为如果在每一个极大理想局部同构,则整体也同构。任取$R$的一个极大理想$\mathfrak{n}$. 先考虑长度为1的情况,此时$M$是一个单模,于是$M\cong R/\mathfrak{m}$,其中$\mathfrak{m}$是一个极大理想,不妨就直接写成$M=R/\mathfrak{m}$. 如果$\mathfrak{n}=\mathfrak{m}$,此时$\mathfrak{n}$外的元素在$R/\mathfrak{m}=R/\mathfrak{n}$中都是可逆元,于是$M_{\mathfrak{n}}=(R/\mathfrak{m})_{\mathfrak{n}}=R/\mathfrak{m}=M$. 

	如果$\mathfrak{n}\neq \mathfrak{m}$,因此$\mathfrak{m}$是极大理想,所以$\mathfrak{m}\not\subset \mathfrak{n}$以及$\mathfrak{n}\not\subset \mathfrak{m}$,因此$\mathfrak{m}_\mathfrak{n}=R_\mathfrak{n}$. 于是$M_\mathfrak{n}=(R/\mathfrak{m})_\mathfrak{n}=R_\mathfrak{n}/\mathfrak{m}_\mathfrak{n}=0$. 由$(M_\mathfrak{m})_\mathfrak{n}=R_\mathfrak{n}\otimes R_\mathfrak{m} \otimes M$,其中张量积都是$R$-模的张量积. 于是有同构$(M_\mathfrak{m})_\mathfrak{n}\cong R_\mathfrak{n}\otimes M\otimes R_\mathfrak{m}$,任取一个$a\in \mathfrak{m}$但$a\not\in \mathfrak{n}$,于是任取$(r/p)\otimes m \in R_\mathfrak{n}\otimes M$有
	\[
		\frac{r}{p}\otimes a=\frac{ra}{pa}\otimes m =\frac{r}{pa}\otimes (am),
	\]
	因为$a\in \mathfrak{m}$,但$m\in M=R/\mathfrak{m}$,所以$am=0$. 所以$R_\mathfrak{n}\otimes M=0$也就推出了$(M_\mathfrak{m})_\mathfrak{n}=0=M_\mathfrak{n}$. 由于已经遍历了极大理想,都有$(M_\mathfrak{m})_\mathfrak{n}=M_\mathfrak{n}$,于是$M_\mathfrak{m}=M$.

	现在考虑一般的情况,设$0=M_0\subset M_1\subset \cdots\subset M_n=M$是一个合成列。对任意一个极大理想$\mathfrak{m}$进行局部化,我们有
	\[
	0=(M_0)_\mathfrak{m}\subset (M_1)_\mathfrak{m}\subset \cdots\subset (M_n)_\mathfrak{m}=M_\mathfrak{m}.
	\]
	由于$M_i/M_{i-1}$是单模,长度为1,所以当$\mathfrak{m}=\ann(M_i/M_{i-1})$时,$(M_i/M_{i-1})_\mathfrak{m}\cong M_i/M_{i-1}\cong R/\mathfrak{m}$,否则$(M_i/M_{i-1})_{\mathfrak{m}}=(M_i)_{\mathfrak{m}}/(M_{i-1})_{\mathfrak{m}}=0$给出了$(M_i)_{\mathfrak{m}}=(M_{i-1})_{\mathfrak{m}}$. 如果$\mathfrak{m}$在$I$中不出现,则上面的推理给出$M_\mathfrak{m}=(M_n)_\mathfrak{m}=(M_0)_\mathfrak{m}=0$. 反过来,只有当$\mathfrak{m}$在$I$中出现,在链中将重复项删去就得到了$M_\mathfrak{m}$的一个合成列
	\[
	0=(M_0)_\mathfrak{m}\subset (M_{i_1})_\mathfrak{m}\subset \cdots\subset (M_{i_k})_\mathfrak{m}=M_\mathfrak{m}.
	\]
	任取单模$(M_{i_k})_\mathfrak{m}/(M_{i_{k-1}})_\mathfrak{m}$都有
	\[
	(M_{i_k})_\mathfrak{m}/(M_{i_{k-1}})_\mathfrak{m}=(M_{i_k}/M_{i_{k-1}})_\mathfrak{m}=M_{i_k}/M_{i_{k-1}}\cong R/\mathfrak{m}.
	\]

	取不同于$\mathfrak{m}$的另一个极大理想$\mathfrak{n}$,然后再局部化,可以得到另一条链
	\[
	0=((M_0)_\mathfrak{m})_\mathfrak{n}\subset ((M_{i_1})_\mathfrak{m})_\mathfrak{n}\subset \cdots\subset ((M_{i_k})_\mathfrak{m})_\mathfrak{n}=(M_\mathfrak{m})_\mathfrak{n},
	\]
	由于$((M_{i_k})_\mathfrak{m})_\mathfrak{n}/((M_{i_{k-1}})_\mathfrak{m})_\mathfrak{n}=((M_{i_k})_\mathfrak{m}/(M_{i_{k-1}})_\mathfrak{m})_{\mathfrak{n}}\cong R/\mathfrak{m}=((M_{i_k}/M_{i_{k-1}})_\mathfrak{m})_\mathfrak{n}=((R/\mathfrak{m})_{\mathfrak{m}})_\mathfrak{n}=0$,所以推出了$(M_\mathfrak{m})_\mathfrak{n}=0$. 正如长度为1的情况一般。

	现在来看命题,有限长模$M$的一个极大理想$\mathfrak{m}$是否属于$I$,只取决于$M_\mathfrak{m}$是否为零,而与合成列的选取无关。现在只剩下那个同构了,设$i_\mathfrak{m}:M\to M_{\mathfrak{m}}$是局部化诱导的同态,进而定义$\alpha=\sum_{\mathfrak{m}\in I}i_{\mathfrak{m}}:M\to \bigoplus_{\mathfrak{m}\in I} M_{\mathfrak{m}}$. 这检验这是一个同构,只要对每一个极大理想$\mathfrak{n}$进行局部化即可,而这由已知的,对不同的两个极大理想$\mathfrak{m}$和$\mathfrak{n}$有$(M_\mathfrak{m})_\mathfrak{n}=0$,以及如果$\mathfrak{m}\not\in I$则$M_\mathfrak{m}=0$这两点是清楚的。
\qed

\pro 设$M$是一个有限长模,而$\mathfrak{m}$是一个极大理想,则$M=M_{\mathfrak{m}}$当且仅当存在一个正整数$n$使得$\mathfrak{m}^nM=0$.

\proof
	任取另一个极大理想$\mathfrak{n}$,那么$\mathfrak{m}$存在一个不属于$\mathfrak{n}$中的元素$a$,这个元素在$M_\mathfrak{n}$中是可逆的,但是$a^n\in \mathfrak{m}^n$给出$a^nM_\mathfrak{n}=0$,由可逆性就得到了$M_\mathfrak{n}=0$. 所以只有可能$M_\mathfrak{m}\neq 0$,由上面的命题,$M\cong M_\mathfrak{m}$.

	反过来,由于$M\cong M_\mathfrak{m}$,所以对于任意的合成列$0=M_0\subset M_1\subset \cdots\subset M_n=M$都有$M_i/M_{i-1}\cong R/\mathfrak{m}$或者$\ann(M_i/M_{i-1})=\mathfrak{m}$再或者$\mathfrak{m}M_{i}\subset M_{i-1}$,于是
	\[
	\mathfrak{m}^nM=\mathfrak{m}^nM_{n}\subset \mathfrak{m}^{n-1}M_{n-1}\subset \cdots \subset \mathfrak{m}M_1\subset M_0=0,
	\]
	因此$\mathfrak{m}^nM=0$.
\qed

\para 由于$R$本身就是一个$R$-模,如果$R$是有限长$R$-模,已经知道,在$R$的任何合成列中出现的极大理想不依赖于合成列的选取,个数不大于$R$的长度。记这些极大理想的集合为$I$,作为$R$-模成立同构$R\cong\bigoplus_{\mathfrak{m}\in I}R_\mathfrak{m}$.

\lem 回忆\eqref{oka}里面的定义,这里再指出一个理想族是Oka理想族,这个理想族由$R$中所有使得$R/\mathfrak{a}$是有限长$R$-模的理想$\mathfrak{a}$组成。

\proof 
	给定$\mathfrak{a}$和$a\in R$,如果$R/(\mathfrak{a}+(a))$和$R/(\mathfrak{a}:a)$都是有限长$R$-模,要证明$R/\mathfrak{a}$是有限长$R$-模。可以假设$a\not\in \mathfrak{a}$,否则$R/(\mathfrak{a}+(a))=R/\mathfrak{a}$. 首先注意到短正合列
	\[
		0\to R/(\mathfrak{a}:a)\xrightarrow{\times a} R/\mathfrak{a} \to R/(\mathfrak{a}+(a))\to 0,
	\]
	其中第二个箭头是乘以$a$,是单射,因为如果$(r+(\mathfrak{a}:a))a=ar+\mathfrak{a}\in R/\mathfrak{a}$为零,则$r\in (\mathfrak{a}:a)$,第三个箭头是商环的泛性质诱导的,显然是满射。

	检验正和性最好回到$R$中,由于商同态$R\to R/(\mathfrak{a}:a)$是一个满射,所以$R/(\mathfrak{a}:a)\xrightarrow{\times a} R/\mathfrak{a}$的像就是$R\xrightarrow{\times a} R/\mathfrak{a}$的像,也就是$(a)$在$R/\mathfrak{a}$的像。另一方面,考虑$R/\mathfrak{a} \to R/(\mathfrak{a}+(a))$的核,任取$r+\mathfrak{a}\in R/\mathfrak{a}$,如果他在$R/(\mathfrak{a}+(a))$中为零,则$r\in (a)$,因此$R/\mathfrak{a} \to R/(\mathfrak{a}+(a))$的核也是$(a)$在$R/\mathfrak{a}$的像,正和性得证。

	由于短正合列正中一个模是Noether/Artin模当且仅当两边的模是Noether/Artin模,所以从正合列两边都是有限长模,就得到了$R/\mathfrak{a}$是有限长模。因此$R$中所有使得$R/\mathfrak{a}$是有限长$R$-模的理想$\mathfrak{a}$构成的理想族是Oka理想族。
\qed

\pro 一个Noether环$R$,如果所有的素理想都是极大理想,则他是一个有限长$R$-模。

\proof
	由于是Noether环,非空理想族满足极大性条件,考虑$R$中所有使得$R/\mathfrak{a}$是有限长$R$-模的理想$\mathfrak{a}$构成的Oka理想族$\mathcal{F}$. 假设$R$不是有限长模,则$(0)$就在其中,因此$\mathcal{F}^c$非空,由极大性条件,$\mathcal{F}^c$中存在极大元$\mathfrak{p}$,并且由$\mathcal{F}$是一个Oka理想族,$\mathfrak{p}$是一个素理想,进而由条件是一个极大理想。由于$R/\mathfrak{p}$是一个单模,长度为1. 但是$\mathfrak{p}\not\in \mathcal{F}$,所以$R/\mathfrak{p}$不是有限长模,矛盾。
\qed

\pro 一个环$R$如果是Artin环,则$R$的每一个素理想都是极大理想且$R$极大理想数目有限。

\proof
	令$\mathfrak{p}$是$R$的一个素理想,则$R/\mathfrak{p}$是一个Artin整环。取非零$x\in R/\mathfrak{p}$,有降链$(x)\supset (x^2)\cdots$,由于是Artin整环,所以存在正整数$n$使得$(x^n)=(x^{n+1})$,因此$x^n=x^{n+1}y$,由整环的消去律得到$xy=1$. 所以$R/\mathfrak{p}$是一个域。

	剩下我们考虑所有有限个极大理想的交构成的理想族,由极小条件,这个理想族有极小元$\mathfrak{m}_1\cap \cdots \cap \mathfrak{m}_n$. 任取极大理想$\mathfrak{m}$,由于$\mathfrak{m}\cap\mathfrak{m}_1\cap \cdots \cap \mathfrak{m}_n\subset \mathfrak{m}_1\cap \cdots \cap \mathfrak{m}_n$,因此极小性给出$\mathfrak{m}\cap\mathfrak{m}_1\cap \cdots \cap \mathfrak{m}_n=\mathfrak{m}_1\cap \cdots \cap \mathfrak{m}_n$,故$\mathfrak{m}_1\cap \cdots \cap \mathfrak{m}_n\subset \mathfrak{m}$. 由Proposition \eqref{primeau},存在$i$使得$\mathfrak{m}_i\subset \mathfrak{m}$,由于是极大理想,所以$\mathfrak{m}=\mathfrak{m}_i$.
\qed

\pro 一个环$R$是Artin环,当且仅当,$R$是一个Noether环且所有素理想都是极大理想。

\proof
	一个环$R$是Noether环且所有素理想都是极大理想,我们已经看到这是一个有限长$R$-模,进而是一个Artin环。反过来,我们也已经看到,Artin环的所有素理想都是极大理想。

	Artin性就是极小性条件。考虑$R$所有极大理想的任意有限乘积构成的理想族,它存在极小元$\mathfrak{a}$. 由于$\mathfrak{a}^2\subset \mathfrak{a}$也是有限个极大理想的乘积,由极小性$\mathfrak{a}^2=\mathfrak{a}$. 类似地,任取极大理想$\mathfrak{m}$,有$\mathfrak{m}\mathfrak{a}=\mathfrak{a}$. 这就是说,$\mathfrak{a}$中的元素在任意的极大理想$\mathfrak{m}$中。我们下面证明$\mathfrak{a}$只能为零。

	如果$\mathfrak{a}$非零,在所有使得$\mathfrak{b}\mathfrak{a}\neq 0$的理想$\mathfrak{b}$中选择极小的那个,我们有$(\mathfrak{b}\mathfrak{a})\mathfrak{a}=\mathfrak{b}\mathfrak{a}^2=\mathfrak{b}\mathfrak{a}\neq 0$,于是$\mathfrak{b}\mathfrak{a}\subset \mathfrak{b}$,由极小性,$\mathfrak{b}\mathfrak{a}=\mathfrak{b}$. 必然存在一个$b\in\mathfrak{b}$使得$b\mathfrak{a}\neq 0$,于是由极小性$\mathfrak{b}=(b)$,因此$\mathfrak{b}\mathfrak{a}=\mathfrak{b}$,所以存在一个$a\in \mathfrak{a}$使得$ab=b$,或者$(a-1)b=0$. 由于$a$在任意的极大理想中,所以$a-1$是可逆的,故$b=0$. 矛盾。

	现在已经知道,$0$可以写成有限个极大理想的乘积,不妨写作$0=\mathfrak{m}_1\cdots \mathfrak{m}_n$. 考虑链
	\[
	0=\mathfrak{m}_1\cdots \mathfrak{m}_n\subset \mathfrak{m}_1\cdots \mathfrak{m}_{n-1}\subset \cdots\subset \mathfrak{m}_1\subset R,
	\]
	有$(\mathfrak{m}_1\cdots \mathfrak{m}_{k-1})/(\mathfrak{m}_1\cdots \mathfrak{m}_{k})$是一个$R/\mathfrak{m}_{k}$-矢量空间。由于$(\mathfrak{m}_1\cdots \mathfrak{m}_{k-1})/(\mathfrak{m}_1\cdots \mathfrak{m}_{k})$作为一个Artin模子模的商模也是一个Artin模。看作$R/\mathfrak{m}_{k}$-矢量空间时,Artin性与Noether性等价,所以这是一个Noether模。

	考虑正合列
	\[
	0\to \mathfrak{m}_1\cdots \mathfrak{m}_{k}\hookrightarrow\mathfrak{m}_1\cdots \mathfrak{m}_{k-1}\to (\mathfrak{m}_1\cdots \mathfrak{m}_{k-1})/(\mathfrak{m}_1\cdots \mathfrak{m}_{k})\to 0,
	\]
	正中间的Noether性来自于两边的Noether性,所以有限次归纳后,可以得到$R$是一个Noether模。
\qed

\para 结合上面两个命题,一个环$R$是Artin环等价于他是一个有限长$R$-模,也当且仅当他是Noether环且所有的素理想都是极大理想。因此,与模的情况不一样,到了环这里,Artin性就是一个远比Noether性强的一个性质。附带的,还断言了,Artin环的素理想(极大理想)有限。

由于Artin环是一个有限长模,所以可以把有限长模的结构定理用到Artin环上,这里就不表了。

\section{主理想整环上的有限生成模}

\para 在很久之前,我们已经有了一个结论:域上的有限生成模是自由模。这节就将推广这个结论。

\section{微分模}

\para 设$S$是一个环,$M$是一个$S$-模,则一个加法群同态$d:S\to M$被称为一个导子,如果它满足Leibniz法则
\[
	d(fg)=fdg+gdf.
\]
如果$S$是一个$R$-代数,设$\varphi:R\to S$是定义代数的环同态,那么$r\cdot m=\varphi(r)m$定义出了$M$上的一个$R$-模结构。此时,导子$d$如果是$R$-模同态,则称呼$d$是$R$-线性的。$S$到$M$所有的$R$-线性导子构成了一个$R$-模。

由于$d(1\cdot 1)=d(1)+d(1)$,因此对任意的导子有$d(1)=0$. 如果$d$是$R$-线性的,对$r\in R$,我们有$d(r)=rd(1)=0$. 设$s\in S$,则$d(s^n)=ns^{n-1}d(s)$,这个可以用Leibniz法则直接计算得到。

如果$\varphi:R\to S$是一个满同态,如果将$S$看成$R$-代数,则$\varphi(r)=r\cdot 1$,于是从上面的构造有$d(r\cdot 1)=rd(1)=0$,所以此时$\Omega_{S/R}=\{0\}$.

\pro 给定环$R$和$R$-代数$S$,则存在一个$S$-模$\Omega_{S/R}$和一个$R$-线性的导子$d:S\to R$使得任意的$S$-模$M$与$R$-线性导子$\delta: S\to M$都存在唯一的$R$-线性映射$\varphi:\Omega_{S/R}\to M$使得分解
\[
	\delta:S\xrightarrow{d}\Omega_{S/R}\xrightarrow{\varphi}M
\]
成立。

\proof
	考虑集合$\{d(f)\,:\, f\in S\}$生成的自由$S$-模$M$,再考虑其中由
	\[
	d(fg)-fd(g)-gd(f),\quad d(rf)-rd(f),\quad d(f+g)-d(f)-d(g)
	\]
	生成的子模$N$,于是$\Omega_{S/R}$就被构造为$M/N$,而$d:f\mapsto d(f)$很容易检验这是一个$R$-线性导子。

	泛性质的检验从构造是简单的。
\qed

今后,$d(f)$的括号不产生理解上的差异的时候会略去记成$df$. 上面定义的模就被称为$S$的微分模。由泛性质,微分模在同构意义下唯一。

\para 在继续关于微分模的讨论之前,先谈谈他相关的几何概念。在光滑流形$M$上,我们知道,给定一个光滑函数$f:M\to \mathbb{R}$,那么在每一点$f_{*p}:T_pM\to T_{f(p)}\mathbb{R}=\mathbb{R}$将$p$点的切矢量$v$变成了一个数$f_{*p}(v)=v(f)=df(v)(p)$,然后让$p$在流形上跑动起来,这样就得到了$f_*=df$实际上是余切丛的一个光滑截面。

记$\mathcal{C}^\infty(M)$是流形$M$上的光滑函数环,这是一个$\mathbb{R}$-代数,而$\Gamma(T^*M)$是$M$的余切丛的光滑截面构成的集合,他是一个加法群,通过逐点定义数乘,可以赋予一个$\mathcal{C}^\infty(M)$-模结构,此时$f\mapsto f_*$就是一个导子。

由微分模的泛性质,因此存在一个$\mathcal{C}^\infty(M)$-模同态$\Omega_{\mathcal{C}^\infty(M)/\mathbb{R}}\to \Gamma(T^*M)$. 由于$T^*M$与$\Gamma(T^*M)$是等价的对象,所以$\Omega_{\mathcal{C}^\infty(M)/\mathbb{R}}$这里就可以理解成余切丛。他的对偶就可以理解成切丛。

\para 设$S$是一个有限生成$R$-代数,他被$\{f_1,\cdots,f_n\}$生成,那么任意的单项式$f_1^{k_1}\cdots f_n^{k_n}\in S$求导后有
\[
	d(f_1^{k_1}\cdots f_n^{k_n})=\sum_{i=1}^n f_1^{k_1}\cdots d(f_i^{k_i})\cdots f_n^{k_n},
\]
如果$k_i=0$则$d(f_i^{k_i})=0$,如果$k_i>0$则$d(f_i^{k_i})=f_i^{k_i-1}df_i$,所以这个单项式可以由$\{df_i\}$线性组合而言,继而$\Omega_{S/R}$就可以由$\{df_i\}$线性组合。

因此,我们求出了,如果$S=R[x_1,\cdots,x_n]$,则$\Omega_{S/R}=\bigoplus_{i=1}^n Sdx_i$,即由$\{dx_i\}$生成的自由模。

\para 回忆一个$R$-代数$S$实际上是一个三元组$(R,S,\varphi)$,其中$\varphi:R\to S$是一个环同态。考虑所有这样三元组构成的范畴,其中态射$\rho:(R,S,\varphi)\to (R',S',\varphi')$是两个环同态$\rho_R:R\to R'$以及$\rho_S:S\to S'$,它们需要满足交换图
\[
\begin{xy}
	\xymatrix
	{
		R\ar[rr]^{\rho_R} \ar[d]_{\varphi}&&R' \ar[d]^{\varphi'}\\
		S\ar[rr]^{\rho_S}&&S'
	}
\end{xy}
\]
不难检验复合的成立。

由于每个代数都可以考虑它的微分模,那么我们自然可以考虑微分模在上述态射下的变化,为此,我们记$\Omega(S,R,\varphi)$来代替$\Omega_{S/R}$,则上述态射诱导了如下的交换图
\[
\begin{xy}
	\xymatrix
	{
		R\ar[r]^{\varphi}\ar[d]_{\rho_R}&S\ar[rr]^{d} \ar[d]_{\rho_S}&&\Omega(S,R,\varphi) \ar[d]\\
		R'\ar[r]^{\varphi'}&S'\ar[rr]^{d'}&&\Omega(S',R',\varphi')
	}
\end{xy}
\]
没有标记的那个同态来自于微分模的泛性质,因此,我们可以将$\Omega$看成一个代数范畴到模范畴的一个函子。

\para 考虑$R=R'$的情况,此时代数的态射约化成了$S\to S'$的一个环同态,换而言之,即考虑三个环
$R\to S\to T$的情况。

考虑$T$-模$\Omega_{T/R}$和$\Omega_{T/S}$,从构造,他们都可以写成自由$T$-模$M$关于两个子模$N_R$和$N_S$的商模,由于$S$-线性可以推出$R$-线性,所以$N_R\subset N_S$. 从商同态$M\to M/N_S=\Omega_{T/S}$可以诱导出满同态$\psi:\Omega_{T/R}=M/N_R\to \Omega_{T/S}$,具体到元素就是$\psi:dc\mapsto dc$. 

对$S$-模$\Omega_{S/R}$,我们可以标量扩张成一个$T$-模,通过$T\otimes_S \Omega_{S/R}$. 对环同态$\pi:S\to T$. 由于同样都是$R$-线性的,所以通过$\varphi:t\otimes ds\mapsto tds$可以给出$R$-线性的$T$-模同态$\varphi:T\otimes_S \Omega_{S/R}\to \Omega_{T/R}$.

现在我们来求$\ker\psi$,由构造,他就是$N_S/N_R$. $N_S$由所有$d(fg)-df-dg$, $d(fg)-df-dg$和$d(f+g)-df-dg$生成,由于$d(fg)-df-dg$, $d(f+g)-df-dg\in N_R$, 所以$\ker\psi$被$d(st)-sdt=tds$生成元,其中$s\in S$,由于$tds=\varphi(t\otimes ds)$,于是$\ker\psi=\im\varphi$.

综上,我们有正合列
\[
	T\otimes_S \Omega_{S/R}\xrightarrow{\varphi} \Omega_{T/R}\xrightarrow{\varphi} \Omega_{T/S}\to 0,
\]
其中$\varphi:t\otimes ds\mapsto tds$以及$\psi:dt\mapsto dt$.

利用这个正合列,当$R\to S$是一个满同态的时候,由于此时$\Omega_{S/R}=\{0\}$,所以$\Omega_{T/R}\cong \Omega_{T/S}$.

\para 依然考虑三个环$R\to S\to T$的情况,但此时假设$\pi:S\to T$是一个满射,因此$\Omega_{T/S}=0$,上述正合列就只剩
\[
	T\otimes_S \Omega_{S/R}\xrightarrow{\psi} \Omega_{T/R}\to 0,
\]
此时$\psi$是一个满同态。

设$I=\ker \pi$,由于$\pi:S\to T$满同态,由于张量积是右正和的,所以$1\otimes \pi:\Omega_{S/R}=S\otimes\Omega_{S/R}\to T\otimes\Omega_{S/R}$是一个满同态,并且$\ker(1\otimes \pi)=I\Omega_{S/R}$,所以$\Omega_{S/R}/I\Omega_{S/R}\cong T\otimes\Omega_{S/R}$,这个同构具体写出来就是$df\to 1\otimes df$. 利用这个同构,可以断言$T\otimes_S\Omega_{S/R}$由$\{1\otimes_S ds\,:\,s\in S\}$张成。

考虑标准导子$d:S\to \Omega_{S/R}$限制在$I$上得到的$S$-模同态$d|_I$. 任取$s\in S$以及$r\in I$,有$d(rs)=sdr+rds$,在$\Omega_{S/R}/I\Omega_{S/R}\cong T\otimes\Omega_{S/R}$中$rds$将被模去,这样$d|_I$就诱导了一个$S$-模同态$I\to \Omega_{S/R}/I\Omega_{S/R}$,我们依然记作$d$。如果$s\in I$,则$d(rs)$在$\Omega_{S/R}/I\Omega_{S/R}$中为零,因此$d(I^2)=0$. 由商模的泛性质,他诱导了$S$-模同态$d:I/I^2\to \Omega_{S/R}/I\Omega_{S/R}$,他同时也是$S/I=T$-模同态。综上,我们就有一列$T$-模同态
\[
	I/I^2\xrightarrow{D} T\otimes_S\Omega_{S/R}\xrightarrow{\varphi} \Omega_{T/R}\to 0,
\]
其中$D:f\mapsto 1\otimes df$. 下面我们证明这还是正合列。

\pro 设$R\to S\to T$是三个环,并且$\pi:S\to T$是满同态,记$I=\ker\pi$,我们有正合列
\[
	I/I^2\xrightarrow{D} T\otimes_S\Omega_{S/R}\xrightarrow{\varphi} \Omega_{T/R}\to 0,
\]
其中$D:f\mapsto 1\otimes df$,以及$\varphi:1\otimes df\mapsto df$.

\proof
	只要求$\ker \varphi$. 由于$T\otimes_S\Omega_{S/R}$由$\{1\otimes_S ds\,:\,s\in S\}$张成,并且,由于$S\to T$是满同态,所以$\Omega_{T/R}$由$\{ds\,:\,s\in S\}$张成。于是$\varphi$将生成元$1\otimes_S ds$变成生成元$ds$,并且$T\otimes_S\Omega_{S/R}$和$\Omega_{T/R}$满足相同的Leibniz法则和$R$-线性性,所以$\ker \varphi$由那些$1\otimes ds\neq 0$张成,其中$ds=0\in\Omega_{T/R}$,如果$1\otimes ds\neq 0$,则只可能$s\in I$. 此时$1\otimes ds=Ds\in\im D$,所以$\ker\varphi=\im D$. 
\qed

利用同构基本定理,上面的正合列给出了同构
\[
	\Omega_{T/R}\cong \coker\left(I/I^2\xrightarrow{D} T\otimes_S\Omega_{S/R}\right).
\]
当$T$是一个有限生成$R$-代数,即$T=R[x_1,\cdots,x_n]/I$,取$S=R[x_1,\cdots,x_n]$,则
\[
	T\otimes_S\Omega_{S/R}=T\otimes_S\bigoplus_{i=1}^n Sx_i=\bigoplus_{i=1}^n Tx_i,
\]
所以
\[
	\Omega_{T/R}\cong \coker\left(I/I^2\xrightarrow{D} \bigoplus_{i=1}^n Tx_i\right),
\]
此时$\Omega_{T/R}$是一个有限生成$T$-模。