%!TEX root = main.tex
\chapter{模(二)}
\ThisULCornerWallPaper{1}{../Pictures/5.png}

\section{投射模、内射模、平坦模}

从Proposition \ref{homleftexact},我们已经知道,对给定的$R$-模$M$,$\Hom(M,-)$与$\Hom(-,M)$都是左正和函子。关注它们什么时候正和是同调代数的基本问题之一。

\begin{para}[投射对象]
	设$\mathcal C$是一个Abel范畴,$M$是一个对象,如果$\Hom(M,-)$是正和函子,则称$M$是一个投射对象。
\end{para}

具体地,因为已经知道这是左正和的了,所以$Z$投射等价于,对任何满态射$f:X\to Y$,以及态射$g:Z\to Y$,都存在(不一定唯一)$h:Z\to X$使得如下图交换
\[
	\xymatrix{
Z \ar@{->}[rd]^g \ar@{-->}[d]^h &  &  \\
X \ar@{->}[r]^f & Y \ar@{->}[r] & 0
}
\]
等价是直接的,因为此时$\Hom(Z,X)\to \Hom(Z,Y)$是满的。

\begin{para}[内射对象]
	设$\mathcal C$是一个Abel范畴,$M$是一个对象,如果$\Hom(-,M)$是正和函子,则称$M$是一个内射对象。
\end{para}

具体地,$Z$内射等价于,对任何单态射$f:X\to Y$,以及态射$g:X\to Z$,都存在(不一定唯一)$h:Y\to Z$使得如下图交换
\[
	\xymatrix{
 &  Z & \\
0 \ar@{->}[r] & X \ar@{->}[r]^f \ar[u]^g & Y \ar@{-->}[ul]_h
}
\]
等价是直接的,因为此时$\Hom(Y,Z)\to \Hom(X,Z)$是满的。

\begin{pro}
	一个模是投射模,当且仅当,它是一个自由模的直和项。
\end{pro}

\begin{para}[关系]
	考虑一个$R$-模$X$,一组$X$中的元素$x_1$, $\dots$, $x_n$,以及一组标量$r_1$, $\dots$, $r_n$使得$\sum_{i}r_ix_i=0$. 这样的一个等式被称为$X$上的一个关系。特别地,如果$X$是有限生成的,则其为一个自由模$R^n$的商模,而商掉的自然就是所有关系构成的子模。用关系(即商模)是表示某些模的最自然的方式。

	现在,考虑另一个$R$-模$Y$,考虑$X\otimes Y$上的关系$\sum_i x_i\otimes y_i=0$. 显然,原本$X$上的一族关系$\sum_i r_i^{j}x_i=0$,其中$j$是关系的指标,都可以诱导出$X\otimes Y$上的关系如下:
	\[
		y_i=\sum_{j}r^j_i y^j,
	\]
	其中$y^j\in Y$是任意的关系。现在,反过来,我们问什么时候$X\otimes Y$上的关系都是这样来自于$X$上的关系呢?这自然要对$Y$有一些要求,而这要求正是下面要定义的平坦性。
	
	略微粗糙地说,一个平坦模在张量积的时候不会引入新的关系。这是非常重要的性质,比方说,在代数几何中,我们将任意的态射$f:X\to Y$看成参数空间$Y$上的概形族,对每一个$y$,纤维$f^{-1}(y)$就是参数为$y$时的概形。如果$X$会引入新的关系,则极限行为就往往不是我们想要的极限行为了,即极限行为应该仅依赖于参数。具体一些,当考虑$Y$中的一个极限$b\to a$时,我们会希望$f^{-1}(b)\to f^{-1}(a)$,即我们要求一种“连续性”。
	
	设$f:X\to Y$是一个给定的概形族,因为要考虑极限行为,所以与其考虑纤维$f^{-1}(a)=X\times_Y \operatorname{Spec}(k(a))$,我们可以转而考虑一下“无穷小邻域”$X\times_Y \operatorname{Spec}(\mathcal O_{Y,a})$. 进一步假设$X$和$Y$都是仿射的,$X=\operatorname{Spec} A$, $Y=\operatorname{Spec} B$,则我们等价于考虑张量积$A\otimes_B B_{a}$. 为满足我们希求的极限行为,我们会要求$A$不要再引入任何极限行为中额外的关系,于是,我们会要求$A$作为$B$-模是平坦的。
	更一般地,概形间的态射$f:X\to Y$是平坦的,如果其诱导的局部环态射$f^\#_{x}:\mathcal O_{Y,f(x)}\to \mathcal O_{X,x}$是平坦的,即$\mathcal O_{X,x}$作为$\mathcal O_{Y,f(x)}$-模是平坦的。

	某种角度来看,概形的平坦族是纤维丛的类似物,其在每个局部都近似为平凡丛。在概形的平坦族这里,在每个局部,所有的极限行为都由$Y$来控制,$X$并不会引入新的“扭曲”,是为“平坦”。
\end{para}

\begin{para}[平坦性]
	设$M$是一个$R$-模,若$-\otimes_R M$是正和的,则称$M$是一个平坦$R$-模。一个$R$-代数$S$如果看成$R$-模是平坦的,则称其为平坦的,同时,也称定义同态$R\to S$被称为平坦的。
\end{para}

由于张量积是左伴随函子,所以其是右正和的,所以平坦就只是加上了如下新的条件:任取单同态$f:N'\to N$,诱导的同态$N'\otimes M \to N\otimes M$也是单的。

回忆Proposition \ref{flat_loc},任取乘性子集$S$,$S^{-1}R$是平坦$R$-模。

\begin{pro}
	设$A$是一个环,我们有:
	\begin{compactenum}
		\item 自由$A$-模平坦;
		\item 设$M$, $N$都是平坦$A$-模,则$M\otimes_A N$也是平坦$A$-模;
		\item 设$B$是一个$A$-代数,如果$M$是平坦$A$-模,则$M\otimes_A B$是平坦$B$-模; 
		\item 如果$B$是平坦$A$-代数,则任何$B$-模都是平坦$A$-模。
	\end{compactenum}
\end{pro}

\begin{proof}
	都是显然的。
\end{proof}

\begin{pro}
	平坦性是局部性条件,即$R$-模$M$平坦等价于任取极大理想$\mathfrak m$,$M_{\mathfrak m}$都平坦。
\end{pro}

\begin{proof}
	从局部化,已经知道$S^{-1}R$是平坦$R$-模,假设$M$是平坦$R$-模,则$S^{-1}M$也是。反之,设$f:N\to N'$是一个$R$-模的单同态,记$L$为诱导的同态的$N\otimes M\to N'\otimes M$的核。任取极大理想$\mathfrak m$,因为$R_{\mathfrak m}$平坦,所以我们有正和列
	\[
		0\to L\otimes R_{\mathfrak m}\to (N\otimes M)\otimes R_{\mathfrak m}\to (N'\otimes M)\otimes R_{\mathfrak m},
	\]
	或等价地
	\[
		0\to L_{\mathfrak m}\to N_{\mathfrak m}\otimes M_{\mathfrak m}\to N'_{\mathfrak m}\otimes M_{\mathfrak m},
	\]
	由于$M_{\mathfrak m}$平坦,所以$L_{\mathfrak m}=0$. 遍历极大理想$\mathfrak m$,立刻得知$L=0$.
\end{proof}

\begin{pro}
	$R$-模$M$是平坦的当且仅当,任取$R$的理想$\mathfrak a$,典范同态$\mathfrak a\otimes M\to \mathfrak aM$是同构。
\end{pro}

从这个命题也可以再次看到,乘法基本并不会引入新的关系,如同张量积一般。

\begin{proof}
	从$M$平坦到同构是直接的。\notprove
\end{proof}

作为推论,如果$R$是一个主理想整环,且对任意的$m\in M$,$am=0$都可以推出$a=0$,此时$M$是平坦的。

\begin{pro}
	设$R$是一个环,而$0\to M'\to M\to M''\to 0$是一个$A$-模的正和列,现在设$M''$是平坦的,在对任何$A$-模$N$,$0\to M'\otimes N \to M\otimes N\to M''\otimes N\to 0$也是正和列。
\end{pro}

\begin{proof}
	我们只要真美国$M'\otimes N\to M\otimes N$是单的。设$N$可以写成自由模$L$模去$K$,则我们有如下交换图
	\[
		\xymatrix{
			& M'\otimes K \ar[r]\ar[d]& M\otimes K \ar[r]\ar[d]& M''\otimes K\ar[d]_{\alpha}\\
		0\ar[r]& M'\otimes L\ar[r]\ar[d]_\beta& M\otimes L\ar[r]\ar[d] & M''\otimes L\\
		& M'\otimes N \ar[r] & M\otimes N
		}
	\]
	其中$\alpha$是单的(因为$M''$是平坦的),$\beta$是满的。然后不难通过追图得到结论。
\end{proof}

\begin{pro}
	局部环上的有限生成平坦模是自由模。
\end{pro}

\section{准素分解}

这节在某种角度来说,我们要开始研究$R$-模$M$的结构中的标量乘法。标量乘法并不总像$k$在$k[x]$上的作用那么简单,比方说,对$m\in M$,可能存在一个$r\in R$使得$rm=0$. 我们这里就是要考虑这样的问题。

\para[零因子] 考虑一个非零$R$-模$M$,如果对$r\in R$存在一个非零的$m\in M$使得$rm=0$,则这个$r$称为$M$的一个零因子。对固定的$m\in M$,使得$rm=0$的$r\in R$的集合记作$\ann(m)$. 这是一个$R$中的理想,因为$ram=r0=0$对$a\in \ann(m)$成立。设$S$是$M$的一个子集,定义$\ann(S)=\bigcap_{m\in S}\ann(m)$.\endpara

设$M$是一个$R$-模,设$\mathfrak{a}\subset \ann(M)$是$R$中的一个理想,此时$M$可以看成一个$R/\mathfrak{a}$-模,因为$(r+\mathfrak{a})m=rm$. 特别地,$M$是一个$R/\ann(M)$-模,此时,如果$rm=0$,则可以推出$r$和$m$中有一个零。

\begin{lem}
	如果$\ann(m)$是素理想,则$\ann(m)\in \supp(M)$.
\end{lem}

\begin{proof}
	实际上,任取$r\in R-\ann(m)$,则$rm\neq 0$. 所以$m$在$M_{\ann(m)}$中的像不为零,因此$\ann(m)\in \supp(M)$.
\end{proof}

\para $R$中形如$\ann(m)$的那些素理想被称之为$M$的associated primes,定义$\Ass_R(M)$为所有associated primes的集合,如果下标$R$是清楚的,我们会略去它。
\endpara

上一个引理告诉了我们$\Ass(M)\subset \supp(M)$. 此外,当然有可能所有的$\ann(m)$都不是素理想,即$\Ass(M)$为空集。但如果$R$是一个Noether环,而$M$非零,则下面的引理保证了$\Ass(M)$非空。当然,如果是零模$M$,则$\Ass(M)$一定为空。

\begin{lem}
设$\mathcal{P}$是所有形如$\ann(m)$的理想按照包含构成的一个偏序集,则$\mathcal{P}$中的极大元都是素理想。
\end{lem}

\begin{proof}
	设$\ann(m)$是一个极大元,再设$r\notin \ann(m)$,即$rm\neq 0$. 很清楚,$\ann(m)\subset \ann(rm)$,所以由极大性可知$\ann(rm)=\ann(m)$. 现在如果$rs\in \ann(m)$但$r\notin \ann(m)$,则$srm=0$可知$s\in \ann(rm)=\ann(m)$,所以$\ann(m)$是一个素理想。
\end{proof}

如果$R$是一个Noether环,则它的理想构成的非空子集一定存在极大元,这就保证了$\Ass(M)$非空。并且,如果$r$是$M$的一个零因子,含有$r$的那个极大的$\ann(m)$属于$\Ass(M)$,这就是说,任意的零因子必然属于$\Ass(M)$中的某个元素,或者如下包含关系
\[
	\bigcup_{0\neq m\in M}\ann(m)\subset \bigcup_{\mathfrak{a}\in \Ass(M)}\mathfrak{a}. 
\]
利用Proposition \ref{primeav},我们还可以断言,其实存在一个$\pp\in \Ass(M)$使得
\[
	\bigcup_{0\neq m\in M}\ann(m)\subset \pp.
\]

\begin{lem}
设$M$是一个非零$R$-模,如果$\ann(m)$是素理想,而$r\in R$使得$rm\neq 0$,则$\ann(rm)=\ann(m)$. 特别地,如果$\pp$是$R$的一个素理想,则$\Ass(R/\pp)=\{\pp\}$.
\end{lem}

下面可能要用这个非常简单的推论:对$M$的子模$Rm$的任意非零子模$N$有$\ann(N)=\ann(m)$.

\begin{proof}
	显然,$\ann(m)\subset \ann(rm)$. 同时,由于$rm\neq 0$,所以$r\not\in \ann(m)$. 反之,任取$s\in \ann(rm)$,则$srm=0$给出$sr\in \ann(m)$. 由于$r\not\in \ann(m)$且$\ann(m)$是素理想,所以$s\in \ann(m)$. 这就给出了$\ann(rm)\subset \ann(m)$.

	现在,考虑$M=R/\pp$而$m=1$的情况,此时任取$m\neq 0$,我们都有$\ann(m)=\ann(1)=\pp$,所以$\Ass(R/\pp)=\{\pp\}$.
\end{proof}

\begin{pro}
如果我们有$R$-模的正合列$0\to M'\to M\to M''\to 0$,则我们有包含关系
\[
	\Ass(M')\subset \Ass(M)\subset \Ass(M')\cup \Ass(M'').
\]
\end{pro}

\begin{proof}
	由于单同态并不会改变$\Ass$,所以可以将$M'$看成$M$的子模,所以第一个包含显然。对第二个包含,任取$\mathfrak{a}\in \Ass(M)-\Ass(M')$,我们要证明$\mathfrak{a}\in \Ass(M')$. 设$\mathfrak{a}=\ann(m)$,其中$m\in M$,由$m$生成的子模$Rm$同构于$R/\mathfrak{a}$. 考虑$N=Rm\cap M'$,他是$Rm$的子模,如果$N$非零,因为$\mathfrak{a}\notin \Ass(M')$,$\mathfrak{a}N\neq 0$,因此$\mathfrak{a}\not\subset \ann(N)=\ann(m)=\mathfrak{a}$,矛盾,所以只可能$Rm\cap M'=0$. 在这种情况下,$Rm$同构于他在$M''$中的像,所以$\mathfrak{a}\in \Ass(M'')$.
\end{proof}

考虑$M=M'\oplus M''$的情况,由于$M'$和$M''$都有到$M$的单同态,所以$\Ass(M')\subset \Ass(M)$以及$\Ass(M'')\subset \Ass(M)$,继而他给出了$\Ass(M)=\Ass(M')\cup \Ass(M'')$.

从下个命题开始,我们假设这节下面出现的环都是Noether环,除非特别申明。

\begin{pro}
设$M$是一个有限生成$R$-模,则存在一个子模链
\[
	0=M_0\subset M_1\subset \cdots\subset M_n=M,
\]
使得$M_{i+1}/M_i\cong R/\pp_i$,其中$\pp_i$是$R$的一个素理想。
\end{pro}

\begin{proof}
利用Hilbert基定理,$M$是一个Noether模。由于$R$是一个Noether环,所以$\Ass(M)$非空,即存在一个素理想$\pp_1=\ann(m_1)$,于是$Rm_1\cong R/\pp_1$. 我们将$M_1$取做$Rm_1$,然后对$M/M_1$重复上面的逻辑,由于$M$是一个Noether模,我们只能这样做有限次,进而得到了我们需要的链。
\end{proof}

由于$M_{i+1}/M_i\cong R/\pp_i$,所以
\[
	\Ass(M_i)\subset \Ass(M_{i+1})\subset \Ass(M_i)\cup \Ass(R/\pp_i).
\]
因此,我们得到了一个很粗的包含关系
\[
	\Ass(M)\subset \bigcup_i \Ass(R/\pp_i)=\bigcup_i \{\pp_i\}.
\]
但至少,对有限生成$R$-模$M$,我们知道,$\Ass M$有限。如果应用到本征值问题,可以断言有限维$k$-矢量空间之间的映射的本征值是有限的(当然也可能不存在)

\begin{pro}
设$M$是一个$R$-模,而$S$是$R$的一个乘性子集,则
\[
	\Ass_{S^{-1}R}(S^{-1}M)=\{S^{-1}\pp\,:\,\pp\in \Ass_RM\;\;\text{and}\;\;\mathfrak p\cap S=\varnothing\}.
\]
\end{pro}

\begin{proof}
如果$\pp\in \Ass_R M$,那么存在一个$M$的一个子模同构于$R/\pp$. 局部化之,我们得到了$S^{-1}M$的一个子模同构于$S^{-1}R/S^{-1}\pp$,所以,如果$S^{-1}\pp$是$S^{-1}R$的素理想,即$\pp\cap S= \varnothing$,则$S^{-1}\pp\in\Ass_{S^{-1}R}(S^{-1}M)$.

反之,任取$S^{-1}R$的素理想$\mathfrak{q}$,我们可以将$\mathfrak{q}$写为$S^{-1}\pp$,其中$\pp$是$R$的一个素理想且$\pp\cap S=\varnothing$. 如果$S^{-1}\pp\in \Ass_{S^{-1}R}(S^{-1}M)$,则存在一个单射$\varphi:S^{-1}R/S^{-1}\pp\to S^{-1}M$. 因为$R$是一个Noether环,$R/\pp$是有限生成$R$-模,所以$R/\pp$也是有限表示模。因此,Proposition \ref{3.5.5}告诉我们存在典范同构
\[
	\Hom_{S^{-1}R}(S^{-1}R/S^{-1}\pp,S^{-1}M)\cong S^{-1}\Hom_R(R/\pp,M),
\]
所以,我们可以将$\varphi$写为$s^{-1}f$的形式,其中$f\in \Hom_R(R/\pp,M)$而$s\in S$. 因为$S$并不包含$R/\pp$的零因子,所以$f$是一个单射,这就给出了反方向的论断。
\end{proof}

\begin{pro}\label{pro:5.1.9}
设$M$是一个非零$R$-模,则$\supp(M)$中的极小素理想在$\Ass (M)$中。
\end{pro}

\begin{proof}
任取一个$\supp(M)$中的极小素理想$\pp\in \supp(M)$,我们需要说明$\pp\in \Ass(M)$. 首先,$\pp$包含$\ann(M)$. 局部化之,考虑环$R_\pp$和模$M_\pp$,因为$M_\pp\neq \{0\}$,所以$\Ass_{R_\pp}(M_\pp)$非空,且其中的元素对应于$\mathfrak{q}\in \Ass_R M$且$\mathfrak{q}\subset \pp$. 由于每一个$\mathfrak{q}\in \Ass_R M$都在$\supp(M)$中,而$\pp$是极小的,如果$\pp$不在$\Ass_R M$中,则$\Ass_{R_\pp}(M_\pp)$为空,矛盾。
\end{proof}

注意到,尽管两个素理想的交并不一定是素理想,这就意味着极小素理想可能不唯一。此外,如果$M$是有限生成的,则$\supp(M)$即所有包含$\ann(M)$的素理想的集合,由于素理想降链的交还是素理想(也包含$\ann(M)$),所以此时Zorn引理告诉我们,极小素理想一定存在。所以,即使没有Noether条件,如果$M$有限生成,$\Ass M$也非空。

\begin{para}
设$N$是$M$的一个子模,如果$\Ass(M/N)$只有一个元素,则称呼$N$是$M$的一个准素子模,如果$\Ass(M/N)=\{\mathfrak{p}\}$,则称呼$N$是$\mathfrak{p}$-准素的。
\end{para}

前面我们已经看到,素理想$\pp$总是$R$的一个准素子模,而且是$\pp$-准素的。

\begin{lem}
	如果$N_1$和$N_2$都是$\mathfrak{p}$-准素的,则$N_1\cap N_2$也是. 
\end{lem}

经过有限次归纳可以得到:有限个$\mathfrak{p}$-准素子模的交依然是一个$\mathfrak{p}$-准素子模。

\begin{proof}
	事实上,考虑单同态$M/(N_1\cap N_2)\to M/N_1\oplus M/N_2$,因此$\Ass(M/(N_1\cap N_2))\subset \Ass(M/N_1)\cup \Ass(M/N_2)=\{\mathfrak{p}\}\cup \{\mathfrak{p}\}=\{\mathfrak{p}\}$. 但是由于$\Ass(M/(N_1\cap N_2))$非空,所以$\Ass(M/(N_1\cap N_2))=\{\mathfrak{p}\}$. 
\end{proof}

\begin{pro}\label{pro:5.1.11}
设$R$是Noether环,$\pp$是它的一个素理想,而$M$是一个有限生成$R$-模,则下面三个命题等价:
\begin{compactenum}[~~~(1)]
\item $\Ass(M)=\{\pp\}$.
\item $\pp$是包含$\ann(M)$的极小素理想,且任取$r\not\in \pp$,则$r$都不是$M$的零因子。
\item 存在一个正整数$n$使得$\pp^n M=0$,且任取$r\not\in \pp$,则$r$都不是$M$的零因子。
\end{compactenum}
\end{pro}

注意,条件“任取$r\not\in \pp$,$r$都不是$M$的零因子”这个条件也可以等价为条件“局部化同态$M\to M_{\mathfrak p}$是单同态”。

\begin{proof}
$(1)\Rightarrow (2)$:因为$\Ass M$只包含一个$\pp$,但$\Ass M$应包含所有的极小素理想,所以$\pp$是包含$\ann(M)$的唯一极小素理想。此外,由于
\[
	\bigcup_{0\neq m\in M}\ann(m)\subset \bigcup_{\mathfrak{a}\in \Ass(M)}\mathfrak{a}=\pp. 
\]
所以,任取$r\not\in \pp$,则$r$不包含于任意的$\ann(m)$中,即$r$不是$M$的零因子。

$(2)\Rightarrow (3)$:因为不在$\pp$中的每个元素都不是$M$的零因子(此时$\ann_R(M_\pp)=\ann_R(M)$),所以我们在对$\pp$局部化后考虑问题,此时$\pp$可以看成极大理想,同时$\pp$还是包含$\ann_R(M)$的唯一极小素理想,所以$\pp=\sqrt{\ann_R(M)}$,而且$\pp$又是有限生成的,所以存在一个正整数$n$使得$\pp^n\subset \ann_R(M)$. 

$(3)\Rightarrow (1)$:由于模掉$\ann(M)$后$\pp$是幂零的,所以它就是一个包含$\ann(M)$的一个极小素理想,于是$\pp \in \Ass(M)$. 又因为每一个不属于$\pp$的元素都不是零因子,所以每一个$\mathfrak{q}\in \Ass(M)$都有$\mathfrak{q}\subset \pp$,所以$\pp$也是$\Ass(M)$唯一的元素。
\end{proof}

\begin{pro}\label{pro:5.2.13}
	设$M$是一个有限生成$R$-模,而$\pp\in \supp(M)$是一个包含$\ann(M)$的极小素理想,设$\alpha:M\to M_\pp$是局部化同态,则$\ker\alpha$是一个$\pp$-准素子模。
\end{pro}

\begin{proof}
	由于$(M/\ker\alpha)_{\pp}=M_{\pp}$,所以同态$\overline{\alpha}:M/\ker\alpha\to M_{\pp}=(M/\ker\alpha)_{\pp}$是一个单射。为利用上一个命题,我们只要说明$\pp$是包含了$\ann(M/\ker \alpha)$的极小素理想,这样$\ker \alpha$就是$\pp$-准素的了。

	一方面,$\pp$包含$\ann(M/\ker \alpha)$. 实际上,因为$\pp$是包含$\ann(M)$的极小素理想,则$\pp=\ann(m_0)\in \Ass(M)$. 如果$a\overline{m}=0$对任意的$m\in M/\ker \alpha$都成立,则$\alpha(am)=0$对所有的$m\in M$都成立。这意味着存在$b\not\in \pp$使得$abm_0=0$,所以$ab\in \ann(m_0)=\pp$将给出$a\in \pp$. 另一方面,$\ann(M)\subset \ann(M/\ker \alpha)$,所以如果$\mathfrak q\subset \pp$是包含$\ann(M/\ker \alpha)$的素理想,则它也是包含$\ann(M)$的素理想,这与$\pp$极小矛盾。
\end{proof}

% \begin{para}[主理想整环的情况]
% 	我们这里考虑$R$是一个主理想整环,此时,任意两个不同的素理想是互素。实际上,考虑$\pp_1=(a_1)$和$\pp_2=(a_2)$生成的理想$(b)$,则$a_1\in (b)$和$a_2\in b$告诉我们$b$是$a_1$和$a_2$的共同因子。但注意到,在PID中,素元等价于不可约元,此时$b$只能是可逆元,所以$(b)=R$.

% 	进而,取$\pp\in \Ass(M)$,定义$M^\pp:=\{m\in M\,:\, \pp \subset \ann(m)\}$. 但是,由于不存在包含$\pp$的素理想,所以$\Ass(M^\pp)=\{\pp\}$. 此外,再取$\mathfrak q\in \Ass(M)$,则我们有$M^{\pp}\cap M^{\mathfrak q}=\{0\}$. 实际上,如果$m\in M^{\pp}\cap M^{\mathfrak q}$,则任取$r\in (\pp+\mathfrak{q})$都有$rm=0$,但是,由于$\pp$和$\mathfrak q$互素,所以$r$可以取做$1$,所以$m=0$.

% 	进一步,考虑$M$是有限生成的。我们来求局部化同态$M\to M_\pp$的核,即使得$m/1=0$的$m$,换言之,我们需要$\ann(m)\not\subset \pp$.
% \end{para}

利用Proposition \ref{pro:5.1.11},下面考虑$M=R/\mathfrak{a}\neq 0$的情况,这将给出准素理想的经典定义。

\begin{pro}
真理想$\mathfrak{a}$是$R$的准素子模,当且仅当,$R/\mathfrak{a}$中的零因子都是幂零的。
\end{pro}

等价地,也可以将上面的条件写为:$xy\in \mathfrak{a}$可以推出$x\in \mathfrak{a}$或者$y\in \sqrt\mathfrak{a}$.

\begin{proof}
如果$\mathfrak{a}$是$\pp$-准素的,从上面我们可以看到,存在一个$n$使得$\pp^n\subset \mathfrak{a}\subset \mathfrak{p}$. 此外,从前面也已知道,任取$\bar{a}\in R/\mathfrak{a}$,$\ann(\bar{a})\subset \mathfrak{p}$,所以如果$\bar r$是$R/\mathfrak{a}$中的零因子,则其原像$r\in \mathfrak{p}$,所以$r^n\in \pp^n\subset \mathfrak{a}$,即$\bar r^n=0$. 

反过来,我们证明$\mathfrak{a}$是$\sqrt{\mathfrak{a}}$-准素的。首先,需要指出$\sqrt{\mathfrak{a}}$是一个素理想,因为任取$xy\in \sqrt{\mathfrak{a}}$,其中$x\not\in \sqrt{\mathfrak{a}}$,则存在一个$m$使得$x^m y^m\in \mathfrak{a}$. 由于$x^m\not\in \mathfrak{a}$,利用条件,$y^m$在$R/\mathfrak{a}$中的像是幂零的,即存在一个$n$使得$(y^m)^n=y^{mn}\in \mathfrak{a}$,所以$y\in\sqrt{\mathfrak{a}}$. 此外,由于$\sqrt{\mathfrak{a}}$是所有包含$\mathfrak{a}$的素理想的交,如果其本身是素理想,则其必然是极小素理想。此外,$\mathfrak{a}\subset \sqrt\mathfrak{a}$告诉我们$\sqrt\mathfrak{a}$外的元素都不是零因子,继而上面的命题告诉我们,$\mathfrak{a}$是$\sqrt{\mathfrak{a}}$-准素的。
\end{proof}

如果$\mathfrak{a}$是$\pp$-准素的,则$\pp^n\subset \mathfrak{a}\subset \mathfrak{p}$,所以$\sqrt{\mathfrak{a}}=\pp$. 所以对于准素理想,我们只要读出它的根也就读出了其准素于哪个理想。

\begin{lem}
设$R$是Noether环,$M$是有限生成$R$-模,则$M$的不可约子模$N$是准素子模。
\end{lem}

\begin{proof}
	如果$\Ass(M/N)$中至少存在两个素理想$\pp_1=\ann(\overline{m}_1)$和$\pp_2=\ann(\overline{m}_2)$,那么在$M/N$中有子模$R\overline{m}_1\cong R/\pp_1$和$R\overline{m}_2\cong R/\pp_2$. 如果$R\overline{m}_1\cap R\overline{m}_2$非零,则$R\overline{m}_1\cap R\overline{m}_2$作为$R\overline{m}_1$的非零子模,$\ann(R\overline{m}_1\cap R\overline{m}_2)=\pp_1$,作为$R\overline{m}_2$的非零子模,$\ann(R\overline{m}_1\cap R\overline{m}_2)=\pp_2$,矛盾,于是推出$R\overline{m}_1\cap R\overline{m}_2=\{\overline{0}\}$. 回到$M$中的原像,我们就可以知道$N$可以写成两个$M$的子模的交,这与$N$不可约矛盾。
\end{proof}

准素子模的重要性来自于下面几个命题,是将一个子模写成几个准素子模的交,它们被称为准素分解。特别地,理想作为子模,将一个理想写成几个理想的交,在几何上,就是将一个簇写成一些簇的并。下面我们首先陈述存在性。

\begin{pro}
设$R$是Noether环,$M$是有限生成$R$-模,如果$N$是$M$的一个真子模,那么他是有限个准素子模的交。
\end{pro}

\begin{proof}
由于Noether环上的有限生成模是Noether模,回忆Proposition \ref{irrde},$N$可以写成$M$中的有限个不可约子模的交的形式,而每一个不可约子模都是准素子模。
\end{proof}

\begin{thm}\label{thm:5.2.17}
	令$R$是Noether环,而$M$是有限生成$R$-模,$N$是$M$的一个真子模。如果$N$可以写成交$N=\bigcap_{i=1}^n M_i$,其中$M_i$是$M$的$\mathfrak p_i$-准素子模,记$P$为所有的$\mathfrak p_i$的集合,则
	\begin{compactenum}
		\item $\Ass(M/N)\subset P$.
		\item 如果分解中不能再略去任何一个$M_i$,则$\Ass(M/N) = P$.
		\item 如果分解是极小的,即不存在个数更小的分解,则$\pp_i$各不相同,且刚好就是$\Ass(M/N)$中出现的素理想。此外,如果$\mathfrak p_i$是包含$\ann(M/N)$的极小素理想,则$M_i=\alpha_i^{-1}(N_{\mathfrak p_i})$,其中$\alpha_i:M\to M_{\mathfrak p_i}$是局部化的典范同态。
	\end{compactenum}
\end{thm}

\begin{proof}
	因为前两个命题都是对$M/M_i$或者$M/N$进行的,而$M_i$又都包含$N$,所以我们可以假设$N=0$. 然后在处理第三个命题的时候再回到一般的$N$.

	首先,假设$\bigcap_{i=1}^n M_i=\{0\}$,则我们可以从商映射$\pi_i:M\to M/M_i$构造出映射的积$\pi:M\to \bigoplus_{i=1}^n M/M_i$. 由于$\bigcap_{i=1}^n M_i=\{0\}$,所以$\pi$是一个单射,继而
	\[
		\Ass(M)\subset \Ass\left(\bigoplus_{i=1}^n M/M_i\right)=P.
	\]

	如果分解中不能再略去任何一个$M_i$,则$\bigcap_{i\neq j}M_i\neq \{0\}$,其中$j$是任意的。但是,由于$M_j\cap \bigcap_{i\neq j}M_i= \{0\}$,所以
	\[
		\bigcap_{i\neq j}M_i=\left(\bigcap_{i\neq j}M_i\right)\bigg/ \left(M_j\cap \bigcap_{i\neq j}M_i\right)\cong \left(\bigcap_{i\neq j}M_i+M_j\right)/M_j\subset M/M_j,
	\]
	所以
	\[
		\varnothing\neq \Ass\left(\bigcap_{i\neq j}M_i\right)\subset \Ass(M/M_j)=\{\pp_j\},
	\]
	这就意味着$\{\pp_j\}=\Ass\left(\bigcap_{i\neq j}M_i\right)\subset \Ass(M)$. 遍历$j$,我们就得到了$P\subset \Ass(M)$. 结合第一点,我们就得到了$\Ass(M)=P$.

	现在,假设分解是极小的。因为$\pp$-准素子模的有限交也是$\pp$-准素的,因而如果分解不是极小的,则我们可以将准素于同一个素理想的子模先交起来。所以,极小分解中,$\pp_i$各不相同。同时,因为极小分解中不能再略去任何一个$M_i$,所以$\pp_i$刚好就是$\Ass(M)$中出现的素理想。

	现在回到一般的$N$. 假设$\pp_i$是包含$\ann (M/N)$的极小素理想。已经看到,$\pi:M\to \bigoplus_{i=1}^n M/M_i$的核是$N$,所以局部化之后,我们有
	\[
		\pi_{\mathfrak p_i}:M_{\mathfrak p_i}\to \bigoplus_{j=1}^n (M/M_j)_{\mathfrak p_i}
	\]
	的核为$N_{\mathfrak p_i}$. 因为$\pp_i$是极小的,所以存在一个$a\in \ann(M/M_j)\subset \pp_j$也在$R-\pp_i$中,否则,$\pp_i$包含$\ann(M/M_j)$,利用$\pp_j$的极小性(见Proposition \ref{pro:5.1.11}),我们立刻得到$\pp_j\subset \pp_i$,但这与$\pp_i$的极小性矛盾。进而,$(M/M_i)_{\mathfrak p_i}=0$,因为任取$m/s\in (M/M_i)_{\mathfrak p_i}$,我们都有$am=0$. 综上,
	\[
		(\pi_i)_{\mathfrak p_i}:M_{\mathfrak p_i}\to  (M/M_i)_{\mathfrak p_i}
	\]
	的核为$N_{\mathfrak p_i}$. 再复合上同态$\alpha_i:M\to M_{\mathfrak p_i}$,我们就得到了同态
	\[
		(\pi_i)_{\mathfrak p_i}\alpha_i:M\to  (M/M_i)_{\mathfrak p_i},
	\]
	它也等于先局部化再求商$\beta_i:M\to  M_{\mathfrak p_i}/(M_i)_{\mathfrak p_i}$,所以
	\[
		\ker ((\pi_i)_{\mathfrak p_i}\alpha_i)=\ker \beta_i=M_i.
	\]
	求$\beta_i$的核我们应用了Proposition \ref{pro:3.5.19}. 另一方面
	\[
		\ker ((\pi_i)_{\mathfrak p_i}\alpha_i)=\alpha_i^{-1}(\ker (\pi_i)_{\mathfrak p_i})=\alpha_i^{-1}(N_{\mathfrak p_i}).
	\]
	所以$\alpha_i^{-1}(N_{\mathfrak p_i})=M_i$,此即所需。
\end{proof}

准素分解不是唯一的,比方说,在$k[x,y]$中,$(x^2,xy)=(x)\cap (x^2,y)=(x)\cap (x^2,xy,y^2)$,但是,可以看到,对于极小分解,我们还是可以得到许多有趣的信息。

\section{微分模}

\para 设$S$是一个环,$M$是一个$S$-模,则一个加法群同态$d:S\to M$被称为一个导子,如果它满足Leibniz法则
\[
	d(fg)=fdg+gdf.
\]
如果$S$是一个$R$-代数,设$\varphi:R\to S$是定义代数的环同态,那么$r\cdot m=\varphi(r)m$定义出了$M$上的一个$R$-模结构。此时,导子$d$如果是$R$-模同态,则称呼$d$是$R$-线性的。$S$到$M$所有的$R$-线性导子构成了一个$R$-模。

由于$d(1\cdot 1)=d(1)+d(1)$,因此对任意的导子有$d(1)=0$. 如果$d$是$R$-线性的,对$r\in R$,我们有$d(r)=rd(1)=0$. 设$s\in S$,则$d(s^n)=ns^{n-1}d(s)$,这个可以用Leibniz法则直接计算得到。

如果$\varphi:R\to S$是一个满同态,如果将$S$看成$R$-代数,则$\varphi(r)=r\cdot 1$,于是从上面的构造有$d(r\cdot 1)=rd(1)=0$,所以此时$\Omega_{S/R}=\{0\}$.

\begin{pro}
给定环$R$和$R$-代数$S$,则存在一个$S$-模$\Omega_{S/R}$和一个$R$-线性的导子$d:S\to R$使得任意的$S$-模$M$与$R$-线性导子$\delta: S\to M$都存在唯一的$R$-线性映射$\varphi:\Omega_{S/R}\to M$使得分解
\[
	\delta:S\xrightarrow{d}\Omega_{S/R}\xrightarrow{\varphi}M
\]
成立。
\end{pro}

\begin{proof}
考虑集合$\{d(f)\,:\, f\in S\}$生成的自由$S$-模$M$,再考虑其中由
	\[
	d(fg)-fd(g)-gd(f),\quad d(rf)-rd(f),\quad d(f+g)-d(f)-d(g)
	\]
	生成的子模$N$,于是$\Omega_{S/R}$就被构造为$M/N$,而$d:f\mapsto d(f)$很容易检验这是一个$R$-线性导子。

	泛性质的检验从构造是简单的。
\end{proof}

今后,$d(f)$的括号不产生理解上的差异的时候会略去记成$df$. 上面定义的模就被称为$S$的微分模。由泛性质,微分模在同构意义下唯一。

\para 在继续关于微分模的讨论之前,先谈谈他相关的几何概念。在光滑流形$M$上,我们知道,给定一个光滑函数$f:M\to \mathbb{R}$,那么在每一点$f_{*p}:T_pM\to T_{f(p)}\mathbb{R}=\mathbb{R}$将$p$点的切矢量$v$变成了一个数$f_{*p}(v)=v(f)=df(v)(p)$,然后让$p$在流形上跑动起来,这样就得到了$f_*=df$实际上是余切丛的一个光滑截面。

记$\mathcal{C}^\infty(M)$是流形$M$上的光滑函数环,这是一个$\mathbb{R}$-代数,而$\Gamma(T^*M)$是$M$的余切丛的光滑截面构成的集合,他是一个加法群,通过逐点定义数乘,可以赋予一个$\mathcal{C}^\infty(M)$-模结构,此时$f\mapsto f_*$就是一个导子。

由微分模的泛性质,因此存在一个$\mathcal{C}^\infty(M)$-模同态$\Omega_{\mathcal{C}^\infty(M)/\mathbb{R}}\to \Gamma(T^*M)$. 由于$T^*M$与$\Gamma(T^*M)$是等价的对象,所以$\Omega_{\mathcal{C}^\infty(M)/\mathbb{R}}$这里就可以理解成余切丛。他的对偶就可以理解成切丛。

\para 设$S$是一个有限生成$R$-代数,他被$\{f_1,\cdots,f_n\}$生成,那么任意的单项式$f_1^{k_1}\cdots f_n^{k_n}\in S$求导后有
\[
	d(f_1^{k_1}\cdots f_n^{k_n})=\sum_{i=1}^n f_1^{k_1}\cdots d(f_i^{k_i})\cdots f_n^{k_n},
\]
如果$k_i=0$则$d(f_i^{k_i})=0$,如果$k_i>0$则$d(f_i^{k_i})=f_i^{k_i-1}df_i$,所以这个单项式可以由$\{df_i\}$线性组合而言,继而$\Omega_{S/R}$就可以由$\{df_i\}$线性组合。

因此,我们求出了,如果$S=R[x_1,\cdots,x_n]$,则$\Omega_{S/R}=\bigoplus_{i=1}^n Sdx_i$,即由$\{dx_i\}$生成的自由模。

\para 回忆一个$R$-代数$S$实际上是一个三元组$(R,S,\varphi)$,其中$\varphi:R\to S$是一个环同态。考虑所有这样三元组构成的范畴,其中态射$\rho:(R,S,\varphi)\to (R',S',\varphi')$是两个环同态$\rho_R:R\to R'$以及$\rho_S:S\to S'$,它们需要满足交换图
\[
\begin{xy}
	\xymatrix
	{
		R\ar[rr]^{\rho_R} \ar[d]_{\varphi}&&R' \ar[d]^{\varphi'}\\
		S\ar[rr]^{\rho_S}&&S'
	}
\end{xy}
\]
不难检验复合的成立。

由于每个代数都可以考虑它的微分模,那么我们自然可以考虑微分模在上述态射下的变化,为此,我们记$\Omega(S,R,\varphi)$来代替$\Omega_{S/R}$,则上述态射诱导了如下的交换图
\[
\begin{xy}
	\xymatrix
	{
		R\ar[r]^{\varphi}\ar[d]_{\rho_R}&S\ar[rr]^-{d} \ar[d]_{\rho_S}&&\Omega(S,R,\varphi) \ar[d]\\
		R'\ar[r]^{\varphi'}&S'\ar[rr]^-{d'}&&\Omega(S',R',\varphi')
	}
\end{xy}
\]
没有标记的那个同态来自于微分模的泛性质,因此,我们可以将$\Omega$看成一个代数范畴到模范畴的一个函子。

\para 考虑$R=R'$的情况,此时代数的态射约化成了$S\to S'$的一个环同态,换而言之,即考虑三个环
$R\to S\to T$的情况。

考虑$T$-模$\Omega_{T/R}$和$\Omega_{T/S}$,从构造,他们都可以写成自由$T$-模$M$关于两个子模$N_R$和$N_S$的商模,由于$S$-线性可以推出$R$-线性,所以$N_R\subset N_S$. 从商同态$M\to M/N_S=\Omega_{T/S}$可以诱导出满同态$\psi:\Omega_{T/R}=M/N_R\to \Omega_{T/S}$,具体到元素就是$\psi:dc\mapsto dc$. 

对$S$-模$\Omega_{S/R}$,我们可以标量扩张成一个$T$-模,通过$T\otimes_S \Omega_{S/R}$. 对环同态$\pi:S\to T$. 由于同样都是$R$-线性的,所以通过$\varphi:t\otimes ds\mapsto tds$可以给出$R$-线性的$T$-模同态$\varphi:T\otimes_S \Omega_{S/R}\to \Omega_{T/R}$.

现在我们来求$\ker\psi$,由构造,他就是$N_S/N_R$. $N_S$由所有$d(fg)-df-dg$, $d(fg)-df-dg$和$d(f+g)-df-dg$生成,由于$d(fg)-df-dg$, $d(f+g)-df-dg\in N_R$, 所以$\ker\psi$被$d(st)-sdt=tds$生成元,其中$s\in S$,由于$tds=\varphi(t\otimes ds)$,于是$\ker\psi=\im\varphi$.

综上,我们有正合列
\[
	T\otimes_S \Omega_{S/R}\xrightarrow{\varphi} \Omega_{T/R}\xrightarrow{\varphi} \Omega_{T/S}\to 0,
\]
其中$\varphi:t\otimes ds\mapsto tds$以及$\psi:dt\mapsto dt$.

利用这个正合列,当$R\to S$是一个满同态的时候,由于此时$\Omega_{S/R}=\{0\}$,所以$\Omega_{T/R}\cong \Omega_{T/S}$.

\para 依然考虑三个环$R\to S\to T$的情况,但此时假设$\pi:S\to T$是一个满射,因此$\Omega_{T/S}=0$,上述正合列就只剩
\[
	T\otimes_S \Omega_{S/R}\xrightarrow{\psi} \Omega_{T/R}\to 0,
\]
此时$\psi$是一个满同态。

设$I=\ker \pi$,由于$\pi:S\to T$满同态,由于张量积是右正和的,所以$1\otimes \pi:\Omega_{S/R}=S\otimes\Omega_{S/R}\to T\otimes\Omega_{S/R}$是一个满同态,并且$\ker(1\otimes \pi)=I\Omega_{S/R}$,所以$\Omega_{S/R}/I\Omega_{S/R}\cong T\otimes\Omega_{S/R}$,这个同构具体写出来就是$df\to 1\otimes df$. 利用这个同构,可以断言$T\otimes_S\Omega_{S/R}$由$\{1\otimes_S ds\,:\,s\in S\}$张成。

考虑标准导子$d:S\to \Omega_{S/R}$限制在$I$上得到的$S$-模同态$d|_I$. 任取$s\in S$以及$r\in I$,有$d(rs)=sdr+rds$,在$\Omega_{S/R}/I\Omega_{S/R}\cong T\otimes\Omega_{S/R}$中$rds$将被模去,这样$d|_I$就诱导了一个$S$-模同态$I\to \Omega_{S/R}/I\Omega_{S/R}$,我们依然记作$d$。如果$s\in I$,则$d(rs)$在$\Omega_{S/R}/I\Omega_{S/R}$中为零,因此$d(I^2)=0$. 由商模的泛性质,他诱导了$S$-模同态$d:I/I^2\to \Omega_{S/R}/I\Omega_{S/R}$,他同时也是$S/I=T$-模同态。综上,我们就有一列$T$-模同态
\[
	I/I^2\xrightarrow{D} T\otimes_S\Omega_{S/R}\xrightarrow{\varphi} \Omega_{T/R}\to 0,
\]
其中$D:f\mapsto 1\otimes df$. 下面我们证明这还是正合列。

\begin{pro}
设$R\to S\to T$是三个环,并且$\pi:S\to T$是满同态,记$I=\ker\pi$,我们有正合列
\[
	I/I^2\xrightarrow{D} T\otimes_S\Omega_{S/R}\xrightarrow{\varphi} \Omega_{T/R}\to 0,
\]
其中$D:f\mapsto 1\otimes df$,以及$\varphi:1\otimes df\mapsto df$.
\end{pro}

\begin{proof}
只要求$\ker \varphi$. 由于$T\otimes_S\Omega_{S/R}$由$\{1\otimes_S ds\,:\,s\in S\}$张成,并且,由于$S\to T$是满同态,所以$\Omega_{T/R}$由$\{ds\,:\,s\in S\}$张成。于是$\varphi$将生成元$1\otimes_S ds$变成生成元$ds$,并且$T\otimes_S\Omega_{S/R}$和$\Omega_{T/R}$满足相同的Leibniz法则和$R$-线性性,所以$\ker \varphi$由那些$1\otimes ds\neq 0$张成,其中$ds=0\in\Omega_{T/R}$,如果$1\otimes ds\neq 0$,则只可能$s\in I$. 此时$1\otimes ds=Ds\in\im D$,所以$\ker\varphi=\im D$. 
\end{proof}

利用同构基本定理,上面的正合列给出了同构
\[
	\Omega_{T/R}\cong \coker\left(I/I^2\xrightarrow{D} T\otimes_S\Omega_{S/R}\right).
\]
当$T$是一个有限生成$R$-代数,即$T=R[x_1,\cdots,x_n]/I$,取$S=R[x_1,\cdots,x_n]$,则
\[
	T\otimes_S\Omega_{S/R}=T\otimes_S\bigoplus_{i=1}^n Sx_i=\bigoplus_{i=1}^n Tx_i,
\]
所以
\[
	\Omega_{T/R}\cong \coker\left(I/I^2\xrightarrow{D} \bigoplus_{i=1}^n Tx_i\right),
\]
此时$\Omega_{T/R}$是一个有限生成$T$-模。