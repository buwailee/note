\chapter{线性代数}
\ThisULCornerWallPaper{1}{../Pictures/3.png}
\section{矢量空间与对偶空间}

\section{张量代数}

\section{本征值问题}

模论的方法给了我们研究线性代数的新思路。考虑一个非零$k$-矢量空间$V$,以及一个线性映射$A:V\to V$. 那么,我们可以通过$x\cdot v=Av$定义$V$的一个$k[x]$-模结构。此时,研究线性映射$A$,也就归结到了研究作为$k[x]$-模的$V$,即研究$k[x]$-模的标量作用。

此外,在有些命题中,$V$需要是一个有限维$k$-矢量空间,不难看到它此时是有限生成$k[x]$-模。反过来并不正确。所以有时候,条件可以放宽到有限生成$k[x]$-模,但有时候就只能放在有限维$k$-矢量空间上。

在标量作用中,零因子(生成的理想)是比较简单和重要的研究对象,从准素分解那里可以看到,associated primes是一类研究比较深入的对象。而将associated primes放到$k[x]$-模$V$上,我们自然就遇到了本征值问题。

\begin{para}[本征值与本征矢量]
    设$A:V\to V$是一个线性映射,在其对应的$k[x]$-模$V$中,如果$\langle x-a\rangle\in \Ass_{k[x]}(V)$,则称$a$为$A$的一个本征值,而对应的$\ann(v)=\langle x-a\rangle$中的$v$被称为本征值$a$对应的本征矢量。
\end{para}

如果$v$是本征值$a$对应的本征矢量,则$(x-a)v=0$等价于$Av=av$. 这就是经典的本征方程。

\begin{lem}
    如果$V$是有限维$k$-矢量空间,则首一多项式$\det(xI-A)\in \ann_{k[x]}(V)$.
\end{lem}

\begin{proof}
    参见Hamilton-Cayley定理Theorem \ref{Hamilton-Cayley}的证明。
\end{proof}

注意到,$k[x]$是PID,则$\ann_{k[x]}(V)$由某个多项式生成,我们将其中的那个首一多项式称为极小多项式,记作$p(V)$,而首一多项式$\det(xI-A)$称为本征多项式,不难看到,本征多项式的次数为$\dim V$.

\begin{pro}
    如果$k$是代数闭域,而$V$是$k$-矢量空间,$A:V\to V$是线性映射,如果对应的$\ann_{k[x]}(V)$是非零理想,则$A$存在本征值。特别地,如果$V$是有限维的,则任意线性映射$A:V\to V$都存在本征值。
\end{pro}

\begin{proof}
    因为$k[x]$总是Noether环,所以associated prime一定存在,同时,因为所有的associated prime一定包含$\ann_{k[x]}(V)$,所以它们是非零素理想。因为$k$是代数闭的,所以只能具有形式$\langle x-a\rangle$,此时$a$就是一个本征值。后一论断结合上一个引理是直接的。
\end{proof}

当然,对于非代数闭域,associated primes可能就不是一次式生成的(尽管依然是主理想),所以就可能不存在本征值。此外,对于无限维的情况,此时$\ann(V)$可以为零,所以也可能不存在本征值。

\begin{pro}
    设有限生成$k[x]$-模$V$的极小多项式为$p(V)$,则由于$k[x]$是PID,所以$p(V)$可以唯一分解为一族不可约首一多项式$\{p_i\}$的乘积,即
    \[
        p(V)=p_1^{i_1}\cdots p_k^{i_k},
    \]
    则$A$具有本征值$a$,当且仅当$(x-a)\in \{p_i\}$.
\end{pro}

\begin{proof}
    如果$A$具有本征值$a$,则$\langle x-a\rangle \in \Ass(V)$,其包含$\ann(V)=\langle p(V)\rangle$,因此$p(V)$的分解中必有因子$(x-a)$. 反过来,如果$p(V)$的分解中具有因子$(x-a)$,则$\langle x-a\rangle$是包含$\langle p(V)\rangle =\ann(V)$的一个极小素理想,由于$V$是有限生成的,Lemma \ref{lem:3.5.28}告诉我们$\supp(V)$由所有包含$\ann(V)$的素理想构成,继而Proposition \ref{pro:5.1.9}立刻告诉我们,$\langle x-a\rangle \in \Ass(V)$.
\end{proof}

\begin{pro}
    如果$V$是有限维矢量空间,则$\deg(p(V))\leq \dim (V)$. 特别地,$A$的本征值数目不大于$V$的维数。
\end{pro}

\begin{proof}
    依然是Hamilton-Cayley定理Theorem \ref{Hamilton-Cayley},因为$V$可以由$\dim V$个元素生成,所以$\ann(V)=\langle p(V)\rangle$包含一个最高次数为$\dim V$的多项式,这就给出了$\deg(p(V))\leq \dim (V)$. 此外,上一个命题告诉我们,$A$的本征值数目不能超过$\deg(p(V))$.
\end{proof}

\begin{para}[根子空间]
    设$p\in k[x]$是一个多项式,对$k[x]$-模$V$,考虑集合
    \[
        V(p)=\{v\in V\;:\; \text{存在自然数$n$使得$p(x)^nv=0$}\},
    \]
    不难检验,这是$V$的一个$k[x]$-子模(当然也是一个子空间),它被称为$V$的对应于多项式$p$的根子空间。
\end{para}

\begin{lem}\label{lem:6.3.9}
    设$k[x]$-模$V$的极小多项式可以分解为$p(V)=p_1^{n_1}\cdots p_k^{n_m}$,则存在一个$v\in V(p_i)$使得$\ann(v)=\langle p_i^{n_i}\rangle$.
\end{lem}

\begin{proof}
    实际上,我们可以找到一个$v\in V(p_i)$使得$p_i^{n_i-1}(x)v\neq 0$且$p_i^{n_i}(x)v=0$,此时自然有$\ann(v)=\langle p_i^{n_i}\rangle$. 现在,取$w\in V$,则
    \[
        v_w=\left(p_1^{n_1}\cdots p_{i-1}^{n_{i-1}}p_{i+1}^{n_{i+1}}\cdots p_{m}^{n_{m}}\right)(x)w,
    \]
    满足$p_i^{n_i}(x)v_w=0$. 但是,如果对所有的$w\in V$都有$p_i^{n_i-1}(x)v_w=0$,则多项式
    \[
        p_1^{n_1}\cdots p_{i-1}^{n_{i-1}}p_i^{n_i-1}p_{i+1}^{n_{i+1}}\cdots p_{m}^{n_{m}}
    \]
    作用在任意的$w\in V$都为零,所以它处于极小多项式生成的理想中,但他比极小多项式的次数严格小,矛盾。
\end{proof}

\begin{pro}\label{pro:6.3.6}
    设$V$是有限生成$k[x]$-模且极小多项式$p(V)$非零,则存在$v\in V$使得$\ann(v)=\ann(V)$.
\end{pro}

\begin{proof}
    设$p(V)$可以分解为$p(V)=p_1^{n_1}\cdots p_k^{n_m}$,首先,存在$v_i\in V(p_i)$且$\ann(v_i)=\langle p_i^{n_i}\rangle$. 此外,由于分解中的$p_i$都两两互素,所以任意两个$V(p_i)$都只相交于$\{0\}$. 因此,$\{v_i\}$是线性无关的,并且,由于每个$V(p_i)$都是$k[x]$-子模,所以任取$f\in k[x]$,则$\{f(x)v_i\}$中的非零矢量族是线性无关的。

    最后,考虑$v=\sum_i v_i$. 任取$f\in k[x]$使得$\deg f<\deg (p(V))$,则总存在一个$v_i$使得$f(x)v_i\neq 0$,因此$f(x)v\neq 0$,或者说$f\not\in \ann(v)$. 由于$\ann(v)$包含$\ann(V)$非零,所以他由一个多项式$q\in k[x]$生成,如果$\ann(v)$真包含$\ann(V)$,则$\deg q< \deg (p(V))$,所以$q\not\in \ann(v)$,矛盾,所以$\ann(v)=\ann(V)$.
\end{proof}

\begin{pro}\label{pro:6.3.7}
    设$k[x]$-模$V$是一个$n$-维$k$-矢量空间,$p(V)$为其极小多项式,$q(V)$为其本征多项式,则一下命题等价:
    \begin{compactenum}
        \item $\deg(p(V))=\dim V$;
        \item $p(V)=q(V)$;
        \item 存在$v\in V$使得$v$, $xv$, $\dots$, $x^{n-1}v$是线性无关的;
        \item $V$是一个循环$k[x]$-模。
    \end{compactenum}
\end{pro}

\begin{proof}
    其中只有1和3的等价性没那么平凡。如果$\deg(p(V))=\dim V=n$,上一个命题告诉我们,存在一个$v\in V$使得任取$(n-1)$-次多项式$q\in k[x]$都有$q(x)v\neq 0$,而这又意味着$v$, $xv$, $\dots$, $x^{n-1}v$是线性无关的。反过来,如果$v\in V$使得$v$, $xv$, $\dots$, $x^{n-1}v$是线性无关的,考虑$\ann(v)=\langle q\rangle$,其中$\deg q\geq n$. 但同时,因为$\ann(v)$包含$\ann(V)=\langle p(V)\rangle$,所以$\deg q\leq \deg(p(V))$,所以
    \[
        \dim V=n\leq \deg q\leq \deg(p(V)),
    \]
    结合$\deg(p(V))\leq \dim V$,我们就得到了$\deg(p(V))=\dim V$.
\end{proof}

下面我们首先研究循环$k[x]$-模。

\begin{pro}
    设有限维$k$-矢量空间$V$是一个循环$k[x]$-模,极小多项式的分解为$p(V)=p_1^{n_1}\cdots p_m^{n_m}$,则我们有分解
    \[
       V=\bigoplus_i V(p_i),
    \]
    且$\dim_k V(p_i)=n_i\deg(p_i)$. 此外,$V(p_i)$的极小多项式$p(V(p_i))$就是$p_i^{n_i}$,所以$V(p_i)$是一个循环$k[x]$-模。
\end{pro}

\begin{proof}
    从Lemma \ref{lem:6.3.9}可知$\langle p(V(p_i))\rangle=\ann (V(p_i))\subset \langle p_i^{n_i}\rangle$,所以$n_i\deg(p_i)=\deg (p_i^{n_i})\leq \deg (p(V(p_i)))$. 同时,$V(p_i)$作为$k[x]$-模,我们有$\deg (p(V(p_i))) \leq \dim V(p_i)$. 所以$n_i\deg(p_i)\leq \dim V(p_i)$. 
    
    另一方向,由于$V$是循环$k[x]$-模,所以$\deg p(V)=\dim V$,所以我们有
    \[
        \dim V=\sum_i n_i\deg(p_i)\leq \sum_i\dim V(p_i)\leq \dim V,
    \]
    其中最后一个不等号来自于$\{p_i\}$两两互素,直和$\bigoplus_i V(p_i)\subset \dim V$. 因此直和分解成立。此外,如果有一个$n_i\deg(p_i)$严格小于$\dim V(p_i)$,则这将违背$\dim V=\sum_i n_i$,所以$\dim V(p_i)=n_i\deg(p_i)$.

    最后,因为
    \[
        \deg(p_i^{n_i})\leq \deg (\deg(p(V)))\leq \dim V(p_i),
    \]
    所以从$\dim V(p_i)=n_i\deg(p_i)$以及Proposition \ref{pro:6.3.7}立刻得到$V(p_i)$是一个循环$k[x]$-模。
\end{proof}

有趣的是,我们还可以通过中国剩余定理来直接看到这个命题。甚至,我们可以这个定理推广到PID上的循环模,因为PID上任意两个不同的素元也是互素的。

依然是$R$是PID,而$M$是有限生成$R$-模,设$\ann_R(M)=\langle p_1^{n_1}\cdots p_k^{n_m}\rangle$,则$\langle p_i\rangle$是包含$\ann_R(M)$的极小素理想,设$\alpha_i:M\to M_{\langle p_i\rangle}$是局部化映射,则Proposition \ref{pro:5.2.13}告诉我们,$\ker \alpha_i$是一个$\langle p_i\rangle$-准素子模,且Theorem \ref{thm:5.2.17}告诉我们,

\begin{para}[本征空间]
    如果$a\in k$是线性映射$A:V\to V$的本征值,则所有以$a$为本征矢量的$v\in V$以及$0$构成了$V$的一个子空间,他被称为$A$的对应于本征值$a$的本征空间。

    显然,如果$v$, $w$的本征值为$a$,则$rv+sw\neq 0$的本征值也为$a$,所以定义合理。并且,从定义,不难看到$V$的本征空间是$V$的一个$k[x]$-子模。
\end{para}

\section{主理想整环上的有限生成模}

在这节中,设$R$是一个主理想整环,而$M$是一个有限生成模。我们首先利用中国剩余定理给出的一个循环$R$-模的刻画。

\begin{pro}
    设$R$是一个PID,$M$是一个循环$R$-模,且$\ann_R(M)=\langle p_1^{n_1}\cdots p_k^{n_m}\rangle$,其中$p_i$都是$R$中的素元(可以为零),此时,我们有$R$-模同构
    \[
        M\cong \bigoplus_i R/\langle p_i^{n_i}\rangle.
    \]
\end{pro}

\begin{proof}
    对循环$R$-模$M$,则模同态$R\to M$给出了$R$-模同构$M\cong R/\ann(M)$. 此外,由于$\{p_i\}$两两互素,所以中国剩余定理告诉我们,存在环同构
    \[
        R/\langle p_1^{n_1}\cdots p_k^{n_m}\rangle = R/\left(\bigcap_i \langle p_i^{n_i}\rangle\right)\cong \prod_i R/\langle p_i^{n_i}\rangle,
    \]
    这个环同构是商映射族$\pi_i:R\to R/\langle p_i^{n_i}\rangle$的积$\pi=\prod_i \pi_i:R\to \prod_j R/\langle p_i^{n_i}\rangle$的商。如果将两侧看成$R$-模,则作为环同态的$\pi$将给出一个$R$-模同态(正是$\pi_i$给出了$R/\langle p_i^{n_i}\rangle$的$R$-模结构),所以,我们有$R$-模同构
    \[
        M\cong \bigoplus_i R/\langle p_i^{n_i}\rangle
    \]
    此即所需。
\end{proof}

\begin{pro}
	如果$M$是有限生成自由$R$-模,则$M$的子模也是一个自由$R$-模。
\end{pro}

一般地,如果$R$不是PID,则这个命题未必成立。

\begin{proof}
	首先,可以设$M=R^n$. 然后对子模的$\rank$做归纳。对于$\rank(M)=1$的情况,此时$M$可以看成$R$,则$M$的子模对应于$R$的理想$\langle a\rangle$,同时由于$\ann_R (a)=0$,所以是一个自由模。

	现在假设命题对于$k<n$的情况都正确,那么任取$R^n$的子模$S$,考虑$S_1$是所有形如$(s_1,\dots,s_{n-1},0)$的元素的集合,以及$S_2$是所有形如$(0,\dots,0,s_n)$的元素的集合与$S$的交。显然,$S_1$同构于$R^{n-1}$的一个子模,归纳假设告诉我们这是自由的,所以设$\{\alpha_1,\dots,\alpha_k\}$是$S_1$的一组自由基。类似地,$S_2$同构于$R$的一个理想,如果$S_2$平凡,则$S=S_1$自由,否则$\rank S_2=1$,设基为$\{\beta=(0,\dots,0,t_n)\}$,这里$(t_1,\dots,t_n)\in S$. 

	最后,我们检查$\{\alpha_1,\dots,\alpha_k,\beta\}$是$S$的基。从最后一个坐标来看,$\beta$与$\{\alpha_1,\dots,\alpha_k\}$线性无关,同时$\{\alpha_1,\dots,\alpha_k\}$作为自由基本身也是线性无关的。然后,任取$\alpha=(s_1,\dots,s_n)\in S$,则存在$r\in R$使得$s_n=rt_n$,所以$\alpha-r\beta\in S_1$,同时$\{\alpha_1,\dots,\alpha_k,\beta\}$生成$S_1$,于是$\{\alpha_1,\dots,\alpha_k,\beta\}$生成$S$.
\end{proof}

\begin{para}[挠元、挠模]
	设$M$是一个$R$-模,如果对$m\in M$有$\ann(m)\neq 0$,则$m$被称为一个挠元。如果$M$中每个元素都是挠元,则$M$被称为一个挠模。如果对非零的$m\in M$都有$\ann(m)=0$,则称$M$是一个无挠模。
\end{para}

对$M$来说,其所有挠元构成的集合是它的一个子模,这被称为$M$的挠子模,记作$M_{\text{tor}}$. 显然,$M$无挠当且仅当$M_{\text{tor}}=0$. 所以,$M/M_{\text{tor}}$总是无挠的。

\begin{lem}
	设$R$是一个PID,$M$是一个有限生成无挠模,则$M$是一个自由模。
\end{lem}

PID上的自由模当然是无挠的。

\begin{proof}
	以后再证明。
\end{proof}

\begin{pro}
设$R$是一个PID,$M$是一个有限生成模,则成立分解
\[
	M=M_{\mathrm{tor}}\oplus M_{\mathrm{free}}
\]
这里$M_{\mathrm{free}}$是一个自由模,其秩只依赖于$M$.
\end{pro}

\begin{para}[根子模]
    设$r\in R$,对$R$-模$M$,局部化同态$\alpha_r:M\to M_r$的核$\ker \alpha_r$被称为$M$关于$r$的根子模,记作$M(r)$. 不难看到,
    \[
        M(r)=\{m\in V\;:\; \text{存在自然数$n$使得$r^nv=0$}\}.
    \]
\end{para}

\begin{lem}
    如果$r$和$s$互素,则$M(r)\cap M(s)=\{0\}$.
\end{lem}

\begin{proof}
    取$m\in M(r)\cap M(s)$,所以存在一个足够大的$n$使得$r^nm=0=s^nm$. 如果$r$, $s$生成单位理想,则$r^{n}$和$s^{n}$也生成单位理想\footnote{以素谱的语言,我们知道:如果$\{D(f_i)\}$构成了$\operatorname{Spec} R$的一个开覆盖,当且仅当$\{f_i\}$生成了$R$. 同时,再注意到$D(f)=D(f^n)$对任意的正整数$n$都成立。},因此存在$a$, $b\in R$使得$ar^{n}+bs^{n}=1$. 乘以$m$后立刻得到$m=0$.
\end{proof}

\begin{lem}
    设$M$是一个$R$-模且$\ann_R(M)=\langle p_1^{n_1}\cdots p_m^{n_m}\rangle \neq 0$,则存在一个$m_i\in M(p_i)$使得$\ann(m_i)=\langle p_i^{n_i}\rangle$,因此$\ann(M(p_i))=\langle p_i^{n_i}\rangle$.
\end{lem}

\begin{proof}
    我们可以找到一个$m_i\in M(p_i)$使得$p_i^{n_i-1}m_i\neq 0$且$p_i^{n_i}m_i=0$,此时自然有$\ann(m_i)=\langle p_i^{n_i}\rangle$. 实际上,此时,$p_i^{n_i}\in \ann(m_i)=\langle q_i\rangle$,所以$p_i^{n_i}=rq_i$,但是根据唯一分解,立刻得到$\langle q_i\rangle=\langle p_i^n\rangle$,其中$n\leq n_i$. 再从$p_i^{n_i-1}m_i\neq 0$,就有$\ann(m_i)=\langle q_i\rangle = \langle p_i^{n_i}\rangle$.
    
    现在,取$w\in M$,则
    \[
        m_w=\left(p_1^{n_1}\cdots p_{i-1}^{n_{i-1}}p_{i+1}^{n_{i+1}}\cdots p_{m}^{n_{m}}\right)w,
    \]
    满足$p_i^{n_i}m_w=0$. 但是,如果对所有的$m\in M$都有$p_i^{n_i-1}m_w=0$,则$R$中的元素
    \[
        p_1^{n_1}\cdots p_{i-1}^{n_{i-1}}p_i^{n_i-1}p_{i+1}^{n_{i+1}}\cdots p_{m}^{n_{m}}
    \]
    作用在任意的$m\in M$都为零,所以它处于极小多项式生成的理想中,这与唯一分解性质矛盾。

    最后,欲证$\ann(M(p_i))=\langle p_i^{n_i}\rangle$,现在只需说明任取$m\in M(p_i)$,我们都有$p_i^{n_i}m=0$. 这从零因子的分解是显然的。
\end{proof}

作为推论,我们有$\Ass(M(p_i))=\{\langle p_i\rangle\}$,如果存在$M$的一个子模$N$使得$M/N\cong M(p_i)$,此时$N$就是一个$\langle p_i\rangle$-准素子模,后面我们会具体构造出这个子模$N$.

\begin{pro}\label{pro:6.3.6}
    设$R$-模$M$是一个挠模,存在$m\in M$使得$\ann(m)=\ann(M)$.
\end{pro}

\begin{proof}
    设$\ann_R(M)=\langle p_1^{n_1}\cdots p_k^{n_m}\rangle$,根据上一个引理,存在$m_i\in M(p_i)$且$\ann(m_i)=\langle p_i^{n_i}\rangle$. 此外,由于分解中的$p_i$都两两互素\footnote{因为主理想整环的非零素理想是极大理想,所以两个不同的素元互素。},所以任意两个$M(p_i)$都只相交于$\{0\}$. 由于每个$M(p_i)$都是$k[x]$-子模,所以任取$r\in R$,则$rm_i$依然在$M(p_i)$中。

    最后,考虑$m=\sum_i m_i$. 由于$\ann(M)\subset \ann(m)=\langle q\rangle$,所以$p_1^{n_1}\cdots p_k^{n_m}=qr$,其中$r\in R$,于是唯一分解告诉我们
    \[
        \ann(m)=\langle p_1^{l_1}\cdots p_k^{l_m}\rangle,
    \]
    其中$l_i\leq n_i$. 由于$\ann(m_i)=\langle p_i^{n_i}\rangle$,将$p_1^{l_1}\cdots p_k^{l_m}$作用到$m_i$上,除非$l_i=n_i$,否则不为零,因此,只有当$l_i=n_i$对所有的$i$都成立时,$p_1^{l_1}\cdots p_k^{l_m}$乘以$m$才为零,此即$\ann(m)=\ann(M)$.
\end{proof}

\begin{pro}
    设$M$是一个有限生成$R$-模且$\ann_R(M)=\langle p_1^{n_1}\cdots p_k^{n_k}\rangle \neq 0$,则
    \[
        M=M(p_1)\oplus \cdots  \oplus M(p_k).
    \]
\end{pro}

\begin{proof}
    任取$m\in M$,由于$\ann_RM\subset \ann_R m=\langle q\rangle$,所以,可以设
    \[
        q=p_1^{l_1}\cdots p_k^{l_k},
    \]
    其中$l_i\leq n_i$. 现在,构造
    \[
    	q_i=p_1^{l_1}\cdots \widehat{p_i^{l_i}}\cdots p_k^{l_k},
    \]
    我们下面证明存在$r_i$使得$\sum_i r_iq_i=1$. 实际上,逐级下降即可,降到$k=2$的时候是平凡的,因为$\{p_i^{l_i}\}$两两互素。比方说,从$k$开始,此时存在$r$, $s$使得$rp_{k}^{l_{k}}+sp_i^{l_i}=1$,所以从
    \[
    q_i=p_1^{l_1}\cdots \widehat{p_i^{l_i}}\cdots p_k^{l_k} \quad\text{和}\quad q_{k}=p_1^{l_1}\cdots p_{k-1}^{l_{k-1}}
    \]
    可以构造出
    \[
    p_1^{l_1}\cdots \widehat{p_i^{l_i}}\cdots p_{k-1}^{l_{k-1}}=rq_i+sq_{k}.
    \]

    最后,注意到$m_i=q_im\in M(p_i)$,所以
    \[
    m=\sum_i r_iq_im=\sum_i r_im_i\in M(p_1)\oplus \cdots  \oplus M(p_k),
    \]
    此即所证。
\end{proof}

现在,我们来考察有限生成挠模$M$的准素子模。

\begin{pro}
    设$M$是一个有限生成$R$-模且$\ann_R(M)=\langle p_1^{n_1}\cdots p_m^{n_m}\rangle \neq 0$,则$\Ass_RM=\{\langle p_1\rangle ,\dots,\langle p_m\rangle \}$.
\end{pro}

\begin{proof}
    显然,$\{\langle p_1\rangle ,\dots,\langle p_m\rangle \}\subset \Ass_RM$,因为每个$\langle p_i\rangle$都是包含$\ann_R(M)$的极小素理想,见Proposition \ref{pro:5.1.9}. 所以,我们只需证明不存在其他的associated primes. 如果$\ann(m)=\langle q\rangle$是一个素理想,由于$\ann(M)\subset \ann(m)$,所以$p_1^{n_1}\cdots p_m^{n_m}=qr$,其中$r\in R$. 根据唯一分解,$q$只能等于某个$p_i$.
\end{proof}

 \begin{pro}
    设$M$是一个有限生成$R$-模且$\ann_R(M)=\langle p_1^{n_1}\cdots p_m^{n_m}\rangle \neq 0$,从而局部化映射$M\to M_{\langle p_i\rangle}$的核给出了$\langle p_i\rangle$-准素子模$M_i$,且$\bigcap_{i}M_i=\{0\}$.
\end{pro}

\begin{proof}
    前一个论断就是Proposition \ref{pro:5.2.13},我们只需证明$\bigcap_{i}M_i=\{0\}$. 取$m\in \bigcap_{i}M_i$,所以存在$r_i\not\in \langle p_i\rangle$使得$r_im=0$. 因为$\ann(M)\subset \ann(m)$,所以可设$\ann(m)=\langle p_1^{l_1}\cdots p_m^{l_m}\rangle$. 从$r_i\not\in \langle p_i\rangle$,立刻得到$l_i=0$,所以$\ann(m)=R$,继而$m=0$.
\end{proof}

此时,可以看到$\bigcap_i M_i=\{0\}$就是$\{0\}$的极小分解了,从而利用Theorem \ref{thm:5.2.17}给出了如下推论。

\begin{coro}
设$M$是一个有限生成$R$-挠模,且$\ann_R(M)=\langle p_1^{n_1}\cdots p_k^{n_k}\rangle \neq 0$,则$N_i=M(p_1)\oplus \cdots \oplus \widehat{M(p_i)}\oplus \cdots  \oplus M(p_k)$是$M$的一个$\langle p_i\rangle$-准素子模,且$N_i=\ker(\alpha_i)$,这里$\alpha_i:M\to M_{\langle p_i\rangle}$是典范的局部化同态。
\end{coro}

\begin{proof}
    只需注意到$\Ass(M/N_i)=\Ass(M(p_i))=\{\langle p_i\rangle\}$以及$\bigcap_i N_i=\{0\}$即可。
\end{proof}

但这并没有给出所有的$\langle p_i\rangle$-准素子模。如果$M(p_i)$存在直和分解$M(p_i)=M'\oplus M''$,则我们有$\Ass(M')=\Ass(M'')=\Ass(M(p_i))$,此时
\[
    N_i\oplus M'\quad \text{和}\quad  N_i\oplus M'',
\]
都是$\langle p_i\rangle$-准素子模。但是,从某种角度而言,我们给出了“极小的”准素子模们。

\begin{para}
迄今为止,给定一个有限生成$R$-模,则我们有分解
\[
    M=M_{\mathrm{free}}\oplus M_{\mathrm{tor}},
\]
并且对挠模$M_{\mathrm{tor}}$,如果$\ann(M_{\mathrm{tor}})=\langle p_1^{n_1}\cdots p_k^{n_k}\rangle$,我们有分解
\[
    M_{\mathrm{tor}}=M(p_1)\oplus \cdots  \oplus M(p_k),
\]
所以我们有分解
\[
    M=M_{\mathrm{free}}\oplus M(p_1)\oplus \cdots  \oplus M(p_k).
\]
\end{para}

最后,我们来分解$M(p_i)$,注意到$\ann(M(p_i))=\langle p_i^{n_i}\rangle$,而$p_i$是一个素元。

\begin{pro}
设$M$是一个$R$-挠模且$\ann(M)=\langle p^n\rangle$,其中$p\in R$是一个素元,则$M$可以写成几个循环$R$-模的直和。
\end{pro}

\section{矢量空间在线性算子下的分解}
