\chapter{Banach空间的对偶空间}

\section{有界线性映射空间的范数}

\begin{para}
设$X$和$Y$是两个拓扑矢量空间,记$\mathcal{B}(X,Y)$是所有$X$到$Y$的有界线性映射构成的集合。不难看到$\mathcal{B}(X,Y)$有一个自然的矢量空间结构,因为两个有界线性映射相加依然是有界的,而一个有界线性映射乘以一个标量还是有界的。

设$X$是可度量化空间,因为可度量化空间上的线性映射的有界性等价于连续性,所以此时$\mathcal{B}(X,Y)$就是$X\to Y$的所有连续线性映射的构成的矢量空间,特别地,$X^*=\mathcal{B}(X,k)$.

如果$X$和$Y$都是赋范空间,则$\mathcal{B}(X,Y)$也将拥有一个自然的范数,使得$\mathcal{B}(X,Y)$也是一个赋范空间。设$f\in \mathcal{B}(X,Y)$,定义
\[
	|f|=\sup\bigl\{|f(x)|\,:\,x\in X,\,|x|\leq 1\bigr\},
\]
由于$f$是有界的,所以$|f|<\infty$. 我们下面验证,$|f|$构成了$\mathcal{B}(X,Y)$上的一个范数。

\begin{proof}
首先证明三角不等式$|f+g|\leq |f|+|g|$,其中$f$, $g\in \mathcal{B}(X,Y)$. 由于对任意满足$|x|\leq 1$的$x\in X$,有
\[
	|(f+g)x|=|fx|+|gx|\leq |f|+|g|,
\]
所以三角不等式$|f+g|\leq |f|+|g|$成立。此外,对标量$\alpha\in k$,由于$|\alpha f(x)|=|\alpha||f(x)|$,所以
\[
	|\alpha f|=|\alpha||f|.
\]
最后,对$f\neq 0$,由于存在一个$x\in X$使得$f(x)\neq 0$,此时$|f(x/|x|)|=|f(x)|/|x|>0$,所以$|f|\neq 0$.
\end{proof}

关于这个范数有一个有用的不等式$|f(x)|\leq |f||x|$:显然,$|x/|x||=1$,所以由
\[
	\left|f\left(\frac{x}{|x|}\right)\right|\leq |f|
\]
可以推知$|f(x)|\leq |f||x|$. 
\end{para}

\begin{pro}
如果$Y$是Banach空间,则$\mathcal{B}(X,Y)$依上述范数也构成一个Banach空间。
\end{pro}

注意,这与$X$是否完备无关。因为标量域$k$,$\rr$或者$\cc$都是完备的,所以$X^*$在上面定义的范数下总是完备的。

\begin{proof}
	设$\{f_n\}$是$\mathcal{B}(X,Y)$中的一个Cauchy列,任取$x\in X$,我们有不等式
	\[
	|f_n(x)-f_m(x)|\leq |f_n-f_m||x|,
	\]
	所以$\{f_n(x)\}$是$Y$中的一个Cauchy列,它的极限记作$f(x)$. 遍历$x\in X$,我们就得到了一个线性映射$f:X\to Y$. 为证明其有界性,给定一个小正数$\epsilon$,注意到不等式
	\[
	|f(x)-f_n(x)|\leq \epsilon |x|
	\]
	对足够大的$n$一定成立,因此
	\[
	|f(x)|\leq |f_n(x)|+\epsilon|x|\leq \bigl(|f_n|+\epsilon\bigr) |x|,
	\]
	此外,由于$\{f_n\}$是一个Cauchy列,所以对于足够大的$m$和$n$成立$|f_n-f_m|\leq \epsilon$,固定一个$m$,于是对足够大的$n$满足$|f_n|\leq |f_m|+\epsilon$. 所以,对足够大的$n$,我们有
	\[
	|f(x)|\leq \bigl(|f_n|+\epsilon\bigr) |x|\leq \bigl(|f_m|+2\epsilon\bigr) |x|
	\]
	对任意的$x$都成立。于是$f:X\to Y$是一个有界线性映射。

	重新利用
	\[
	|f(x)-f_n(x)|\leq \epsilon |x|,
	\]
	对$|x|\leq 1$时,我们有$|f-f_n|\leq \epsilon$,即$\lim_{n\to\infty}f_n=f$.
\end{proof}

以后,我们可能会使用$x^*$来表示$X^*$里面的元素,而$x^*(x)$会被记作$\langle x^*,x\rangle$. 于是,我们有不等式$|\langle x^*,x\rangle|\leq |x^*||x|$. 采用这个记号,是因为我们需要考虑$X^*$的对偶空间$(X^*)^*$. 将$(X^*)^*$简记为$X^{**}$.

\begin{lem}
	若$X$是一个赋范空间,则$x\mapsto \langle x^*,x\rangle$将定义一个$X^*$上的有界线性泛函$\psi(x)$,在$X^{**}$的范数下,我们有等式$|\psi(x)|=|x|$.
\end{lem}

线性是显然的,而有界性直接来自于不等式$|\psi(x)(x^*)|=|\langle x^*,x\rangle|\leq |x^*||x|$. 所以$\psi(x)$是$X^*$上的有界线性泛函。此外,对$|x^*|\leq 1$,我们有
\[
	|\psi(x)(x^*)|=|\langle x^*,x\rangle|\leq |x^*||x|\leq |x|.
\]
于是$|\psi(x)|\leq |x|$,下面我们证明存在一个$y^*$使得$\langle y^*,x\rangle=|x|$,于是$|\psi(x)|=|x|$.

考虑$X$中由$x$张成的子空间$\langle x\rangle$,通过$\alpha x\mapsto \alpha|x|$,其中$\alpha$是$k$中任意的标量,可以定义$\langle x\rangle$上的一个线性泛函$f:\langle x\rangle\to k$. 所以,任取$z\in \langle x\rangle$,我们有$|f(z)|=|z|$,由Hahn-Banach定理,我们可以将$f$延拓成$X$上的一个有界线性泛函$y^*$,满足$|y^*(z)|\leq |z|$,所以$|y^*|\leq 1$且
\[
	|\psi(x)(y^*)|=\langle y^*,x\rangle = f(x) =|x|,
\]
此即所需。

\begin{para}
从上一个引理,我们有了一个线性映射$\psi:X\to X^{**}$,由等式$|\psi(x)|=|x|$,$\psi$是一个等距映射,即成立$|\psi(x)-\psi(y)|=|x-y|$. 此外,如果$\psi(x)=0$,则由等式$|\psi(x)|=|x|$可以推出$x=0$,所以$\psi$还是一个单射。

于是,$X$等距同构于$X^{**}$的子空间$\psi(X)$. 通常把$X$等同于$\psi(X)$,即把$x$等同于$\psi(x)$,所以$X$可以看成$X^{**}$的一个子空间。如果$X$是Banach空间,则$X$还可以看成$X^{**}$的闭子空间。

由于$\langle x^*,\psi(x)\rangle=\langle x,x^*\rangle$,而$x$也已经等同于$\psi(x)$,所以在记号上,我们有$\langle x^*,x\rangle=\langle x,x^*\rangle$. 

对有限维空间,或者某些无限维空间(比如$\mathsf{L}^1(\mu)$),我们有$X=X^{**}$. 这样的空间被称为自反的。反过来,比如当$X$不是完备的,则$X$必然是$X^{**}$的真子空间,因为$X^{**}$总是完备的。此外,即使存在$X$到$X^{**}$的等距同构$\varphi$,如果$\langle x^*,x\rangle=\langle \varphi(x),x^*\rangle$不成立,则$X$依然有可能是$X^{**}$的真子空间。
\end{para}

\begin{pro}
设$X$, $Y$都是赋范拓扑矢量空间,而$x_0\in X$以及$f\in \mathcal{B}(X,Y)$,则
\[
	|x_0|=\sup\left\{|\langle x^*,x_0\rangle|\,:\, |x^*|\leq 1\right\},\quad
	|f|=\sup \left\{|\langle y^*,f(x)\rangle |\,:\, |x|\leq 1,\, |y^*|\leq 1\right\}.
\]
\end{pro}

这个命题的证明是简单的,从前面已经知道,对$|x^*|\leq 1$的$x^*$,成立$|\langle x^*,x_0\rangle|\leq |x^*||x_0|\leq |x_0|$. 并且,从上一个Lemma的证明中已经知道,存在一个$x_1^*$使得$|x_1^*|\leq 1$且$\langle x_1^*,x_0\rangle=|x_0|$,所以第一个等式得证。

对第二个等式,应用第一个等式,我们有
\[
	|f(x)|=\sup\left\{|\langle y^*,f(x)\rangle|\,:\, |y^*|\leq 1\right\},
\]
然后应用$|f|$的定义
\[
	|f|=\sup\left\{|f(x)|\,:\, |x|\leq 1\right\},
\]
所以
\[
	|f|=\sup \left\{|\langle y^*,f(x)\rangle |\,:\, |x|\leq 1,\, |y^*|\leq 1\right\}.
\]

\section{伴随算子}

\begin{thm}
设$X$和$Y$都是赋范空间,对每一个$f\in \mathcal{B}(X,Y)$,我们都可以定义唯一的一个$f^*\in \mathcal{B}(Y^*,X^*)$,使得对任意的$x\in X$以及$y^*\in Y$成立
\[
	\langle f^*(y^*),x\rangle=\langle y^*,f(x)\rangle.
\]
此外,还成立$|f^*|=|f|$.
\end{thm}

$f^*$即所谓的拉回,对于有限维空间,$f^*$即$f$的转置。不难看到,它的定义为$f^*(y^*)=y^*\circ f$. 唯一性来自于等式对任意$x$都成立,因为$f^*(y^*)$是一个$X$上的线性函数。$f^*$线性的检验是平凡的,最后我们只需要证明$|f^*|=|f|$,这也就顺便说明了$f^*$有界。

由上一节最后的命题,
\begin{align*}
|f|&=\sup \left\{|\langle y^*,f(x)\rangle |\,:\, |x|\leq 1,\, |y^*|\leq 1\right\}\\
&=\sup \left\{|\langle f^*(y^*),x\rangle |\,:\, |x|\leq 1,\, |y^*|\leq 1\right\}\\
&=\sup \left\{|f^*(y^*)|\,:\, |y^*|\leq 1\right\}\\
&=|f^*|.
\end{align*}