\chapter{拓扑矢量空间}

\section{一些拓扑学基础}

一个拓扑空间$X$被称为\textit{紧}的,如果对于他的任意开覆盖$\{U_\alpha\}_{\alpha\in I}$,都存在$I$的某个有限子集$J$使得$\{U_\alpha\}_{\alpha\in J}$也是$X$的开覆盖。或者说简单点,$X$是紧的,如果对于任意$X$的开覆盖都存在有限子覆盖。

称$X$的一个子集$E$是紧的,如果$E$继承$X$的子空间拓扑时是一个紧拓扑空间。直接的判别法是,$X$中任意$E$的开覆盖都可以找到有限子覆盖。比如,$X$中的有限集一定是紧集。

\begin{pro}
下列命题等价:
\begin{compactenum}[(1)]
\item $X$是紧的。

\item 对任意的闭集族$\{V_\alpha\}_{\alpha\in I}$,如果$\bigcap_{\alpha\in I} V_\alpha=\varnothing$,则存在有限子集$J\subset I$使得$\bigcap_{\alpha\in J} V_\alpha=\varnothing$.

\item 对任意的闭集族$\{V_\alpha\}_{\alpha\in I}$,如果对任意有限子集$J\subset I$都有$\bigcap_{\alpha\in J} V_\alpha\neq\varnothing$,则$\bigcap_{\alpha\in I} V_\alpha\neq\varnothing$.
\end{compactenum}
\end{pro}

利用第二点,可以知道紧空间的闭子集也是紧的。利用第三点,我们可以知道,如果$X$是紧的,则其中任意递减非空闭集构成的链中所有元素的交非空。

\begin{proof}
	(1)等价(2)因为交和并可以用补集联系,而(2)等价(3)因为他们是逆否命题。
\end{proof}

\begin{para}
设$X$是一个拓扑空间,如果对于任意不同的两点$x$和$y$,他们有不相交的开邻域,则称$X$是一个Hausdorff空间。换而言之,点与点是可以用开集分离的。很容易看出,Hausdorff空间的子空间是Hausdorff空间,有限个Hausdorff空间的积是Hausdorff空间。

这里可以给一个直接的论断,可知Hausdorff空间中的单点集是闭集。因为单点集的补中任意一点都可以用开集与单点集分离。

更一般地,Hausdorff空间中的紧集是闭集。实际上,设$E$是紧的,而$x\not\in E$,由于$X$是Hausdorff空间,所以对任意的$y\in E$,都存在开集$U_y$和$U_x$使得$x\in U_x$和$y\in U_y$且$U_y\cap U_x=\varnothing$. 遍历$y$,$U_y$构成了$E$的开覆盖,进而有有限子覆盖$U_{y_1}$, $\dots$, $U_{y_n}$,然后$x$的开邻域$\bigcap_{i=1}^n U_{x_i}$就与它们无交,进而与$E$无交,故$x\in \bigcap_{i=1}^n U_{x_i}\subset X-E$,这就告诉我们$E$是闭集。
\end{para}

\begin{pro}
设$X$是拓扑空间,而$Y$是Hausdorff空间,且$f:X\to Y$是连续映射,则$f$的图像$\Gamma(f):=\{(x,f(x))\in X\times Y\,:\, x\in X\}$是一个闭集。
\end{pro}

\begin{proof}
设$E=X\times Y- \Gamma(f)$,取$(x,y)\in E$,由于$f(x)\neq y$,所以存在$x$的开邻域$U_{f(x)}$和$y$的开邻域$U_y$分离。由于$f$连续,所以存在$x$的开邻域$U_x$使得$f(U_x)\subset U_{f(x)}$. 注意到,$U_x\times U_y$是$(x,y)$的开邻域且$U_x\times U_y\subset E$. 所以$E$是开集,或者说$\Gamma(f)$是闭集。
\end{proof}

\begin{pro}\label{pro.1.4}
设$X$是拓扑空间,则$X$是Hausdorff空间当且仅当$\id_X:X\to X$的图像$\Delta=\{(x,x)\in X\times X\,:\, x\in X\}$是一个闭集。
\end{pro}

\begin{proof}
如果是Hausdorff空间,$\Delta$是闭集由上一个命题保证。反过来,如果$\Delta=\{(x,x)\in X\times X\,:\, x\in X\}$,令$E=X\times X-\Delta$. 任取$(x,y)\in E$,由于矩形邻域构成了积空间的一组基,所以存在开邻域$U_x\times U_y\subset E$,而$U_x$和$U_y$就分离了$x$与$y$,因此$X$是Hausdorff空间。
\end{proof}

\begin{para}
称一个Hausdorff空间$X$在$x$处局部紧\footnote{实际上,局部紧性可以脱离Hausdorff空间来定义:如果对$x$存在一个紧集包含它的一个邻域,则$X$在$x$处是局部紧的。定义的相容性这里按下不表。},如果对于$x$的任意开邻域$U$,都存在一个$x$的开邻域$V$使得其闭包$\bar{V}$紧,且$\bar{V}\subset U$. 如果$X$处处是局部紧的,则称其为局部紧Hausdorff空间。不难从定义看出,紧Hausdorff空间是局部紧的,局部紧Hausdorff空间的任意非空开子集或者闭子集都是局部紧Hausdorff空间。

一个局部紧Hausdorff空间可以看成一个紧Hausdorff空间的一个子空间(即继承了子空间拓扑),加一点紧致化定理保证了这点。

\begin{quote}\it
	加一点紧致化定理:设$X$是一个局部紧Hausdorff空间,则存在一个紧Hausdorff空间$Y$使得$X$是$Y$的子空间,且$Y-X$是一个单点集。并且,$Y$在同胚意义下唯一。
\end{quote}

上述定理的证明这里不打算给出,但是可以给出$Y$的拓扑的描述:$Y$中的开集,要么是$X$的开集,要么是$Y-C$的形式,其中$C$是$X$中的紧集。

于是我们可以给出局部紧Hausdorff空间的另一个等价描述:$X$同胚于紧Hausdorff空间的一个开子集当且仅当$X$是一个局部紧Hausdorff空间。
\end{para}

\begin{para}
	设$X$是一个拓扑空间,而$\{x_i\,:\, i\geq 0\}$是其中的一个序列,称$x\in X$是这个序列的极限,如果任取$x$的一个开邻域$U$,都存在一个正整数$n(U)$使得当$i>n(U)$时有$x_i\in U$. 称一个序列是收敛的,如果他有一个极限。

	对于非Hausdorff空间来说,序列可能有几个不同的极限,这并不是很奇怪的事情,因为可能包含一个点的开集一定包含着另一点,从极限的定义就知道了此时极限会有多个。但是对Hausdorff空间,极限如果存在则只有一个。
\end{para}

\begin{lem}
设$X$是一个拓扑空间,则
\begin{compactenum}[(a)]
\item 如果$E$是$X$的一个子集,若存在$E$中点的序列收敛于$x$,则$x\in \bar{E}$.
\item 设$Y$也是一个拓扑空间,而$f:X\to Y$连续,则对$X$中任意的收敛序列$x_i\to x$,序列$\{f(x_i)\}$收敛于$f(x)$.
\end{compactenum}
\end{lem}

\begin{para}
设$X$是一个拓扑空间,它的子集$E$如果其闭包的补$X-\bar{E}$是一个稠密开集,则$E\subset X$被称为无处稠密的。由于$X-\bar{E}$是$X-E$的内部,所以$E$是无处稠密的等价于它的补有稠密内部。此外,因为$X-(\bar{E})^\circ=\overline{X-\bar{E}}=X$,所以$E$无处稠密又等价于$\bar{E}$有空内部。显然,一个无处稠密集$E$的子集$S$也是无处稠密的,因为$S$闭包的内部包含于$E$闭包的内部,同样是空集。此外,空集显然是无处稠密的,无处稠密结的闭包是无处稠密的。

$X$中的第一纲集被定义为无处稠密集的可数并。注意到,取一个有限无处稠密族,然后补上可数个空集,则可知有限无处稠密族的并是第一纲集。$X$中不是第一纲的子集被称为第二纲集。显然,第一纲集的子集是第一纲集,第一纲集的任意可数并也是第一纲集,无处稠密集是第一纲集,所以比如有空内部的闭集是第一纲集。

利用第一纲集的子集是第一纲的,所以可以给出第一纲集一个更宽松的定义:$X$中的第一纲集是一个无处稠密闭集的可数并的子集,或者说是包含于一个无处稠密闭集的可数并中。实际上,考虑无处稠密集的可数并$\bigcup E_\alpha$中,则它也包含于$\bigcup \bar{E}_\alpha$中,其中$\bar{E}_\alpha$也是无处稠密的。于是,第二纲集的定义就可以写作:不能包含于任意无处稠密闭集的可数并中。

称一个空间$X$为Baire空间,如果给定$X$中可数个无处稠密闭集,则它们的并依然有空内部。如果用开集表述,则是:给定$X$中可数个稠密开子集,则它们的交依然在$X$中稠密。
\end{para}

\begin{lem}
	一个空间$X$是Baire空间当且仅当$X$的任意非空开集是第二纲集。
\end{lem}

\begin{proof}
	设$X$是Baire空间,而$U$是一个非空开集,只要证明不是第一纲集即可。反正,如果$U$包含于一个无处稠密集的可数并$\bigcup E_\alpha$中,则它也包含于$\bigcup \bar{E}_\alpha$中,由于$X$是Baire空间,我们可知$\bigcup \bar{E}_\alpha$有空内部,于是$U\subset \varnothing$与$U$非空矛盾。

	反过来,设$X$的任意非空开集是第二纲集。如果$X$不是Baire空间,则存在一个无处稠密闭集的可数族,使得它们的并的内部$U$非空。此时,$U$作为非空开集是第二纲集,但同时,$U$作为第一纲集的子集,是第一纲集,矛盾。
\end{proof}

作为推论,如果$X$是Baire空间,$X$在$X$中是第二纲集。

\begin{thm}[Baire纲定理]
	如果$X$是完备度量空间或者局部紧Hausdorff空间,则$X$是一个Baire空间。
\end{thm}

\begin{proof}
	设$\{V_n\}$是$X$中的可数稠密开子集族,而$U_0$是$X$中的任意非空开集,我们要证明$U_0\cap \bigcap V_n$非空,于是$\bigcap V_n$稠密。

	采用归纳法,若对$1$, $\dots$, $n-1$,我们已经构造了一族非空开集$U_{n-1}\subset U_{n-2}\subset \cdots \subset U_1\subset U_0$,取非空开集$U_n$使得
	\[
	\overline {U_n}\subset V_n\cap U_{n-1},
	\]
	因为$V_n$稠密,所以右侧非空。对完备度量空间,可以取$U_n$是半径小于$1/n$的球,对局部紧Hausdorff空间,可以取$\overline {U_n}$为紧集。然后考虑集合
	\[
	K=\bigcap_{n=1}^\infty \overline {U_n},
	\]
	对完备度量空间,由于$\overline {U_n}$的中心构成Cauchy列,所以$K$非空,对局部紧Hausdorff空间,从$\overline {U_1}$是紧的,考虑$\overline {U_1}$中的闭集链$\overline {U_1}\supset \overline {U_2}\supset \cdots$,因为它们的任意有限交非空,所以它们的交非空。

	由构造,$K$包含于每一个$V_n$,且$K$包含于$U_0$,所以$U_0\cap \bigcap V_n$非空。
\end{proof}

\section{拓扑群}

一个群$G$是拓扑群,如果群上有一个拓扑结构使得群乘法和逆都是连续映射。由于左乘是一个同胚,所以如果$U$是一个开集(闭集),那么$gU=\{gh\,:\,h\in U\}$也是一个开集(闭集)。对应到拓扑矢量空间,如果$U$是一个开集(闭集),那么$a+U=\{a+x\,:\,x\in U\}$也是一个开集(闭集)。

对拓扑群$G$,以及子集$A$, $B$,引入两个记号:$A^{-1}=\{a^{-1}\,:\,a\in A\}$以及$AB=\{ab\,:\,a\in A,\, b\in B\}$. 如果$U$是一个开集(闭集),则$U^{-1}$也是一个开集(闭集)。任取开集$U$,以及子集$A$,由于$AU=\bigcup_{a\in A}aU$,所以$AU$是开集,类似地,$UA$也是开集。

从定义来看,如果$A=A_0\cup A_1$,则$BA=BA_0\cup BA_1$,同样,如果$A=A_0\cap A_1$,则$BA=BA_0\cap BA_1$. 因此至少对于有限交或者有限并来说,乘积是可以分配进去的。如果单位元的一个邻域$U$满足$U^{-1}=U$,这样的开集就被称为对称的。

\begin{lem}
设$U$是单位元的一个邻域,则存在单位元的一个对称邻域$V$使得$V\subset VV\subset U$. 
\end{lem}

\begin{proof}
	由于乘法是连续的,所以存在$e$的邻域$V_1$和$V_2$使得$V_1V_2\subset U$. 令$V=(V_1\cap V_1^{-1})\cap (V_2\cap V_2^{-1})$. 显然这是对称的,并且$VV\subset U$. 由于$e\in V$,所以$V=eV\subset VV\subset U$.
\end{proof}

\begin{lem}\label{lem:2}
设$A\subset G$是一个拓扑群的一个子集,记$\mathscr{O}$是所有包含单位元$e$的开集构成的集合,则对$A$的闭包$\overline{A}$有等式
\[
	\overline{A}=\bigcap_{U\subset \mathscr{O}}AU=\bigcap_{U\subset \mathscr{O}}\overline{AU}.
\]
\end{lem}

\begin{proof}
	任取$x\in \overline{A}$以及$U\in \mathscr{O}$. 我们有$xU^{-1}$是$x$的一个邻域,于是存在$a\in A\cap x U^{-1}$,记$a=xu^{-1}$,我们有$x=au\in AU$. 于是$\overline{A}\subset \bigcap_{U\subset \mathscr{O}}AU$. 这也给出了一个下面会用到的结论:任取$e$的一个开集$U$,有$\overline{A}\subset AU$.

	反过来,考虑闭集的并$\bigcap_{U\subset \mathscr{O}}\overline{AU}$,这是一个闭集,且有显然的包含关系$\bigcap_{U\subset \mathscr{O}}AU\subset \bigcap_{U\subset \mathscr{O}}\overline{AU}$. 任取$x\in \bigcap_{U\subset \mathscr{O}}\overline{AU}$以及$U\in \mathscr{O}$,由上一个引理,存在$U'\in \mathscr{O}$使得$U'U'\subset U$. 由于$x\in \overline{AU'}\subset AU'U'\subset AU$,所以有$\bigcap_{U\subset \mathscr{O}}\overline{AU}\subset \bigcap_{U\subset \mathscr{O}}AU$. 进而$\bigcap_{U\subset \mathscr{O}}AU=\bigcap_{U\subset \mathscr{O}}\overline{AU}$.

	令$U_x$是$x\in \bigcap_{U\subset \mathscr{O}}AU$的一个邻域,再令$U=U_x^{-1}x$,他是$e$的一个邻域且$x\in AU$. 于是存在$a\in A$使得$a\in xU^{-1}=xx^{-1}U_x=U_x$,这就是在说$A\cap U_x\neq \varnothing$. 假设$x\in G-\overline{A}$,则$x$附近一定有一个邻域$U_x$使得$U_x\subset G-\overline{A}$,这与$A\cap U_x\neq \varnothing$矛盾,所以$x\in \overline{A}$.
\end{proof}

% 对于$A=\{e\}$的情况会特别有趣。记$H=\overline{\{e\}}=\bigcap_{U\subset \mathscr{O}}U$. 于是$x\in H$就是在说,对于任意包含$e$的开集$U$,$x\in U$. 我们下面证明这是一个子群。

% 任取$x\in H$,由于$U$是包含$e$的开集当且仅当$U^{-1}$是包含$e$的开集,所以$x^{-1}\in H$. 任取$x$, $y\in H$,任取$xy$的一个邻域$U$,我们要证明他总是包含$e$的,这样就有$xy\in H$. 由于$e\in U(xy)^{-1}$,由于$x^{-1}\in H$,所以$x^{-1}\in U(xy)^{-1}$. 两边乘以$x$得到$e \in Uy^{-1}$. 类似地,有$y^{-1}\in Uy^{-1}$,两边乘以$y$得到$e\in U$. 于是$H$是$G$的一个子群,而且还是一个闭子群。

\begin{pro}\label{pro:3}
设$G$是一个拓扑群,如果$\{e\}$是一个闭集,则

\begin{compactenum}
\item $G$中任意单点集是一个闭集,这样的拓扑空间被称为$\mathsf{T}_1$空间。

\item $G$是一个Hausdorff空间,或者被称为$\mathsf{T}_2$空间。

\item $G$是正则Hausdorff空间\footnote{单点集是闭集,且任取一个$x$以及一个不包含$x$的闭集$V$,存在两个不相交的开集$U_x$和$U_V$使得$x\in U_x$, $V\subset U_V$. 换而言之,点与开集是可以用开集分离的。},或者被称为$\mathsf{T}_3$空间。
\end{compactenum}
\end{pro}

\begin{proof}
	对于第一点,任意一个单点集都是单位元的平移,所以成立。因为$\mu:(x,y)\mapsto x^{-1}y$是连续映射且$\{e\}$闭集,$\Delta=\mu^{-1}(e)\subset G\times G$是闭集,其中$\Delta=\{(x,x):x\in G\}$,利用Hausdorff性的对角线判别法Proposition \ref{pro.1.4},$G$是一个Hausdorff空间。而一个$\mathsf{T}_1$空间是$\mathsf{T}_3$的,当且仅当对任意的$x$以及他的一个邻域$U$,存在$x$的邻域$V$使得$\overline{V}\subset U$. 由于$x^{-1}U$是$e$的一个邻域,所以存在对称开集$U'$满足$U'\subset U'U'\subset x^{-1}U$. 所以$xU'\subset xx^{-1}U=U$是$x$的一个邻域,满足$\overline{xU'}\subset xU'U'\subset xx^{-1}U=U$.
\end{proof}

脱离上面的命题,下面给出一个分离性条件的直接证明:

\begin{pro}
设$V$是拓扑群$G$的一个闭子集,而$K$是$G$的一个紧子集,并且$V\cap K=\varnothing$. 那么存在$e$的一个邻域$U$使得$VU\cap KU=\varnothing$.
\end{pro}

由于有限集是紧集,当假设$\{e\}$是闭集的时候,这个命题可以直接推出$\mathsf{T}_3$条件。

\begin{proof}
	假定$K$非空,否则命题显然正确。已经知道,任意一个包含原点的开集$U$,可以找到对称开集$V$使得$VV\subset U$. 现在任取$x\in K$,于是$G-V$是一个开集,所以可以找到包含原点的开集$U_x$使得$xU_x\cap V=\varnothing$. 取对称开集$U'_x$使得$U'_xU'_x\subset U_x$,再取对称开集$U''_x$使得$U''_xU''_x\subset U'_x$此时$xU'_xU''_xU''_x\subset xU'_xU'_x\subset xU_x$,所以$xU'_xU''_xU''_x\cap V=\varnothing$. 

	任取$y\in xU'_xU''_x\cap VU''_x$,由于$y\in xU'_xU''_x$,所以$yU''_x\subset xU'_xU''_xU''_x$,这意味着$yU''_x\cap V=\varnothing$,或者说,任取$a\in U''_x$,都有$ya\not\in V$. 另一方面,由于$y\in VU''_x$,所以一定存在一个$b\in U''_x$和$v\in V$使得$y=vb$成立,反过来,$yb^{-1}=v\in V$,由于$U''_x$是对称的,所以$b^{-1}\in U''_x$,矛盾。因此$xU'_xU''_x\cap VU''_x=\varnothing$.

	遍历$x$,$xU'_x$构成$K$的一个开覆盖,由于$K$是紧集,取一个有限子覆盖,对应于$x_1$, $\cdots$, $x_n\in K$. 令$U=\bigcap_{i=1}^n U''_{x_i}$,则
	\[
	KU\subset \bigcup_{i=1}^n {x_i}U'_{x_i}U\subset \bigcup_{i=1}^n x_iU'_{x_i}U''_{x_i}.
	\]
	任取$x_iU'_{x_i}U''_{x_i}$,由于
	\[
		x_iU'_{x_i}U''_{x_i}\cap VU=x_iU'_{x_i}U''_{x_i}\cap \bigcap_{j=1}^n VU''_{x_j}=(x_iU'_{x_i}U''_{x_i}\cap VU''_{x_i})\cap \bigcap_{j\neq i}^n VU''_{x_j}=\varnothing \cap \bigcap_{j\neq i} VU''_{x_j}=\varnothing,
	\]
	所以$KU\cap VU=\varnothing$.
\end{proof}

由于$VU$是一个开集,所以$\overline{KU}$都与$VU$不交,进而与$V$不交。由于有限集都是紧的,取$K=\{e\}$,然后就有$\overline{U}$与$V$不交。所以,任取$e$的一个邻域$W$,有闭集$G-W$,于是存在一个$e$的邻域$U$使得$\overline{U}$与$G-W$不交,或者说$\overline{U}\subset W$.

上面讨论了所有单位元邻域的交,由此产生了分离性的讨论。下面说明,任意一个单位元邻域都可以生成整个含有单位元的连通分支。

\begin{lem}\label{lem:116}
对于连通拓扑群,设$U$是单位元的任意一个邻域,则$G=\bigcup_{n\geq 1}U^n$,其中$U^k=\{g_1\cdots g_k\,:\,g_i\in U,\, 1\leq i \leq k\}$是开集。
\end{lem}

\begin{proof}
令$V=U\cap U^{-1}$,显然$V=V^{-1}\subset U$以及$H=\bigcup_{n\geq 1}V^n\subset \bigcup_{n\geq 1}U^n$,而且$H$还是一个子群。下面我们只要证明$H=\bigcup_{n\geq 1}V^n$既是开的也是闭的,那么连通性自然给出了结论。他是开的,如果$\sigma\in V^k$,那么$\sigma V\in V^{k+1}\subset H$就是他的一个开邻域。他是闭的,因为每一个$\sigma H$都是开的,于是$H=G-\bigcup_{\sigma\notin H}\sigma H$是一个闭集。
\end{proof}

结合上面两点不难看到,因为拓扑群有着代数结构,一般而言,这就使得拓扑群的拓扑结构都来自于其单位元附近的邻域。

举个例子,一个拓扑空间是局部紧的,需要每一点处有一个紧邻域,到了拓扑群,只需要单位元有一个紧邻域。

\begin{para}
一个拓扑空间$X$的拓扑可以由某个度量$d$诱导出来,则这个拓扑空间被称为可度量化的。对一般的拓扑空间$X$,Nagata-Smirnov度量化定理给出了可度量化的等价条件为$X$是$\mathsf{T}_3$且存在可数局部有限基。所谓的可数局部有限基就是说拓扑空间的基是可数个局部有限子族的并。所谓的局部有限族,就是对任意一点$p\in X$,存在一个邻域只与族中的有限个元素相交。

到了拓扑群语境,这个条件可以减弱到单位元附近存在局部基,使得他是数个局部有限子族的并。而$\mathsf{T}_3$可以减弱到$\{e\}$是一个闭集。
\end{para}

\begin{para}
在交换拓扑群上可以谈Cauchy列。由于下面我们谈论的都是交换群,所以这里用加法来表示群运算。其实在拓扑群上也可以谈类似的概念,不过这里就不把问题弄得太一般了。

在拓扑群$G$中,一个Cauchy列是$G$中的一个序列$\{g_1$, $g_2$, $\dots\}$,对每一个$G$中$0$的开邻域$U$,都存在一个正整数$n(U)$使得当$s$, $t>n(U)$时都有$g_s-g_t\in U$. 

称两个列$\{x_i\}$和$\{y_i\}$是等价的(共尾的),如果$x_i-y_i\to 0$. 这显然是一个等价关系。如果一个列收敛,则与他等价的列也收敛,并且有着相同的极限。事实上,给定单位元处的一个开邻域$U$,我们可以找一个对称邻域$V$使得$V\subset V+V\subset U$. 如果$\{x_i\}$收敛于$a$,则存在一个$n(V)$使得$i>n(V)$时一定成立$x_i-a\in V$,同样,由于$x_i-y_i\to 0$,所以存在一个$m(V)$使得$i>m(V)$时一定成立$y_i-x_i\in V$(如有必要,利用一下$V$是对称的),于是当$i>N(U)=\max\{n(V),m(V)\}$时,有
\[
	y_i-a=y_i-x_i+x_i-a\in V+V\subset U.
\]
于是$a$也是$\{y_i\}$的极限。

将这两个定义结合起来,不难看到,与Cauchy列等价的列也是Cauchy列,于是它诱导了所有Cauchy列上的一个等价关系。实际上,任取$0$的邻域$U$,选一个对称邻域$V$使得$V+V\subset U$,再取一个对称邻域$W$使得$W+W\subset V$. 设$\{x_n\}$是一个Cauchy列,而$\{y_n\}$与他等价,那么对$W$,我们有一个正整数$N$使得,$i$, $j>N$时成立
\[
	y_i-x_i\in W,\quad x_j-y_j\in W,\quad x_i-x_j\in W,
\]
其中前两个是等价,第三个是Cauchy列的条件,于是
\[
	y_i-y_j=(y_i-x_i)+(x_i-x_j)+(x_j-y_j)\subset W+W+W\subset V+V\subset U.
\]
因此$\{y_n\}$也是一个Cauchy列。

\end{para}

\begin{lem}
	Cauchy列的无穷子列与原列等价,进而也是Cauchy列。
\end{lem}

\begin{proof}
	设$\{x_i\}$是一个Cauchy列,而$\{x_{n_i}\}$是一个无穷子列,其中$n_i$随着$i$严格递增。下面证明$x_{n_i}-x_i\to 0$,这等价于说,任取$0$的一个邻域$U$,存在一个$N$使得$i>N$时候一定有$x_{n_i}-x_{i}\in U$. 事实上,由于$n_i$是选成递增的,所以$n_i\geq i$. 由于$\{x_i\}$是一个Cauchy列,所以存在一个$n(U)$使得$i$, $j>n(U)$时候一定有$x_i-x_j\in U$,取$N=n(U)$,则$i>N$时也有$n_i>N$,于是我们就得到了结论。
\end{proof}

一个简单的推论就是,如果Cauchy列的某个无穷子列收敛,则Cauchy列也收敛。

\begin{para}
	设$G$是一个交换拓扑群,而$\hat G$是$G$上所有Cauchy列的等价类,则$\hat G$被称为$G$的完备化。

	不难在$\hat G$上定义出一个交换群结构。Cauchy列$\{x_i\}$和$\{y_i\}$的加法定义为序列$\{x_i+y_i\}$,我们需要验证它也是Cauchy列。实际上,任取$0$的开邻域$U$,找一个对称邻域$V$使得$V\subset V+V\subset U$. 存在一个足够大的$N$使得$i$, $j>N$时有$x_i-x_j\in V$以及$y_i-y_j\in V$成立,此时$(x_i+y_i)-(x_j+y_j)\in V+V\subset U$. 因此$\{x_i+y_i\}$也是一个Cauchy列。

	从$G$到$\hat G$有一个自然的同态$\varphi$,即将$x\in G$变成常序列$\{x$, $x$, $\dots\}$. 但是注意到,这个同态可能不是单的。因为如果存在非零的$a\in\bigcap_{U\subset \mathscr{O}}U$,则$\varphi(x)$与$\varphi(x+a)$诱导的两个Cauchy列是等价的。反过来,不难从Cauchy列等价的定义证明也只有当$\bigcap_{U\subset \mathscr{O}}U=\{0\}$时,$\varphi$才是单同态。回忆Lemma \ref{lem:2},这就是说$\{0\}=\overline{\{0\}}$或者说$\{0\}$是一个闭集,回忆Proposition \ref{pro:3},这等价于说$G$是一个Hausdorff空间。

	那么$\varphi$什么时候才是满射呢?注意到,如果$G$中的Cauchy列是收敛的,则他与极限的常序列是等价的,所以如果$G$中的Cauchy列都是收敛的,则$\varphi:G\to \hat{G}$是满的。
\end{para}

\begin{para}
	假设$G$是第一可数的,即在每一点处都有可数局部基,在拓扑群上,只需要在原点处有可数局部基即可。这里我们可以赋予$\hat G$一个拓扑空间结构。这个拓扑空间自然要使得$\varphi$以及$\hat G$上的运算成为连续映射,这点我们慢慢检验。

	首先称序列$\{x_i\}$最终处于$0$的邻域$U$中,如果存在一个$N$使得$n>N$时候一定有$x_n\in U$. 主义,如果一个Cauchy列最终处于$U$中,与他等价的Cauchy列不一定也最终处于$U$中。比如考虑开集$(-1,1)\subset \rr$以及序列$\{1-2^{-n}\}$以及$\{1+(-2)^{-n}\}$,它们显然都是Cauchy列,并且相互等价,但是第二个就并不是最终处于$(-1,1)$中的。

	对$G$中的$0$的任意邻域$U$,定义$\hat U$是$\hat G$中那些等价类中的Cauchy列都最终处于$U$的等价类的集合。注意到,我们选取的等价类中所有的元素都最终处于$U$中,这就避免了$\{1+(-2)^{-n}\}$这种会与“边界”无限接近的序列的存在。

	下面罗列一些性质,它们都是简单的:
	\begin{compactenum}
	\item 如果$U\subset V$,则$\hat U\subset \hat V$.
	\item $\hat{U}\cap \hat{V}=\widehat{U\cap V}$.

	%首先,由第一点,显然$\widehat{U\cap V}\subset \hat{U}\cap \hat{V}$. 反过来,由于$\hat{U}$是所有最终处于$U$中的Cauchy列的等价类的集合,而$\hat{V}$是所有最终处于$V$中的Cauchy列的等价类的集合,于是$\hat{U}\cap \hat{V}$中的Cauchy列都最终处于$U\cap V$中。
	\item $\hat{U}\cup \hat{V}=\widehat{U\cup V}$.
	\item $\hat{U}+\hat{V}\subset \widehat{U+V}$.
	\end{compactenum}

	考虑所有形如$\hat{a}+\hat{U}$的集合构成的族,其中$a$跑遍$\hat G$,而$U$跑遍$G$中$0$的开邻域。我们验证这是一个拓扑基,即,我们要检验,如果$(a+\hat{U})\cap (b+\hat{V})\neq \varnothing$,则对$c\in (a+\hat{U})\cap (b+\hat{V})$,一定存在一个$c+\hat{W}\subset (a+\hat{U})\cap (b+\hat{V})$. 而这等价于证明存在一个$\hat{W}$使得
	\[
	\hat{W}\subset (a-c+\hat{U})\cap (b-c+\hat{V})
	\]
	而这只需要对$a-c+\hat{U}$找到$U_0$使得$\hat{U}_0\subset a-c+\hat{U}$,对$b-c+\hat{V}$找到$V_0$,然后取$W=U_0\cap V_0$即可。

	由于$c\in a+\hat{U}$,所以$c-a\in \hat{U}$. 考虑Cauchy列$\{x_i=c_i-a_i\}$是$x=c-a$的一个代表元,由于它最终落于$U$中,所以不妨直接假设它完全就处于$U$中。

	我们下面寻找一个$0$的对称邻域$U_0$,使得对于足够大的$n$,成立$x_n+U_0\subset x_n+U_0+U_0\subset U$. 这样对于最终落在$U_0$中的Cauchy列$\{y_n\}$,对足够大的$n$,一定有
	\[
	y_n+x_n\in x_n+U_0\subset U,
	\]
	如果选取$x$的等价类中其他的序列$\{x'_n\}$,那对于足够大的$n$,有$x'_n-x_n\in U_0$,于是
	\[
	y_n+x'_n=(y_n+x_n)+(x'_n-x_n)\in x_n+U_0+U_0\subset U,
	\]
	因此这是良定的。由于$\{y_n\}$是任取的最终落在$U_0$中的Cauchy列,故$\hat{U_0}+x\subset \hat{U}$或者写作$\hat{U_0}\subset -x+\hat{U}=a-c+\hat{U}$,此即所需。

	注意到,如果存在一个$0$的邻域$U_1$使得对于足够大的$n$一定有$x_n+U_1\subset U$成立,则取$U_0$为使得$U_0\subset U_0+U_0\subset U_1$的对称邻域就满足了上面的要求。最后,我们即要找一个$0$的邻域$U_1$,使得对于足够大的$n$,成立$x_n+U_1\subset U$.

	反证,因为假设了$G$是第一可数的,所以找$0$处的可数局部基,并选出一个严格递减的链$O_1\supset O_2\supset \cdots$使得对于任意的$0$的邻域$V$,存在一个整数$n(V)$使得$i>n(V)$时一定成立$O_i\subset V$. 由于不存在$U_1$使得$x_n+U_1\subset U$对足够大的$n$成立,所以对每一个$O_i$,都存在一个$x_{n_i}$使得$x_{n_i}+O_i\not\subset U$,即存在$y_{i}\not\in U$但$y_i\in x_{n_i}+O_i$. 不妨将$n_i$选成按$i$严格递增的,此时,$y_i-x_{n_i}\in O_i$对每一个$i$都成立。由于$O_i$可以选得任意小,故$y_i-x_{n_i}\to 0$,所以$\{y_i\}$与$\{x_{n_i}\}$等价也就与$\{x_i\}$等价,但$\{y_i\}$最终不处于$U$中。由于$x=c-a\in \hat{U}$,按照$\hat{U}$的定义,$\{y_i\}$处于该等价类中也应该最终处于$U$中。矛盾。
\end{para}

\begin{lem}\label{lem:11}
	假设$G$是第一可数的。对$x\in \hat{G}$以及给定$\hat{U}$,若存在对称邻域$V$使得$V\subset V+V\subset U$使得$x$的一个代表元$\{x_i\}$最终处于$V$中,则$x\in\hat{U}$.
\end{lem}

\begin{proof}
	只要证明与$\{x_i\}$等价的序列都最终处于$U$中即可。设$\{x'_i\}$与之等价,则足够大的$i$,我们有$x'_i-x_i\in V$,所以$x'_i=x_i+x'_i-x_i\in V+V\subset U$. 
\end{proof}

\begin{pro}
	如果交换拓扑群$G$是第一可数的,则其完备化$\hat G$有一个自然的拓扑结构使得它是一个拓扑群,并且使得自然同态$\varphi:G\to \hat G$是一个连续同态。此外,$\varphi(G)$是$\hat{G}$的一个稠密子集。
\end{pro}

当交换拓扑群$G$是第一可数的时候,我们会默认$\hat G$有引理中的这个拓扑群结构。

\begin{proof}
	拓扑结构在上面已经给出。考虑基中的任意一个元素$a+\hat{U}$,关于群的逆运算,它的原像是$-a-\hat{U}=-a+\widehat{-U}$也是一个开集,所以逆运算是连续的。

	然后考虑加法$\mu:\hat G\times \hat G\to \hat G$. 设$a+b\in c+\hat{U}$,存在一个$\hat{V}$使得$a+b+\hat{V}\subset c+\hat{U}$,其中$V$是$0$的一个邻域。然后对$V$找一个对称邻域$W$使得$W\subset W+W\subset V$,于是
	\[
	a+\hat{W}+b+\hat{W}=a+b+\hat{W}+\hat{W}\subset a+b+\widehat{W+W}\subset a+b+\widehat{V}\subset c+\hat{U},
	\]
	所以$\mu^{-1}(c+\hat{U})$是一个开集。

	现在,我们来证明$\varphi^{-1}(a+\hat{U})\subset G$是一个开集。如果是空集,则没什么好证的,否则任取$g\in \varphi^{-1}(a+\hat{U})$,于是$\varphi(g)\in a+\hat{U}$,对它可以找到$0$的一个邻域$V$使得$\varphi(g)+\hat{V}\subset a+\hat{U}$,直接可以计算得到
	\[
	g+V\subset \varphi^{-1}(\varphi(g)+\hat{V})\subset \varphi^{-1}(a+\hat U).
	\]
	所以$\varphi^{-1}(a+\hat U)$是一个开集,进而$\varphi$是一个连续同态。

	最后我们证明,$\varphi(G)$是$\hat{G}$的一个稠密子集。任取$x\in \hat{G}$,只需在$\hat{G}$中找一个序列趋向于$x$. 选定$x$的一个代表元$\{x_i\}$,然后考虑$\hat{G}$中的序列$\{\varphi(x_i)\}$,下面说明它的极限就是$x$,或者说$x-\varphi(x_i)\to 0$. 

	任取$\hat{U}$,我们需要说明对于足够大的$i$,$x-\varphi(x_i)\in \hat{U}$. 取$U$的一个对称邻域$V$使得$V\subset V+V\subset U$,由Lemma \ref{lem:11},只要$\{x_j-x_i\}_j$最终处于$V$中就得到了$x-\varphi(x_i)\in \hat{U}$. 由于$\{x_i\}$是Cauchy列,所以对$V$,存在$n(V)$,当$i$, $j>n(V)$时,$x_j-x_i\in V$成立。此即所需。
\end{proof}

\begin{para}
	既然$\hat{G}$是一个拓扑群了,我们自然也可以继续谈它的一些拓扑性质。
	\begin{compactenum}
	\item $\hat{G}$是Hausdorff空间。

		实际上,考虑$x\in \bigcap\hat{U}$,其中$U$跑遍$G$上$0$处的开邻域,我们证明$x=0$,即$x$中的任意代表元都与常序列$\{0$, $0$, $\dots\}$等价,或者说,任取$x$的代表元$\{x_i\}$,都有$x_i\to 0$. 任取$0$的开邻域$U$,因为$x\in \hat{U}$,$\{x_i\}$最终处于$U$中,所以存在一个$N$使得$n>N$时有$x_n\in U$,这就推出了$x_i\to 0$.
	\item $\hat{G}$是第一可数的。这当然直接来自于$\hat{G}$的拓扑基的构造与$G$第一可数的假设。

	\item $\hat{G}$是完备的,即$\hat{G}$中所有的Cauchy列收敛。
	\item 因此,对$\hat{G}$的完备化$\hat{\hat G}$以及自然同态$\hat\varphi:\hat{G}\to\hat{\hat G}$. 由第一点,$\hat\varphi$是单射,由第三点,$\hat\varphi$是满射,所以$\hat\varphi$是一个同构。
	\end{compactenum}

	\begin{proof}
		只需证明第三点。因为$G$是第一可数的,所以找$0$处的可数局部基,并选出一个严格递减的链$O_1\supset O_2\supset \cdots$使得对于任意的$0$的邻域$U$,存在一个整数$n(U)$使得$i>n(U)$时一定成立$O_i\subset U$. 

		考虑$\{\hat{x}_i\}_i$是$\hat{G}$中的一个Cauchy列,由于$\varphi(G)$在$\hat{G}$中是稠密,所以$\varphi^{-1}(\hat{x}_i+\hat{O}_i)$非空,让我们取一个$y_i\in\varphi^{-1}(\hat{x}_i+\hat{O}_i)$. 于是$\varphi(y_i)\in \hat{x}_i+\hat{O}_i$或者说$\varphi(y_i)-\hat{x}_i\in \hat{O}_i$,因此$\{\varphi(y_i)\}_i$与$\{\hat{x}_i\}_i$等价,所以只需证明$\{\varphi(y_i)\}_i$收敛。这里先可以得出$\{\varphi(y_i)\}_i$是一个Cauchy列。

		任取开集$U$,由于$\{\varphi(y_i)\}_i$是Cauchy列,所以存在$n(U)$使得$i$, $j>n(U)$时有
		\[
		\varphi(y_i-y_j)=\varphi(y_i)-\varphi(y_j)\in \hat{U},
		\]
		所以$y_i-y_j\in \varphi^{-1}(\hat{U})=U$. 于是$\{y_i\}$是Cauchy列,它在$\hat G$中的等价类记作$y$. 最后,我们证明它就是$\{\varphi(y_i)\}_i$的极限,或者说$\varphi(y_i)-y\to 0$. 与证明稠密性时候相似,利用Lemma \ref{lem:11}和$y$是一个Cauchy列,我们就得到了结论。
	\end{proof}
\end{para}

\section{拓扑矢量空间}

本文称矢量空间一般是指$\rr$-模,称复矢量空间空间一般是指$\cc$-模. 因此,对标量有绝对值函数$|*|:k\to \rr$.

\begin{para}
一个矢量空间$X$被称为赋范空间,如果存在一个非负函数$|*|:X\to \rr$使得如下性质成立:
\begin{compactenum}
\item 三角不等式,任取$x$, $y\in X$,成立不等式:$|x+y|\leq |x|+|y|$.
\item 任取标量$a$,以及$x\in X$,成立等式:$|ax|=|a||x|$.
\item $|x|=0$当且仅当$x=0$.
\end{compactenum}
这个非负函数被称为$X$上的一个范数。一旦给出$X$的一个范数,那么就可以定义$X$上的一个度量,$d(x,y)=|x-y|$. 这个度量被称为赋范空间的由范数诱导的度量。一般而言,除非特别声明,赋范空间上的范数都是指这个。
\end{para}

\begin{para}
一个度量空间的拓扑结构是清楚的。记$B(x,r)=\{y\,:\, d(x,y)<r\}$为圆心在$x$,半径为$r$的开球。度量空间的所有开球构成一组拓扑基,继而给出了度量空间的一个拓扑结构。

赋范空间的许多结构实际上完全来自于原点附近,这来自于这个事实,$d(x-z,y-z)=|x-z-y+z|=|x-y|=d(x,y)$,如果在平移下,两点之间的距离是不变的。然后可以证明,平移映射$T_a:x\mapsto x+a$是一个同胚,实际上,只要证明这是连续的即可,因为$T_{-a}$此时构成$T_a$的连续逆,而连续性来自于开球的原像也是开球。类似地,放缩映射$M_r:x\mapsto rx$是一个连续同胚,其中$r$是一个非零标量。由于$r$可以是负数,所以$x\mapsto -x$也是连续同胚。乘以零虽然不是一个同胚,但依然是一个连续映射,实际上,任取不包含原点的开集,原像是空集所以是开集,选取包含原点的开集,原像是整个空间所以是开集。

更一般的,可以证明加法$p:X\times X\to X$是一个连续映射。考虑开球$B(x,r)$,我们有$p^{-1}(B(x,r))=\{(y,z)\,:\,|y+z-x|<r\}$,这是一个开集。实际上,任取$(y,z)\in p^{-1}(B(x,r))$,我们有$|y+z-x|=a<r$. 考虑开集$U=B(y,(r-a)/2)\times B(z,(r-a)/2)$,任取$(y',z')\in U$,有
\[
	|y'+z'-x|=\left|y'-y+z'-z+y+z-x\right|\leq |y'-y|+|z'-z|+|y+z-x|<r-a+a=r,
\]
于是$U\subset p^{-1}(B(x,r))$. 所以$p^{-1}(B(x,r))$是一个开集。
\end{para}

\begin{para}
赋范空间是拓扑矢量空间的重要实例,而拓扑矢量空间也可以看成赋范空间的抽象。一个拓扑矢量空间是一个矢量空间$X$,上面有一个拓扑结构使得$X$是一个Hausdorff空间,且加法与标量乘法是连续映射。因此,一个拓扑矢量空间看成加法群的时候是一个拓扑群。

Hausdorff空间的假设在许多材料里这个条件不是必须的,但是引入会很方便,而且基本上感兴趣的所有情况都会满足$\mathsf{T}_2$的假设,因为对于拓扑群来说$\mathsf{T}_2$空间假设其实是一个很容易满足的强力条件,只需要原点作为单点集是一个闭集即可。

此外,拓扑矢量空间是道路连通的,任取$x$, $y\in X$,则$tx+(1-t)y$就构成连接$x$和$y$的一条连续道路。
\end{para}

\begin{para}
由于拓扑矢量空间比交换拓扑群多了一个数乘的结构,这就导致了如下的一些定义:
\begin{compactenum}
\item 一个$X$的子集$C$是凸的,如果任取$0\leq t\leq 1$,都有$tC+(1-t)C\subset C$. 凸集是平移不变的,实际上,任取$a\in X$,都有$a+tC+(1-t)C=t(a+C)+(1-t)(a+C)\subset a+C$.

\item 一个$X$的子集$C$是均衡的,如果任取标量$\alpha$使得$|\alpha|\leq 1$的时候有$\alpha C\subset C$.

\item 一个拓扑矢量空间被称为局部凸的,如果$0$有一个局部基使得他的元素都是凸集。这等价于说每一点都有局部基,它的元素都是凸集。

\item 一个$X$的子集$C$是有界的,如果对每一个$0$的邻域$U$,存在一个正实数$s_U$使得当$t>s_U$的时候有$C\subset tU$.

\item 一个拓扑矢量空间被称为局部有界的,如果他有一个$0$的有界邻域$U$。

\item 称一个拓扑矢量空间$X$为F-空间,如果它的拓扑是由一个完备\footnote{Cauchy列都收敛。}度量$d$诱导的,且满足$d(x-z,y-z)=d(x,y)$对任意的$x$, $y$, $z\in X$都成立。最后一点往往被称为平移不变,或者简单叫做不变。

\item 称一个拓扑矢量空间$X$为Fr\'{e}chet空间,如果$X$是一个局部凸F-空间。

\item 如果拓扑矢量空间$X$的每一个有界闭子集都是紧的,则$X$被称为满足Heine-Borel性质。这个名字自然来自于Heine-Borel定理:$\rr^n$的每一个有界闭子集都是紧的。

\item 赋范空间已经定义了。一个完备的赋范空间被称为Banach空间。
\end{compactenum}
\end{para}

从定义,可以得到:
\begin{compactenum}
\item 凸子集的闭包与内部都是凸子集。
\item 子空间的闭包是子空间。
\item 有界子集的闭包是有界子集。
\item 均衡子集的闭包是均衡子集。
\item 如果均衡子集包含原点,则它的内部也是均衡子集。
\end{compactenum}

如果拓扑矢量空间的拓扑结构可以由一个不变度量出来,那么在度量意义下的Cauchy列与在拓扑群意义下的Cauchy列是等价的。于是,如果两个不变度量诱导了同一个拓扑,则它们的Cauchy列等价,特别地,如果拓扑矢量空间对其中一个完备,则对每一个也完备。

\begin{pro}
任取一个原点的(凸)邻域$U$,都存在一个均衡(凸)邻域$V$使得$V\subset U$. 
\end{pro}

\begin{proof}
	由于标量乘法是连续的,所以存在一个$\delta$和开集$V$使得当$0<|\alpha|<\delta$时,$\alpha V\subset U$. 遍历$\alpha$,所有$\alpha V$的并就是需要的均衡开子集。

	如果$U$是凸的,设$A$是所有$\alpha U$的交,其中$|\alpha|=1$. 由于在$U$中存在均衡开子集$V$,所以对任意使得$|\alpha|=1$的$\alpha$,有$V\subset \alpha V$以及$\alpha V\subset \alpha^{-1}\alpha V=V$,所以$V=\alpha V$给出了$V\subset \alpha U$以及$V\subset A$. 这说明$A$具有非空内部,下面我们证明这就是想要的均衡凸邻域。首先作为凸集的交,$A$是凸集,其内部也是凸的。然后任取$\beta$使得$0\leq |\beta|\leq 1$,将其分解为径部$r=|\beta|$以及$\gamma=\beta/r$,其中$|\gamma|=1$. 于是
	\[
	\beta A=r\gamma A=\bigcap_{|\alpha|=1}r\gamma\alpha U=\bigcap_{|\gamma\alpha|=1}r(\gamma\alpha) U=\bigcap_{|\alpha|=1}r\alpha U,
	\]
	由于$U$包含原点,所以$rU=rU+0\subset rU+(1-r)U\subset U$,进而
	\[
	\beta A=\bigcap_{|\alpha|=1}r\alpha U\subset \bigcap_{|\alpha|=1}\alpha U=A
	\]
	推出$A$是均衡的,其内部也是均衡。
\end{proof}

上面的命题意味着,拓扑矢量空间局部基的元素可以都选成均衡的。如果空间还是局部凸的,则局部基的元素可以都选成均衡凸的。

\begin{para}
任取一个原点的开邻域$U$,由于拓扑矢量空间$X$是连通的,所以$X$实际上可以写成所有形如$U+U+\cdots+U$的开集的并,这是拓扑群那里的Lemma \ref{lem:116}. 这只是利用了加法结构的一个推论,到了拓扑矢量空间上,标量的引入可以让我们做得更多。
\begin{compactenum}
\item 考虑任意一个严格递增正实数序列$\{r_n\}$,$U$是原点的一个邻域。如果当$n\to \infty$时$r_n\to \infty$,则$X=\bigcup_{n=1}^\infty r_n U$.

证明是简单的,固定$x$,由于$\alpha\to \alpha x$是连续映射,所以所有使得$\alpha x\in U$的标量$\alpha$构成一个开集$V$,且$0$属于这个开集,这就意味这,对于足够大的$n$,$1/r_n$都在$V$中,所以$(1/r_n) x\in U$给出了$x\in r_n U$.

\item 类似地,考虑任意一个严格递减正实数序列$\{r_n\}$,$V$是原点的一个有界邻域。如果当$n\to \infty$时$r_n\to 0$,则$\{r_n V\}$构成原点的一个局部基。

实际上,设$U$是原点的一个邻域。由于$V$是有界的,则存在一个$s$使得$r>s$时,$V\subset rU$. 对于足够大的$n$,我们有$sr_n<1$,所以$V\subset (1/r_n)U$或者$r_n V\subset U$.
\end{compactenum}
\end{para}

作为推论,拓扑矢量空间的紧子集$K$一定是有界的。任取一个$0$的开邻域$V$,取$0$的一个均衡邻域$U\subset V$,则$K\subset \bigcup_{n=1}^\infty nU$. 由于$K$是紧的,所以存在有限个$n_1< \cdots <n_k$使得
\[
	K\subset \bigcup_{i=1}^k n_iU=n_k U.
\]
最后的等号来自于,对均衡邻域$U$以及$r>1$一定成立$U\subset rU$. 所以$K\subset n_k U\subset n_k V$.

\begin{thm}
第一可数拓扑矢量空间$X$是可度量化的,且该度量可以选成不变度量,使得中心在$0$的开球是均衡开集。如果$X$还是局部凸的,则所有开球是凸的。
\end{thm}

度量空间显然是第一可数的,但第一可数一般不足以说明一个拓扑空间是可度量化的。不过因为矢量空间多了两个运算,这就导致了拓扑矢量空间上第一可数成了可度量化的充要条件。这个定理的证明见[Rudin, Theorem 1.10].

\section{线性映射}

\section{Hilbert空间}

设$\mathcal{H}$是复数域上的矢量空间。假设$x$, $y\in \mathcal{H}$,有一个二元运算写作$(x,y)$,需要满足如下性质
\begin{itemize}
\item $(x,y)=(y,x)^*,$

这里用$a^*$表示$a$的复共轭。

\item 对第二个变量线性
\[
(x,\xi_1y_1+\xi_2y_2)=\xi_1(x,y_1)+\xi_2(x,y_2),
\]
结合上面两个性质,很容易证明内积对第一个变量反线性,即
\[
(\eta_1x_1+\eta_2x_2,y)=\eta_1^*(x_1,y)+\eta_2^*(x_2,y).
\]
\item 对任意的$x\in \mathcal{H}$,$(x,x)\geq 0$,等号当且仅当$x=0$的时候取到。
\end{itemize}

则称这个二元运算是一个内积,而$\mathcal{H}$被称为一个内积空间。在内积空间上可以定义一个矢量的长度
\[
	|x|=\sqrt{(x,x)},
\]
注意到,我们重用了绝对值符号。在具体语境下,含义基本还是可以分辨的,否则我们会具体指出。从内积的定义,长度为零的矢量只有零。

\begin{para}
两个矢量$y$和$x$被称为正交的,如果$(y,x)=0$. 于是我们就有勾股定理,如果$y$和$x$正交,则
\[
	|x+y|^2=|x|^2+|y|^2+(x,y)+(y,x)=|x|^2+|y|^2.
\]
\end{para}

\begin{pro}
Cauchy-Schwarz不等式:$|(x,y)|\leq |x||y|$.
\end{pro}

\begin{proof}
等式对$x=0$的情况是显然成立的,下面设$x\neq 0$. 令$\alpha=(x,y)$,考虑
	\[
		0\leq |\lambda x+y|^2=|\lambda|^2|x|^2+|y|^2+\lambda^* \alpha+\lambda \alpha^*.
	\]
	在这之中,令$\lambda=-\alpha/|x|^2$,则上述不等式变为
	\[
		0\leq |\lambda x+y|^2=|y|^2-\frac{|\alpha|^2}{|x|^2},
	\]
	这就是我们的需要的不等式。
\end{proof}

\begin{pro}
$|y|\leq |\lambda x+y|$对每一个$\lambda \in \cc$成立当且仅当$(x,y)=0$.
\end{pro}

\begin{proof}
对于$(x,y)=0$,由勾股定理
\[
	|\lambda x+y|^2=|\lambda x|^2+ |y|^2\geq |y|^2.
\]
反过来,如果$(x,y)\neq 0$,只要找一个$\lambda$使得$|y|^2>|\lambda x+y|^2$即可。同上一个命题取$\lambda=-\alpha/|x|^2$,则
\[
	|\lambda x+y|^2=|y|^2-\frac{|\alpha|^2}{|x|^2},
\]
在$(x,y)\neq 0$时$|y|^2>|\lambda x+y|^2$. 
\end{proof}

此时,我们可以检验$|*|$构成了一个范数,因为
\[
	|x|\geq 0,\quad |\lambda x|=|\lambda||x|,
\]
以及最重要的
\[
	|x+y|^2=|x|^2+|y|^2+(x,y)+(y,x)\leq |x|^2+|y|^2+2|x||y|=(|x|+|y|)^2,
\]
这就是三角不等式$|x+y|\leq |x|+|y|$.

\begin{para}
有了范数,就自然有了一个度量$d(x,y)=|x-y|$,于是内积空间就是一个度量空间。有了度量,就很容易定义出拓扑,由于是矢量空间,是加法群,所以在单位元$0$附近定义拓扑即可,剩下的可以通过平移移过去。选取所有$|x|<r$的$x$的集合作为单位元处的开集基,这样我们就定义出了$\mathcal{H}$上的一个拓扑。

固定$x$,$(x,y)$对$y$是一个连续函数,因为
\[
	|(x,y+\Delta y)-(x,y)|=|(x,\Delta y)|\leq |x||\Delta y|,
\]
对选定的$\delta$,都可以选取$\epsilon=\delta/|x|$使得$\Delta y<\epsilon$时有$|(x,y+\Delta y)-(x,y)|<\delta$. 反过来,固定$y$,$(x,y)$对$x$也是一个连续函数。

利用这点,我们可以说明$x$的正交补空间$N(x)=\{y\in\mathcal{H}\,:\, (x,y)=0\}$是一个闭集,因为$N(x)$是$\Lambda_x(y)=(x,y)$关于$\{0\}$的原像。
\end{para}

\begin{para}
如果我们的内积空间$\mathcal{H}$在这个范数下是完备的(即Cauchy列收敛),则称$\mathcal{H}$是一个Hilbert空间。由于Cauchy列收敛,所以Hilbert空间的非空闭凸子集$E$\footnote{所谓凸集就是说,如果$x$和$y$都在该集合内,则对任意的$t\in [0,1]$都有$tx+(1-t)y$也在该集合内。}必然有着唯一的范数最小的元素。
\end{para}

\begin{proof}
考虑$E$中所有元素范数的下界$d$,然后找一个序列$\{x_i\}$使得$|x_i|\to d$,由于$(x_m+x_n)/2\in E$,故$|(x_m+x_n)/2|\geq d$,考虑恒等式
\[
	0\leq |x_m-x_n|^2=2|x_m|^2+2|x_n|^2-|x_m+x_n|^2\leq 2(|x_m|^2+|x_n|^2)-4d^2,
\]
或者
\[
	|x_m-x_n|^2\leq 2(|x_m|^2-d^2)+2(|x_m|^2-d^2)\leq 2\bigl||x_m|^2-d^2\bigr|+2\bigl||x_m|^2-d^2\bigr|,
\]
因此$\{x_i\}$是一个Cauchy列,故而收敛到某个$x\in H$使得$|x|=d$. 由于$x$是$E$的极限点,而$E$是闭集,所以$x\in E$. 至于唯一性,如果$|y|=d$,考虑$\{x$, $y$, $x$, $y$, $\cdots\}$的极限即可。
\end{proof}

\begin{pro}
对于Hilbert空间$\mathcal{H}$以及他的任意一个子空间$M$,$M$的正交补$M^\bot=\{y\in H\,:\, \forall x\in M,\, (x,y)=0\}$是一个闭子空间。如果$M$还是闭的,则成立直和分解$\mathcal{H}=M\oplus M^\bot$.
\end{pro}

\begin{proof}
由于
\[
	M^\bot=\bigcap_{x\in M}N(x),
\]
而$N(x)$是闭集,所以$M^\bot$自然是闭集。由于$(x,y)$关于$y$线性,所以$M^\bot$也是一个子空间。

现在假设$M$是闭的。剩下的就是要证明$M\cap M^\bot=\{0\}$,以及任意的$x\in \mathcal{H}$都可以分解为$x=x_1+x_2$,其中$x_1\in M$, $x_2\in M^\bot$. 对于前者,假设$x\in M\cap M^\bot$,因此$(x,x)=0$就推出了$x=0$. 

对后者,设$x\in \mathcal{H}$,考虑集合$x-M$,这是$\mathcal{H}$的一个闭凸子集,闭来自于他是$M$的一个平移,而凸来自于$M$是一个子空间。所以$x-M$中必然有唯一的$x_1$使得$|x-x_1|$最小。令$x_2=x-x_1$,则对于所有的$y\in M$都成立$|x_2|\leq |x_2+y|$,特别地,对于任意的$\lambda\in \cc$有$|x_2|\leq |x_2+\lambda x_1|$,因此$(x_1,x_2)=0$. 综上,任取$x\in \mathcal{H}$,成立分解$x=x_1+x_2$,$x_1\in M$,而$x_2\in M^\bot$.
\end{proof}

\begin{para}
Hilbert空间$\mathcal{H}$的对偶空间$\mathcal{H}^*$是由$\mathcal{H}$上所有线性函数张成的矢量空间,上面可以如下赋予范数
\[
	|T|=\sup\bigl\{|Tx|\,:\,x\in \mathcal{H},\,|x|\leq 1\bigr\},
\]
很容易检验这是一个范数,比如三角不等式:
\[
	|(T+S)x|=|Tx|+|Sx|\leq |T|+|S|,
\]
对任意的$|x|\leq 1$都成立,所以$|T+S|\leq |T|+|S|$.

关于这个范数还有一个有用的不等式$|Tx|\leq |T||x|$:显然,$|x/|x||=1$,所以由
\[
	\left|T\frac{x}{|x|}\right|\leq |T|
\]
可以推知$|Tx|\leq |T||x|$. 
\end{para}

使用内积,可以在$\mathcal{H}$和$\mathcal{H}^*$之间建立一个线性映射:通过$\Lambda_y(x)=(y,x)$可以定义出一个线性函数$\Lambda_y\in \mathcal{H}^*$,这样我们就得到一个线性映射$\Lambda:\mathcal{H}\to \mathcal{H}^*$,他将$y$变为$\Lambda_{y}$.

\begin{thm}
Riesz表示定理:设$\mathcal{H}$是一个Hilbert空间,则$\Lambda:\mathcal{H}\to \mathcal{H}^*$是一个同构,且$|\Lambda_y|=|y|$.
\end{thm}

\begin{proof}
	注意到,$|\Lambda_y|\geq |y|$就可以推出$\Lambda$是一个单射。因为如果$|\Lambda_y|=0$,则$|y|=0$可以推出$y=0$. 所以我们先推$|\Lambda_y|=|y|$,这样单射就是显然的了。

	由Cauchy-Schwarz不等式,在$|x|\leq 1$时,$|\Lambda_y(x)|\leq |y||x|\leq |y|$,所以$\Lambda_y|\leq |y|$. 反过来,由于
	\[
		|y|^2=\Lambda_y(y)=|\Lambda_y(y)|\leq |\Lambda_y||y|,
	\]
	所以$|y|\leq |\Lambda_y|$. 这样就得到了$|\Lambda_y|=|y|$. 

	剩下的就是要证明$\Lambda$是满射。若$y=0$,则取$\Lambda_y=0$. 若$y\neq 0$,令$N(\Lambda_y)$为$\Lambda_y$的零空间,即
	\[
		N(\Lambda_y)=\bigl\{x\in\mathcal{H}\,:\, \Lambda_y(x)=(y,x)=0\bigr\}.
	\]
	由于$N(\Lambda_y)=N(y)$是一个闭子空间,所以存在直和分解$\mathcal{H}=N(\Lambda_y)\oplus N(\Lambda_y)^\bot$. 取定一个非零的$z\in N(\Lambda_y)^\bot$,因为
	\[
		(\Lambda_y(x))z-(\Lambda_y(z))x\in N(\Lambda_y),
	\]
	所以
	\[
		(z,(\Lambda_y(x))z-(\Lambda_y(z))x)=\Lambda_y(x)(z,z)-\Lambda_y(z)(z,x)=0,
	\]
	或者
	\[
		\Lambda_y(x)=\left(\frac{\Lambda_y(z)}{(z,z)}z,x\right),
	\]
	此时当$y=\Lambda_y(z)z/(z,z)$时上式成立。
\end{proof}

\section{伴随算子}