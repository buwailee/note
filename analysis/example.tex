%!TEX root = main.tex
\chapter{例子}

说在最前面,经常地,在同一个矢量空间上,我们可以为了不同的目的选取不同的拓扑,下面各节中出现的拓扑往往是最熟悉的那些,但这不代表这是唯一的拓扑。一个重要的反例是有限维空间,可以看到,作为拓扑矢量空间,它上面具有唯一的拓扑。

\section{有限维空间}

\begin{lem}
设$X$是$k$-拓扑矢量空间(比如$k$可以取作$\rr$或者$\cc$). 则任意的线性映射$f:k^n\to X$是连续映射。
\end{lem}

这个小引理表示,有限维空间上的线性映射是都是连续的。所以这也同时给出了有限维矢量空间的拓扑很大的限制。事实上下一个命题就将我们,有限维矢量空间唯一的拓扑就是熟知的欧式空间拓扑(对实数域和复数域)。

\begin{proof}
设$\{e_i\}$是$k^n$的一组基,令$x_i=f(e_i)$. 对每一个$v=\sum_{i=1}^nv^ie_i$,我们有
\[
	f(v)=\sum_{i=1}^nv^ix_i.
\]
由于分量函数$v\to v^i$是连续的,则由有限个加法与标量乘法的连续性,$f$是连续的。
\end{proof}

\begin{pro}
	设域$k$为$\cc$或者$\rr$,而$X$是一个$k$-拓扑矢量空间。设$Y$是$X$的一个有限维子空间,$\dim Y=n$. 则每一个同构$k^n\to Y$是同胚,且$Y$是一个闭子空间。
\end{pro}

\begin{proof}
	设$f:k^n\to Y$是一个同构,首先它是连续的,所以我们只要证明$f^{-1}$连续即可。从
\end{proof}

\begin{pro}\label{1.46}
局部紧拓扑矢量空间是有限维的。因此,只有形如$\cc^n$或者$\rr^n$的拓扑矢量空间才是局部紧的,无限维的拓扑矢量空间都不可能是局部紧的。
\end{pro}

这俩命题这里都不证了,可见[Rudin Theorem 1.21]和[Rudin Theorem 1.22].

作为推论,如果一个局部有界拓扑矢量空间满足Heine-Borel性质,则原点处可以找到有界闭子集是紧的,这是一个局部紧拓扑矢量空间,继而是有限维的。所以,在无限维拓扑矢量空间上,不可能同时满足局部有界以及Heine-Borel性质。

\section{连续函数空间}

这节讨论欧式空间中的开集上的连续函数空间。首先,我们给出这样一个拓扑学上的命题。

\begin{thm}\label{thm:3.4}
设$M$是一个局部紧第二可数Hausdorff空间,则存在一列开集$\{U_i\}$,使得每个$\overline{U}_i$都是紧集,$\overline{U}_i\subset U_{i+1}$,且$\bigcup_iU_i=M$.这样的一列开集被称为$M$上的一个穷竭。
\end{thm}

\begin{proof}
任取$x\in M$,由于$M$是局部紧Hausdorff空间,所以我们可以找一个邻域$V_x$使得$\overline{V}_x$是紧集。由于$M$第二可数,有可数基,所以我们可以找至多可数个$x$使得$V_x$构成$M$的一个开覆盖,由于至多可数,所以我们可以利用正数将其编号。随便选定一个编号就好,令$U_1=V_1$,并假设在$0<i<n$的时候$U_i$都已经定义,那么考虑紧集$\bigcup_{i<n}\overline{U}_i$,开覆盖$\{V_j\}$中可以找到一个有限子覆盖,对应着有限指标集$I$,使得$\bigcup_{j\in I}V_j$也是$\bigcup_{i<n}\overline{U}_i$的开覆盖,取$k=\sup(I)$,则
\[
	\bigcup_{i<n}\overline{U}_i\subset \bigcup_{j\leq k}V_j,
\]
那么定义$U_n=\bigcup_{j\leq k}V_j$,我们就归纳地找到了$M$上的一个穷竭。
\end{proof}

由于欧式空间中的非空开集总是局部紧的第二可数Hausdorff空间,所以对每一个开集$U$都存在一个穷竭。

下面,我们将讨论欧式空间上的非空开子集$U$上所有的复值连续函数构成的矢量空间$\mathsf{C}(U)$. 首先,即使$U$是任意集合,矢量空间结构是自然的,取连续函数$f$和$g$和标量$a$,可以定义加法和数乘如下:
\[
	(f+g)(x):=f(x)+g(x),\quad (af)(x)=af(x).
\]

\begin{para}
为了给出$\mathsf{C}(U)$上的拓扑结构,我们先回忆一些中学就学过的定义和定理,它们涉及连续函数族的极限等。
\begin{compactenum}[(1)]
\item 设$f:X\to Y$是两个度量空间之间的映射,如果对任意的整数$\epsilon$,都存在$\delta$使得对任意满足$d_X(x,y)<\delta$的$x$, $y\in X$都有$d_Y(f(x),f(y))<\epsilon$,则称$f$是一致连续的。

一致连续与普通的连续性的区别在于,普通的连续性只需要在一点附近有一个$\delta$就行,每点处的$\delta$可以不同,而一致连续需要在全局都有一个$\delta$. 因此,不难得到,一致连续函数是连续的,而如果在全局加上一个有限性条件,比如全局是一个紧集,则此时连续函数是一致连续的。

\item 设$\{f_n\}$是同一个集合$E$到一个度量空间$X$的函数族,如果对每一个正数$\epsilon$,都存在一个$N$使得$n>N$的时候有
\[
	d_X(f_n(x),f(x))< \epsilon
\]
对任意的$x\in E$都成立,则称$\{f_n\}$一致收敛于函数$f$. 

显然,如果$\{f_n\}$一致收敛于$f$,则也必然逐点收敛于$f$,即对每一个$x\in E$,序列$\{f_n(x)\}$收敛于$f(x)$. 反过来,如果$\{f_n\}$逐点收敛,则可以定义出一个$E$上的函数$f$通过$f(x)=\lim_{n}f_n(x)$,但是此时$\{f_n\}$不一定一致收敛于$f$.

\item 在(2)中,如果度量空间$X$是完备的,则我们有如下的Cauchy准则:$\{f_n\}$一致收敛当且仅当,对每一个正数$\epsilon$,都存在一个$N$使得$m$, $n>N$的时候有
\[
	d_X(f_n(x),f_m(x))< \epsilon
\]
对任意的$x\in E$都成立。

\item 在(2)中,如果$E$是一个拓扑空间,则此时可以考虑连续函数族。这里有一个非常重要的一致极限定理:一致收敛的连续函数序列将收敛于一个连续函数。
\end{compactenum}
\end{para}

\begin{para}
现在考虑考虑欧式空间中的紧子集$K$,以及$K$上的复值连续函数空间$\mathsf{C}(K)$. 由于紧子集上的复值连续函数都是有界的,所以对每一个$f\in \mathsf{C}(K)$
\[
	|f|:=\sup_{x\in K}|f(x)|
\]
都是一个正实数,并且只有当$f(x)$恒为零的时候才有$|f|=0$. 此外,由于
\[
	|(f+g)(x)|=|f(x)+g(x)|\leq |f(x)|+|g(x)|\leq |f|+|g|
\]
对任意的$x\in K$都成立,所以三角不等式$|f+g|\leq |f|+|g|$也成立。这样我们就在$\mathsf{C}(K)$上给出的一个范数,使得其成为了一个赋范空间。

在这个度量下,$|f-g|< \epsilon$等价于
\[
	\sup_{x\in K}|f(x)-g(x)|<\epsilon,
\]
所以,$\mathsf{C}(K)$中的序列在这个度量下的收敛等价于一致收敛。
\end{para}

\begin{pro}
$\mathsf{C}(K)$是一个Banach空间。
\end{pro}

\begin{proof}
即证明完备。考虑$\mathsf{C}(K)$的Cauchy列,由一致收敛性的Cauchy准则,它一致收敛于一个函数,而一致收敛的连续函数序列收敛于连续函数,所以它在$\mathsf{C}(K)$中收敛。
\end{proof}

\begin{para}[$\mathsf{C}(U)$的拓扑]
考虑$U$的一个穷竭,进而得到一个至多可数紧集族$K_1\subset K_2\subset \cdots$使得每一个$K_n\subset K_{n+1}^\circ$且$\bigcup_i K_i=U$. 考虑如下半范数族
\[
	p_n(f):=\sup_{x\in K_n}|f(x)|,
\]
因为$K_n\neq U$,所以它们不是范数。而这个半范数族是可分点的,这里即没有非平凡的公共零点,实际上,如果存在一个$f$和$x_0\in U$有$f(x_0)\neq 0$,则必然有一个$K_n$使得$x_0\in K_n$,此时$p_n(f)\geq |f(x_0)|>0$.

由Theorem \ref{1.61},集合族
\[
	V_n=\left\{f\in \mathsf{C}(U)\,:\, p_n(f)< \frac 1 n\right\}
\]
的任意有限交构成的族是$\mathsf{C}(U)$原点的一个均衡凸局部基,这就给出了$\mathsf{C}(U)$的一个拓扑。由于$V_n$中总可以找到一个$f$使得在$p_{n+1}(f)$任意大,所以由Theorem \ref{1.61},$V_n$都不是有界的,进而$\mathsf{C}(U)$不是局部有界的,由Theorem \ref{1.56},$\mathsf{C}(U)$是不可赋范的。

或者,可以由
\[
	d(f,g)=\max_n \frac{2^{-n}p_n(f-g)}{1+p_n(f-g)}
\]
直接给出一个与上述拓扑相容的不变度量。此时,$d(f,g)<\epsilon$可以写成
\[
	\frac{2^{-n}p_n(f-g)}{1+p_n(f-g)}<\epsilon
\]
对任意的$n$都成立,即
\[
	(2^{-n}-\epsilon)p_n(f-g)<\epsilon,
\]
对于足够大的$n$,左边是负的所以自然成立,因此我们只要转而考虑使得$2^{-n}>\epsilon$的那些有限的$n$,成立
\[
	p_n(f-g)<\frac{\epsilon}{2^{-n}-\epsilon}=a_n \epsilon.
\]
上述逻辑链反过来,也可以推出$d(f,g)<\epsilon$成立。
\end{para}

\begin{pro}
$\mathsf{C}(U)$是一个完备度量空间。
\end{pro}

\begin{proof}
考虑一个Cauchy列$\{f_n\}$,对给定的小$\epsilon$,我们有$N$以及$m$, $n>N$时候成立$d(f_n,f_m)<\epsilon$很成立,按照前面所言,存在一个最大的正整数$k(\epsilon)$使得$2^{-k(\epsilon)}>\epsilon$,对$m$, $n>N$以及正整数$i\leq k(\epsilon)$成立
\[
	p_i(f_n,f_m)<\frac{\epsilon}{2^{-i}-\epsilon}.
\]
显然,$k(\epsilon)$是随着$\epsilon$减小而增大。

对于给定的正数$\delta$,只要$\epsilon$足够小,总是可以保证$p_{k(\epsilon)}(f_n,f_m)<\delta$成立,故$\{f_n|_{K_{k(\epsilon)}}\}$在$\mathsf{C}(K_{k(\epsilon)})$中是Cauchy列。而$\mathsf{C}(K_{k(\epsilon)})$完备,所以收敛于一个$f_\epsilon$. 显然,对于给定的$x$,如果$x$同时处于$K_{k(\epsilon_1)}$和$K_{k(\epsilon_2)}$中,则$f_{\epsilon_1}(x)=f_{\epsilon_2}(x)$.

由于任取$x_0$,都存在一个正整数$\epsilon_0$使得$x_0\in K_{k(\epsilon_0)}$,定义
\[
	f(x_0)=f_{\epsilon_0}(x_0),
\]
不难检验它在$U$上定义了一个良定的函数$f$. 此外,$f$在$U$上连续。实际上,只要局部检验就行了,取$x$,它必然包含于一个$K_n$中,于是$x$包含于$K_{k}$的内部中,其中$k>n$. 选一个足够小的$\epsilon$保证$k(\epsilon)>n$,则按前面的定义,$f|_{K_{k(\epsilon)}}=f_{\epsilon}$在点$x$处连续,因此$f$在点$x$连续。而$x$又是$U$上任取的,所以$f\in \mathsf{C} (U)$.

最后,$d(f_n,f)\to 0$的检验是直接的。所以$\mathsf{C}(U)$是一个完备度量空间。
\end{proof}

\section{光滑函数空间}

这里我们考虑的都是$\rr^n$及其子集上的(复)函数构成的空间,所以先引入一些记号。首先,设$\alpha=(\alpha_1$, $\dots$, $\alpha_n)\in \zz^n$是一个自然数组(非负整数组),则称其为一个多重指标。然后,即
\[
	|\alpha|=\sum_{i=1}^n\alpha_i,\quad D^\alpha=\prod_{i=1}^n \partial_i^{\alpha_i}.
\]
类似地,我们可能还需要二项式系数,阶乘等:
\[
	\binom{\beta}{\alpha}=\prod_{i=1}^n\binom{\beta_i}{\alpha_i},\quad \alpha!=\prod_{i=1}^n \alpha_i!.
\]

\begin{para}[光滑函数空间]
任取开集$U$,如果$f\in \msf C(U)$对任意的多重指标$\alpha$,都有$D^\alpha f\in \msf C(U)$,则称$f$在$U$上光滑。最简单的光滑函数是常值函数,稍微复杂一些,多项式函数,再比如我们知道,指数映射也是光滑的。所有在$U$上光滑的函数构成的集合我们记作$\msf C^\infty(U)$. 实(复)光滑函数显然构成一个实(复)矢量空间,所以$\msf C^\infty(U)$被叫做$U$上的光滑函数空间。
\end{para}

\begin{para}[$\msf C^\infty(U)$的拓扑]
考虑$U$的一个穷竭,进而得到一个至多可数紧集族$K_1\subset K_2\subset \cdots$使得每一个$K_n\subset K_{n+1}^\circ$且$\bigcup_i K_i=U$. 如果$\rr^n$是局部凸的,所以这些紧集也总可以取成凸的\footnote{参见Theorem \ref{thm:3.4}的证明。}。考虑如下半范数族
\[
	p_n(f):=\sup\{|D^\alpha f(x)|\,:\,x\in K_n,\, |\alpha|\leq n\}.
\]
因为$K_n\neq U$,所以它们不是范数。而这个半范数族是可分点的,这里即没有非平凡的公共零点,实际上,如果存在一个$f$和$x_0\in U$有$f(x_0)\neq 0$,则必然有一个$K_n$使得$x_0\in K_n$,此时$p_n(f)\geq |f(x_0)|>0$.

此时,Lemma \ref{lem:1.45}或者Theorem \ref{1.61}可以给出$\msf C^\infty(U)$上的一个第一可数局部凸拓扑矢量空间结构。局部基由
\[
	V_n=\left\{f\in \msf C^\infty(U)\,:\, p_n(f)<\frac 1n \right\}
\]
给出。此外,任取$x\in U$,我们都给出了一个线性泛函$f\mapsto f(x)$,它们是连续的。

$\msf C^\infty(U)$和$\msf C(U)$一样是完备的。任取$\msf C^\infty(U)$中的Cauchy列$\{f_i\}$,给定正整数$n$,对足够大的$i$, $j$,总存在一个$N$使得$i$, $j>N$时成立
\[
	\sup_{x\in U}\left|D^\alpha f_i(x)-D^\alpha f_j(x)\right|<\frac 1n,
\]
其中$|\alpha|\leq n$. 因此,固定$\alpha$,$\{D^\alpha f_i\}$在$\msf C(U)$拓扑下是一个Cauchy列,所以利用$\msf C(U)$的完备性,$\{D^\alpha f_i\}$都收敛到一个连续函数$g_\alpha$. 特别地,$\alpha=0$时候给出了$\{f_i\}$收敛到一个连续函数$f=g_0$. 验证$f$的光滑性的法子与$\msf C(U)$中验证连续性是相似的,局部检验在紧集中检验,然后利用函数列在紧集上的一致收敛性,这里我们引入一个数学家们幼儿园就知道的定理:
\begin{quote}
定义在赋范空间的一个凸有界集$E$上的复可微函数序列$\{f_i\}$,如果其导函数$\{f'_i\}$序列一致收敛到$g$且$\{f_i\}$至少在$E$的一点收敛,则$\{f_i\}$一致收敛于一个函数$f$,且$f'=g$.
\end{quote}
所以$f$光滑,且$D^\alpha f=g_\alpha$.
\end{para}

\begin{para}[测试函数空间]
设$K$是$\rr^n$的一个紧子集,所有支集处于$K$中的$\rr^n$上的光滑函数构成的集合我们记作$\mathscr D_K$,称为紧集$K$上的测试函数空间。它显然是一个矢量空间,是$\msf C^\infty(\rr^n)$的子空间,我们赋予其子空间拓扑。

我们还可以用另一种方式来描述$\mathscr D_K$,任取$x\not\in K$,泛函$x$的零点集为$x^{-1}(0)$,他是所有在$x$处为零的光滑函数构成的集合,由$x$是连续的,它也是闭的。此时,$\mathscr D_K=\bigcap_{x\not\in K}x^{-1}(0)$. 所以$\mathscr D_K$是$\msf C(U)$的闭子空间,其中$U$是任意一个包含$K$的开集。

由于$\msf C(U)$是完备的,作为其闭子空间,$\mathscr D_K$也是完备的。
\end{para}

\section{Lebesgue空间}

\section{H\"{o}lder空间}