\chapter{Hilbert空间}

\section{Hilbert空间}
\begin{para}
设$\mathcal{H}$是复数域上的矢量空间。假设$x$, $y\in \mathcal{H}$,有一个二元运算写作$(x,y)$,满足如下性质:
\begin{compactenum}
\item $(x,y)=(y,x)^*$,这里用$a^*$表示$a$的复共轭。

\item 对第二个变量线性,即对任意$\xi_1$, $\xi_2\in\cc$以及$x$, $y_1$, $y_2\in \mathcal{H}$满足
\[
(x,\xi_1y_1+\xi_2y_2)=\xi_1(x,y_1)+\xi_2(x,y_2),
\]
结合上面两个性质,很容易证明内积对第一个变量反线性,即
\[
(\eta_1x_1+\eta_2x_2,y)=\eta_1^*(x_1,y)+\eta_2^*(x_2,y).
\]
\item 对任意的$x\in \mathcal{H}$,$(x,x)\geq 0$,等号当且仅当$x=0$的时候取到。
\end{compactenum}
则称二元运算$(*,*)$是空间$\mathcal{H}$上的一个(复)内积,而$\mathcal{H}$就被称为一个(复)内积空间。对于实数域,不难类比出一个实内积空间的定义。

在内积空间上可以定义一个矢量的长度,
\[
	|x|=\sqrt{(x,x)},
\]
注意到,我们重用了绝对值符号。在具体语境下,含义基本还是可以分辨的,否则我们会具体指出。从内积的定义,长度为零的矢量只有零。
\end{para}

\begin{para}
两个矢量$x$和$y$被称为正交的,如果$(x,y)=0$. 两个矢量正交,则它们的模方成立勾股定理:如果$y$和$x$正交,则
\[
	|x+y|^2=|x|^2+|y|^2+(x,y)+(y,x)=|x|^2+|y|^2.
\]
\end{para}

\begin{pro}[Cauchy-Schwarz不等式]
$|(x,y)|\leq |x||y|$.
\end{pro}

\begin{proof}
等式对$x=0$的情况是显然成立的,下面设$x\neq 0$. 令$\alpha=(x,y)$,考虑
	\[
		0\leq |\lambda x+y|^2=|\lambda|^2|x|^2+|y|^2+\lambda^* \alpha+\lambda \alpha^*.
	\]
	在这之中,令$\lambda=-\alpha/|x|^2$,则上述不等式变为
	\[
		0\leq |\lambda x+y|^2=|y|^2-\frac{|\alpha|^2}{|x|^2},
	\]
	这就是我们的需要的不等式。
\end{proof}

\begin{pro}
$|y|\leq |\lambda x+y|$对每一个$\lambda \in \cc$成立当且仅当$(x,y)=0$.
\end{pro}

\begin{proof}
对于$(x,y)=0$,由勾股定理
\[
	|\lambda x+y|^2=|\lambda x|^2+ |y|^2\geq |y|^2.
\]
反过来,如果$(x,y)\neq 0$,只要找一个$\lambda$使得$|y|^2>|\lambda x+y|^2$即可。同上一个命题取$\lambda=-\alpha/|x|^2$,则
\[
	|\lambda x+y|^2=|y|^2-\frac{|\alpha|^2}{|x|^2},
\]
在$(x,y)\neq 0$时$|y|^2>|\lambda x+y|^2$. 
\end{proof}

此时,我们可以检验$|*|$构成了一个范数,因为
\[
	|x|\geq 0,\quad |\lambda x|=|\lambda||x|,
\]
以及最重要的
\[
	|x+y|^2=|x|^2+|y|^2+(x,y)+(y,x)\leq |x|^2+|y|^2+2|x||y|=(|x|+|y|)^2,
\]
这就是三角不等式$|x+y|\leq |x|+|y|$.

\begin{para}
有了范数,就自然有了一个不变度量$d(x,y)=|x-y|$,这样我们就在内积空间上定义出了一个拓扑。在这个拓扑下,$(*,*)$是一个连续函数:实际上,由于
\begin{align*}
	|(x+\Delta x,y+\Delta y)-(x,y)|&=|(x+\Delta x,y+\Delta y)-(x+\Delta x,y)+(x+\Delta x,y) -(x,y)|\\
	&\leq |(x+\Delta x,\Delta y)|+|(\Delta x,y)|\\
	&\leq |x+\Delta x||\Delta y|+|\Delta x||y|\\
	&\leq |x||\Delta y|+|\Delta x||y|+|\Delta x||\Delta y|,
\end{align*}
对于给定的正数$\delta$,都可以选取足够小的正数$\epsilon$使得
\[
	|x|\epsilon<\frac{\delta}{3},\quad |y|\epsilon<\frac{\delta}{3},\quad \epsilon^2<\frac{\delta}{3},
\]
此时,只要$|\Delta y|<\epsilon$和$|\Delta x|<\epsilon$,则$|(x+\Delta x,y+\Delta y)-(x,y)|<\delta$. 所以$(*,*)$是一个连续函数。

利用这点,我们可以说明$x$的正交补空间$N(x)=\{y\in\mathcal{H}\,:\, (x,y)=0\}$是一个闭集,因为$N(x)$是$\Lambda_x(y)=(x,y)$关于$\{0\}$的原像。
\end{para}

如果内积空间$\mathcal{H}$在这个范数下是完备的(即Cauchy列收敛),即当内积空间是一个Banach空间时,$\mathcal{H}$被称为一个Hilbert空间。

\begin{pro}
Hilbert空间的非空凸闭子集$E$必然有着唯一的范数最小的元素。
\end{pro}

\begin{proof}
考虑$E$中所有元素范数的下界$d$,然后找一个序列$\{x_i\}$使得$|x_i|\to d$,由于$(x_m+x_n)/2\in E$,故$|(x_m+x_n)/2|\geq d$,考虑恒等式
\[
	0\leq |x_m-x_n|^2=2|x_m|^2+2|x_n|^2-|x_m+x_n|^2\leq 2(|x_m|^2+|x_n|^2)-4d^2,
\]
或者
\[
	|x_m-x_n|^2\leq 2(|x_m|^2-d^2)+2(|x_m|^2-d^2)\leq 2\bigl||x_m|^2-d^2\bigr|+2\bigl||x_m|^2-d^2\bigr|,
\]
因此$\{x_i\}$是一个Cauchy列,故而收敛到某个$x\in H$使得$|x|=d$. 由于$x$是$E$的极限点,而$E$是闭集,所以$x\in E$. 至于唯一性,如果$|y|=d$,考虑$\{x$, $y$, $x$, $y$, $\cdots\}$的极限即可。
\end{proof}

\begin{pro}
对于Hilbert空间$\mathcal{H}$以及他的任意一个子空间$M$,$M$的正交补$M^\bot=\{y\in H\,:\, \forall x\in M,\, (x,y)=0\}$是一个闭子空间。如果$M$还是闭的,则成立直和分解$\mathcal{H}=M\oplus M^\bot$.
\end{pro}

\begin{proof}
由于
\[
	M^\bot=\bigcap_{x\in M}N(x),
\]
而$N(x)$是闭集,所以$M^\bot$自然是闭集。由于$(x,y)$关于$y$线性,所以$M^\bot$也是一个子空间。

现在假设$M$是闭的。剩下的就是要证明$M\cap M^\bot=\{0\}$,以及任意的$x\in \mathcal{H}$都可以分解为$x=x_1+x_2$,其中$x_1\in M$, $x_2\in M^\bot$. 对于前者,假设$x\in M\cap M^\bot$,因此$(x,x)=0$就推出了$x=0$. 

对后者,设$x\in \mathcal{H}$,考虑集合$x-M$,这是$\mathcal{H}$的一个闭凸子集,闭来自于他是$M$的一个平移,而凸来自于$M$是一个子空间。所以$x-M$中必然有唯一的$x_1$使得$|x-x_1|$最小。令$x_2=x-x_1$,则对于所有的$y\in M$都成立$|x_2|\leq |x_2+y|$,特别地,对于任意的$\lambda\in \cc$有$|x_2|\leq |x_2+\lambda x_1|$,因此$(x_1,x_2)=0$. 综上,任取$x\in \mathcal{H}$,成立分解$x=x_1+x_2$,$x_1\in M$,而$x_2\in M^\bot$.
\end{proof}

使用内积,可以在$\mathcal{H}$和$\mathcal{H}^*$之间建立一个线性映射:通过$\Lambda_y(x)=(y,x)$可以定义出一个线性函数$\Lambda_y\in \mathcal{H}^*$,这样我们就得到一个线性映射$\Lambda:\mathcal{H}\to \mathcal{H}^*$,他将$y$变为$\Lambda_{y}$.

\begin{thm}[Riesz-Frechet表示定理]
设$\mathcal{H}$是一个Hilbert空间,则$\Lambda:\mathcal{H}\to \mathcal{H}^*$是一个同构,且$|\Lambda_y|=|y|$.
\end{thm}

\begin{proof}
	注意到,$|\Lambda_y|\geq |y|$就可以推出$\Lambda$是一个单射。因为如果$|\Lambda_y|=0$,则$|y|=0$可以推出$y=0$. 所以我们先推$|\Lambda_y|=|y|$,这样单射就是显然的了。

	由Cauchy-Schwarz不等式,在$|x|\leq 1$时,$|\Lambda_y(x)|\leq |y||x|\leq |y|$,所以$|\Lambda_y|\leq |y|$. 反过来,由于
	\[
		|y|^2=\Lambda_y(y)=|\Lambda_y(y)|\leq |\Lambda_y||y|,
	\]
	所以$|y|\leq |\Lambda_y|$. 这样就得到了$|\Lambda_y|=|y|$. 

	剩下的就是要证明$\Lambda$是满射。若$y=0$,则取$\Lambda_y=0$. 若$y\neq 0$,令$N(\Lambda_y)$为$\Lambda_y$的零空间,即
	\[
		N(\Lambda_y)=\bigl\{x\in\mathcal{H}\,:\, \Lambda_y(x)=(y,x)=0\bigr\}.
	\]
	由于$N(\Lambda_y)=N(y)$是一个闭子空间,所以存在直和分解$\mathcal{H}=N(\Lambda_y)\oplus N(\Lambda_y)^\bot$. 取定一个非零的$z\in N(\Lambda_y)^\bot$,因为
	\[
		(\Lambda_y(x))z-(\Lambda_y(z))x\in N(\Lambda_y),
	\]
	所以
	\[
		(z,(\Lambda_y(x))z-(\Lambda_y(z))x)=\Lambda_y(x)(z,z)-\Lambda_y(z)(z,x)=0,
	\]
	或者
	\[
		\Lambda_y(x)=\left(\frac{\Lambda_y(z)}{(z,z)}z,x\right),
	\]
	此时当$y=\Lambda_y(z)z/(z,z)$时上式成立。
\end{proof}
