%!TEX program = xelatex
\documentclass[9pt]{extbook}
\usepackage[book,zh]{noteheader}
\usepackage[fontset = fandol]{ctex}
\usepackage[chapter]{../egastyle}
\usepackage{hyperref}
	\hypersetup{
		pdftoolbar=true,  
		pdfmenubar=true,  
	}

\definecolor{shadecolor}{rgb}{0.92,0.92,0.92}

\newcommand{\no}[1]{{$(#1)$}}
% \renewcommand{\not}[1]{#1\!\!\!/}
\newcommand{\rr}{\mathbb{R}}
\newcommand{\zz}{\mathbb{Z}}
\newcommand{\aaa}{\mathfrak{a}}
\newcommand{\pp}{\mathfrak{p}}
\newcommand{\mm}{\mathfrak{m}}
\newcommand{\dd}{\mathrm{d}}
\newcommand{\oo}{\mathcal{O}}
\newcommand{\calf}{\mathcal{F}}
\newcommand{\calg}{\mathcal{G}}
\newcommand{\bbp}{\mathbb{P}}
\newcommand{\bba}{\mathbb{A}}
\newcommand{\osub}{\underset{\mathrm{open}}{\subset}}
\newcommand{\csub}{\underset{\mathrm{closed}}{\subset}}

\DeclareMathOperator{\im}{Im}
\DeclareMathOperator{\Hom}{Hom}
\DeclareMathOperator{\id}{id}
\DeclareMathOperator{\rank}{rank}
\DeclareMathOperator{\tr}{tr}
\DeclareMathOperator{\supp}{supp}
\DeclareMathOperator{\coker}{coker}
\DeclareMathOperator{\codim}{codim}
\DeclareMathOperator{\height}{height}
\DeclareMathOperator{\sign}{sign}

\DeclareMathOperator{\ann}{ann}
\DeclareMathOperator{\Ann}{Ann}
\DeclareMathOperator{\ev}{ev}
\newcommand{\ii}{\mathrm{i}}
\newcommand{\cc}{\mathbb{C}}

\begin{document}
\title{分析学笔记}
\author{Buwai. Lee@School of Physics, NJU}
\date{2016秋季学期}
\maketitle %标题
\frontmatter
\tableofcontents

\mainmatter
\chapter{测度与拓扑矢量空间}
\section{Hilbert空间}


\chapter{haha}
\section{Hilbert空间}

设$\mathcal{H}$是复数域上的矢量空间。假设$x$, $y\in \mathcal{H}$,有一个二元运算写作$(x,y)$,需要满足如下性质
\begin{itemize}
\item $(x,y)=(y,x)^*,$

这里用$a^*$表示$a$的复共轭。

\item 对第二个变量线性
\[
(x,\xi_1y_1+\xi_2y_2)=\xi_1(x,y_1)+\xi_2(x,y_2),
\]
结合上面两个性质,很容易证明内积对第一个变量反线性,即
\[
(\eta_1x_1+\eta_2x_2,y)=\eta_1^*(x_1,y)+\eta_2^*(x_2,y).
\]
\item 对任意的$x\in \mathcal{H}$,$(x,x)\geq 0$,等号当且仅当$x=0$的时候取到。
\end{itemize}

则称这个二元运算是一个内积,而$\mathcal{H}$被称为一个内积空间。在内积空间上可以定义一个矢量的长度
\[
	|x|=\sqrt{(x,x)},
\]
注意到,我们重用了绝对值符号。在具体语境下,含义基本还是可以分辨的,否则我们会具体指出。从内积的定义,长度为零的矢量只有零。

\para 两个矢量$y$和$x$被称为正交的,如果$(y,x)=0$. 于是我们就有勾股定理,如果$y$和$x$正交,则
\[
	|x+y|^2=|x|^2+|y|^2+(x,y)+(y,x)=|x|^2+|y|^2.
\]

\pro Cauchy-Schwarz不等式:$|(x,y)|\leq |x||y|$.

\proof 等式对$x=0$的情况是显然成立的,下面设$x\neq 0$. 令$\alpha=(x,y)$,考虑
	\[
		0\leq |\lambda x+y|^2=|\lambda|^2|x|^2+|y|^2+\lambda^* \alpha+\lambda \alpha^*.
	\]
	在这之中,令$\lambda=-\alpha/|x|^2$,则上述不等式变为
	\[
		0\leq |\lambda x+y|^2=|y|^2-\frac{|\alpha|^2}{|x|^2},
	\]
	这就是我们的需要的不等式。\qed

\pro $|y|\leq |\lambda x+y|$对每一个$\lambda \in \cc$成立当且仅当$(x,y)=0$.

\proof 对于$(x,y)=0$,由勾股定理
\[
	|\lambda x+y|^2=|\lambda x|^2+ |y|^2\geq |y|^2.
\]
反过来,如果$(x,y)\neq 0$,只要找一个$\lambda$使得$|y|^2>|\lambda x+y|^2$即可。同上一个命题取$\lambda=-\alpha/|x|^2$,则
\[
	|\lambda x+y|^2=|y|^2-\frac{|\alpha|^2}{|x|^2},
\]
在$(x,y)\neq 0$时$|y|^2>|\lambda x+y|^2$. \qed

此时,我们可以检验$|*|$构成了一个范数,因为
\[
	|x|\geq 0,\quad |\lambda x|=|\lambda||x|,
\]
以及最重要的
\[
	|x+y|^2=|x|^2+|y|^2+(x,y)+(y,x)\leq |x|^2+|y|^2+2|x||y|=(|x|+|y|)^2,
\]
这就是三角不等式$|x+y|\leq |x|+|y|$.

\para 有了范数,就自然有了一个度量$d(x,y)=|x-y|$,于是内积空间就是一个度量空间。有了度量,就很容易定义出拓扑,由于是矢量空间,是加法群,所以在单位元$0$附近定义拓扑即可,剩下的可以通过平移移过去。选取所有$|x|<r$的$x$的集合作为单位元处的开集基,这样我们就定义出了$\mathcal{H}$上的一个拓扑。

固定$x$,$(x,y)$对$y$是一个连续函数,因为
\[
	|(x,y+\Delta y)-(x,y)|=|(x,\Delta y)|\leq |x||\Delta y|,
\]
对选定的$\delta$,都可以选取$\epsilon=\delta/|x|$使得$\Delta y<\epsilon$时有$|(x,y+\Delta y)-(x,y)|<\delta$. 反过来,固定$y$,$(x,y)$对$x$也是一个连续函数。

利用这点,我们可以说明$x$的正交补空间$N(x)=\{y\in\mathcal{H}\,:\, (x,y)=0\}$是一个闭集,因为$N(x)$是$\Lambda_x(y)=(x,y)$关于$\{0\}$的原像。

\para 如果我们的内积空间$\mathcal{H}$在这个范数下是完备的(即Cauchy列收敛),则称$\mathcal{H}$是一个Hilbert空间。由于Cauchy列收敛,所以Hilbert空间的非空闭凸子集$E$\footnote{所谓凸集就是说,如果$x$和$y$都在该集合内,则对任意的$t\in [0,1]$都有$tx+(1-t)y$也在该集合内。}必然有着唯一的范数最小的元素。

\proof 考虑$E$中所有元素范数的下界$d$,然后找一个序列$\{x_i\}$使得$|x_i|\to d$,由于$(x_m+x_n)/2\in E$,故$|(x_m+x_n)/2|\geq d$,考虑恒等式
\[
	0\leq |x_m-x_n|^2=2|x_m|^2+2|x_n|^2-|x_m+x_n|^2\leq 2(|x_m|^2+|x_n|^2)-4d^2,
\]
或者
\[
	|x_m-x_n|^2\leq 2(|x_m|^2-d^2)+2(|x_m|^2-d^2)\leq 2\bigl||x_m|^2-d^2\bigr|+2\bigl||x_m|^2-d^2\bigr|,
\]
因此$\{x_i\}$是一个Cauchy列,故而收敛到某个$x\in H$使得$|x|=d$. 由于$x$是$E$的极限点,而$E$是闭集,所以$x\in E$. 至于唯一性,如果$|y|=d$,考虑$\{x$, $y$, $x$, $y$, $\cdots\}$的极限即可。\qed

\pro 对于Hilbert空间$\mathcal{H}$以及他的任意一个子空间$M$,$M$的正交补$M^\bot=\{y\in H\,:\, \forall x\in M,\, (x,y)=0\}$是一个闭子空间。如果$M$还是闭的,则成立直和分解$\mathcal{H}=M\oplus M^\bot$.

\proof 由于
\[
	M^\bot=\bigcap_{x\in M}N(x),
\]
而$N(x)$是闭集,所以$M^\bot$自然是闭集。由于$(x,y)$关于$y$线性,所以$M^\bot$也是一个子空间。

现在假设$M$是闭的。剩下的就是要证明$M\cap M^\bot=\{0\}$,以及任意的$x\in \mathcal{H}$都可以分解为$x=x_1+x_2$,其中$x_1\in M$, $x_2\in M^\bot$. 对于前者,假设$x\in M\cap M^\bot$,因此$(x,x)=0$就推出了$x=0$. 

对后者,设$x\in \mathcal{H}$,考虑集合$x-M$,这是$\mathcal{H}$的一个闭凸子集,闭来自于他是$M$的一个平移,而凸来自于$M$是一个子空间。所以$x-M$中必然有唯一的$x_1$使得$|x-x_1|$最小。令$x_2=x-x_1$,则对于所有的$y\in M$都成立$|x_2|\leq |x_2+y|$,特别地,对于任意的$\lambda\in \cc$有$|x_2|\leq |x_2+\lambda x_1|$,因此$(x_1,x_2)=0$. 综上,任取$x\in \mathcal{H}$,成立分解$x=x_1+x_2$,$x_1\in M$,而$x_2\in M^\bot$.\qed

\para Hilbert空间$\mathcal{H}$的对偶空间$\mathcal{H}^*$是由$\mathcal{H}$上所有线性函数张成的矢量空间,上面可以如下赋予范数
\[
	|T|=\sup\bigl\{|Tx|\,:\,x\in \mathcal{H},\,|x|\leq 1\bigr\},
\]
很容易检验这是一个范数,比如三角不等式:
\[
	|(T+S)x|=|Tx|+|Sx|\leq |T|+|S|,
\]
对任意的$|x|\leq 1$都成立,所以$|T+S|\leq |T|+|S|$.

关于这个范数还有一个有用的不等式$|Tx|\leq |T||x|$:显然,$|x/|x||=1$,所以由
\[
	\left|T\frac{x}{|x|}\right|\leq |T|
\]
可以推知$|Tx|\leq |T||x|$. 

\para 使用内积,可以在$\mathcal{H}$和$\mathcal{H}^*$之间建立一个线性映射:通过$\Lambda_y(x)=(y,x)$可以定义出一个线性函数$\Lambda_y\in \mathcal{H}^*$,这样我们就得到一个线性映射$\Lambda:\mathcal{H}\to \mathcal{H}^*$,他将$y$变为$\Lambda_{y}$.

\theo Riesz表示定理:设$\mathcal{H}$是一个Hilbert空间,则$\Lambda:\mathcal{H}\to \mathcal{H}^*$是一个同构,且$|\Lambda_y|=|y|$.

\proof
	注意到,$|\Lambda_y|\geq |y|$就可以推出$\Lambda$是一个单射。因为如果$|\Lambda_y|=0$,则$|y|=0$可以推出$y=0$. 所以我们先推$|\Lambda_y|=|y|$,这样单射就是显然的了。

	由Cauchy-Schwarz不等式,在$|x|\leq 1$时,$|\Lambda_y(x)|\leq |y||x|\leq |y|$,所以$\Lambda_y|\leq |y|$. 反过来,由于
	\[
		|y|^2=\Lambda_y(y)=|\Lambda_y(y)|\leq |\Lambda_y||y|,
	\]
	所以$|y|\leq |\Lambda_y|$. 这样就得到了$|\Lambda_y|=|y|$. 

	剩下的就是要证明$\Lambda$是满射。若$y=0$,则取$\Lambda_y=0$. 若$y\neq 0$,令$N(\Lambda_y)$为$\Lambda_y$的零空间,即
	\[
		N(\Lambda_y)=\bigl\{x\in\mathcal{H}\,:\, \Lambda_y(x)=(y,x)=0\bigr\}.
	\]
	由于$N(\Lambda_y)=N(y)$是一个闭子空间,所以存在直和分解$\mathcal{H}=N(\Lambda_y)\oplus N(\Lambda_y)^\bot$. 取定一个非零的$z\in N(\Lambda_y)^\bot$,因为
	\[
		(\Lambda_y(x))z-(\Lambda_y(z))x\in N(\Lambda_y),
	\]
	所以
	\[
		(z,(\Lambda_y(x))z-(\Lambda_y(z))x)=\Lambda_y(x)(z,z)-\Lambda_y(z)(z,x)=0,
	\]
	或者
	\[
		\Lambda_y(x)=\left(\frac{\Lambda_y(z)}{(z,z)}z,x\right),
	\]
	此时当$y=\Lambda_y(z)z/(z,z)$时上式成立。\qed

\section{伴随算子}
\end{document}