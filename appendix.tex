\chapter{Appendix: Exterior Algebra}

\para 这里复习一下外代数的内容。设$V$是一个$n$维矢量空间,记$\Omega^k(V)$为所有反对称线性函数
\[
	f:\underbrace{V\times\cdots\times V}_{k\text{个}}\to \rr
\]
的集合。这个集合有一个显然的线性结构。此外再约定$\Omega^1(V)=V^*$和$\Omega^0(V)=\rr$.我们将$\Omega^k(V)$称为$V$的$k$-次外代数。

\para 	设$\xi \in \Omega^m(V)$和$\eta \in \Omega^n(V)$,定义$\xi$和$\eta$的外积$\xi \wedge \eta \in \Omega^{m+n}(V)$为
	\[
		\xi \wedge \eta=\frac{1}{m!n!}\sum_{\sigma\in S^{m+n}}(-1)^{\mathrm{sign}(\sigma)}\sigma (\xi \otimes \eta).
	\]
	其中$\sigma$属于$m+n$阶置换群且$\sigma (\xi \otimes \eta)$被定义为
	\[
		\sigma (\xi \otimes \eta)(v_1,\dots,v_{m+n})=(\xi \otimes \eta)(v_{\sigma(1)},\dots,v_{\sigma(m+n)}).
	\]
	而$\xi \otimes \eta$(称为张量积)被定义为双线性的运算
	\[
		\xi \otimes \eta(v_1,\dots,v_m,v_{m+1},\dots,v_{m+n})
		=\xi(v_1,\dots,v_m)\eta(v_{m+1},\dots,v_{m+n}).
	\]
	容易验证张量积满足结合律。

\pro 设$\xi$, $\xi_1$, $\xi_2 \in \Omega^m(V)$和$\eta$, $\eta_1$, $\eta_2\in \Omega^n(V)$还有一个$\zeta\in \Omega^h(V)$,有

	\no{1}分配律:
	\[
		\begin{split}
			(\xi_1+\xi_2)\wedge \eta&=\xi_1 \wedge \eta+\xi_2 \wedge \eta, \\
			\xi \wedge (\eta_1+\eta_2)&=\xi \wedge \eta_1+\xi \wedge \eta_2.
		\end{split}
	\]

	\no{2}反变换律:$\xi \wedge \eta=(-1)^{mn}\eta \wedge \xi$.

	\no{3}结合律:$(\xi \wedge \eta)\wedge \zeta=\xi \wedge (\eta\wedge \zeta)$.

\proof \no{1} 由$\sigma$线性和张量积线性显然。

	\no{2} 设置换
	\[
		\tau=
		\begin{pmatrix}
			1& \cdots & m & m+1 &\cdots &m+n\\
			1+n& \cdots & m+n & 1 &\cdots &n
		\end{pmatrix},
	\]
	容易证明$(-1)^{\mathrm{sign}(\tau)}=(-1)^{mn}$.

	由线性性,我们只要对一个分量证明就可以了。按基打开直接写出
	\[
		\begin{split}
			\xi \wedge \eta&=\frac{\xi^{i_1\dots i_m}\eta^{j_1 \dots j_n}}{m!n!}\sum_{\sigma\in S^{m+n}}(-1)^{\mathrm{sign}(\sigma)}\sigma(v_{i_1}\otimes \cdots \otimes v_{i_m}\otimes v_{j_1}\otimes \cdots \otimes v_{j_n})\\
			\eta \wedge \xi&=\frac{\xi^{i_1\dots i_m}\eta^{j_1 \dots j_n}}{m!n!}\sum_{\sigma\in S^{m+n}}(-1)^{\mathrm{sign}(\sigma)}\sigma(v_{j_1}\otimes \cdots \otimes v_{j_n}\otimes v_{i_1}\otimes \cdots \otimes v_{i_m})
		\end{split}
	\]
	注意到$\sigma$跑遍所有置换,那么$\sigma$和$\tau$的复合$\sigma'=\sigma\circ\tau$也跑遍所有置换,且$(-1)^{\mathrm{sign}(\sigma')}=(-1)^{\mathrm{sign}(\sigma)}(-1)^{\mathrm{sign}(\tau)}$,所以
	\[
		\begin{split}
			\eta \wedge \xi&=\frac{\xi^{i_1\cdots i_m}\eta^{j_1 \cdots j_n}}{m!n!}\sum_{\sigma'\in S^{m+n}}(-1)^{\mathrm{sign}(\sigma')}\sigma'(v_{j_1}\otimes \cdots \otimes v_{j_n}\otimes v_{i_1}\otimes \cdots \otimes v_{i_m})\\
			&=\frac{\xi^{i_1\cdots i_m}\eta^{j_1 \cdots j_n}}{m!n!}\sum_{\sigma\in S^{m+n}}(-1)^{\mathrm{sign}(\sigma)}(-1)^{\mathrm{sign}(\tau)}\sigma(v_{i_1}\otimes \cdots \otimes v_{i_m}\otimes v_{j_1}\otimes \cdots \otimes v_{j_n})\\
			&=(-1)^{\mathrm{sign}(\tau)}\frac{\xi^{i_1\cdots i_m}\eta^{j_1 \cdots j_n}}{m!n!}\sum_{\sigma\in S^{m+n}}(-1)^{\mathrm{sign}(\sigma)}\sigma(v_{i_1}\otimes \cdots \otimes v_{i_m}\otimes v_{j_1}\otimes \cdots \otimes v_{j_n})\\
			&=(-1)^{\mathrm{sign}(\tau)} \xi \wedge \eta.
		\end{split}
	\]
	代入$(-1)^{\mathrm{sign}(\tau)}=(-1)^{mn}$即得证\no{2}.

	第三个性质,也就是结合律的证明就是死算,略去计算最后得到:
	\[
		(\xi \wedge \eta) \wedge \zeta=\frac{1}{m!n!h!}\sum_{\sigma\in S^{m+n+h}}(-1)^{\mathrm{sign}(\sigma)}\sigma(\xi \otimes \eta \otimes \zeta)=\xi \wedge (\eta \wedge \zeta).
	\]
	\qed

\para 留意$v,w\in \Omega^1(V)=V^*$的外积$v\wedge w$是有趣的。首先,显然地,$\omega\wedge\omega=0$。然后容易证明对$a,b\in V$
\[
	v\wedge w(a,b)=v(a)w(b)- w(a)v(b)=\begin{vmatrix}v(a)&v(b)\\w(a)&w(b)\end{vmatrix}.
\]
对于多个$\omega_i \in \Omega^1(V)$的外积,我们可以用归纳法证明
\[
	\omega_1\wedge \cdots \wedge\omega_p(v_1,\dots,v_p)=
	\begin{vmatrix}
		\omega_1(v_1)&\cdots&\omega_p(v_p)\\
		\vdots&\ddots&\vdots\\
		\omega_p(v_1)&\cdots&\omega_p(v_p)
	\end{vmatrix}.
\]

\para 还有一个很类似的结论,对于多个$\omega_i \in \Omega^1(V)$的外积来说我们有$\omega_i \wedge \omega_j=-\omega_j \wedge \omega_i$,而且是线性的。我们考虑$p$个$\omega_i \in \Omega^1(V)$之间的外积$F(\omega_1, \dots,\omega_n)=\omega_1\wedge \cdots \wedge \omega_p$,
如果矢量空间$V^*$的基是$\{v_i\}$,由于$F$是反对称线性映射,我们有熟知的分解:
\[
	F(\omega_1, \dots,\omega_p)=\det(\omega_1, \dots,\omega_p)F(v_1, \dots,v_p),
\]
或者写作
\[
	\omega_1\wedge \cdots \wedge \omega_p=
	\begin{vmatrix}
		\omega_{11}&\cdots&\omega_{p1}\\
		\vdots&\ddots&\vdots\\
		\omega_{1p}&\cdots&\omega_{pp}
	\end{vmatrix}
	v_1\wedge \cdots \wedge v_p.
\]
其中$\omega_{ij}$值的是$\omega_i$在$v_j$方向的分量值。从这里可以看到,若$\omega_i$们线性相关的,则$\omega_1\wedge \cdots \wedge \omega_p=0$.若$V^*$的维度是$n$,而$p>n$,则$\omega_i$必然线性相关,则$\omega_1\wedge \cdots \wedge \omega_p=0$.

\para 特别地,如果$\omega'=\dd x'$和$\eta'=\dd y'$,且$x'$和$y'$可以看做$x$和$y$的函数,而
\[
	\dd x'=\partial_x x' \dd x+\partial_y x' \dd y,\quad \dd y'=\partial_x y' \dd x+\partial_y y' \dd y,
\]
因此
\[
	\dd x'\wedge \dd y'=\det\left(\frac{\partial(x',y')}{\partial (x,y)}\right)\dd x\wedge \dd y,
\]
其中$\partial(x',y')/\partial (x,y)$是$(x',y')$关于$(x,y)$的Jacobian. 如果还记得积分学的一些基本事实,会发现这就是和重积分变量替换公式是如此的相似。实际上,对于二个变量的重积分变换公式,他写作
\[
	\int_S\dd x' \dd y'=\int_S\left|\det\left(\frac{\partial(x',y')}{\partial (x,y)}\right)\right|\dd x \dd y,
\]
其中绝对值的引入是因为换变量可能会改变积分区域的定向。

所以,从这个角度来看,正如我们对于$\dd x$是无穷小距离的直观一样,我们可以认为$\dd x\wedge \dd y$是无穷小面积。

\para 设$\{e^i:1\leq i \leq n\}$是$V^*$的一组基,则$\{e^{i_1}\wedge \cdots \wedge e^{i_k}\}$是$\Omega^k(V)$的一组基。于是当$k>n$时$\Omega^k(V)=0$,当$0\leq k \leq n$时,$\dim \Omega^k(V)=\tbinom nk$.

\lem (Cartan引理)设$\{v_i:1\leq i \leq r\}$和$\{w_i:1\leq i \leq r\}$是$V^*$中的两组矢量,且$\sum_{i=1}^rv_i\wedge w_i=0$。如果$\{v_i\}$线性无关,则每个$w_i$可以由$\{v_i\}$线性组合而成$w_i=\sum_{i=1}^r\alpha_{ij}v_j$,且$\alpha_{ij}=\alpha_{ji}$.

\proof
	将$\{v_i\}$扩充为一组基,则$w_i=\sum_{i=1}^r\alpha_{ij}v_j$,将其代入条件$\sum_{i=1}^rv_i\wedge w_i=0$有
	\[
		0=\sum_{j=1}^n\sum_{i=1}^r \alpha_{ij}v_i \wedge v_j=\sum_{1\leq i < j \leq n}(\alpha_{ij}-\alpha_{ji}) v_i \wedge v_j+\sum_{j=r+1}^n\sum_{i=1}^r \alpha_{ij}v_i \wedge v_j,
	\]
	由于$v_i \wedge v_j$是$\Omega^2(V)$的一组基,所以$\alpha_{ij}=0$当$j>r$,且$\alpha_{ij}=\alpha_{ji}$当$j\leq r$.\qed

% \lem 设$\{v_i:1\leq i \leq r\}$是$V^*$中的一族线性无关的矢量,设$w$是一个$V$上的$p$-形式,则存在$\{\psi_i\in \Omega^{k-1}(V):1\leq i \leq r\}$使得$w=\sum_{i=1}^rv_j\wedge \psi_i$,当且仅当$v_1\wedge\cdots\wedge v_r\wedge w=0$.
\chapter{Appendix: Partition of Unity}