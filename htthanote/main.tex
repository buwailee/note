% author: buwailee@nmhs
\documentclass[12pt]{article}
\usepackage[b5paper]{../noteheader}

% \usepackage[chapter]{../egastyle}
\theoremstyle{definition}
	\newtheorem{para}{}[section]
		\renewcommand{\thepara}{\thesection.\arabic{para}}
	\newtheorem{exa}[para]{Example}
\theoremstyle{plain}
	\newtheorem{lem}[para]{Lemma}
	\newtheorem{theo}[para]{Theorem}
	\newtheorem{thm}[para]{Theorem}
	\newtheorem{pro}[para]{Proposition}
    \newtheorem{coro}[para]{Corollary}
\renewcommand{\proofname}{Proof}

\usepackage{titletoc,makeidx,paralist}
	\makeindex
	% \renewcommand{\indexname}{Index}
\usepackage[titletoc,title]{appendix}
	\renewcommand{\appendixname}{Appendix}

\usepackage{hyperref}
	\hypersetup{bookmarksnumbered=true}

\definecolor{shadecolor}{rgb}{0.92,0.92,0.92}

\newcommand{\no}[1]{{$(#1)$}}
% \renewcommand{\not}[1]{#1\!\!\!/}
\newcommand{\rr}{\mathbb{R}}
\newcommand{\zz}{\mathbb{Z}}
\newcommand{\aaa}{\mathfrak{a}}
\newcommand{\pp}{\mathfrak{p}}
\newcommand{\mm}{\mathfrak{m}}
\newcommand{\dd}{\mathrm{d}}
\newcommand{\oo}{\mathcal{O}}
\newcommand{\calf}{\mathcal{F}}
\newcommand{\calg}{\mathcal{G}}
\newcommand{\bbp}{\mathbb{P}}
\newcommand{\bba}{\mathbb{A}}
\newcommand{\osub}{\underset{\mathrm{open}}{\subset}}
\newcommand{\csub}{\underset{\mathrm{closed}}{\subset}}

\DeclareMathOperator{\im}{Im}
\DeclareMathOperator{\Hom}{Hom}
\DeclareMathOperator{\id}{id}
\DeclareMathOperator{\rank}{rank}
\DeclareMathOperator{\tr}{tr}
\DeclareMathOperator{\supp}{supp}
\DeclareMathOperator{\coker}{coker}
\DeclareMathOperator{\codim}{codim}
\DeclareMathOperator{\height}{height}
\DeclareMathOperator{\sign}{sign}

\DeclareMathOperator{\ann}{ann}
\DeclareMathOperator{\Ann}{Ann}
\DeclareMathOperator{\ev}{ev}
	\newcommand{\cc}{\mathbb{C}}
	\DeclareMathOperator{\spec}{Spec}

\begin{document}

\section{Simplicial Set}

\begin{para}[Simplicial set]
    Suppose $\mathsf{TOFS}$ is the category of totally ordered finite sets and order-preserving morphisms. Simplicial sets are just set-valued presheaive on $\mathsf{TOFS}$, {\it i.e.} functors $\mathsf{TOFS}^{\text{op}}\to \mathsf{Set}$. The category of simplicial sets is denoted by $\mathsf{sSet}$.
\end{para}

Let $[n]=0<1<\cdots<n$, the natural presheaf $\operatorname{Mor}_{\mathsf{TOFS}}(-,[n])$ is the standard $n$-simplex, denoted by $\Delta^n$. Therefore, the Yoneda lemma tells us that
\[
    K([n])\cong\operatorname{Mor}_{\mathsf{sSet}}(\Delta^n,K) 
\]
for any simplicial set $K$.

% Usually, we only care about the skeleton $\{[n]\,:\, n\in\mathbb N\}$ of $\mathsf{TOFS}$, where $[n]=0<1<\cdots<n$. 

One important example of simplicial sets is the nerve of a small category.

\begin{para}[Nerve of a small category]
    Every partially ordered set $P$ could be seen as a small category $i(P)$ in a natural way, where morphisms are order relations between objects. We thus obtain a functor $i$ from $\mathsf{TOFS}$ to the category of small categories. We can now describe the nerve of the category $C$ as the functor
    \[
        N(C): P \mapsto \operatorname{Fun}(i(P),C),
    \]
    where $\operatorname{Fun}(A,B)$ are set of functors between small categories $A$ and $B$.
\end{para}

Usually $N(C)_k:=N(C)([k])$ is described as a set, whose elements are $k$-tuples of composable morphisms
\[
    P_0\to P_1\to P_2\to \cdots \to P_{k-1}\to P_k
\]
of $C$. Elements of $N(C)_k$ are usually called $k$-cycles, especially, $0$-cycles are objects of $C$, $1$-cycles are morphisms of $C$. 

% Suppose $\eta:P\to Q$ is a morphism, then the morphism under the functor $N(C)_k$ is given by
% \[
%     \xymatrix{
%         P_0\ar[r]\ar[d]_{\eta_0} & P_1\ar[r]\ar[d]_{\eta_1} & P_2\ar[r]\ar[d]_{\eta_2} & \cdots \ar[r] & P_{k-1}\ar[r]\ar[d]_{\eta_{k-1}} & P_k\ar[d]_{\eta_k}\\
%         Q_0\ar[r] & Q_1\ar[r] & Q_2\ar[r] & \cdots \ar[r] & Q_{k-1}\ar[r] & Q_k
%     }
% \]
Now let's consider the face and degeneracy maps (functors), they can be used to connect different $k$. The face maps 
\[
     d_{i}\colon N(C)_{k}\to N(C)_{k-1}
\]
are given by composition of morphisms at the $i$-th object (or removing the $i$-th object from the sequence, when $i$ is $0$ or $k$). This means that $d_i$ sends the $k$-tuple 
\[P_{0}\to \cdots \to P_{i-1}\xrightarrow{f_{i-1}} P_{i}\xrightarrow{f_{i}} P_{i+1}\to \cdots \to P_{k}\]
to the $(k-1)$-tuple 
\[P_{0}\to \cdots \to P_{i-1}\xrightarrow{f_if_{i-1}} P_{i+1}\to \cdots \to P_{k}.\]
Similarly, the degeneracy maps 
\[s_{i}:N(C)_{k}\to N(C)_{k+1}\] 
are given by inserting an identity morphism at the object $P_i$. It's not hard to see that any order-preserving morphism is a composition of $d$ and $s$.

\begin{para}
    Every simplicial set $K$ can be seen as a colimit,
    \[
        K = \varinjlim_{\Delta^n\to K}\Delta^n.
    \]
    It's very easy to check.
\end{para}

\begin{exa}[Geometric relazation of the standard $n$-simplex]
    By definition, $C([n]) = \Delta^n$, {\it i.e.} the standard $n$-simplex $\Delta^n$ is the nerve of $[n]$. Generally, suppose $J$ is a linearly order set, denote $\Delta^J$ as the nerve of $J$.
    
    We can realize $\Delta^n$ geometrically as the subset 
    \[
        |\Delta^n|=\{(x_0,\dots,x_n)\in [0,1]^{n+1}\,:\, x_0+\cdots+x_n=1\}
    \]
    of $\mathbb R^{n+1}$, and for each $J=n_0<n_1<\cdots<n_k$, we associate it a $k$-face (cycle)
    \[
        |\Delta^J|=\{x\in \Delta^n\,:\, x_i=0\text{ when $i\not\in J$}\},
    \]
    which corrsponds to a (strict) order-preserving morphism $n_*:[k] \to [n]$. In this case, $d_i$ makes $x_{n_i}=0$ in $|\Delta^J|$.

    Now consider the $i$-th horn of $\Delta^n$, denoted by $\Lambda_i^n$, it's obtained by deleting the interior and the face opposite the $i$-th vertex of $|\Delta^n|$. In other words, 
    \[
        |\Lambda_i^n| = \{x\in\Delta^n\,:\, x_i\neq 0 \text{ and  $x_j=0$ for a $j\neq i$}\}
    \]
    whose faces (cycles) are proper subsets of $[n]$ containing $i$. It's a simplicial set again.
\end{exa}

\begin{para}[Geometric relazation of simplicial sets]
    % From now on, denote
    % \[
    %     |\Delta^n| = \{(x_0,\dots,x_n)\in [0,1]^{n+1}\,:\, x_0+\cdots+x_n=1\}.
    % \]
    % It's the geometric relazation of a standard $n$-simplex. Any morphism $f:[m]\to [n]$ defines a linear map $|\Delta|(f):|\Delta^m|\to |\Delta^n|$ mapping vertice $e_i$ to $e_{f(i)}$.
    
    % Now consider a simplicial set $K$, we associate it a topological space 
    % \[
    %     |K|=\coprod_{n=0}^\infty |\Delta^n|\times K([n])/\sim,
    % \]
    % where $\sim$ is the weakest equivalence relation such that $(t,y)\sim (s,x)$ if there exists a morphism $f:[m]\to [n]$ such that
    % \[
    %     y = K(f)x,\quad s=|\Delta|(f)t.
    % \]
    % The topology on $|K|$ is the weakest one for which the factorization by $\sim$ is continuous. 

    The geometric relazation of a simplicial set $K$ is given by the colimit
    \[
        |K| = \varinjlim_{\Delta^n\to K}|\Delta^n|.
    \]
\end{para}

\begin{pro}[HTT 1.1.2.2]
    Suppose $K$ is a simplicial set, then $K\cong N(C)$ for a small category $C$ if and only if for each $0<i<n$ and each morphism $\Lambda_i^n\to K$ can be lifted to an unique morphism $\Delta^n \to K$.
\end{pro}

\begin{para}[Kan complex]
    Let $K$ is a simplicial set, we say that $K$ is a Kan complex if for any $0\leq i\leq n$ any morphism $\Lambda_i^n\to K$ can be lifted to an unique morphism $\Delta^n \to K$.
\end{para}

\begin{para}[$\infty$-category]
    A simplicial set $K$ is an $\infty$-category if for any $0< i< n$ any morphism $\Lambda_i^n\to K$ can be lifted to a morphism $\Delta^n \to K$.
\end{para}

\end{document}
