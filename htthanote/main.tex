% author: buwailee@nmhs
\documentclass[11pt]{article}
\usepackage{../noteheader}

% \usepackage[chapter]{../egastyle}
\theoremstyle{definition}
	\newtheorem{para}{}[section]
		\renewcommand{\thepara}{\thesection.\arabic{para}}
	\newtheorem{exa}[para]{Example}
\theoremstyle{plain}
	\newtheorem{lem}[para]{Lemma}
	\newtheorem{theo}[para]{Theorem}
	\newtheorem{thm}[para]{Theorem}
	\newtheorem{pro}[para]{Proposition}
	\newtheorem{coro}[para]{Corollary}
\renewcommand*{\proofname}{Proof}

\usepackage{titletoc,makeidx,paralist}
	\makeindex
	% \renewcommand{\indexname}{Index}
\usepackage[titletoc,title]{appendix}
	\renewcommand{\appendixname}{Appendix}

\usepackage{hyperref}
	\hypersetup{bookmarksnumbered=true}

\definecolor{shadecolor}{rgb}{0.92,0.92,0.92}

\newcommand{\no}[1]{{$(#1)$}}
% \renewcommand{\not}[1]{#1\!\!\!/}
\newcommand{\rr}{\mathbb{R}}
\newcommand{\zz}{\mathbb{Z}}
\newcommand{\aaa}{\mathfrak{a}}
\newcommand{\pp}{\mathfrak{p}}
\newcommand{\mm}{\mathfrak{m}}
\newcommand{\dd}{\mathrm{d}}
\newcommand{\oo}{\mathcal{O}}
\newcommand{\calf}{\mathcal{F}}
\newcommand{\calg}{\mathcal{G}}
\newcommand{\bbp}{\mathbb{P}}
\newcommand{\bba}{\mathbb{A}}
\newcommand{\osub}{\underset{\mathrm{open}}{\subset}}
\newcommand{\csub}{\underset{\mathrm{closed}}{\subset}}

\DeclareMathOperator{\im}{Im}
\DeclareMathOperator{\Hom}{Hom}
\DeclareMathOperator{\id}{id}
\DeclareMathOperator{\rank}{rank}
\DeclareMathOperator{\tr}{tr}
\DeclareMathOperator{\supp}{supp}
\DeclareMathOperator{\coker}{coker}
\DeclareMathOperator{\codim}{codim}
\DeclareMathOperator{\height}{height}
\DeclareMathOperator{\sign}{sign}

\DeclareMathOperator{\ann}{ann}
\DeclareMathOperator{\Ann}{Ann}
\DeclareMathOperator{\ev}{ev}
	\newcommand{\cc}{\mathbb{C}}
	\DeclareMathOperator{\spec}{Spec}

\begin{document}

\section{One}

\begin{para}[Simplicial set]
    Suppose $\Delta$ is the category of totally ordered finite sets and order-preserving morphisms. Simplicial sets are just set-valued presheaive on $\Delta$, {\it i.e.} functors $\Delta^{\text{op}}\to \mathsf{Set}$.
\end{para}

Usually, we only care about the skeleton $\{[n]\,:\, n\in\mathbb N\}$ of $\Delta$, where $[n]=0<1<\cdots<n$. 

\begin{exa}
    A classical example of simplicial set is $\Delta^n$. It can be seen as the subset 
    \[
        \{(x_0,\dots,x_n)\in [0,1]^{n+1}\,:\, x_0+\cdots+x_n=1\}
    \]
    of $\mathbb R^{n+1}$. Now suppose $A=n_0<n_1<\cdots<n_k$ is a subset of $[n]$, we associate it a face 
    \[
        F_A=\{x\in \Delta^n\,:\, x_i=0\text{ when $i\not\in A$}\}.
    \]
\end{exa}

One important example of simplicial sets is the nerve of a small category.

\begin{para}[Nerve of a small category]
    Every partially ordered set $P$ could be seen as a small category $i(P)$ in a natural way, where morphisms are order relations between objects. We thus obtain a functor $i$ from $\Delta$ to the category of small categories. We can now describe the nerve of the category $C$ as the functor
    \[
        N(C): P \mapsto \operatorname{Fun}(i(P),C),
    \]
    where $\operatorname{Fun}(A,B)$ are set of all functors between small categories $A$ and $B$.
\end{para}

We can describe the nerve on each level $k$ in another way. Let's consider the category of totally ordered finite sets with length $k+1$, denoted by $\Delta_k$, and the similar functor $N(C)_k: P \mapsto \operatorname{Fun}(i(P),C)$. $N(C)_k$ can be described as a set, whose elements are $k$-tuples of composable morphisms
\[
    P_0\to P_1\to P_2\to \cdots \to P_{k-1}\to P_k
\]
of $C$. Elements of $N(C)_k$ are usually called $k$-cycles, especially, $0$-cycles are objects of $C$, $1$-cycles are morphisms of $C$. 

Suppose $\eta:P\to Q$ is a morphism, then the morphism under the functor $N(C)_k$ is given by
\[
    \xymatrix{
        P_0\ar[r]\ar[d]_{\eta_0} & P_1\ar[r]\ar[d]_{\eta_1} & P_2\ar[r]\ar[d]_{\eta_2} & \cdots \ar[r] & P_{k-1}\ar[r]\ar[d]_{\eta_{k-1}} & P_k\ar[d]_{\eta_k}\\
        Q_0\ar[r] & Q_1\ar[r] & Q_2\ar[r] & \cdots \ar[r] & Q_{k-1}\ar[r] & Q_k
    }
\]
Now let's consider the face and degeneracy maps (functors), they can be used to connect different $k$. The face maps 
\[
     d_{i}\colon N(C)_{k}\to N(C)_{k-1}
\]
are given by composition of morphisms at the $i$-th object (or removing the $i$-th object from the sequence, when $i$ is $0$ or $k$). This means that $d_i$ sends the $k$-tuple 
\[P_{0}\to \cdots \to P_{i-1}\xrightarrow{f_{i-1}} P_{i}\xrightarrow{f_{i}} P_{i+1}\to \cdots \to P_{k}\]
to the $(k-1)$-tuple 
\[P_{0}\to \cdots \to P_{i-1}\xrightarrow{f_if_{i-1}} P_{i+1}\to \cdots \to P_{k}.\]
Similarly, the degeneracy maps 
\[s_{i}:N(C)_{k}\to N(C)_{k+1}\] 
are given by inserting an identity morphism at the object $P_i$. 

\begin{pro}[HTT 1.1.2.2]
    Suppose $K$ is a simplicial set, then $K\cong N(C)$ for a small category $C$ if and only if for each $0<i<n$ and each morphism $\Lambda_i^n\to K$ can be lifted to an unique morphism $\Delta^n \to K$.
\end{pro}

\end{document}
