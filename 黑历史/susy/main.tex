\documentclass[11pt]{article}
\usepackage{amsmath,amsfonts,amsthm,paralist,graphicx}
\usepackage[a4paper, top=16mm, text={170mm, 248mm}, includehead, includefoot, hmarginratio=1:1, heightrounded]{geometry}
\usepackage{ccfonts}
\usepackage[T1]{fontenc}

\newcommand{\zz}{\mathbb{Z}}
\DeclareMathOperator{\sgn}{sgn}

\theoremstyle{definition}
	\newtheorem{para}{}[part]
		\renewcommand{\thepara}{\arabic{para}}
	\newtheorem{conj}[para]{Conjecture}
\theoremstyle{plain}
	\newtheorem{lem}[para]{Lemma}
	\newtheorem{thm}[para]{Theorem}
	\newtheorem{pro}[para]{Proposition}

\begin{document}

\section{6.27}

\begin{thm}[Haag-Lapuszanski-Sohnius Theorem]
Sypersymmetry algebra is the only graded lie algebra of symmetries of the
$S$-matrix consistent with relativistic QFT.
\end{thm}

\begin{thm}[Coleman-Mandula Theorem]~
\begin{compactenum}
\item the $S$-matrix is based on a local relativistic QFT in $4$-D.
\item There are only a finite number of different particles associated 
with one-particle states of a given mass.
\item There is a energy gap between the vacumm and one particle ststes.
\end{compactenum}
The most general Lie algebra of symmetries of the $S$-matrix contains
the energy-momentum operator $P_m$, the Lorentz rotation operator
$M_{mn}$, and a finite number of Lorentz scalar operators $B_l$.
\end{thm}

Witten's understanding: The additional space-time symmetries beyond
energy, momentum and angular momentum would, in a relativistic convariant
theory, overconstrain the elastic scattering amplitudes, and allow the 
non-zero scattering amplitudes only for discrete scattering angles. 
The assumption of analyticity therefore rules out such symmetries.

\vspace{2ex}

$\zz_2$-graded algebra: $\{Q,Q'\}=X$, $[X,X']=X''$, $[Q,X']=Q''$

\[
	Q=\sum Q_{\alpha_1,\cdots \alpha_a,\dot\alpha_1,\cdots,\dot\alpha_b}
\]
The spin of $Q_{\alpha_1,\cdots \alpha_a,\dot\alpha_1,\cdots,\dot\alpha_b}$
is $a/2+b/2=(a+b)/2$, where $a+b$ is a odd number.

\vspace{2ex}

Assumptions are
\begin{compactenum}[(1)]
\item The operator $Q$ act in a Hilbert space with positive definite metric.
\item Both $Q$ and its Hermitian conjugate $\bar Q$ belong to the algebra.
\end{compactenum}

The element
\[
	P_{**}=\left\{Q_{\alpha_1,\cdots \alpha_a,\dot\alpha_1,\cdots,\dot\alpha_b},
	\bar Q_{\dot\beta_1,\cdots \dot\beta_a,\beta_1,\cdots,\beta_b}
	\right\}
\]
is in the algebra with the weight $((a+b))/2,(a+b))/2)$, so 
the Coleman-Mandula theorem tells us that $P_{**}\propto P_{m}$, 
so $a+b=1$ and the weight is $(1/2,1/2)$. So
\[
	\{Q_\alpha^L,\bar{Q}_{\dot \alpha M}\}=P_{\alpha\dot\alpha}C^L_M
\]
where $P_{\alpha\dot\alpha}=\sigma_{\alpha\dot\alpha}^mP_m$, $C^L_M$ is 
Hermitian. We could redefine $Q$ and $\bar Q$ (Use the first 
assumption?) such that
\[
	\{Q_\alpha^L,\bar{Q}_{\dot \alpha M}\}=2P_{\alpha\dot\alpha}\delta^L_M,
\]
where $1\leq L,M \leq N$ and $N\geq 1$.

Since $\{Q_\alpha^L,Q_\beta^M\}$ belongs to $(1/2,0)\otimes (1/2,0)
=(0,0)\oplus (1,0)$, so we have
\[
	\{Q_\alpha^L,Q_\beta^M\}=\epsilon_{\alpha\beta}X^{LM}+M_{\alpha\beta}
	Y^{LM},
\]
where $X$ and $Y$ are anti-symmtric. 

\begin{lem}
$[P_m,Q_\alpha^L]=0$.
\end{lem}

\begin{proof}
Since 
\[
	[P_{\alpha\dot\alpha},Q_\gamma^L]=Z^L_M \epsilon_{\alpha\gamma}
	\bar Q_{\dot\alpha}^M.
\]
Jacobi identity gives that
\[
	[P_{\beta\dot\beta},[P_{\alpha\dot\alpha},Q_\gamma^L]]+\cdots=0,
\]
so $Z^L_M=0$ and $[P_m,Q_\alpha]$.
\end{proof}

\[
	\{Q_\alpha^L,Q_\beta^M\}=\epsilon_{\alpha\beta}a^{l,LM}B_l.
\]

Supersymmetry algebra:

\[
	\{Q_\alpha^L,\bar{Q}_{\dot \alpha M}\}=2P_{\alpha\dot\alpha}\delta^L_M,
\]
\[
	[P_m,Q_\alpha^L]=[P_m,\bar Q_{\dot\beta M}]=0,
\]
\[
	\{Q_\alpha^L,Q_\beta^M\}=\epsilon_{\alpha\beta}a^{l,LM}B_l=
	\epsilon_{\alpha\beta}X^{LM},
\]
\[
	\{\bar Q_{\dot\alpha L},\bar Q_{\dot\beta M}\}
	=\epsilon_{\dot\alpha\dot\beta}a^*_{l,LM}B^l=
	\epsilon_{\alpha\beta}X^\dagger_{LM},
\]
\[
	[Q_\alpha^L,B_l]=(S_l)^L_M Q_\alpha^M,
\]
\[
	[B^l, \bar Q_{\dot\alpha L}]=(S^{l*})_L^M \bar Q_{\dot\alpha M},
\]
\[
	[B_l,B_m]=i C_{lm}^k B_k,
\]
\[
	[Q_\alpha^L,M_{mn}]=\frac 12 (\sigma_{mn})_\alpha^\beta Q_{\beta}^L,
\]
\[
	[Q_{\dot\alpha L},M_{mn}]=-\frac 12 \bar Q_{L\dot\beta}(\bar\sigma_{mn})_{\dot \alpha}^{\dot\beta} 
\]

\begin{pro}[Jacobi identity]
\[
	\{A,\{B,C]]\pm \{B,\{C,A]]\pm \{C,\{A,B]]=0.
\]
\end{pro}

For example,
\[
	[B_l,\{Q_\alpha^L,\bar Q_{\dot \beta M}\}]+\{Q_\alpha^L,[\bar Q_{\dot \beta M}\},B_l]\}-\{\bar Q_{\dot \beta M},[B_l,Q_\alpha^L]\}=0.
\]
And it tells us that
\[
	2P_{\alpha\dot\beta}\left((S^{l*})_M^L-(S_l)_M^L\right)=0,
\]
so $(S^{l*})_M^L=(S_l)_M^L$.

\begin{lem}
$\{X^{LM}\}$ form an invariant subalgebra.
\end{lem}

\begin{proof}
Check the Jacobi identity of $B_l$, $Q_\alpha^L$ and $Q_{\beta}^M$, it 
gives that
\[
	\epsilon_{\alpha\beta}\left(
	[B_l,X^{LM}]+(S_l)^M_K X^{LK}-(S_l)^{L}_K X^{MK}
	\right)=0.
\]
The first term is $[B_l,a^{m,LM}B_m]=ia^{m,LM}C_{lm}^k B_k=0$, and so
\[
	(S_l)^M_K X^{LK}-(S_l)^{L}_K X^{MK}=0.
\]

Similarily, check the Jacobi identity of $Q$, $Q$ and $\bar Q$, it 
will be
\[
	\epsilon_{\alpha\beta}[\bar Q_{\dot\gamma K},X^{LM}]=0.
\]
And we have that 
\[
	[X^{KN},X^{LM}]=\frac 12 \epsilon^{\beta\alpha}[\{Q_{\alpha}^K,Q_\beta^N\},X^{LM}]=0,
\]
so
\[
	S_{lK}^Ma^{k,LK}-(S_l)_{K}^La^{a,MK}=0
\]
or
\[
	S_{lK}^Ma^{k,LK}=-(S^{l*})_{K}^La^{a,MK}
\]
\end{proof}

\begin{compactenum}[(1)]
\item $S=0$.
\item $S_l=-S_l^*$.
\item $\{Q_\alpha^L,Q_\beta^M\}=\epsilon_{\alpha\beta}\epsilon^{LM}(C_1Z_1+iC_2Z_2)$.
\end{compactenum}

\vspace{2ex}

Casimir Operators

\begin{compactenum}[(1)]
\item $C_1=P^\mu P_\mu=P^2$, since $[P^2,Q]=0$.

\[
	m_F^2|F\rangle =P^2|F\rangle=P^2Q|B\rangle
	=QP^2|B\rangle=m_B^2Q|B\rangle=m_B^2|F\rangle,
\]
so $m_F=m_B$.
\item $W^2=W^\mu W_\mu$ is not a Casimir operator now because
$[W^2,Q_\alpha]\neq 0$. However, define
\[
	B_\mu=W_\mu +\frac 18 \bar Q_M \gamma_\mu \gamma_5 Q_M
\]
and
\[
	C_{\mu\nu}=B_\mu P_\nu-B_\nu P_\mu,
\]
where 
\[
	Q_M:=\begin{pmatrix}
	Q_\alpha\\\bar Q^{\dot\alpha}
	\end{pmatrix}
\]	
Now we will have $C^2$ is supersymmetry Casimir operator.
\end{compactenum}

$C^2=2m^4J_\mu J^\mu$, where 
\[
	J_\mu=S_\mu-\frac 1{4m}(\bar Q\sigma_\mu Q).
\]
Then
\[
	[J_\mu,J_\nu]=i\epsilon_{\mu\nu\rho}J_\rho.
\]

Representation of the Supersymmetry algebra
\[
	Q_\alpha=\int d^3x J_\alpha^0,
\]
and $\partial_mJ^m_\alpha=0$.

A fermion number operator 
\[
	(-)^{N_F}:=\begin{cases}
	1&\text{bosonic states}\\
	-1&\text{fermionic states}
	\end{cases}
\]
then
\begin{align*}
\operatorname{Tr}\left((-)^{N_F}\{Q_\alpha^A,\bar Q_{\dot \beta B}\}\right)&=\operatorname{Tr}\left((-)^{N_F}Q_\alpha^A\bar Q_{\dot \beta B}+(-)^{N_F}\bar Q_{\dot \beta B}Q_\alpha^A\right)\\
&=\operatorname{Tr}\left(-Q_\alpha^A(-)^{N_F}\bar Q_{\dot \beta B}+(-)^{N_F}\bar Q_{\dot \beta B}Q_\alpha^A\right)\\
&=0.
\end{align*}
Using that 
\[
	\{Q_\alpha^A,\bar Q_{\dot \beta B}\}=2\sigma^m_{\alpha\dot\alpha}P_m \delta^A_B,
\]
we have that
\[
	\operatorname{Tr}\left((-)^{N_F}P_m\right)=0.
\]
Since the spectrum of $P_0$ is all positive, so we also have that $\operatorname{Tr}\left((-)^{N_F}\right)=0$.

For massless particle, let $P_\mu=(E,0,0,E)$ and 
\[
	W_0=L\cdot P,
\]
where $L=(M_{23},M_{31},M_{12})$. One-particle states will be denoted
as $|E,\lambda\rangle$, where $\lambda$ is the helicity of particle. 
Then 
\[
	W_u|E,\lambda\rangle =\lambda P_\mu |E,\lambda\rangle.
\]

dd
\[
	W_0Q_\alpha|E,\lambda\rangle=Q_\alpha W_0|E,\lambda\rangle+[W_0,Q_\alpha]|E,\lambda\rangle=E\left(\lambda I-\frac 12 \sigma^3\right)^{\beta}_\alpha Q_\beta |E,\lambda\rangle,
\]
where we have used that $[W_0,Q]=[L\cdot P,Q]=-\frac 12(\sigma\cdot P) Q$.
So
\[
	\begin{cases}
	Q_{1L}& -1/2\\
	Q_{2L}& +1/2
	\end{cases}
	\quad 
	\begin{cases}
	\bar Q_{\dot 1L}& +1/2\\
	\bar Q_{\dot 2L}& -1/2
	\end{cases}
\]

Since $\{Q_1^L,\bar Q_{\dot 1 M}=4\delta^L_M E\}$, define
\[
	a^A=\frac{1}{2\sqrt E}Q_1^A,\quad a^\dagger_A=
	\frac{1}{2\sqrt E}\bar Q_{\dot 1}^A=(a^A)^\dagger,
\]
for $|E,\lambda_0\rangle$, $a^A|E,\lambda_0\rangle=0$, then
\[
	a^{\dagger A}|E,\lambda_0\rangle=|E,\lambda_0+1/2\rangle
\]
and
\[
a^{\dagger i_1}\cdots a^{\dagger i_k}|E,\lambda_0\rangle=
|E,\lambda_0+k/2\rangle,
\]
so the number of states with helicity $\lambda_0+k/2$ is $\binom{N}{k}$. 
The total number is 
\[
	\sum_{k=0}^N \binom{N}{k}=(1+1)^N=2^N,
\]
and since $(1-1)^N=0$, we also have that
\[
	\sum_{k=0}^{N/2} \binom{N}{2k}-\sum_{k=0}^{N/2} \binom{N}{2k+1}=0.
\]

If the theory has the CPT invariance, we should have $-|\lambda_0|-|\lambda_0|+N/2=0$, or $N/4=|\lambda_0|$. There're two important theory,
\begin{compactenum}[(1)]
\item $N=4$ Yang-Mills theory with $\lambda_0=-1$,
\[
\begin{matrix}
\text{helicity}& -1 & -1/2 & 0 & 1/2 & 1\\
\text{states}& 1 & 4 & 6 & 4 & 1
\end{matrix}
\]
SYM $N=4$, Renormalizability. $4$-D $N=4$ $\Rightarrow$ $10$-D $N=1$.

\item $N=8$ Supergravity theory with $\lambda_0=-2$,
\[
\begin{matrix}
\text{helicity}&-2 & -3/2 & -1 & -1/2 & 0 & 1/2 & 1 & 3/2 & 2\\
\text{states}& 1 & 8 & 28 & 56 & 70 & 56 & 28 & 8 & 1
\end{matrix}
\]
Supergravity $N=8$, Consistent coupling to gravity. 
$4$-D $N=8$ $\Rightarrow$ $11$-D $N=1$.
\end{compactenum}

For massive particle with mass $M$, let $P_m=(-M,0,0,0)$. Let
\[
	a_\alpha^A=\frac{1}{2M}Q_\alpha^A,\quad (a_\alpha^A)^\dag=\frac{1}{2M}\bar Q_{\dot \alpha}^A,
\]
since
\[
	\{Q_\alpha^A,\bar Q_{\dot\beta B}\}=2M \delta_{\alpha\dot \beta}\delta^A_B,
\]
and $\{Q,Q\}=0=\{\bar Q,\bar Q\}$, then
\[
	\{a_\alpha^A,(a_{\beta}^B)^\dagger \}=\delta_{\alpha\beta}\delta^{AB}.
\]

Suppose $|\Omega\rangle$ is the highest-weight state, i.e. $a_\alpha^A|\Omega\rangle =0$ for all $\alpha$ and $A$. Then the states are
\[
	(a^{A_1}_{\alpha_1})^\dagger \cdots (a^{A_n}_{\alpha_n})^\dagger |\Omega\rangle.
\]
The number is $\binom{2N}{n}$,
so the total number is $2^{2N}$. So there're $2^{2N-1}$ fermionic and 
bosonic states.

Here's an example: Spin $\Omega_{1/2}$, $N=1$
\[
	\begin{matrix}
	0& a_{\dot 2}^\dagger\Omega\\
	1/2 & \Omega, a_{\dot 1}^\dagger a_{\dot 2}^\dagger\Omega\\
	1 & a_{\dot 1}^\dagger\Omega
	\end{matrix}
\]

Spin $\Omega_0$, $N=2$, here is the pic!!!!!

***************

$\operatorname{SO}(4N)$
\begin{align*}
	\Gamma^l&=\frac{1}{\sqrt 2}(a_1^l+(a_1^l)^\dagger)\\
	\Gamma^{N+l}&=\frac{1}{\sqrt 2}(a_2^l+(a_2^l)^\dagger)\\
	\Gamma^{2N+l}&=\frac{1}{\sqrt 2}(a_1^l-(a_1^l)^\dagger)\\
	\Gamma^{3N+l}&=\frac{1}{\sqrt 2}(a_2^l-(a_2^l)^\dagger)
\end{align*}
Then $[\Gamma^r,\Gamma^s]=\delta^{rs}$.

$\operatorname{SU}(2)\times \operatorname{USp}(2N)$, the letter one
is R symmetry.
\[
	\begin{cases}
	q_\alpha^l=a_\alpha^l\\
	q_\alpha^{N+l}=\sum_{\beta=1}^2\epsilon_{\alpha\beta}(a_\beta^l)^\dagger
	\end{cases}
\]

\[
	(q_\alpha^l)^\dagger =(a_\alpha^l)^\dagger = \epsilon^{\alpha\beta}q_\beta^{N+l}
\]
\[
	(q_\alpha^{N+l})^\dagger = - \epsilon^{\alpha\beta}a_\beta^l
	= - \epsilon^{\alpha\beta}q_{\beta}^l
\]
\[
	(q_\alpha^r)^\dagger =\epsilon^{\alpha\beta}\Lambda^{rt}q^t_\beta
\]
where 
\[
	\Lambda = \begin{pmatrix}
		0&1\\-1&0
	\end{pmatrix}
\]
\[
	\{q_\alpha^r,q_\beta^t\}=-\epsilon_{\alpha\beta}\Lambda^{rt}.
\]

With cneter charge. 
\[
	\{Q_\alpha^A,\bar Q_{\dot\beta B}\}=2M \delta_{\alpha\dot \beta}\delta^A_B,
\]
\[
	\{Q_\alpha^L,Q_\beta^M\}=\epsilon_{\alpha\beta}Z^{LM}
\]
\[
	\{(Q_\alpha^L)^\dagger,(Q_\beta^M)^\dagger\}=\epsilon^{\alpha\beta}Z^*_{LM}
\]
where $Z^{LM}=-Z^{ML}$.

$\tilde Z^{LM}=U^L_K U^M_N Z^{KN}$, where for even $N$
\[
	\tilde Z=\epsilon \otimes D
\]
for odd $N$
\[
	\tilde Z=\begin{pmatrix}
	\epsilon \otimes D&0\\
	0&0
	\end{pmatrix}
\]
where 
\[
	\epsilon=\begin{pmatrix}
	0&1\\
	-1&0
	\end{pmatrix},\quad
	D=\operatorname{diag}(a,b,c,d)
\]

pic here !!!!!!!

**********

$|Z_n|\leq 2M$

**********

From now on, we only consider $N=1$ SUSY.

Component fields 

Anti-commuting parameters $\xi^\alpha$, $\bar \xi_{\dot \alpha}$,
\[
	\{\xi^\alpha,\xi^\beta\}=\{\xi^\alpha,Q_\alpha=0\}
\]
then
\[
	[\xi Q,\bar \xi \bar Q]=2\xi \bar\sigma^m\bar \xi P_m
\]
\[
	[\xi Q,\xi Q]=[\bar \xi \bar Q,\bar \xi \bar Q]=0
\]
\[
	[P^m,\xi Q]=[P^m,\bar \xi \bar Q]=0.
\]
where 
\[
	\xi Q=\xi^\alpha Q_\alpha,\quad \bar \xi \bar Q=\bar \xi_{\dot \alpha}\bar Q^{\dot \alpha}.
\]

Define
\[
	\delta_\xi A=(\xi Q+\bar \xi \bar Q)A,\quad \delta_\xi \psi
	=(\xi Q+\bar \xi \bar Q)\psi,
\]
then
\begin{align*}
	[\delta_\eta,\delta_\xi]A&=2(\eta\sigma^m \bar \xi -\xi\sigma^m \bar \eta)
	P_m A\\
	&=-2i(\eta\sigma^m \bar \xi-\xi \sigma^m\bar\eta)\partial_m A
\end{align*}
so
\[
	\delta_\xi A=\sqrt 2 \xi \psi,\quad \delta_\xi \psi=i\sqrt 2 \sigma^m \bar \xi \partial_m A+\sqrt{2}\xi F
\]
pic here !!!!!!

******
\end{document}