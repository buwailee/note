\documentclass[11pt]{article}
\usepackage[utf8]{inputenc}
\usepackage{hyperref}
\usepackage{amsmath,amsfonts,amsthm,paralist,graphicx,mathrsfs}
\usepackage[a4paper, top=16mm, text={170mm, 240mm}, includehead, includefoot, hmarginratio=1:1, heightrounded]{geometry}
\usepackage{xypic}

\usepackage{ccfonts}
\usepackage[T1]{fontenc}
\renewcommand{\baselinestretch}{1.1}

\newcommand{\zz}{\mathbb{Z}}
\newcommand{\cc}{\mathbb{C}}
\DeclareMathOperator{\sgn}{sgn}
\DeclareMathOperator{\Arg}{Arg}

\theoremstyle{definition}
	\newtheorem{para}{}[section]
		\renewcommand{\thepara}{\thesection.\arabic{para}}
	\newtheorem{conj}[para]{Conjecture}
\theoremstyle{plain}
	\newtheorem{defi}[para]{Definition}
	\newtheorem{lem}[para]{Lemma}
	\newtheorem{thm}[para]{Theorem}
	\newtheorem{pro}[para]{Proposition}

\title{Note on twisted cycles and $m_{\alpha'}$}

\begin{document}

\maketitle

\section{Twisted Cycles}

Let $\{P_i\,:\,1\leq i \leq k\}$ be polynomials on $\cc^n$ and 
\[
	u(z_1,\dots,z_n)=\prod_i P_i(z_1,\dots,z_n)^{s_i}
\]
be a multi-valued function on $M:=\cc^n-Z(P_1\cdots P_k)$, 
where $s_i\in \cc-\zz$.

\begin{defi}[twisted cycle]
A twisted cycle $C$ associated with multi-valued function $u$ 
on $M$ is
the tensor product of a topological cycle $[C]$ and a 
branch of $u$ constrained on $[C]$, i.e. $C=[C]\otimes u_C$.
\end{defi}

When $[C]=\langle 012\cdots m\rangle$ is a $m$-simplex on M, one can
define a boundary operator $\partial$ by
\[
	\partial \langle 012\cdots m\rangle=\sum_i (-1)^{i-1}
	\langle 012\cdots \hat{i}\cdots m\rangle,
\]
and a homology group $H_{\bullet}(M)$ for the free modules
generated by simplexes (i.e.~chain groups). 
Similarily, one could define a new boundary operator 
$\partial_\omega$ here for twisted cycles by
\[
	\partial_\omega (\langle 012\cdots m\rangle\otimes u_C)
	=\sum_i (-1)^{i-1}\langle 012\cdots \hat{i}\cdots m\rangle\otimes 
	u_C|_{\langle 012\cdots \hat{i}\cdots m\rangle},
\]
and a new homology group $H_{\bullet}(M,\mathcal L_\omega)$ of 
twisted chain groups. The new homology group is called
the homology group with cofficients in a local system 
$\mathcal L_\omega$ defined by the differential equation
\[
	d \xi -\xi \omega=0,
\]
where $\omega=d \log(u)=s_idP_i/P_i$ is a well-defined $1$-from
on $M$. This name comes form the fact that local solutions 
of the above differential equation must have the form that 
$\xi = cu$, where $c\in \cc$ is a complex number.

With there terminologies, we can define a homology group 
$H_m(M,\mathcal L_{-w})$ associated with the multi-valued 
function $1/u$ since the differential equation here should 
be $d \xi +\xi \omega=0$. From now on, the twisted cycles 
we mentioned below will be the homology classs in the homology 
group $H_{\bullet}(M,\mathcal L_\omega)$ or 
$H_{\bullet}(M,\mathcal L_{-\omega})$ or their representatives,
i.e.~the boundary-less twisted cycles.

\begin{defi}[intersection number]
Suppose $C$ and $D$ are two twisted $n$-cycles 
associated with multi-valued function $u$ and $u^{-1}$ 
respectively in $M$. 
If they intersect transversely, then $[C]\cap [D]$ is a
finite set, and at each $x\in [C]\cap [D]$, the topological 
intersection number of $[C]$ and $[D]$ is 
\[
	\operatorname{int}_x([C],[D])=
	\begin{cases}
	1, & \text{if $T_x[C]\oplus T_x[D]$ has the same orientation with $T_xM$},\\
	-1,& \text{else},
	\end{cases}
\]
and the intersection number of these two 
twisted cycles is 
\[
	C\cdot D:=\sum_{x\in [C]\cap [D]}\operatorname{int}_x([C],[D])
	u_C(x)u^{-1}_D(x).
\]
\end{defi}

In the above definition, $u_C(x)u_D^{-1}(x)$ is just the locally constant 
phase factor that differs between two branches of $u$.
If one perturbs cycle 
$[C]$ near $x$, the intersection point may be changed, 
however, the value of $u_Cu_D^{-1}$ at the intersection point will 
not be changed, so it only depends on the homology class of $[C]$, 
and so is $C\cdot D$.

Now, we could generalize the above definition to the case where
$[C]$ and $[D]$ don't intersect transversely. Given two twisted
cycles $C$ and $D$, we can move them within their respective 
homology classes until they intersect transversely, then define
$C\cdot D$ as the intersection number after deformation. Therefore,
we have defined the intersection pairing:
\[
	H_n(M,\mathcal L_{\omega})\times H_n(M,\mathcal L_{-\omega})
	\to \cc.
\]

% [Sebastian Mizera] has proven that the KLT inverse matrix is
% \[
% 	m_{\alpha'}(\alpha|\beta)=\left(-2i\right)^{n-3}C(\alpha)\cdot
% 	C^\wedge(\beta),
% \]
% so the monodromy relations of KLT inverse matrix is immediate.

However, the naive definition of intersection pairing 
sometimes\footnote{For example, one of the cycles is non-compact.} 
performs not as good as we want, so additional
conditions should be considered. A natrual condition is that
one of these two twisted cycles is locally finite. 
For locally finite twisted cycles, we can similarily define
its homology group $H^{\text{lf}}_\bullet(M,\mathcal L_\omega)$,
then we have the intersection pairing 
\[
	H^{\text{lf}}_n(M,\mathcal L_{\omega})\times H_n(M,\mathcal L_{-\omega})
	\to \cc,
\]
\cite{aomoto2011theory,kita1994intersection} has showed that this pairing is non-degenerated. What's more,
there's a natrual isomorphism
\[
	\operatorname{reg}:H^{\text{lf}}_n(M,\mathcal L_{\omega})
	\to H_n(M,\mathcal L_{\omega}),
\]
the image of a twisted cycle under the isomorphism
$\operatorname{reg}^{-1}$ is called its regularization.
We will denote the regularization of $C$ as $C_{\text{reg}}$, and 
define a new intersection pairing
\[
	H_n(M,\mathcal L_{\omega})\times H_n(M,\mathcal L_{-\omega})
	\to \cc
\]
by
\[
	C\bullet D = C_{\text{reg}}\cdot D,
\]
it's non-degenerated now.

% \section{Twisted Cycles on the Moduli Space}
\section{Regularization}

In order to define the intersection pairing of two twisted 
cycles, one should regularize one of these two twisted cycles. 

In one dimension, suppose $[C]=(a,b)$ and $a$, $b$ are branch 
points of $u$, then 
\[
	C_{\text{reg}}:=\left(\frac{S_{\epsilon}(a,\epsilon)}{\exp(2\pi i s_a)-1}+[a+\epsilon,b-\epsilon]-\frac{S_{\epsilon}(b,-\epsilon)}{\exp(2\pi i s_b)-1}\right)\otimes u_C,
\]
where $S_{\epsilon}(x,y)$ is a circle centered at $x$ 
with radius $\epsilon$ and starting point $x+y$. 
It can be shown as a diagram 
\begin{center}
\includegraphics[width=0.5\textwidth]{fig_0}
\end{center}
where the arrows represent the orientation of topological cycles.
It's direct to cheak that $\partial C_{\text{reg}}=0$, so it's in
the $H^{\text{lf}}_\bullet(M,\mathcal L_{\omega})$.
The higher dimensional generalization is defined in \cite{aomoto2011theory}, we will
give a very short definition below.

Suppose $u(z_1,\dots,z_n)=\prod_i P_i(z_1,\dots,z_n)^{s_i}$, $L$ is a codimension-$k$ face of $[C]$, $p$ is a point on $L$ which is not in the
$\epsilon$-neighborhood of a face of $[C]$ with higher codimension.
Face $L$ must be contained in $Z(P_{i_1})\cap \cdots \cap Z(P_{i_k})$.
Locally, it's always possible to choose a chart $\{w_1,\cdots,w_n\}$
with the same orientation with $M$
near $p$ such that $Z(P_{i_j})$ is given by $w_j=0$, and
in $[C]$, $w_i>0$ for $1\leq i\leq k$, then
we chould define a cycle
\[
	[C_p]=\frac{S_\epsilon(p,\epsilon)}{\exp(2\pi i s_{i_1})-1}\times \cdots \times
	\frac{S_\epsilon(p,\epsilon)}{\exp(2\pi i s_{i_k})-1}\times (-\epsilon,\epsilon)\times \cdots
	\times (-\epsilon,\epsilon)
\]
and a twisted cycle
\[
	C_p=C=[C_p]\otimes u_{C_p}.
\]
Finally, we can glue $C_p$ along $[C]$ and then get a twisted cycle
$C_{\text{reg}}$. The proof of the boundary-less of $C_{\text{reg}}$
can be found in \cite{aomoto2011theory}.

The addtion of two regularized twisted cycle depends on the choice
of branch of $u$ on each cycle, when the branches are 'compatible',
the result may be very simple. We will show it in dimension one, the
higher dimensional generalization is similar.

For example, suppose $C_1$ and $C_2$ are two twisted cycle 
\[
	C_1=(-\infty,a-\epsilon]\otimes u_1-
	\frac{1}{\exp(2\pi is)-1}S_{\epsilon}(a,-\epsilon)\otimes u_1
\]
and 
\[
	C_2=\frac{1}{\exp(2\pi is)-1}S_{\epsilon}(a,\epsilon)\otimes u_2+[a+\epsilon,\infty) \otimes u_2,
\]
where $u_1$, $u_2$ are the branches of $u(z)=(a-z)^{s}$, 
\begin{center}
\includegraphics[width=0.7\textwidth]{fig_1}
\end{center}

In the above diagram, we have chosen that
\[
	u_1(z)=|a-z|^s\exp(is \arg(a-z))
\]
and 
\[
	u_2(z)=|a-z|^s\exp(-2\pi is+is \arg(a-z))
\]
on $C_1$ and $C_2$ respectively.
% , where the orange 
% numbers are the values of $\Arg(z-a)$ 
% on each branch.
It may be useful to note that
\begin{equation}
	\begin{cases}
	u_1(x)=|a-x|^s, & x\in  (-\infty,a-\epsilon];\\
	u_2(x)=|x-a|^s \exp(-\pi i s),& x\in [a+\epsilon,\infty).
	\end{cases}
\end{equation}

\begin{lem}\label{lem:1}
The sum of the above two twisted cycles is
\[
	C=C_1+C_2=(-\infty,x-\epsilon]\otimes u_1+
	S^+_{\epsilon}(a,-\epsilon)\otimes u_2+
	[x+\epsilon,\infty) \otimes u_2,
\]
where $S^+_\epsilon(a-\epsilon)$ is the upper semi-circle 
centered at $a$ with radius $\epsilon$ and starting point $a-\epsilon$.
\begin{center}
\includegraphics[width=0.5\textwidth]{fig_2}
\end{center}
What's more, $u$ is continuous on $C$.
\end{lem}

\begin{proof}
It's enough to calculate
\[
	D=-\frac{1}{\exp(2\pi is)-1}S_{\epsilon}(a,-\epsilon)\otimes u_1+\frac{1}{\exp(2\pi is)-1}S_{\epsilon}(a,\epsilon)\otimes u_2.
\]
According to the choise of branches, the semi-circles under the axis cancel, so
\begin{align*}
D&=-\frac{\exp(2\pi i s)}{\exp(2\pi is)-1}(-S^+_{\epsilon}(a,-\epsilon))\otimes u_2+\frac{1}{\exp(2\pi is)-1}(-S^+_{\epsilon}(a,-\epsilon))\otimes u_2\\
&=S^+_{\epsilon}(a,-\epsilon)\otimes u_2,
\end{align*}
where the addtional sign comes form the orientation of $S^+_{\epsilon}(a)$.
\end{proof}

% \section{Backgroud and Notations}

% Not important now, I will omit it.

% Notations and definitions: 
% \begin{center}
% \begin{tabular}{cc} 
%  $\alpha$, $\beta$, $\dots$ & permutation \\ 
%  $\Delta(\alpha)$ & $\overline{\{z_{\alpha(1)}<\cdots< z_{\alpha(n)}\}}$ \\ 
%  $(ij)$&  hyperplain $\{z_i-z_j=0\}$
% \end{tabular}
% \end{center}

\section{Monodromy relations}

In the following disscusion of KLT inverse matrix, the space 
$M$ will be the moduli space $\mathcal M_{0,n}$, the multi-valued
function $u$ is the Koba-Nielsen factor
\[
	u(z)=\prod_{i=2}^{n-2}(-z_i)^{\alpha' s_{1i}}\prod_{2\leq i<j\leq n-1}(z_i-z_j)^{\alpha' s_{ij}}\prod_{i=2}^{n-2}(z_i-1)^{\alpha' s_{i,n-1}},
\]
which comes form $\prod_{1\leq i<j \leq n}(z_i-z_j)^{\alpha' s_{ij}}$ after
fixing $z_1=0$, $z_{n-1}=1$, $z_n=\infty$. Sometimes, another
choice of fixing may be useful. For example, when considering
the monodromy relations of $\alpha=(q_1\cdots q_n)$, it's natrual
to fix three $q$s other than the running $q_1$.

The twisted cylces used here are
\[
	C(\alpha)=\Delta^o(\alpha)\otimes u_\alpha,
\]
where $\Delta^o(\alpha)$ is the interior of 
\[
	\Delta(\alpha)=\overline{\{z_{\alpha(1)}<\cdots< z_{\alpha(n)}\}}
\]
and $u_\alpha$ is a chosen branch of $u$ on $\Delta^o(\alpha)$. 

A facet (i.e.~codimention one face) of $C(\alpha)$ is on a hyperplain 
$(ij):=\{z_i-z_j=0\}\cap \{z_i=z_i^*\} \cap \{z_j=z_j^*\}
\subset \mathbb R^n$, 
then the standard orientation of $\mathbb R^n$ gives a normal 
vector of $C(\alpha)|_{(ij)}$ in $\mathbb R^n$, which is denoted by $n_{(ij)}$. 

Suppose $\Delta(\alpha)$ and $\Delta(\beta)$ share the same 
facet $(ij)$, 
% and $\Delta(\alpha)$ is behind $\Delta(\beta)$  
% along the $n_{(ij)}$. Therefore, 
as showed in the last section, if we choose 
$u_\beta=\exp(-2\pi i s_{ij})u_\alpha$, the sum of two 
regularized twisted cycles $C_{\text{reg}}(\alpha)$ and 
$C_{\text{reg}}(\beta)$ will be very simple. 

\begin{pro}[monodromy relation]
For cycles
\[
	\{\Delta(q_1q_2\cdots q_n),\Delta(q_2q_1\cdots q_n),\cdots,
	\Delta(q_2\cdots q_1q_n)\},
\]
we can choose that
\[
	u_{q_2\cdots q_iq_1q_{i+1}\cdots q_n}=\exp(\pm 2\pi \alpha' i (s_{12}+\cdots+s_{1i}))u_{q_1q_2\cdots q_n}
\]
such that
\[
	C_{\mathrm{reg}}(q_1q_2\cdots q_n)+C_{\mathrm{reg}}(q_2q_1\cdots q_n)
	+\cdots+C_{\mathrm{reg}}(q_2q_3\cdots q_1 q_n)=0 \in H_{n-3}(M,\mathcal L_w),
\]
where the sign in $\exp$ depends on the orientation.
\end{pro}

\begin{proof}[Proof]
Cycles
\[
	\{\Delta(q_1q_2\cdots q_n),\Delta(q_2q_1\cdots q_n),\cdots,
	\Delta(q_2\cdots q_1q_n)\},
\]
are on the same side of the hyperplane $(ij)$ for all 
$j>i\geq 2$, and the adjacent two cycles
\[
	\Delta(q_2\cdots q_{i-1}q_1 q_i\cdots q_n)
\quad \text{and}\quad 
	\Delta(q_2\cdots q_{i}q_1 q_{i+1}\cdots q_n)
\]
share the same facet on the hyperplain $(1i)$, and 
$n_{1i}=n_{12}$ for all $2\leq i\leq n-1$.
So the choice of branches in the statement of the propostion is 
possible (see the following diagram), then by Lemma \ref{lem:1}, we
have that
\begin{center}
\includegraphics[width=0.7\textwidth]{fig_3}
\end{center}
where we have used the conversation of momentum that
\[
	u_{q_2\cdots q_1q_n}=\exp(\pm 2\pi i\alpha' (s_{12}+\cdots+s_{1{n-1}}))u_{q_1q_2\cdots q_n}=\exp(\mp2\pi i\alpha' s_{1n})u_{q_1q_2\cdots q_n}
\]
or $u_{q_1q_2\cdots q_n}=\exp(\pm 2\pi i\alpha' s_{1n})u_{q_2\cdots q_1q_n}$.
\end{proof}

Another useful choice of branch of $u$ on $\Delta^o(\alpha)$ is
\textit{the standard loading}:
\[
	\mathsf{SL}_\alpha(u)=|u|=\prod_{1\leq i<j \leq n} (z_{\alpha(i)}-z_{\alpha(j)})^{\alpha' s_{\alpha(i)\alpha(j)}}
\]
or equivalently
\[
	\arg(\log \mathsf{SL}_\alpha(u)(z))=0 \quad \text{when $z\in \Delta^o(\alpha)$}.
\]
If we set $u_{q_1\cdots q_n}=\mathsf{SL}_{q_1\cdots q_n}(u)$ and
$C^{\mathsf{SL}}(\alpha)=\Delta^o(\alpha)\otimes \mathsf{SL}_\alpha(u)$, then the branches chosen in the last propostion are
\[
	u_{q_2q_3\cdots q_i q_1 q_{i+1}\cdots q_n}=\exp(\pm i\pi\alpha' (s_{q_1q_2}+\cdots+s_{q_1q_{i-1}}))
	\mathsf{SL}_{q_2q_3\cdots q_i q_1 q_{i+1}\cdots q_n}(u),
\]
and the above proposition is equivalent to 
\[
	C_{\text{reg}}^{\mathsf{SL}}(q_1\cdots q_n)+\sum_{i=2}^{n-1}\exp(\pm i\pi\alpha' (s_{q_1q_2}+\cdots+s_{q_1q_{i-1}}))
	C_{\text{reg}}^{\mathsf{SL}}(q_2q_3\cdots q_i q_1 q_{i+1}\cdots q_n)=0.
\]

As a corollary, since integral is bilinear, 
\[
	\int_{C^{\mathsf{SL}}_{\text{reg}}(q_1q_2\cdots q_n)}\frac{\mathsf{PT}(\alpha)}{\operatorname{SL}(2,\mathbb C)}+\sum_{i=2}^{n-1}\exp(\pm i\pi\alpha' (s_{q_1q_2}+\cdots+s_{q_1q_{i-1}}))\int_{C^{\mathsf{SL}}_{\text{reg}}(q_2q_3\cdots q_i q_1 q_{i+1}\cdots q_n)}\frac{\mathsf{PT}(\alpha)}{\operatorname{SL}(2,\mathbb C)}
	=0
\]
is valid for all permutation $\alpha$. In old story, 
it is just the monodromy relation of $Z$-integral that
\[
	Z(q_1q_2\cdots q_n|\alpha)+e^{i\pi\alpha' s_{q_1q_2}}Z(q_2q_1\cdots q_n|\alpha)+\cdots+e^{i\pi\alpha' (s_{q_1q_2}+\cdots+s_{q_1q_{n-1}})}Z(q_2q_3\cdots q_1 q_n|\alpha)=0.
\]

% If the blowing-up $\widetilde{\mathcal M}_{n,0}$ is orientable, it may be 
% possible to choose branches such that for adjacent associahedrons, their 
% sum behaves like the one in Lemma \ref{lem:1}.

One may want to choose a `compatible' set of branches of $u$ 
such that for every permutation $\alpha=(q_1q_2\cdots q_n)$, 
\[
	C_{\mathrm{reg}}(q_1q_2\cdots q_n)+C_{\mathrm{reg}}(q_2q_1\cdots q_n)
	+\cdots+C_{\mathrm{reg}}(q_2q_3\cdots q_1 q_n)=0.
\]
% However, because of the 
% non-trivial topology on the Moduli space $\mathcal M_{n,0}$ and its
% blowing-up $\widetilde{\mathcal M}_{n,0}$, it's impossible to choose
% a compatible set of branches on the whole space. 
However, it's impossible.

\begin{pro}\label{pro:1}
If the monodromy relations
\[
	C_{\mathrm{reg}}(q_1q_2\cdots q_n)+C_{\mathrm{reg}}(q_2q_1\cdots q_n)
	+\cdots+C_{\mathrm{reg}}(q_2q_3\cdots q_1 q_n)=0
\]
are valid for every permutation $(q_1q_2\cdots q_n)$, it is impossible to write down a matrix of phase factor
$A(\alpha)$ such that
\[ 
	C_{\mathrm{reg}}(\alpha) = A(\alpha)C_{\mathrm{reg}}^{\mathsf{SL}}(\alpha)
\]
for all $\alpha \in S_n/\zz_n$. So there doesn't exist a `compatible' set 
of branches on the whole space.
\end{pro}

Counting the dimension may be very useful here, but we will give
a direct calculation.

% For a fixed $q$, it's convenient to absorb a global phase factor in the choice
% of all branches, so we will assume that 
% \[
% 	A(\hat q_1)=1.
% \]
% As said in the last section, 
% \[
% 	A(\hat q_i)
% 	=\exp\left(\pm i\pi\alpha' (s_{q_1q_2}+\cdots+s_{q_1q_{i}})\right).
% \]
% Then the usual monodromy relations are 
% \begin{align*}
% 	C_{\text{reg}}^{\mathsf{SL}}(\hat q_1)
% 	+e^{\pm i\pi\alpha' s_{q_1q_2}}C_{\text{reg}}^{\mathsf{SL}}(\hat q_2)
% 	+\cdots+e^{\pm i\pi\alpha' (s_{q_1q_2}+\cdots+s_{q_1q_{n-1}})}
% 	C_{\text{reg}}^{\mathsf{SL}}(\hat q_{n-1})=0,
% \end{align*}

\begin{proof}
After setting $A(q_1q_2\cdots q_n)=1$ and using the monodromy relations, 
$A(q_2q_3\cdots q_iq_1q_{i+1}\cdots q_n)$ has the form that
\[
	A(q_2q_3\cdots q_iq_1q_{i+1}\cdots q_n)
	=\exp(i\pi\alpha'S(q_2q_3\cdots q_iq_1q_{i+1}\cdots q_n)),
\]
where $S(q_2q_3\cdots q_iq_1q_{i+1}\cdots q_n)$ is an element of 
the free $\zz$-module $\bigoplus_{1\leq i<j\leq n}\zz\langle s_{ij}\rangle$.

Suppose that $A(123\cdots)=1$ and the usual monodromy relation is 
\[
	C_{\text{reg}}^{\mathsf{SL}}(123\cdots)+e^{i\pi\alpha's_{12}}C_{\text{reg}}^{\mathsf{SL}}(213\cdots)
	+e^{i\pi\alpha'(s_{12}+s_{13})}C_{\text{reg}}^{\mathsf{SL}}(231\cdots)+\cdots =0.
\]
Thus we chould give that
\[
	\begin{matrix}
		S(123\cdots)& S(213\cdots) & S(231\cdots) & \cdots \\
		0&s_{12}&s_{12}+s_{13}&\cdots
	\end{matrix}
\]
For $\beta=(213\cdots )$, the monodromy relation must be
\[
	e^{i\pi\alpha's_{12}}C_{\text{reg}}^{\mathsf{SL}}(213\cdots)+C_{\text{reg}}^{\mathsf{SL}}(123\cdots)
	+e^{i\pi\alpha'(-s_{23})}C_{\text{reg}}^{\mathsf{SL}}(231\cdots )+\cdots =0,
\]
so
\[
	\begin{matrix}
		S(213\cdots )& S(123\cdots ) & S(132\cdots ) & \cdots\\
		s_{12}&0&-s_{23}&\cdots
	\end{matrix}
\]

Similarily, we chould give that
\[
	\begin{matrix}
		S(321\cdots )& S(231\cdots ) & S(213\cdots ) & \cdots \\
		s_{12}+s_{13}+s_{23}&s_{12}+s_{13}&s_{12}&\cdots \\
		S(132\cdots )& S(312\cdots ) & S(321\cdots ) & \cdots \\
		-s_{23}&???&s_{12}+s_{13}+s_{23}&\cdots 
	\end{matrix}
\]
where it's impossible to determine $S(312\cdots )$. 
\end{proof}

However, it may be possible to choose branches that
\[
	C(1,\alpha(2),\cdots ,\alpha(n-2),n-1,n)
	=\exp(-i\pi \alpha' s_{\alpha})C^{\mathsf{SL}}(1,\alpha(2),\cdots ,\alpha(n-2),n-1,n),
\]
where 
\[
	s_{\alpha}=\sum_{i<j,\alpha(j)<\alpha(i)}s_{ij}.
\]
This choice may be useful to calculate the determinant or the inverse of
the intersection matrix.


% However, it's possible to choose branches to make some 
% monodromy relations simple. For example, in the proof
% of last proposition, we chould write down the following relations:
% \[
% 	C_{\text{reg}}(12345)+C_{\text{reg}}(21345)+C_{\text{reg}}(23145)+C_{\text{reg}}(23415)=0,
% \]
% \[
% 	C_{\text{reg}}(21345)+C_{\text{reg}}(12345)+C_{\text{reg}}(13245)+C_{\text{reg}}(13425)=0,
% \]
% \[
% 	C_{\text{reg}}(32145)+C_{\text{reg}}(23145)+C_{\text{reg}}(21345)+C_{\text{reg}}(21435)=0,
% \]
% \[
% 	e^{2\pi i\alpha' s_{23}}C_{\text{reg}}(13245)+C_{\text{reg}}(31245)+C_{\text{reg}}(32145)+C_{\text{reg}}(32415)=0,
% \]
% \[
% 	C_{\text{reg}}(23145)+C_{\text{reg}}(32145)+e^{2\pi i\alpha' s_{12}}C_{\text{reg}}(31245)+e^{2\pi i\alpha' s_{12}}C_{\text{reg}}(31425)=0
% \]
% \[
% 	e^{2\pi i\alpha' (s_{23}+s_{24})}C_{\text{reg}}(13425)+C_{\text{reg}}(31425)+C_{\text{reg}}(34125)+C_{\text{reg}}(34215)=0
% \]
% and so on, where
% \[
% 	\begin{matrix}
% 		S(12345)& S(21345) & S(23145) & S(23415)\\
% 		0&s_{12}&s_{12}+s_{13}&s_{12}+s_{13}+s_{14}\\
% 		S(21345)& S(12345) & S(13245) & S(13425)\\
% 		s_{12}&0&-s_{23}&-s_{23}-s_{24}\\
% 		S(32145)& S(23145) & S(21345) & S(21435)\\
% 		s_{12}+s_{13}+s_{23}&s_{12}+s_{13}&s_{12}&s_{13}-s_{34}\\
% 		S(13245)& S(31245) & S(32145) & S(32415)\\
% 		-s_{23}&s_{23}+s_{13}&s_{12}+s_{13}+s_{23}&s_{12}+s_{13}+s_{23}+s_{14}\\
% 		S(23145)&S(32145)&S(31245)&S(31425)\\
% 		s_{12}+s_{13}&s_{12}+s_{13}+s_{23}&s_{13}+s_{23}&s_{13}+s_{23}+s_{24}\\
% 		S(13425)&S(31425)&S(34125)&S(34215)\\
% 		-s_{23}-s_{24}&s_{13}+s_{23}+s_{24}&s_{13}+s_{14}+s_{23}+s_{24}&
% 		s_{13}+s_{14}+s_{12}+s_{23}+s_{24}
% 	\end{matrix}
% \]
% \section{Another choice of branches}

% Define
% \[
% 	\Omega(\alpha)=\bigl\{(i,j)\,:\, i<j,\quad \alpha(j)<\alpha(i)\bigr\}
% \]
% and 
% \[
% 	S(\alpha)=\sum_{(i,j)\in \Omega(\alpha)}s_{ij},
% \]
% and set
% \[
% 	u_{\alpha}=\exp(-2\pi i S(\alpha))\mathsf{SL}_I(u),
% \]
% where $I$ is the identity permutation. It's easy to see that
% \[
% 	u_\alpha=\exp(-\pi i S(\alpha))\mathsf{SL}_\alpha(u).
% \]

\section{Calculation of Intersection Numbers}

In the last section, we have introduced two choice of branches.
The `compatible' one in Lemma \ref{lem:1} and the proof of monodromy
relations is easy to calculate, however, it's difficult to choose 
it globally, see Proposition \ref{pro:1}. The second one is 
the standard loading, it's well-defined globally but more difficult
to calculate. We will take the standard loading here.

Define that
\[
	\langle \alpha,\beta\rangle= 
	C_{\text{reg}}^{\mathsf{SL}}(\alpha)\cdot C^{\mathsf{SL}}(\beta).
\]
From definition, it is linear for the first variable and 
anti-linear for the second, and it's symmetric:
\[
	\langle \alpha,\beta\rangle=\langle \beta,\alpha\rangle.
\]

The monodromy relations of intersection number are
\[
	\langle q_1\cdots q_n,\alpha\rangle +
	\sum_{i=2}^{n-1}\exp(-i\pi\alpha' 
	(s_{q_1q_2}+\cdots+s_{q_1q_{i-1}}))
	\langle q_2q_3\cdots q_i q_1 q_{i+1}\cdots q_n,\alpha\rangle =0.
\]

In order to calculate $\langle \alpha,\beta \rangle$, one need to
deform $C^{\mathsf{SL}}(\beta)$. \cite{dijkgraaf1988c} has proven that
there is a vector field $v(\operatorname{Re} z)$ on $\Delta^o(\beta)$ 
which has transversal zero\footnote{
It's equivalent to say that, if $v(z_0)=0$, then $v'(z_0)\neq 0$.}
exactly at the barycenter of the faces of $\Delta^o(\beta)$. Then
\[
	z\mapsto z+iv(\operatorname{Re} z)
\]
gives a deformation in the same homology class. 

For example,
\[
	\langle \alpha,\alpha \rangle=(-1)^{n-1}\sum_{\text{barycenter of the faces $H$}}
	\frac{1}{d_H},
\]
where $H\subset \bigcap_i H_{i}$ and then
\[
	d_H=\prod_i \left(e^{2\pi i \alpha' s_{H_i}}-1\right) \quad
	\text{and} \quad d_{\Delta^o(\alpha)}=1.
\]
If $K$ is the shared face with the highest codimension $k$
of $\alpha$ and $\beta$, and they are on the different 
side of $K$. Suppose $K=H_1\cap \cdots \cap H_k$, then
\[
	\langle \alpha,\beta\rangle
	=\prod_{i=1}^k \frac{\exp(\pi i\alpha' s_{H_i})}{\exp(2\pi i \alpha' s_{H_i})-1}K_{\text{reg}}\cdot K
	=\left(\frac{1}{2i}\right)^k\prod_{i=1}^k\frac{1}{\sin(\pi \alpha' s_{H_i})}K_{\text{reg}}\cdot K,
\]
where
\[
	K_{\text{reg}}\cdot K=\sum_{\text{barycenter of the faces $H$ of $K$}}
	\frac{1}{d_H}.
\]
The first calculation of $\langle \alpha,\alpha\rangle$ 
can be seen in \cite{mimachi2003intersection}, and then
\cite{mizera2017combinatorics} proven that the KLT inverse matrix is
\[
	m_{\alpha'}(\alpha|\beta)=\left(-2i\right)^{n-3}\langle
	\alpha,\beta\rangle,
\]
so the monodromy relations of KLT inverse matrix is immediate:
\[
	m_{\alpha'}(q_1\cdots q_n|\alpha)+
	\sum_{i=2}^{n-1}\exp(-i\pi\alpha' 
	(s_{q_1q_2}+\cdots+s_{q_1q_{i-1}}))
	m_{\alpha'}(q_2q_3\cdots q_i q_1 q_{i+1}\cdots q_n|\alpha)=0.
\]

\begin{pro}
$m_{\alpha'}(\alpha|\beta)=m_{\alpha'}(\beta|\alpha)\in \mathbb R$.
\end{pro}

\begin{proof}
It's direct from the construction. Since the standard loadings 
and topological intersection numbers are always real, 
so the imaginary part only comes from the different
choise of branches and the complex cofficients of regularized
twisted cycles. For example, if $\alpha$ and $\beta$ are
on the different side of the hyperplane $H$, then this will give
a factor 
\[
	\frac{\exp(\pi i\alpha' s_H)}{\exp(2\pi i \alpha' s_H)-1}
	=\frac{1}{2i}\frac{1}{\sin(\pi \alpha' s_H)}.
\]
If $K$ is the shared face with the highest codimension $k$
of $\alpha$ and $\beta$, it will give a factor
\[
	K_{\text{reg}}\cdot K = (-1)^{k}
	\sum_{\text{barycenter of the faces $H$ of $K$}}
	\frac{1}{d_H}=\left(-\frac{1}{2i}\right)^k \left(
	1+\frac{1}{\tan(\pi \alpha' s_{H_1})\cdots}+\cdots\right),
\]
then
\[
	\langle \alpha,\beta\rangle=(-1)^{n-3}\left(\frac{1}{2i}\right)^{n-3-k}\left(\frac{1}{2i}\right)^k f(s)=\left(-\frac{1}{2i}\right)^{n-3}f(s),
\]
where $f(s)$ is a real function of $s$.
\end{proof}

\section{KLT relation}

Kawai-Lewellen-Tye (KLT) relations, in our language, are just the 
twisted Riemann period relations \cite{cho1995intersection}.

In the following diagram, every $\leftrightarrow$ represent a morphism
defined by the non-degenerated pairing:
\[
\begin{xy}
\xymatrix{
	H_{n-3}(M,\mathcal L_{\omega})\ar@{<->}[rr]^a\ar@{<->}[d]_b&& H^{n-3}(M,\nabla_{\omega})
	\ar@{<->}[d]^d\\
	H_{n-3}(M,\mathcal L_{-\omega})\ar@{<->}[rr]^c&& H^{n-3}(M,\nabla_{-\omega})
}
\end{xy}
\]
where $a$ and $c$ is defined by the integral, 
and $b$ is defined by the intersection pairing of twisted cycles, 
and $d$ is defined by the intersection pairing of cocyles defined
in \cite{cho1995intersection}. Here, it is defined by
\[
	d_{\alpha,\beta}=\int_M \mathsf{PT}(\alpha)\wedge \mathsf{PT}^\vee(\beta).
\]

****

We have known that $C_{\text{reg}}(\alpha)$ forms a  $H_{n-3}()$, 
\[
	Z_{\alpha}(\beta)=\int_{C(\alpha)}\mathsf{PT}(\beta)=
	\int_{C_{\text{reg}}(\alpha)}\mathsf{PT}(\beta)
\]

****

The Poincar\'e duality gives that
\[
	b=\theta^T a,\quad d=c\theta,
\]
and in our case $a=c$, so the twisted Riemann's period relation is
\[
	d=a^T(b^{-1})^Ta.
\]
This is the KLT realtion:
\[
	J(\delta|\epsilon)=\sum_{\beta,\gamma}Z_{\beta}(\delta)
	m^{-1}_{\alpha'}(\beta|\gamma)Z_\gamma(\epsilon).
\]

\section{Embedding Polytope in the Moduli space}

As we all know, the blowing-up Moduli space 
$\widetilde{\mathcal M}_{n,0}$
is divided into a lot of associahedrons labled by elements of 
$S_n/\zz_n$. 

As a generalization, [1708.08701] has showed that Cayley polytopes
can be decomposed into associahedrons, so it's natural to consider 
Cayley polytopes in the moduli space, write down their
accociated twisted cycles and calculate their intersection
pairings. 

\begin{lem}
Suppose $\{f_i\,:\,1\leq i\leq n\}$ are hyperplanes in $M$, 
$u=f_1^{\alpha_1}\cdots f_n^{\alpha_n}$ is a multi-valued 
function on $M$. If a polytope $[\Delta]$ is cutted into 
a lot of cycles $[\Delta_i]$ by hyperplanes $\{f_i\}$, and 
there exists a `compatible' choice of the branches of $u$ on 
each $[\Delta_i]$, then the sum of twisted cycles 
$\Delta_i=[\Delta_i]\otimes u_i$ is also a twisted cycle, and
its self-intersection number is
\[
	\sum_{i,j}(\Delta_i)_{\mathrm{reg}}\cdot \Delta_j.
\]
\end{lem}

\begin{proof}
By Lemma \ref{lem:1}, $u$ is continuous on $[\sum_i\Delta_i]$, 
so it's obvious.
\end{proof}

Then $\Delta=\sum_i \Delta_i$ defines the embedding of 
polytope in the Moduli space. The other constructions are
similar here. However, it's not clear whether the intersection 
number of two Cayley polytope is a amplitude. In fact,
it's hard to say that the KLT inverse in string theory
is a kind of amplitude, although it will be the
amplitude of bi-adjoint $\phi^3$-theory in the field
theory limit.

\section{Amplitude as the Intersection of Currents}

Poincar\'e duality will give us a very different but
equivalent definition of intersection number between two
twisted cycles. We will not
go into details here but only talk about definitions.
For twisted cycle $C\in H_n(M,\mathcal L_\omega)$ on the 
dimension $2n$ space $M$, 
the Poincar\'e duality $\theta_{\omega}$ will give a 
current $\theta_{\omega}(C)$. 
A current, roughly speaking, is a 
regular form multiplied with Dirac delta functions. 
Similarily, for $D\in H_n(M,\mathcal L_{-\omega})$, there's
a distribution valued $n$-form $\theta_{-\omega}(D)$. Then,
the intersection number $C\cdot D$ is given by
\[
	C\cdot D=\int_{M}\theta_{\omega}(C)\wedge \theta_{-\omega}(D).
\]
One may notice that the form of this formula is very similar with the 
Cachazo-He-Yuan (CHY) formula giving the tree amplitudes in
field theorys,
\[
	\mathscr M_n^{\text{tree}}=\int_{\mathcal M_{0,n}}
	\mu_\sigma \prod_a \delta(E_a)I(\sigma),
\]
where $\mathcal M_{0,n}$ is the moduli space, $\mu_\sigma$ is the 
invariant measure on $M$, $\{E_a\}$ are scattering equations and 
$I$ depends on the specific field theory.

% Denote the linear space of $p$-forms on $M$ by $\Omega^p(M)$.
\begin{defi}
Suppose $M$ is a $n$-dimension manifold. A current of degree $p$ 
is a linear function on $\Omega^{n-p}(M)$.
\end{defi}

Here are two examples of current. Suppose $C$ is a $(n-p)$-cycle on $M$, 
then
\[
	T_C :\varphi \in \Omega^{n-p}(M)\mapsto \int_C i^*_C(\varphi)
\]
is a $p$-current, well $i^*_C$ is the pullback of $i_C:C\hookrightarrow M$.
Suppose $\psi$ is a $p$-form on $M$, then
\begin{equation}
T_\psi:\varphi\in \Omega^{n-p}(M)\mapsto \int_M\psi\wedge\varphi.
\end{equation}

If $T$ is a $p$-current and $\varphi$ is a $q$-form, then
for any form $\psi$,
\[
	T\wedge \varphi(\psi)=T(\varphi\wedge \psi)
\]
will define a new $(p+q)$-current $T\wedge \varphi$. Locally, it's always possible to write a $p$-current $T$ as
\[
	T=\sum_{|I|=p} T_Idx^I,
\]
where $\{T_I\}$ are $0$-currents, and for any $(n-p)$-form $\psi$,
\[
	T(\psi)=\sum_{|I|=p} T_Idx^I(\psi)=\sum_{|I|=p} T_I(dx^I\wedge\psi).
\]

Now suppoer $T$ and $S$ are currents, they can be 
locally written as
\[
	T=\sum_{|I|=p} T_Idx^I,\quad S=\sum_{|J|=q} S_Jdx^J,
\]
then we chould define $T\wedge S$ locally by
\[
	T\wedge S=\sum_{|I|=p,|J|=q}T_IS_J dx^I\wedge dx^J.
\]
It's also well-defined globally. Let us call it \textit{intersection of
currents $T$ and $S$}. This name comes from the following example.

Suppose $C$ is a $p$-cycle, $D$ is a $(n-p)$-cycle, and $T_C$ and $T_D$
are associated currents for $C$ and $D$, then 
$T_C\wedge T_D(1)$ is the triditional intersection number 
for cycles.

****

\[
	m_{\alpha'}(\alpha,\beta)=\left(-2i\right)^{n-3}T_{C^{\mathsf{SL}}_{\text{reg}}(\alpha)}\wedge
	T_{C^{\mathsf{SL}}(\beta)}(1)
\]

****

The construction of Poincar\'e dual.

****

In \cite{mizera2018scattering}, amplitude can be seen as the intersection
number of two twisted cocyles, it can also writen as the intersection
of currents:
\[
	\langle \psi_L,\psi_R\rangle_\omega= T_{\psi_L}\wedge T_{\psi_R}(1),
\]
where $T_{\psi_L}$ is defined in (\theequation).

****

\begin{conj}
In field theory limit, 
\[
	\theta_\omega(C^{\mathsf{SL}}_{\text{reg}}(\alpha))\longrightarrow \frac{1}{(\alpha')^{n-3}}\prod_i \delta(E_i)\delta(\sigma_i-\sigma_i^*)\overline{\mathsf{PT}(\alpha)}.
\]
\end{conj}


% and the definition gives that
% \[
% 	\int_{C_{\text{reg}}(\alpha)}\mathsf{PT}(\beta)
% 	=\int_M \theta_\omega(C_{\text{reg}}(\alpha))\wedge \mathsf{PT}(\beta)
% \]
% and then
% \[
% 	\lim_{\alpha'\to 0}Z(\alpha|\beta)\longrightarrow\frac{1}{(\alpha')^{n-3}}m(\alpha|\beta).
% \]


% \section{hahajaja}

% Near $\sigma_0\in [C]\cap [D]$, and it's on hyperplain $H_{ij}$, then 
% \[
% 	\theta_{\omega}(C)
% 	\propto\prod_i\frac{u_C}{\exp(2\pi i\alpha' s_{\alpha(i),\alpha(i+1)})-1}
% 	\delta(|\sigma_i-\sigma_j|-\epsilon)
% 	\frac{d(|\sigma_i-\sigma_j|-\epsilon)}{d\sigma_i}(\sigma_i-\sigma_j)
% 	\mathsf{PT}(\alpha)
% \]
% and we could construct a normal vector field $v$ on $[D]$, see \cite{dijkgraaf1988c}, and then 
% \[
% 	[D']=\{(\{\sigma_i,v_i(\sigma)\})\in \cc^n\,:\, (\{\sigma_i\}) \in [D]\}
% \]
% and $D'=D\in H_n(M,\mathcal L_{-\omega})$.

% \[
% 	\theta_{-\omega}(D')
% 	\propto u_D^{-1}
% 	\prod_i \delta(\operatorname{Im}\sigma_i-v_i(\operatorname{Re} \sigma))
% 	(\sigma_{\beta(i)}-\sigma_{\beta(i+1)})^*
% 	\mathsf{PT}(\beta)^*,
% \]
% % \[
% % 	\theta_{-\omega}(D)
% % 	\propto u_D^{-1}(\sigma)
% % 	\delta^{n-3}(\sigma+iv(\sigma))\frac{d(\sigma+iv(\sigma))}{d\sigma_{\beta(i)}^*}\prod_i (\sigma_{\beta(i)}-\sigma_{\beta(i+1)})^*
% % 	\mathsf{PT}(\beta)^*
% % \]
% then
% \[
% 	\theta_{\omega}(C)\wedge \theta_{-\omega}(D)
% 	=\propto\prod_i\frac{u_Cu_D^{-1}}{\exp(2\pi i\alpha' s_{\alpha(i),\alpha(i+1)})-1}
% 	\frac{\delta(\sigma_i-\sigma_j-\epsilon)}{(\sigma_j-\sigma_i)^{-2}}\delta(\sigma_i-\sigma_i^*)
% 	\mathsf{PT}(\alpha)\wedge \mathsf{PT}(\beta)^*
% \]
% In the field theory limit,
% \[
% 	\theta_{\omega}(C)\wedge \theta_{-\omega}(D)
% 	\propto \frac{1}{2\pi{\alpha'}^{n-3}}
% 	\delta\left(\frac{s_{ij}}{\sigma_{ij}}-a\right)
% 	\mathsf{PT}(\alpha) \widehat{\mathsf{PT}(\beta)}
% \]
% on $\mathcal M_{0,n}(\mathbb R)$, where
% \[
% 	\mathsf{PT}(\beta)=\widehat{\mathsf{PT}(\beta)}\bigwedge_i dz_i^*.
% \]

\bibliographystyle{plain} 
\bibliography{3.bib} 

*****

\end{document}
