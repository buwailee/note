\documentclass[8pt]{article}
\usepackage{amssymb,amsfonts,amsthm,amsmath,bm,mathrsfs}
%\usepackage{extarrows}
\usepackage[left=21mm,text={148mm,200mm},paperwidth=185mm,paperheight=250mm,includehead,vmarginratio=1:1]{geometry}
\usepackage[all]{xy}
\usepackage{tikz}
\usepackage{titletoc}
	\theoremstyle{plain}
	\newtheorem{pro}{Proposition}
	\newtheorem{theo}{Theorem}
	\newtheorem{defi}{Definition}
	\newtheorem{lem}{Lemmma}
	\newtheorem{cor}{Corollary}

\definecolor{shadecolor}{rgb}{0.92,0.92,0.92}

\newcommand{\no}[1]{{$(#1)$}}
% \renewcommand{\not}[1]{#1\!\!\!/}
\newcommand{\rr}{\mathbb{R}}
\newcommand{\zz}{\mathbb{Z}}
\newcommand{\aaa}{\mathfrak{a}}
\newcommand{\pp}{\mathfrak{p}}
\newcommand{\mm}{\mathfrak{m}}
\newcommand{\dd}{\mathrm{d}}
\newcommand{\oo}{\mathcal{O}}
\newcommand{\calf}{\mathcal{F}}
\newcommand{\calg}{\mathcal{G}}
\newcommand{\bbp}{\mathbb{P}}
\newcommand{\bba}{\mathbb{A}}
\newcommand{\osub}{\underset{\mathrm{open}}{\subset}}
\newcommand{\csub}{\underset{\mathrm{closed}}{\subset}}

\DeclareMathOperator{\im}{Im}
\DeclareMathOperator{\Hom}{Hom}
\DeclareMathOperator{\id}{id}
\DeclareMathOperator{\rank}{rank}
\DeclareMathOperator{\tr}{tr}
\DeclareMathOperator{\supp}{supp}
\DeclareMathOperator{\coker}{coker}
\DeclareMathOperator{\codim}{codim}
\DeclareMathOperator{\height}{height}
\DeclareMathOperator{\sign}{sign}

\DeclareMathOperator{\ann}{ann}
\DeclareMathOperator{\Ann}{Ann}
\DeclareMathOperator{\ev}{ev}

\begin{document}

A topological space $X$ can be seen as a category if we take open sets as its objects and inclusion maps as its morphisms, i.e. 
\[
	\Hom_{X}(U,V)=\begin{cases}
	\bigl\{i_{UV}:U\hookrightarrow V\bigr\},& \text{if }U\osub V\osub X\text{;}\\
	\varnothing,&\text{otherwise}.
	\end{cases}
\]

\begin{defi}
	A presheaf is a functor $\calf:X^\circ\to \textit{AG}$, where $X^\circ$ is the dual category of a topological space $X$ and \textit{AG} is the category of abelian groups. In other words, $\calf:X\to \textit{AG}$ is a contravariant functor. The elements $s\in \calf(U)$ are called the sections on $U$. Denote that $\rho_{UV}:=\calf(i_{UV}):\calf(V)\to\calf(U)$ and $\rho_{UV}(s)=s|_V$ for $s \in \calf(U)$.
\end{defi}

The morphisms $\varphi$ between two presheaf $\calf$ and $\calg$ is just the morphisms between two functors, that is, there exist a family of morphisms $\varphi(U):\calf(U)\to \calg(U)$ such that the following diagram is commutative.
\[
	\begin{xy}
	\xymatrix{
		\calf(U)\ar[rr]^{\varphi(U)} \ar[d]_{\rho_{UV}}&&\calg(U) \ar[d]^{\rho'_{UV}}\\
		\calf(V)\ar[rr]^{\varphi(V)}&&\calg(V)
	}
	\end{xy}
\]
\begin{defi}
	If $\calf$ is a presheaf on $X$, and if $p$ is a point on $X$, we define the stalk $\calf_p\in \textit{AG}$ of $\calf$ at $p$ to be the direct limit\footnote{$X$ is equiped with an order that $U>V$ if $U\subset V$. Since $\calf(U)$ are abelian groups, as $\zz$-modules, the direct limit always exists.} of the group $\calf(U)$ for all $U \osub X$ containing $p$, via the restriction maps $\rho$.
\end{defi}

We can directly construct $\calf_p=\varinjlim_{U\ni p}\calf(U)$ that 
\[
	\calf_p=\bigl\{\langle U,s\rangle: p\in U\osub X, s\in \calf(U)\bigr\}/\sim,
\]
where $\sim$ is defined as follows: Suppose $s\in\calf(U)$ and $t\in\calf(V)$, if there exists a $W\osub U\cap V\neq \varnothing$ such that $s|_W=t|_W$, then $\langle U,s\rangle=\langle V,t\rangle$ or $s\sim t$. $\calf_p$ is indeed a group since the addition can be defined by $\langle U,s\rangle+\langle V,t\rangle=\langle W,s|_W+t|_W\rangle$. 

Here $\calf(U)\ni s\mapsto \langle U,s\rangle=s_p$ defines a family of morphisms $\rho_{Up}:\calf(U)\to \calf_p$ for any $U\osub X$. We can then vertify the universal property of direct product that
\[
\begin{xy}
	\xymatrix{
	&&\ar[dl]^{\rho_{Up}}\calf(U)\ar[dd]^{\rho_{UV}}\ar@/_/[lld]_{\varphi_{Vp}} \\
	W&\ar@{-->}[l]_(0.4){\theta}\calf_p&\\
	&&\ar[ul]_{\rho_{Up}}\calf(V)\ar@/^/[llu]^{\varphi_{Vp}}
	}
\end{xy}
\]

Now, suppose $\varphi:\calf \to \calg$ is a functor. The universal property of direct product 
\[
	\begin{xy}
	\xymatrix{
		&&\ar[dl]^{\rho_{Up}}\calf(U)\ar[dd]^{\rho_{UV}}\ar@/_/[lld]_{\rho'_{Up}\circ\varphi(U)} \\
		\calg_p&\ar@{-->}[l]_(0.4){\varphi_p}\calf_p&\\
		&&\ar[ul]_{\rho_{Vp}}\calf(V)\ar@/^/[llu]^{\rho'_{Vp}\circ\varphi(V)}
	}
	\end{xy}
\]
gives the existence of the morphism $\varphi_p:\calf_p\to \calg_p$, and the diagram
\begin{figure}[h]
\[
\begin{xy}
	\xymatrix{
		\calf(U)\ar[rr]^{\varphi(U)} \ar[d]_{\rho_{Up}}&&\calg(U) \ar[d]^{\rho'_{Up}}\\
		\calf_p\ar@{-->}[rr]^{\varphi_p}&&\calg_p
	}
\end{xy}
\]
\caption{The existence of $\varphi_p$}
\label{fig1}
\end{figure}

\noindent is commutative.
\begin{defi}
	A sheaf is a presheaf that for any open set $U\subset X$, the complex 
\[
	0\rightarrow \calf(U) \xrightarrow{d_0}\prod_{i\in I}\calf(U_i) \xrightarrow{d_1}\prod_{i,j\in I}\calf(U_i\cap U_j)
\]
is exact for any open cover $\{U_i\}$ of $U$, where
\[
\begin{array}{cccl}
	d_0:&s&\mapsto& \displaystyle{\prod_{i\in I}s|_{U_i}},\\
	d_1:&\displaystyle{\prod_{i\in I}s_i}&\mapsto& \displaystyle{\prod_{i,j\in I}\bigl(s_i|_{U_i\cap U_j}-s_j|_{U_i\cap U_j}\bigr)}.
\end{array}
\]
\end{defi}

The definition can be rewritten as: For any open cover $\{U_i\}$ of any open set $U\subset X$,
\begin{itemize}
\item If $\forall i\in I$, $s|_{U_i}=0$, then $s=0$.

\item If $\forall i,j\in I$, $s_i|_{U_i\cap U_j}=s_j|_{U_i\cap U_j}$, then there's a section $s\in\calf(U)$ such that $s|_{U_i}=s_i$.
\end{itemize}
It is not so difficult to vertify that this definition is equivalent to the old one.

The following proposition (which would be false for presheves) illustrates the local nature of a sheaf.
\begin{pro}
	Suppose $\varphi:\calf\to\calg$ is a morphism of sheaves on a topological space $X$. Then $\varphi$ is a isomorphism if and only if the induced map on the stalk $\varphi_p:\calf_p\to \calg_p$ is an isomorphism for every $p\in X$.
	\label{pro:1}
\end{pro}
\begin{proof}
	p.63 on Hartshorne.
\end{proof}
\begin{defi}
	Let $\varphi:\calf\to\calg$ be a morphism of presheaves. We define the presheaf kernel of $\varphi$, presheaf of cokernel of $\varphi$, and presheaf image of $\varphi$ to be the presheaves given by $U\mapsto \ker(\varphi(U))$, $U\to \coker(\varphi(U))$, and $U\mapsto \im(\varphi(U))$ respectively.
\end{defi}
\begin{pro}
If $\varphi:\calf\to\calg$ is a morphism of sheaves, then the presheaf $U\mapsto \ker(\varphi(U))$ is a sheaf.
\end{pro}
\begin{proof}
	Let $\{U_i\}$ be an open cover of $U$, and $s_i$ is local section on $U_i$. 

	\begin{itemize}
		\item Suppose $s\in \ker(\varphi(U))$ and $s|_{U_i}=0$, since $\calf$ is a sheaf, $s=0$.

		\item Suppose $s_i|_{U_i\cap U_j}=s_j|_{U_i\cap U_j}$, we need to show that there exists a global section $s\in \ker(\varphi(U))$ such that $s|_{U_i}=s_i$. Since $\calf$ is a sheaf, it's nature that there's $s\in \calf(U)$ such that $s|_{U_i}=s_i$. The last thing to vetify is $\varphi(U)(s)=0$. Restrict $\varphi(U)(s)$ on $U_i$, then
		\[
			\rho'_{UU_i}\circ \varphi(U)(s)=\varphi(U_i)(\rho_{UU_i}s)=\varphi(U_i)(s_i)=0,
		\]
		so $\varphi(U)(s)\in \calg(U)$ vanishes locally. Since $\calg$ is a sheaf, it also 	vanishes globally, i.e. $\varphi(U)(s)=0$.
	\end{itemize}
	Thus $U\mapsto \ker(\varphi(U))$ is a sheaf.
\end{proof}
However, the presheaves $\coker(\varphi)$ and $\im(\varphi)$ need not to be sheaves. Actually, the key point in the proof above is that $\ker$ is compatible with the sheaf property of $\calg$. Then we come to an important notion of a sheaf associated to a presheaf, i.e. sheafification.

Roughly speaking, the sheafification of a presheaf $\calf$ is the "smallest" sheaf with the same stalks as $\calf$. Because of the "smallest", sheafification should have the universal property.

\begin{pro}
	Given a presheaf $\calf$, there is a sheaf $\calf^+$ and a morphism $\theta$ make the diagram
	\begin{center}
	\leavevmode
		\xymatrix{
			\calf \ar[rr]^\theta\ar[d]_\varphi&& \calf^+\ar@{-->}[lld]^{\psi}\\
			\calg&&
		}
	\end{center}
	commutative for any sheaf $\calg$. $\calf^+$ is called the sheaf associated to the preshead $\calf$ or the sheafification of $\calf$.
\end{pro}
\begin{proof}
	We construct the sheaf $\calf^+$ as follows. For any open set $U$, let $\calf^+(U)$ be the set of functions $s:U\to \cup_{p\in U}\calf_p$, such that
	\begin{itemize}
		\item for each $p\in U$, $s(p)\in \calf_p$, and

		\item for each $p\in U$, there is a neighborhood $V\osub U$ of $P$, and an element $t\in\calf(V)$, such that $\forall q\in V$, $s(q)=t_q:=t^+(q)$.
	\end{itemize}

	The addition on $\calf^+(U)$ is that $(s+t)(p)=s(p)+t(p)$, so $\calf^+(U)$ is indeed a group. If $V\osub U$, there's a nature map (function restriction) $i_{UV}:\calf^+(U)\to \calf^+(V)$ such that $i_{UV}(s)=s|_V$, so $\calf^+$ is a presheaf. Let $\{U_i\}$ be a open cover of $U$ and $s_i\in \calf^+(U_i)$ be local sections. If $s_i|_{U_i\cap U_j}=s_j|_{U_i\cap U_j}$, we can define a function $s:U\to \cup_{p\in U}\calf_p$ by setting $s|_{U_i}=s_i$. If $s|_{U_i}=0$ for all $i\in I$, then $s=0$ since it is a function. Thus $\calf^+$ is a sheaf.

	For each $s\in \calf(U)$, we can associate it a section $s^+\in \calf^+(U)$ by $s^+(p)=s_p$, then there's a morphism $\theta(U):s\mapsto s^+$. $\forall s\in \calf(U)$, since
	\[
		i_{UV}\bigl(\theta(U)(s)\bigr)=i_{UV}(s^+)=s^+|_{V}
	\]
	and
	\[
		\theta(V)\bigl(\rho_{UV}(s)\bigr)=\theta(V)(s|_V)=s^+|_{V}.
	\]
	$\theta$ is a morphism.

	Let $\bar{s}\in \calf^+(U)$, because of the construction of $\calf^+$, we can find an open cover $\{U_i\}$ of $U$ such that $\bar{s}|_{U_i}=\bar{s}^+_i$, where $\bar{s}_i\in \calf(U_i)$. Firstly define $\psi(U_i):\bar{s}|_{U_i}\mapsto \varphi(U_i)(\bar{s}_i)$, and we can use the sheaf condition of $\calg$ to get a global section $s'$ on $U$ such that $s'|_{U_i}=\varphi(U_i)(\bar{s}_i)$. Then, define $\psi(U):\bar{s}\mapsto s'$, and $\psi$ will become the morphism $\psi:\calf^+\to \calg$. Finally, because of the construction of $\psi$, for any $s^+\in \calf^+(U)$, $\psi(U)(s^+)=\varphi(U)(s)$, so that
	\[
		\psi(U)(\theta(s))=\psi(U)(s^+)=\varphi(U)(s)
	\]
	makes the diagram commutative. 
\end{proof}
\begin{pro}
	$\calf_p\cong \calf^+_p$, so if $\calf$ is a sheaf, then $\calf\cong \calf^+$.
	\label{pro:4}
\end{pro}
\begin{proof}
	According to the Figure \ref{fig1}, there's a morphism $\theta_p:\calf_p\to \calf^+_p$, such that $\theta_p(\langle U,s\rangle)=\langle U,s^+\rangle$.
	\begin{itemize}
	\item It is injective. If $\theta_p(\langle U,s\rangle)=\langle V,0\rangle$, then $s^+|_V=0$, and $s_p=s^+(p)=0$.

	\item It is surjective. $\forall \langle U,\bar{s}\rangle\in \calf^+_p$, there exists an open subset $V$ and $t\in \calf(V)$ such that $\langle U,\bar{s}\rangle=\langle V,t^+\rangle$, then $\theta_p(\langle V,t\rangle)=\langle U,\bar{s}\rangle$.
	\end{itemize}
\end{proof}
\begin{defi}
	If $\varphi:\calf\to\calg$ is a morphism of sheaves, we define the kernel(repsectively cokernel, image) of $\varphi$, denoted $\ker \varphi$(repsectively $\coker \varphi$, $\im \varphi$), to be the sheaf associated to the presheaf of kernel(respectively, coker, image) of $\varphi$.
\end{defi}
\begin{defi}
	We say that a morphism of sheaves $\varphi:\calf\to\calg$ is injective(respectively, surjective) if $\ker\varphi=0$(respectively, $\im\varphi\cong\calg$).
\end{defi}
\begin{pro}
	For any morphism of sheaves $\varphi:\calf\to\calg$, $(\ker \varphi)_p=\ker(\varphi_p)$ and $(\im \varphi)_p=\im(\varphi_p)$ for each $p$.
	\label{pro:5}
\end{pro}
\begin{proof}
We will show the following statements.
	\begin{itemize}
		\item $\ker(\varphi_p)\subset (\ker \varphi)_p$: Suppose $s_p=\langle U,s\rangle\in \ker(\varphi_p)$, then $\varphi_p\langle U,s\rangle=\langle U,\varphi(U)s\rangle=0$, so there exists $W\osub U$ such that 
		\[
			(\varphi(U)s)|_W=\varphi(W)(s|_W)=0,
		\]
		thus $s|_W\in \ker(\varphi(W))$ and $s_p=\langle W,s|_W\rangle\in (\ker \varphi)_p$.

		\item $(\ker \varphi)_p\subset\ker(\varphi_p)$: Suppose $\langle U,s\rangle\in (\ker \varphi)_p$, then $\varphi_p\langle U,s\rangle=\langle U,\varphi(U)s\rangle=\langle U,0\rangle=0\in \calg_p$, thus $\langle U,s\rangle\in \ker(\varphi_p)$.

		\item $\im(\varphi_p)\subset (\im \varphi)_p$: Suppose $t=\varphi_ps_p\in\im (\varphi_p)$ and $s_p=\langle U,s\rangle$, then $t=\varphi_ps_p=\varphi_p\langle U,s\rangle=\langle U,\varphi(U)s\rangle\in (\im \varphi)_p$.

		\item $(\im \varphi)_p\subset \im(\varphi_p)$: Suppose $t=\langle U,\varphi(U)s\rangle\in (\im \varphi)_p$, then $t=\rho'_{Up}\circ\varphi(U)s=\varphi_p s_p \in \im(\varphi_p)$.
	\end{itemize}
\end{proof}
\begin{cor}
	For any morphism of sheaves $\varphi:\calf\to\calg$, it is injective(respectively, surjective) if and only if $\varphi_p$ is injective(respectively, surjective) for all $p$.
	\label{cor:1}
\end{cor}
\begin{proof}
	According to Proposition \ref{pro:1} and Proposition \ref{pro:5}, $\ker \varphi=0$ if and only if $(\ker \varphi)_p=\ker(\varphi_p)=0$. Similarly, $\im \varphi\cong \calg$ if and only if $(\im \varphi)_p=\im(\varphi_p)\cong \calg_g$.
\end{proof}
\begin{cor}
	A morphism of sheaves is an isomorphism if and only if it is injective and surjective.
\end{cor}
\begin{proof}
	A morphism of sheaves $\varphi$ is an isomorphism if and only if $\varphi_p$ is an isomorphism for all $p$. As a morphism of groups, $\varphi_p$ is an isomorphism for all $p$ if and only if it is injective and surjective for all $p$, and according to Corollay \ref{cor:1}, if and only if $\varphi$ is injective and surjective.
\end{proof}
\begin{defi}
	Let $f:X\to Y$ be a continuous map of topological spoaces. For any sheaf $\calf$ on $X$, we define the direct image sheaf $f_*\calf$ on $Y$ by $(f_*\calf)(U)=\calf(f^{-1}(U))$ for any open set $U\subset Y$. For any sheaf $\calg$ on $Y$, we define the inverse image sheaf $f^{-1}\calg$ on $X$ to be the sheaf associated to the presheaf $U\mapsto \varinjlim_{V \supset f(U)}\calg(V)$, where $U$ is any open set in $X$, and the limit is taken over all open sets $V\subset Y$ containing $f(U)$. 
\end{defi}
Especially, if $f(U)$ is an open set, then $\varinjlim_{V \supset f(U)}\calg(V)=\calg\bigl(f(U)\bigr)$.
\end{document}