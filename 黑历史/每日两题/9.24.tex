\documentclass[11pt,a4paper,openany]{article}
\usepackage{amssymb}
\usepackage{amsfonts}
\usepackage{amsmath}
\usepackage{ctex}
\usepackage{bm}

\begin{document}

1.有一个伟大的公式,我们叫他欧拉公式:
\[
e^{it}=\cos t+i\sin t
\]
试证明:
\[
1+\frac{\cos x}{1!}+\cdots+
\frac{\cos nx}{n!}+\cdots
=e^{\cos x}\cos{(\sin{x})}
\]
\[
\frac{\sin x}{1!}+\cdots+
\frac{\sin nx}{n!}+\cdots
=e^{\cos x}\sin{(\sin{x})}
\]
\\
2.为了体现出$\Gamma$函数(见9/22)实际在应用上已经很和初等函数地位类似了。所以,尽可能习惯起来$\Gamma$函数比较好。\\
\\
(1).假设一维势能函数为$U(x)$.试着根据能量守恒推出周期$T$和总能量$E$的关系。特别当$U(x)=A|x|^n$,试求$T(E)$.\\(提示,动能和势能和为定值$E$,当势能等于总能量的大小时,他必然处于边界,由于只能在两个边界中移动,所以从一个边界到另一个边界的时间的两倍就是周期)\\
\\
(2).如果$n$维球是指$\{(x_1,x_2,\dots,x_n)|x_1^2+x_2^2+\dots+x_n^2 \leq r^2\}$,且$n$维球的体积元为$\mathrm{d}V=\mathrm{d}x_1 \mathrm{d}x_2 \cdots \mathrm{d}x_n$.试求半径为$r$的$n$维球的体积公式。(假设我们已知$n$维球的体积与$r^n$成正比)
\newpage
解答:
这题恐怕是我发过最水的了。\\
1.\\
因为$\cos{(\sin x)}+i\sin{(\sin x)}=e^{i\sin x}$,又因为
\[
e^{\cos x}e^{i\sin x}=e^{\cos x+i\sin x}=e^{e^{ix}}
\]
又因为$e^x$的幂级数展开为,
\[
e^x=\sum_{k=0}^\infty \frac{x^k}{k!}
\]
所以,
\[
e^{e^{ix}}
=\sum_{k=0}^\infty \frac{e^{ikx}}{k!}
=\left(1+\frac{\cos x}{1!}+\cdots+
\frac{\cos nx}{n!}+\cdots\right)
+i\left(\frac{\sin x}{1!}+\cdots+
\frac{\sin nx}{n!}+\cdots\right)
\]
而,
\[
e^{\cos x}\cos{(\sin x)}+ie^{\cos x}\sin{(\sin x)}=e^{\cos x}e^{i\sin x}
=e^{e^{ix}}
\]
联系上面两个式子就可得证。
\\
\\
2.\\
(1).第一小题得到了一个有关力学一般结论,用他可以推导出开普勒三定律。\\
首先能量守恒,
\[
\frac{1}{2}m\dot{x}^2+U(x)=E
\]
然后分离变量,
\[
\frac{\mathrm{d}x}{\mathrm{d}t}
=\sqrt{\frac{2}{m}(E-U(x))}
\]
积分可得,
\[t=\sqrt{\frac{m}{2}}\int
\frac{\mathrm{d}x}{\sqrt{E-U(x)}}
\]
\indent 积分限的确定,其物理意义也就是运动的界的确定。由于在势能等于总能量的时候,动能等于$0$,此时的位置则必然就是运动的边界。所以运动的边界由$U(x)=E$这个方程确定的,方程可能有多解,具体的形式需要具体分析。这里我们就在方程有两解$x_1 (E)$和$x_2 (E)$(即最远和最近)的情况下讨论。根据提示,容易得到,
\[
T(E)=\sqrt{2m}\int^{x_2 (E)}_{x_1 (E)}\frac{\mathrm{d}x}{\sqrt{E-U(x)}}
\]
这是一般解,题目给出了势能的具体形式$U(x)=A|x|^n$,注意到势能关于原点对称,代入积分,
\[
T(E)=\sqrt{2m}\int^{x_2 (E)}_{x_1 (E)}\frac{\mathrm{d}x}{\sqrt{E-A|x|^n}}
=2\sqrt{2m}\int^{(E/A)^{1/n}}_{0}\frac{\mathrm{d}x}{\sqrt{E-Ax^n}}\]
换元,令$E-Ax^n=1-u$,所以
\[
T(E)=\frac{2\sqrt{2m}E^{1/n-1/2}}{nA^{1/n}}\int u^{(1-n)/n}(1-u)^{-1/2}\mathrm{d}u
=\frac{2\sqrt{2m}E^{1/n-1/2}}{nA^{1/n}}\text{B}\left(\frac{1}{2},\frac{1}{n}\right)
\]
(2).\\ \indent 由提示,设$V_k (r)=a_k r^k $,设$n$维球为$S_n$.\\同时,
\[V_k=
\int_{S_k} 
\mathrm{d}x_1 \mathrm{d}x_2 \cdots \mathrm{d}x_k
\]
由于$k$维球是$\{(x_1,x_2,\dots,x_k)|x_1^2+x_2^2+\dots+x_k^2 \leq r^2\}$,不妨单独分离出一个$x_1$,注意此时的半径$r'=\sqrt{r^2-x_1^2}$.
\[
\begin{split}
V_k=&
\int^r_{-r} \left[\int_{S_{k-1}} 
\mathrm{d}x_2 \cdots \mathrm{d}x_k \right]\mathrm{d}x_1\\
=&\int^r_{-r} \left[V_{k-1}\left(\sqrt{r^2-x^2_1}\right)\right]\mathrm{d}x_1\\
=&\int^r_{-r} \left[a_{k-1}\left(\sqrt{r^2-x^2_1}\right)^{k-1}\right]\mathrm{d}x_1\\
=&a_{k-1}r^k \text{B}\left(\frac{1}{2},\frac{k+1}{2} \right)
\end{split}
\]
于是可以得到递推式,
\[
\begin{split}
a_k=a_{k-1}\text{B}\left(\frac{1}{2},\frac{k+1}{2} \right)
&=\sqrt{\pi}a_{k-1}\frac{\Gamma((k+1)/2)}{\Gamma(1+k/2)}\\
&=\pi a_{k-2}
\frac{\Gamma((k+1)/2)}{\Gamma(1+k/2)}
\frac{\Gamma(k/2)}{\Gamma((k+1)/2)}\\
&=\pi a_{k-2}
\frac{\Gamma(k/2)}{\Gamma(1+k/2)}
=\frac{2\pi}{k} a_{k-2}
\end{split}
\]
根据$a_1=2$,$a_2=\pi$,所以,当$k=2p$的时候,
\[
a_k
=\frac{\pi}{p} a_{2p-2}
=\frac{\pi^{p}}{p!}
\]
因为$\Gamma(p+1)=p!$,所以该公式为,
\[
a_k
=\frac{\pi^{p}}{\Gamma(p+1)}
=\frac{\pi^{k/2}}{\Gamma(k/2+1)}
\]
当$k=2p-1$的时候,
\[
a_k
=\frac{2\pi}{2p-1} a_{2p-3}
=\frac{2^{p} \pi^{p-1}}{(2p-1)!!}
\]
根据9/22证明的结论,有
$\displaystyle{\Gamma\left(\frac{1}{2}\right)=\sqrt{\pi}}$和$\displaystyle{\Gamma\left(x+1\right)=x\Gamma\left(x\right)}$,所以很容易证明,
\[
\Gamma\left(\frac{k}{2}+1\right)
=\Gamma\left(p+\frac{1}{2}\right)=\frac{(2p-1)!!}{2^p}\Gamma\left(\frac{1}{2}\right)
=\frac{(2p-1)!!}{2^{p}}\sqrt{\pi}
\]
将上面两个式子直接相乘可以得到,
\[
a_k\Gamma\left(\frac{k}{2}+1\right)
=\pi^{p-\frac{1}{2}}
=\pi^{k/2}
\]
所以,再一次得到了公式
\[
a_k
=\frac{\pi^{k/2}}{\Gamma(k/2+1)}
\]
因此公式的形式和$k$的奇偶是没有关系的。\\
\indent 综上,$n$维球体积的一般公式为,
\[
V_n
=\frac{\pi^{n/2}r^n }{\Gamma(n/2+1)}
\]
\end{document}