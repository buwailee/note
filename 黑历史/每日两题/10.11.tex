\documentclass[11pt,a4paper,openany]{article}
\usepackage{amssymb,amsfonts,
amsmath,ctex,bm}

\begin{document}
由于微积分课老师一直在那读证明。导致我觉得微积分课和数分课差不多了 = =!\\
1.如果$x_1,x_2 \in E \subset \mathbb{R}$,以及$f:E \rightarrow \mathbb{R}$,定义$\displaystyle{\omega(\delta) =\sup_{|x_1-x_2|<\delta}|f(x_1)-f(x_2)|}$,其中$x_1,x_2$取遍所有可能的点对。试证明:\\
(1).下列极限存在:
\[\lim_{\delta\rightarrow +0}\omega(\delta)\]
(2).若$E$是闭区间或开区间或半开区间(换而言之就是中间没断开的实数集的子集),则对于
$f$满足:
\[
\omega(\delta_1+\delta_2)
\leq \omega(\delta_1)+\omega(\delta_2)
\]
2.若$P_n(x)$为$n$阶多项式。定义多项式于有界函数$f$的偏差为,
\[
\Delta(P_n)=\sup_{x\in [a,b]}|f(x)-P_n(x)|
\]
试证明:当$P_n$是固定的多项式的时候,且$\lambda$为任意实数,那么对于多项式$Q_\lambda=\lambda P_n$,则存在实数$\lambda_0$,使得下式成立,
\[
\Delta(Q_{\lambda_0})=\min_{\lambda}\Delta(Q_{\lambda})
\]
\newpage
\noindent
不知道有木有说过,如果木有说过,那我就说一遍。这些答案都是我自己做的,当然会有不好或者不对的。若是您可以给出您的正确的漂亮的解答,欢迎和我探讨。\\ 
1.
\\(1).显然$\omega(\delta)$关于$\delta$单调递增。构造单调递减序列$\{\delta_n\}$,他的极限为0,则序列$\omega(\delta_n)$单调递减。$\omega(\delta_n)$显然有下界0,所以$\omega(\delta_n)$有极限。所以,
\[
\lim_{n\rightarrow +\infty}
\omega(\delta_n)=
\lim_{\delta\rightarrow +0}\omega(\delta)
\]
(2).对于$|x_1-x_2|<\delta_1+\delta_2$,不妨假设$x_1<x_2$,则从$[x_2-\delta_2,x_1+\delta_1]$可以挑出$x$(且这样的$x$是存在的),同时满足,
\[
\begin{split}
&x-x_1<\delta_1\\
&x_2-x<\delta_2
\end{split}
\]
那么,则
\[
|f(x_1)-f(x)|+|f(x_2)-f(x)|\geq |f(x_1)+f(x_2)|
\]
对两边同时取两次sup,则
\[
\sup_{|x_1-x|<\delta_1}|f(x_1)-f(x)|
+
\sup_{|x_2-x|<\delta_2}|f(x_2)-f(x)|
\geq \sup_{
\substack{x_2-x<\delta_2 
\\ 
x-x_1<\delta_1}}
|f(x_1)-f(x_2)|
\]
由于,
\[
\sup_{|x_2-x_1|<\delta_1+\delta_2}|f(x_1)-f(x_2)|=\sup_{
\substack{x_2-x<\delta_2 
\\ 
x-x_1<\delta_1}}
|f(x_1)-f(x_2)|
\]
所以,
\[
\omega(\delta_1+\delta_2)
\leq \omega(\delta_1)+\omega(\delta_2)
\]
2.
由于$\Delta$现在只和$\lambda$有关,所以不妨记作$\Delta(\lambda)$,这是一个从$\mathbb{R}$到非负实数集的映射。首先我们证明他是连续的。\\
由于,
\[
|f(x)-(\lambda+\delta)P_n(x)|
\leq |f(x)-\lambda P_n(x)|+|\delta P_n(x)|
\]
在两边同时取sup,则
\[
\sup |f(x)-(\lambda+\delta)P_n(x)|
\leq \sup |f(x)-\lambda P_n(x)|+\sup |\delta P_n(x)|
\]
即,
\[
\Delta(\lambda+\delta)
-\Delta(\lambda)
\leq
\sup|\delta P_n(x)|
\]
设$|P_n(x)|$在$[a,b]$上的最大值为$M$,取$\varepsilon=\delta/M$,则
\[
\Delta(\lambda+\delta)
-\Delta(\lambda)
\leq
\varepsilon
\]
所以这是一个连续函数。但是连续函数的条件不能说明他一定有最小值,即使他有下界。所以还需要考察下面一步。\\
\indent 写出$\Delta(\lambda)$的具体表达式,
\[
\Delta(\lambda)=\sup |f(x)-\lambda P_n(x)|
\]
发现如果$|\lambda| \rightarrow +\infty$,则$\Delta(\lambda)$可以大于任意给定的实数,那么自然可以在$|\lambda|>m$的时候,$\Delta(\lambda)>\Delta(0)$。那么考虑$\Delta(\lambda)$的最小值,自然只要考虑$\lambda \in [-m,m]$的情况。在闭区间内,连续函数必然可以取到最小值。于是命题得证。
\end{document}