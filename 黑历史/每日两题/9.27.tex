\documentclass[11pt,a4paper,openany]{article}
\usepackage{amssymb}
\usepackage{amsfonts}
\usepackage{amsmath}
\usepackage{ctex}
\usepackage{bm}

\begin{document}
\indent 这是一个有趣的命题,重新回到集合论上的一些结论,有关于集合的势的概念,可以看9/25。\\
1.若一个集合$X$是可数的,当且仅当$\text{card}X=\text{card}\mathbb{N}$.且已知有限个或者可数个可数集的并是可数的。\\
(1).若一个集合中的元素可以排成一个数列,试证明这个集合是可数的。\\
(2).试证明:整数集$\mathbb{Z}$是可数的。\\
(3).若$a\in A$和$b\in B$,$A$和$B$的笛卡尔积$A\times B$被定义为所有二元有序组的集合$\{(a,b)|a\in A\wedge b\in B \}$,并且简记为$A\times A=A^2$.\\
试证明:$\mathbb{N}^2$是可数的。
\\
\\
2.若一系列$\mathbb{R}$上的闭区间$I_n$满足,
\[
I_1\supset I_2 \supset \cdots
\supset I_n \supset \cdots
\]
(1).试证明:
存在一点$x\in \mathbb{R}$,属于这些闭区间中的每一个。\\
(2).试证明:$\text{card}\mathbb{N}<\text{card}\mathbb{R}$(或者说实数集不可数)
\newpage
解答:\\
(1).根据数列的定义$f:\mathbb{N}\rightarrow X$,这里的$f$是一个双射。所以显然,这个集合$X$是可数的。\\
(2).定义集合$\mathbb{N}-i$,就是说$\mathbb{N}$中的每个元素都减去$i$。形式化的定义就是,
\[
\mathbb{N}-i:=
\{x|x=n-i\wedge n\in \mathbb{N}\}
\]
所以这个集合可数。那么显然,集合$\mathbb{Z}$就满足,\\
\[
\mathbb{Z}=\bigcup_{i=1}^{\infty}
\left(\mathbb{N}-i\right)
\]
根据给出的定理,可数集的可数并是可数的,于是$\mathbb{Z}$是可数的。\\
(3).对于$(a,b)$,固定$a$,那么$(a,b)$和$\mathbb{N}$之间有自然的一对一映射,因而是可数的。然后所有的$(a,b)$构成的集合为
\[
\bigcup_{b=1}^{\infty}(a,b)
\]
根据可数集的可数并是可数的,可以得到他是可数的。于是$\mathbb{N}^2$可数。
\\
\indent 并且,很容易证明$\mathbb{Z}^2$也是可数的。由于任何一个有理数$a/b$可以表示成一个有序组$(a,b)$的形式(会有不同的有序组对应同一个有理数,比如$a'=ka\bigwedge b'=kb$),所以有理数集$\mathbb{Q}$是$\mathbb{Z}^2$的子集。因为$\mathbb{Z}^2$可数,所以$\mathbb{Q}$或者有限或者可数,因为它是无限的,所以有理数集也是可数的。
\\
\\
2.
(1).假设$I_n=[a_n,b_n]$,根据条件,一定满足
\[
a_1 \leqslant \cdots \leqslant a_n \leqslant \cdots
\leqslant b_n \leqslant \cdots \leqslant b_1
\]
由于实数集是稠密的,换而言之,一定存在$x$ 满足对任意的$a_ n\leqslant b_n$有$a_ n\leqslant x\leqslant b_n$成立,所以有,
\[
a_1 \leqslant \cdots \leqslant a_n \leqslant \cdots \leqslant x  \leqslant \cdots
\leqslant b_n \leqslant \cdots \leqslant b_1
\]
于是这个x在所有的$I_n$中。\\
(2).
因为$A=[0,1]$和$\mathbb{R}$之间存在一一映射,所以只要证明
$\text{card}\mathbb{N}<\text{card} A$就可以了。
假设$A$中的点可以排成一个数列$\{x_n\}$.在闭区间$A=I_0$上,取定一闭区间$I_1$,使得$I_1$长度不为零,且$I_1$不含$x_1$,这样不断做下去。那么,从A中可以截取长度不为0的$I_n$,使得$x_n$不在$I_n$内。显然,$I_1\supset I_2 \supset \cdots
\supset I_n \supset \cdots$,根据第一小题,存在一点$x$,他属于所有的$I_n$,但是根据上面$I_n$的构造法,$x$不能是点列$\{x_n\}$中的任何一点,于是矛盾。所以$A$不可数,所以$\mathbb{R}$不可数。
\end{document}