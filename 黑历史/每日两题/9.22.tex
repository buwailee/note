\documentclass[11pt,a4paper,openany]{article}
\usepackage{amssymb}
\usepackage{amsfonts}
\usepackage{amsmath}
\usepackage{ctex}
\usepackage{bm}

\begin{document}

\noindent 
1.试求
$$\int_0^1 \frac{x}{e^x+e^{1-x}}\, \mathrm{d}x$$
\indent 提示:换元$t=1-x$试试。
\\

\noindent 2.
$\mathbb{R}^{+}$ $\left(x\neq 0\right)$上存在一个函数
$\Gamma\left(x\right)$(第二类欧拉积分),其定义为,
\[
\Gamma\left(x\right)=
\int^{\infty}_{0}e^{-t}t^{x-1}\,\mathrm{d}t.
\]
\indent (1).求证:
$$\Gamma\left(n+1\right)=n!.$$
\indent 提示:可以先证明$\Gamma\left(1\right)=1$再证明
$\Gamma\left(x+1\right)=x\Gamma\left(x\right)$.

(2).求证:
\[
\int_0^{\infty } e^{-x^2} \,\mathrm{d}x=\frac{\sqrt{\pi }}{2}
\]
\indent 然后证明:
\[
\Gamma\left(\frac{1}{2}\right)=
\sqrt{\pi }
\]
\indent 提示:设$\text{I}=\int_0^{\infty } e^{-x^2} \, dx$,\ $\text{I}^2=\iint e^{-(x^2+y^2)} \,\mathrm{d}x\mathrm{d}y$,又有$\mathrm{d}x\mathrm{d}y=r\mathrm{d}r\mathrm{d}\theta$和$r^2=x^2+y^2$可以求得(即所谓“直角坐标”转换为“极坐标”,注意全平面上$r\in[0,\infty]$,\ $\theta\in[0,2\pi]$)。\\
\indent (3).定义第一类欧拉积分为\[
\mathrm{B}(x,y)= \int_0^1t^{x-1}(1-t)^{y-1}\,\mathrm{d}t
\]
\indent 求证:
\[
\mathrm{B}(x,y)=\frac{\Gamma(x)\Gamma(y)}{\Gamma(x+y)}
\]
\indent 提示:先计算$\Gamma(x)\Gamma(y)$,方法同第二小题。
\newpage
解答:\\
1.$$\text{A}=\int_0^1 \frac{x}{e^x+e^{1-x}}\, \mathrm{d}x
$$
换元$t=1-x$,于是
$$\text{A}=-\int_1^0 \frac{1-t}{e^t+e^{1-t}}\, \mathrm{d}t
=\int_0^1 \frac{1-x}{e^x+e^{1-x}}\, \mathrm{d}x
$$
所以,
$$2\text{A}=
\int_0^1 \frac{1}{e^x+e^{1-x}}\, \mathrm{d}x
$$
换元$\sqrt{e}u=e^x$并且$\mathrm{d}u=u\mathrm{d}x$。所以,
$$2\text{A}=
\frac{1}{\sqrt{e}}\int_{1/\sqrt{e}}^{\sqrt{e}} \frac{1}{u^2+1}\, \mathrm{d}u
=\frac{1}{\sqrt{e}}\arctan{u}\bigg|_{1/\sqrt{e}}^{\sqrt{e}}
\approx 0.291366
$$
所以$A\approx 0.145683$
\\
\\
2.\\
(1).\[
\Gamma\left(1\right)=
\int^{\infty}_{0}e^{-t}\,\mathrm{d}t=-e^{-t}\bigg|^{\infty}_{0}=1
\]

\[
\Gamma\left(x+1\right)=
\int^{\infty}_{0}e^{-t}t^{x}\,\mathrm{d}t
=-\int^{\infty}_{0}t^{x}\,\mathrm{d}e^{-t}
=\left[\frac{-t^x}{\mathrm{e}^t}\right]_0^\infty + x \int_0^\infty \mathrm{e}^{-t} t ^{x - 1} {\rm{d}} t
\]
前面的一部分在$t=0$时显然为零,在$t\rightarrow \infty$时连续运用洛必达法则,
\[
\lim_{t \rightarrow \infty} \frac{-t^n}{\mathrm{e}^t} = \lim_{t \rightarrow \infty} \frac{-n! \cdot 0}{\mathrm{e}^t} = 0
\]
于是,
\[
\Gamma\left(x+1\right)=
\int^{\infty}_{0}e^{-t}t^{x}\,\mathrm{d}t
=x \int_0^\infty \mathrm{e}^{-t} t ^{x - 1} {\rm{d}} t
=x\Gamma\left(x\right)
\]
当$x$为整数的时候,运用数学归纳法即可得证。\\
(2).
设$\text{I}=\int_0^{\infty } e^{-x^2} \, \mathrm{d}x$(这货叫泊松积分,证明方法貌似很多很多),则
$$\text{I}^2=\iint_{\text{第一象限}} e^{-(x^2+y^2)} \,\mathrm{d}x\mathrm{d}y$$
又$\mathrm{d}x\mathrm{d}y=r\mathrm{d}r\mathrm{d}\theta$和$r^2=x^2+y^2$
可以求得(即所谓“直角坐标”转换为“极坐标”,注意第一象限上$r\in[0,\infty]$,\ $\theta\in[0,\pi/2]$),
$$
\text{I}^2
=\int^{\pi/2}_0 \int^{\infty}_0 r e^{-r^2} \,\mathrm{d}r\mathrm{d}\theta
=\frac{\pi}{4} \int^{\infty}_0 e^{-r^2} \,\mathrm{d}r^2=\frac{\pi}{4} 
$$
所以
$$\int_0^{\infty } e^{-x^2} \, dx=\frac{\sqrt{\pi}}{2}$$
又
\[
\Gamma\left(\frac{1}{2}\right)=
\int^{\infty}_{0}e^{-t}t^{-1/2}\,\mathrm{d}t
\]
令$t=x^2$,并且$\mathrm{d}t=2x\mathrm{d}x$,所以
\[
\Gamma\left(\frac{1}{2}\right)=
2\int^{\infty}_{0}e^{-x^2}\,\mathrm{d}x=\sqrt{\pi}
\]
(3).和第二小题一样,可以把第二类欧拉积分改写一下,
\[
\Gamma\left(p\right)=
2\int^{\infty}_{0}e^{-x^2}x^{2p-1}\,\mathrm{d}x
\]
注意到$x=r\cos\theta,\ y=r\sin\theta$,则,
\[
\begin{split}
\Gamma\left(p\right)\Gamma\left(q\right)=&
4\int^{\infty}_{0}e^{-y^2}y^{2p-1}\int^{\infty}_{0}e^{-x^2}x^{2q-1}\,\mathrm{d}x\mathrm{d}y \\
=& 4\iint_{\text{第一象限}} e^{-(x^2+y^2)}y^{2p-1}x^{2q-1}\,\mathrm{d}x\mathrm{d}y\\
=& 4\iint_{\text{第一象限}} re^{-r^2}y^{2p-1}x^{2q-1}\,\mathrm{d}r\mathrm{d}\theta\\
=& 4\iint_{\text{第一象限}} e^{-r^2}r^{2p+2q-1}\sin^{2p-1}\theta\cos^{2q-1}\theta \,\mathrm{d}r\mathrm{d}\theta\\
=& 2\left(2\int_{0}^{\infty} e^{-r^2}r^{2p+2q-1}\,\mathrm{d}r\right)\int_{0}^{\pi/2}\sin^{2p-1}\theta\cos^{2q-1}\theta \,\mathrm{d}\theta\\
=& 2\Gamma\left(p+q\right)\int_{0}^{\pi/2}\sin^{2p-1}\theta\cos^{2q-1}\theta \,\mathrm{d}\theta\\
\end{split}
\]
令$t=\sin^2\theta$,$\mathrm{d}t=2\cos\theta\sin\theta\mathrm{d}\theta$,则
\[
\begin{split}
\Gamma\left(p\right)\Gamma\left(q\right)
=& 2\Gamma\left(p+q\right)\int_{0}^{\pi/2}\sin^{2p-1}\theta\cos^{2q-1}\theta \,\mathrm{d}\theta\\
=& \Gamma\left(p+q\right)
\int_{0}^{1}\sin^{2p-2}\theta\cos^{2q-2}\theta \,\mathrm{d}t\\
=& \Gamma\left(p+q\right)
\int_{0}^{1}t^{p-1}(1-t)^{q-1}\,\mathrm{d}t\\
=& \Gamma\left(p+q\right) \mathrm{B}(p,q)
\end{split}
\]
\end{document}
