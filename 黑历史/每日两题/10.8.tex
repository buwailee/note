\documentclass[11pt,a4paper,openany]{article}
\usepackage{amssymb,amsfonts,
amsmath,ctex,bm}

\begin{document}
\noindent 1.如果$k$是自然数,且
$S_k(n)=1^k+2^k+\cdots+n^k$\\
试证明:\\
(1).
\[
S_k(n)=a_{k+1}n^{k+1}+\cdots+a_1 n+a_0
\]
(2).
\[
\lim_{n\rightarrow\infty}
\frac{S_k(n)}{n^{k+1}}
=\frac{1}{k+1}
\]
\\
\noindent 2.
函数
\[R(x)=\left\{
\begin{array}{ll}
\frac{1}{n}&\text{若}x=\frac{m}{n}\in \mathbb{Q}(\frac{m}{n}\text{是既约分数})\\
0&\text{若}x \text{为无理数}
\end{array}
\right .
\]
试证明:$R(x)$在所有无理点连续.
\newpage
解答:\\
1.注意到,
\[
i^{k}-(i-1)^{k+1}
=\sum_{j=0}^{k}{k \choose j}(i-1)^{j}
\]
那么,
\[
\begin{split}
n^{k+1}=\sum^{n}_{i=1}\left(i^{k+1}-(i-1)^{k+1}\right)
=&\sum^{n}_{i=1}\sum_{j=0}^{k}{k +1\choose j}(i-1)^{j}\\
=&\sum_{j=0}^{k}{k+1 \choose j}\sum^{n}_{i=1}(i-1)^{j}\\
=&\sum_{j=0}^{k-1}{k+1 \choose j}\sum^{n}_{i=1}(i-1)^j
+(k+1)\sum^{n}_{i=1}(i-1)^{k}
\end{split}
\]
这就是说,
\[
\begin{split}
\sum^{n}_{i=1}(i-1)^{k}&=1^k+2^k
+\cdots+(n-1)^k\\
&=
\frac{1}{k+1}\left(n^{k+1}+\sum_{j=0}^{k-1}{k+1 \choose j}\sum^{n}_{i=1}(i-1)^j\right)
\end{split}
\]
后面系数不必写出具体形式,再将左右各加上$n^k$,则有,
\[
S_k(n)=\frac{n^{k+1}}{k+1}+k\text{次多项式}
\]
那么极限就是显然的了。
\\
2.对于任何点$a$,他的有界邻域内,以及无论怎么的有理数$N$,在邻域中都仅有有限个有理数$\displaystyle{\frac{m}{n}}$使得$n<N$.\\
\indent 因此,只要缩小领域,就可以认为位于其中的一切有理数的分母都比$N$大(可能除去$a$)。这样一来,在邻域内的任意点$x$处,$\displaystyle{|R(x)|<\frac{1}{N}}$\\
\indent 这样就证明了,在任意点$a$处,
\[
\lim_{x\rightarrow a}R(x)=0
\]
于是,黎曼函数在所有无理点连续。
\end{document}