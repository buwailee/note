\documentclass[11pt,a4paper,openany]{article} \usepackage{amssymb,amsfonts, amsmath,ctex,bm}
\begin{document}
\noindent 1.如果$f$在$\mathbb{R}$上无限次可导。试证明,当$x \neq 0$的时候
\[
\frac{1}{x^{n+1}}f^{(n)}\left(\frac{1}{x}\right)=(-1)^n\left[x^{n-1}f\left(\frac{1}{x}\right)\right]^{(n)}
\]
其中上角标的$(n)$是指对这个函数求$n$阶导数。\\
2.这是一个有趣的命题。设$A$是有数码1和数码0构造的一切序列的集合。证明这个集合是不可数的。$A$的元素就是像$\{0,1,0,0,1,1,\cdots\}$这样的序列。(显然的推论就是一切二进制数的集合是不可数的)\\
\\
解答:\\
1.这题可以用数学归纳法来做。首先$n=1$的时候是正确的,这就不检验了。假设当$n=k$的时候,命题正确。那么就是已知
\begin{equation}
\frac{1}{x^{k+1}}f^{(k)}\left(\frac{1}{x}\right)=(-1)^k\left[x^{k-1}f\left(\frac{1}{x}\right)\right]^{(k)}
\end{equation}
求证,
\begin{equation}
\frac{1}{x^{k+2}}f^{(k+1)}\left(\frac{1}{x}\right)=(-1)^{k+1}\left[x^{k}f\left(\frac{1}{x}\right)\right]^{(k+1)}
\end{equation}
将式$(1)$两边乘以$x$(这步比较有技巧性,乘以的原因是,接下来对左边求导的话,就会出现式$(1)$和式$(2)$左边的项),
\begin{equation}
\frac{1}{x^{k}}f^{(k)}\left(\frac{1}{x}\right)=(-1)^kx\left[x^{k-1}f\left(\frac{1}{x}\right)\right]^{(k)}
\end{equation}
将式$(3)$两边求导,可得,
\begin{equation}
\begin{split}
\frac{-k}{x^{k+1}}f^{(k)}\left(\frac{1}{x}\right)
-\frac{1}{x^{k+2}}&f^{(k+1)}\left(\frac{1}{x}\right)=\\
&(-1)^k\left(\left[x^{k-1}f\left(\frac{1}{x}\right)\right]^{(k)}+
x\left[x^{k-1}f\left(\frac{1}{x}\right)\right]^{(k+1)}\right)
\end{split}
\end{equation}
代入我们假设正确的式$(1)$,然后可以整理得,
\begin{equation}
\begin{split}
\frac{1}{x^{k+2}}f^{(k+1)}\left(\frac{1}{x}\right)
=
(-1)^{k+1}(k+1)&\left[x^{k-1}f\left(\frac{1}{x}\right)\right]^{(k)}\\
&+(-1)^{k+1}x\left[x^{k-1}f\left(\frac{1}{x}\right)\right]^{(k+1)}
\end{split}
\end{equation}
那么剩下的就是要计算一下式$(5)$右边的式子,毫无疑问,右边第二项在外面的一个$x$是比较讨厌的。所以要凑出这种形式,我们可以先计算$式(2)$的右边,
\begin{equation}
\left[x^kf\left(\frac{1}{x}\right)\right]^{(k+1)}
=
\left[\left[xx^{k-1}f\left(\frac{1}{x}\right)\right]'\right]^{(k)}
\end{equation}
将右边的式子里面先求导一次,得
\begin{equation}
\left[x^kf\left(\frac{1}{x}\right)\right]^{(k+1)}
=\left[x^{k-1}f\left(\frac{1}{x}\right)\right]^{(k)}
+\left[x\left[x^{k-1}f\left(\frac{1}{x}\right)\right]'\right]^{(k)}
\end{equation}
右边的第二个式子用莱布尼兹公式打开,
\begin{equation}
\begin{split}
\left[x\left[x^{k-1}f\left(\frac{1}{x}\right)\right]'\right]^{(k)}
=&
\sum_{i=0}^{k}{k \choose i}x^{(k-i)}\left[x^{k-1}f\left(\frac{1}{x}\right)\right]^{(i+1)}\\
=&
x\left[x^{k-1}f\left(\frac{1}{x}\right)\right]^{(k+1)}
+
k\left[x^{k-1}f\left(\frac{1}{x}\right)\right]^{(k)}
\end{split}
\end{equation}
将式$(8)$代入到式$(7)$,可以得到
\begin{equation}
x\left[x^{k-1}f\left(\frac{1}{x}\right)\right]^{(k+1)}
=
\left[x^kf\left(\frac{1}{x}\right)\right]^{(k+1)}
-(k+1)\left[x^{k-1}f\left(\frac{1}{x}\right)\right]^{(k)}
\end{equation}
那么,式$(9)$就是式$(5)$中很难求得的那项。将其代入式$(5)$,有
\begin{equation}
\frac{1}{x^{k+2}}f^{(k+1)}\left(\frac{1}{x}\right)
=
(-1)^{k+1}\left[x^kf\left(\frac{1}{x}\right)\right]^{(k+1)}
\end{equation}
这也就是式$(2)$,那么命题得证。\\
2.相对于上面一题,这题篇幅很少。中心思想就是证明$A$的所有可数子集都是他的真子集。如果$A$可数,则他等于他的一个真子集,显然这是不可能的,于是可以得到$A$不可数。

\indent 设$E$是$A$的一个可数子集,且设$E$是由一列元素$s_1,s_2,s_3,\cdots$组成,现在构造一个序列$s$如下。\\
\indent 如果在$s_n$中第$n$个数码是1,就令$s$的第$n$个数码为0,如果$s_n$的第$n$个数码为0,就令$s$的第$n$个数码取1.于是序列$s$与$E$的每一个序列至少有一位不同。从而$s\notin E$,但是显然$s \in A$,所以$E$是$A$的真子集。因此$A$是不可数集。
\end{document}