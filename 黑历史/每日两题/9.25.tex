\documentclass[11pt,a4paper,openany]{article}
\usepackage{amssymb}
\usepackage{amsfonts}
\usepackage{amsmath}
\usepackage{ctex}
\usepackage{bm}

\begin{document}
\noindent 1.这是刘寸土同学问我的一个题目,由于上课讲了数学归纳法,于是就发一个数学归纳的题吧。\\
\indent 试证明对大于等于1的整数$n$,有
\[
n!>\left(\frac{n+1}{3}\right)^n
\]
\\
2.算的题目太多,大家都不喜欢,不如放一个证明题。\\
\indent 当一个集合的个数有限时,比较两个集合的个数多少可以直接比较。但当集合元素的个数是无限的时候,个数无法得到了,因此无法比较元素的多少。但我们有时候也要去比较两个无限集谁的元素多。比如实数集和有理数集哪一个元素多?\\
\indent 所以人类引入势的概念。有限集的势等于这个集合元素的个数。而对于两个集合势的比较,则依靠于两个集合间的关系,比如一一映射。\\
\indent 我们定义,当两个集合的势相同的时候,则称两个集合是等势的。如果一个集合和另一个集合之间存在一一对应关系,即$X$的元素$x$和$Y$的元素$y$同时满足$f(x)=y$和$f^{-1}(y)=x$,那么这两个集合是等势的。而当集合$X$和$Y$的某个子集等势的时候,则$X$的势小于等于$Y$的势。\\
\indent (1).试证明$(0,1)$和$\mathbb{R}$等势。\\
\indent (2).试证明任何构成有界闭合不自交(即不像双扭线那样,自己和自己有交点)曲线的点集是等势的。\\
\indent (3).如果用$P(X)$记所有$X$的子集构成的集合。试证明,$P(X)$的势大于$X$的势。
\newpage

解答:\\1.
我们先证明一个普遍的结论,即对正整数n,有
\[\left(1+\frac{1}{n}\right)^n<3.\]
先二项式展开,
\[
\begin{split}
\left(1+\frac{1}{n}\right)^n
&=\sum_{i=0}^n
\frac{n!}{(n-i)!i!}\frac{1}{n^i}\\
&=\sum_{i=0}^n
\frac{1}{i!}
\left(1-\frac{i-1}{n}\right)
\left(1-\frac{i-2}{n}\right)
\cdots
\left(1-\frac{i-i}{n}\right)\\
&<\sum_{i=0}^n
\frac{1}{i!}\\
&<2+\sum_{i=2}^n
\frac{1}{i(i-1)}=3-\frac{1}{n}<3.
\end{split}
\]
好,现在证明原结论。
首先$n=1$的时候显然,然后假设$n=k$的时候成立,让我们来看$n=k+1$的情况。
\[
(k+1)!=(k+1)k!>(k+1)\left(\frac{k+1}{3}\right)^k
\]
那么只要证明下式成立即可。
\[
(k+1)\left(\frac{k+1}{3}\right)^k \geq
\left(\frac{k+2}{3}\right)^{k+1}
\]
稍稍整理一下,就是证明:
\[
\left(1+\frac{1}{k+1}\right)^{k+1}\leq 3
\]
这刚刚已经证过,是成立的。所以根据数学归纳法,原不等式成立。
\\
2.\\
(1).很简单,对$(0,1)$的任意一点$x$,都有一一映射$\displaystyle{f(x)=\tan \left(\pi\left(x-\frac{1}{2}\right)\right)}$\\
(2).虽然没必要,但毕竟比较直观,我先证明,一个圆和一个闭合曲线的情况。设这个圆的半径比这个闭合曲线上到原点的最大距离还要大一点。从原点做射线,可以交闭合曲线和圆上各一点,这是一个一一映射。所以成立。由于圆也是闭合曲线,所以任意两个圆之间也成立。最后由于一一映射是可以传递的,所有构成任何构成有界闭合不自交(即不像双扭线那样,自己和自己有交点)曲线的点集是等势的。\\
(3).这是一个有趣的定理,它属于集合论的创始者康托尔。\\
证:对于空集来说,上述结论显然成立,所以可设$X\neq\varnothing$。
因为$P(X)$含有$X$的一切单元素子集,故$\text{card}X \leq \text{card}P(X)$,
现只需证明两者不相等。若相等,假定$f:X\rightarrow P(X)$是双射,
考察集合$A=\{  x \in X|x\notin f(x) \}$,
它由那样一些元素$x\in X$,$x$不含于它对应的集$f(x)\in P(X)$组成的。
因为$A\in P(X)$,所以必能找到一个元素$a\in X$,使$f(a)=A$,这个元素$a\in X$既不能有$a\in A$(据$A$的定义),也不能有$a\notin A$(也是根据$A$的定义)。得证。\\

\[
\begin{split}
[f,g]v_i&=fgv_i-gfv_i=fv_i=f_iv_i\\
&=fv_jg^j_i-f_igv_i\\
&=fv_jg^j_i-f_iv_jg^j_i\\
&=(f_j-f_i)v_jg^j_i\\
\end{split}
\]
这里$g^j_i$是$g$对基$\{v_i\}$的矩阵,比较$v_i$的系数,有
\[
f_iv_i=0
\]
所以$f_i=0$,因为$v_i$是任取的,所以所有的$f_i$都为0.
\end{document}