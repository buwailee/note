\documentclass[11pt,a4paper,openany]{article} \usepackage{amssymb,amsfonts, amsmath,ctex,bm}

\begin{document}
\noindent 1.还记得紧集吗?(见10/9解答)一个开集的并可以覆盖一个集合,如果从这些开集中可以挑选出有限个开集,他们的并依旧可以覆盖这个集合。那么这个集合就被称为紧集。试证明:\\
(1).紧集必然闭。\\
(2).紧集的闭子集是紧集。\\
2.$\mathbb{R}^n$是所有$n$元有序实数组$(a_{1},a_{2},\dots,a_{n})$的集合,而$(a_{1},a_{2},\dots,a_{n})$简记为$\bm a$。如果$\mathbb{R}^n$中的定义一个函数$|\cdot|$,其作用在两个$\mathbb{R}^n$中的点上$\bm x,\bm y$的时候,
$$
|\bm{x}-\bm{y}|=\sqrt{(x_{1}-y_{1})^{2}+(x_{2}-y_{2})^{2}+\cdots+(x_{n}-y_{n})^{2}}
$$
试证明:\\
(1).$|\cdot|$是$\mathbb{R}^n$的一个度量。\\
(2).$\mathbb{R}^n$中的有界闭集是紧集。
(这就是$\mathbb{R}^n$中的的Heine-Borel定理的一部分。10/9已经给出了$\mathbb{R}^1$上的形式了)

\newpage
\noindent 1.\\
(1).设$K$是一个紧集,只要证明$K$的补集是一个开集就可以了。\\
\indent 取一点$p$在$K$的补集里。再取一点$q$在$K$里面,令$V_p$和$W_q$分别是$p$和$q$的邻域,他们的半径小于$\displaystyle{\frac{1}{2}d(p,q)}$.因为$K$是紧集,所以$K$中有有限多个点$q_1,q_2,\cdots,q_n$,使得,

\[
K \subset W_{q_1}\cup W_{q_2} \cup \cdots \cup W_{q_n}=W
\]
\indent 如果令$V=V_{P_1}\cap V_{P_2} \cap \cdots \cap V_{P_n}$,那么$V$就是$p$的一个和$W$不相交的邻域。因此$V$在$K$的补集里面,所以$p$是$K$的补集的内点。证毕。\\
(2).设$F\subset K\subset X$,其中$F$是闭的,$K$是紧的,而$X$是一个度量空间。(值得一提的是,闭集是相对度量空间而言的,相同的集合,在不同的度量空间下是可能有着不同的开闭情况。但是紧集却不会,就是说,无论安置在哪个空间里,紧集永远是紧的。这个命题很有趣,有兴趣可以尝试证明看看。)\\
\indent 取$F$的一个开覆盖$\{V_a\}$,然后把$F$的补集加进去(开的),他构成了一个$K$的开覆盖,由于$K$是紧集,所以可以从中挑出有限的开集依旧覆盖$K$。如果$F$的补集不在这些有限开覆盖里面,则这些开覆盖也覆盖$F$,则命题证毕。如果在里面,把他去掉,则剩下的那些有限个开集的并依旧可以覆盖$F$。命题证毕。\\
2.\\
(1).略,Cauthy不等式即可。\\
(2).我不想给出完整的证明。因为这个证明完全和10/9的第二题的证明是类似的。大致上分两步:\\
($\alpha$).先定义一个$k$-方格$I_n$,其中的点$I_n \ni \bm{x}=(x_1,x_2,\cdots,x_k)$满足,$x_i \in [a_{n,i},b_{n,i}]$.而且满足$I_1\supset I_2 \supset \cdots \supset I_n \supset \cdots$.可以证明,在所有$\{I_n\}$中一定有一点$\bm{x}$.(可见9/27的第二题)\\
($\beta$).一个有界闭集一定在一个$k$-方格里面,那么只要证明一个有界闭的$k$-方格是紧集就可以了(见10/9第二题)。因为紧集的闭子集也是紧集(上一题第二小题)。命题就证明完毕了。
\end{document}