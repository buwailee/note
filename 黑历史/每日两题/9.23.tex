\documentclass[11pt,a4paper,openany]{article}
\usepackage{amssymb}
\usepackage{amsfonts}
\usepackage{amsmath}
\usepackage{ctex}
\usepackage{bm}

\begin{document}
上次积分不太接地气,所以换几道看起来比较接地气的题目,\\
1.试求:
\[
\int_{0}^{1}\frac{\ln (1+x)}{1+x^2}\,\mathrm{d}x.
\]
2.一个$\pi$的近似公式。\\
(1).首先规定($k$为整数)
$$(2k)!!=2\cdot4\cdot6\cdots\cdot(2k-2)\cdot(2k),
$$
$$(2k-1)!!=1\cdot3\cdot5\cdots\cdot(2k-3)\cdot(2k-1).
$$
求证($n$为整数):
$$ 
\int_0^{\pi/2}\sin^n x\,\mathrm{d}x
=
\left\{
\begin{array}{rcl}
&((n-1)!!/n!!)(\pi/2) &n\text{为偶数,}\\
&(n-1)!!/n!! &n\text{为奇数.}\\
\end{array}
\right .
$$
(2).当$x\in(0,\pi/2)$时,显然有,
\[
\sin^{2n+1}x < \sin^{2n}x <\sin^{2n-1}x.
\]
试用以上关系推导出,
\[
\frac{\pi}{2}=\lim_{n\rightarrow\infty}\left[\frac{2n!!}{(2n-1)!!}\right]^2
\frac{1}{2n+1}
.\]

\newpage
解答:\\
1.我错了,这题超恶心,花了我一节微积分课 ,原谅我这个大弱 = =
\[
J=\int^1_0 \frac{\ln(1+x)}{1+x^2}\,\mathrm{d}x
\]
换元$x=\tan t$,则,
\[
J=\int^{\pi/4}_0 \ln(1+\tan t)\,\mathrm{d}t
\]
由于这个恒等式成立,
\[
1+\tan t=\sqrt{2}\sin(\frac{\pi}{4}+t)/\cos t
\]
所以,
\[
J=\int^{\pi/4}_0 
\ln\sqrt{2}
\,\mathrm{d}t
+
\int^{\pi/4}_0 
\ln(\sin(\frac{\pi}{4}+t))
\,\mathrm{d}t
-
\int^{\pi/4}_0 
\ln(\cos t)
\,\mathrm{d}t
\]
分项积分,第一项有,
\[
\int^{\pi/4}_0 
\ln\sqrt{2}
\,\mathrm{d}t
=\frac{\pi \ln \sqrt{2}}{4}
\]
令$u=\pi/4-t$,则第二项为,
\[
-\int^{0}_{-\pi/4}
\ln(\sin(\frac{\pi}{2}-u))
\,\mathrm{d}u
=
\int^{\pi/4}_{0}
\ln(\cos u)
\,\mathrm{d}u
\]
所以第二项和第三项相等,之差为0.故而,
\[
J
=\frac{\pi \ln \sqrt{2}}{4}
=\frac{\pi \ln 2}{8}
\]
2.\\
(1).先用分部积分推导递推式:
\[
\int^{\pi/2}_{0}\sin^{n}x\,\mathrm{d}x
=\int^{\pi/2}_{0}\cos x\,\mathrm{d}\sin^{n-1}x
=(n-1)\int^{\pi/2}_{0}(\sin^{n-2}x-\sin^n x)\,\mathrm{d}x
\]
所以可以解出递推关系,
\[
\int^{\pi/2}_{0}\sin^{n}x\,\mathrm{d}x
=\frac{n-1}{n}\int^{\pi/2}_{0}
\sin^{n-2}x\,\mathrm{d}x
\]
其余步骤略。\\
(2).对不等式积分,
\[
\int^{\pi/2}_{0} \sin^{2n+1}x\,\mathrm{d}x <\int^{\pi/2}_{0} \sin^{2n}x \,\mathrm{d}x<\int^{\pi/2}_{0} \sin^{2n-1}x\,\mathrm{d}x
\]
\[
\frac{(2n)!!}{(2n+1)!!}
<
\frac{(2n-1)!!}{(2n)!!}\frac{\pi}{2}
<
\frac{(2n-2)!!}{(2n-1)!!}
\]
等价于
\[
\left[\frac{(2n)!!}{(2n-1)!!}
\right]^2\frac{1}{2n+1}
<
\frac{\pi}{2}
<
\left[\frac{(2n)!!}{(2n-1)!!}
\right]^2\frac{1}{2n}
\]
最左边和最右边的差为,
\[
\frac{1}{2n(2n+1)}
\left[\frac{(2n)!!}{(2n-1)!!}
\right]^2
<
\frac{1}{2n}\frac{\pi}{2}
\]
所以当$n\rightarrow\infty$的时候,不等式左右两边的差趋向于零,所以说左右两边有着相同的极限,根据夹逼,这个极限是$\displaystyle{\frac{\pi}{2}}$.于是,
\[
\lim_{n\rightarrow\infty}\left[\frac{(2n)!!}{(2n-1)!!}
\right]^2\frac{1}{2n+1}
=\frac{\pi}{2}
\]
此即所求。\\
附:经过计算机的计算,当$n=500$时,
\[
2\left[\frac{1000!!}{999!!}\right]^2
\frac{1}{1001}
\approx 3.14002
\]
可见收敛速度相当慢。
\end{document}
