\documentclass[11pt,a4paper,openany]{article}
\usepackage{amssymb,amsfonts,
amsmath,ctex,bm}

\begin{document}
解答:\\
1.$f(x)$显然单调不减,所以若$f(x)<x$,则
\[
f^n(x)\le f^{n-1}(x) \le f(x)<x
\]
同理,若$f(x)>x$,则,
\[
f^n(x)\ge f^{n-1}(x) \ge f(x)>x
\]
若$f(x)\neq x$,则上面两个式子告诉我们,$f^n(x)\neq x$。所以一旦有一个$n$使得$f^n(x)=x$,则$f(x)\equiv x$\\
2.显然$n=1$成立。$[0,1]$上的连续函数存在的最大值$f(a)$,使得$f(a)\ge f(a+1/n)$,以及$f(a-1/n)\ge f(a)$,所以在$[a-1/n,a]$(我们称为区间1)中存在使得$f(x)= f(x+1/n)$的$x$,同理对$f(x-1/n)$和$f(x)$也做类似讨论,则在$[a,a+1/n]$(我们称为区间2)中也存在。若$0<a<1/n$,则使用区间1,若$1-1/n<a<1$,则使用区间2,其余两者皆可。总之,可以挑出长度为$1/n$的水平线段,这就是说,存在$x$满足,$f(x)= f(x+1/n)$。而第二问的反例也比较简单,取$f(x)=\sin{2\pi x}$,则$[0,1]$上取不到使得$f(x+2/3)=f(x)$成立的$x$.
\end{document}