\documentclass[11pt,a4paper,openany]{article}
\usepackage{amssymb}
\usepackage{amsfonts}
\usepackage{amsmath}
\usepackage{ctex}
\usepackage{bm}

\begin{document}
\noindent 1.刘寸土同学很勤奋。今天他又问了我一个问题。我以为难度不高,但很亲民接地气,所以用来缓解一下前几天太难的气氛。\\
试证明,对一切实数$a$有,
\[
\lim_{n\rightarrow \infty}
\frac{a^n}{n!}=0
\]

\noindent 2.
如果$|q|<1$,\\
(1).试用极限定义证明
\[
\lim_{n\rightarrow\infty} n^2 q^n=0
\]
(2).不用定义,换一种方法证明上式。
\newpage
解答:\\
1.很简单,设$\displaystyle{x_n=\frac{a^n}{n!}}$,然后\\
\[
|x_{n+1}|
=\frac{|a|}{n+1}|x_n|
\]
当$n$足够大的时候,即对一切$n>N$,满足$|a|<n+1$,所以在$n>N$的时候,数列$|x_n|$单调递减。因为$|x_n|$下有界且在有限项后单调递减,所以这个数列是有极限的。用极限的乘法法则,可以得到:
\[
A=\lim_{n\rightarrow \infty}
|x_{n+1}|=
\lim_{n\rightarrow \infty}
\frac{|a|}{n+1}|x_n|
=\lim_{n\rightarrow \infty}
\frac{|a|}{n+1}\lim_{n\rightarrow \infty}|x_n|=0 \cdot A=0
\]
如果数列的绝对值收敛到0,那么他自然也收敛到0.得证。\\
\\
2.定义证明难度不低。但如果从这开始,也可以通过对比看出不用定义证明十分简单和易懂。\\
证:(1).设$x_n=n^2 q^n$,令$\displaystyle{a_n=\frac{x_{n+1}}{x_n}=\left(1+\frac{1}{n}\right)^2 q}$,取$q<t<1$.\\
可见
\[
\left(1+\frac{1}{M}\right)^2 q<1
\Leftrightarrow
M>\frac{\sqrt{q}}{\sqrt{t}-\sqrt{q}}
\]
固定$M$,根据$a_n$的定义,对任意的$n>M$有
\[
x_{n+1}=x_M(a_{M+1}\cdots a_n)<x_M t^{n-M}
\]
于是
\[
\begin{split}
&
\forall \varepsilon >0 \exists N>\left(\frac{\ln \varepsilon-\ln x_M}{\ln t}-M\right) \forall n>N \left(0<x_{n+1}<x_M t^{n-M}<x_M t^{N-M}<\varepsilon \right)\\
&\Rightarrow
\lim_{n\rightarrow \infty}n^2 q^n=0
\end{split}
\]
(2).这个证明非常像第一题。可以证明,对足够大的$n$,$x_n$是单调递减的,所以他有极限,可以使用极限的乘法法则。于是解一个一元一次方程可得其极限为0.
\end{document}