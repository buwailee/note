\documentclass[11pt,a4paper,openany]{article}
\usepackage{amssymb}
\usepackage{amsfonts}
\usepackage{amsmath}
\usepackage{ctex}
\usepackage{bm}
\begin{document}
尝试着问一些不动点的问题。\\
1.\\
(1).试证明:任何连续映射$f:[0,1]\rightarrow[0,1]$都有不动点,即存在某点$x \in [0,1]$使得$f(x)=x$.\\
(2).试证明:若映射连续$f:[0,1]\rightarrow[0,1]$满足$f(0)=0$,$f(1)=1$且在$[0,1]$上满足$f(f(x))\equiv x$,则$f(x)\equiv x$.\\
(提示:$f(x)$是单调不减函数)\\
2.\\
试证明:如果一组开区间的并可以覆盖一个有界的闭区间,即这个闭区间包含于开区间的并,那么从这一组开区间的集合中可以挑出有限个开区间,他们的并可以覆盖原本那个闭区间。\\(提示:可以参考一下9/27的第二题的结论)
\newpage
解答:\\
1.构造函数$g(x)=f(x)-x$,显然$g(x)$在$[0,1]$上是连续的。由于$f(x)$的值域是$[0,1]$,所以一定有一点$a$,使得$f(a)=1>a$,或者说$g(a)>0$。同理一定有一点b,使得$g(b)<0$。因为$g(x)$在$[a,b]$上面连续,所以一定存在点$x\in [a,b]$使得$g(x)=0$,即$f(x)=x$.这就是所要证明的。\\
2.这就是著名的有限覆盖引理。满足有限覆盖引理的集合一般叫他紧集。换而言之,就个定理就是在说,在$\mathbb{R}$上的有界闭集是紧集。同时,凡是紧集,必是闭集。这两条就是著名的Heine-Borel定理。\\
证:\\
如果$S$是无限个开集的并,且不能从其中选出有限个开集,他们的并可以覆盖闭区间$I_1$.那么不妨将$I_1$对半分,那么至少其中一个也不能被覆盖,把他记作$I_2$.同样的手续可以无限做下去。那么就得到了这样的一个闭区间的序列。
\[
I_1\supset I_2 \supset \cdots \supset I_n \supset \cdots,
\]
其中必然有长度任意小的闭区间。根据9/27的第二题的结论,可以得到,必然有一点$c$,存在每一个闭区间之中。因为$c$在闭区间$I_1$中,则必然有一个$S$中的开区间$(a,b)$包含$c$,令$\varepsilon=\min{c-a,b-c}$,因为做出来的那一些闭区间中有长度任意小的闭区间,所以一定可以找到$n$,使得$I_n<\varepsilon$,因为$c\in I_n$,且$I_n<\varepsilon$,因此$I_n\subset (a,b)$。但是这和,闭区间不能由从$S$中选出有限个开区间覆盖相悖。这就得到了所要证的命题。
\end{document}