\documentclass[11pt,a4paper,openany]{article}
\usepackage{amssymb,amsfonts,
amsmath,ctex,bm}

\begin{document}
\noindent 1.继续上次的问题。\\
若$P_n(x)$为$n$阶多项式。我们用多项式来逼近有界函数$f:[a,b]\rightarrow \mathbb{R}$。设,
\[
\Delta(P_n)=\sup_{x\in [a,b]}|f(x)-P_n(x)|
\text{以及}
E_n(f)=\inf{P_n}\Delta(P_n)
\]
其中下界取遍一切可能的$n$阶多项式。如果$\displaystyle{E_n(f)=\inf_{P_n}\Delta(P_n)}$,则多项式$P_n$叫做函数$f$的 {\kaishu 最佳逼近多项式}。
试证明:\\
(1).存在0阶最佳逼近多项式,并且尝试写出来。\\
(2).如果$n$阶最佳逼近多项式存在,则$n+1$阶也存在。\\
(3).对任意的$n$,最佳逼近多项式总是存在的。\\
2.引进切比雪夫多项式。\\
在$x\in [-1,1]$上定义切比雪夫多项式,
\[
T_n(x)=\arccos{(\cos{(x)})}
\]
(1).证明$T_n$为n阶多项式,且$x^n$项为1。\\
(2).证明对于任意的$x^n$项为1的多项式$P_n(x)$,一定满足,
\[
\max_{x\in[-1,1]}|P_n(x)|\geq 1
\]
\newpage
\noindent 1.\\
(1).如果$f(x)$在$[a,b]$上的最大值是$M$,最小值是$m$,则其零阶最佳逼近多项式为
\[
P_0=\frac{M+m}{2}
\]
(2).
不妨令,
\[
\Delta(P_{n+1})
=\Delta(P_n,\lambda)
=\sup_{x\in [a,b]}|f(x)-P_n(x)-\lambda x^n|
\]
像10/11里面的证明一样,可以证明,对确定的$P_n$存在使得$\Delta(P_n,\lambda)$达到最小值的$\lambda$。同时,
\[
E_{n+1}(f)=\inf_{P_{n+1}}\Delta(P_{n+1})
=\min_{\lambda}\inf_{P_n}\Delta(P_n,\lambda)
\]
根据假设,$\inf\Delta(P_n)$也存在。所以可以看到,如果$n$阶最佳逼近多项式存在,则$n+1$阶也存在。\\
(3).数学归纳法\\
2.这题我题目写错了,2阶以上的$T_n$首项都不是1.所以第二小题应该改成首项为$2^{n-1}$的多项式。\\
(1).令$t=\arccos x$,则显然$\cos t=x$且
\[
\frac{\sin{(nt)}}{\sin t}
=\sum_{k=0}^{n-1}\cos((n-1-k)t)\cos^k t
\]
这个式子证明用数学归纳法就可以了,略。所以,
\[
\sin{(nt)}\sin t
=(1-\cos^2 t)\sum_{k=0}^{n-1}\cos((n-1-k)t)\cos^k t
\]
假设$k$从1到$n$,$cos(kt)$为$k$次多项式,则$\sin{(nt)}\sin t$必然是$n+1$次多项式,那么
\[
\cos((n+1)t)=\cos{(nt)} \cos t-\sin{(nt)} \sin t
\]
显然也是$n+1$次多项式,根据数学归纳法,可知命题成立。\\
(2).这个很简单说穿了很简答$\max |T_n|=1$,显然,所以只要和$T_n$比就可以了。如果$|P_n|$在每个点都比1小,那么$P_n-T_n$是一个$n-1$次多项式。因为$|T_n|$会有$n+1$个最大值点,所以$P_n-T_n$有$n$个零点。但是$n-1$阶多项式为0最多有$n-1$个根,相悖。所以$\max |P_n|\geq 1$.
\end{document}
