\documentclass[11pt,a4paper,openany]{article}
\usepackage{amssymb,amsfonts,
amsmath,ctex,bm}

\begin{document}
\noindent 这次来点点集拓扑吧。\\
0.定义:如果一个空间$X$上定义了一个函数$d:X^2\rightarrow\mathbb{R}$(称之为距离或者度量),对于$p,q\in X$,距离函数满足:\\
(a).如果$p\neq q$,那么$d(p,q)>0$;$d(p,p)=0$\\
(b).$d(p,q)=d(q,p)$\\
(c).对任意的$r\in X$有,$d(p,r)+d(r,q)\ge d(p,q)$\\
那么我们就称这个空间$X$为一个度量空间。\\
假定下面的讨论都在度量空间$X$中。再定义:\\
(0.1).点$p$的邻域$N_r(p)$是指满足$d(p,q)<r$的一切点$q$的集合,$r$称为这个邻域的半径。\\
(0.2).点$p$叫做点集$E$的极限点,如果$p$的每个邻域内都含有一点$q\in E$,但$q \neq p$.如果一个点不是极限点,就叫他孤立点。\\
(0.3).如果点集$E$的每个点都是他的极限点,则称$E$为闭的,$E$是一个闭集。\\
(0.4).点$p$称为$E$的一个内点,如果存在一个$p$的邻域$N$,有$N\subset E$\\
(0.5).如果$E$的每个点都是他的内点,则称$E$是一个开集,$E$是开的。\\
(0.6).如果存在一个实数$M$,和一个点$q \in X$,使得一切$p \in E$,都满足$d(p,q)<M$,则称E是有界的。\\
\\
1.试证明:如果$p$是$E$的一个极限点,则$p$的每个邻域有$E$的无数个点。\\
2.试证明:\\
(1).$E$是开集的充要条件是他的补集是闭集。\\
(2).任意开集的并开;任意闭集的交闭;有限开集的交开;有限闭集的并闭。
\newpage
\noindent 1.假设$p$有个邻域$N$只有$E$的有限个点,令$q_1,\cdots,q_n$是这些点,再选取一个$r$等于所有$d(p,q_i)$里面最小的那个。显然$r>0$。所以$p$以$r$为半径的那个邻域里面再也没有$E$的点了,但这和极限点的定义相悖,所以命题错误。\\
2.\\
(1).首先设$E$为开集,令$x$为$E$的补集中的一个极限点,那么$x$的每个领域都含有$E$的补集的点,所以$x$不是$E$的内点,因为$E$是开集,所以$x$在$E$的补集中。所以$E$的补集是闭集。\\
\indent 再设$E$的补集是闭集。取$E$中的一个点$x$,显然他不能是$E$的补集的极限点,于是$x$有一个邻域$N$不交于$E$的补集。这就是说$N$在$E$内,所以$x$是$E$的内点,从而$E$是开的。\\
(2).令$G_a$是一组开集,再令$G=\bigcup_a G_a$。如果$x\in G_a$,那么他一定在$G$里面。因为$x$是$G_a$的一个内点,所他也是$G$的内点,从而$G$是开集。\\
\indent 补集的交等于并集的补。所以令$F_a$是一组开集,再令$F=\bigcap_a F_a$。$F$的补集就是$F_a$们补集的并,同时根据第一题,$F_a$们的补集是开的。根据第一小题,$F_a$们补集的并也是开的,即$F$的补集是开的,所以$F$是闭的。\\
\indent 令$H=\bigcap_{a=1}^n G_a$,对于任意的$x\in H$,存在$x$的领域$N_i$,其半径为$r_i$,使得$N_i$在$G_i$里面。取$r$为$r_i$中最小的那个,因为有限,所以$r>0$。令$N$为以$r$为半径的邻域。于是对于所有的$i$,有$N$在$G_i$里面,从而$N$在$H$里面,所以$H$是开集。\\
\indent 上面这段取补集即可。
\end{document}