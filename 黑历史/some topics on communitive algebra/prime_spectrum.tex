%!TEX program = xelatex
\documentclass[11pt]{article}
\usepackage{amssymb,amsfonts,amsthm,amsmath,bm,microtype}
\usepackage[left=21mm,text={148mm,200mm},paperwidth=185mm,paperheight=250mm,includehead,vmarginratio=1:1]{geometry}
\usepackage[all]{xy}
\usepackage{tikz}
\usepackage{titletoc}%使用目录
	\theoremstyle{plain}%定理环境样式
	\newtheorem{pro}{Proposition}% 定义命题环境
	\newtheorem{theo}{Theorem}% 定义定理环境
	\newtheorem{lem}{Lemma}% 定义引理环境
	\newtheorem{defi}{Definition}% 定义定义环境

%定义你的命令
\newcommand{\dd}{{\mathrm{d}}}  % 微分号
\newcommand{\no}[1]{{$(#1)$}}  % 编号
\newcommand{\rr}{{\mathbb{R}}}  % 实数域
\newcommand{\cc}{{\mathbb{C}}}  % 复数域
\newcommand{\fr}[1]{{\mathfrak{#1}}}
\newcommand{\ad}{{\mathrm{ad}}}
\definecolor{shadecolor}{rgb}{0.92,0.92,0.92}
\newcommand{\re}[1]
	{\begin{center}
		\colorbox{shadecolor}{
			\begin{minipage}{135mm}
				\emph{''#1''}
			\end{minipage}}
	\end{center}}

\begin{document}
The main reference is the \textit{Introduction of Commutative Algebra} by Atiyah\&Macdonald (for short, I will call it A\&M from now on.). This article can be seem as the answers of some problems on the prime spectrum of A\&M. Throughout this article the word ``ring'' shall mean a commutative ring with an identity element.

\begin{pro}{\rm Ex 1.15}:

	Let $A$ be a ring and let $X$ be the set of all prime ideals of $A$. For each subset E of A, let $V(E)$ denote the set of all prime ideals of $A$ which contain $E$, then the sets $V(E)$ satisfy the axioms for closed sets in a topological space.
\end{pro}

\begin{proof}
	We shall prove this proposition in four parts:

	\no{0} if $\fr{a}$ is the ideal generated by $E$, then $V(E)=V(\fr{a})=V(r(\fr{a}))$.

	\no{1} $V(0)=X$ and $V(1)=\varnothing$.

	\no{2} if $(E_i)_{i\in I}$ is any family of subsets of $A$, then
	\[
		V\left(\bigcup_{i\in I}E_i\right)=\bigcap_{i\in I} V(E_i).
	\]

	\no{3} $V(\fr{a}\cap \fr{b})=V(\fr{ab})=V(\fr{a})\cup V(\fr{b})$ for any ideals $\fr{a}$, $\fr{b}$ of $A$.

	\vspace{1em}

	Firstly, let $\fr{p}$ is a prime ideal which contains $E$, then $\fr{a}\subseteq \fr{p}$ since $\forall a\in A$ we have $xa\in \fr{p}$, so $V(E)\subseteq V(\fr{a})$. Conversely, $V(\fr{a})\subseteq V(E)$ because $E\subseteq \fr{a}$. Using that the radical of an ideal $\fr{a}$ is the intersection of the prime ideals which contain $\fr{a}$, $V(\fr{a})=V(r(\fr{a}))$. Thus $V(E)=V(\fr{a})=V(r(\fr{a}))$.

	Secondly, every ideal containing $0$ implies $V(0)=X$, and $V(1)=V(A)=\varnothing$ is trivial by \no{0}.

	Next, an ideal contains $\cup_i E_i$ if and only if it contains each $E_i$.

	Finally, using $r(\fr{a}\cap \fr{b})=r(\fr{ab})$,
	\[
		V(\fr{a}\cap \fr{b})=V(r(\fr{a}\cap \fr{b}))=V(r(\fr{ab}))=V(\fr{ab}),
	\]

	We now should prove $V(\fr{a}\cap \fr{b})=V(\fr{a})\cup V(\fr{b})$.

	$V(\fr{a}\cap \fr{b})\subseteq V(\fr{a})\cup V(\fr{b})$: $\forall \fr{p}\in V(\fr{a})\cup V(\fr{b})$, $\fr{p}$ contains $\fr{a}$ or $\fr{b}$, thus $\fr{a}\cap \fr{b}\subseteq \fr{p}$ which is equivalent that $\fr{p}\in V(\fr{a}\cap \fr{b})$.

	$V(\fr{a}\cap \fr{b})\supseteq V(\fr{a})\cup V(\fr{b})$: $\forall \fr{p}\in V(\fr{a}\cap \fr{b})$, thus $\fr{a}\cap \fr{b}\subseteq \fr{p}$. If $\fr{a}\nsubseteq \fr{p}$ and $\fr{b}\nsubseteq \fr{p}$, there exist $x\in \fr{a}$, $y\in \fr{b}$ and $x,y \notin \fr{p}$, and therefore $xy\in \fr{a}\fr{b}\subseteq \fr{a}\cap \fr{b}$, but $xy \notin \fr{p}$ (since $\fr{p}$ is prime), which means $\fr{a}\cap \fr{b}\nsubseteq \fr{p}$. Hence if $\fr{a}\cap \fr{b}\subseteq \fr{p}$, $\fr{p}$ contains $\fr{a}$ or $\fr{b}$.
\end{proof}

The resulting topology is called the \textit{Zariski} topology.
\begin{defi}
	The topological space $X$ is called the prime spectrum of $A$, and is written $\mathrm{Spec}(A)$.
\end{defi}

\begin{pro}{\rm Ex 1.17}:

	For each $f\in A$, let $X_f$ denote the complement of $V(f)$ in $X=\mathrm{Spec}(A)$, i.e. $X_f=\mathrm{Spec}(A)-V(f)$. The sets $X_f$ are open and form a basis of open sets for the Zariski topology.
\end{pro}

\begin{proof}
	$X_f$ is open because $V(f)$ is closed. If $P$ is a open set, it has the form $\mathrm{Spec}(A)-V(E)$ for a $E$, then
	\[
		P=\mathrm{Spec}(A)-V(E)=\mathrm{Spec}(A)-\bigcap_{f\in E}V\left(f\right)=\bigcap_{f\in E} \left(\mathrm{Spec}(A)-V\left(f\right)\right)=\bigcap_{f\in E} X_f.
	\]
\end{proof}

\begin{pro}{\rm Ex 1.17}:

	\no{0} $X_f\cap X_g=X_{fg}$;

	\no{1} $X_f=\varnothing \, \Leftrightarrow\, f$ is nilpotent;

	\no{2} $X_f=X \, \Leftrightarrow\, f$ is a unit;

	\no{3} $X_f=X_g \, \Leftrightarrow\, r(f)=r(g)$;

	\no{4} $X$ is compact;

	\no{5} $X_f$ is compact;

	\no{6} An open subset of $X$ is compact if and only if it is a finite union of sets $X_f$.
\end{pro}

\begin{proof} The sets $X_f$ are often called basic open sets of $X=\mathrm{Spec}(A)$.

	\no{0} $X_f\cap X_g=\mathrm{Spec}(A)-(V(f)\cup V(g))=\mathrm{Spec}(A)-V(fg)=X_{fg}$;

	\no{1} $X_f=\varnothing \, \Leftrightarrow\, V(f)=V(r(fA))=X \, \Leftrightarrow\, r(f)=0$;

	\no{2} $X_f=X \, \Leftrightarrow\, V(f)=\varnothing \, \Leftrightarrow\, f$ is not in any maximal ideal $\Leftrightarrow\, f$ is a unit;

	\no{3} $X_f=X_g \, \Leftrightarrow\, V(f)=V(g) \, \Leftrightarrow\, r(f)=r(g)$;

	\no{4} It is enough to consider a covering of $X$ by basic open sets $X_{f_i}$ because $X=\bigcup_\alpha U_\alpha$ and $U_\alpha=\bigcup_\beta X_{f_{\alpha \beta}}$. Thus
	\[
		\varnothing=V(X)=V\left(\bigcup_i f_i\right)=\bigcap_i V\left(f_i\right),
	\]
	This means that $\{f_i\}$ generates $A$, so exist $\{g_i\}$ s.t.
	\[
		\sum_i f_ig_i=1
	\]
	with cofinitely many of the $i$ non-zero. Thus, $X$ is the union of the $X_{f_i}$ for which $g_i\neq 0$, so $X$ is the union of finitely many $U_\alpha$.

	\no{5} Suppose $X_f\subseteq \bigcup_i X_{f_i}$, then $\bigcap_i V(f_i)\subseteq V(f)$. Let $\fr{a}$ is the ideal generated by the $f_i$, then $f\in \fr{a}$, so there is an equation
	\[
		f^n=\sum_i f_ig_i
	\]
	with cofinitely many of the $i$ non-zero. Let $f_1,\dots,f_n$ with $g_i\neq 0$, then
	\[
		\bigcap_{i=1}^n V(f_i)\subseteq V(f^n)=V(f),
	\]
	so $X_f\subseteq \bigcup_{i=1}^n X_{f_i}$. Now we can say $X_f$ is compact.

	\no{6}
	The union of finitely many $X_f$ is open and compact. Conversely, suppose $U$ is compact and open, then $U$ is the union of some $X_f$ as an open cover, it must have a finite subcover because of compactness.
\end{proof}

\begin{defi}
	A topological space $X$ is said to be irreducible if $X\neq \varnothing$ and if every pair of non-empty open sets in $X$ intersect, or equivalently if every non-empty open set is dense in $X$.
\end{defi}

\begin{pro}{\rm Ex 1.19}:

	$\mathrm{Spec}(A)$ is irreducible if and only if the nilradical of $A$ is a prime ideal.
\end{pro}

\begin{proof}
	If not prime, there exist $f$ and $g$ s.t. $fg\in \fr{p}$ but $f\notin \fr{p}$ and $g\notin \fr{p}$, and
	\[
		X_f\neq \varnothing,X_g\neq \varnothing \,\Rightarrow \,X_f\cap X_g=X_{fg}=\varnothing,
	\]
	which means $X$ is not irreducible.

	Conversely, if not irreducible, there exist $X_f\subseteq U$, $X_g\subseteq V$ and $U\cap V=\varnothing$, thus $X_f \cap X_g=X_{fg}=\varnothing$ but neithor $f$ or $g$ is nilpotent.
\end{proof}
\end{document}
