% author: buwailee@nmhs
\documentclass[10pt]{article}
\usepackage[zh]{noteheader}
\usepackage{titletoc}%使用目录
	\theoremstyle{plain}%定理环境样式
	\newtheorem{pro}{Proposition}[section]% 定义命题环境
	\newtheorem{defi}{Definition}[section]% 定义命题环境
	\newtheorem{theo}{Theorem}[section]% 定义定理环境

\usepackage{hyperref}
	\hypersetup %一些选项
	{
		bookmarksnumbered=true,%书签中章节编号
		colorlinks=true,  % 彩色链接 false:边框链接 ; true: 彩色链接
	}

\definecolor{shadecolor}{rgb}{0.92,0.92,0.92}

\newcommand{\no}[1]{{$(#1)$}}
% \renewcommand{\not}[1]{#1\!\!\!/}
\newcommand{\rr}{\mathbb{R}}
\newcommand{\zz}{\mathbb{Z}}
\newcommand{\aaa}{\mathfrak{a}}
\newcommand{\pp}{\mathfrak{p}}
\newcommand{\mm}{\mathfrak{m}}
\newcommand{\dd}{\mathrm{d}}
\newcommand{\oo}{\mathcal{O}}
\newcommand{\calf}{\mathcal{F}}
\newcommand{\calg}{\mathcal{G}}
\newcommand{\bbp}{\mathbb{P}}
\newcommand{\bba}{\mathbb{A}}
\newcommand{\osub}{\underset{\mathrm{open}}{\subset}}
\newcommand{\csub}{\underset{\mathrm{closed}}{\subset}}

\DeclareMathOperator{\im}{Im}
\DeclareMathOperator{\Hom}{Hom}
\DeclareMathOperator{\id}{id}
\DeclareMathOperator{\rank}{rank}
\DeclareMathOperator{\tr}{tr}
\DeclareMathOperator{\supp}{supp}
\DeclareMathOperator{\coker}{coker}
\DeclareMathOperator{\codim}{codim}
\DeclareMathOperator{\height}{height}
\DeclareMathOperator{\sign}{sign}

\DeclareMathOperator{\ann}{ann}
\DeclareMathOperator{\Ann}{Ann}
\DeclareMathOperator{\ev}{ev}
\newcommand{\cc}{\mathbb{C}}

\begin{document}
\title{你必须知道的常微分方程常识}
\author{Buwai Lee}
\date{\today}
\maketitle %标题

\section{微积分}

记号上,所有$k$-阶导数在开集$U$上连续的$Y$-值函数的集合记作$C^k(U,Y)$,如果$Y=\rr$,则简单记作$C^k(U)$. 特别地,约定$C^0(U,Y)$为$U$上的$Y$-值连续函数。

\begin{theo}
	隐函数定理:设$f=(f^1$, $\cdots$, $f^n)$是从$\rr^{n}\times \rr^k$到$\rr^n$的函数,不妨记前$k$个变元为$\{t^1$, $\cdots$, $t^k\}$,后$n$个变元为$\{x^1$, $\cdots$, $x^n\}$. 如果在$(t_0,x_0)\in \rr^{n+k}$附近,存在一个矩形邻域$W=U\times V$使得

	\no{1} $f$在$(t_0,x_0)$处连续且$f(t_0,x_0)=0$;

	\no{2} 对每一个$1\leq i$, $j\leq n$,Jacobi矩阵$(\partial_i f^j)=(\partial f^j/\partial x^i)$在$W$上有定义且在$(t_0,x_0)$处连续;

	\no{3} Jacobi行列式$|\partial_i f^j|$在$(t_0,x_0)$处不为零;\\
\noindent 则存在一个足够小的矩形邻域$U'\times V'\subset U\times V$,以及函数$\varphi=(\varphi_1$, $\cdots$, $\varphi_n):U' \to V'$使得方程$f(t,\varphi(t))=0$成立,并且这是满足初始条件$x_0=\varphi(t_0)$的唯一解。$\varphi$在$t_0$处连续。
\end{theo}

简单来说,就是一组方程$f(t,x)=0$在$(t_0,x_0)$处关于$x$的Jacobi矩阵如果可逆,则局部存在方程的唯一解$x=\varphi(t)$.

隐函数定理的条件还是比较弱的,可以适当加强,比如设$f$不仅仅在$(t_0,x_0)$一点连续,还在它的一个邻域上连续。那么可以推知$\varphi$也不仅仅在$t_0$连续,而是在$t_0$的一个邻域上连续。实际上,由$\partial_i f^j$的连续性,在$(t_0,x_0)$的某个邻域上,所有形如$(t,\varphi(t))$的点都满足隐函数定理的条件,所有这些点附近都可以求解隐函数,然后由解的唯一性,可以拼出一个比较大的隐函数,他在$t_0$的某个邻域上连续。

另一方面,如果矩阵$(\partial f^i/\partial t^j)$在$(t_0,x_0)$的领域上存在,且在$(t_0,x_0)$处连续,则$\varphi(t)$在$t_0$处可微,导数就是按照一般的法则去求,将$f(t,\varphi(t))=0$在$t_0$处对$t$求导,有
\[
	0=\frac{\partial f^i(t,\varphi(t))}{\partial t^j}\bigg|_{t=t_0}=\frac{\partial f^i}{\partial t^j}(t_0,x_0)+\frac{\partial f^i}{\partial x^k}(t_0,x_0)\frac{\partial \varphi^k}{\partial t^j}(t_0),
\]
所以,为求$\varphi'(t_0)$,只要将
\[
	\frac{\partial f^i}{\partial x^k}(t_0,x_0)\frac{\partial \varphi^k}{\partial t^j}(t_0)=-\frac{\partial f^i}{\partial t^j}(t_0,x_0),
\]
两边乘以$\frac{\partial f^i}{\partial x^k}(t_0,x_0)$的逆矩阵即可。证明就是验证这样求出来的确实是导数。

类似地,如果在$(t_0,x_0)$的一个邻域$W$内,$f$对$x$和$t$的偏导数都存在且连续,即$f\in C^1(W,\rr^n)$,则在$t_0$的某个邻域$V$内,$f(t,x)=0$的解$\varphi\in C^1(V,\rr^n)$, 且满足
\[
	\frac{\partial f^i}{\partial x^k}(t,\varphi(t))\frac{\partial \varphi^k}{\partial t^j}(t)=-\frac{\partial f^i}{\partial t^j}(t,\varphi(t)).
\]
将这个式子两边对$t$求导,就得到了下面的命题。

\begin{pro}
	隐函数的可微性:如果存在一个$(t_0,x_0)$的邻域$W$使得$f\in C^k(W,\rr^n)$,则在$t_0$的某个邻域$V$内,$f(t,x)=0$的解$\varphi\in C^1(V,\rr^n)$. 换而言之,$\varphi$与$f$具有相同的可微性。
\end{pro}

\section{常微分方程}

从最简单的一阶常微分方程开始。设$F:\rr^3\to \rr$是一个函数,具体到变量写作$F(t,x,v)$. 所谓的一阶常微分方程就是形如$F(t,x,\dot{x})=0$的方程,其中$\dot{x}$是$x$关于$t$的导函数,这是Newton的记号。这个方程的解就是函数$\varphi(t)$,使得$F(t,\varphi(t),\varphi'(t))=0$对一定范围内的$t$成立。

一般而言,$F$不会在$\rr^3$上定义,而只在一个更小的集合上有定义,此时方程的解$\varphi$的定义域就要保证$\varphi$和$\varphi'$的值域不会在$F$的定义域外边。这样确定下来的$t$的范围就被称为解$\varphi$的定义域。

考虑特殊的一阶常微分方程$\dot{x}=f(t,x)$,其中$f$是一个给定的函数。这个方程会很普遍,因为如果$F$满足隐函数定理的条件,将$\dot{x}$解出来就得到了上面的方程。并且,关于这个方程,我们有著名的存在与唯一性定理。

\begin{theo}
	给定一阶常微分方程$\dot{x}=f(t,x)$,其中$f:U\to \rr$是开集$U\subset \rr^2$上的函数,且$f$以及$\partial_t f$在$U$上都连续,则

	\no{1} 对$U$的每一点$(t_0,x_0)$,方程$\dot{x}=f(t,x)$都存在解$x=\varphi(t)$满足初始条件$x_0=\varphi(t_0)$.

	\no{2} 如果$x=\varphi(t)$和$x=\psi(t)$都是方程的解,如果存在一个$t_0$使得$\varphi(t_0)=\psi(t_0)$成立,则在它们都有解的地方,都成立$\varphi(t)=\psi(t)$.
\end{theo}

上述定义第一条是存在性,第二条是唯一性。利用唯一性,我们可以考虑解的延拓。比方说,在开集$U_0$上有一个解,在开集$U_1$上也有一个解,且在$U_0\cap U_1$的某个点上它们是相同的,则它们在$U_0\cap U_1$上处处相同,继而我们可以将他粘起来得到$U_0\cup U_1$上面的一个解。

\end{document}