\documentclass[9pt]{extarticle}
\usepackage[zh]{../noteheader}

\newcommand{\calt}{\mathcal{T}}
\newcommand{\ee}{\mathrm{e}}
\newcommand{\ii}{\mathrm{i}}
\newcommand{\jj}{\mathcal{J}}
\newcommand{\su}{\mathrm{SU}(N)}
\newcommand{\rr}{\mathbb{R}}
\newcommand{\zz}{\mathbb{Z}}
\newcommand{\dd}{\mathrm{d}}

\DeclareMathOperator{\im}{im}
\DeclareMathOperator{\Hom}{Hom}
\DeclareMathOperator{\id}{id}
\DeclareMathOperator{\rank}{rank}
\DeclareMathOperator{\tr}{tr}
\DeclareMathOperator{\sgn}{sgn}
\DeclareMathOperator{\abs}{abs}
\DeclareMathOperator{\Arg}{Arg}

\begin{document}

已经计算得到
\[
	G(x,\tau)=\langle T_\tau \psi(x,t)\psi^\dag(0,0)\rangle=G_R(x,\tau)\,\ee^{-\ii k_F x}+G_L(x,\tau)\,\ee^{\ii k_F x},
\]
其中,$G_R(x,\tau)$如果引入(玻色到费米)修正,则具有形式
\[
	G_R(x,\tau)\sim \delta_{ab}F\left(x,|\tau|,u_c,\alpha /2\right)\exp\left(-\frac {L_c(x)} N -\left(1-\frac 1 N\right) L_s(x)\right),
\]
其中
\[
	F(x,\tau,u,a):=\left[\frac{\epsilon^2}{x^2+(u\tau+\epsilon)^2}\right]^{a},\quad L_{c(s)}(x):=\ln \frac{y_{c(s)\epsilon}-\ii x}{\epsilon}.
\]
而$G_L(x,\tau)=G_R(-x,\tau)$. 

如果重复类似的计算,可以得到
\[
	\langle T_\tau \psi^\dag(x,t)\psi(0,0)\rangle = G_R(x,\tau)\,\ee^{\ii k_F x}+G_L(x,\tau)\,\ee^{-\ii k_F x}.
\]
我们将使用它来计算延时Green函数。

计算谱函数需要实时间的延时Green函数,首先,我们先通过虚时间的Green函数计算实时间的Green函数。而这只需要进行适当的解析延拓即可,见各类量子多体的材料,比如\cite{giamarchi2004quantum}.

取$\tau =\ii t +\sgn(t)\delta$,其中$\delta=0^+$,就可以得到实时间的Green函数,对$t>0$有
\[
	G_R(x,t)\sim \delta_{ab}\left[\frac{\epsilon^2}{x^2+(\ii u_c t+\epsilon)^2}\right]^{\alpha/2}\left(\frac{\epsilon}{\ii(u_ct-x)+\epsilon}\right)^{1/N}\left(\frac{\epsilon}{\ii(u_st-x)+\epsilon}\right)^{1-1/N}.
\]
同样,$G_R(x,t)=G_L(-x,t)$.

实时间的(费米)延时Green函数被定义为
\[
	G^\text{ret}(x,t)=-\ii H(t)\left[\langle \psi(x,t)\psi^\dag(0,0)\rangle+\langle \psi^\dag(0,0)\psi(x,t)\rangle\right],
\]
注意到,对$t>0$,
\[
	\langle \psi^\dag(0,0)\psi(x,t)\rangle^*=\langle \psi^\dag(x,t)\psi(0,0)\rangle=G_R(x,t)\,\ee^{\ii k_F x}+G_L(x,t)\,\ee^{-\ii k_F x}.
\]
所以
\begin{align*}
	G^\text{ret}(x,t)&=-\ii H(t)\left[G_R(x,t)\,\ee^{-\ii k_F x}+G_L(x,t)\,\ee^{\ii k_F x}+G_R(x,t)^*\,\ee^{-\ii k_F x}+G_L(x,t)^*\,\ee^{\ii k_F x}\right]\\
	&=-2\ii H(t)\left[\mathrm{Re}(G_R(x,t))\,\ee^{-\ii k_F x}\,+\,\mathrm{Re}(G_L(x,t))\,\ee^{\ii k_F x}\right].
\end{align*}
记
\[
	G^\text{ret}_R(x,t)=-2\ii H(t)\mathrm{Re}(G_R(x,t)),\quad 	G^\text{ret}_L(x,t)=-2\ii H(t)\mathrm{Re}(G_L(x,t)),
\]
它们有关系$G^\text{ret}_L(x,t)=G^\text{ret}_R(-x,t)$.

差一个相乘因子,一个系统的谱函数被定义为延时Green函数Fourier变换后的虚部,即
\[
	A(k,\omega)=-\frac{1}{\pi}\,\mathrm{Im}\,G^\text{ret}(k,\omega).
\]
由$G^\text{ret}_R(x,t)$的表示,可以计算得
\[
	A(k,\omega)=A_R(k-k_F,\omega)+A_L(k+k_F,\omega),
\]
其中
\[
	A_{R}(k-k_F,\omega)=\frac{2\delta_{ab}}{\pi}\, \mathrm{Re}\int_{\rr^2}\dd t\dd x\, \exp(\ii \omega t-\ii k x)H(t)\mathrm{Re}(G_R(x,t)).
\]
\[
	A_{L}(k+k_F,\omega)=\frac{2\delta_{ab}}{\pi}\, \mathrm{Re}\int_{\rr^2}\dd t\dd x\, \exp(\ii \omega t-\ii k x)H(t)\mathrm{Re}(G_L(x,t)).
\]

现在以(右行粒子)谱函数的为例,按定义,有如下积分
\begin{align*}
	A_R&(k-k_F,\omega)=\frac{2\delta_{ab}}{\pi} \, \mathrm{Re}\int_{\rr^2}\dd t\dd x\, \exp(\ii \omega t-\ii k x)H(t)\mathrm{Re}(G_R(x,t))\\
	&\sim \frac{2\delta_{ab}}{\pi}\,\mathrm{Re}\int_{-\infty}^\infty \dd x\int_{0}^\infty \dd t \,\exp(\ii \omega t-\ii k x)\\
	&\hspace{5em}\mathrm{Re}\left\{\left[\frac{\epsilon^2}{x^2+(\ii u_c t+\epsilon)^2}\right]^{\alpha/2}\left(\frac{\epsilon}{\ii(u_ct-x)+\epsilon}\right)^{1/N}\left(\frac{\epsilon}{\ii(u_st-x)+\epsilon}\right)^{1-1/N}\right\}.
\end{align*}
为了对这二重积分进行计算,我们引入新的变量$z=x/t$以及$s=t$,不难计算得到
\[
	\dd x \dd t=s\dd z\dd s,
\]
注意被积分函数都可以写成$s^a f(z)$的形式,于是我们就可以先对$s$从$0$到$\infty$积分,得到
\begin{align*}
	A_R&(k-k_F,\omega)\\
	&\sim \frac{2\delta_{ab}}{\pi}\,\mathrm{Re}\int_{-\infty}^\infty \dd x \frac{f(\omega-kx)}{F\left(x,\ii,u_c,-\alpha/2\right)}\left(\frac{\epsilon}{\ii(u_c-x)+\epsilon}\right)^{1/N}\left(\frac{\epsilon}{\ii(u_s-x)+\epsilon}\right)^{1-1/N},
\end{align*}
其中
\[
	f(y)=\mathrm{Re}\int_{0}^\infty \dd t \,t^{-\alpha} \exp(\ii y t)=\Gamma \left(1-\alpha\right)\,\mathrm{Re}\left((-\ii y)^{\alpha-1}\right),
\]
这个积分只在$\alpha<1$时候才收敛,不过已经知道$\alpha$是一个很小的正数,所以这个条件是不难满足的。

结合这个积分,我们得到了
\begin{align*}
	A_R&(k-k_F,\omega)\\
	&\propto \delta_{ab}\,\mathrm{Re}\int_{-\infty}^\infty \dd x \frac{\mathrm{Re}\bigl((kx-\omega)^{\alpha-1}\bigr)}{(x^2+(\ii u_c+\epsilon)^2)^{\alpha/2}}\left(\frac{1}{\ii(u_c-x)+\epsilon}\right)^{1/N}\left(\frac{1}{\ii(u_s-x)+\epsilon}\right)^{1-1/N},
\end{align*}
其中$\epsilon=0^+$. 

为简化这个积分,首先注意到,被积式除去一个实数$\mathrm{Re}\bigl((kx-\omega)^{\alpha-1}\bigr)$的部分在$x^2-\max\{u_s,u_c\}^2>0$的时候(我们下面假设$u_c>u_s$,一般来说是这样的),等于
\[
	\frac{\ii}{(x^2-(u_c-\ii\epsilon)^2)^{\alpha/2}}\left(\frac{1}{x-u_c+\ii\epsilon}\right)^{1/N}\left(\frac{1}{x-u_s+\ii\epsilon}\right)^{1-1/N},
\]
令$\epsilon\to 0^+$,不难看到是纯虚的,其实部为零。由积分恒等式
\[
	\mathrm{Re}\int_\rr \dd x \,f(x)=\int_\rr \dd x \, \mathrm{Re}(f(x)),
\]
我们可以把积分改写为
\begin{align*}
	A_R&(k-k_F,\omega)\\
	&\propto \delta_{ab}\,\mathrm{Re}\int_{-u_c}^{u_c} \dd x \frac{\mathrm{Re}\bigl((kx-\omega)^{\alpha-1}\bigr)}{(x^2+(\ii u_c+\epsilon)^2)^{\alpha/2}}\left(\frac{1}{\ii(u_c-x)+\epsilon}\right)^{1/N}\left(\frac{1}{\ii(u_s-x)+\epsilon}\right)^{1-1/N}.
\end{align*}
这个表达式对下面的渐进分析是实用的。

给出了右行粒子的表达式,对左行粒子,我们同样可以计算得到
\begin{align*}
	A_L&(k+k_F,\omega)\\
	&\propto \delta_{ab}\,\mathrm{Re}\int_{-\infty}^{\infty} \dd x \frac{\mathrm{Re}\bigl((-kx-\omega)^{\alpha-1}\bigr)}{(x^2+(\ii u_c+\epsilon)^2)^{\alpha/2}}\left(\frac{1}{\ii(u_c+x)+\epsilon}\right)^{1/N}\left(\frac{1}{\ii(u_s+x)+\epsilon}\right)^{1-1/N}.
\end{align*}
由积分恒等式
\[
	\int_\rr \dd x\, f(x)=\int_\rr \dd x\, f(-x),
\]
可以断言
\[
	A_L(k+k_F,\omega)=A_R(k-k_F,\omega).
\]
因此,我们只要分析$A_R(k-k_F,\omega)$即可。

\section{数值积分与渐进分析}

选取参数$N=2$, $\alpha=0.2$, $k_F=1$, $u_c=0.3$而$u_s=0.2$的情况下,数值积分作图如下:
\begin{center}\includegraphics[scale=1]{1.pdf}\end{center}
其中第一个峰在$u_s k$处,第二个峰在$u_c k$处。此外,在$\omega=-u_c k$处,也有一个突起,数值积分作图如下:
\begin{center}\includegraphics[scale=1]{1_2.pdf}\end{center}
注意,两幅图的纵坐标不是一个量级的。

为求出在$\omega=u_c k$处发散的程度,代入$\omega=u_c k$,此时
\[
(x^2-u_c^2)^{\alpha/2}\sim (x-u_c)^{\alpha/2}
\]
所以
\[
	A_R(k-k_F,\omega)\sim \int_{u_c-\delta}^{u_c} \dd x \,(x-u_c)^{\alpha-1-\alpha/2-1/N}\sim \delta^{\alpha/2-1/N},
\]
因此,可以得到
\[
	A_R(k-k_F,\omega)\sim |\omega - u_c k|^{\alpha/2-1/N}.
\]
在这里,$A_R(k-k_F,\omega)$形如$|\omega - u_c k|^{\alpha/2-1/N}$发散。

但是,注意,这也仅仅只是一个估计。比如取参数$N=10$, $\alpha=0.6$, $k_F=1$, $u_c=0.3$而$u_s=0.2$时候,数值积分的结果是
\begin{center}\includegraphics[scale=1]{1_3.pdf}\end{center}
在$\omega=u_ck$处依然发散。但是此时前面的估计却给出了
\[
	A_R(k-k_F,\omega)\sim |\omega - u_c k|^{0.3-0.1}=|\omega - u_c k|^{0.2},
\]
他将在$\omega=u_ck$处为零!所以,在$\omega=u_ck$处,发散应该比$\alpha/2-1/N$更快一些,但也快不了多少。因为前面已经看到,积分是在有限区间上进行的,在$\omega=u_ck$处其他奇点处的积分很快收敛,所以在$x=u_c$处积分的发散占据的是主导地位。

类似地,在$\omega=u_s k$附近,使用同一手段可以得到
\[
	A_R(k-k_F,\omega)\sim \int_{u_s k -\delta}^{u_s k +\delta} \dd x \,(x-u_s)^{(\alpha-1)-(1-1/N)}\sim \delta^{\alpha+1/N-1},
\]
于是,在$\omega=u_s k$附近有
\[
	A_R(k-k_F,\omega)\sim H(\omega-u_s k)|\omega - u_s k|^{\alpha+1/N-1}.
\]
由于$\alpha$是一个很小的正数,所以只要当$N$足够大的时候,$A_R(k-k_F,\omega)$在$\omega=u_s k$处也确确实实是发散的。

最后,在$\omega=-u_c k$附近,依然使用同一手段可以得到
\[
	A_R(k-k_F,\omega)\sim \int_{-u_c}^{-u_c+\delta} \dd x \,(x+u_c)^{\alpha-1-\alpha/2}\sim \delta^{\alpha/2},
\]
因为$\alpha>0$,所以在$\omega=-u_c k$处积分是不发散的。但是该处的斜率为
\[
	\frac{\partial }{\partial \delta} \delta^{\alpha/2}=\delta^{\alpha/2-1},
\]
故他与横轴垂直相交。此外,在$\omega=-u_c k$附近,
\[
	A_R(k-k_F,\omega)\sim H(-\omega-u_c k)|\omega + u_c k|^{\alpha/2}.
\]
\end{document}