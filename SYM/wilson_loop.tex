\documentclass[10pt]{article}
\usepackage[zh]{../noteheader}
\usepackage{titletoc}%使用目录

\usepackage{hyperref}
	\hypersetup %一些选项
	{
		bookmarksnumbered=true,%书签中章节编号
		colorlinks=true,  % 彩色链接 false:边框链接 ; true: 彩色链接
	}

\definecolor{shadecolor}{rgb}{0.92,0.92,0.92}

\newcommand{\no}[1]{{$(#1)$}}
% \renewcommand{\not}[1]{#1\!\!\!/}
\newcommand{\rr}{\mathbb{R}}
\newcommand{\zz}{\mathbb{Z}}
\newcommand{\aaa}{\mathfrak{a}}
\newcommand{\pp}{\mathfrak{p}}
\newcommand{\mm}{\mathfrak{m}}
\newcommand{\dd}{\mathrm{d}}
\newcommand{\oo}{\mathcal{O}}
\newcommand{\calf}{\mathcal{F}}
\newcommand{\calg}{\mathcal{G}}
\newcommand{\bbp}{\mathbb{P}}
\newcommand{\bba}{\mathbb{A}}
\newcommand{\osub}{\underset{\mathrm{open}}{\subset}}
\newcommand{\csub}{\underset{\mathrm{closed}}{\subset}}

\DeclareMathOperator{\im}{Im}
\DeclareMathOperator{\Hom}{Hom}
\DeclareMathOperator{\id}{id}
\DeclareMathOperator{\rank}{rank}
\DeclareMathOperator{\tr}{tr}
\DeclareMathOperator{\supp}{supp}
\DeclareMathOperator{\coker}{coker}
\DeclareMathOperator{\codim}{codim}
\DeclareMathOperator{\height}{height}
\DeclareMathOperator{\sign}{sign}

\DeclareMathOperator{\Gal}{Gal}
\DeclareMathOperator{\ann}{ann}
\DeclareMathOperator{\Ann}{Ann}
\DeclareMathOperator{\ev}{ev}
\newcommand{\cc}{\mathbb{C}}

\title{$\mathcal N=4$ SYM 超对称振幅与超对称Wilson loop对偶}
\author{Buwai Lee}
\date{\today}

\begin{document}

\maketitle %标题

\section{(假装)熟知的事实,非超对称的情况}

从弦论和AdS/CFT,我们已经知道$\mathcal N=4$ SYM与在$AdS_5\times S^5$上的type IIB 超弦之间有着一个看上去证据极强的“对偶”存在。在振幅层次,历史上首先知道的是胶子振幅与$AdS_5\times S^5$种的开弦散射之间的关系。即使$\mathcal N=4$ SYM是紫外有限的,但在处理具体问题的时候,难免会看到红外发散,即使最终这种发散被抵消了,我们也需要一种手段去做正规化。在弦论侧,我们可以做如下的事情。

首先写下$AdS_5$的度规
\[
    ds^2=R^2\frac{dx_{3+1}^2+dz^2}{z^2},
\]
然后在$z=z_{\text{IR}}\sim \infty$附近垂直地放一排$N$个D3-膜,我们考虑的渐进态都来自于端点在这些D3-膜上的开弦的散射。对于端点位于两个平行放置位置为$z_1$和$z_2$的D3-膜的开弦来说,膜间距将赋予它一个正比于$|z_1-z_2|/\alpha$的经典质量。在这些膜非常靠近和低能的两种极限(故只考虑无质量的激发)下,他们将在$x_{3+1}$-空间中给出一个具有$SU(N)$规范群($SU(N)$指标来自于开弦端点所处的膜的指标,即所谓的Chan-Paton因子)的无质量规范理论。同时,由于我们要考虑弦微扰,我们会考虑大$N$极限下,即弱耦合的情况,
则这些开弦的微扰振幅似乎就会对应胶子的振幅。在$AdS_5$中的动量为
\[
    k_{AdS_5}=\frac{k z_{\text{IR}}}{R},
\]
其中$k$是平直$x_{3+1}$-空间中的(类光)四动量,在将截断$z_{\text{IR}}$推向无穷时可以理解为是固定的。

现在我们做T-对偶,原$x$变量变成了
\[
	\partial_\alpha y^\mu = \sqrt{-1}w^2(z)\epsilon_{\alpha\beta}\partial_\beta x^\mu
\]
原度规
\[
	ds^2=w^2(z)dx_\mu dx^\mu+\cdots
\]
在重定义$r=R^2/z$之后,$AdS_5$的度规则变成了同种形式
\[
	ds^2=R^2\frac{dy^2_{3+1}+dr^2}{r^2},
\]
而原本的$z_{\text{IR}}$则变到了边界$r\sim 0$附近。而插入顶点算子的边界条件给出了限制
\[
	\Delta y^\mu=2\pi k^\mu.
\]

\end{document}