\documentclass[10pt]{article}
\usepackage[zh]{../noteheader}
\usepackage{titletoc}%使用目录

\usepackage{hyperref}
	\hypersetup %一些选项
	{
		bookmarksnumbered=true,%书签中章节编号
		colorlinks=true,  % 彩色链接 false:边框链接 ; true: 彩色链接
	}

\definecolor{shadecolor}{rgb}{0.92,0.92,0.92}

\newcommand{\no}[1]{{$(#1)$}}
% \renewcommand{\not}[1]{#1\!\!\!/}
\newcommand{\rr}{\mathbb{R}}
\newcommand{\zz}{\mathbb{Z}}
\newcommand{\aaa}{\mathfrak{a}}
\newcommand{\pp}{\mathfrak{p}}
\newcommand{\mm}{\mathfrak{m}}
\newcommand{\dd}{\mathrm{d}}
\newcommand{\oo}{\mathcal{O}}
\newcommand{\calf}{\mathcal{F}}
\newcommand{\calg}{\mathcal{G}}
\newcommand{\bbp}{\mathbb{P}}
\newcommand{\bba}{\mathbb{A}}
\newcommand{\osub}{\underset{\mathrm{open}}{\subset}}
\newcommand{\csub}{\underset{\mathrm{closed}}{\subset}}

\DeclareMathOperator{\im}{Im}
\DeclareMathOperator{\Hom}{Hom}
\DeclareMathOperator{\id}{id}
\DeclareMathOperator{\rank}{rank}
\DeclareMathOperator{\tr}{tr}
\DeclareMathOperator{\supp}{supp}
\DeclareMathOperator{\coker}{coker}
\DeclareMathOperator{\codim}{codim}
\DeclareMathOperator{\height}{height}
\DeclareMathOperator{\sign}{sign}

\DeclareMathOperator{\ann}{ann}
\DeclareMathOperator{\Ann}{Ann}
\DeclareMathOperator{\ev}{ev}
\newcommand{\cc}{\mathbb{C}}

\title{$\mathcal N=4$ SYM 超对称振幅与超对称Wilson loop对偶}
\author{Buwai Lee}
\date{\today}

\begin{document}

\maketitle %标题

\section{一些(假装)熟知的事实}

从弦论和AdS/CFT,我们已经知道$\mathcal N=4$ SYM与在$\text{AdS}_5\times S^5$上的type IIB 超弦之间有着一个看上去证据极强的“对偶”存在。在振幅层次,历史上首先知道的是胶子振幅与$\text{AdS}_5\times S^5$种的开弦散射之间的关系。即使$\mathcal N=4$ SYM是紫外有限的,但作为无质量理论,难免会看到红外发散,即使最终这种发散被合适地抵消了,我们也往往需要一种手段去做正规化,我们会看到这将带来相当多问题。

首先在弦论侧,我们在Poincar\'e坐标卡里写下$\text{AdS}_5$的度规
\[
	ds^2=R^2\frac{dx_{3+1}^2+dz^2}{z^2},
\]
其中$R$是$\text{AdS}_5$的半径,在这个坐标下,整个$\text{AdS}_5$其可以看成$z>0$给出的半空间,而$z=0$是边界,在靠近边界的时候,共形等价于一个$3+1$
的Minkowski时空,所以可以说$\text{AdS}_5$边界是一个平坦时空。
然后我们在$z=z_{\text{IR}}>\!\!> 0$附近垂直地放一排$N$个D3-膜,我们考虑的渐进态都来自于端点在这些D3-膜上的开弦的散射。对于端点位于两个平行放置位置为$z_1$和$z_2$的D3-膜的开弦来说,膜间距将赋予它一个正比于$|z_1-z_2|/\alpha'$的经典质量。在这些膜非常靠近(合并为一)和低能的两种极限(故只考虑无质量的激发)下,他们将在$x_{3+1}$-时空中给出一个具有$\text{SU}(N)$规范群($\text{SU}(N)$指标来自于开弦端点所处的膜的指标,即所谓的Chan-Paton因子)的无质量规范理论。同时,由于我们要考虑弦微扰,我们会考虑大$N$极限下,即弱耦合的情况,
则这些开弦的微扰振幅就会对应胶子的振幅。

为考虑弦论的散射振幅,我们需要在D3-膜上插入顶点算符来表示渐进态然后计算相应的路径积分。但是,D3-膜上的
动量并不完全是在$\text{AdS}_5$边界上的动量。考虑一个没有$z$方向的动量$p$,那么为了动量的平方$g_{\mu\nu}p^\mu p^\nu=R^2\eta_{\mu\nu} p^\mu p^\nu/z^2$是一个不变量,那么位于$z=z_{\text{IR}}$处的
D3-膜上的渐进态的动量形如
\[
	p=\frac{z_{\text{IR}}}{R}k,
\]
在$z_{\text{IR}}\to 0$的时候,可以看到$k$是边界处真正平直$x_{3+1}$-时空中的(类光)四动量,他并不随着$z_{\text{IR}}$变动而改变。此时,在红外极限$z_{\text{IR}}\to \infty$时,$p$会变得足够大,此时插入的顶点算符$V\propto \exp(ip\cdot y)$会使得路径积分主要由极值点处的贡献主导,即
\[
	A\propto \exp(iS),
\]
其中$S$为极值点处的经典世界面作用量。但是,在$\text{AdS}_5$中计算这个经典作用量并不容易,
所幸,我们可以通过T-对偶将$z_{\text{IR}}\to \infty$变到另一个$\text{AdS}_5$中的边界附近,
进而,我们可以跑到一个平直时空中去考虑这个问题。

% 众所周知,T-对偶保持左行坐标$x^\mu_L(\tau+\sigma)$不变,
% 而将右行坐标$x^\mu_R(\tau-\sigma)$变成$-x^\mu_R(\tau-\sigma)$. 
% 将变换后的坐标叫做$y^\mu$,我们需要有
% \[
% 	(\partial_\tau+\partial_\sigma)(y^\mu-x^\mu)=0,\quad 
% 	(\partial_\tau-\partial_\sigma)(y^\mu+x^\mu)=0,
% \]
% 换言之
% \[
% 	\partial_\tau y^\mu=\partial_\sigma x^\mu,\quad \partial_\sigma y^\mu=\partial_\tau x^\mu,
% \]
% 或者$\partial_\alpha y^\mu=\epsilon_{\alpha\beta}\partial_\beta x^\mu$.
% 我们也可以
% 在$\sigma$-模型下考虑这个问题,考虑世界面上的作用量(这里我们略去时空指标)
% \[
% 	\int\left(\frac{1}{2} v^\alpha v_\alpha-\epsilon^{\alpha \beta} x \partial_\beta v_\alpha\right) d^2 \sigma,
% \]
% 乘子$x$对应的运动方程写作$\epsilon^{\alpha\beta}\partial_\beta v_\alpha=0$,此时他的解形如
% $v_\alpha=\partial_\alpha \tilde x$,此外,对$v$变分还有运动方程
% \[
% 	v_\alpha=-\sigma_{\alpha\beta}\partial^\beta x,
% \]
% 这就有$\partial_\alpha \tilde x=-\epsilon_{\alpha\beta}\partial_\beta x$,在
% $x$和$\tilde x$坐标下,作用量写作
% \[
% 	\frac{1}{2} \int \partial^\alpha \tilde{x} \partial_\alpha \tilde{x} d^2 \sigma
% 	=
% 	\frac{1}{2} \int \partial^\alpha x \partial_\alpha x d^2 \sigma.
% \]
这里我们考虑T-对偶
\[
	\partial_\alpha y^\mu = \sqrt{-1}w(z)^2\epsilon_{\alpha\beta}\partial_\beta x^\mu,
\]
在世界面的光锥坐标系下,在差一个相乘因子下,他保持一个方向不变,而将另一个方向变号。
原度规形如
\[
	ds^2=w(z)^2dx_\mu dx^\mu+\cdots,
\]
在重定义$r=R^2/z$之后,我们得到了另一个$\text{AdS}_5$的度规
\[
	ds^2=R^2\frac{dy^2_{3+1}+dr^2}{r^2},
\]
而原本的$z_{\text{IR}}$则变到了这个$\text{AdS}_5$的边界$r\sim 0$附近。在T-对偶
$\partial_\alpha y^\mu = \sqrt{-1}w(z)^2\epsilon_{\alpha\beta}\partial_\beta x^\mu$
两边对$d\sigma$积分,右侧的动量变成了开弦两点之差(即环绕数),形如
\[
	\Delta y^\mu=2\pi k^\mu.
\]
这个类光多边形当$z_{\text{IR}}\to \infty$时处于$r=0$处,散射的计算此时也就等价于这样的Wilson圈在强耦合的期望值。在这个区域中,弦作用量的鞍点就是止于对偶空间种的$k_1,\dots,k_n$的世界面的极小面积$A_{\text{min}}(k_1,\dots,k_n)$,因此,类光Wilson圈和振幅的领头阶都可以写作
\[
    \widehat A_{n;0}\sim \exp\bigl(-\frac{\sqrt{\lambda}}{2\pi}A_{\text{min}}(k_1,\dots,k_n)\bigr).
\]

% 原则上,振幅的弱耦合将$T$-对偶到Wilson圈的强耦合,但是振幅-Wilson圈对偶的故事中,弱耦合的Wilson圈也占有一席之地。Wilson圈被定义为
% \[
%     W(\mathcal{C})=\frac{1}{N}\left\langle 0\left|\operatorname{Tr} \mathcal{P} \mathrm{e}^{i g \oint_{\mathcal{C}} A_{\mu} \mathrm{d} x^{\mu}}\right| 0\right\rangle,
% \]
% 它的微扰展开写作
% \[
%     W(\mathcal{C}) \sim 1+i g \oint_{\mathcal{C}} A_{\mu} \mathrm{d} x^{\mu}+(i g)^{2} \oint_{\mathcal{C}} \mathrm{d} x^{\mu} \int_{x_{j}>x_{i}} \mathrm{~d} x^{\nu} A_{\mu}\left(x_{i}\right) A_{\nu}\left(x_{j}\right)+\cdots.
% \]

\section{超对称的情况}

在引入超对称后,我们应考虑相应的超对称振幅和超对称Wilson圈。对$\mathcal N=4$ SYM
来说,在壳的态可以通过超对称组织成一个超态,其包含了gluon, gluino以及标量粒子。不同helicity的态
通过一族Grassmann变量$\eta_i^A$打包在一起,其中$A=1,2,3,4$,即
\[
	\Omega=g^{+}+\eta_A \lambda^A+\frac{1}{2 !} \eta_A \eta_B \phi^{A B}+\frac{1}{3 !} \eta_A \eta_B \eta_C \lambda^{A B C}+\eta_1 \eta_2 \eta_3 \eta_4 g^{-},
\]
进而所有的振幅可以打包为一个超振幅$A$,具有展开
\[
	A_n=A_{n;0}+A_{n;1}+\cdots+A_{n;n-4}
\]
分别对应$\text{N}^k\text{MHV}$的振幅,并且,我们可以通过除掉MHV振幅来剔除
动量和超荷守恒必然有的$\delta$函数,记
\[
	A_{n;k}=\frac{\delta^{(4)}(p^{\dot{\alpha} \alpha}) \delta^{(8)}(q_\alpha^A)}{\langle 12\rangle\langle 23\rangle \ldots\langle n 1\rangle} \widehat{A}_{n ; k}(p, \eta ; a),
\]
其中$a$是耦合常数,相乘因子即MHV树图振幅,故对MHV振幅而言,对耦合常数展开我们有$A_{n;0}=1+O(a)$.


类似地,$n$-点超Wilson圈(先不管其如何定义)可以在对偶空间按照超对称的分量去展开,其展开应具有形式
\[
	W_n=\sum_{k=0}^n W_{n;k}(x_i,\theta_i^A,a),
\]
其中$x_i$是对偶坐标,而$\theta_i^A$是对偶超坐标,满足
\[
p_i^{\dot{\alpha} \alpha}=\widetilde{\lambda}_i^{\dot{\alpha}} \lambda_i^\alpha=\left(x_i-x_{i+1}\right)^{\dot{\alpha} \alpha}, \quad q_i^{\alpha A}=\lambda_i^\alpha \eta_i^A=\left(\theta_i-\theta_{i+1}\right)^{\alpha A}.
\]
从$R$-对称性,$W_{n;k}(x_i,\theta_i^A,a)$应当是一个
$\chi_i^A=\langle \lambda_i\theta_i^A\rangle$的一个$4k$-次齐次不变多项式,
每个$k$对应着$\text{N}^k\text{MHV}$.

超振幅与超Wilson圈的对偶写作
\[
	W_{n;k}(x,\theta;a)=a^k \widehat A_{n,k}(p,\eta;a),
\]
右侧的$a^k$是因为$W_{n;k}$在弱耦合按耦合常数展开时是从$a^k$开始的。于是,
对完整的超对称Wilson圈代入上面的对偶并在弱耦合展开后,我们有
\[
	\sum_{L=0}^\infty W_n^{(L)}a^L=\sum_{L=0}^\infty\sum_{k=0}^n a^{k+L}\widehat A_{n,k}^{(L)}(p,\eta),
\]
于是
\[
	W_n^{(L)}=\sum_{k=0}^{\min(n,L)} \widehat A_{n,k}^{(L-k)}(p,\eta).
\]
可以看到,给定圈数的Wilson圈包含了更低圈的振幅。



\end{document}