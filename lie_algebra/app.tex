\appendix
\noindent \textbf{\huge Appendix}

\cftaddtitleline{toc}{section}{Appendix}{}
\renewcommand{\thepara}{\Alph{section}.\arabic{para}}

\section{Lie群基础}

\para 设$G$为群,单位元记做$e$,群运算记做$\mu:G\times G\to G$,如果$G$是一个光滑流形,且$\mu$是一个光滑映射,则称$G$是一个光滑Lie群。当然可以谈论不怎么光滑的Lie群,但是下面所指的Lie群都是光滑的。记$l_g$是左作用算符,即$l_g=\mu(g,\cdot)$,或者写作$l_gh=gh$,同样,右作用算符记做$r_g$,即$r_gh=hg$.显然这些都是光滑映射。

现在考虑方程$\mu(x,y)=e$,由于$(l_e)_{*e}$是一个恒同映射,所以在$e$附近,按隐函数定理,方程$\mu(x,y)=e$在$e$附近有光滑解,即$y=\nu(x)=x^{-1}$中的逆函数$\nu$在$e$附近光滑,由于$(\nu\circ l_g)(h)=h^{-1}g^{-1}=(r_{g^{-1}}\circ \nu)(h)$,所以逆函数$\nu$处处光滑。

因为Lie群有着光滑流形结构,那么我们就可以对其局部线性化,特别地,单位元附近的局部线性化就构成了Lie代数的内容。

\para 一个Lie群$G$的Lie代数$\lag$就是其单位元处的切空间。
\endpara

由于$l_g$和$r_g$都是$G\to G$的光滑同胚,所以$(l_g)_*$或者$(r_g)_*$就是将光滑切矢量场映射到光滑切矢量场的双射。如果矢量场$X_x$满足$(l_a)_*X_x=X_{ax}$,则称$X$是一个左不变矢量场。对于左不变矢量场$X$而言,由于\footnote{设$f:M\to N$是一个光滑单射,而$X$是$M$上的一个光滑矢量场,则$f_*X:p\mapsto f_{*f^{-1}(p)}X_{f^{-1}(p)}$是$N$上的一个矢量场。因为$(f_*X)g=X(g\circ f)$成立,所以这也是一个光滑矢量场。}$(l_a)_*X:g\mapsto (l_a)_{*a^{-1}g}X_{a^{-1}g}=X_g$,所以$(l_a)_*X=X$.

和任意的矢量场一样,左不变矢量场$X$在$e$处诱导了$\lag$中的元素$X_e$. 反过来,设$X_e\in\lag$,我们可以构造一个左不变矢量场$x\mapsto X_x=(l_{x})_*X_e$,因此我们就建立了Lie代数和左不变矢量场之间的一一对应。

\para 左不变矢量场都是光滑的且完备的。
\endpara

实际上,任取$f\in\calf(G)$以及$g\in G$,我们来看$(Xf)(g)=X_gf=(l_g)_{*}X_ef=X_e(f\circ l_g)$. 取一个在$e$处切矢量为$X_e$的曲线$\sigma$,则
\[
	Xf(g)=\frac{\dd }{\dd t}\bigg|_{t=0}f(g\sigma(t))=\frac{\dd }{\dd t}\bigg|_{t=0}f\circ \mu(g,\sigma(t)),
\]
是一个光滑函数,所以左不变矢量场都是光滑矢量场。至于完备性,设$\sigma$是左不变矢量场$X$的积分曲线,则$l_g\circ \sigma$也是$X$的积分曲线,这就使得我们可以把一条局部积分曲线拼到无穷远,这就是说$X$是完备的。

\para 设$X$是左不变矢量场,因为$X$是完备的,所以他诱导的单参数变换群$\{\sigma^X_t\}$在整个Lie群上是有定义的,特别地,在单位元上,我们定义$\exp(t,X):=\sigma^X_t(e)$,其中$X$是左不变矢量场,因为左不变矢量场一一对应着Lie代数,所以也可以说$X\in \lag$. 同样,对固定的$X$,映射$\exp(t,X)$对改变的$t$就构成了$G$的一个子群,这被称为单参子群。由积分曲线的存在唯一性,我们也得到了单参子群与Lie代数的一一对应关系。

{\pro 设$f:G\to G$是一个微分同胚,我们有$f_*[X,Y]=[f_*X,f_*Y]$,若$f=l_a$,那么我们立刻就得到了左不变矢量场的对易子也是左不变的。因此对于Lie代数来说,他继承了切矢量场的Lie括号$[\star,\star]:\lag\times \lag\to \lag$,这是一个二元线性运算,所以Lie代数确实是一个代数。\endpro}

通过直接的计算,Lie代数上满足:

\no{1} $[X,Y]=-[Y,X]$,

\no{2} $[X,[Y,Z]]+[Y,[Z,X]]+[Z,[X,Y]]=0$.

第一条反对称性从矢量场的$[X,Y]=XY-YX$来看是显然的。而第二条称为Jacobi恒等式,直接计算即可验证。可以如下记忆Jacobi恒等式,$X$, $Y$和$Z$的三种右手方向构成的置换和为$0$,或者说,$[X_i,[X_j,X_k]]$中$ijk$是$123$的偶置换。

\para Lie代数上的双线性映射$f$如果满足$f([X,Y])=[f(X),Y]+[X,f(Y)]$,则$f$被称为一个导子。\endpara

在Lie代数上我们可以找到一个自然的导子。适当改写Jacobi恒等式,我们可以得到$[X,[Y,Z]]=[[X,Y],Z]+[Y,[X,Z]]$,如果记$\ad(X):Y\mapsto [X,Y]$,于是
\[
	\ad(X)([Y,Z])=[\ad(X)Y,Z]+[Y,\ad(X)Z],
\]
因此$\ad(X)$就是一个Lie代数上面的导子,他被称为内导子。

\para[指数映射] 前面我们定义了$\exp:\lag\times \rr\to G$,特别地,我们记$\exp(1,X)$为$\exp(X)$,这样定义的映射$\exp:\lag\to G$他被称为指数映射。

可以看到$\exp(tX)=\sigma^{tX}_1(e)=\sigma^{X}_t(e)=\exp(t,X)$,所以实际上,我们指数映射已经能够完全包含$\exp:\lag\times \rr\to G$的内容了。特别地,
\[
	\exp(tX)\exp(sX)=\exp(t,X)\exp(s,X)=\exp(t+s,X)=\exp((t+s)X).
\]
就和一般的指数表现得那样。但如果$[X,Y]\neq 0$,一般来说$\exp(X)\exp(Y)\neq \exp(X+Y)$.我们有时候也会通过$e^{X}$来记$\exp(X)$.\endpara

\begin{lem} \label{exp}我们找一个光滑函数$f:G\to \rr^n$,那么$g(t)=f(xe^{tX})$就是一个$\rr$上的光滑函数,我们来归纳证明他的$n$阶导数为
\[
	\frac{\dd^n}{\dd t^n}g(t)=(X^nf)(x e^{tX}).
\]
\end{lem}

\proof $n=0$是显然的,$n=1$需要直接计算验证
\[
	(Xf)(x)=\left\{\frac{\dd}{\dd t}f(x e^{tX})\right\}_{t=0},
\]
这个的计算只要使用链式法则
\[
	\left\{\frac{\dd}{\dd t}f(x e^{tX})\right\}_{t=0}=f_{*x}(e^{tX})_{*0}=f_{*x}X=(Xf)(x).
\]
注意最后一个等式要依赖于$f$是矢量值的,某种程度来说这就是$T\rr^n=\rr^n$的结果。由于矩阵也可以看成在欧氏空间$\rr^{n\times n}$里,所以$f$也可以取值为矩阵。

假设$n=k$是成立的,那么因为$X^{k+1}=X\circ X^k$,
\[
	(X^{k+1}f)(x e^{tX})=(X(X^{k}f))(x e^{tX})=\left\{\frac{\dd}{\dd s}(X^kf)(x e^{(s+t)X})\right\}_{s=0}=\frac{\dd}{\dd t}(X^kf)(x e^{tX})=\frac{\dd^{k+1}}{\dd t^{k+1}}g(t).
\]\endproof

% \para 为了下面的讨论,我们先将实数值的1-形式拓展到矢量值的1-形式。设$V$是一个矢量空间,对于切矢量场的$V$-值函数$\omega:\Gamma(TM)\to V$被称为一个$V$-值1-形式。如果对任意的光滑切矢量场$X$,我们都有$\omega(X)$是$M$上的$V$-值光滑函数,则$\omega$被称为光滑的。

% \para Lie群$G$的切丛$TG$倒是相当简单,因为我们可以定义$(l_{a^{-1}})_*$把$T_aG$始终映射到$T_eG=\lag$来考虑,所以切丛就被平凡化了。与这相关的概念即Maurer-Cartan形式。设$G$是一个Lie群,他的切丛记做$TG$,映射$\omega_G:(g,v)\mapsto (l_{g^{-1}})_*v$被称为Maurer-Cartan形式。可以看到$\omega_G:\Gamma(TG)\to \lag$,因此Maurer-Cartan形式可以看做一个$\lag$值1-形式。且对于任意的$l_h^*$,我们都有
% \[
% 	(l_h^*\omega_G)v=\omega_G((l_h)_*v)=(l_{(hg)^{-1}})_*(l_h)_*v=(l_{(g)^{-1}})_*v=\omega_G(v).
% \]
% 所以Maurer-Cartan形式是左不变的。

\para 现在来看具体的例子,设所有$n\times n$的实(复)矩阵构成的集合为$\operatorname{GL}(n,\rr)$($\operatorname{GL}(n,\cc)$),其中$\det A\neq 0$的矩阵按矩阵乘法构成一个群$\operatorname{GL}(n,\rr)$($\operatorname{GL}(n,\cc)$),我们称为一般线性群,单位元是$I$。由于他可以开嵌入$\rr^{n^2}$($\cc^{n^2}\cong \rr^{2n^2}$)内,所以他有自然的光滑流形结构。因此一般线性群是一个Lie群,矩阵群上的微分定义使得我们可以直接计算一般线性群的Lie代数,他的Lie代数为$\mathfrak{gl}(n,\rr)$。在一般线性群$G$上
\[
	(l_g)_{*a}v=\frac{1}{t}(l_g(a+tv)-l_g(a))
	=\frac{1}{t}(l_g(tv))=l_g(v)=gv.
\]
其中$v\in T_aG$.

% 所以一般线性群上面的Maurer-Cartan形式即为$\omega_G(v)=l_{g^{-1}}(v)=g^{-1}v$,其中$g$和$v$都是矩阵,矩阵乘矩阵还是矩阵,所以Lie代数$\mathfrak{gl}(n,\rr)$也是矩阵的形式。设$\dd g=(\dd x_{ij})$,那么$v$就可以写成$\dd g(v)$,因为$\dd x_{ij}(v)=v_{ij}$,则$\omega_G=g^{-1}\dd g$.

由于$\operatorname{GL}(n,\rr)$的微分结构是熟知的,我们可以直接计算其Lie代数$\mathfrak{gl}(n,\rr)$上的交换子形式。设$A\in\mathfrak{gl}(n,\rr)$而$g\in\operatorname{GL}(n,\rr)$,容易验证$A_g=gA$是左不变矢量场,因为$(l_h)_{*}A_g=(l_h)_{*}gA=hgA=A_{hg}$.

记$g=(x_{ij})$,考虑与$A=(a_{ij})$和$B=(b_{ij})$相关的左不变矢量场为
\[
A_g=\sum_{i,j,k}x_{ij}a_{jk}\partial_{ik},\quad B_g=\sum_{i,j,k}x_{ij}b_{jk}\partial_{ik},
\]
于是
\[
[A_g,B_g]=\left[\sum_{i,j,k}x_{ij}a_{jk}\partial_{ik},\sum_{i,j,k}x_{ij}b_{jk}\partial_{ik}\right]=\sum_{i,k}\left(\sum_{j}x_{ij}\sum_{r}(a_{jr}b_{rk}-b_{jr}a_{rk})\right)\partial_{ik},
\]
或者$[A_g,B_g]=(AB-BA)_g$,所以$\mathfrak{gl}(n,\rr)$上的对易子为$[A,B]=AB-BA$,其中的乘法就是矩阵乘法。

\para 对于$A\in\mathfrak{gl}(n,\rr)$,指数映射有如下级数展开
\[
	e^A=1+\sum_{n=1}^\infty \frac{A^n}{n!}=\sum_{n=0}^\infty \frac{A^n}{n!},
\]
对于任意的矩阵$A$都是收敛的。可以看到其完全类似于实数值指数函数的展开$e^x=\sum_{n=0}^\infty x^n/n!$.

\proof 由 Lemma \ref{exp},对一般的Lie群$G$和光滑函数$f:G\to \rr^n$,使用Taylor公式
\[
	f(xe^{tX})=\sum_{k=0}^n\frac{(tX)^{k}}{k!}f(x)+O(t^{n+1}),
\]
如果可以展开无数项,那么
\[
	f(xe^{tX})=\sum_{k=0}^\infty\frac{(tX)^{k}}{k!}f(x).
\]

现在取$f(A)=A$, $x=I$和$t=0$就可以了。至于收敛性,因为对于任意一个矩阵,$A$的范数都是有界的,那么$e^A$就被$A$的范数的级数控制,因此收敛。\endproof

% \para 以下矩阵群构成一般线性群的子群:

% \no{1} 特殊线性群:$\operatorname{SL}(n,\rr)=\{A\in \operatorname{GL}(n,\rr)|\det A=1\};$

% \no{2} 正交群:$\mathrm{O}(n) = \{ Q \in \operatorname{GL}(n,\rr) \mid Q^T Q = Q Q^T = I \};$

% \no{3} 酉群:$\mathrm{U}(n) = \{ Q \in \operatorname{GL}(n,\cc) \mid Q^\dag Q = Q Q^\dag = I \};$

% \no{4} 特殊正交群:$\mathrm{SO}(n) =\{ Q \in \mathrm{O}(n) \mid \det Q=1 \};$

% \no{5} 特殊酉群:$\mathrm{SU}(n) =\{ Q \in \mathrm{U}(n) \mid \det Q=1 \};$

% 我们来考虑最简单的一个特殊正交群$\mathrm{SO}(2)$,他的群元素由矩阵
% \[
% 	\begin{pmatrix}
% 	\cos \theta&-\sin \theta\\
% 	\sin \theta&\cos \theta\\
% 	\end{pmatrix}
% \]
% 构成。这是一个Abel群,而且可以注意到,他同构于群$\mathrm{S}^1=\{e^{i\theta}:\theta\in\rr\}=\mathrm{U}(1)$,这是一个圆周。

% \para 对于矩阵群,我们可以使用Heine-Borel定理断言有界闭子群是紧的,所以$\mathrm{O}(n)$, $\mathrm{SO}(n)$, $\mathrm{U}(n)$, $\mathrm{SU}(n)$都是紧的,但是一般线性群不是紧的。

\clearpage

\section{一些基本的表示论}

上节谈了Lie群的“内部线性化”,即Lie代数的内容。这节我们来谈论群表示,他使得我们可以把群进行“外部线性化”,粗略地来说就是我们把群元素看做了一个线性变换。

\para 令$V$是一个域$k$(后面我们只会考虑$\rr$和$\cc$的情况)上的有限维矢量空间,群$G$的一个表示$(\pi, V)$指存在这样的一个群同态$\pi:G\rightarrow \operatorname{GL}(V)$,使得
\[
	\pi(g)\pi(g')=\pi(gg'),\quad \pi(g^{-1})=\pi(g)^{-1},\quad \pi(e_G)=e_{\operatorname{GL}(V)}
\]
成立。表示$(\pi, V)$的维度被定义为$V$的维度。如果$\pi$是一个单同态,那么我们称这个表示为忠实表示。

如果没什么会混淆的话,就直接略去$\pi$,写$gx$来表达$\pi(g)x$,当然还有用$Gx$来表达$\pi(G)x=\{\pi(g)x:g\in G\}$。

设$\lag$是一个Lie代数,则他的表示是一个Lie代数同态$\rho:\lag \to \mathfrak{gl}(V)$,即一个线性空间同态满足$\rho([a,b])=[\rho(a),\rho(b)]$.

\para 同一个群$G$(Lie代数$\lag$)的两个表示$(\pi_1,V_1)$和$(\pi_2,V_2)$可以构造出一个新的表示,即直和表示$(\pi_1\oplus \pi_2,V_1\oplus V_2)$,他满足
\[
	(\pi_1\oplus \pi_2)(g)(x,y)=(\pi_1(g)x,\pi_2(g)y).
\]
或者简单地写作$g(x,y)=(gx,gy)$.

\para 一个矢量空间$V$,所有线性函数$f:V\to k$也构成一个矢量空间,记作$V^*$。如果$V$是有限维的,他和$V$是同构的,如果存在一个非退化的双线型(比如一个内积),则我们可以建立$V^*$和$V$之间的一个自然同构。

考虑群表示$\rho:G\to \operatorname{GL}(V)$,我们可以构造一个新的表示$\rho^*:G\to \operatorname{GL}(V^*)$通过
\[
	\rho^*(g)(f):v\mapsto f(\rho(g^{-1})(v)),
\]
很容易检验这是一个群表示。他被称为对偶表示。

对于Lie群,我们考虑$g=\exp(tX)$,我们有$f(\rho(\exp(-tX))(v))$,在$t=0$处求导有$f(-\rho_*(X)v)$,所以我们定义Lie代数的表示$(\rho,V)$的对偶表示$(\rho^*,V^*)$如下
\[
	\rho^*(X)(f):v\mapsto -f(\rho(X)v).
\]

\para 设$(\pi,V)$是群$G$(Lie代数$\lag$)的一个表示,$W\subset V$是一个子空间,如果$\pi(G)W\subset W$(对于Lie代数,$\pi(\lag)W\subset W$)则$W$称为不变子空间。显然,$\{0\}$和$V$是两个不变子空间,略去这两个平凡不变子空间,如果没有其他不变子空间了,则$V$被称为是不可约的,此时表示被称为不可约表示。如果一个表示能被分解成几个不可约表示的直和,则称该表示为完全可约(或者叫完全分解)的。

\para 可以断言,如果一个有限维表示是可以完全分解的,那分解出来的直和,在允许直和顺序可以交换下是唯一的。比如我们考虑两个分解$\bigoplus_i V_i$和$\bigoplus_j W_j$,对$V_i$,必然存在一个$W_j$使得$V_i\cap W_j\neq \varnothing$,然后考虑这里面的任意元素$a$,任取$g\in G$($g\in\lag$),因为$ga\in V_i$以及$ga\in W_j$,所以$ga\in V_i\cap W_j$,也即$V_i\cap W_j$构成了一个不变子空间。利用$V_i$和$W_j$是不可约的,我们有$V_i=V_i\cap W_j=W_j$.
\endpara

下面的一系列定义涉及表示的等价,当然还有很重要的Schur引理。

\para 设有同一个群(Lie代数)的两个表示$(\pi_1,V_1)$和$(\pi_2,V_2)$,如果存在线性映射$T:V_1\to V_2$,对任意的$g\in G$($g\in \lag$)都满足$\pi_2(g)\circ T=T\circ \pi_1(g)$. 这样的$T$被称为缠结映射\footnote{这不是一个主流的名字。一般我们把表示看成$G$-模或者$\lag$-模,那么这样的映射被称为$G$-模同态或者$\lag$-模同态。两个表示是等价的,就是说这个同态是一个同构。}。当缠结映射是同构的时候,两个表示被称为是等价的。显然,这确实是一个等价关系。自反对称显然,而传递性自然来自两个同构复合还是同构。
\endpara

{\lem 如果$\pi_1$和$\pi_2$之间存在缠结映射$T$,那么$\ker T$是$\pi_1$的不变子空间,而$\im T$是$\pi_2$的不变子空间。\endlem}

\proof 因为$T(\pi_1(g)x)=\pi_2(g)Tx$对于任何$x$使得$Tx=0$的,都有$\pi_1(g)x$使得$T(\pi_1(g)x)=0$,所以前半句话证明完了。对于$\im T$中的元素$y$可以找到原象$x$,由于$\pi_2(g)y=\pi_2(g)Tx=T(\pi_1(g)x)$,则$\pi_2(g)y$也在$\im T$中,后半句话证完。\endproof

{\lem[Schur引理]如果$(\pi_1,V_1)$和$(\pi_2,V_2)$是不等价的不可约表示,若存在缠结映射$T$,则$T=0$.换句话说,不等价的表示没有非平凡的缠结映射。\endlem}

\proof 如果两个表示不等价,则$T$不是双射,所以$\ker T\neq \{0\}$,而他是不变子空间,由不可约性,则$\ker T= V_2$,同理$\im  T= \{0\}$,这就是说$T=0$。\endproof

\para 反之,在复表示情况下,我们考虑不可约表示$(\pi_1,V_1)=(\pi_2,V_2)$的情况下。此时如果存在一个非平凡的双的缠结映射$T\neq I$使得$T\circ \pi(g)=\pi(g)\circ T$.

在复数域上,$T$一定存在一个本征值$\lambda$,令$E_\lambda$是$\lambda$的本征空间,任取$v\in E_\lambda$,我们有$T(\pi(g)(v))=\pi(g)\circ T(v)=\lambda\pi(g)(v)$,所以$\pi(G)E_\lambda\subset E_\lambda$,由$\pi$的不可约性,所以他要么是零空间,要么是全空间。而本征值的存在性说明了零空间不可能,所以本征空间就是全空间,这也就是说$T=\lambda I$.

\para 如果我们遇到的群是Abel群,那么他的不可约群表示也是可交换的,同时其本身就构成了一个缠结映射,即$T=\pi(g)$对任意的$h$成立$T\circ \pi(h)=\pi(h)\circ T$,所以通过上面的定理我们就可以知道,$\pi(g)=T=\lambda I$。

如果$V$是大于$1$维的,那么任意的$V$的子空间都是$\pi$的不变子空间,但不可约性否决了这点,所以我们得到了:凡Abel群的不可约复表示都是一维的。

\para 如果$(\star,\star)$是$V$上的一个内积,如果对任意的$g\in G$, $u$, $v\in V$有$(u,v)=(gu,gv)$,则我们称呼这个表示为幺正表示,幺正表示也可以写作$\pi(g)^{-1}=\pi(g)^\dag$。对于有限群我们总可以找到幺正表示,因为我们可以重新构造内积
\[
	(u,v)'=\frac{1}{|G|}\sum_{g\in G}(gu,gv),
\]
那么
\[
	(hu,hv)'=\frac{1}{|G|}\sum_{g\in G}(hgu,hgv)=\frac{1}{|G|}\sum_{hg\in G}(hgu,hgv)=\frac{1}{|G|}\sum_{k\in G}(ku,kv)=(u,v)'.
\]

幺正表示的好处是,一个不变子空间的正交空间也是不变的。对于一个有限维的幺正表示,如果不是完全可约的,那么就分解出一个不变子空间,他是不可约的,然后对这个不变子空间的正交空间,我们又得到了一个可约或不可约的不变子空间,靠着有限归纳(因为有限维),我们就得到了如果群存在一个有限维幺正表示,则这个表示一定是完全可约的。

应用到有限群上,这就是Maschke定理的内容:有限群的有限维表示总是完全可约的。 

{\thm 在局部紧的拓扑群上存在Haar测度$\mu$,他是一个正则的左不变的Borel测度。所谓的左不变就是指对于一个集合$S$,通过左作用$L_g$,我们有$\mu(S)=\mu(L_gS)$,当然还有右作用和右不变的概念。\endthm}

对于$G$上的可测函数$f$,上述测度对任意的$h\in G$都满足
\[
	\int_G f(l_h g)\dd \mu(g)=\int_G f(g)\dd \mu(g),
\]
可以从中直接看出左不变的意义。以后我们将$\dd \mu(g)$直接写作$\dd g$.

上面的定理我们不证明,同时也不证明如下命题:如果$G$是紧Lie群,则我们存在双不变(既左不变也右不变)测度$\mu$,且$\mu(G)$有限。

\para 紧Lie群上存在幺正表示,所以紧Lie群的有限维表示总是完全可约的。 
\endpara

换而言之,我们可以找到内积$\langle \star,\star\rangle$使得$\langle gx,gy\rangle=\langle x,y\rangle$。假设原本存在内积$\langle \star,\star \rangle$,那么再设$\dd g$是$G$上的Haar测度(因为紧所以存在,而且已经归一化),那么
\[
	\langle x,y\rangle'=\int_G \langle gx,gy \rangle \dd g
\]
就满足了要求。下面我们谈论紧Lie群的表示的时候,总使用幺正表示。

从这可见,紧Lie群表示的完全可约性,又是一个紧性作为有限性条件的例证。

\subsection*{Character Theory}

\para 对于有限群$G$上的复值函数$f$和$g$,我们可以定义内积为
\[
	(f,g)=\frac{1}{|G|}\sum_{h\in G} f(h)g^*(h),
\]
其中$*$代表的是复共轭。类似地,对于Lie群$G$上的复值光滑函数$f$和$g$,我们可以定义内积为
\[
	(f,g)=\int_G f(h)g^*(h)\dd h,
\]
其中积分已经被归一化过。

\para 对于矢量空间$V$和$W$,$\Hom(V,W)\cong V^*\otimes W$也有自然的矢量空间结构,所以如果已知$V$上有群表示$\pi_1$,以及$W$上有群表示$\pi_2$,则对于$A\in \Hom(V,W)$,我们可以定义群表示$(\pi_1^*\otimes\pi_2,\Hom(V,W))$如下
\[
	\pi_1^*\otimes\pi_2(g)A=\pi_2(g)A\pi_1(g)^{-1}.
\]
很容易验证$\pi_1^*\otimes\pi_2(g)\pi_1^*\otimes\pi_2(h)=\pi_1^*\otimes\pi_2(gh)$,所以这是一个群表示。为了省略空间,我们经常也直接记做$gA$.

对每一个群元$g$,表示诱导了映射$g:A\mapsto gA$,现在如果$A$是该映射的不动点$gA=A$,则有$A\pi_1(g)=\pi_2(g)A$。如果对每一个$g$,$A$都是其不动点,那么我们就称$A$是表示$\pi_1^*\otimes\pi_2$的不动点。那么对于群表示$(\pi_1,V)$和$(\pi_2,W)$,$A$就是一个缠结映射。

\para 利用Schur引理,假若$\pi_1$, $\pi_2$都是不可约的且不等价的,则群表示$(\pi_1^*\otimes\pi_2,\Hom(V,W))$的不动点集平凡(即只有$A=0$这一个不动点)。

假如$\pi$, $\rho$都是不可约的且不等价的,现在我们考虑线性映射$A\in \Hom(V,W)$在紧Lie群上的平均\footnote{在有限群上也有平均,所以下面一套对有限群也成立。}$\bar{A}=\int_G \pi^*\otimes\rho(g)A \dd g$,显然$\bar{A}$是一个不动点,因此由Schur引理,
\[
	\bar{A}=\int_G \rho(g)A\pi(g)^{-1}\dd g=0
\]
对任意的线性映射$A\in \Hom(V,W)$都成立。

同样,如果$\pi$, $\rho$等价,此时不妨就直接记$\pi=\rho$.因为$\bar{A}$是一个缠结映射,那么由Schur引理,$\bar{A}=\lambda_{\bar{A}} I$,所以
\[
	\int_G \pi(g)A\pi(g)^{-1}\dd g=\lambda_{\bar{A}} I
\]
对任意的线性映射$A\in \Hom(V,V)$都成立,但是由于$\lambda_{\bar{A}}$并不已知,所以这公式用着不是很方便。为了求出$\lambda_{\bar{A}}$,两边求迹,就有$\tr(A)=\lambda_{\bar{A}} \dim V$,即$\lambda_{\bar{A}}=\tr(A)/\dim V$,所以我们也可以将上式写作更实用的形式
\[
	\bar{A}=\int_G \pi(g)A\pi(g)^{-1}\dd g=\frac{\tr(A)}{\dim V}I.
\]

\para 现在,为了做一点小小的计算,我们将固定矢量空间的基,此时任意的线性算子都可以看做矩阵。记$E_{ij}$为只在$(a,b)$位置为1,其他位置为0的矩阵。很容易可以计算得到$E_{ij}AE_{kl}=A_{jk}E_{il}$.现在对
\[
	\bar{A}=\int_G \rho(g)A\pi(g)^{-1}\dd g
\]
左乘$E_{ij}$,右乘$E_{kl}$,并令$A=E_{ab}$,可以得到
\[
	\int_G E_{ij}\rho(g)E_{ab}\pi(g)^{-1}E_{kl}\dd g=\int_G \rho(g)_{ja}E_{ib}\pi(g)^{-1}E_{kl}\dd g=\int_G (\rho(g))_{ja}\left(\pi(g)^{-1}\right)_{bk}E_{il}\dd g,
\]
由于$\pi$是幺正表示,所以$\left(\pi(g)^{-1}\right)_{bk}=\left(\pi(g)^{\dag}\right)_{bk}=\pi(g)_{kb}^*$,并记$\pi_{ia}:g\to \pi(g)_{ia}$为分量函数,则利用复函数内积的写法,上式可以写作
\[
	E_{ij}\overline{E_{ab}}E_{kl}=\left(\rho_{ja},\pi_{kb}\right)E_{il}.
\]

\para 设$\rho$和$\pi$都是不可约的,对于$\rho$和$\pi$不等价的情况,我们有$\overline{E_{ab}}=0$,所以$\left(\rho_{ja},\pi_{kb}\right)=0$.类似地,对于$\rho=\pi$的情况,我们有$\overline{E_{ab}}=\tr(E_{ab})I/\dim V=\delta_{ab}I/\dim V$,所以
\[
	\left(\pi_{ja},\pi_{kb}\right)E_{il}=E_{ij}\overline{E_{ab}}E_{kl}=\frac{\delta_{ab}}{\dim V}E_{ij}IE_{kl}=\frac{\delta_{ab}\delta_{jk}}{\dim V}E_{il},
\]
或者
\[
	\left(\pi_{ja},\pi_{kb}\right)=\frac{\delta_{ab}\delta_{jk}}{\dim V}.
\]

\para 令$\rho:G\to \operatorname{GL}(V)$是一个$G$的复表示,那么复值函数$\chi_\rho:G\xrightarrow{\rho}\operatorname{GL}(V)\xrightarrow{\tr}\cc:g\mapsto \tr(\rho(g))$被称为$\rho$的特征标。
\endpara

可以注意到特征标的一些基本运算法则:
\[
	\chi_\rho(gh)=\chi_\rho(hg),\quad \chi_\rho(h^{-1}gh)= \chi_\rho(g).
\]
这些都来自于迹的运算法则。以及
\[
	\chi_\rho(e)=\tr(I)=\dim(V).
\]

\para 对于紧Lie群(有限群)的两个不可约的不等价表示$\rho$和$\pi$,我们有
\[
(\chi_\rho, \chi_\pi)=\sum_{i,j}\left(\rho_{ii},\pi_{jj}\right)=0,
\]
以及对于一个不可约表示
\[
(\chi_\pi, \chi_\pi)=\sum_{i,j}(\pi_{ii},\pi_{jj})=\sum_{i,j}\frac{\delta_{ij}^2}{\dim V}=\sum_{i}\frac{1}{\dim V}=1.
\]
对于有限群,后者我们通常写作
\[
	\sum_{g\in G} |\chi_\pi(g)|^2=|G|.
\]

令$\hat{G}$是紧Lie群(有限群)$G$的复不可约表示的等价类。任取$\rho$, $\pi\in\hat{G}$,前面的推论现在可以总结为$(\chi_\rho,\chi_\pi)=\delta_{\rho\pi}$.

由于是紧Lie群(有限群),有限维表示必然是完全可约的,就是说任何一个有限维表示能写作$\rho=\bigoplus_{\pi\in\hat{G}}m(\rho,\pi)\pi$,其中$m(\rho,\pi)$是乘数,是一个自然数,就是说分解出来的等价的不可约表示$\pi$的个数。直接计算就可以得到$m(\rho,\pi)=(\chi_\rho,\chi_\pi)$,以及
\begin{equation}
	(\chi_{\rho_1},\chi_{\rho_2})=\sum_{\pi\in\hat{G}}m(\rho_1,\pi)m(\rho_2,\pi).
\end{equation}

{\pro 一个紧Lie群(有限群)的复有限维表示$\pi$是不可约的当且仅当$(\chi_{\pi},\chi_{\pi})=1$. \endpro}

\proof 前面说过,紧Lie群(有限群)的一个不可约的复有限维表示$\pi$满足$(\chi_\pi, \chi_\pi)=1$. 反过来,如果$(\chi_\pi, \chi_\pi)=1$,则$(\chi_{\pi},\chi_{\pi})=\sum_{\rho\in\hat{G}}m(\rho,\pi)^2=1$,因此存在一个$\rho\in\hat{G}$使得$m(\rho,\pi)=1$,而其他的$\pi\in\hat{G}$都有$m(\rho,\psi)=0$.\endproof

{\pro 如果$G$是一个紧Lie群(有限群),那么两个表示$\pi_1$, $\pi_2$等价当且仅当$\chi_{\pi_1}=\chi_{\pi_2}$.\endpro}

\proof 假若$\pi_1$和$\pi_2$等价,那么存在$A$使得
$A\pi_1(g)=\pi_2(g)A$,于是$\tr(\pi_1(g))=\tr(A^{-1}\pi_2(g)A)=\tr(\pi_2(g))$.反过来,如果两个表示$\pi_1$, $\pi_2$特征标相等,则$	m(\pi_1,\pi)=(\chi_{\pi_1},\chi_\pi)=(\chi_{\pi_2},\chi_\pi)=m(\pi_2,\pi)$,对任意的$\pi\in\hat{G}$都成立,因此$\pi_1$, $\pi_2$等价。\endproof

\para 设$G$是一个群,而$g\in G$,则同态$\mathrm{Ad}(g):h\mapsto ghg^{-1}$被称为共轭作用。共轭作用衡量了一个群的可交换性,这是因为,如果$\mathrm{Ad}(g)(h)=h$,则$gh=hg$.

固定一个$h\in G$,考虑子集$H_h=\mathrm{Ad}(G)(h)$,这样的子集被称为共轭作用的一条轨道(或者在这里叫做共轭类)。两条不同的轨道是不能相交的,这是因为如果$H_h$和$H_g$有一个相交点$k$,分别记作$k=php^{-1}=qgq^{-1}$,则$g=q^{-1}ph(q^{-1}p)^{-1}$,所以$H_h=H_g$.

\para 回到表示,由于$\chi_\rho(h^{-1}gh)= \chi_\rho(g)$,所以$\chi_\rho$可以定义到共轭类的集合上面。对于有限群,所有共轭类集合上的复值函数构成一个复有限维矢量空间,他上面有着从$G$的复值函数那儿继承的内积,显然,他的维度等于共轭类的个数。

由于$(\chi_\rho,\chi_\pi)=\delta_{\rho\pi}$,所以$\{\chi_\rho\,:\, \rho\in \hat{G}\}$构成这个矢量空间的一组正交线性无关组。由于维度是极大的线性无关组的个数,故有限群不可约表示的数目小于等于其共轭类的数目。后面引入所谓的正规表示,我们可以用它证明一个更强的命题:对于有限群,不可约表示的数目等于其共轭类的数目!

\para 对于有限群$G=\{g_1,\cdots,g_n\}$,我们定义一个矢量空间
\[
	\cc [G]=\cc\langle g_1\rangle \oplus \cdots \oplus \cc\langle g_n\rangle,
\]
这是矢量空间同时拥有幺环的结构,我们通过
\[
\sum_{i=1}^n a_i g_i\sum_{j=1}^n b_j g_j=\sum_{i,j=1}^n a_ib_i(g_ig_j)
\]
来定义环乘法。

有了乘法,通过
\[
	R(g)\sum_{i=1}^n c_i g_i=\sum_{i=1}^n c_i(gg_i),
\]
可以定义出线性映射$R(g):G\to \operatorname{GL}(\cc [G])$,此时$(R,\cc [G])$自然地成为$G$的一个表示,称为正规表示,他的特征标记作$\chi_R$. 

实际上,正规表示的值域我们可是适当缩小。注意到环$\cc [G]$可以看成自己的右$\cc [G]$-模,且任取$v$, $w\in \cc [G]$,我们有
\[
	R(g)(vw)=gvw=(gv)w=(R(g)v)w.
\]
所以$R(g)$还是一个右$\cc [G]$-模同构。

可以验证,$\cc [G]$上所有右$\cc [G]$-模同态也构成一个矢量空间,我们记作$\mathrm{End}(\cc [G])$. 他通过复合也构成了一个含幺环。

{\pro 左乘$l_v:w\mapsto vw$建立了矢量空间$\cc [G]$与$\mathrm{End}(\cc [G])$的同构,同时也建立了环之间的同构。\endpro}

\proof
	线性性显然。先看看是不是单射,如果$l_v=0$,则$v=ve=l_ve=0$,所以这是单的。然后只要检验是不是满射就行了。设$\theta\in \operatorname{GL}(\cc [G])$,我们考虑$\theta(e)=v\in \cc [G]$,因为
	\[
		\theta(g)=\theta(eg)=\theta(e)g=vg=l_vg,
	\]
	因此$\theta=R(v)$. 满性证明完毕。于是矢量空间同构自然。

	对于环结构,由于$l_{vw}=l_v\circ l_w$,所以左乘也是一个环同态,而且既单又满故而是一个环同构。
\endproof

\para 正规表示的特征标比较简单。显然的一点是$\chi_R(e)=\dim (\cc [G])=|G|$,对于$g\neq e$,矩阵分量$g_{hh}$的意义是$g:h\mapsto gh$后对应到$h$的分量,所以$g_{hh}=0$,故而此时$\chi_R(g)=\tr(g:h\mapsto gh)=0$.

直接计算就有$(\chi_{R},\chi_{R})=|G|$,利用上面的判据,可以断言这不是一个不可约表示(除非是平凡群)。将其进行分解,
\[
	R=\bigoplus_{\pi\in\hat{G}}m(R,\pi)\pi,
\]
其中
\[
	m(R,\pi)=(\chi_R,\chi_\pi)=\frac{1}{|G|}\sum_{g\in G}\chi_R(g)\chi_{\pi}^*(g)=\chi_{\pi}^*(e)=\dim V_{\pi},
\]
其中$V_\pi$是分解出来的不可约表示$\pi$作用的子空间。因而利用式(\theequation),我们有
\[
	|G|=(\chi_{R},\chi_{R})=\sum_{\pi\in\hat{G}}m(R,\pi)m(R,\pi)=\sum_{\pi\in\hat{G}}(\dim V_{\pi})^2.
\]
这就使得有限群的表示可能变成一个组合的问题。比如对称群$S_3$,$|S_3|=3!=6=2^2+1^2+1^2$.

\para 由于
\[
	\cc [G]=\bigoplus_{\pi\in\hat{G}}m(R,\pi)V_\pi=\bigoplus_{\pi\in\hat{G}}\dim(V_\pi)V_\pi,
\]
考虑环$\mathrm{End}(\cc [G])$的中心$Z_{\mathrm{End}(\cc [G])}$,即$\{\psi\in \mathrm{End}(\cc [G])\,:\, \forall \phi\in \mathrm{End}(\cc [G]),\, \phi\circ\psi=\psi\circ\phi\}$,显然,这还是一个矢量空间。

不妨将$\mathrm{End}(\cc [G])$看成$\cc [G]$上所有矩阵的全体,任取$C\in Z_{\mathrm{End}(\cc [G])}$,我们利用$CE_{ii}=E_{ii}C$以及$C(E_{ij}+E_{ji})=(E_{ij}+E_{ji})C$,可以断言$Z_{\mathrm{End}(\cc [G])}$中所有矩阵具有$\sum_{\pi\in \hat{G}}c_\pi I_\pi$的形式,其中$I_\pi$是$V_\pi$的单位矩阵而$c_\pi\in\cc$,所以$\dim (Z_{\mathrm{End}(\cc [G])})=|\hat{G}|$.

{\thm 对于有限群$G$,其共轭类的个数等于其有限维不可约复表示的个数。\endthm}

\proof 设$G$的共轭类为$\{K_i\,:\, 1\leq i\leq k\}$. 我们考虑$Z_{\cc [G]}$,任取$a\in Z_{\cc [G]}$,他对任意的$h\in G$成立$hah^{-1}=a$. 

考虑分解$a=\sum_i a_ig_i$,我们有
\[
	\sum_i a_ig_i=\sum_i a_ihg_ih^{-1},
\]
如果$g_i$和$g_j$处于同一个共轭类里面。由于左边是不变的,而右边总存在$h$使得$hg_ih^{-1}=g_j$,由线性无关性,可以得知$a_i=a_j$,继而可以断言处于同一个共轭类里面的$g_i$的系数都是相同的。令$z_i=\sum_{g\in K_i}g$,所以$a$可以作如下分解
\[
	a=\sum_{i=1}^kd_iz_i,
\]
其中$d_i\in \cc$. 反之,如上形式的$a$必然处于$Z_{\cc [G]}$之中。所以$\dim (Z_{\cc [G]})=k$.

最后,由于$\cc [G]$与$\mathrm{End}(\cc [G])$的同构,所以$k=\dim (Z_{\cc [G]})=\dim (Z_{\mathrm{End}(\cc [G])})=|\hat{G}|$.\endproof

\para 作为推论,对于有限群,$\{\chi_\rho\,:\, \rho\in \hat{G}\}$构成了共轭类上复值函数空间的一组正交归一基。

\subsection*{伴随表示}

Lie代数本身就是一个矢量空间,我们这里感兴趣的是表示是一种矢量空间为$\lag$的表示。这需要从伴随作用开始。先说几个记号,设$l_g:h\mapsto gh$以及$r_g:h\mapsto hg$分别是左平移与右平移,他们都是群的自同构。

\para 设$\lag$是一个Lie代数,那么不难看出$\lag$的自同构群
\[\mathrm{Aut}_{\mathrm{Lie}}(\lag)=\{T\in \operatorname{GL}(\lag)\,:\,T[u,v]=[Tu,Tv],\,\forall u,\,v\in\lag\}\]
构成一个Lie群,他的Lie代数是
\[\mathfrak{gl}_{\mathrm{Lie}}(\lag)=\{T\in \mathfrak{gl}(\lag)\,:\,T[u,v]=[Tu,v]+[u,Tv],\,\forall u,\,v\in\lag\}.\]
由于$[X,*]:Y\mapsto [X,Y]\in \lag$,且根据Jacobi恒等式,我们可以得知$[X,*]\in \mathfrak{gl}_{\mathrm{Lie}}(\lag)$,这被称为Lie代数的内导子。

\para 设$G$是一个Lie群,他的Lie代数是$\lag$。Lie代数的伴随来自于Lie群的伴随$\mathbf{Ad}(g):h\mapsto ghg^{-1}$或者$\mathbf{Ad}(g)=r_{g^{-1}}\circ l_g=l_g\circ r_{g^{-1}}$在单位元上的导数$\mathrm{Ad}_g=\mathbf{Ad}(g)_{*e}:T_eG\to T_eG$,但注意到Lie代数$\lag$就是Lie群在单位元的切空间$T_eG$,所以$\mathrm{Ad}_g:\lag\to \lag$。因为$\mathbf{Ad}(g)$是Lie群的一个自同构,所以$\mathrm{Ad}_g:\lag\to\lag$是线性空间的同构,即$\mathrm{Ad}_g\in \operatorname{GL}(\lag)$.

利用指数函数,把$\mathrm{Ad}_g$和$\mathbf{Ad}(g)$之间的微分关系联系起来即,$\mathbf{Ad}(g)\exp(X)=\exp(\mathrm{Ad}_gX)$成立。

\para 我们也可以通过左不变矢量场来描述Lie代数,这时候最好把$\mathrm{Ad}_g$理解成$(l_g)_*\circ (r_{g^{-1}})_*$,此时我们有断言:如果$X$是$G$上的左不变矢量场,那么$\mathrm{Ad}_gX$对于任意$g\in G$也是左不变矢量场:注意到左作用和右作用是可交换的,因此他们的导数也是可以交换的,那么
\[
	(l_h)_*(\mathrm{Ad}_gX)=(l_h)_*\circ (l_g)_*\circ (r_{g^{-1}})_*(X)=(r_{g^{-1}})_*(X)=(r_{g^{-1}})_*\circ (l_g)_*(X)=\mathrm{Ad}_gX.
\]

\para $\mathrm{Ad}_g$是Lie代数$\lag$的一个自同构,即$\mathrm{Ad}_g\in \mathrm{Aut}_{\mathrm{Lie}}(\lag)$.\endpara

线性空间同构已经是清楚的了,下面只要证明他是Lie代数同态即可,为此只要检验$\mathrm{Ad}_g([X,Y])=[\mathrm{Ad}_gX,\mathrm{Ad}_gY]$,其中$X$和$Y$都是左不变矢量场。

由于$\mathrm{Ad}_g=(l_g)_*\circ (r_{g^{-1}})_*$,且$l_g$与$r_{g^{-1}}$作为Lie群的自同构,有$(l_g)_*[X,Y]=[(l_g)_*X,(l_g)_*Y]$和$(r_{g^{-1}})_*[X,Y]=[(r_{g^{-1}})_*X,(r_{g^{-1}})_*Y]$成立,于是$\mathrm{Ad}_g([X,Y])=[\mathrm{Ad}_gX,\mathrm{Ad}_gY]$自然得证。

\para 到目前为止,我们看到$\mathrm{Ad}_g\in \mathrm{Aut}_{\mathrm{Lie}}(\lag)$,所以我们可以构造映射$\mathrm{Ad}:g\mapsto \mathrm{Ad}_g$. 可以检验$\mathrm{Ad}:G\to \mathrm{Aut}(\lag)$是一个Lie群同态:
\[\mathrm{Ad}_g\circ \mathrm{Ad}_h=(l_g)_*\circ (r_{g^{-1}})_*\circ (l_h)_*\circ (r_{h^{-1}})_*=(l_g)_*\circ (l_h)_*\circ (r_{g^{-1}})_*\circ (r_{h^{-1}})_*=(l_{gh})_*\circ (r_{(hg)^{-1}})_*=\mathrm{Ad}_{gh},\]
这个表示被称为Lie群的伴随表示。

% 第三个设$v\in T_hG$,因此$(r_g)_*v\in T_{hg}G$,于是
% \[
% 	(r_g)^*\omega_G(v)=\omega_G((r_g)_*v)=(l_{{hg}^{-1}})_*(r_g)_*
% 	=(l_{{g}^{-1}})_*(r_g)_*(l_{{h}^{-1}})_*v=\mathrm{Ad}(g^{-1})\omega_G.
% \]

{\pro 令$G$的Lie代数为$\lag$,则$\mathrm{Ad}:G\to \mathrm{Aut}_{\mathrm{Lie}}(\lag)$在$e$的导数$\ad=\mathrm{Ad}_{*e}:\lag\to \mathfrak{gl}_{\mathrm{Lie}}(\lag)$满足$\ad(X)Y=[X,Y]$. 这被称为Lie代数的伴随表示。\endpro}

\proof 因为$\mathrm{Ad}$是$G$上矢量值的函数,证明就是很直接地要去计算
\[
	\ad(X)Y=\left.\frac{\dd}{\dd t}\right|_{t=0}\mathrm{Ad}_{\exp(tX)}Y,
\]
找一个$G$上的光滑函数$f$,我们计算其单位元处的导数,注意到
\[
	Yf=\left.\frac{\dd}{\dd u}\right|_{u=0}f(\exp(uY)),
\]
以及$(\mathrm{Ad}_{g}Y)f=Y(f\circ l_g\circ r_{g^{-1}})$,所以
\[
	\ad(X)Yf=\left.\frac{\dd}{\dd t}\right|_{t=0}(\mathrm{Ad}_{\exp(tX)}Y)f=\left.\frac{\dd}{\dd t}\frac{\dd}{\dd u}\right|_{u=t=0}f\bigl(\exp(tX)\exp(uY)\exp(-tX)\bigr),
\]
注意到对$t$求导的时候,可以利用多元实函数$F(t_1,t_2)$的求导等式
\[
	\left.\frac{\dd}{\dd t}\right|_{t=0}F(t,t)=\frac{\partial F}{\partial t_1}(0,0)+\frac{\partial F}{\partial t_2}(0,0),
\]
所以我们得到了
\[
\begin{split}
	&\left.\frac{\dd}{\dd t}\frac{\dd}{\dd u}\right|_{u=t=0}f\bigl(\exp(tX)\exp(uY)\exp(-tX)\bigr)\\
	=&\left.\frac{\dd}{\dd t_1}\frac{\dd}{\dd u}\right|_{u=t_1=0}f\bigl(\exp(t_1X)\exp(uY)\bigr)-\left.\frac{\dd}{\dd t_2}\frac{\dd}{\dd u}\right|_{u=t_2=0}f\bigl(\exp(uY)\exp(t_2X)\bigr)\\
	=&(XY-YX)f(e),
\end{split}
\]
所以$\ad(X)Y=[X,Y]$.\endproof

\para 对于一般线性群,前面已经计算过了$(l_g)_*=l_g$,那么同样$(r_g)_*=r_g$,所以
\[
	\mathrm{Ad}_g=(l_g)_*(r_{g^{-1}})_*=l_gr_{g^{-1}}.
\]
那么
\[
	\mathrm{Ad}_g(v)=(l_g)_*(r_{g^{-1}})_*v=l_gr_{g^{-1}}v=gvg^{-1}.
\]
我们现在求他的Lie代数,考虑$u,v\in \lag$,我们令$u(t)$是一个以$u$为初速度的单参子群,那么我们有
\[
	\frac{\dd}{\dd t}\bigl(\mathrm{Ad}_{u(t)}(v)\bigr)=u'(t)vu^{-1}(t)+u(t)v(u^{-1}(t))'=u'(t)vu^{-1}(t)-u(t)vu^{-1}(t)u'(t)u^{-1}(t).
\]
然后令$t=0$,那么$u(0)=u^{-1}(0)=I$,而$u'(0)=u$,那么就得到了单位元处的切矢量,也就是Lie代数$\ad(u)v=uv-vu=[u,v]$.

% 类似的手段譬如对$T(t)[u,v]=[T(t)u,T(t)v]$求个导,然后在$t=0$处的值为
% \[
% 	T'(0)[u,v]=[T'(0)u,T(0)v]+[T(0)u,T'(0)v],
% \]
% 注意到$T(0)$是恒等变换,而$T'(0)$就是我们需要的Lie代数$B$,他需要满足的关系就是
% \[
% 	B[u,v]=[Bu,v]+[u,Bv],
% \]
% 其显然是Lie代数上面的一个导子。
\newpage

\section{一些矩阵恒等式}

矩阵幂被定义为
\[
	\exp(A)=\sum_{n=0}^\infty \frac{1}{n!}A^n,
\]
其中定义$A^0=I$. 这个级数对任意的$A$都收敛,因为固定一个$A$,他总是有界的,设$a$是他的一个上界,而$\exp(a)$总是收敛的,所以$\exp(A)$也是收敛的。

对于对角矩阵$\Lambda=\mathrm{diag}(\lambda_1$, $\cdots$, $\lambda_n)$,他的矩阵幂容易通过定义证明为$\exp(\Lambda)=\mathrm{diag}(\exp(\lambda_1)$, $\cdots$, $\exp(\lambda_n))$. 利用$DA^nD^{-1}=\bigl(DAD^{-1}\bigr)^n$,我们有$\exp(DAD^{-1})=D\exp(A)D^{-1}$. 所以我们可以计算出在$A$是可对角化的时候的矩阵幂。

如果$A$是幂零的,比如$A^k=0$,此时
\[
	\exp(A)=\sum_{n=0}^{k-1} \frac{1}{n!}A^n,
\]
也是可以计算的。

\lem 如果$A$是可逆矩阵而$B$是幂零矩阵且$AB=BA$,则$\det(A+tB)=\det(A)$,其中$t$是一个任意常数。

\proof 现在
\[
	\det(A+tB)=\det(A)\det(I+tA^{-1}B),
\]
由于$AB=BA$或者$A^{-1}B=BA^{-1}$,所以$(A^{-1}B)=(A^{-1})^n B^n$,这就推出$A^{-1}B$也是幂零的。从Jordan标准型,幂零矩阵$A^{-1}B$相似于一个对角元素全是$0$上三角矩阵,因此$\det(I+tA^{-1}B)=1$,这也就推出了
\[
	\det(A+tB)=\det(A),
\]
因此$A+tB$也是可逆的。\endproof

\lem 如果$A$是幂零矩阵,则$\det(\exp(A))=1$.

\proof 应用上面的引理,$\det(I+A)=\det(I)=1$,再应用一次
\[
	\det(I+A+A^2/2)=\det(I+A)=\det(I)=1,
\]
如果$A^n=0$,则反复应用$n$次就得到了结论。\endproof

{\pro 指数映射联系了矩阵的迹与行列式,$\det(\exp(A))=\exp(\tr(A))$.\endpro}

\proof 从Jordan标准型,我们只要考虑如下矩阵即可
\[
	A=\begin{pmatrix}
	\lambda &1&&&\\
	&\lambda&1&&\\
	&&\ddots&\ddots&\\
	&&&\lambda&1\\
	&&&&\lambda
	\end{pmatrix}=\lambda I+ B,
\]
其中$B$是一个幂零矩阵,因为$[\lambda I,B]=0$,所以
\[
	\exp(A)=\exp(\lambda I)\exp(B),
\]
以及
\[
	\det \left(\exp(A)\right)=\det\left(\exp(\lambda I)\right)\det\left(\exp(B)\right).
\]
利用上面的引理,$\det\left(\exp(B)\right)=1$,所以
\[
\det \left(\exp(A)\right)=\det\left(\exp(\lambda I)\right)=\exp(n\lambda)=\exp(\tr A).
\]
\endproof

{\pro 当$A$和$B$属于矩阵群的Lie代数的时候,$\tr(\ad(A)\ad(B))=2n \tr(AB) -2\tr(A)\tr(B)$. \endpro}

\proof 选取$M(\cc^n)$中的$\{E_{ij}\,:\, 1\leq i,j\leq n\}$作为一组基,我们来计算这组基下$A=\ad(E_{ij})\ad(E_{kl})$的矩阵$A_{pq;rs}$,使用矩阵的定义
\[
\ad(E_{ij})\ad(E_{kl})E_{rs}=\sum_{r,s}A_{pq;rs}E_{pq},
\]
左边可以计算出来为
\[
[E_{ij},[E_{kl},E_{rs}]]=\delta_{lr}\delta_{jk}E_{is}+\delta_{ks}\delta_{li}E_{rj}-\delta_{lr}\delta_{si}E_{kj}-\delta_{ks}\delta_{jr}E_{il},
\]
与右侧比较就可以得到
\[
	A_{pq;rs}=\delta_{lr}\delta_{jk}\delta_{ip}\delta_{sq}+\delta_{ks}\delta_{li}\delta_{rp}\delta_{jq}-\delta_{lr}\delta_{si}\delta_{kp}\delta_{jq}-\delta_{ks}\delta_{jr}\delta_{ip}\delta_{lq},
\]
以及
\[
	\tr(\ad(E_{ij})\ad(E_{kl}))=\sum_{p,q}A_{pq;pq}=2n \delta_{kj}\delta_{il}-2\delta_{ij}\delta_{kl}.
\]
注意到$\tr(E_{ij}E_{kl})=\delta_{kj}\delta_{il}$,所以
\[
\tr(\ad(E_{ij})\ad(E_{kl}))=2n \tr(E_{ij}E_{kl}) -2\delta_{ij}\delta_{kl}.
\]
对于一般的矩阵$A$和$B$,利用展开$A=\sum_{i,j}A_{ij}E_{ij}$和$B=\sum_{k,l}A_{kl}E_{kl}$,以及上面的式子可以得到
\[
\tr(\ad(A)\ad(B))=2n \tr(AB) -2\tr(A)\tr(B).
\]\endproof

在我们处理$\tr(A)=\tr(B)=0$的情况的时候,有$\tr(\ad(A)\ad(B))=2n \tr(AB)$. 

\clearpage
\section{$\mathfrak{su}(N)$一组常用的基}

$\mathfrak{su}(N)$的一组常用的基选作
\[
	\frac{E_{ij}+E_{ji}}{2},\quad \frac{i(E_{ij}-E_{ji})}{2},\quad H_k=\frac{1}{\sqrt{2k(k+1)}}\sum_{i=1}^{k} ie_{i}=\frac{1}{\sqrt{2k(k+1)}}
	\mathrm{diag}(1,1,\cdots,1,-k,0,\cdots,0),
\]
其中$1\leq i<j\leq N$以及$1\leq k \leq N-1$,这三类生成元构成的集合记作$\Omega$,其中子集$\{H_k\,:\,1\leq k \leq N-1\}$张成了$\mathfrak{su}(N)$的Cartan子代数。直接的计算可以得到
\[
	\tr(H_iH_j)=\frac{\delta_{ij}}{2}.
\]

下面我们用$\mathcal{T}^A$来表示一个生成元,$A$是用来标记$\Omega$的中元素的一个指标。

{\pro 如下等式成立
\[
	\sum_{A\in \Omega}\mathcal{T}^A_{ab}\mathcal{T}^A_{cd}=\frac{1}{2}\left(\delta_{ad}\delta_{bc}-\frac{1}{N}\delta_{ab}\delta_{cd}\right),
\]
其中$A$走遍所有的生成元。\endpro}

\proof 我们自然分成三类求和,为记号上的方便,我们用$[ij]$来表示$\delta_{ij}$. 第一类对应
\begin{equation}
\label{1}
\begin{split}
	\frac{1}{4}\sum_{1\leq i < j\leq N}&(E_{ij}+E_{ji})_{ab}(E_{ij}+E_{ji})_{cd}\\
	&=\frac{1}{8}\left(\sum_{1\leq i,j\leq N}(E_{ij}+E_{ji})_{ab}(E_{ij}+E_{ji})_{cd}-\sum_{i=1}^N(E_{ii}+E_{ii})_{ab}(E_{ii}+E_{ii})_{cd}\right)\\
	&=\frac{1}{8}\left(\sum_{1\leq i,j\leq N}\bigl([ia][bj]+[ja][ib]\bigr)\bigl([ic][dj]+[jc][id]\bigr)-4\sum_{i=1}^N[ia][ib][ic][id]\right)\\
	&=\frac{1}{8}\left(2\bigl([ac][bd]+[ad][bc]\bigr)-4[ab][ac][ad]\right)\\
	&=\frac{1}{4}\bigl([ac][bd]+[ad][bc]\bigr)-\frac{1}{2}[ab][ac][cd],
\end{split}
\end{equation}
第二类对应
\begin{equation}
\label{2}
\begin{split}
	-\frac{1}{4}\sum_{1\leq i < j\leq N}&(E_{ij}-E_{ji})_{ab}(E_{ij}-E_{ji})_{cd}\\
	&=\frac{-1}{8}\left(\sum_{1\leq i,j\leq N}(E_{ij}-E_{ji})_{ab}(E_{ij}-E_{ji})_{cd}-\sum_{i=1}^N(E_{ii}-E_{ii})_{ab}(E_{ii}-E_{ii})_{cd}\right)\\
	&=\frac{-1}{8}\left(\sum_{1\leq i,j\leq N}\bigl([ia][bj]-[ja][ib]\bigr)\bigl([ic][dj]-[jc][id]\bigr)\right)\\
	&=\frac{1}{4}\bigl([ad][bc]-[ac][bd]\bigr),
\end{split}
\end{equation}
第三类对应
\begin{equation}
\label{4}
\begin{split}
	\sum_{k=1}^{N-1}(H_k)_{ab}(H_k)_{cd}&=\sum_{k=1}^{N-1}\frac{1}{2k(k+1)}\left(\sum_{i=1}^k i(e_{i})_{ab}\right)\left(\sum_{i=1}^k i(e_{i})_{cd}\right)\\
	&=\sum_{k=1}^{N-1}\frac{[ab][cd]}{2k(k+1)}\left(\sum_{i=1}^k i([ia]-[i+1,a])\right)\left(\sum_{i=1}^k i([ic]-[i+1,c])\right),
\end{split}
\end{equation}
剩下的部分,就是所有计算中最难的一部分了,我们要去算上式中的
\begin{equation}
\sum_{k=1}^{N-1}\frac{1}{2k(k+1)}\left(\sum_{i=1}^k i\bigl([ia]-[i+1,a]\bigr)\right)\left(\sum_{i=1}^k i\bigl([ic]-[i+1,c]\bigr)\right).
\end{equation}

注意到如下等式成立:
\[
	i\bigl([ia]-[i+1,a]\bigr)=
	\begin{cases}
	0,& i+1<a \text{ or } i>a;\\
	-(a-1),&i+1=a;\\
	a,&i=a,
	\end{cases}
\]
所以
\[
	\sum_{i=1}^ki\bigl([ia]-[i+1,a]\bigr)=
	\begin{cases}
	0,& k+1<a;\\
	-(a-1),&k=a-1;\\
	1,&k\geq a.
	\end{cases}
\]

由于式$(\theequation)$是关于$a$和$c$对称的,不妨先假设$a\leq c<N$,利用上面的式子,式$(\theequation)$将分解成
\[
\left(\sum_{k=1}^{c-2}+\sum_{k=c-1}^{c-1}+\sum_{k=c}^{N-1}\right)\frac{1}{2k(k+1)}\left(\sum_{i=1}^k i\bigl([ia]-[i+1,a]\bigr)\right)\left(\sum_{i=1}^k i\bigl([ic]-[i+1,c]\bigr)\right).
\]
三个部分,第一个部分是零,第二个部分是
\[
	-\frac{1}{2c}\left(\sum_{i=1}^{c-1} i\bigl([ia]-[i+1,a]\bigr)\right)=\frac{1}{2}
	\begin{cases}
	1-1/c,& a=c;\\
	-1/c,&a<c;
	\end{cases}=\frac{1}{2}\left([ac]-\frac{1}{c}\right),
\]
第三个部分是
\[
	\sum_{k=c}^{N-1}\frac{1}{2k(k+1)}=\frac{1}{2c}-\frac{1}{2N},
\]
所以
\[
	(\theequation)=\frac{1}{2}\left([ac]-\frac{1}{c}+\frac{1}{c}-\frac{1}{N}\right)=\frac{1}{2}\left([ac]-\frac{1}{N}\right).
\]

对于$a\leq c=N$的情况,式$(\theequation)$将分解成
\[
\left(\sum_{k=1}^{c-2}+\sum_{k=c-1}^{c-1}\right)\frac{1}{2k(k+1)}\left(\sum_{i=1}^k i\bigl([ia]-[i+1,a]\bigr)\right)\left(\sum_{i=1}^k i\bigl([ic]-[i+1,c]\bigr)\right),
\]
第一个部分是零,第二个部分在上面的第二个部分里面令$c=N$就可以,因此
\[
	(\theequation)=\frac{1}{2}\left([ac]-\frac{1}{N}\right).
\]

综合上面两个计算,不管$a$和$c$是什么,我们都得到了
\begin{equation}
	\sum_{k=1}^{N-1}\frac{1}{2k(k+1)}\left(\sum_{i=1}^k i\bigl([ia]-[i+1,a]\bigr)\right)\left(\sum_{i=1}^k i\bigl([ic]-[i+1,c]\bigr)\right)=\frac{1}{2}\left([ac]-\frac{1}{N}\right).
\end{equation}
将式$(\theequation)$带入\eqref{4},有
\begin{equation}
\label{3}
\sum_{k=1}^{N-1}(H_k)_{ab}(H_k)_{cd}=\frac{[ab][cd]}{2}\left([ac]-\frac{1}{N}\right),
\end{equation}
最后,将式\eqref{1},式\eqref{2}和式\eqref{3}全部加起来,就得到了
\[
\begin{split}
	\sum_{A\in \Omega}\mathcal{T}^A_{ab}\mathcal{T}^A_{cd}&=\frac{1}{4}\bigl([ad][bc]-[ac][bd]\bigr)+\frac{1}{4}\bigl([ac][bd]+[ad][bc]\bigr)-\frac{1}{2}[ab][ac][cd]+\frac{[ab][cd]}{2}\left([ac]-\frac{1}{N}\right)\\
	&=\frac{1}{2}\left([ad][bc]-\frac{1}{N}[ab][cd]\right)\\
	&=\frac{1}{2}\left(\delta_{ad}\delta_{bc}-\frac{1}{N}\delta_{ab}\delta_{cd}\right).
\end{split}
\]
\endproof


\clearpage
\section{$\mathfrak{sl}(N,\cc)$的Killing形式非退化的证明}

选取$\mathfrak{sl}(N,\cc)$的$(N^2-1)$个基如下:$\{E_{ij}:1\leq i\neq j \leq N\}$以及$\{H_k\,:\, 1\leq k \leq N-1\}$,前者有$N^2-N$个,后者有$N-1$个。

{\pro $\mathfrak{sl}(N,\cc)$的Killing形式非退化。\endpro}

\proof Killing形式非退化等价于Killing形式的矩阵可逆,或者于Killing形式的矩阵的行列式非零。下面我们就算出$(N^2-1)\times (N^2-1)$个Killing形式的矩阵的分量。

将$\{E_{ij}:1\leq i\neq j \leq N\}$排在前面,而$\{H_k\,:\, 1\leq k \leq N-1\}$排在后面。Killing形式的矩阵应该长成如下形式
\[
K=\begin{pmatrix}
A&B\\
B^T&C
\end{pmatrix},
\]
其中$A$是一个$(N^2-N)\times (N^2-N)$的矩阵,$C$是一个$(N-1)\times (N-1)$的矩阵,而$B$是一个$(N^2-N)\times (N-1)$的矩阵。

矩阵$A$的分量为
\[
	A_{ij;kl}=(E_{ij},E_{kl})_K=\delta_{jk}\delta_{il},
\]
矩阵$B$的分量为$B_{ij;k}=(E_{ij},H_k)_K$,由于
\[
	(E_{ij},e_k)_K=(E_{ij},E_{kk}-E_{(k+1),(k+1)})_K=\delta_{jk}\delta_{ik}-\delta_{j,(k+1)}\delta_{i,(k+1)}=0,
\]
所以$B_{ij;k}=(E_{ij},H_k)_K=0$,故$B=0$. 矩阵$C$的分量为
\[
	C_{ij}=(H_i,H_j)_K=\frac{\delta_{ij}}{2}.
\]
因此
\[
\det K =\det A \cdot \det C = \frac{\det A}{2^{n-1}}.
\]

现在问题就归结到了证明$\det A\neq 0$上面了。注意到分量$A_{ij;kl}=\delta_{jk}\delta_{il}$,对固定的$(i,j)$,即固定的一行,只有当$(k,l)=(j,i)$时候才不为零。因此,这个矩阵,每一行只有一个$1$,因为他是对称矩阵,所以每一列也只有一个$1$,这就是一个置换矩阵,他的行列式为$\pm 1\neq 0$. \endproof
\clearpage

\section{对称群以及Young图}

对称群的不可约表示和$\operatorname{GL}(n)$以及$\operatorname{SL}(n)$的不可约表示紧密相连,所以这里考虑对称群。对于对称群不可约表示的结构定理,下面会陈述,但是不做证明。这节主要是Young的工作,是他开启了表示论的大门。

\para 考虑集合$=\{1,\cdots,n\}$,以及任意的双射$\sigma : I_n\to I_n$,我们可以直接用表
\[
	\begin{pmatrix}
	1&2&\cdots &n\\
	\sigma(1)&\sigma(2)&\cdots &\sigma(n)
	\end{pmatrix}
\]
来描述这样一个双射。所有这样的双射按照复合构成一个群,我们称为对称群,记作$S_n$。这是一个有限群,他的元素个数是$n!$个。

\para 对于对称群$S_n$,对于任意的置换$\sigma\in S_n$,记$\rr^n$的标准基为$e_i$,则置换$\sigma$可以通过$\sigma:e_i\mapsto e_{\sigma(i)}$看成一个可逆线性映射$D(\sigma): \rr^n\to \rr^n$,他的矩阵由$D(\sigma)_{ij}=\delta_{i,\sigma(j)}$给出。经过数次调换行和列,总可以将这个矩阵变换到单位矩阵,于是断言$D(\sigma)$是可逆的且$\det(D(\sigma))=\pm 1$. 记$\sgn(\sigma)=\det(D(\sigma))$,当$\sgn(\sigma)=1$的时候,我们称这个置换是偶置换,反之,则叫他奇置换。实际上,任意的置换都可以分解成数个两两对换,当置换是偶置换的时候,分解出两两置换的数目是偶数,反之是奇数。
\endpara

上面这个例子构成了两个表示,其中$D:S_n\to \operatorname{GL}(\rr^n)$被称为$S_n$的标准表示,他不一定是不可约的。而$\sgn=\det\circ D:S_n\to \{1,-1\}$又是另一个表示,因为$\pm 1\in \operatorname{GL}(\rr)$,这是一个很简单的一维表示。

\para 现在,固定$\sigma$,考虑任意的$i\in I$,以及集合$\{i,\sigma(i),\sigma^2(i),\cdots,\sigma^k(i),\cdots\}$,由于这是一个有限群,所以集合必然也是有限的,即总存在一个$1\leq p\leq n$,使得$\sigma^p(i)=i$. 我们就得到了一个$p$个元素的$I$的子集,这被称为一个cycle,长度为$p$,有时候被称作一个$p$-cycle. 由于$I_n$是有限集,所以我们可以将其分解为不同cycle的并。

将$\{i,\sigma(i),\sigma^2(i),\cdots,\sigma^{p-1}(i)\}$直接计算出来,记作$(i,j,k,\cdots,l)$. 由于$\sigma^{k+1}(i)=\sigma(\sigma^k(i))$,所以这组数里面,第$j$个数是第$j-1$个数进行应用$\sigma$后得到的,而第一个数是最后一个数应用$\sigma$后得到的。因此,cycle的名字恰如其分。

由于$I_n$可以分解成这样的集合的并,即
\[
	(i_{11},\cdots,i_{1k_1})(i_{21},\cdots,i_{2k_2})\cdots (i_{m1},\cdots,i_{mk_m}),
\]
其中每一个$I_n$中的正整数都出现一次。所以如果我们得到了一个分解,则我们也完全确定了$\sigma$. 由于每一个$(i_{j1},\cdots,i_{jk_j})$都可以反过来确定一个$\sigma_j$,所以上面的分解又可以看成许多置换的复合,即$\sigma=\sigma_1 \sigma_2\cdots \sigma_m$,其中$\sigma_i\sigma_j=\sigma_j\sigma_i$. 

比如
\[
	\sigma=\begin{pmatrix}
	1&2&3&4&5\\
	2&3&1&4&5
	\end{pmatrix}
\]
我们就可以把$\sigma$分解为
\[
	\sigma=(1,2,3)(4)(5).
\]
其中$(1,2,3)$被称为3-cycle,类似地,$(4)$和$(5)$是1-cycle. 我们称$k$-cycle比$k'$-cycle高,如果$k>k'$. 在已知$S_n$的$n$的情况下,我们会在循环的cycle表示里略去所有的1-cycle. 上述的$\sigma=(1,2,3)(4)(5)$就会被记作$\sigma=(1,2,3)$.

{\pro 形如$(i,j)$的置换被称为两两置换。对于两两置换$\sigma$,显然有$\sigma^{-1}=\sigma$,或者$\sigma^2=e$. 我们断言,对称群中的任意置换都可以写成两两置换的复合。\endpro}

\proof 对于单位元$e$,这点是自明的,因为$(i,j)^2=e$. 其他的置换都至少存在一个2-cycle. 由于
\[
	\sigma=(i_{11},\cdots,i_{1k_1})(i_{21},\cdots,i_{2k_2})\cdots (i_{m1},\cdots,i_{mk_m})=\sigma_1 \cdots \sigma_m,
\]
所以,如果一个置换没有比$2$-cycle更高的cycle,则这个置换可以写作两两置换的复合。

考虑置换$\sigma$,他的1-cycle的个数是$k$,即他保持$(1,\cdots,n)$中$k$个数字位置不变,且至少有一个$m$-cycle,其中$m\geq 3$,所以存在一个$i$使得$\sigma^2(i)\neq i$,这样我们就有如下表示
\[
	\sigma=\begin{pmatrix}
	1&2&\cdots&i&\cdots &j&\cdots&n\\
	\sigma(1)&\sigma(2)&\cdots&j&\cdots &\sigma(j)&\cdots&\sigma(n)\\
	\end{pmatrix},
\]
然后考虑一个两两置换$\pi=(i,j)$,则
\[
	\pi\sigma=\begin{pmatrix}
	1&2&\cdots&i&\cdots &j&\cdots&n\\
	\sigma(1)&\sigma(2)&\cdots&\sigma(j)&\cdots &j&\cdots&\sigma(n)\\
	\end{pmatrix}.
\]
$\pi\sigma$的1-cycle的个数是$k+1$. 如果此时$\pi\sigma$还有比2-cycle高的cycle,则我们可以这样继续归纳下去,该过程总是可以终止的,因为每这样一次,我们的1-cycle的个数就涨一,当1-cycle的个数涨为$n-2$的时候,这个置换没有比2-cycle高的cycle。

假设进行了$l$步我们得到了$\pi_l\cdots\pi_1 \sigma$,他没有比2-cycle高的cycle了,此时
\[
\pi_l\cdots\pi_1 \sigma=(i_{11},i_{12}) \cdots  (i_{r1},i_{r1})=\sigma_1 \cdots  \sigma_r,
\]
其中$\sigma_i$都是两两置换,因此$\sigma=\pi_1\cdots\pi_l\sigma_1\cdots \sigma_r$,这个置换就写作了两两置换的复合。

最后,我们得到了结论,任意的置换都可以写成两两置换的复合。即在对称群中,两两置换生成了整个群。\endproof

\para 设$G$是一个群,而$g\in G$,则同态$\mathrm{Ad}(g):h\mapsto ghg^{-1}$被称为共轭作用。共轭作用衡量了一个群的可交换性,这是因为,如果$\mathrm{Ad}(g)(h)=h$,则$gh=hg$.

固定一个$h\in G$,考虑子集$H_h=\mathrm{Ad}(G)(h)$,这样的子集被称为共轭作用的一条轨道(或者在这里叫做共轭类)。两条不同的轨道是不能相交的,这是因为如果$H_h$和$H_g$有一个相交点$k$,分别记作$k=php^{-1}=qgq^{-1}$,则$g=q^{-1}ph(q^{-1}p)^{-1}$,所以$H_h=H_g$.

{\pro 对于对称群,在伴随作用下,$\sigma (i_1,\cdots,i_k)\sigma^{-1}= (\sigma(i_1),\cdots,\sigma(i_k))$. \endpro}

这就是说,他将一个$k$-cycle变成$k$-cycle.

\proof 如果$i$在$(i_1,\cdots,i_k)$作用下不变,则
\[\sigma (i_1,\cdots,i_k)\sigma^{-1}:i\mapsto \sigma^{-1}(i)\mapsto \sigma^{-1}(i) \mapsto i,\]
故他在$\sigma (i_1,\cdots,i_k)\sigma^{-1}$作用下也不变。否则
\[
	\sigma (i_1,\cdots,i_k)\sigma^{-1}:\sigma(i_j)\mapsto i_j\mapsto i_{j+1} \mapsto \sigma(i_{j+1}).
\]
这就得到了我们的结论。\endproof

\para 现在来看Young图,对于一个正整数$n$,我们可以将其分解成一个加式$n=\sum_{i=1}^k n_i$,其中$0\leq n_k\leq n_{k-1}\leq \cdots \leq n_1$,这样一个分解可以简单记作$\{n_1,n_2,\cdots,n_k\}$,或用一个Young图表示,第$i$行$n_i$个小方块。比如$8$被分解成$\{5,3\}$或者$\{4,2,2\}$,他们的Young图分别是
\[\Yvcentermath1
	\yng(5,3)\quad \yng(4,2,2)
\]

将循环群和Young图联系起来,即是通过循环的cycle表示,比如$\sigma=(1,2,3)(4)(5)$,他对应于分解为$3+1+1$,即$\{3,1,1\}$,对应的Young图是
\[
	\yng(3,1,1)
\]
我们可以通过另一个方法来得到$\{3,1,1\}$. 先分别列出$i$-cycle的个数$k_i$,然后计算出$l_i=\sum_{j=i}^nk_j$. 比如$\sigma=(1,2,3)(4)(5)$对应的$i$-cycle的个数序列为$(2,0,1,0,0)$,然后$\{3,1,1\}$中的第$i$个数字由$(2,0,1,0,0)$中的累加第$i$个数字后面所有的数字,我们可以得到$\{3,1,1,0,0\}$然后略去末尾的零就可以了。这点在以前的Lie代数的最高权的Young图表示中提过了。

Young图并不唯一确定置换。这是因为在伴随作用下,$\sigma (i_1,\cdots,i_k)\sigma^{-1}= (\sigma(i_1),\cdots,\sigma(i_k))$,故置换$\pi$和$\sigma\pi\sigma^{-1}$有着相同的Young图。更确切地说,Young图的数目正好等于$S_n$中共轭类的数目。

在有限群那里已经证明了,有限群的有限维不可约复表示的个数等于其共轭类的个数,所以$S_n$的不可约表示的个数正好就是他能有的Young图的个数。

\para 为了通过Young图确定置换,往Young图里面填数字就可以了,比如
\[\Yvcentermath1
	\young(123,4,5)\quad \text{or}\quad \young(123,4,5)\quad \text{or}\quad \young(231,4,5)
\]
就确定了置换$(1,2,3)(4)(5)$. 在Young图中填入$1$到$n$这些正整数后得到的表被称为Young表。

每一个Young图,都共有$n!$种填表的方式,其中如下的填表方式得到的Young表
\[\Yvcentermath1
	\young(12345,678),\quad \young(1234,56,78),
\]
被称为标准Young表。

\para 对于有限群$G=\{g_1,\cdots,g_n\}$,前面我们定义一个矢量空间
\[
	\cc [G]=\cc\langle g_1\rangle \oplus \cdots \oplus \cc\langle g_n\rangle,
\]
在$\cc [G]$上可以定义乘法如下
\[
	\sum_{g\in G} a_g g \sum_{h\in G} b_h h=\sum_{g,h\in G}a_gb_h gh,
\]
这就使得$\cc [G]$成为了一个含幺环。他的单位元记作$e$.

{\thm $S_n$的不可约表示由总格数为$n$的Young图一一确定。\endthm}

这个定理比以前数不可约表示个数的命题要强很多,这意味着$S_n$的不可约表示由其共轭类一一确定。

\para 上面定理的证明是直接的构造,来自于Young. 给定一个Young图,给出他的标准Young表,比如
\[
	\young(12345,678)
\]
记$P_i$为所有保持第$i$行数字不变的置换构成的$S_n$的子群(比如上面的Young表中第一行可以变成$(12453)$,但不能变成$(12645)$),以及$Q_j$为所有保持第$j$列数字不变的置换构成的子群,再令
\[
	P=\bigcap_i P_i,\quad Q=\bigcap_j Q_j,
\]
这是两个$S_n$的子群。我们有如下定理:

{\thm 给定$S_n$的一个Young图$\lambda$,我们对应有子群$P_\lambda$和$Q_\lambda$,令$a_\lambda=\sum_{\sigma\in P_\lambda}\sigma$以及$b_\lambda=\sum_{\sigma\in Q_\lambda}\sgn(\sigma)\sigma$,这是两个$\cc[S_n]$中的元素。定义$c_\lambda=a_\lambda b_\lambda$,称为Young图$\lambda$的Young对称化子。则$V_\lambda=\cc[S_n]c_\lambda$以及$\rho_\lambda(g)=l_g$构成$S_n$的一个不可约表示,且不同Young图对应的不可约表示不等价。因此,$S_n$的不可约表示一一对应着他的Young图(或者说共轭类)。\endthm}

最后一点是自明,其他的我们不加证明,并且也不证明如下命题:

{\pro 对于$S_n$的一个Young图$\lambda=\{\lambda_1,\lambda_2,\cdots,\lambda_k\}$,记$l_i=\lambda_i+k-i$,则
\[
	\dim V_\lambda=\frac{n!}{l_1!\cdots l_k!}\prod_{1\leq i< j\leq k}(l_i-l_j)=\frac{n!}{l_1!\cdots l_k!}\begin{vmatrix}
	1& l_k & l_k^2&\cdots\\
	\vdots&\vdots&\vdots&\vdots\\
	1& l_1 & l_1^2&\cdots\\
	\end{vmatrix},
\]
其中的行列式为van der Monde行列式。\endpro}

{\pro 设$\lambda$是$S_n$的一个Young图,则$c_\lambda^2=n_\lambda c_\lambda$,其中$n_\lambda=n!/\dim V_\lambda$. $n_\lambda$被称为hook number,他可以通过Young图直接确定:\endpro}

给定Young图$\lambda$,记$n_{ij}$为他直着往下的格子数+直着往右的格子数+1,这个数称为hook length. 比如下面的Young图
\[
	\young(\hfil \hfil \hfil \hfil ,\hfil ***,\hfil *\hfil ,\hfil *)
\]
中$n_{22}$为有*的格子的数目,所以$n_{22}=5$. 我们的结论是$n_\lambda=\prod_{i,j}n_{ij}$.

比如考虑如下Young图\[\Yvcentermath1
	\yng(3,2,1)
\]
在每一个格子里面填入相应的hook length,得到
\[
	\young(531,31,1)
\]
所以$n_\lambda= 45$.

\para 由映射
\[
	(v_{1}\otimes\cdots\otimes v_{n})\sigma=v_{{\sigma(1)}}\otimes\cdots\otimes v_{{\sigma(n)}}
\]
线性扩张,我们可以定义出$\cc[S_n]$在$V^{\otimes n}$上面的右作用.

比如$c=e+4(1,3)$,那么
\[
	(v_{i_1i_2i_3})c=v_{i_1i_2i_3}+4v_{i_3i_2i_1},
\]
其中$v_{ijk}=v_{i}\otimes v_{j}\otimes v_{k}$.

\para 对于一个$S_n$的Young图$\lambda$,我们有Young对称化子$c_\lambda$,我们记$S_\lambda V= (V^{\otimes n})c_\lambda=\im(c_\lambda)$. 为了阐明$S_\lambda V$的意义,下面我们考虑$n=2$的情况,以及
\[\Yvcentermath1
	\lambda_1=\yng(1,1)
\]
他对应的标准Young表为
\[\Yvcentermath1
	\young(1,2)
\]
所以$P_{\lambda_1}=\{e\}$,以及$Q_{\lambda_1}=\{e,(1,2)\}=S_2$,故$c_{\lambda_1}=a_{\lambda_1} b_{\lambda_1} =e (e- (1,2))=e- (1,2)$.

考虑任意的矢量$v\otimes w\in V^2$,
\[
c_{\lambda_1}(v\otimes w)=v\otimes w-w\otimes v=v\wedge w,
\]
所以$S_{\lambda_1} V$就是那些反对称的张量构成的矢量空间。

我们接着考虑
\[\Yvcentermath1
	\lambda_2=\yng(2)
\]
他对应的标准Young表为
\[\Yvcentermath1
	\young(12)
\]
所以$P_{\lambda_2}=\{e,(1,2)\}=S_2$,以及$Q_{\lambda_2}=\{e\}$,故$c_{\lambda_2}=a_{\lambda_2} b_{\lambda_2} =(e+(1,2))e =e+(1,2)$.

考虑任意的矢量$v\otimes w\in V^2$,
\[
(v\otimes w)c_{\lambda_2}=v\otimes w+w\otimes v,
\]
所以$S_{\lambda_2} V$就是那些对称的张量构成的矢量空间。

更广义地,对于Young图
\[\Yvcentermath1
	\lambda=\yng(5,3,2)
\]
$S_{\lambda} V$就是那些按列反对称、按行全对称的张量张成的矢量空间。

\para 任取$\sigma\in S_n$,以及任意的$\varphi$诱导来的$\Phi$,我们有$\Phi\circ \sigma=\sigma\circ \Phi$,这次定义来看是显然的,因为他们都成立
\[
	e_{i_1}\otimes\cdots\otimes e_{i_n}\mapsto \varphi e_{\sigma(i_1)}\otimes\cdots\otimes \varphi e_{\sigma(i_n)}.
\]

\para 注意到$\Lambda^i\bigl(\cc^N\bigr)\subset (\cc^N)^{\otimes i}$以及$\sym^a(V)\subset V^{\otimes a}$,所以
\[
	V_{(\lambda_1,\cdots,\lambda_{N-1})}\subset \bigotimes_{i=1}^{N-1}\sym^{\lambda_i}\left((\cc^N)^{\otimes i}\right)\subset  \bigotimes_{i=1}^{N-1}(\cc^N)^{\otimes i\lambda_i}=(\cc^N)^{\otimes d},
\]
其中$d=\sum_{i=1}^{N-1}i\lambda_i$. 

让我们回忆一下Young图,对应于$\lambda=(\lambda_1,\cdots, \lambda_{N-1})$,我们采用另一个记法
\[
	\lambda=\{l_1,\cdots,l_{N-1}\},
\]
其中$l_i$是第$i$行的格子数,$l_i=\sum_{k=i}^{N-1}\lambda_k$,所以
\[
	d=\sum_{i=1}^{N-1}i\lambda_i=\sum_{i=1}^{N-1}l_i
\]
正是Young图的格子数。

{\thm $\mathfrak{sl}(n,\cc)$最高权为$\lambda=\{l_1,\cdots,l_{N-1}\}$的不可约表示正是由$S_N$的Young图$\lambda'=\{l_1,\cdots,l_{N-1},0\}$确定的$S_{\lambda'} (\cc^N)$.\endthm}

\clearpage

\section*{Reference}
\addcontentsline{toc}{section}{Reference}

[1] GTM 222, Hall,

[2] Conforaml Field Theory, Chapter 13, Philippe, etc. ,

[3] 李群讲义, 项武义等, 

[4] GTM 225, Bump,

[5] Symmetries, Lie Algebras and Representations, Jurgen Fuch, etc. ,

[6] Lie Algebras, Shlomo Sternberg,

[7] Quantum Theory of Field V1, Chapter 2, Weinberg,

[8] GTM 203, S. Axler, etc.

[9] GTM 129, William Fulton, etc.