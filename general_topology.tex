%!TEX program = xelatex
\documentclass[10pt]{book}
\usepackage{amssymb,amsfonts,amsthm,amsmath,bm}
\usepackage{ctex}
%\usepackage[adobefonts]{ctex}%ubuntu用
%\usepackage{extarrows}
\usepackage[left=21mm,text={148mm,200mm},paperwidth=185mm,paperheight=250mm,includehead,vmarginratio=1:1]{geometry}
\usepackage[all]{xy}
\usepackage{tikz}
\usepackage{titletoc}%使用目录
	\theoremstyle{plain}%定理环境样式
	\newtheorem{pro}{Proposition}[section]% 定义命题环境
	\newtheorem{theo}{Theorem}% 定义定理环境
	%\newtheorem{lem}{Lemma}[section]% 定义引理环境
	\newtheorem*{rem}{Remark}% 定义注记环境
	\newtheorem{que}{Question}[section]% 定义问题环境
	\newtheorem{defi}{Definition}% 定义定义环境
	\newtheorem{exa}{Example}[section]% 定义例子环境
	%\newtheorem{exe}{Exercise}[section]% 定义习题环境
\newcommand{\dd}{{\mathrm{d}}}%微分号
\newcommand{\no}[1]{{$(#1)$}}%编号
\begin{document}
\title{一些基础拓扑什么的}
\author{理不歪}
\date{\today}
\maketitle %标题
基础拓扑的性质决定了这个笔记老多老多的定义,估摸着可以用来当字典查2333.本笔记是GTM 202的笔记。
\begin{defi}
现设有集合$X$,其上的一个拓扑(topology)是他的一个子集族$\mathcal{J}$,子集族中的元素称为开集,我们说某个集$U$是开集就是说$U \in \mathcal{J}$。子集族,或者这里说开集族,满足下面的性质:

\no{1}全集和空集都是开集:$X\in \mathcal{J}$以及$\varnothing\in \mathcal{J}$.

\no{2}开集的有限交是开集:假如$U_1,\dots,U_n \in \mathcal{J}$,那么$U_1\cap\dots\cap U_n\in \mathcal{J}$.

\no{3}开集的任意并是开集:任意的(有限或无限的)开集族$\{U_\alpha\}$的并$\cup_\alpha U_\alpha \in \mathcal{J}$.\\
有一个拓扑的集合我们称为拓扑空间。
\end{defi}

空间中的元素通常被称呼为一个点,虽然这是$\mathbb{R}^3$来的几何直观,但是拓扑空间的元素可以是很随意的东西,比如一条直线,但是我们还是称呼其为一个点。一个点的一个邻域就是包含这个点的一个开集。

上面的定义在操作上并不是那么实用的,所以下面的拓扑基(basis)在很多时候更加常用一些。
\begin{defi}
假设$X$是任意集合,所谓的拓扑基$\mathcal{B}$是他上的子集族,满足下列条件:

\no{1}集合上的任意点都在拓扑基的某一个元素里,即任取$p \in X$,那么存在$B \in \mathcal{B}$使得$p \in B$.这就是说$\cup_{B \in \mathcal{B}} B = X$.

\no{2}如果$B_1,B_2\in \mathcal{B}$,而且$x \in B_1 \cap B_2$,那么存在$B_3 \in \mathcal{B}$,使得$x \in B_2 \subset B_1 \cap B_2$.
\end{defi}

\begin{theo}
如果集合$X$有一个拓扑基$\mathcal{B}$,那么令$\mathcal{J}$为所有$\mathcal{B}$中元素的任意并的集族,则$\mathcal{J}$是$X$的一个拓扑。此时$\mathcal{J}$被称$\mathcal{B}$生成的拓扑。
\end{theo}

一个集合$U$被称为满足基的判别法则就是说,如果对于任意一点$x\in U$都存在一个集$B\in \mathcal{B}$使得$x \in W$且$W\subset U$.

\begin{theo}
如果集合$X$有一个拓扑基$\mathcal{B}$,那么其生成的拓扑$\mathcal{J}$为$X$中所以满足基的判别法则的集合构成的集族。简单来说,就是满足基的判别法则的集合是开集。
\end{theo}

上面的定理给出了判别一个集合是开集的实用方法。下面的定理则类似基生成拓扑的逆定理。
\begin{theo}
如果$X$是一个拓扑空间,$\mathcal{B}$是其所有开集的集族。如果$X$的每一个开集都对$\mathcal{B}$满足基的判别法则,然后$\mathcal{B}$是$X$这个拓扑的拓扑基。
\end{theo}

拓扑结构的引入,允许我们谈论一些分析上的基本概念,比如收敛和连续。当然,这就要涉及映射,涉及映射就要涉及两个拓扑空间。

\begin{defi}
拓扑空间$X$上面的点序列$\{q_n\}$称有极限$q \in X$的,就是对于$q$的任意邻域$U$,都存在$N$,当$n>N$的时候,所有的$q_n$都属于$U$.极限一般记做$\lim_{n\to \infty} q_n=q$.
\end{defi}

\begin{defi}
我们说映射$f:X\to Y$是连续的,就是对于任意开集$U \in Y$,他的原象集$f^{-1}(U) \in X$也是开的。
\end{defi}
下面的定理给出了整体连续和局部连续之间的关系。

\begin{theo}
一个函数$f:X\to Y$是连续的,当且仅当他在每一点$x \in X$都存在一个邻域$U_x$,在这个邻域上的限制$f|_{U_x}$也是连续的。
\end{theo}

两个拓扑空间$X,Y$被称为同胚的,就是说存在连续双射$f:X\to Y$,其逆也是连续的。同胚是拓扑空间范畴的同构,也就是说两个拓扑空间存在同胚映射的话,两个空间有相同的拓扑性质。连续函数的复合也是连续的,所以同胚的复合还是同胚的,那么同胚的拓扑空间就构成一个等价类。直观上,两个空间是同胚的,就是说我们可以连续变形之使两者相同。

很简单可以证明,$\mathbb{R}^n$中的单位开球和$\mathbb{R}^n$同胚。所以说“有界”并不是拓扑性质。同样,单位开立方体也同胚于单位开球,所以有没有棱角也不是拓扑性质。但是怎么变,一个洞也不会连续变没了,所以有没有洞才是拓扑性质。当然还有其他拓扑性质。拓扑学就是研究这些性质的。   

\begin{defi}
一个集是闭集当他补集是开集。
\end{defi}
用集合论熟知的并的补和交的补的公式,我们很自然可以得到,任意闭集的交是闭集,有限闭集的并是开集。此外,连续函数也可以改写成闭集版。
\begin{theo}
映射$f:X\to Y$是连续的,当且仅当就是对于任意闭集$U \in Y$,他的原象集$f^{-1}(U) \in X$也是闭的。
\end{theo}
\begin{defi}
一个集$A$的闭包$\overline{A}$为包含他的最小闭集,一个集的开核(或内部)$\mathrm{Int} A$为他包含的最大开集。
\end{defi}
这样我们就可以定义一个集的拓扑边界了。
\begin{defi}
一个集$A \in X$的拓扑边界$\partial A$定义为:$\partial A=\overline{A}-\mathrm{Int} A$.
\end{defi}

\begin{defi}
一个集$A \in X$的外部$\mathrm{Ext} A$定义为:$\mathrm{Ext}A=X-\overline{A}$.
\end{defi}

很容易看出$\mathrm{Ext}A$是一个开集,因为他是$\overline{A}$的补集,而且$\mathrm{Int} A$也是开集。所以这就给出了判别一个点是不是在内部或者外部的方式,即是否存在一个邻域含于$\mathrm{Ext}A$或$\mathrm{Int} A$,如果都不是,那么就是在$\partial A$上了,因为$X=\mathrm{Int} A\cup \partial A\cup \mathrm{Ext}A$.

\begin{defi}
一个拓扑空间被称为Hausdorff的,就是说,对于任意两个不相同的点,存在其两个不相交的邻域。这样的空间被称为Hausdorff空间。
\end{defi}
这个定义看上去有些突兀,不过确实存在非Hausdorff空间的拓扑空间。比如集$\{1,2,3\}$,上面的一个拓扑为$\{\varnothing,\{1\},\{2,3\},\{1,2,3\}\}$,虽然2和3是不想同的点,但是他们却不存在不相交的邻域。

Hausdorff性一般被理解为分离性,但他也保证了拓扑空间有足够多的开集,在上面的例子里面,就是邻域不够用才出现不存在不相交的邻域。

Hausdorff空间内的单点集是闭集。非Hausdorff空间这个结论不一定成立,比如上面的$\{2\}$,他的补不是开集。

一般来说,一个非Hausdorff空间我们是不感兴趣的。 因为这个时候,一个收敛的点序列可能有着两个极限。比如在上面的例子里面,一个全是2的序列,他可能的极限为2还有3.

Hausdorff性说开集不能太少,那么可数性就限制开集不能太多。
\begin{defi}
一个拓扑空间,如果在每一点的邻域,拓扑基是可数的,那么这个拓扑空间被称为第一可数的。如果一个拓扑空间的拓扑基是可数的,那么这个拓扑空间被称为第二可数的。
\end{defi}
第二可数比第一可数限制更强一些。
\begin{defi}
一个第二可数的Hausdorff空间,如果对他的任意一点,都存在一个邻域,这个邻域同胚于$\mathbb{R}^n$,那么这个空间被称为一个$n$维拓扑流形,或者简称流形。
\end{defi}

现在转向从老空间构造新空间。先说从大到小的。

从子集$A\subset X$到原集$X$有一个自然的映射$\iota: A \hookrightarrow X$满足$\iota(x)=x$.子集作为一个集合,自然可以有着子集的拓扑。一方面,我们自然希望$\iota: A \hookrightarrow X$是一个连续函数,那么任意在$X$上连续的函数$f$,都有$f\circ \iota$在$A$上是连续的。另一方面,我们自然也希望映到$X$上的连续函数也要是一个映到$A$上的函数。前者要求$A$要有足够多的开集,后者要求这样的开集不能太多。下面要定义的拓扑,则是一个同时满足上面两点的拓扑,所以他享有子空间拓扑这样的名字。

\begin{defi}
子空间拓扑:对于拓扑空间$X$的子集$A$,$\mathcal{J}_A=\{U\subset A: U=A \cap V \text{对于某个开集}V\subset X\}$被称为子空间拓扑。赋予了子空间拓扑的$A$被称为子空间。
\end{defi}
上面定义的合理性要证明$\mathcal{J}_A$确实是一个拓扑。
\begin{theo}
假设$A\subset X$是一个子空间,对于任意的拓扑空间$Y$,一个映射$f:Y \rightarrow A$ 是连续的当且仅当$Y \xrightarrow{f} A \stackrel{\iota}{\hookrightarrow} X$是连续的。
\end{theo}
这个定理很本质地刻画了子空间拓扑,下面的唯一性定理直接表明了这点。
\begin{theo}
假设$A\subset X$是一个子集,$A$上的子空间拓扑是唯一的拓扑使得上面的定理成立。
\end{theo}

现在转向从小到大的拓扑,积拓扑。假设$X_1,\dots,X_n$为$n$个拓扑空间,则积拓扑是集合$X_1\times\cdots\times X_n$上的拓扑。

\begin{defi}
积拓扑的基:定义$X_1\times\cdots\times X_n$上的集族$\mathcal{B}=\{U_1\times \cdots \times U_n : U_1 \text{在}   X_i\text{是开集},\,i=1,\dots,n\}$称为积拓扑的基。由这组基生成的拓扑就是积拓扑。赋予了积拓扑的$X_1\times\cdots\times X_n$被称为积空间。
\end{defi}
积空间上也自然有一个映射$\pi_i:X \to X_i$,我们当然希望这个映射是连续的。和上面的子空间拓扑一样,也有一个定理本质地刻画了积拓扑。
\begin{theo}
假设$X_1\times\cdots\times X_n$是一个积空间。对于任意的拓扑空间$B$,一个映射$f:B\to X_1\times\cdots\times X_n$是连续的当且仅当任意的$f_i=\pi_i\circ f$是连续的:
\[
   \xymatrix{
    &  X_1\times\cdots\times X_n \ar[dd]^{\pi_i} \\
    &&\\
   B \ar[uur]^{f}  \ar[r]_{f_i}  &X_i.
    }
\]
\end{theo}
正因为和子空间的相似性,我们也有理由相信积拓扑也成立唯一性定理。

现在转向商拓扑,这个拓扑依然是从大到小的一个拓扑。商拓扑的名字来自类似于商群,经常用来粘合等价类,所以很容易可以看到应该是从大到小的。

\begin{defi}
令$X$是一个拓扑空间,$Y	$是一个集合。称满射$\pi$给予了$Y$商拓扑就是说对于任意的集合$U$是开的当且仅当原象集$\pi^{-1}(U)$是开的。该满射被称为商映射。
\end{defi}

因为确定等价类的映射都是满射,类似于$\pi:p\mapsto [p]$这种映射,所以等价类构成的空间$X/\sim$的拓扑由$\pi$引入。一个点的原象$\pi^{-1}(p)$有个特殊的名字,称为在$p$点的纤维。

\begin{theo}
假设$\pi:X\to Y$是商映射,对于任意拓扑空间$B$,映射$f:Y \to B$是连续的当且仅当$f \circ \pi$是连续的。
\[
   \xymatrix{
   X\ar[dd]_\pi \ar[ddrr]^{f\circ\pi}&&\\
   &&\\
   Y\ar[rr]_f&&B.
    }
\]
\end{theo}
\begin{theo}
对于任意满射$\pi:X\to Y$,满足上面定理的就是商映射。
\end{theo}
一个集合$E\subset X$的开覆盖就是说存在$X$中的开集族$\mathcal{B}=\{B_v\}$(这里指标不是说这是可数的),使得$E \subset \cup_v B_v$.所谓子覆盖就是说,一个$\mathcal{B}$的子族也是$E$的开覆盖。一个开覆盖如果是有限的,那么就称呼其为有限开覆盖。一个集合是紧的,就是当他的所有开覆盖都有有限子开覆盖。

紧集就是拓扑意义下的有限集,一个集合如果是紧的,那么这个集合有着很多类似于有限集的优良性质。
\begin{theo}
设$X,Y$都是拓扑空间,而$f:X\to Y$是连续函数。对于$X$中的紧集$E$,$f(E)$是$Y$中的紧集。
\end{theo}
紧集的闭子集也是紧的。Hausdorff空间的紧集是闭的(注意Hausdorff空间的有限集是闭的)。所以在紧Hausdorff空间下,紧集和闭集是等价的。当然,紧空间的有限积和商都是紧的。
\begin{defi}
令$X$是一个拓扑空间,当他的每一个无穷子集都有一个极限点的时候,他被称为极限点紧致的。当他的任意点列都存在收敛到他的元素的子列的时候,他被称为列紧的。
\end{defi}
可以很简单看出,紧性包含着极限点紧。
\begin{theo}
如果拓扑空间是第一可数的Hausdorff空间,那么极限点紧包含着列紧。
\end{theo}
\begin{theo}
如果拓扑空间是第二可数的列紧空间,则他是紧的。
\end{theo}
上面的几个定理使得下面一个推论成立:
\begin{theo}
如果拓扑空间是第二可数的Hausdorff空间,则极限点紧、列紧和紧性等价。
\end{theo}
所以说,在流形上,我们不必区分这三种紧性。我们知道,度量空间是第一可数而且是Hausdorff的,此外列紧的度量空间是第二可数的,那么从上面的推论可以得到:在度量空间上,三种紧性等价。
\begin{theo}
闭映射引理:设$F$是闭映射且连续。那么如果$F$是单的,那么这是一个拓扑嵌入。如果$F$是满的,那么这是一个商映射。如果是双射,那么$F$是一个同胚。
\end{theo}
\begin{theo}
如果$F$是从一个紧空间到Hausdorff空间的连续映射,那么$F$是闭映射。
\end{theo}
上面说了,紧是很强的条件,可是实际上,我们有时候并没有那么强的条件,可是一些紧的性质确实可以保留下来。所谓的局部紧空间就是如此。如果空间任意一点的都存在一个邻域,他被包含于这个空间中的一个紧集,那么这个空间就被称为局部紧的。

显然紧可以推出局部紧,这是因为每一个邻域的闭包都是紧的。如果当一个集合的闭包是紧的,那么这个集合就被称为预紧的。下面一个定理更为常用一些。

\begin{theo}
在Hausdorff空间上,局部紧和每一点都存在一个预紧邻域等价。
\end{theo}
下面这个引理挺常见的。
\begin{theo}
收缩引理:设$X$是一个局部紧的Hausdorff空间。如果$x\in X$的一个邻域为$U$,那么存在一个$x$的预紧邻域$V$使得$\overline{V}\subset U$.
\end{theo}
比紧空间性质还好一些的是,局部紧空间的任何开集或者闭集都是局部紧的。
\begin{theo}
如果$f:X\to Y$是两个局部紧空间间的连续函数,如果他的每一个紧集的原象也是紧的(称为proper的),那么他是闭映射。
\end{theo}
这使得我们可以使用闭映射引理。
\begin{defi}
抽象单纯复形(simplicial complex)是抽象单形(simplex)的集合。单形就是有限集。单纯复形要求每一个他的单形的非空子集也是它的单形。
\end{defi}
单形的元素被称为顶点(vertex),任何一个单形的非空子集被称为单形的面(face)。对于顶点的面,我们还称为顶点。

单形的维度定义为集合的$\mathrm{order}-1$,比如集合有$k+1$个元素,则被称为
$k$维的。
单纯复形的维度定义为其所有单形维度的上确界,当然可能是无限的。

单纯复形被称为有限的,如果他是有限集。单纯复形称为局部有限的,如果他的每一个顶点都只包含于有限个单形中。

如果单纯复形的子集也是单纯复形,则被称为子复形。我们称单纯复形的$k$-骨架(skeleton)为单纯复形的所有维度小于等于$k$的单形的集合,是一个$k$维子复形。0-骨架就是所有顶点的集合。

两个单纯复形之间的映射被称为单形映射(simplicial map)如果他将单形映射到单形且将顶点映射到顶点。$f(\{v_0,\dots,v_k\})=\{f_0(v_0),\dots,f_0(v_n)\}$,其中$f_0$是0-骨架到0-骨架的映射,称为顶点映射。

对于任意的欧几里得单纯复形,可以令一个抽象单纯复形为这个欧几里得单纯复形的0-骨架的任意有限子集的集族,这个抽象单纯复形被称为顶点系(vertex scheme)。

不是所有抽象单纯复形都是一个欧几里得单纯复形的顶点系,欧几里得单纯复形要求局部有限且有限维的。

对于任意有限集,我们可以定义自由模(free module)。为每一个抽象单形$\{v_0,\dots,v_k\}$定义一个$\mathbb{R}$上的自由模,当然这样就成了一个矢量空间$\mathbb{R}\langle v_0,\dots,v_k\rangle$。
每一个$\mathbb{R}\langle v_0,\dots,v_k\rangle$中的元素都可以写作
$t=t_iv^i$,其中$v^i:\{v_0,\dots,v_k\}\to\mathbb{R}$,满足$v^i(v_j)=\delta_j^i$.
其中我们定义抽象单形$\{v_0,\dots,v_k\}$的几何实现为$\langle v_0,\dots,v_k\rangle\subset \mathbb{R}\langle v_0,\dots,v_k\rangle$,几何实现为所有使得$0\leq t_i\leq 1$且$\sum_{i=0}^kt_i=1$的$t=t_iv^i$的集合。单形$\sigma$的几何实现一般记做$|\sigma|$。

然后定义复形的几何实现,复形$\mathcal{K}$的几何实现$|\mathcal{K}|$定义为他的所有单形的几何实现的并。现在我们对$|\mathcal{K}|$引入拓扑。这个是从所有单形的几何实现的不交并的商拓扑引入的。商映射如下:
\[\pi:\coprod_{\sigma\in\mathcal{K}}|\sigma|\to|\mathcal{K}|.\] 
这就是说,$|\mathcal{K}|$中的开集当且仅当他交每一个$|\sigma|$的时候也是开集。
任意的单形映射$f:\mathcal{K}\to \mathcal{L}$都引入了他们几何实现之间的映射$​|f|​:|\mathcal{K}|\to|\mathcal{L}|$,如果$\mathcal{K}$是一个欧几里得单纯复形$K$的顶点系,那么$|\mathcal{K}|$就同胚于$|K|$。

任何一个同胚于单纯复形几何实现的拓扑空间被称为多面体(polyhedron)。这样的同胚映射被称为三角剖分(triangulation),可以三角剖分的拓扑空间(即任意多面体)被称为可三角剖分的。

我们称呼图为带有三角剖分的1维多面体。

\begin{theo}
任意的一维流形都被一个一维单纯复形三角剖分。
\end{theo}
这个定理在二维三维也正确,但是四维就是不正确的了,更高维的性质尚未探索清楚。

下面是一维流形的分类。

\begin{theo}
任意连通的一维流形,如果他是紧的,则同胚于单位圆,如果不是,则同胚于实直线。
\end{theo}
被三角剖分之后,如果是周期的就是圆,不是就是实直线。

\end{document}
