\documentclass[12pt]{article}
\usepackage{../noteheader}
\usepackage{ctex}

\theoremstyle{definition}
	\newtheorem{para}{}[section]
		\renewcommand{\thepara}{\thesection.\arabic{para}}
	\newtheorem{defi}[para]{Definition}
\theoremstyle{plain}
	\newtheorem{lem}[para]{Lemma}
	\newtheorem{thm}[para]{Theorem}
	\newtheorem{coro}[para]{Corollary}
	\newtheorem{pro}[para]{Proposition}
\renewcommand*{\proofname}{Proof}

\begin{document}

闲谈2D-CFT的中心荷

一般来说,2D-CFT中,中心荷$c$可能会出现在这些地方:

\begin{itemize}
    \item 能动张量与自身的OPE:
    \[
        T(z)T(w)\sim \frac{c/2}{(z-w)^4}+\frac{2T(w)}{(z-w)^2}+\frac{\partial T(w)}{z-w}.
    \]
    \item Casimir能量:
    \[
        \langle T(w)\rangle = -\frac{ac}{L^2},
    \]
    其中$L$是系统的尺度,$a$是一个常数。
    \item 迹反常:若空间不平坦,则
    \[
        \langle T^\mu_\mu(x)\rangle=\frac{c}{24\pi}R(x),
    \]
    其中$R(x)$是$x$处的标量曲率。
\end{itemize}
能动张量与自身的OPE一般是最方便计算中心荷的方式,也常常作为定义,而后两者都引入了一个全局的“量”,前者是系统的尺度,而后者是全局的度量,它们告诉我们凡是有这种全局的量存在,中心荷$c$都有可能出现。然而,我们却不能把中心荷直接与这种全局的“量”联系起来。一方面,我们考虑的是CFT,而不管是Casimir能量还是迹反常都不是共形不变的,而中心荷$c$是共形不变的,它仅仅依赖于我们所考虑的CFT. 数学一些说,这是一个Casimir算符,它的本征值(这里也就是常数c)是用来标记这类CFT的。另一方面,从能动张量与自身的OPE可以看到,中心荷$c$也不需要知道全局的信息,因为能动张量与自身的OPE仅仅依赖于场局部的表现。所以我们直接将其理解为比如Casimir能量显然是荒谬的,它仅仅与$c$成正比而已,$c$应当是个更广泛的东西。

所以,$c$到底是什么?为回答这个问题,我们先回到能动张量的定义,以标量场为例,能动张量定义为
\[
    T(z)=-\frac{1}{\alpha'}:\partial X\partial X:(z),
\]
其中$:O:$是求$O$的normal ordering,可以形式地写作
\[
    :O:=\exp\left(\frac{\alpha'}{4}\int d^2z_1d^2z_2 \log |z_1-z_2|^2 \frac{\delta}{\delta X(z_1,\bar z_1)}\frac{\delta}{\delta X(z_2,\bar z_2)}
        \right)O,
\]
引入normal ordering是为了消去真空无穷大的真空期望,在这里我们有
\[
    :A(z)B(w):=A(z)B(w)-\langle A(z)B(w)\rangle.
\]
对于标量场,它满足
\[
    :X(z,\bar z)X(w,\bar w):=X(z,\bar z)X(w,\bar w)+\frac{\alpha'}{2}\log |z-w|^2,
\]
于是
\[
    \partial_z\bar\partial_z :X(z,\bar z)X(w,\bar w):=0.
\]
现在,让我们来看“裸”能动张量,为使得其不直接发散,我们考虑
\[
    T(z,w)=-\frac{1}{\alpha'}\partial X(z)\partial X(w),
\]
从上面,我们立刻知道$T(z,w)$在$z\to w$时候发散,且发散形如
\[
    \langle T(z,w)\rangle =\frac12\partial_z\partial_w \log |z-w|^2=\frac 1{2 (z - w)^2},
\]
回忆,$T(z)$告诉我们系统在点$z$处的“能量密度”,所以$T(z,w)$在$z\to w$之发散我们也不陌生,即场论里常见的能量零点问题。所以,我们可以从$T(z,w)$中减去这个发散的能量密度来得到我们“相对的”能动张量$G(z,w)$,在这里,正是$G(z,w)=:T(z,w):$. 但是,注意到,我们减去之能量密度在不同的坐标处是不同的,我们并不是均匀地减去了一个常量,这就体现在
\[
    G(z,w)G(z',w')=\frac{1}{4(z-w)^2(z'-w')^2}
\]

若



\end{document}