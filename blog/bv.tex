% !TEX program = xelatex
\documentclass[10pt]{article}
\usepackage{../noteheader}
\usepackage{ctex}

\theoremstyle{definition}
	\newtheorem{para}{}[section]
		\renewcommand{\thepara}{\thesection.\arabic{para}}
	\newtheorem{defi}[para]{Definition}
\theoremstyle{plain}
	\newtheorem{lem}[para]{Lemma}
	\newtheorem{thm}[para]{Theorem}
	\newtheorem{coro}[para]{Corollary}
	\newtheorem{pro}[para]{Proposition}
\renewcommand*{\proofname}{Proof}

\begin{document}

Batalin-Vilkovisky量子化科普

\section{分次结构}

分次结构的出现在物理中是自然的,其中最自然的可能是来自于粒子统计% https://www.doubilee.com/particle\_statistics/
,其告诉我们,粒子分为两类,其一是玻色子,其二是费米子,分别描述它们的经典理论是两类非常不同的代数,在前者中,两个量$a$和$b$的乘法满足$ab=ba$,在后者中,乘法则满足$ab=-ba$. 历史上,数学家和物理学家就定义了一系列具有$\mathbb{Z}_2$-分次的对象,他们通常以 super 为名,如 supermanifold 等。

同时,从自旋统计定理,自旋为半整数的粒子是费米子,自旋为整数的粒子是玻色子。所以,从这种角度来看,$\mathbb{Z}_2$-分次并不是最自然的,$\mathbb{Z}$-分次应当会更自然些,它可以告诉我们相应粒子(场)的自旋到底是什么。所以我们从$\mathbb{Z}$-分次的矢量空间开始。

\begin{defi}
    设$V$是一个$k$-矢量空间,可以写成直和$V=\bigoplus_{i\in\mathbb Z} V_i$,其中$V_i$中的元素被称为$i$-次元,这样一个矢量空间被称为分次矢量空间。如果$v\in V$处于某个$V_i$中,则称其为齐次的,此时若$v\neq 0$是$i$-次元,定义$\deg(v)=i$. 如果对两个分次$k$-矢量空间,$k$-同态$f:V\to W$满足$f(V_i)\subset W_i$,则称$f$是一个分次$k$-矢量空间同态。
\end{defi}

然后是分次代数。

\begin{defi}
    若分次$k$-矢量空间$V$上有一个乘法使得$V_iV_j\subset V_{i+j}$,则称其为一个分次$k$-代数。一般来说,我们要求这个乘法是有单位元的,且其为零次的,此时我们有自然的含入$k\subset V_0$. 对两个齐次元$v$, $w\in V$,若
\[
    vw=(-1)^{\deg(w)\deg(v)}wv\quad \Leftrightarrow \quad 
    [v,w]=vw-(-1)^{\deg(w)\deg(v)-1}wv=0,
\]
则称他们是可交换的(对易的)。若任意齐次元之间是可交换的,则称这是一个交换$k$-代数。
\end{defi}

这里$[*,*]$可以线性地延拓到整个$V$上,他是对玻色子括号$[*,*]$和费米子$\{*,*\}$括号的推广。所以,可以看到,经典的对易和反对易,在这里都是对易的。注意,因为全同粒子假设,玻色子和费米子之间是可以区分的,所以在物理构造中,玻色算符和费米算符的对易关系是可以任意给定的,而这里,偶数次元和任意元素都可以对易。

数学上,这样的交换分次代数也不少见,其中非常常见的是流形的de Rham complex,流形$M$上的余切丛的截面的反对称化$\Gamma(\Lambda^*M)$. 但是de Rham complex上还有一个非常重要的构造,外微分算子。这样对象的推广被称为\textit{微分分次代数} (differential graded algebra),简记为 dga. 

\begin{defi}
    若$V$是一个交换分次$k$-代数,如果有一个$k$-代数映射$d:V\to V$满足:$dV_i\subset V_{i+1}$, $d^2=0$和 Leibniz 法则,即对齐次元$a$满足
    \[
        d(ab)=(da)b+(-1)^{\deg(a)}a(db).
    \]
    则称$(V,d)$是一个微分分次代数。一个dga同态$f$不仅是一个交换$k$-分次代数同态,还需满足$df=fd$,即de Rham complex间拉回的推广。
\end{defi}

对每个分次交换代数$V$,$(V,0)$就是一个平凡的 dga. 
显然,对每个 dga,都有对应的上同调群$H^i(V,d)=\ker d/\operatorname{im} d$,且 dga 同态会诱导相应的上同调群同态。

\begin{para}[对称化]
    设$V$是一个分次$k$-矢量空间,我们可以定义一个其生成的最大的交换分次代数$\operatorname{Sym}(V)$. 以范畴论的语言,即相应的自由对象 (free object). 回忆,在交换$k$-代数范畴,这样的自由对象是多项式代数。

    $\operatorname{Sym}(V)$的构造是直接的,我们考虑多项式环$k[V]$,上面可以定义自然的分次:对齐次元$a$和$b$,定义形式乘法$ab$的分次为$\deg(a)+\deg(b)$. 所以,$ab$和$ba$具有相同分次。这时候考虑所有$[a,b]=vw-(-1)^{\deg(w)\deg(v)-1}wv$生成的理想$I$,则
    \[
        \operatorname{Sym}(V)=k[V]/I.
    \]
    显然,任何$V$生成的交换分次代数都可以经由$\operatorname{Sym}(V)$分解。

    我们来看一个非常简单的例子,$V=V_0\oplus V_1=k\oplus V_1$,这个分次矢量空间只有两个分次。此时$\operatorname{Sym}(V)$就是外代数$\Lambda^*(V_1)$,其中$\Lambda^0(V_1)=k$. 类似地,如果$V$只有$0$-分次,则$\operatorname{Sym}(V)$就是普通的$\operatorname{Sym}(V)$.
\end{para}

\begin{para}[分次代数的谱]
    设$V$是一个分次交换$k$-代数,我们可以定义出$V$-模,特别地,$V$的理想。设$W\subset V$是$V$的一个$k$-子模,且任取$a\in V$都有$aW\subset W$,则称$W$是一个左理想,同理可以定义右理想。此外,如果(左、右)理想可以由齐次元生成,则称其为(左、右)齐次理想。

    我们断言,此时左齐次理想等价于右齐次理想。实际上,如果$W$是右理想,则任取齐次元$a$,都有$Wa\subset W$,因为如果$ba\in W$,则$-ba\in W$,所以$aW\subset W$.
\end{para}

\begin{para}[分次对偶空间和谱]
    设$V$是一个分次矢量空间,$V^*$是其对偶空间,则我们可以将其分解为$V^*=\bigoplus_{i\in\mathbb Z}V_i^*$. 由于$V_i^*$中的元素将$i$-次齐次元变成$0$-次的,所以我们赋予分次$-i$,即$(V^*)_i=V_{-i}^*$.

    大家都知道,在代数几何中,一个有限维矢量空间$V$可以看成概形$\operatorname{Spec}(\operatorname{Sym}(V^*))$的$k$-值点的集合。这是因为,设$\{e_i\}$是$V$的一组基,$\{e^i\}$是$V^*$的对偶基,则$\operatorname{Spec}(\operatorname{Sym}(V^*))$的$k$-值点一一对应极大理想$\langle e^i-a_i\rangle_i$,其中$a_i\in k$,而这也自然对应着$a=\sum_i a_ie_i\in V$,因为通过赋值映射,自然就有$e^i(a)=a_i$.

    这里,对分次$k$-矢量空间$V$,类似地可以定义出$\operatorname{Spec}(\operatorname{Sym}(V^*))$.
\end{para}

\section{例子}

方便起见,我们考虑一个$k$-矢量空间$V$,函数空间$\mathscr O(V)=\operatorname{Sym}(V^*)$,以及Lie 群$G$在$V$上的作用。回忆,$V$可以自然等同于$\operatorname{Spec}(\operatorname{Sym}(V^*))$的$k$-值点的集合。同时,通过
\[
g\cdot\psi(v)=\psi(g^{-1}\cdot v),
\]
我们立刻可以将$G$作用定义到$\mathscr O(V)$上。

现在,我们考虑“路径积分”
\[
\int_{V} \exp(-S),
\]
其中我们要求$S$是$G$-不变函数。为使路径积分有意义,实际上我们考虑的积分应该具有形状
\[
\int_{V/G}\exp(-S),
\]
因为我们是对$V$中的$G$-轨道积分。

现在,让我们先忘掉积分,首先考虑$V/G$的结构,等价地,考虑其上所有可能的函数,再等价地,就是那些$\mathscr O(V)$上的$G$-不变函数,或者说$\mathfrak{g}$-不变函数。同时,如果我们不仅只考虑这些函数,还考虑更“高阶”的“不变量”,则我们还可以得到更多有趣的信息。

回忆,设$M$是一个$\mathfrak g$-模,可以定义其不变子模$M^{\mathfrak g}:=\{g \in \mathfrak g\,:\, gM=0\}$,此时$M\to M^{\mathfrak g}$是一个左正和函子,进而我们可以考虑其右导出函子,其刻画了“不变性的障碍”。为计算导出函子,我们需要找一个resolution,最常见的是Chevalley–Eilenberg complex $C^\bullet(\mathfrak g,M):=\operatorname{Hom}(\Lambda^\bullet \mathfrak g,M)=M\otimes \Lambda^\bullet \mathfrak g^*$,上面的微分算符$\delta:\operatorname{Hom}(\Lambda^k \mathfrak g,M)\to \operatorname{Hom}(\Lambda^{k+1} \mathfrak g,M)$定义为
\[
    \delta\omega(g_1,\dots,g_{k+1})=\sum_{j=1}^{k+1} (-1)^{j}g_j\omega(\dots,\hat{g}_j,\dots)+\sum_{i<j}^{k+1} (-1)^{i+j-1}\omega([g_i,g_j],\dots,\hat{g}_i,\dots,\hat{g}_j,\dots),
\]继而可以计算这个complex的cohomology group,其中$H^0=M^{\mathfrak g}$.

现在将$M$换成我们的函数空间,即考虑如下complex
\[
C^\bullet(\mathfrak g,\mathcal O(V))=\operatorname{Sym}(V^*)\otimes \Lambda^\bullet \mathfrak g^*
=\operatorname{Sym}(V^*\oplus (\mathfrak g[1])^*),
\]这里,为了将对称化和反对称化统一,即“超对称化”,我们移动了$\mathfrak g$的分次,使之变为负一次,即$(\mathfrak g[1])^{-1}=\mathfrak g$. 此时,Chevalley–Eilenberg complex上自然的微分算符$d$诱导了$\operatorname{Sym}(V^*\oplus (\mathfrak g[1])^*)$上的$1$-次微分算符$\delta$,而所有上同调群$H^k$中的元素刻画了“不变量”,它们应当满足$d\omega=0$. 方便起见,我们下面列出一些可能需要用到的计算结果。首先,取$f\in M$,则
\[
    \delta f=-(g_if)g^i.
\]
记$\{g_i\}$是$\mathfrak g$的基地,而$\{g^i\}$是$\{g_i\}$的对偶基,则
\[
    \delta g^i(g_j,g_k)=g_k\delta^i_j-g_j\delta^i_{k}+g^i([g_j,g_k])=g_k\delta^i_j-g_j\delta^i_{k}+f^i_{jk},
\]
其中$f$是结构常数,一般来说,$g_i$作用在常数上为零,所以
\[
    \delta g^i=\frac 12 f^i_{jk}g^j\wedge g^k.
\]

对于我们的空间而言,$\delta S=0$是“规范不变”性,同时,经典运动方程是由$dS=0$给出的。所以,同时将这两者结合起来,是一个自然的想法。我们考虑如下空间
\[
    \mathfrak g[1]\oplus V\oplus V^*[-1]\oplus \mathfrak g^*[-2]
\]

现在回到积分,回忆最原始的 Faddeev-Popov 量子化,我们可以通过一个规范固定函数$F:V\to \mathbb R$来指定$V$中每条$G$-轨道中的一个点,即$V/G$中的一个点。这是容易的,只要保证超曲面$F^{-1}(0)$与每条轨道相交且只相交一次即可。但是,由于相交横截的角度不同,所以需要配上相应的Jacobian. 具体一些,若我们可以解掉$v_n=f(v_1,\dots,v_{n-1})$,使得局部超曲面$F^{-1}(0)$可以用$v_1$, $\dots$, $v_{n-1}$表示,则我们应该考虑的积分其实是
\[
    \int_{V} \exp(-S)\delta(F)=\int_{F^{-1}(0)\cong V/G} \exp(-S)\left|\frac{1}{\partial_nF(v_1,\dots,v_{n-1};f(v_1,\dots,v_{n-1}))}\right|.
\]
但这有诸多不便,比方说这样的规范固定函数可能不存在。注意到,在规范固定中,超曲面才是主位重要的,而其方程是次要的,所以,我们可以适当放宽给出超曲面的方程。比如,找一个$1$-形式$\omega$,将超曲面定义为其零点集,同时,我们要求其是“可积的”,即$d\omega\wedge \omega=0$,这样可以保证有一个超曲面。易见,所有$\omega\in H^1$都可以保证这点,同时若$\omega=dF$,则回到我们之前的情况。

现在,给定$V\oplus \mathfrak g[1]$上的一个函数$f$满足$df=0$,我们来刻画所有使得$df=0$的点。

\end{document}