% !TEX program = aplatex
\documentclass[11pt]{article}
\usepackage{ctex}
\usepackage{../noteheader}

\theoremstyle{definition}
	\newtheorem{para}{}[section]
		\renewcommand{\thepara}{\thesection.\arabic{para}}
	\newtheorem{defi}[para]{Definition}
\theoremstyle{plain}
	\newtheorem{lem}[para]{Lemma}
	\newtheorem{thm}[para]{Theorem}
	\newtheorem{coro}[para]{Corollary}
	\newtheorem{pro}[para]{Proposition}
\renewcommand*{\proofname}{Proof}

\begin{document}

Batalin-Vilkovisky量子化

\section{分次结构}

分次结构的出现在物理中是自然的,其中最自然的可能是来自于粒子统计% https://www.doubilee.com/particle\_statistics/
,其告诉我们,粒子分为两类,其一是玻色子,其二是费米子,分别描述它们的经典理论是两类非常不同的代数,在前者中,两个量$a$和$b$的乘法满足$ab=ba$,在后者中,乘法则满足$ab=-ba$. 摆在一起,这就构成了一个$\mathbb Z_2$-代数,其中玻色子对应0分量,而费米子对应1分量。

根据自旋统计定理,自旋为半整数的粒子是费米子,自旋为整数的粒子是玻色子。$\mathbb{Z}_2$-分次还是丢掉了具体自旋数的信息,相比起来$\mathbb{Z}$-分次应当会更自然些。同时,从$\mathbb Z$-分次很容易回到$\mathbb Z_2$-分次,所以我们从$\mathbb{Z}$-分次的矢量空间开始。

\begin{defi}
    设$V$是一个$k$-矢量空间,可以写成直和$V=\bigoplus_{i\in\mathbb Z} V_i$,其中$V_i$中的元素被称为$i$-次元,这样一个矢量空间被称为分次矢量空间。如果$v\in V$处于某个$V_i$中,则称其为齐次的,此时若$v\neq 0$是$i$-次元,定义$\deg(v)=i$. 
\end{defi}

有时候平移分次是有用的,给定$V$,我们定义$V[k]=\bigoplus_{i\in\mathbb Z} (V[k])_i$,其中的分次为$(V[k])_i=V_{k+i}$. 

\begin{defi}
    设有两个分次$k$-矢量空间$V$和$W$,$k$-同态$f:V\to W$满足$f(V_i)\subset W_{i+a}$,则称$f$是一个$a$-次同态。特别地,$0$-次同态简称为同态。
\end{defi}

然后是分次代数。

\begin{defi}
    若分次$k$-矢量空间$V$上有一个乘法使得$V_iV_j\subset V_{i+j}$,则称其为一个分次$k$-代数。一般来说,我们要求这个乘法是有单位元的,且其为零次的,此时我们有自然的含入$k\subset V_0$. 对两个齐次元$v$, $w\in V$,若
\[
    vw=(-1)^{\deg(w)\deg(v)}wv\quad \Leftrightarrow \quad 
    [v,w]=vw-(-1)^{\deg(w)\deg(v)-1}wv=0,
\]
则称他们是可交换的(对易的)。若任意齐次元之间是可交换的,则称这是一个交换$k$-代数。
\end{defi}

这里$[*,*]$可以线性地延拓到整个$V$上,他是对玻色子括号$[*,*]$和费米子$\{*,*\}$括号的推广。所以,经典的对易和反对易,在这里都是对易的。注意,因为全同粒子假设,玻色子和费米子之间是可以区分的,所以在物理构造中,玻色算符和费米算符的对易关系是可以任意给定的,而这里,我们可以要求,偶数次元和任意元素都可以对易。

数学上,这样的交换分次代数也不少见,其中非常常见的是流形的de Rham 复形,流形$M$上的余切丛的截面的反对称化$\Gamma(\Lambda^*M)$. 但是de Rham 复形上还有一个非常重要的构造,外微分算子。这样对象的推广被称为\textit{微分分次代数} (differential graded algebra),简记为 dga. 

\begin{defi}[微分分次代数]
    若$V$是一个交换分次$k$-代数,如果有一个$1$-次$k$-代数同态$d:V\to V$满足:$dV_i\subset V_{i+1}$, $d^2=0$和 Leibniz 法则,即对齐次元$a$满足
    \[
        d(ab)=(da)b+(-1)^{\deg(a)}a(db).
    \]
    则称$(V,d)$是一个微分分次代数。一个dga同态$f$不仅是一个交换$k$-分次代数同态,还需满足$df=fd$,即de Rham 复形间拉回的推广。
\end{defi}

对每个分次交换代数$V$,$(V,0)$就是一个平凡的 dga. 
显然,对每个 dga,都有对应的上同调群$H^i(V,d)=\ker d/\operatorname{im} d$,且 dga 同态会诱导相应的上同调群同态。

\begin{para}[对称化]
    设$V$是一个分次$k$-矢量空间,我们可以定义一个其生成的最大的交换分次代数$\operatorname{Sym}(V)$. 以范畴论的语言,即相应的自由对象 (free object). 回忆,在交换$k$-代数范畴,这样的自由对象是多项式代数。

    $\operatorname{Sym}(V)$的构造是直接的,我们考虑多项式环$k[V]$,上面可以定义自然的分次:对齐次元$a$和$b$,定义形式乘法$ab$的分次为$\deg(a)+\deg(b)$. 所以,$ab$和$ba$具有相同分次。这时候考虑所有$[a,b]=vw-(-1)^{\deg(w)\deg(v)-1}wv$生成的理想$I$,则
    \[
        \operatorname{Sym}(V)=k[V]/I.
    \]
    显然,任何$V$生成的交换分次代数都可以经由$\operatorname{Sym}(V)$分解。
\end{para}

我们来看一个非常简单的例子,$V=V_0\oplus V_1=k\oplus V_1$,这个分次矢量空间只有两个分次。此时$\operatorname{Sym}(V)$就是外代数$\Lambda^*(V_1)$,其中$\Lambda^0(V_1)=k$. 类似地,如果$V$只有$0$-分次,则$\operatorname{Sym}(V)$就是普通的$\operatorname{Sym}(V)$.

\begin{para}[分次代数的谱]
    设$V$是一个分次交换$k$-代数,我们可以定义出$V$-模,特别地,$V$的理想。设$W\subset V$是$V$的一个$k$-子模,且任取$a\in V$都有$aW\subset W$,则称$W$是一个左理想,同理可以定义右理想。此外,如果(左、右)理想可以由齐次元生成,则称其为(左、右)齐次理想。

    我们断言,此时左齐次理想等价于右齐次理想。实际上,如果$W$是右理想,则任取齐次元$a$,都有$Wa\subset W$,因为如果$ba\in W$,则$ba=\pm ab\in W$,所以$aW\subset W$.
\end{para}

\begin{para}[分次对偶空间和谱]
    设$V$是一个分次矢量空间,$V^*$是其对偶空间,则我们可以将其分解为$V^*=\bigoplus_{i\in\mathbb Z}V_i^*$. 由于$V_i^*$中的元素将$i$-次齐次元变成$0$-次的,所以我们赋予分次$-i$,即$(V^*)_i=V_{-i}^*$.

    大家都知道,一个有限维矢量空间$V$可以看成概形$\operatorname{Spec}(\operatorname{Sym}(V^*))$的$k$-值点的集合。这是因为,设$\{e_i\}$是$V$的一组基,$\{e^i\}$是$V^*$的对偶基,则$\operatorname{Spec}(\operatorname{Sym}(V^*))$的$k$-值点一一对应极大理想$\langle e^i-a_i\rangle_i$,其中$a_i\in k$,而这也自然对应着$a=\sum_i a_ie_i\in V$,因为通过赋值映射,自然就有$e^i(a)=a_i$.
    这里,对分次$k$-矢量空间$V$,类似地可以定义出$\operatorname{Spec}(\operatorname{Sym}(V^*))$.
\end{para}

\begin{para}[超流形]
一个$(n|m)$-超流形$\mathcal M$,是一个$n$-维的流形$M$上的层$\mathcal O_{\mathcal M}$,局部同构于一超代数$C^\infty(U)\otimes \Lambda^* V^*$,其中$V$是一个给定的$m$-维矢量空间。
\end{para}

\begin{para}[分裂超流形]
设$E$是一个$n$-维流形$M$上秩为$m$的矢量丛,则我们有一个自然的$(n|m)$-超流形$\Pi E$,其结构层即$\Lambda^* E^*$的结构层。
\end{para}

\begin{para}[$\mathbb Z$-分次超流形]
给定一个超流形$\mathcal M$,如果局部地,开集$U$通过坐标卡同胚于一个分次矢量空间$W$,其只有偶数分次,且
$V$是只有奇数分次的分次矢量空间,则局部坐标可以视为分次矢量空间$W^*$中的元素,他们诱导了$C^\infty(U)\otimes \Lambda^* V^*$的子代数$\operatorname{Sym}(W^*)\otimes \Lambda^* V^*$上的一个$\mathbb Z$-分次。如果这个分次相容于坐标变换,则我们称$\mathcal M$是一个$\mathbb Z$-分次超流形。
\end{para}

局部地,我们可以选取坐标$\{x_i\}$使得其中每个坐标都是齐次元,此时如果坐标$x^i$是$W_{2k}$线性函数,则分次为$-2k$,若为$V_{2k+1}$上的线性函数,则分次为$-(2k+1)$.

\begin{para}[超流形上的分次矢量场]
    $\mathcal M$上的$k$-次矢量场为$k$-次导子$v:\mathcal O_{\mathcal M}\to \mathcal O_{\mathcal M}$,即满足$\deg (v(f))=k+\deg(f)$以及
    \[
        v(fg)=v(f)g+(-1)^{k\deg(f)}fv(g),
    \]
    其中$f$是齐次元。记号上,我们记$k$为$\deg(v)$.
\end{para}

一个典型的分次矢量场为Euler矢量场$E$,它是零次矢量场,满足$Ef=\deg(f)f$,其中$f$为任意齐次元,局部地,如果坐标为$\{x^i\}$,则他可以写作
\[
    E=\sum_i \deg(x^i)x^i\partial_{x^i}.
\]
对任意的分次矢量场$v$,我们可以知道$[E,v]=\deg(v)v$. 实际上,任取齐次元$f$,我们有
\[
    [E,v]f=Evf-(-1)^{\deg(E) \deg(v)}vEf=\deg(vf)vf-\deg(f)vf=\deg(v)vf.
\]

\begin{para}[${T[1]M}$ 与 ${T^*[-1]M}$]
设$M$是一个流形,局部地,局部坐标$x^i$的分次定义为0,而切矢量的局部坐标$\partial_{x^i}$的分次定义为$-1$,或者说对偶坐标$dx^i$的分次定义为$1$. 则切丛$TM$构成了一个超流形,记作$T[1]M$. 这里分次平移$[1]$的意思是,即切矢量的局部坐标$\partial_{x^i}$在$TM$中对应$0$次。类似地,可以定义${T^*[-1]M}$,其中对偶坐标$\partial_{x^i}$的分次定义为$-1$.
\end{para}

\section{Gauss积分}

\begin{para}[Gauss积分]
    设$Q$是一个正定二次型,则
    \[
        \int_{\mathbb R^n}d^nx \exp(-Q(x,x))=\pi^{n/2}(\det Q)^{-1/2}.
    \]
\end{para}

\begin{para}[Fresnel积分]
    设$Q$是一个非退化二次型,则
    \[
        \int_{\mathbb R^n}d^nx \exp(iQ(x,x))=\pi^{n/2}\exp\biggl(\frac{\pi i}{4} \operatorname{sign}(Q)\biggr)|\det Q|^{-1/2},
    \]
    其中$\operatorname{sign}(Q)$是二次型的特征,即正本征值的数量减去负本征值的数量。
\end{para}

\begin{para}[费米Gauss积分]
    设$V$是一个分次$k$-交换代数,$\{\eta_i\}_{i=1,\dots n}$是某个奇次分次$V_i$的一组线性无关矢量,设$Q$是一个反对称$n\times n$矩阵,则
    \[
        \int D\eta_n\cdots D\eta_1 \exp\left(\frac{1}{2}Q(\eta,\eta)\right)=\operatorname{Pf}(Q),
    \]
    其中$Q(\eta,\eta)=\sum_{i,j}Q_{ij}\eta_i\eta_j$.
    这里$\int D\eta_n\cdots D\eta_1$即提取$\eta_1\cdots \eta_n$之前的系数。不难证明,设$\eta'_i=\sum_jA_{ij}\eta_j$,则
    \[
        \int D\eta'_n\cdots D\eta'_1=\det(A)^{-1}\int D\eta_n\cdots D\eta_1,
    \]
    这与玻色情况的测度变化刚好相反。
\end{para}

\begin{para}
    设$V$是一个分次$k$-交换代数,$\{\eta_i\}_{i=1,\dots n}$是某个奇次分次$V_i$的一组线性无关矢量,而$\{\bar \eta_i\}_{i=1,\dots n}$是另外某个奇次分次$V_j$的一组线性无关矢量,设$Q$是任意的$n\times n$矩阵,则
    \[
        \int (D\eta_nD\bar\eta_n)\cdots (D\eta_1D\bar\eta_1) \exp\left(Q(\bar\eta,\eta)\right)=\det(Q),
    \]
    其中$Q(\bar\eta,\eta)=\sum_{i,j}Q_{ij}\bar\eta_i\eta_j$.
\end{para}

实际上,只有$\exp$展开中的第$n$阶才贡献,
\[
    \frac{1}{n!}\sum Q_{i_1j_1}\cdots Q_{i_nj_n}\int (D\eta_nD\bar\eta_n)\cdots (D\eta_1D\bar\eta_1) (\bar\eta_{i_1}\eta_{j_1})\cdots (\bar\eta_{i_n}\eta_{j_n}),
\]
由于每一对$\bar\eta\eta$都交换都没有符号,所以可以把$i_1,\dots,i_n$固定成$1,\dots,n$,此时所有对这些$i$的求和把$1/n!$消掉,然后可以把所有的$\bar\eta$积掉得到
\[
    \sum_{j\in S_n} Q_{1j_1}\cdots Q_{nj_n}\int D\eta_n\cdots D\eta_1 \eta_{j_1}\cdots \eta_{j_n},
\]
后面的积分即一个反对称因子,继而得到了行列式。

\section{Faddeev-Popov构造}

方便起见,我们考虑一个有限维流形$X$,以及Lie 群$G$在$V$上的作用,$G$的李代数为$\mathfrak g$. 现在,我们考虑“路径积分”
\[
\int_{X} \mu \exp(iS),
\]
其中我们要求$S$是$G$-不变函数,$\mu$是一个$G$-不变测度。在物理上,$G$-等价的点被认为是相同的,所以实际考虑的路径积分应该是
\[
I=\int_{X/G}\tilde \mu\exp(iS),
\]
即我们是对$X$中的$G$-轨道积分,$\tilde \mu$是$X/G$的测度。

设$\dim G=m$,$\dim X=n$,$\mu$是$X$的$G$-不变测度,$\pi:X\to X/G$是商映射。再设$\{v_1,\dots,v_{m}\}$是$\mathfrak g$的一组基$\{t_1,\dots,t_{m}\}$对应的基本矢量场,即$(v_i)_p\in T_p X$是单参曲线$\exp(\lambda t_i)\cdot p$在$\lambda=0$处的切矢量。于是,$\pi_* v_i=0$,这来自于任何$G$-不变函数$f$都有$\frac{d}{dt}f(\exp(\lambda t_i)\cdot p)=0$. 进而,我们构造$X/G$的测度$\tilde \mu$如下:
\[
    \iota_{v_m}\cdots \iota_{v_1}\mu=\pi^*\tilde \mu.
\]
容易从左边是$G$-不变和再作用任意的$\iota_{v_k}$都为零可知这一定是一个商映射的拉回,所以用来定义$\tilde \mu$合理。

在实际操作中,我们通常是在原空间$X$上取一些“规范固定”函数来给出$X/G$,即选择一族函数$f=(f_1, \dots, f_{\dim G}):X\to \mathfrak g$,使得子流形$W=f^{-1}(0)\xrightarrow{i} X$与$G$-轨道们横截相交(方便起见,假设只交一次),此时,我们考虑的不是$X/G$上的积分,而是$W$上的积分。
为此,首先将$\tilde \mu$通过$\pi i$拉回到$W$上,得到$W$上的体积形式
\[
    i^*\pi^*\tilde \mu=i^*(\iota_{v_m}\cdots \iota_{v_1}\mu).
\]

% 为进一步计算这个体积形式,我们考虑其在任意$n-m$个$W$上的矢量场
% $\{w_i\}$上的值$i^*(\iota_{v_m}\cdots \iota_{v_1}\mu)(w_1,\dots,w_{n-m})$. 

% 我们想求

但是,在实际计算中,这个$i^*\pi^*\tilde \mu$并不方便,实际上,不如考虑$X$上的体积形式
\[
    \alpha=df_1\wedge \cdots \wedge df_m\wedge (\iota_{v_m}\cdots \iota_{v_1}\mu),
\]
容易看到,任取$n-m$个$W$上的矢量场$\{w_i\}$,利用所有$W$上的矢量场$w$都满足$df_i(w)=0$,则
\[
    \iota_{w_{n-m}}\cdots \iota_{w_1}\alpha=(-1)^{m}(\iota_{w_{n-m}}\cdots \iota_{w_1} \pi^*\tilde \mu)\, df_1\wedge \cdots \wedge df_m
\]
中$df_1\wedge \cdots \wedge df_m$的系数即$i^*\pi^*\tilde \mu$在任意$\{w_i\,:\,i=1,\dots,n-m\}$上的取值,因此$W$上的体积形式可以形式地写作
\[
    i^*\pi^*\tilde \mu=\int \delta(f)\alpha.
\]
现在我们来求$\alpha$,不妨设$\mu=df_1\wedge \cdots \wedge df_m\wedge\beta$,其中$\beta$是$\bigcap_i\ker \iota_{v_i}\subset T^*X$的基外积而成的形式,于是
\[
    \iota_{v_m}\cdots \iota_{v_1}\mu=df_1\wedge \cdots \wedge df_m(v_1,\dots,v_m)\,\beta=\det(df_i(v_j))\,\beta,
\]
两边外积上$df_1\wedge \cdots \wedge df_m$即得到$\alpha=\det(df_i(v_j))\mu$. 所以,最后我们的积分可以写作
\[
    I=\int_{X}\delta(f)\exp(iS)\det(df_i(v_j))\mu.
\]
其中$df_i(v_j)$可以更内蕴地理解为一个$\mathfrak g\to \mathfrak g$的线性映射$\mathsf{FP}_f(x)$.

Faddeev-Popov构造的最后一步是将$\delta(f)$和$\det(\mathsf{FP}_f(x))$重写为积分,引入如下玻色和费米积分
\begin{align*}
    \delta(f(x))&=\frac{1}{(2\pi)^m}\int_{\mathfrak{g}^*}d^m\lambda \,\exp(i\langle \lambda,f(x)\rangle),\\
    \det(\mathsf{FP}_f(x))&=(-i)^m\int_{\prod \mathfrak{g}\oplus \mathfrak{g}^*} \prod_{i=1}^m (Dc_iD\bar c_i) \exp\left(i\langle\bar c,\mathsf{FP}_f(x)c\rangle\right),
\end{align*}
我们就得到了
\[
    I=\frac{1}{(2\pi i)^m}\int_{X}\mu\int_{\prod \mathfrak{g}\oplus \mathfrak{g}^*} \prod_{i=1}^m (Dc_iD\bar c_i)\int_{\mathfrak{g}^*}d^m\lambda \,\exp(iS+i\langle \lambda,f(x)\rangle+i\langle\bar c,\mathsf{FP}_f(x)c\rangle).
\]
于是,我们可以引入新的作用量
\[
    S_{\text{FP}}=S+\langle \lambda,f(x)\rangle+\langle\bar c,\mathsf{FP}_f(x)c\rangle.
\]

对上述的作用量,我们首先来研究其极值点(即运动方程):
\[
    c=\bar c=0,\quad f(x)=0,\quad \partial_i S(x)+\langle \lambda,\partial_i f\rangle=0.
\]
由于$S$是$G$-不变的,所以$S$的极值轨道与规范固定$f=0$是横截相交的,这就说明,$S$在$p\in W$处给出的条件极值($dS(T_pW)=0$)其实也是$S$的真正极值($dS(T_pX)=0$),换而言之,最后一个条件极值方程中的$\lambda$也为零。于是,$S_{\text{FP}}$的极值点形如$(p,c=\bar c=\lambda=0)$,其中$p$是$S$的某条极值轨道与规范固定子流形之交。此外,也正是$\lambda$这些乘子的引入,导致了$S_{\text{FP}}$在极值点处的Hessian也不再退化,继而我们可以用其去写出传播子,这里就不展开了。

\section{正文}

这就是Faddeev-Popov 量子化最开始所作的事情。然而,从上述过程中可以看到,“规范固定”的选取使得我们很难重新还原$G$-作用的影响,因为我们只在每条轨道中选了一个点,所以我们希望可以找到一种更“不变”的方式来定义这个积分。

为解决这个问题,注意到,在上面构造中,子流形是比规范固定函数更加基本的东西。而在所有与$G$-轨道横截相交的子流形,最“正则”的子流形是与$G$-轨道“垂直”的那些。考虑一个从$x$出发的单参群作用给出的轨道$\gamma(t)=g(t)\cdot x$,若$f$是一个$G$-不变函数,则$\partial_t f(\gamma(t))=0$,在$x$点处也即$\langle df,\dot{\gamma}(0)\rangle(x)=0$. 于是$\dot \gamma(0)$给出$G$-轨道,而与其垂直的方向由$df=0$给出。

举个例子,在$\mathbb R^2$中,如果$G$的作用是$x$-方向的平移,$y$是$G$-不变函数,而$dy=0$将给出我们想要的子流形$x=c$.

现在,让我们先忘掉积分,首先考虑$V/G$的结构,等价地,考虑其上所有可能的函数,再等价地,就是那些$\mathscr O(V)$上的$G$-不变函数,或者说$\mathfrak{g}$-不变函数。同时,如果我们不仅只考虑这些函数,还考虑更“高阶”的“不变量”,则我们还可以得到更多有趣的信息。

回忆,设$M$是一个$\mathfrak g$-模,可以定义其不变子模$M^{\mathfrak g}:=\{g \in \mathfrak g\,:\, gM=0\}$,此时$M\to M^{\mathfrak g}$是一个左正和函子,进而我们可以考虑其右导出函子,其刻画了“不变性的障碍”。为计算导出函子,我们需要找一个resolution,最常见的是Chevalley-Eilenberg 复形 $C^\bullet(\mathfrak g,M):=\operatorname{Hom}(\Lambda^\bullet \mathfrak g,M)=M\otimes \Lambda^\bullet \mathfrak g^*$,上面的微分算符$\delta:\operatorname{Hom}(\Lambda^k \mathfrak g,M)\to \operatorname{Hom}(\Lambda^{k+1} \mathfrak g,M)$定义为
\[
    \delta\omega(g_1,\dots,g_{k+1})=\sum_{j=1}^{k+1} (-1)^{j}g_j\omega(\dots,\hat{g}_j,\dots)+\sum_{i<j}^{k+1} (-1)^{i+j-1}\omega([g_i,g_j],\dots,\hat{g}_i,\dots,\hat{g}_j,\dots),
\]继而可以计算这个复形的上同调群,其中$H^0=M^{\mathfrak g}$.

现在将$M$换成我们的函数空间,即考虑如下复形
\[
C^\bullet(\mathfrak g,\mathcal O(V))=\operatorname{Sym}(V^*)\otimes \Lambda^\bullet \mathfrak g^*
=\operatorname{Sym}(V^*\oplus (\mathfrak g[1])^*),
\]这里,为了将对称化和反对称化统一,即“超对称化”,我们移动了$\mathfrak g$的分次,使之变为负一次,即$(\mathfrak g[1])^{-1}=\mathfrak g$. 此时,Chevalley-Eilenberg 复形上自然的微分算符$d$诱导了$\operatorname{Sym}(V^*\oplus (\mathfrak g[1])^*)$上的$1$-次微分算符$\delta$,而所有上同调群$H^k$中的元素刻画了“不变量”,它们应当满足$d\omega=0$. 方便起见,我们下面列出一些可能需要用到的计算结果。首先,取$f\in M$,则
\[
    \delta f=-(g_if)g^i.
\]
记$\{g_i\}$是$\mathfrak g$的基地,而$\{g^i\}$是$\{g_i\}$的对偶基,则
\[
    \delta g^i(g_j,g_k)=g_k\delta^i_j-g_j\delta^i_{k}+g^i([g_j,g_k])=g_k\delta^i_j-g_j\delta^i_{k}+f^i_{jk},
\]
其中$f$是结构常数,一般来说,$g_i$作用在常数上为零,所以
\[
    \delta g^i=\frac 12 f^i_{jk}g^j\wedge g^k.
\]

对于我们的空间而言,$\delta S=0$是“规范不变”性,同时,经典运动方程是由$dS=0$给出的。所以,同时将这两者结合起来,是一个自然的想法。我们考虑如下空间
\[
    \mathfrak g[1]\oplus V\oplus V^*[-1]\oplus \mathfrak g^*[-2]
\]

现在回到积分,回忆最原始的 Faddeev-Popov 量子化,我们可以通过一个规范固定函数$F:V\to \mathbb R$来指定$V$中每条$G$-轨道中的一个点,即$V/G$中的一个点。这是容易的,只要保证超曲面$F^{-1}(0)$与每条轨道相交且只相交一次即可。但是,由于相交横截的角度不同,所以需要配上相应的Jacobian. 具体一些,若我们可以解掉$v_n=f(v_1,\dots,v_{n-1})$,使得局部超曲面$F^{-1}(0)$可以用$v_1$, $\dots$, $v_{n-1}$表示,则我们应该考虑的积分其实是
\[
    \int_{V} \exp(-S)\delta(F)=\int_{F^{-1}(0)\cong V/G} \exp(-S)\left|\frac{1}{\partial_nF(v_1,\dots,v_{n-1};f(v_1,\dots,v_{n-1}))}\right|.
\]
但这有诸多不便,比方说这样的规范固定函数可能不存在。注意到,在规范固定中,超曲面才是主位重要的,而其方程是次要的,所以,我们可以适当放宽给出超曲面的方程。比如,找一个$1$-形式$\omega$,将超曲面定义为其零点集,同时,我们要求其是“可积的”,即$d\omega\wedge \omega=0$,这样可以保证有一个超曲面。易见,所有$\omega\in H^1$都可以保证这点,同时若$\omega=dF$,则回到我们之前的情况。

现在,给定$V\oplus \mathfrak g[1]$上的一个函数$f$满足$df=0$,我们来刻画所有使得$df=0$的点。

\end{document}