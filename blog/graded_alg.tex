\documentclass[11pt]{article}
\usepackage{../noteheader}
\usepackage{ctex,paralist}

\theoremstyle{definition}
	\newtheorem{para}{}[section]
		\renewcommand{\thepara}{\thesection.\arabic{para}}
	\newtheorem{defi}[para]{Definition}
\theoremstyle{plain}
	\newtheorem{lem}[para]{Lemma}
	\newtheorem{thm}[para]{Theorem}
	\newtheorem{coro}[para]{Corollary}
	\newtheorem{pro}[para]{Proposition}
\renewcommand*{\proofname}{Proof}

\begin{document}
    
该文的目标是整理一些分次结构、超结构、复结构的常用构造、记号等,以便今后使用。参考文献繁杂,且大部分证明是非常基础的线性代数习题,所以会略去不少证明。

\section{分次结构}

分次结构的出现在物理中是自然的,其中最自然的可能是来自于粒子统计% https://www.doubilee.com/particle\_statistics/
,其告诉我们,粒子分为两类,其一是玻色子,其二是费米子,分别描述它们的经典理论是两类非常不同的代数,在前者中,两个量$a$和$b$的乘法满足$ab=ba$,在后者中,乘法则满足$ab=-ba$. 历史上,数学家和物理学家就定义了一系列具有$\mathbb{Z}_2$-分次的对象,他们通常以 super 为名,如 supermanifold 等。

同时,从自旋统计定理,自旋为半整数的粒子是费米子,自旋为整数的粒子是玻色子。所以,从这种角度来看,$\mathbb{Z}_2$-分次并不是最自然的,$\mathbb{Z}$-分次应当会更自然些,它可以告诉我们相应粒子(场)的自旋到底是什么。同时,可以看到,从$\mathbb Z$-分次很容易回到$\mathbb Z_2$-分次,所以我们从$\mathbb{Z}$-分次的矢量空间开始。

\begin{defi}
    设$V$是一个$k$-矢量空间,可以写成直和$V=\bigoplus_{i\in\mathbb Z} V_i$,其中$V_i$中的元素被称为$i$-次元,这样一个矢量空间被称为分次矢量空间。如果$v\in V$处于某个$V_i$中,则称其为齐次的,此时若$v\neq 0$是$i$-次元,定义$\deg(v)=i$. 
\end{defi}

\begin{defi}
    设有两个分次$k$-矢量空间$V$和$W$,$k$-同态$f:V\to W$满足$f(V_i)\subset W_{i+a}$,则称$f$是一个$a$-次同态。特别地,$0$-次同态简称为同态。
\end{defi}

许多矢量空间上的构造可以直接移到分次矢量空间上。

\begin{defi}设$V$和$W$都是$k$-分次矢量空间,$f:V\to W$是一个$a$-次同态,我们定义:
    \begin{compactenum}[\quad (1)]
        \item 直和$V\oplus W$可以定义如下分次$(V\oplus W)_n=V_n\oplus W_n$.
        \item 张量积$V\otimes_k W$可以定义如下分次$(V\otimes_k W)_n=\bigoplus_{a+b=n}V_a\otimes W_b$.
        \item 移动$i$-次分次$V[i]$,其分次结构为$(V[i])_{j}=V_{i+j}$.
        \item $V^*$是$V$的对偶空间,则我们可以将其分解为$V^*=\bigoplus_{i\in\mathbb Z}V_i^*$. 由于$V_i^*$中的元素将$i$-次齐次元变成$0$-次的,所以我们赋予分次$-i$,即$(V^*)_i=(V_{-i})^*$.
    \end{compactenum}
\end{defi}

然后是分次代数。

\begin{defi}
    若分次$k$-矢量空间$V$上有一个乘法使得$V_iV_j\subset V_{i+j}$,则称其为一个分次$k$-代数。一般来说,我们要求这个乘法是有单位元的,且其为零次的,此时我们有自然的含入$k\subset V_0$. 对两个齐次元$v$, $w\in V$,若
\[
    vw=(-1)^{\deg(w)\deg(v)}wv\quad \Leftrightarrow \quad 
    [v,w]=vw-(-1)^{\deg(w)\deg(v)-1}wv=0,
\]
则称他们是可交换的(对易的)。若任意齐次元之间是可交换的,则称这是一个交换$k$-代数。
\end{defi}

这里$[*,*]$可以线性地延拓到整个$V$上,他是对玻色子括号$[*,*]$和费米子$\{*,*\}$括号的推广。所以,可以看到,经典的对易和反对易,在这里都是对易的。注意,因为全同粒子假设,玻色子和费米子之间是可以区分的,所以在物理构造中,玻色算符和费米算符的对易关系是可以任意给定的,而这里,偶数次元和任意元素都可以对易。

数学上,这样的交换分次代数也不少见,其中非常常见的是流形的de Rham complex,流形$M$上的余切丛的截面的反对称化$\Gamma(\Lambda^*M)$. 但是de Rham complex上还有一个非常重要的构造,外微分算子。这样对象的推广被称为\textit{微分分次代数} (differential graded algebra),简记为 dga. 

\begin{defi}
    若$V$是一个交换分次$k$-代数,如果有一个$1$-次$k$-代数同态$d:V\to V$满足:$dV_i\subset V_{i+1}$, $d^2=0$和 Leibniz 法则,即对齐次元$a$满足
    \[
        d(ab)=(da)b+(-1)^{\deg(a)}a(db).
    \]
    则称$(V,d)$是一个微分分次代数。一个dga同态$f$不仅是一个交换$k$-分次代数同态,还需满足$df=fd$,即de Rham complex间拉回的推广。
\end{defi}

对每个分次交换代数$V$,$(V,0)$就是一个平凡的 dga. 
显然,对每个 dga,都有对应的上同调群$H^i(V,d)=\ker d/\operatorname{im} d$,且 dga 同态会诱导相应的上同调群同态。

其中,我们可以把 Leibniz 法则抽象出来,定义:
\begin{defi}
    若$V$是一个交换分次$k$-代数,如果有一个同态满足
    \[
        d(ab)=(da)b+(-1)^{n\deg(a)}a(db).
    \]
    则称$d$是一个$n$-次导子,其中$n$也被记作$\deg(d)$.
\end{defi}

\begin{para}[对称化]
    设$V$是一个分次$k$-矢量空间,我们可以定义一个其生成的最大的交换分次代数$\operatorname{Sym}(V)$. 以范畴论的语言,即相应的自由对象 (free object). 回忆,在交换$k$-代数范畴,这样的自由对象是多项式代数。

    $\operatorname{Sym}(V)$的构造是直接的,我们考虑多项式环$k[V]$,上面可以定义自然的分次:对齐次元$a$和$b$,定义形式乘法$ab$的分次为$\deg(a)+\deg(b)$. 所以,$ab$和$ba$具有相同分次。这时候考虑所有$[a,b]=vw-(-1)^{\deg(w)\deg(v)-1}wv$生成的理想$I$,则
    \[
        \operatorname{Sym}(V)=k[V]/I.
    \]
    显然,任何$V$生成的交换分次代数(即以$V$中元素为生成元)都可以经由$\operatorname{Sym}(V)$分解。
\end{para}

我们来看一个非常简单的例子,$V=V_0\oplus V_1=k\oplus V_1$,这个分次矢量空间只有两个分次。此时$\operatorname{Sym}(V)$就是外代数$\Lambda^*(V_1)$,其中$\Lambda^0(V_1)=k$. 类似地,如果$V$只有$0$-分次,则$\operatorname{Sym}(V)$就是普通的$\operatorname{Sym}(V)$.

\begin{para}[分次代数的谱]
    设$V$是一个分次交换$k$-代数,我们可以定义出$V$-模,特别地,$V$的理想。设$W\subset V$是$V$的一个$k$-子模,且任取$a\in V$都有$aW\subset W$,则称$W$是一个左理想,同理可以定义右理想。此外,如果(左、右)理想可以由齐次元生成,则称其为(左、右)齐次理想。

    我们断言,此时左齐次理想等价于右齐次理想。实际上,如果$W$是右理想,则任取齐次元$a$,都有$Wa\subset W$,因为如果$ba\in W$,则$-ba\in W$,所以$aW\subset W$.
\end{para}

\begin{para}[分次对偶空间的谱]
    众所周知,在代数几何中,有限维矢量空间$V$可以看成概形$\operatorname{Spec}(\operatorname{Sym}(V^*))$的$k$-值点的集合。这是因为,设$\{e_i\}$是$V$的一组基,$\{e^i\}$是$V^*$的对偶基,则$\operatorname{Spec}(\operatorname{Sym}(V^*))$的$k$-值点一一对应极大理想$\langle e^i-a_i\rangle_i$,其中$a_i\in k$,而这也自然对应着$a=\sum_i a_ie_i\in V$,因为通过赋值映射,自然就有$e^i(a)=a_i$. 这里,对分次$k$-矢量空间$V$,类似地可以定义出$\operatorname{Spec}(\operatorname{Sym}(V^*))$.
\end{para}

\section{复结构}

我们这节统统记$V$是一个有限维$\mathbb R$-矢量空间。

\begin{defi}[近复结构]
    一个自同态$I:V\to V$满足$I^2=-\mathrm{id}$被称为$V$上的一个近复结构。
\end{defi}

对$I^2=-\mathrm{id}$求一次行列式立刻得到$|I|^2=(-1)^{\dim V}$,所以只有当$\dim V$为偶数时,才可能存在近复结构。

\end{document}