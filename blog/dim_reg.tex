\documentclass[12pt]{article}
\usepackage{ctex}
\usepackage{../noteheader}

\theoremstyle{definition}
	\newtheorem{para}{}[section]
		\renewcommand{\thepara}{\thesection.\arabic{para}}
	\newtheorem{defi}[para]{Definition}
\theoremstyle{plain}
	\newtheorem{lem}[para]{Lemma}
	\newtheorem{thm}[para]{Theorem}
	\newtheorem{coro}[para]{Corollary}
	\newtheorem{pro}[para]{Proposition}
\renewcommand*{\proofname}{Proof}

\title{维数正规化碎碎念}
\author{李卜外}

\begin{document}

\maketitle

\subsection{标量费曼积分与维数正规化}

我们下面只考虑标量费曼积分,其一般形式如下
\[
	I=e^{L\epsilon \gamma_{\mathrm{E}}}\left(\mu^{2}\right)^{v-\frac{L D}{2}} \int \prod_{r=1}^{l} \frac{d^{D} l_{r}}{i \pi^{\frac{D}{2}}} \prod_{j=1}^{n_{\text {int }}} \frac{1}{\left(-q_{j}^{2}+m_{j}^{2}\right)^{v_{j}}},
\]
其中$L$是圈数,$\{l_r\}_{r=1,\dots,L}$为相应的圈动量. $D$是理论的维数,在维数正规化的时候通常会取做$D=n-2\epsilon$的形式,$n$是一个正整数,之前的$e^{L\epsilon \gamma_{\mathrm{E}}}$用来吸收积分展开中所有的Euler常数
\[
	\gamma_{\mathrm{E}}:=\lim_{n\to \infty}\biggl(-\log n+\sum_{k=1}^n\frac{1}{k}\biggr).
\]
而$v_j\geq 1$是传播子在壳因子$(-q_{j}^{2}+m_{j}^{2})$上的幂次,$v=\sum_i v_i$是所有$v_i$的和,$\mu^{2}$具有质量量纲,可以理解为一个给定的重整化标度,用来消去后面积分中的质量量纲。

对于一圈积分(圈动量记作$l$),我们通常首先利用费曼参数化将传播子写成一个积分
\[
	\prod_{i=1}^n\frac{1}{A_i^{k_i}}=\frac{\Gamma(k_1)\cdots \Gamma(k_n)}{\Gamma(K)}\int_{\mathbb R^K_+}d^K a_i \prod_{i=1}^n a_i^{k_i-1}\frac{\delta(a_1+\cdots+a_n-1)}{(\sum_{i=1}^n a_iA_i)^K},
\]
其中$K=k_1+\cdots+k_n$. 此时,每个$A_i$形如$-(l-x_i)^2+m_i^2$,通过配方,可以写作
\[
	\sum_{i=1}^n a_iA_i= -(l-l_0)^2+U,
\]
其中$U,V,l_0$是无关圈动量的函数,于是,通过将$l$做平移,积分归结为如下类型
\[
	\int \frac{d^{D} k}{i \pi^{\frac{D}{2}}} \frac{\left(-k^{2}\right)^{a}}{\left(-U k^{2}+V\right)^{v}}=\frac{\Gamma\left(\frac{D}{2}+a\right)}{\Gamma\left(\frac{D}{2}\right)} \frac{\Gamma\left(v-\frac{D}{2}-a\right)}{\Gamma(v)} \frac{U^{-\frac{D}{2}-a}}{V^{v-\frac{D}{2}-a}}.
\]
对于高圈积分,类似的技巧也是可以使用的,但我们这里不展开了。

对于维数正规化$D=n-2\epsilon$,在上面的圈积分积完后,我们会得到如下因子
\[
	\frac{\Gamma\left(\frac{n}{2}+a-\epsilon\right)}{\Gamma\left(\frac{n}{2}-\epsilon\right)} \frac{\Gamma\left(v-\frac{n}{2}-a+\epsilon\right)}{\Gamma(v)} \frac{U^{-\frac{n}{2}-a+\epsilon}}{V^{v-\frac{n}{2}-a+\epsilon}}
\]
这使得费曼参数变得非常复杂,被积函数不再是有理函数。处理这类积分的传统方法之一,是通过Mellin变换来得到其结果。下面我们解释Mellin变换为何会自然出现于维数正规化的积分中。

\subsection{维数正规化与Mellin变换}

取$D=n+2k$,$n$还是正整数,则圈积分可以分解为$l=l_1+l_2$,其中$l_1$处于$n$维动量空间,而$l_2$处于与其正交的$2k$维动量空间,并且,Lorentz度规完全由$l_1$实现,则$l_2$的度规是欧几里得度规。此时由于所有外动量都处于$l_1$的$n$维动量空间中,所以被积分函数关于$l_2$的依赖只可能是$\gamma:=l_2^2$. 则
\[
	\int \frac{d^{D} l}{i \pi^{\frac{D}{2}}}=\int \frac{d^{n} l_1}{i \pi^{\frac{n}{2}}}\int \frac{d^{2k} l_2}{\pi^{k}}=\int \frac{d^{n} l_1}{i \pi^{\frac{n}{2}}}\int_0^\infty \frac{d |l_2|}{\pi^{k}} |l_2|^{2k-1} \Omega_{2k-1}=\frac{1}{2k \pi^k}\int \frac{d^{n} l_1}{i \pi^{\frac{n}{2}}}\int_0^\infty d\gamma^k \Omega_{2k},
\]
其中$\Omega_{2k}$是$2k$维空间中$2k-1$维单位球面的面积,
\[
	\Omega_{2k}=\frac{2\pi^k}{\Gamma(k)},
\]
则
\[
	\int \frac{d^{D} l}{i \pi^{\frac{D}{2}}}=\frac{1}{k\Gamma(k)}\int \frac{d^{n} l_1}{i \pi^{\frac{n}{2}}}\int_0^\infty d\gamma^k=\frac{1}{\Gamma(k)}\int \frac{d^{n} l_1}{i \pi^{\frac{n}{2}}}\int_0^\infty d\gamma\, \gamma^{k-1}.
\]
此时,我们可以将$k$取做任意实数(解析延拓),而积分变换$\int_0^\infty d\gamma\, \gamma^{k-1}$实际即是Mellin变换!

以一圈积分为例,容易看到在费曼参数化之后,分母写作
\[
	\sum_{i=1}^n a_iA_i= -(l-l_0)^2+U+\gamma.
\]
对其做Mellin变换时,会遇到如下积分
\[
	\frac{1}{\Gamma(k)}\int_0^\infty d\gamma \, \gamma^{k-1}\, \frac{1}{(A+B\gamma)^n}=\frac{\Gamma(n-k)}{\Gamma(n)}\frac{1}{A^{n-k}B^k},
\]
具体到一圈标量积分(无分子),则
\[
	\frac{1}{\Gamma(k)}\int_0^\infty d\gamma \, \gamma^{k-1}\, \frac{1}{(\sum_{i=1}^n a_iA_i)^v}=\frac{\Gamma(v-k)}{\Gamma(v)}\frac{1}{(-(l-l_0)^2+U)^{v-k}}.
\]
然后再做圈积分,得到
\[
	\frac{\Gamma\left(v-k-\frac{n}{2}\right)}{\Gamma(v)}\frac{1}{U^{v-k-\frac{n}{2}}}=\frac{\Gamma\left(v-\frac{D}{2}\right)}{\Gamma(v)} \frac{1}{U^{v-\frac{D}{2}}},
\]
就得到了我们之前$D=n+2k$的结果。

于是,在一些例子中,我们可以调换积分顺序,先计算圈积分,
\[
	\int \frac{d^{D} l}{i \pi^{\frac{D}{2}}}=\frac{1}{\Gamma(k)}\int_0^\infty d\gamma\, \gamma^{k-1}\int \frac{d^{n} l}{i \pi^{\frac{n}{2}}}
\]
然后再计算“额外维”的动量积分,即进行Mellin变换。比如对于一圈标量积分(无分子)来说,圈动量的积分做完后
\[
	\int \frac{d^{n} l}{i \pi^{\frac{n}{2}}} \frac{1}{(-(l-l_0)^2+U+\gamma)^v}
	=\frac{\Gamma\left(v-\frac{n}{2}\right)}{\Gamma(v)} \frac{1}{(U+\gamma)^{v-\frac{n}{2}}}
\]
新的费曼参数积分可能会变得相对容易计算,比如对于$n=4$,$(U+\gamma)^{v-\frac{n}{2}}=(U+\gamma)^{v-2}$将是费曼参数的多项式。从另一种角度来看,维数正规化变成了某种“质量”正规化,我们叫做$\gamma$-正规化。

我们可以进一步考虑更高圈的(标量)积分,以两圈为例,两个圈的$2k$-维积分可以写作
\[
	\frac{1}{\Gamma(k)^2}\int_0^\infty d\gamma_1
	\int_0^\infty d\gamma_2 (\gamma_1\gamma_2)^{k-1}.
\]
注意到,对含$l_1$的传播子多一个$\gamma_1$,对含有$l_2$的传播子多一个$\gamma_2$,
利用费曼参数化,我们即要考虑如下积分
\[
	I=\frac{1}{\Gamma(k)^2}\int_0^\infty d\gamma_1
	\int_0^\infty d\gamma_2 (\gamma_1\gamma_2)^{k-1}
	\frac{1}{(A+B\gamma_1+C\gamma_2)^n},
\]
积掉$\gamma_1$和$\gamma_2$,我们得到
\[
	I=\frac{\Gamma(n-k)}{\Gamma(n)}\frac{\Gamma(n-2k)}{\Gamma(n-k)}
	\frac{1}{A^{n-2k}B^kC^k}=\frac{\Gamma(n-2k)}{\Gamma(n)}
	\frac{1}{A^{n-2k}(BC)^k}=\frac{\Gamma(n-2k)}{\Gamma(n)}
	\frac{1}{A^{n-k}(BC/A)^{k}},
\]
我们可以重写为另一个$\gamma$的积分
\[
	I=\frac{\Gamma(n-2k)}{\Gamma(n-k)\Gamma(k)}\int_0^\infty d\gamma\,\gamma^{k-1}
	\frac{A^n}{(A^2+BC\gamma)^n},
\]
这里我们保持了被积函数是有理的,代价就是$\Gamma$函数系数变复杂了。
此外,处理多圈积分时候,容易遇见我们可以先做掉几个圈保证不发散,此时
只需要对剩余的圈积分引入相应的$\gamma$即可。

现在我们来考虑Mellin变换,即
\[
	\mathcal M[f](k):=\int_0^\infty d\gamma\, \gamma^{k-1} f(\gamma).
\]
Mellin逆变换定理说,如果$\psi(k)$在$a<\operatorname{Re}(k)<b$内解析,且在这个区域内在$\operatorname{Im}(k)\to \pm \infty$时一致收敛于$0$,且积分($c$为一个满足$a<c<b$的实数)
\[
	\mathcal M^{-1}[\psi](\gamma):=\frac{1}{2\pi i}\int_{c-i\infty}^{c+i\infty}\gamma^{-k}\psi(k)dk
\]
绝对收敛时,则$\mathcal M^{-1}[\psi]$的Mellin变换为$\psi$,即$\mathcal M[\mathcal M^{-1}[\psi]]=\psi$. 特别地,如果$\psi$是某个函数$f$的Mellin变换,$f$在正实轴上连续,且Mellin变换的积分在$a<\operatorname{Re}(k)<b$中绝对收敛,则$f=M^{-1}[\psi]$.

将其应用到我们的积分,对一圈来说,则我们需要考虑
\[
	\mathcal M^{-1}[\Gamma(k)I(k)](\gamma)=\frac{1}{2\pi i}\int_{c-i\infty}^{c+i\infty}\gamma^{-k}\Gamma(k)I(k)dk,
\]
其中$I(k)$是费曼参数化下的积分结果,而$\mathcal M^{-1}[\Gamma(k)I(k)](\gamma)$是$\gamma$-正规化的结果。现在,假设在$k\sim 0$时,积分结果可以展开为
\[
	\exp (-\gamma_{\mathrm{E}} k)I(k)=\sum_{i=-N}^0 (-1)^iI_i k^i+O(k).
\]
或者令$k=-\epsilon$,
\[
	\exp (\gamma_{\mathrm{E}} \epsilon)I(-\epsilon)=\sum_{i=-N}^0 I_i \epsilon^i+O(\epsilon).
\]
从物理上,我们可以知道,$I(k)$在有限处的极点只会出现在整数处,而$\Gamma(k)$只会出现在非负整数处,我们考虑解析的区间$0<\operatorname{Re}(k)<1$,然后将$c$取得接近$0$,再考虑左无穷大半圆围道,这部分围道上的贡献由$\gamma^{-k}\Gamma(k)I(k)$的极限行为控制,这里假设收敛地够快使其为零,则上述Mellin反变换的结果是这样的围道中$\gamma^{-k}\Gamma(k)I(k)$的留数之和。特别地,对于负整数$k=-N$,留数会正比于$\gamma^N$,所以,如果我们只关心具体$O(\gamma)$的积分结果,则其完全由$k=0$处的留数给出,即由$\gamma^{-k}\Gamma(k)I(k)$的$k^{-1}$系数确定。比如说,若$N=2$,则
\[
	\mathcal M^{-1}[\Gamma(k)I(k)](\gamma)=I_0+\frac{\zeta_2}{2}I_{-2}+I_{-1}\log(\gamma)+\frac{1}{2}I_{-2}\log(\gamma)^2 + O(\gamma).
\]
其中可能会用到$\Gamma$函数在$k=0$附近的分解
\[
	\Gamma(1+k)=k\Gamma(k)=\exp \left(-\gamma_{\mathrm{E}} k+\sum_{n=2}^{\infty} \frac{(-1)^{n}}{n} \zeta_{n} k^{n}\right),
\]
这里$\exp (-\gamma_{\mathrm{E}} k)$已经用来吃掉了所有$I(k)$展开中的$\gamma_{\mathrm{E}}$. 所以,至少可以看到,如果$I(k)$具有$k^{-N}$次发散,则$\mathcal M^{-1}[\Gamma(k)I(k)](\gamma)$至多具有$\log(\gamma)^N$次发散。于是,在一些情况下,我们只需计算$\mathcal M^{-1}[\Gamma(k)I(k)](\gamma)$的$\log(\gamma)$展开就可以反过来推出费曼参数化下的$I(k)$的展开。

\subsection{$\gamma$-正规化下费曼积分的计算}

现在的问题是怎么算$\gamma$-正规化下的费曼积分,当然也不一定是$\gamma$-正规化,比如
可以是小质量正规化下的积分,他们统统都可以写作如下形式
\[
	\int_{\mathbb R_+^n}\frac{dx_1}{x_1}\cdots \frac{dx_n}{x_n}f(x;\gamma),
\]
其中$f$是一个有理函数,$x$是费曼参数,而$\gamma$是一个小的正参数。这个积分在$\gamma\to 0$的时候至多
具有$\log(\gamma)$的幂次的发散,来自于$f(x;0)$在边界上不为零。所以实际上也可以看到
$\log(\gamma)$的幂次最高是$n$.

下面我们考虑一个例子,四维的2-mass easy的一圈四边形,
\[
	I_{\text{2me}}=\int \frac{d^{D} x_0}{i \pi^{\frac{D}{2}}} \frac{1}{(x_0-x_1)^2(x_0-x_2)^2(x_0-x_3)^2(x_0-x_4)^2},
\]
这里$x_{i+1}-x_i=p_i$,且$p_2^2=p_4^2=0$,利用费曼参数化在$D=4-2\epsilon$中可以计算得到
\[
\begin{aligned}
	I_{\text{2me}}&=\Gamma(4)\int_{\mathbb R_+^4}d^4a\int \frac{d^{D} x_0}{i \pi^{\frac{D}{2}}} \frac{\delta(a_1+a_2+a_3+a_4-1)}{(a_1(x_0-x_1)^2+a_2(x_0-x_2)^2+a_3(x_0-x_3)^2+a_4(x_0-x_4)^2)^4}\\
	&=\Gamma(4)\int_{\mathbb R_+^4}d^4a\int \frac{d^{D} x_0}{i \pi^{\frac{D}{2}}} \frac{\delta(a_1+a_2+a_3+a_4-1)}{(x_0^2+\sum_{i<j}a_i a_j (x_i-x_j)^2)^4}\\
	&=\Gamma(4)\frac{1}{\Gamma(-\epsilon)}\int_0^\infty d\gamma \, \gamma^{-\epsilon-1}\int_{\mathbb R_+^4}d^4a\int \frac{d^{D} x_0}{i \pi^{\frac{D}{2}}} \frac{\delta(a_1+a_2+a_3+a_4-1)}{(x_0^2+\sum_{i<j}a_i a_j (x_i-x_j)^2-\gamma)^4}\\
	&=\Gamma(4)\frac{\Gamma(2)}{\Gamma(4)}\frac{1}{\Gamma(-\epsilon)}\int_0^\infty d\gamma \, \gamma^{-\epsilon-1}\int_{\mathbb R_+^4}d^4a\frac{\delta(a_1+a_2+a_3+a_4-1)}{(\sum_{i<j}a_i a_j (x_i-x_j)^2-\gamma)^2}\\
	&=\frac{1}{\Gamma(-\epsilon)}\int_0^\infty d\gamma \, \gamma^{-\epsilon-1}\int_{\mathbb R_+^4}d^4a\frac{\delta(a_1+a_2+a_3+a_4-1)}{(\sum_{i<j}a_i a_j (x_i-x_j)^2-\gamma)^2}\\
\end{aligned}
\]
所以,我们只需计算里面这个费曼参数积分的$\log(\gamma)$展开系数即可,将分母具体写出,即
\[
	U=\sum_{i<j}a_i a_j (x_i-x_j)^2-\gamma=a_1a_2 m_1^2+a_1a_3s+a_2a_4t+a_3a_4m_3^2-(a_1+a_2+a_3+a_4)^2\gamma,
\]
其中$m_i^2=p_i^2$,而$s=(p_1+p_2)^2$, $t=(p_2+p_3)^2$.

对于积分 
\[
	\int_{\mathbb R_+^4}d^4a\frac{\delta(a_1+a_2+a_3+a_4-1)}{U^2},
\]
由于$1/U^2$关于参数是$-4$次齐次的,所以我们设$a_4=1$,这样积分变成了
\[
	\int_{\mathbb R_+^3}d^3a\frac{1}{(a_1a_2 m_1^2+a_1a_3s+a_2t+a_3m_3^2-(a_1+a_2+a_3+1)^2\gamma)^2},
\]
可以看到,若$\gamma=0$,则$a_2\sim a_3\sim 0$处或$a_2\sim a_3\sim \infty$会发散。为计算这个积分,我们引入如下技术。



\begin{pro}
设一个光滑函数$f$具有如下极限行为
\[
f(x_1t,\dots,x_nt;\epsilon t)\xrightarrow{t\sim 0^+}\frac{1}{t^n}f_0(x_1,\dots,x_n;\epsilon)+O(t^{-n+1}),
\]
其中自然数$k\leq n$. 则积分
\[
\int_{\mathbb R_+^n} f(\mathbf{x};\epsilon)\,d^nx-\int_{\mathbb R_+^n - B_+(\lambda \epsilon)} f(\mathbf{x};0)\,d^nx
\]
在$\epsilon\to 0^+$时候有限,其中为球心位于原点的半径为$\lambda \epsilon$的球与$\mathbb R_+^n$的交$B_+(\lambda \epsilon):=\{|\mathbf x|<\lambda \epsilon\}\cap \mathbb R_+^n$,而$\lambda$为任意正数。并且,上述积分的有限差值可以表示为$I_1+I_2$,其中
\[
	I_1=\int_{B_+(\lambda\mu)}f_0(\mathbf{x};\mu)\,d^nx,\quad I_2=\int_{\mathbb R_+^n - B_+(\lambda\mu)} (f_0(\mathbf x;\mu)-f_0(\mathbf x;0))\,d^nx
\]
为$f_0$的积分,其中$\mu$为另一个任意正数。
\end{pro}

我们可以用其他区域来替换$B_+(\lambda \epsilon)$,比方说,取边长为$\epsilon$的正方体$[0,\epsilon]^n:=\{ 0<x_i<\epsilon \text{ for }1\leq i\leq n\}$. 这是因为存在$\lambda_1$和$\lambda_2$使得$B_+(\lambda_1\epsilon)\subset [0,\epsilon]^n\subset B_+(\lambda_2\epsilon)$对任意的$\epsilon$都成立。

\begin{proof}
不妨设$\mu=1$. 我们首先将$\int_{\mathbb R_+^n} f(\mathbf{x};\epsilon)\,d^nx$分解为
\[
\int_{\mathbb R_+^n} f(\mathbf{x};\epsilon)\,d^nx=\int_{B_+(\lambda \epsilon)} f(\mathbf{x};\epsilon)\,d^nx+\int_{\mathbb R_+^n - B_+(\lambda \epsilon)} (f(\mathbf{x};\epsilon)-f(\mathbf{x};0))d^nx + \int_{\mathbb R_+^n - B_+(\lambda \epsilon)} f(\mathbf{x};0)d^nx
\]
我们需要算中间两项在$\epsilon\to 0$时候的极限行为,记
\[
I_1=\int_{B_+(\lambda \epsilon)} f(\mathbf{x};\epsilon)\,d^nx
\quad \text{and} \quad
I_2=\int_{\mathbb R_+^n - B_+(\lambda \epsilon)} (f(\mathbf{x};\epsilon)-f(\mathbf{x};0))d^nx.
\]
对$I_1$,我们坐标变换$\mathbf{x}\to \mathbf{x}\epsilon$
\[
I_1=\epsilon^n\int_{B_+(\lambda)} f(\mathbf{x}\epsilon;\epsilon)\,d^nx
\to \int_{B_+(\lambda)} f_0(\mathbf{x};1)\,d^nx,
\]
另一方面,$I_2$只可能在$B_+(\lambda \epsilon)$附近发散,此时我们在附近考虑积分,注意到$|x|$也很小,所以,在坐标变换$\mathbf{x}\to \mathbf{x}\epsilon$下,其极限行为为
\[
\begin{aligned}
\int_{\mathbb R_+^n - B_+(\lambda \epsilon)} (f(\mathbf{x};\epsilon)-f(\mathbf{x};0))d^nx &\sim \int_{|x|\sim \lambda } \left(f_0(\mathbf{x};1)-\frac{1}{|x|^n}f_0(\mathbf{x}/|x|;0)\right)\,d^n x
\end{aligned}
\]
由于其下界不为零,第二项并不会导致发散。同时,给定一个有限截断$B_+(\Lambda)$,我们将积分分解为
\[
	\int_{\mathbb R_+^n - B_+(\lambda \epsilon)}=\int_{\mathbb R_+^n - B_+(\Lambda)}+\int_{B_+(\Lambda) - B_+(\lambda \epsilon)},
\]
在第一个区间里,我们可以安全地Taylor展开$f(\mathbf{x};\epsilon)-f(\mathbf{x};0)$,可以知道他是$O(\epsilon)$的,对于第二个积分,然后替换$\mathbf{x}\to \epsilon \mathbf{x}$,接着应用极限行为,可得到
\[
	\int_{B_+(\Lambda/\epsilon) - B_+(\lambda)}(f_0(\mathbf{x};1)-f_0(\mathbf{x};0))d^nx
\]
令$\epsilon\to 0$,其积分区域就变成了$\mathbb{R}_+^n-B_+(\lambda)$.
\end{proof}

一般来说,在积分中,因为费曼参数在边界的行为并不会那么简单,他可能在许多边界处产生发散。对积分
\[
	\int_{\mathbb R_+^n} f(\mathbf{x};\epsilon)\,\prod_{i=1}^n d\log x_i,
\]
如果存在非零的$\alpha=(\alpha_1,\dots,\alpha_n)\in \mathbb Q^n$使得
\[
f(t^{\alpha_1}x_1,\dots,t^{\alpha_n}x_n;t\epsilon)\xrightarrow{t\to 0^+}f_0(x_1,\dots,x_n;\epsilon)
\]
有限且不为零,则积分就可能在边界上导致发散。在我们的应用中,$\alpha$的分量一般只会取到$\{-1,0,+1\}$. 对于给定的这样一个$\alpha$,我们来到的是边界$\{x_i^{\alpha_i}\to 0\,:\,\alpha_i\neq 0\}_i$附近。我们将所有这样$\alpha$的集合记作$\Phi_f$,如果下标$f$显明,则我们经常会略去他。这个极限行为告诉我们,实际上整个积分只有$\log$的幂次型发散,而我们恰好也只需要考虑这种类型的发散。

我们下面经常会考虑如下变量变换
\[
	b_i=\prod_{j=1}^n a_j^{N_{ij}},
\]
其中$N$是可逆的有理矩阵(最好是整数矩阵),这样,被积区间$\mathbb R_+^n$首先是在积分变换下不变的,但积分测度会产生变化
\[
	\bigwedge_{i=1}^n d\log b_i=\bigwedge_{i=1}^n\sum_{j=1}^n N_{ij}d\log a_j = \det(N) \bigwedge_{i=1}^n d\log a_i.
\]
其次,如果$a_i\to t^{\alpha_i}a_i$,则
\[
	b_i\to t^{(N \alpha)_i} b_i,
\]
这意味着如果$\alpha\in \Phi_f$,对变量替换后的新被积函数$f'$,则有$N \alpha\in \Phi_{f'}$. 


回到2-mass easy的一圈四边形积分,
\[
	\int_{\mathbb R_+^3}d^3a\frac{1}{(a_1a_2 m_1^2+a_1a_3s+a_2t+a_3m_3^2-(a_1+a_2+a_3+1)^2\gamma)^2},
\]
不难分析得到
\[
	\Phi_f=\{(0, -1, -1), (0, 1, 1)\},
\]
或者适当积分变换一下,令$\{a_2\to a_2a_3,a_3\to a_3/a_2\}$,这样新的$\Phi$为
\[
	\Phi_{f'}=\{(0, 0, -1), (0, 0, 1)\}.
\]

再举个2-mass hard的一圈四边形积分,费曼参数化与2-mass easy的一圈四边形积分相同,但是$x_{ij}:=(x_i-x_j)^2$的条件不同,被积函数如下
\[
	\int_{\mathbb R_+^3}d^3a\frac{1}{(a_1 a_2 x_{12}+a_1 a_3 x_{13}+a_1 x_{14}+a_2 x_{24}-\gamma (a_1+a_2+a_3+1)^2)^2}.
\]
容易计算得
\[
	\Phi_f=\{(0,-1,-1),(1,1,0)\}.
\]
我们可以令$\{b_1=a_1,b_2=a_1/a_2,b_3=a_1 a_3/a_2\}$,则
\[
	\Phi_{f'}=\{(0,1,0),(1,0,0)\}.
\]

下面我们来讲如何计算$\Phi_f$. 

\begin{pro}
如果函数$f(\mathbf{x};\epsilon)$是$\mathbf{x}$和$\epsilon$的有理函数,且$f(\mathbf{x};\epsilon)$关于变量$(x_0:=\epsilon,x_1,\dots,x_n)$的Newton多面体是$n+1$维的多面体,则$\alpha\in \Phi_i$当且仅当$X_0+\sum_{i}\alpha_i X_i=0$是$f(\mathbf{x};\epsilon)$的Newton多面体的一个面,其中$\{X_i\}$是Newton多面体所处空间的坐标。
\end{pro}

对一个多项式$g=\sum_I a_{I_1,I_2,\dots,I_n}x_1^{n^I_1}\cdots x_n^{n^I_n}$而言,它的Newton多面体$\operatorname{NP}(g)$是幂次向量们$\{(n^I_1,\dots,n^I_n)\}_I$构成的凸包,于是,Newton多面体的顶点一定是某个幂次向量。可以证明,
\[
	\operatorname{NP}(gh)=\operatorname{NP}(g)+\operatorname{NP}(g),
\]
其中加号的意思是Minkowski求和,定义为
\[
A+B:=\{x+y\,:\,x\in A, y\in B\}.
\]
那么,对于有理式$g/h$,则它的Newton多面体被定义为$\operatorname{NP}(g)-\operatorname{NP}(g)$,减号的意思为
\[
A-B:=\{x-y\,:\,x\in A, y\in B\}.
\]
类似地,对$a\in \mathbb Q$,我们定义$a A:=\{ax\,:\, x\in A\}$,不难看到,如果$a$是正整数,则$\operatorname{NP}(g^a)=a\operatorname{NP}(g)$,于是我们可以利用这个等式反过来定义$a$为任意有理数时候的$g^a$的Newton多面体。

\begin{proof}
仅对多项式的情况予以证明,我们这里将$\epsilon$标记为$x_0$,并采用多重指标记号。如果$f(\mathbf{x};\epsilon)=\sum_I a_I x^{n_I}$,则在$x_i\to t^{\alpha_i}x_i$时候,函数变为
\[
	f(\mathbf{x};\epsilon)=t^{(n_J,\alpha)}\sum_I a_I t^{(n_I-n_J,\alpha)}x^{n_I},
\]
其中$(*,*)$为欧几里得内积,而$n_J$是使得$(n_J,\alpha)$对所有的$I$最小的。于是,我们看到,在取极限下,求和项中只有$(n_I-n_J,\alpha)=0$的保留了,同时我们还要求$(n_J,\alpha)=0$. 前者告诉我们,$\alpha$是这些保留下来的$\{n_I\}$构成的面构成的法向量,于是我们可以将这个面的方程写作$X_0+\sum_{i}\alpha_i X_i=c$. 而后者$(n_J,\alpha)=0$说明$c=0$,即这个面经过原点。反过来也是容易的。
\end{proof}

这个命题告诉我们,对给定的$f$,$\Phi_f$总是有限的。并且,计算一个Newton多面体的边界并不麻烦,有相当多成熟的方法来实现这点。

下面我们正式给出这类积分的计算方法。

\begin{pro}
假设积分
	\[
		I=\int_{\mathbb R_+^n} f(\mathbf{x};\epsilon)\,\prod_{i=1}^n d\log x_i,
	\]
它至多只有$\log$的幂次型发散,对任意的$\alpha\in \Phi_f$,将$f(\{t^{\alpha_i}x_i\};t\epsilon)$在$t\to 0^+$的极限记作$f_0^\alpha(\mathbf{x};\epsilon)$. 可以看到,如果$\alpha\in \Phi_f$,则除了$c=-1$,任意的$c\alpha$都不可能出现于$\Phi_f$中。

1. 如果$\Phi_f=\{\alpha\}$,则我们可以利用之前的命题,首先通过适当的变量替换使得$\alpha$中只有一个非零的$\alpha_i=1$,那么发散只发生于$x_i$的积分上,我们这时候可以先不管其他积分,此时
\[
	\int_{0}^\infty f(\mathbf{x};\epsilon)d\log x_i-\int_{\epsilon}^\infty f(\mathbf{x};0)d\log x_i=I_1+I_2+O(\epsilon),
\]
其中
\[
	I_1=\int_{0}^1f_0^{\alpha}(\mathbf{x};1)\,d\log x_i,\quad I_2=\int_{1}^\infty (f^{\alpha}_0(\mathbf x;1)-f^{\alpha}_0(\mathbf x;0))\,d\log x_i.
\]

2. 如果$\Phi_f=\{\alpha,-\alpha\}$,类似,我们通过适当的变量替换使得$\alpha$中只有一个非零的$\alpha_i=1$,此时
\[
	\int_{0}^\infty f(\mathbf{x};\epsilon)d\log x_i-\int_{\epsilon}^{1/\epsilon} f(\mathbf{x};0)d\log x_i=I_1+I_2+I_3+I_4+O(\epsilon),
\]
其中
\[
	I_1=\int_{0}^1f_0^{\alpha}(\mathbf{x};1)\,d\log x_i,\quad I_2=\int_{1}^\infty (f^{\alpha}_0(\mathbf x;1)-f^{\alpha}_0(\mathbf x;0))\,d\log x_i,
\]
\[
	I_3=\int_{1}^\infty f_0^{-\alpha}(\mathbf{x};1)\,d\log x_i,\quad I_4=\int_{0}^1 (f^{-\alpha}_0(\mathbf x;1)-f^{-\alpha}_0(\mathbf x;0))\,d\log x_i.
\]

3. 一般情况:我们可以选取一些穿过原点的超平面$\{H_I\}_I$,他们将整个$\mathbb R^n$分解为诸多区域,使得每个区域中只有一个$\alpha\in \Phi_f$或者一对$\{\alpha,-\alpha\}\subset \Phi_f$. 同时,这些超平面也将积分区域$\mathbb R_+^n$划分成了诸多区域,经过适当的变量替换,每个区域可以重新写作$\mathbb R_+^n$. 命题归结于以上两种特殊情况。
\end{pro}

我们现在来算完 2-mass easy的一圈四边形
\[
	\int_{\mathbb R_+^3}d^3a\frac{1}{(a_1a_2 m_1^2+a_1a_3s+a_2t+a_3m_3^2-(a_1+a_2+a_3+1)^2\gamma)^2},
\]
记得有
\[
	\Phi=\{(0, -1, -1), (0, 1, 1)\}.
\]
令$\{a_2\to a_2a_3,a_3\to a_3/a_2\}$,这样新的$\Phi$为
\[
	\Phi=\{(0, 0, -1), (0, 0, 1)\}.
\]
注意到Jacobian会多一个因子$2$,令$\gamma\to 0$,发散项来自于
\[
	2\int_0^\infty da_1\int_0^\infty da_2\int_{\gamma}^{1/\gamma} da_3\frac{a_2}{a_3 \left(a_1 \left(a_2^2 m_1^2+s\right)+a_2^2 t+m_3^2\right)^2}=\frac{2}{s t-m_1^2 m_3^2}\log \left(\frac{m_1^2 m_3^2}{s t}\right)\log(\gamma).
\]
然后考虑相应的$I_1$, $I_2$, $I_3$和$I_4$,他们中会出现的函数为
\[
	f^{(0,0,1)}_{0}=\frac{2 a_2 a_3}{\left(-\gamma a_2 \left(a_1+1\right){}^2 +a_3 \left(a_1 s+m_3^2\right)+a_2^2 a_3 \left(a_1 m_1^2+t\right)\right)^2},
\]
以及
\[
	f^{(0,0,-1)}_0=\frac{2 a_2^3}{a_3 \left(-\gamma a_3 a_2^4 -2 a_3 a_2^2 \gamma -a_3 \gamma +a_1 a_2 \left(a_2^2 m_1^2+s\right)+a_2 m_3^2+a_2^3 t\right)^2}.
\]
分别令$\gamma=1$和$0$,$\gamma=0$为$\gamma=1$消去$a_3\to \infty$或者$a_3\to 0$处的发散,我们把分积完,得到
\[
	\frac{2}{s t-m_1^2 m_3^2}\biggl(\mathrm{Ls}_{-1}^{\text{2me}}\left(t, s ; m_1^2, m_3^2\right)-\frac{1}{2} \log ^2\left(-m_1^2\right)-\frac{1}{2} \log ^2\left(-m_3^2\right)+\frac{\log ^2(-s)}{2}+\frac{\log ^2(-t)}{2}\biggr),
\]
其中
\[
	\begin{aligned}
		\mathrm{Ls}_{-1}^{\text{2me}}\left(t, s ; m_1^2, m_3^2\right)=-\mathrm{Li}_{2}\left(1-\frac{m_1^2}{t}\right) &-\mathrm{Li}_{2}\left(1-\frac{m_1^2}{s}\right)-\mathrm{Li}_{2}\left(1-\frac{m_3^2}{t}\right)-\mathrm{Li}_{2}\left(1-\frac{m_3^2}{s}\right) \\
		+& \mathrm{Li}_{2}\left(1-\frac{m_1^2 m_3^2}{t s}\right)-\frac{1}{2} \log ^{2}\left(\frac{-t}{-s}\right) .
		\end{aligned}
\]
将其与发散项合并起来,并通过Mellin变换变到维数正规化下,这就完整地复现了熟知的结果
\[
	\frac{2}{s t-m_1^2 m_3^2} \left(\mathrm{Ls}_{-1}^{\text{2me}}\left(t, s ; m_1^2, m_3^2\right)+\frac{-\left(-m_1^2\right){}^{-\epsilon}-\left(-m_3^2\right){}^{-\epsilon}+(-s)^{-\epsilon}+(-t)^{-\epsilon}}{\epsilon^2}\right).
\]

我们再来算2-mass hard的一圈四边形积分,被积函数如下
\[
	\int_{\mathbb R_+^3}d^3a\frac{1}{(a_1 a_2 x_{12}+a_1 a_3 x_{13}+a_1 x_{14}+a_2 x_{24}-\gamma (a_1+a_2+a_3+1)^2)^2}.
\]
令$\{b_1=a_1,b_2=a_1/a_2,b_3=a_1 a_3/a_2\}$,则
\[
	\Phi=\{(0,1,0),(1,0,0)\},
\]
那么,我们只需将积分区域分为$b_1>b_2$和$b_1<b_2$,即引入$b_1=b_2$这个超平面,他就足以将$\Phi$分解开来。

对$b_1>b_2$的区域,我们可以做积分变换$b_2\to b_1 t/(1+t)$,这样$\{b_1,t,b_3\}$的积分区域就又变成了$\mathbb R_+^3$. 对$b_1<b_2$,故事也类似。我们这里只写出发散项的计算,有限项的完整计算留给读者。

对$b_1>b_2$的区域,发散项来自于积分
\[
	\int_{0}^\infty db_3\int_{\gamma}^\infty db_2\int_{b_2}^\infty db_1\frac{1}{b_1 b_2 \left(b_1 x_{12}+b_3 x_{13}+b_2 x_{14}+x_{24}\right)^2}
\]
不难计算得到
\[
	\frac{1}{x_{13} x_{24}}\biggl(\log(\gamma) \log \left(\frac{x_{12}}{x_{24}}\right)+\frac 12\log(\gamma)^2\biggr)
\]
对$b_1<b_2$的区域,发散项来自于积分
\[
	\int_{0}^\infty db_3\int_{\gamma}^\infty db_1\int_{b_1}^\infty db_2\frac{1}{b_1 b_2 \left(b_1 x_{12}+b_3 x_{13}+b_2 x_{14}+x_{24}\right)^2}
\]
不难计算得到
\[
	\frac{1}{x_{13} x_{24}}\biggl(\log(\gamma) \log \left(\frac{x_{14}}{x_{24}}\right)+\frac 12\log(\gamma)^2\biggr),
\]
所以他们的和给出发散
\[
	\frac{1}{x_{13} x_{24}}\biggl(\log(\gamma) \log \left(\frac{x_{14}x_{12}}{x_{24}^2}\right)+\log(\gamma)^2\biggr).
\]

在$b_1>b_2$的区域,我们考虑其他贡献,
\[
I_1=\int_{0}^\gamma db_2\int_{b_2}^\infty db_1 f(b_1,b_2;\gamma),\quad 
I_2=\int_{\gamma}^\infty db_2\int_{b_2}^\infty db_1 (f(b_1,b_2;\gamma)-f(b_1,b_2;0)),
\]
对于此二者,我们都可以直接将$b_1$的积分限取做$[\gamma,\infty]$,因为对于前者,他们相差
\[
	\int_{0}^\gamma db_2\int_{b_2}^\gamma db_1 f(b_1,b_2;\gamma)=\int_{0}^1 db_2\int_{b'_2}^1 db_1 f(\gamma b'_1,\gamma b'_2;\gamma),
\]
被积函数在$\gamma\to 0$的时候贡献$\gamma$的高次。对于后者,他们相差
\[
	\int_{\gamma}^\infty db_2\int_{\gamma}^{b_2} db_1(f(b_1,b_2;\gamma)-f(b_1,b_2;0)),
\]
由于他们只有在$b_2\sim \gamma$附近贡献,利用上面类似的讨论,可以知道这个也只贡献$\gamma$的高次。
于是,这两项贡献
\[
	-\frac{1}{2x_{13} x_{24}}\log(\gamma)^2+\frac{1}{x_{13} x_{24}}\log\biggl(\frac{-x_{24}}{x_{13}^2}\biggr)
\]


如果我们只关注发散,那么先令$\gamma\to 0$,再把$b_3$积掉,可以得到
\[
	\int_{\gamma}^\infty\int_{\gamma}^\infty \frac{db_1db_2}{b_1 b_2 x_{13} \left(b_1 x_{12}+b_2 x_{14}+x_{24}\right)}
\]
他的结果为
\[
	\frac{1}{x_{13}x_{24}}\log(\gamma)^2-\frac{1}{x_{13} x_{24}}\log \left(\frac{x_{24}^2}{x_{12} x_{14}}\right)\log(\gamma)+\frac{1}{x_{13}x_{24}}\biggl(\frac{\pi^2}{6}+\log\biggl(\frac{x_{12}}{x_{24}}\biggr)\log\biggl(\frac{x_{14}}{x_{24}}\biggr)\biggr).
\]

剩下的就是,
\[
	f_{(0,1,0);0}=\frac{b_1 b_2}{\left(-\left(b_1+b_3\right)^2 \gamma +b_2 b_1^2 x_{12}+b_2 b_3 b_1 x_{13}+b_2 b_1 x_{24}\right)^2}
\]
\[
	f_{(1,0,0);0}=\frac{b_1 b_2}{\left(-\left(b_2+b_3\right)^2 \gamma +b_1 b_3 b_2 x_{13}+b_1 b_2^2 x_{14}+b_1 b_2 x_{24}\right)^2}
\]
首先积分的是$b_2$或者$b_1$,用$\gamma=0$的被积函数消掉$\gamma=1$的无穷远处发散,得到了
\[
	\frac{\log \left(-b_1 \left(b_1 x_{12}+b_3 x_{13}+x_{24}\right)\right)-2\log \left(b_1+b_3\right)-1}{b_1 \left(b_1 x_{12}+b_3 x_{13}+x_{24}\right)^2},
\]
和
\[
	\frac{\log \left(-b_2 \left(b_3 x_{13}+b_2 x_{14}+x_{24}\right)\right)-2\log \left(b_2+b_3\right)-1}{b_2 \left(b_3 x_{13}+b_2 x_{14}+x_{24}\right)^2},
\]
注意其中第一个积分的$b_1$和第二个积分的$b_2$都是在$(\gamma,1/\gamma)$积分,而上界是可以自然取做$\infty$. 
\end{document}