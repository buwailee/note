\chapter{The Bosonic String}

\section{Action}

The bosonic string is the simplest string theory which only
describes bosons, so it is unrealistic since 
there must be fermions in the real world. However, almost
all the phenomena in more complex string theories will happen
in the bosonic string theory, so it's a nice starting point.

A bosonic string is an one-dimensional connected object moving 
through spacetime, and its trajectory forms a two-dimension object, 
called the worldsheet, embedded in the spacetime. 
The spacetime here is assumed as a plain
Minkowski space with metric
\[
	\eta_{\mu\nu}=\operatorname{diag}(-1,1,\dots,1),
\]
however, we do not need to fix the dimension of
spacetime now since it will determined by some physical conditions.

Comparing with old stories, 
string is the analogue of point particle, 
and worldsheet is the analogue of worldline. 
However, there's a main difference
between string and point particle. It's nice to think 
point particles as points on the worldline, but this does not
work for strings. Strings can only be seen as slicings of 
the worldsheet. Thus, generally, we cannot talk about the 
coordinate of a string in spacetime. This picture is fatal 
in the disscussion of string interaction.

Strings as one-dimensional connected objects, can divided 
into two classes topologically ------ \textit{open strings} 
and \textit{closed strings}. Thus, the 
worldsheet of a open string is a band locally, and 
the worldsheet of a closed string is a cylinder locally
(and may be a torus globally).

It's also fine to consider objects with higher dimension moving 
through spacetime, and think strings are the special cases. These 
$p$-dimensional objects are called $p$-branes. When $p$ goes to 
$0$, these will become the usual point particles in the spacetime.

Now, after declaring the basic objects in our theory, it's time to
give their dynamic. As we all know, all dynamic properties of 
a point particle are contained in a functional whose name is 
\textit{action}. For a massive free particle, its (relativistic) 
action is
\begin{equation}\label{free_particle_action}
	S_0[\gamma]=-\alpha \int_\gamma ds,
\end{equation}
where $\gamma$ is the worldline of this particle, $\alpha$ is
a universal parameter\footnote{
	It can be determined by taking its classical limit, and
	we will get that $\alpha$ is the mass of the particle.}
of this particle and $ds$
is the line element of the world line. Suppose 
the line parameter is $\tau$, the worldline coordinate
is $x^\mu$, then
\[
	ds=\sqrt{-\eta_{\mu\nu}\dot x^\mu \dot x^\nu}d\tau.
\]
The real
worldline of this particle is given by the critical point
(or the saddle point in the most cases) of action. This 
is called the \textit{principle of action}.

Usually, the form of the action is \textit{a priori}. If we
allow another more fundamental principles, it can be gaussed
or even determined. For example, the \textit{law of inertia}
tells us that the worldline of a free particle in spacetime
will be a line. Thus, the most natrual candidate of action 
of a free particle is the length of its worldline which is 
shown in the equation \eqref{free_particle_action} since
for all curves connecting two fixed points in Minkowski space,
line is the longest one. Besides, the action 
\eqref{free_particle_action} does not introduce addtional
symmetries (and then redundances) which are not contained in 
this theory. However, it's more convenient to take action 
and the principle of action as our starting point although
there may be more fundamental principles beyond the cognition
of all human beings living on the nowadays Earth.

The generalization to the $p$-brand action is direct:
\[
	S_p=-T_p\int d\mu_p,
\]
here $T_p$ is called the $p$-brane tension and $d\mu_p$ is 
the $(p+1)$-dimensional volume element given by
\[
	d\mu_p=\sqrt{-\det G_{\alpha\beta}}\,d^{p+1}\sigma,
\]
where the induced metric is given by
\[
	G_{\alpha\beta}=\eta_{\mu\nu}(x)\partial_\alpha x^\mu 
	\partial_\beta x^\nu,\quad \alpha,\beta=0,\dots,p.
\]
For string ($1$-brand), the string action, 
called the \textit{Nambu-Goto action}, takes
the form
\[
	S_{\text{NG}}=-T\int d\sigma d\tau 
	\sqrt{(\dot x\cdot x')^2-\dot x^2 (x')^2},
\]
where 
\[
	\dot x^\mu=\partial_\tau x^\mu,\quad (x')^\mu
	=\partial_\sigma x^\mu.
\]

However, it's difficult to quantize the Nambu-Goto action 
because of the appearance of the square root. The same situation
has happened in the case of point particle. In that case, we use the 
following action
\[
	\tilde S_0=\frac{1}{2} \int d\tau\, (e^{-1}\dot x^2-m^2 e)
\]
to replace the old action, where $e(\tau)$ is an auxiliary field.
The equation of motion of auxiliary field is 
\[
	\frac{\delta \tilde S_0}{\delta e(\tau)}=
	-\frac{1}{2}(e^{-2}\dot x^2+m^2)=0,
\]
and we can solve the auxiliary field
\[
	e=m^{-1}\sqrt{-\dot x^2},
\]
then
\[
	\tilde S_0=\frac{1}{2} \int d\tau\, (e^{-1}\dot x^2-m^2 e)
	=\frac{m}{2}\int d\tau\, (-\sqrt{-\dot x^2}-\sqrt{-\dot x^2})
	=-m\int d\tau\, \sqrt{-\dot x^2}=S_0.
\]
Incidentally, this action even works well for massless particles.

Similarily, for bosonic string, we can introduce a auxiliary metric 
$g_{\alpha\beta}$ on the worldsheet and rewrite the action as
\[
	S_\sigma=-\frac{T}{2} \int d^2\sigma \sqrt{-g}
	g^{\alpha\beta}\partial_\alpha x\cdot \partial_\beta x.
\]
It's direct to check the equivalence of these two action
at the level of equations of motion. The equation of motion of
$g_{\alpha\beta}$ gives us that
\begin{equation}\label{amme}
	\partial_\alpha x\cdot \partial_\beta x
	=\frac{1}{2}g_{\alpha\beta}g^{\gamma\delta}
	\partial_\gamma x\cdot \partial_\delta x.
\end{equation}
This gives
\[
	\sqrt{-\det(\partial_\alpha x\cdot \partial_\beta x)}
	=\frac{1}{2}\sqrt{-g}g^{\alpha\beta}
	\partial_\alpha x\cdot \partial_\beta x.
\]
Note that $g_{\alpha\beta}=\partial_\alpha x\cdot \partial_\beta x$
is a solution of equation \eqref{amme}.

The new action introduced abobe is called 
the \textit{string sigma model action}. However,
at the level of symmetries, these two action is very different.
For the string sigma model action, it is invariant under the 
transformation 
\[
	g_{\alpha\beta}(\tau,\sigma)\longmapsto
	\exp(2\phi(\tau,\sigma))g_{\alpha\beta}(\tau,\sigma),
\]
since for any $k$-dimensional matrix $A$, 
$(\det A)^{1/k} A^{-1}$ is invariant under the rescaling 
$A\mapsto \lambda A$. This new (local) symmetry is called 
the \textit{Weyl symmetry}. Note that the equation 
\eqref{amme} can not totally fix $g_{\alpha\beta}$ because
if $g_{\alpha\beta}$ satisfies this equation, then after
any Weyl scaling, $\exp(2\phi(\tau,\sigma))g_{\alpha\beta}$
still satisfies it. So when considering the Lagrangian or
Hamiltonian, one should fix the coordinate and the Weyl
scaling.

The variation of the action with respect to the metric
defined the (Hilbert's) energy momentum tensor
\[
	T_{\alpha\beta}=-\frac{4\pi}{\sqrt{-g}}
	\frac{\delta S_\sigma}{\delta g^{\alpha\beta}}=
	-\frac{1}{\alpha'}\left(
	\partial_\alpha x\cdot \partial_\beta x-
	\frac 12 g_{\alpha\beta}g^{\gamma\delta}
	\partial_\gamma x\cdot \partial_\delta x
	\right),
\]
where $\alpha'=1/(2\pi T)$ or $T=1/(2\pi \alpha')$. 
It's well-known that $\nabla_\alpha T^{\alpha \beta}=0$.

The next lemma is a direct application of the Weyl invariance.
Its proof does not need the equation of motion $T_{\alpha\beta}=0$.

\begin{lem}
	$T_{\alpha\beta}$ is traceless (off shell), i.e. $g^{\alpha\beta}T_{\alpha\beta}=0$.
\end{lem}

\begin{proof}
	Since $S_\sigma[\lambda g]=S_\sigma[g]$, we should have that
	\[
		0=\left.\frac{\delta}{\delta \lambda}S_\sigma[\lambda g]
		\right |_{\lambda=1}
		=g^{\alpha\beta}\frac{\delta S_\sigma}{\delta g^{\alpha\beta}},
	\]
	and then $g^{\alpha\beta}T_{\alpha\beta}=0$.
\end{proof}


There're another terms which are independent of the choice
of the coordinate. For example, one may think a cosmological
term is possible,
\[
	S_1=\lambda \int d^2\sigma \sqrt{-g}.
\]
However, it doesn't have Weyl invariance, and then it 
makes that
\[
	g^{\alpha\beta}\frac{1}{\sqrt{-g}}
	\frac{\delta}{\delta g^{\alpha\beta}} (S_\sigma+S_1)=
	-\frac{\lambda}{2}g^{\alpha\beta}g_{\alpha\beta}=-\lambda,
\]
so the equation of motion 
$\delta (S_\sigma+S_1)/\delta g^{\alpha\beta}=0$ will never 
be true. Anther possible term is 
\[
	S_2=\lambda \int d^\sigma \sqrt{-g} R(g),
\]
where $R$ is the scalar curvature of $h$. 
If our worldsheet is boundless, this term will be Weyl 
invariant, since for $g\mapsto \exp(2\phi)g$, $\sqrt{-g}R$ 
transforms like
\[
\sqrt{-g}R\longmapsto \sqrt{-g}(R-2\nabla^2 \phi),
\]
where $\nabla^2=g^{\alpha\beta}\nabla_\alpha\nabla_\beta$
is the Laplace operator. Using the formula
\[
	\frac{1}{\sqrt{-g}}\partial_a (\sqrt{-g}v^a)=\nabla_a v^a,
\]
it's clear that the addtional term $\sqrt{-g}\nabla^2 \phi$
is a total derivative. For two-dimensional manifold, the
scalar curvature $R$ equal to twice the sectional curvature 
$K$, so the famous formula thanks to Gauss and Bonnet 
tells us that
\[
	\frac{1}{4\pi}\int_M d^2\sigma \sqrt{-g} R
	=\frac{1}{2\pi}\int_M d^2\sigma \sqrt{-g} K=
	\chi (M)
\]
is the Euler characteristic of $M$.

If our worldsheet is not boundless, Gauss-Bonnet formula
tells us that 
\[
	\frac{1}{2\pi}\int_M d^2\sigma \sqrt{-g} K+
	\frac{1}{2\pi}\int_{\partial M}ds \,k=\chi(M),
\]
where $ds$ is the line element along the boundary of $M$
and $k$ is the geodesic curvature of $\partial M$. Now,
it's natural to guess the following proposition.

\begin{pro}
\[
\frac{1}{4\pi} \int_M d^2\sigma \sqrt{-g} R+
\frac{1}{2\pi}\int_{\partial M}ds\, k.
\]
is Weyl invariant.
\end{pro}

\begin{proof}
Let curve $c$ is a connected component of $\partial M$.
We will use the length of curve to parametrize it such
that $\dot c^2=-1$. Choose unit vectors $n(x)\in T_{x}M$ 
such that $n\cdot \dot c=0$ for all $x\in c$, then the geodesic 
curvature is defined by $k=(\nabla_{\dot c}\dot c)\cdot n$.

Since under the transformation 
$g_{\alpha\beta}\mapsto g'_{\alpha\beta}=\exp(2\phi)g_{\alpha\beta}$, 
\[
	\dot c \mapsto \dot c'=\exp(-\phi)\dot c,\quad 
	n \mapsto n'=\exp(-\phi)n
\]
and
\[
	\Gamma^\alpha_{\beta\gamma}
	\longmapsto (\Gamma')^\alpha_{\beta\gamma}=
	\Gamma^\alpha_{\beta\gamma}+
	(\delta^\alpha_\beta \partial_\gamma \phi+
	\delta^\alpha_\gamma \partial_\beta \phi-
	g_{\beta\gamma}g^{\alpha\delta}\partial_\delta \phi,
	)
\]
one should have that
\begin{align*}
	k'&=g'_{\alpha\beta}(\nabla'_{\dot c'}\dot c')^\alpha
	(n')^\beta\\
	&=g_{\alpha\beta}(\nabla'_{\dot c}\dot c')^\alpha n^\beta\\
	&=g_{\alpha\beta}\dot c^\gamma 
	(\nabla'_{\gamma}(e^{-\phi}\dot c))^\alpha n^\beta\\
	&=e^{-\phi}g_{\alpha\beta}\dot c^\gamma
	(\nabla'_{\gamma}\dot c)^\alpha n^\beta\\
	&=e^{-\phi}g_{\alpha\beta}\dot c^\gamma
(\partial_\gamma\dot c^\alpha
+(\Gamma')_{\delta\gamma}^\alpha \dot c^\delta)n^\beta\\
&=e^{-\phi}k+e^{-\phi}g_{\alpha\beta}\dot c^\gamma
(\delta^\alpha_\delta \partial_\gamma \phi+
	\delta^\alpha_\gamma \partial_\delta \phi-
	g_{\delta\gamma}g^{\alpha\rho}\partial_\rho \phi)
	\dot c^\delta n^\beta\\
&=e^{-\phi}(k+n^\alpha \partial_\alpha \phi)
\end{align*}
and
\[
	ds\longmapsto ds'=\exp(\phi)ds,
\]
so
\[
	\frac{1}{2\pi}\int_{\partial M}ds\, k\longmapsto
	\frac{1}{2\pi}\int_{\partial M}ds \, k+
	\frac{1}{2\pi}\int_{\partial M}ds \, n^\alpha 
	\partial_\alpha\phi.
\]

Finally, we have calculated that
\begin{align*}
	-\frac{1}{2\pi}\int_Md^2\sigma \sqrt{-g}
	\nabla_\alpha(g^{\alpha\beta}\nabla_\beta \phi)
	&=-\frac{1}{2\pi}\int_Md^2\sigma
	\partial_\alpha(\sqrt{-g} g^{\alpha\beta}\nabla_\beta \phi)\\
	&=-\frac{1}{2\pi}\int_{\partial M}
	(\iota({\partial_\alpha})d^2\sigma)
\sqrt{-g} g^{\alpha\beta}\partial_\beta \phi\\
&=-\frac{1}{2\pi}\int_{\partial M}
	d\sigma^\gamma \sqrt{-g} g^{\alpha\beta}
	\epsilon_{\alpha\gamma}\partial_\beta \phi\\
&=-\frac{1}{2\pi}\int_{\partial M}
ds \,\dot c^\gamma \sqrt{-g} g^{\alpha\beta}
\epsilon_{\alpha\gamma}\partial_\beta \phi,
\end{align*}
where $\epsilon$ is the Levi-Civita symbol with 
$\epsilon_{01}=-\epsilon_{10}=1$ and $\epsilon_{00}=\epsilon_{11}=0$. 
Let $a^\beta=\dot c^\gamma \sqrt{-g} g^{\alpha\beta}
\epsilon_{\alpha\gamma}$, then 
\[
	g_{\delta\beta}\dot c^\delta a^\beta=\sqrt{-g}
	\epsilon_{\alpha\gamma}\dot c^\gamma\dot c^\alpha=0,
\]
so $a$ and $c$ are orthogonal. What's more,
\begin{align*}
	a^2=g_{\beta\delta}a^\beta a^\delta &=
	(-g)g_{\beta\delta}\dot c^\gamma \dot c^\xi  
	g^{\alpha\beta}\epsilon_{\alpha\gamma}
	g^{\zeta\delta}\epsilon_{\zeta\xi}\\
	&=(-g)\dot c^\gamma \dot c^\xi  
	g^{\alpha\beta}\epsilon_{\alpha\gamma}
	\epsilon_{\beta\xi}\\
	&=(-g)\dot c^\gamma \dot c^\xi
	(g^{-1}g_{\gamma\xi})\\
	&=-\dot c^\gamma \dot c^\xi g_{\gamma\xi}\\
	&=1,
\end{align*}
so $a$ is just the normal vecter $n$ defined above. Therefore,
\[
	\frac{1}{4\pi} \int_M d^2\sigma \sqrt{-g} R
	\longmapsto 
	\frac{1}{4\pi} \int_M d^2\sigma \sqrt{-g} R
	-\frac{1}{2\pi}\int_{\partial M}ds \, n^\alpha
\partial_\alpha\phi
\]
and 
\[
	\frac{1}{4\pi} \int_M d^2\sigma \sqrt{-g} R+
	\frac{1}{2\pi}\int_{\partial M}ds\, k.
\]
is Weyl invariant.
\end{proof}

According to this proposition, the most
general action for bosonic string can be written as
\begin{align*}
	S&=S_\sigma+\lambda \chi\\
&=-\frac{T}{2} \int_M d^2\sigma \sqrt{-g}
g^{\alpha\beta}\partial_\alpha x\cdot \partial_\beta x
-\frac{\lambda}{4\pi} \int_M d^2\sigma \sqrt{-g} R(g)-
\frac{\lambda}{2\pi}\int_{\partial M}ds\, k,
\end{align*}
where $\chi$ is a pure topological term which can be 
omitted when considering local properties.

\section{Quantization}

There are two `standard' ways to quantize a theory where
the action is given --- \textit{canonical quantization}
and \textit{path integral quantization}. 
Canonical quantization is useful to investigate the 
specturm, but symmetries of the theory are not obvious
after quantization, and path integral
quantization is the opposite. Here we will focus on the
specturm, so let's use the canonical quantization at first.

Open Strings and closed strings have very different specturm,
so we should disscuss them separately.

\subsection{Open String}

As mentioned in the last section, if one want to write 
down the Lagrangian or Hamiltonian of a bosonic string,
he/she must fix the coordinate and the Weyl scaling. 
For simplicity, we want to choose parameters such that
\[
	g=-1,\quad X^+=\sigma^0,\quad 
	\partial_0(\sqrt{-g}g_{00})=0.
\]
The first two conditions are trival. For the third,
it's helpful to note that $g_{00}\sqrt{-g}$
is invariant under reparameterizations of $\sigma^0$
where $\sigma$ is left fixed.

% Let $\sigma^2=i\sigma^0$, this is a standard trick called
% \textit{Wick rotation}. Then the path integral can be written
% as 
% \[
% 	\int \frac{\mathcal D x \mathcal D g}
% 	{\operatorname{Diff}\times \operatorname{Weyl}} \exp(-S),
% \]
% where the Euclidean action is b$S=S_x+\lambda \chi$ with
% \[
% 	S_x=\frac{T}{2} \int_M d^2\sigma \sqrt{g}
% g^{\alpha\beta}\partial_\alpha x\cdot \partial_\beta x,\quad
% \chi=\frac{1}{4\pi} \int_M d^2\sigma \sqrt{g} R+
% 	\frac{1}{2\pi}\int_{\partial M}ds\, k.
% \]
% Here $g$ is a Riemannian metric on $M$.