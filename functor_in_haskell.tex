\documentclass[10pt]{article}
\usepackage[zh]{./noteheader}
% \usepackage{amssymb, amsfonts, amsmath, amsthm, bm, mathrsfs}
% \usepackage[a4paper, top=10mm, text={144mm, 208mm}, includehead, includefoot, hmarginratio=1:1, heightrounded]{geometry}
\usepackage{listings,xypic}
\usepackage{tikz,ctex}

\lstset{
    showstringspaces=false,
    columns=flexible,
    basicstyle={\ttfamily},
    numbers=none
}

\title{Haskell中的函子}
\author{}
\begin{document}

范畴和函子是数学家幼儿园就知道的东西,但是Haskell的教科书普遍说得无比繁琐,这里总结一下函子相关的概念。同时,遵循任何范畴论使用者的习惯,范畴、函子会采用一些集合论里的符号,比如$\in$, $\to$等,它们的含义都是清楚的。

Haskell里最大的范畴$\mathsf{Hask}$是所有类型(type)和其他一些东西的范畴,其中的object可以是对象,也可以比如是小一些的范畴,类型间的态射就构成通常意义下的(计算机)函数。一般来说,一个函数的声明形如
\begin{lstlisting}
    f :: a -> b,
\end{lstlisting}
其中{\tt a}和{\tt b}都是对象。此外,范畴$\mathsf{Hask}$是有限完备的(?),故我们可以通过有限乘积定义多参数函数,它们的声明一般形如
\begin{lstlisting}
    f :: a1 -> ... -> an -> b,
\end{lstlisting}
这里{\tt f}是从$\tt a_1\times \cdots \times a_n$到{\tt b}的morphism. 同时,根据有限积的泛性质,我们可以随意加括号,比如此时{\tt f}可以看作
\begin{center}
    \tt f :: a1 -> (a2 -> ...~-> an -> b)\quad {\rm or} \quad  f :: (a1 -> a2 -> ...~-> an ) -> b
\end{center}
前者非常常用,因为Haskell是从左开始作用的。特别地,如果$\tt f$是两个集合间的函数,则任取$a\in {\tt a_1}$,我们定义${\tt g}= {\tt f}\, a$就得到了一个新的函数,它的声明写作
\begin{lstlisting}
    g :: a2 -> ... -> an -> b,
\end{lstlisting}

现在我们来到函子,函子是这样一个类
\begin{lstlisting}
    class Functor f where
        fmap :: (a -> b) -> f a -> f b
\end{lstlisting}
对这样的类里的每个实例,我们都要定义一个叫{\tt fmap}的东西,它最正则的看法应该是{\tt a -> b}到{\tt f a -> f b}的一个函数,但也不妨碍我们以其他加括号的方式来看它,比如说一个二元函数或一个三元函数。这两点就对应着数值的函子定义中的两点,首先是要讲对象{\tt a}变成{\tt f a},然后要将所有可能的{\tt a}到{\tt b}的morphism变成{\tt f a}到{\tt f b}的morphism. 

这里我们举一个例子。现在假设有一个范畴$\mathsf C$,他是集合范畴的子范畴,我们考虑范畴$\mathsf [C]$,后者中的object形如{\tt [a]},其中{\tt a}是$\mathsf C$中的对象,而{\tt [a]}是所有可能的定长列表(长度为$n$)构成的集合,其中的列表元都属于{\tt a}. 则我们可以构造一个典范的函子{$\tt List : a \mapsto \text{\tt [a]}$}如下:对任取的函数{\tt f~::~a -> b},将其$\tt f$挨个作用到{\tt [a]}. 故可见,此时的{\tt fmap}就是熟知的{\tt map},这就是{\tt fmap}得名的原因之一。

有了函子,自然变换就是非常自然的概念,如果范畴很小,则它们也被叫做函子间态射,是函子范畴的态射,所以是一个“更高阶”的存在。若$f$和$g$是两个给定范畴间的函子,则给出一个自然变换$\varphi$等价于给出所有的态射$\varphi(a)$,并且需要对任意的态射$\eta:a\to b$满足如下交换图
\[
\begin{xy}
	\xymatrix{
		f(a)\ar[rr]^{\varphi(a)} \ar[d]_{f(\eta)}&&g(a) \ar[d]^{g(\eta)}\\
		f(b)\ar[rr]^{\varphi(b)}&&g(b)
	}
\end{xy}
\]
我们先将从函子{\tt f}到函子{\tt g}的一个自然变换{\tt nt}的声明写作
\begin{lstlisting}
    nt :: (Functor f, Functor g) => f a -> g a,
\end{lstlisting}
于是,上图的交换条件就写成 $\tt (fmap~\eta).nt = nt.(fmap~\eta)$. 

下面我们考虑一个近似的概念,名为Lens,中文名叫透镜组。比起计算机学家的Lens,我们定义地再广一些,任取两个函子{\tt f}, {\tt g},考虑如下声明
\begin{lstlisting}
    lens :: (Functor f, Functor g) => (a -> b) -> (f a -> g a) -> f b -> g b,
\end{lstlisting}
这里,{\tt lens}吃掉一个morphism {\tt$\eta$~::~a -> b},得到一个新的函数{\tt (f a -> g a) -> f b -> g b},通常这里我们一般会要求复合得到的{\tt f a -> f b}与{\tt g a -> g b}在相应的{\tt fmap}下都等于{\tt fmap $\eta$}. 于是,这样一个Lens用图来表示即,
\[ 
\begin{xy}
	\xymatrix{
		f(a)\ar[rr]^{\varphi(a)} \ar[d]_{f(\eta)}&&g(a) \ar[d]^{g(\eta)}\ar@{-->}[lld]_\tau \\
		f(b)\ar[rr]^{\varphi(b)}&&g(b)
	}
\end{xy}
\]
其中上下两个三角都是交换的,进而$\varphi$就是一个自然变换。于是,对给定的两个函子$f$, $g: \mathsf{C}\to \mathsf{D}$,一个Lens就是一个$f\to g$的自然变换以及一个相应的分解(图中的虚线)。特别地,如果$\eta$就是$\operatorname{id}_a$,则$\tau$就是$\varphi(a)$的逆,于是,我们只可能得到$f\cong g$,即两个函子同构。

特别地,我们回到计算机学家的Lens,考虑函子$\tt f:\mathsf{Hask}\to \mathsf{Hask}$以及恒等函子,我们立刻有Lens的声明
\begin{lstlisting}
    lens :: (Functor f) => (a -> f a) -> b -> f b,
\end{lstlisting}
此时我们去掉了前面的多余的{\tt (a -> b)}. 上面的论断告诉我们,此时函子{\tt f}是一个范畴等价,即一个完全忠实函子,即{\tt fmap}即是单射也是满射,即所有 {\tt a -> b} 都可以一一对应到相应的 {\tt f a -> f b}. 当然,我们可以继续推广,破坏上面的条件,使得{\tt f}不需要是一个完全忠实函子,这可以通过拆掉上面交换图中的两条竖线完成。

在应用中,我们尤其会考虑



\end{document}
