% !TeX program = XeLaTeX
\documentclass[9pt]{extarticle}
\usepackage[article,zh]{noteheader}
% \usepackage{amssymb, amsfonts, amsmath, amsthm, bm, mathrsfs, tikz}
% \usepackage[b5paper, top=10mm, text={144mm, 208mm}, includehead, includefoot, hmarginratio=1:1, heightrounded]{geometry}
\usepackage{ctex, indentfirst}
\usepackage{young, youngtab} % for Young tableaux
\usepackage[section]{egastyle}
% \usepackage[compat=1.1.0]{tikz-feynman}

\pagestyle{plain}

\title{Lie Algebra}
\author{buwailee}
% \date{}

\definecolor{shadecolor}{rgb}{0.92,0.92,0.92}

\newcommand{\no}[1]{{$(#1)$}}
% \renewcommand{\not}[1]{#1\!\!\!/}
\newcommand{\rr}{\mathbb{R}}
\newcommand{\zz}{\mathbb{Z}}
\newcommand{\aaa}{\mathfrak{a}}
\newcommand{\pp}{\mathfrak{p}}
\newcommand{\mm}{\mathfrak{m}}
\newcommand{\dd}{\mathrm{d}}
\newcommand{\oo}{\mathcal{O}}
\newcommand{\calf}{\mathcal{F}}
\newcommand{\calg}{\mathcal{G}}
\newcommand{\bbp}{\mathbb{P}}
\newcommand{\bba}{\mathbb{A}}
\newcommand{\osub}{\underset{\mathrm{open}}{\subset}}
\newcommand{\csub}{\underset{\mathrm{closed}}{\subset}}

\DeclareMathOperator{\im}{Im}
\DeclareMathOperator{\Hom}{Hom}
\DeclareMathOperator{\id}{id}
\DeclareMathOperator{\rank}{rank}
\DeclareMathOperator{\tr}{tr}
\DeclareMathOperator{\supp}{supp}
\DeclareMathOperator{\coker}{coker}
\DeclareMathOperator{\codim}{codim}
\DeclareMathOperator{\height}{height}
\DeclareMathOperator{\sign}{sign}

\DeclareMathOperator{\Gal}{Gal}
\DeclareMathOperator{\ann}{ann}
\DeclareMathOperator{\Ann}{Ann}
\DeclareMathOperator{\ev}{ev}
	\newcommand{\cc}{\mathbb{C}}
	\newcommand{\lag}{{\mathfrak{g}}}
	\DeclareMathOperator{\ad}{ad}
	\DeclareMathOperator{\Int}{int}
	\DeclareMathOperator{\lie}{Lie}

\begin{document}
\maketitle
\section{伴随表示}

Lie代数本身就是一个矢量空间,我们这里感兴趣的是表示是一种矢量空间为$\lag$的表示。这需要从伴随作用开始。先说几个记号,设$l_g:h\mapsto gh$以及$r_g:h\mapsto hg$分别是左平移与右平移,他们都是群的自同构。

\para 设$\lag$是一个Lie代数,那么不难看出$\lag$的自同构群
\[\mathrm{Aut}_{\mathrm{Lie}}(\lag)=\{T\in \mathrm{GL}(\lag)\,:\,T[u,v]=[Tu,Tv],\,\forall u,\,v\in\lag\}\]
构成一个Lie群,他的Lie代数是
\[\mathfrak{gl}_{\mathrm{Lie}}(\lag)=\{T\in \mathfrak{gl}(\lag)\,:\,T[u,v]=[Tu,v]+[u,Tv],\,\forall u,\,v\in\lag\}.\]
由于$[X,*]:Y\mapsto [X,Y]\in \lag$,且根据Jacobi恒等式,我们可以得知$[X,*]\in \mathfrak{gl}_{\mathrm{Lie}}(\lag)$.

\para 设$G$是一个Lie群,他的Lie代数是$\lag$。Lie代数的伴随来自于Lie群的伴随$\mathbf{Ad}(g):h\mapsto ghg^{-1}$或者$\mathbf{Ad}(g)=r_{g^{-1}}\circ l_g=l_g\circ r_{g^{-1}}$在单位元上的导数$\mathrm{Ad}_g=\mathbf{Ad}(g)_{*e}:T_eG\to T_eG$,但注意到Lie代数$\lag$就是Lie群在单位元的切空间$T_eG$,所以$\mathrm{Ad}_g:\lag\to \lag$。因为$\mathbf{Ad}(g)$是Lie群的一个自同构,所以$\mathrm{Ad}_g:\lag\to\lag$是线性空间的同构,即$\mathrm{Ad}_g\in \mathrm{GL}(\lag)$.

利用指数函数,把$\mathrm{Ad}_g$和$\mathbf{Ad}(g)$之间的微分关系联系起来即,$\mathbf{Ad}(g)\exp(X)=\exp(\mathrm{Ad}_gX)$成立。

\para 一般而言,我们也可以通过左不变矢量场来描述Lie代数,这时候最好把$\mathrm{Ad}_g$理解成$(l_g)_*\circ (r_{g^{-1}})_*$,此时我们有断言:如果$X$是$G$上的左不变矢量场,那么$\mathrm{Ad}_gX$对于任意$g\in G$也是左不变矢量场。

注意到左作用和右作用是可交换的,因此他们的导数也是可以交换的,那么
\[
	(l_h)_*(\mathrm{Ad}_gX)=(l_h)_*\circ (l_g)_*\circ (r_{g^{-1}})_*(X)=(r_{g^{-1}})_*(X)=(r_{g^{-1}})_*\circ (l_g)_*(X)=\mathrm{Ad}_gX.
\]

\para 实际上$\mathrm{Ad}_g$还是Lie代数$\lag$的一个自同构。线性空间同构已经是清楚的了,下面只要证明他是Lie代数同态即可,为此只要检验$\mathrm{Ad}_g([X,Y])=[\mathrm{Ad}_gX,\mathrm{Ad}_gY]$,其中$X$和$Y$都是左不变矢量场。

由于$\mathrm{Ad}_g=(l_g)_*\circ (r_{g^{-1}})_*$,且$l_g$与$r_{g^{-1}}$作为Lie群的自同构,有$(l_g)_*[X,Y]=[(l_g)_*X,(l_g)_*Y]$和$r_{g^{-1}}[X,Y]=[r_{g^{-1}}X,r_{g^{-1}}Y]$成立,于是$\mathrm{Ad}_g([X,Y])=[\mathrm{Ad}_gX,\mathrm{Ad}_gY]$自然得证。

\para 到目前为止,我们看到$\mathrm{Ad}_g\in \mathrm{Aut}_{\mathrm{Lie}}(\lag)$,所以我们可以构造映射$\mathrm{Ad}:g\mapsto \mathrm{Ad}_g$. 可以检验$\mathrm{Ad}:G\to \mathrm{Aut}(\lag)$是一个Lie群同态,称为Lie群的伴随表示。

同态来自于$\mathrm{Ad}_g\circ \mathrm{Ad}_h=(l_g)_*\circ (r_{g^{-1}})_*\circ (l_h)_*\circ (r_{h^{-1}})_*=(l_g)_*\circ (l_h)_*\circ (r_{g^{-1}})_*\circ (r_{h^{-1}})_*=(l_{gh})_*\circ (r_{(hg)^{-1}})_*=\mathrm{Ad}_{gh}$.

% 第三个设$v\in T_hG$,因此$(r_g)_*v\in T_{hg}G$,于是
% \[
% 	(r_g)^*\omega_G(v)=\omega_G((r_g)_*v)=(l_{{hg}^{-1}})_*(r_g)_*
% 	=(l_{{g}^{-1}})_*(r_g)_*(l_{{h}^{-1}})_*v=\mathrm{Ad}(g^{-1})\omega_G.
% \]

\para 令$G$的Lie代数为$\lag$,则$\mathrm{Ad}:G\to \mathrm{Aut}_{\mathrm{Lie}}(\lag)$在$e$的导数$\ad=\mathrm{Ad}_{*e}:\lag\to \mathfrak{gl}_{\mathrm{Lie}}(\lag)$满足$\ad(X)Y=[X,Y]$.

因为$\mathrm{Ad}$是$G$上矢量值的函数,证明就是很直接地要去计算
\[
	\ad(X)Y=\left.\frac{\dd}{\dd t}\right|_{t=0}\mathrm{Ad}_{\exp(tX)}Y,
\]
找一个$G$上的光滑函数$f$,我们计算其单位元处的导数,注意到
\[
	Yf=\left.\frac{\dd}{\dd u}\right|_{u=0}f(\exp(uY)),
\]
以及$(\mathrm{Ad}_{g}Y)f=Y(f\circ l_g\circ r_{g^{-1}})$,所以
\[
	\ad(X)Yf=\left.\frac{\dd}{\dd t}\right|_{t=0}(\mathrm{Ad}_{\exp(tX)}Y)f=\left.\frac{\dd}{\dd t}\frac{\dd}{\dd u}\right|_{u=t=0}f\bigl(\exp(tX)\exp(uY)\exp(-tX)\bigr),
\]
注意到对$t$求导的时候,可以利用多元实函数$F(t_1,t_2)$的求导等式
\[
	\left.\frac{\dd}{\dd t}\right|_{t=0}F(t,t)=\frac{\partial F}{\partial t_1}(0,0)+\frac{\partial F}{\partial t_2}(0,0),
\]
所以我们得到了
\[
\begin{split}
	&\left.\frac{\dd}{\dd t}\frac{\dd}{\dd u}\right|_{u=t=0}f\bigl(\exp(tX)\exp(uY)\exp(-tX)\bigr)\\
	=&\left.\frac{\dd}{\dd t_1}\frac{\dd}{\dd u}\right|_{u=t_1=0}f\bigl(\exp(t_1X)\exp(uY)\bigr)-\left.\frac{\dd}{\dd t_2}\frac{\dd}{\dd u}\right|_{u=t_2=0}f\bigl(\exp(uY)\exp(t_2X)\bigr)\\
	=&(XY-YX)f(e),
\end{split}
\]
所以$\ad(X)Y=[X,Y]$.

\para 对于一般线性群,前面已经计算过了$(l_g)_*=l_g$,那么同样$(r_g)_*=r_g$,所以
\[
	\mathrm{Ad}_g=(l_g)_*(r_{g^{-1}})_*=l_gr_{g^{-1}}.
\]
那么
\[
	\mathrm{Ad}_g(v)=(l_g)_*(r_{g^{-1}})_*v=l_gr_{g^{-1}}v=gvg^{-1}.
\]
我们现在求他的Lie代数,考虑$u,v\in \lag$,我们令$u(t)$是一个以$u$为初速度的单参子群,那么我们有
\[
	\frac{\dd}{\dd t}\bigl(\mathrm{Ad}_{u(t)}(v)\bigr)=u'(t)vu^{-1}(t)+u(t)v(u^{-1}(t))'=u'(t)vu^{-1}(t)-u(t)vu^{-1}(t)u'(t)u^{-1}(t).
\]
然后令$t=0$,那么$u(0)=u^{-1}(0)=I$,而$u'(0)=u$,那么就得到了单位元处的切矢量,也就是Lie代数$\ad(u)v=uv-vu=[u,v]$.

类似的手段譬如对$T(t)[u,v]=[T(t)u,T(t)v]$求个导,然后在$t=0$处的值为
\[
	T'(0)[u,v]=[T'(0)u,T(0)v]+[T(0)u,T'(0)v],
\]
注意到$T(0)$是恒等变换,而$T'(0)$就是我们需要的Lie代数$B$,他需要满足的关系就是
\[
	B[u,v]=[Bu,v]+[u,Bv],
\]
其显然是Lie代数上面的一个导子。

\section{半单Lie代数}

\para 设$G$是一个Lie群,而$\lag$是他的Lie代数。如果Lie代数上可以配备一个双不变的内积,则$G$是一个Riemann流形,利用Levi-Civita联络可以定义出一个合适的曲率算符$R_{XV}Y=\ad(Y)\circ \ad(X)(V)/4$,接着我们定义一个正比于Ricci曲率的双线性映射$(X,Y)_K=\tr(\ad(X)\circ \ad(Y))$,他被称为Killing形式。Killing形式显然是对称的,因为迹成立$\tr(AB)=\tr(BA)$.

\para 设$\mu$是$\lag$上的一个自同构,则$(\mu X,\mu Y)_K=(X,Y)_K$.

因为$\mu$是$\lag$上的一个自同构,他满足$[\mu X,\mu Y]=\mu([X,Y])$,所以$\ad(\mu X)(\mu Y)=\mu\ad(X)Y$,或者写作$\ad(\mu X)(Y)=\mu\ad(X)\mu^{-1}Y$,因此$\ad(\mu X)=\mu\circ \ad(X)\circ \mu^{-1}$. 

现在考虑$(\mu X,\mu Y)_K$,由于
\begin{align*}
	(\mu X,\mu Y)_K&=\tr(\ad(\mu X)\circ \ad(\mu Y))\\
	&=\tr(\mu\circ \ad(X)\circ \ad(Y)\circ \mu^{-1})\\&=\tr(\ad(X)\circ \ad(Y)\circ \mu^{-1}\circ \mu)\\&=\tr(\ad(X)\circ \ad(Y))\\&=(X,Y)_K.
\end{align*}

考虑$\mu=\exp(t\ad(Z))$,我们有
\[
	(X,Y)_K=(\exp(t\ad(Z))X,\exp(t\ad(Z))Y)_K,
\]
在$t=0$处做微分即得到了$(\ad(Z)X,Y)_K+(X,\ad(Z)Y)_K=0$,或者$([Z,X],Y)_K+(X,[Z,Y])_K=0$. 换句话说Killing形式是双不变的。

但是Killing形式却不一定是非退化的,如果$\lag$上的Killing形式是非退化的,则我们称呼这样的$\lag$是半单的。

\pro 一个Lie代数是半单,则他没有非平凡交换理想。特别地,如果$[h,\lag]=0$则$h=0$.

\proof 我们来证明逆否命题,如果$\lag$有一个非平凡交换理想$\mathfrak{a}$,使得$[\mathfrak{a},\lag]\subset \mathfrak{a}$,那么我们证明$(x,y)_K=0$对任意的$x\in \mathfrak{a}$和$y\in \lag$都成立,这自然就是告诉我们Killing形式是退化的了。

令$\sigma=\ad x\circ \ad y$,那么$\sigma:\lag\to \mathfrak{a}$以及$\sigma|_\mathfrak{a}:\mathfrak{a}\to 0$. 选择$\lag$这样的一组基$\{a_1,\cdots,a_k,a_{k+1},\cdots,a_{n}\}$,其中$\{a_i\}_{1\leq i \leq k}$是$\mathfrak{a}$的基。那么$\sigma(a_j)=\sum_i\sigma_{ij} a_i$,当$1\leq j\leq k$的时候,因为$\sigma(a_j)=0$,所以$\sigma_{jj}=0$. 当$k+1\leq j \leq n$的时候,因为$\sigma(a_j)=\sum_{i=1}^k\sigma_{ij} a_i\in\mathfrak{a}$,所以$\sigma_{jj}=0$,因此$(x,y)_K=\sum_{i=1}^n\sigma_{ii}=0$.\qed

\theo 对于有限维复半单Lie代数$\lag$,他是一个紧Lie群$H$的Lie代数的复化,即$\lag=\mathfrak{h}_{\cc}=\mathfrak{h}+i\mathfrak{h}$,其中$\mathfrak{h}$是$H$的Lie代数。

由于群$H$是紧的,所以$\mathfrak{h}$上面存在着内积$(\star,\star)$在表示$\mathrm{Ad}:H\to \mathrm{GL}(\mathfrak{h})$作用下不变,即
\[
	\bigl(\exp(t\ad(h))x,\exp(t\ad(h))y\bigr)=(x,y)
\]
对任意的$h\in\mathfrak{h}$和$t\in \rr$都成立,其中$x$, $y\in\mathfrak{h}$。

可以把这个内积唯一扩张为$\lag$内变成取复值的内积,让我们还是使用符号$(\star,\star)$来标记他,那么求导就有
\[
	(\ad(h)x,y)+(x,\ad(h)y)=0.
\]
所以表示$\ad$是反Hermit的,即
\[
	\ad(h)+\ad(h)^\dag=0.
\]
此时$i\ad(h)$就是Hermit的。根据有限维的谱定理,我们一定可以对角化$i\ad(h)$,也就是说可以对角化$\ad(h)$当$h\in \mathfrak{h}$,而且是对角元是纯虚的。

\lem 一个技术引理,设一个$n\times n$的方阵族$\{A^1,\cdots,A^k\}$满足$[A^i,A^j]=0$对$1\leq i$, $j\leq k$都成立,且每一个$A^i$都是可以对角化的。则$\{A^1,\cdots,A^k\}$可以同时对角化。

\para 复半单Lie代数$\lag$的极大交换子代数称为他的Cartan子代数,由于$\lag=\mathfrak{h}+i\mathfrak{h}$,所以如果$\mathfrak{l}$是$\mathfrak{h}$的极大交换子代数,则$\lag$的Cartan子代数为$\mathfrak{l}+i\mathfrak{l}$.

如果$h_1$, $h_2\in\mathfrak{l}$,利用Jacobi恒等式就可以检验$\ad(h_1)$和$\ad(h_2)$是可交换的,因此他们可以同时对角化。当他们是同时对角化的时候,线性组合$h=h_1+ih_2$对应的$\ad(h)$也是对角化的。所以我们证明了,如果$h$在$\lag$的Cartan子代数里面,那么$\ad(h)$是可以对角化的。

\para 我们可以分解有限维半单复Lie代数$\lag$为这样的两个子代数,其中一个是Cartan子代数$\mathfrak{h}$
\[
	\lag=\mathfrak{h}\oplus \lag'.
\]
注意到对于任意的$h\in \mathfrak{h}$都有$\ad(h):\lag'\to \lag'$,这是因为$\ad(h)h'=0$对$h'\in \mathfrak{h}$.

\para 令$\mathfrak{h}$的维度为$l$,设Cartan子代数$\mathfrak{l}$的基为$\{H^1,H^2,\cdots,H^l\}$,因为$\ad(H^i):\lag'\to \lag'$且$H^i$在$\lag$的Cartan子代数里面,所以$\ad(H^i)$是可以对角化的,同时又因为$[\ad(H^i),\ad((H^j)]=0$,故族$\{\ad(H^i)\,:\, 1\leq i\leq l\}$可以同时对角化。换而言之,我们可以找到$\lag'$的一组基$\{E^\alpha\}$使得如下本征方程
\[
	\ad(H^i)E^\alpha=[H^i,E^\alpha]=\alpha^iE^\alpha
\]
对每一个$i$和$\alpha$都成立,其中$\alpha^i$就是$\ad(H^i):\lag'\to \lag'$的本征值,本征矢量$E^\alpha$被称为阶梯算符。经常我们把上面的式子记作$H^i|\alpha\rangle=\alpha^i|\alpha\rangle$.

% 而Cartan子代数作为交换子代数,我们有交换子$[H^j,H^\alpha]=0$成立。但Cartan子代数又是极大的交换子代数,所以对任意的$\alpha$,我们总可以找到一个$j$使得$\alpha_{\alpha}(H^j)\neq 0$,否则$[H^j,E^\alpha]=0$就可以推出$E^\alpha$在Cartan子代数$\mathfrak{h}$里面了。

固定一个$\alpha$,设$h=\mu_jH^j$是任意的Cartan子代数里面的元素,则
\[
	\ad(h)E^\alpha=[\mu_jH^j,E^\alpha]=\mu_j\alpha^jE^\alpha,
\]
我们定义Cartan子代数$\mathfrak{h}$上的线性函数$\alpha:\mathfrak{h}\to\cc$为$\alpha(h)=\mu_j\alpha^j$,则
\[
	\ad(h)E^\alpha=[h,E^\alpha]=\alpha(h)E^\alpha.
\]
这样定义的线性函数$\alpha$被称为$\lag$的根,所有根的集合记做$\Delta$. 由于$\Delta$中的元素都是$\mathfrak{h}\to \cc$的线性函数,所以他是$\mathfrak{h}$的对偶空间$\mathfrak{h}^*$的子集,即$\Delta\subset h^*$。同时,利用$\mathfrak{h}^*$上面的加法和数乘,我们也可以在$\Delta$上通过$(k_1\alpha+k_2\beta)(h)=k_1\alpha(h)+k_2\beta(h)$定义加法和数乘,但是却无法断言$k_1\alpha+k_2\beta\in \Delta$. 下面我们的任务是搞清楚$\Delta$的结构,主要是通过Killing形式出根之间的内积。

\para 对有限维半单复Lie代数$\lag$的一个根$\alpha$,我们将满足方程的
\[
[h,E^\alpha]=\alpha(h)E^\alpha\text{ or } h|\alpha\rangle=\alpha(h)|\alpha\rangle
\]
阶梯算符$E^\alpha$张成的线性空间称为对应于根$\alpha$的根子空间,记作$\lag_{\alpha}$。于是把有限维半单复Lie代数的分解更加细化为
\[
	\lag=\mathfrak{h}\oplus \bigoplus_{\alpha\in\Delta} \lag_\alpha.
\]
其中$\lag_\alpha$是诸根子空间。Cartan子代数的根可以认为是恒为0的,那么$\mathfrak{h}$也可以认为是一个根子空间$\lag_0$。

两个根子空间可以进行对易子运算,即$[\lag_\alpha,\lag_\beta]$,设$a\in\lag_\alpha$, $b\in\lag_\beta$,则
$[a,b]$也是在某个根子空间里面的,为了求他的根,我们用对任意的$h\in \mathfrak{h}$求$\ad(h)[a,b]$,因为$\ad(h)$是导子,所以
\[
	\begin{split}
		\ad(h)[a,b]&=[\ad(h)a,b]+[a,\ad(h)b]\\
		&=[\alpha(h)a,b]+[a,\beta(h)b]\\
		&=(\alpha(h)+\beta(h))[a,b]
	\end{split}
\]
因此,如果$\alpha+\beta\in\Delta$,则$\ad(h)[a,b]\in \lag_{\alpha+\beta}$,于是$[\lag_\alpha,\lag_\beta]\subset \lag_{\alpha+\beta}$,否则$[\lag_\alpha,\lag_\beta]=\{0\}$.

\pro $\Delta$张成了整个$\mathfrak{h}^*$.

\proof 实际上,任取$h\in \mathfrak{h}$,如果$\alpha(h)=0$对任意的$\alpha\in \Delta$成立的话,$[h,\lag_{\alpha}]=0$对任意的$\alpha\in \Delta$成立,于是$[h,\lag]=0$,利用半单性这也就是说$h=0$.\qed

\pro 设$E^\alpha\in\lag_\alpha$, $E^\beta\in\lag_\beta$,如果$\alpha(h)+\beta(h)\neq 0$对于某个$h\in \mathfrak{h}$成立(简单写作$\alpha+\beta\neq 0$),则$(E^\alpha,E^\beta)_K=0$. 如果$\alpha\in\Delta$,则$-\alpha\in\Delta$.

\proof 前半句,因为存在$h\in\mathfrak{h}$成立
\[
	0=([h,E^\alpha],E^\beta)_K+(E^\alpha,[h,E^\beta])_K=(\alpha(h)+\beta(h))(E^\alpha,E^\beta)_K,
\]
且$\alpha(h)+\beta(h)\neq 0$,所以自然得证。

至于后半句,假若$-\alpha\notin\Delta$,那么对于任何$\beta\in\Delta$都有$\alpha+\beta\neq 0$,那么就是说,对于任何的$a\in \lag$都有$(E^\alpha,a)_K=0$,由Killing形式的非退化可以知道此时$E^\alpha=0$,这不可能。\qed

% 这样,我们顺便也推出了,对于任意的$a_\alpha\in\lag_{\alpha}$,一定存在一个$a_{-\alpha}\in\lag_{-\alpha}$使得$(a_\alpha,a_{-\alpha})_K\neq 0$.因为Killing形式是双线性的,调整系数我们可以使得$(a_\alpha,a_{-\alpha})_K$成为任意的复常数。

\para 特别地,我们考虑$h\in \mathfrak{h}$且$E^\alpha\in \lag_\alpha$,因为$0+\alpha=\alpha\neq 0$,所以$(h,E^\alpha)_K=0$. 以此和Killing形式在$\mathfrak{g}$上非退化可以推知,Killing形式在$\mathfrak{h}$上也是非退化的。这是因为,如果$(h,h')_K=0$对所有$h'\in \mathfrak{h}$都成立,利用$(h,E^\alpha)_K=0$,则他也对所有$g\in \lag$成立$(h,g)_K=0$.

那么我们就有可能用非退化的Killing形式来表示根,即对根$\alpha$存在唯一的$h_\alpha\in\mathfrak{h}$使得\footnote{使用类似定义对偶空间的方式。}
\[
	(h_\alpha,h)_K=\alpha(h),
\]
因此$h_{\alpha+\beta}=h_\alpha+h_\beta$。因为Killing形式是对称的,所以$(h_\alpha,h_\beta)_K=(h_\beta,h_\alpha)_K$可以推知$\alpha(h_\beta)=\beta(h_\alpha)$.

\para 我们定义$\Delta$上的一个二元线性运算$\langle \star,\star \rangle$如下:
\[
	\langle \alpha,\beta \rangle=(h_\alpha,h_\beta)_K.
\]
那么$\alpha(h_\beta)=\beta(h_\alpha)=\langle \alpha,\beta \rangle=\langle \beta,\alpha \rangle$.那么$\langle -\alpha,\beta \rangle=-\alpha(h_\beta)=-\langle \alpha,\beta \rangle$以及
\[
	[h_\beta,E^\alpha]=\alpha(h_\beta)E^\alpha=\langle \beta,\alpha \rangle E^\alpha.
\]

因为$\ad(h)$是对角的,且对角元素为$\{\alpha(h)\}$,其中$\alpha\in\Delta$,那么直接根据Killing形式的定义,就有
\[
	(h,h')_K=\sum_{\gamma \in \Delta}(\dim \lag_\gamma)\gamma(h)\gamma(h'),
\]
特别地,我们有
\[
	\langle \alpha,\beta \rangle=(h_\alpha,h_\beta)_K=\sum_{\gamma \in \Delta}(\dim \lag_\gamma)\gamma(h_\alpha)\gamma(h_\beta)=\sum_{\gamma \in \Delta}(\dim \lag_\gamma)\langle \gamma,\alpha\rangle\langle \gamma,\beta\rangle.
\]

\pro 设$E^\alpha \in \lag_\alpha$和$E^{-\alpha} \in \lag_{-\alpha}$,则$[E^{\alpha},E^{-\alpha}]=(E^{\alpha},E^{-\alpha})_Kh_\alpha$.这说明了$[\lag_{\alpha},\lag_{-\alpha}]$是一维的,他由$h_\alpha$张成。

\proof 显然$[E^\alpha,E^{-\alpha}]\in \mathfrak{h}$,所以考虑
\[
	([E^\alpha,E^{-\alpha}],h)_K=-(E^{-\alpha},[E^\alpha,h])_K=(E^{-\alpha},\alpha(h)E^\alpha)_K=(E^{-\alpha},E^\alpha)_K(h_\alpha,h)_K,
\]
然后利用Killing形式的非退化性就得到了结论。\qed

\section{权和$\mathfrak{sl}(2,\cc)$子代数}

我们先来看看几个比较简单的Lie代数$\mathfrak{sl}(2,\cc)$, $\mathfrak{so}(3)$和$\mathfrak{su}(2)$,他们之间存在着紧密的联系。已知$\mathfrak{sl}(2,\cc)$, $\mathfrak{so}(3)$的基和相互的对易关系有
\[
[h,e]=2e,\quad[h,f]=-2f,\quad[e,f]=h,
\]
\[
	[\eta_1,\eta_2]=\eta_3,\quad [\eta_1,\eta_3]=-\eta_2,\quad [\eta_2,\eta_3]=\eta_1.
\]

现在我们来看$\mathfrak{su}(2)$的表现,他是所有满足$B+B^\dag=0$的复二阶零迹矩阵$B$的集合。我们选如下三个矩阵作为基:
\[
\mu_1=\frac{1}{2}\begin{pmatrix}
	0&i\\
	i&0\\
\end{pmatrix},\quad
\mu_2=\frac{1}{2}\begin{pmatrix}
	0&-1\\
	1&0\\
\end{pmatrix},\quad
\mu_3=\frac{1}{2}\begin{pmatrix}
	i&0\\
	0&-i\\
\end{pmatrix}.
\]
容易验证
\[
	[\mu_1,\mu_2]=\mu_3,\quad [\mu_1,\mu_3]=-\mu_2,\quad [\mu_2,\mu_3]=\mu_1.
\]
这和$\mathfrak{so}(3)$的对易关系一模一样,于是我们可以引入$\rr$-线性映射建立两者作为实Lie代数的同构,也就是$\mathfrak{so}(3)\cong \mathfrak{su}(2)$.

很容易证明$\mathfrak{sl}(2,\rr)$的复化即$\mathfrak{sl}(2,\cc)$,这就是说$\mathfrak{sl}(2,\rr)_\cc=\mathfrak{sl}(2,\cc)$. 为了分析$\mathfrak{sl}(2,\cc)$的结构,我们可以先看看
$\mathrm{SL}(2,\mathbb{C})$的结构。

任何一个复可逆$2\times 2$矩阵都可以唯一分解(极分解)为$
\lambda=ue^{h}$,其中$u$幺正而$h$是Hermite矩阵。现在假如$\det \lambda=1$,则
\[
\det(u)e^{\tr(h)}=1
\]
于是$\det(u)=1$而$\tr(h)=0$.前者的一般形式为
\[
u=
\begin{pmatrix}
a+ib&c+id\\
-c+id&a-ib
\end{pmatrix},
\]
且满足$a^2+b^2+c^2+d^2=1$,因此其拓扑上等价为3-球面$\mathbb{S}^3$.而前者的一般形式为
\[
h=\begin{pmatrix}
e&f-ig\\
f+ig&-e
\end{pmatrix},
\]
拓扑上等价于$\rr^4$,因此$\mathrm{SL}(2,\cc)$在拓扑上等价于$\rr^4\times \mathbb{S}^3$.当然拓扑上的结论在我们这里暂时没什么用。

这样来看$\mathrm{SL}(2,\cc)$的Lie代数$\mathfrak{sl}(2,\mathbb{C})$,在$\mathrm{SL}(2,\cc)$的极分解中,令$b=-ih$,则
\[
\tr(b)=0,\quad b^\dag+b=0,
\]
以及$u=e^a$有
\[
\tr(a)=0,\quad a^\dag+a=0,
\]
所以$a$, $b\in\mathfrak{su}(2)$,且$\lambda=e^{a+ib}$.

注意到任取一个实数$t$和$a\in\mathfrak{su}(2)$,还有$ta \in\mathfrak{su}(2)$,所以任意的一个$\lambda \in \mathrm{SL}(2,\cc)$都可以写成
\[
	\lambda=\exp(ta+itb),
\]
他在$t=0$的导数$a+ib$就构成$\mathfrak{sl}(2,\cc)$,那么任意的$c\in \mathfrak{sl}(2,\cc)$都可以写成$c=a+ib$,其中$a$, $b\in\mathfrak{su}(2)$,这就是说$\mathfrak{sl}(2,\cc)$是$\mathfrak{su}(2)$的复化。

单纯从Lie代数来看,我们在$\mathfrak{su}(2)_\cc$中引入$L_n=i\mu_n$,则
\[
	[L_1,L_2]=iL_3,\quad [L_1,L_3]=-iL_2,\quad [L_2,L_3]=iL_1.
\]
再引入$L_\pm=L_1\pm iL_2$,则
\[
	[L_+,L_-]=2L_3,\quad [L_3,L_+]=L_+,\quad [L_3,L_-]=-L_-.
\]
我们令$h'=2h,e'=2e,f'=2f$,则$\mathfrak{sl}(2,\cc)$的三个基的对易关系变成
\[
[e',f']=2h',\quad[h',e']=e',\quad[h',f']=-f'.
\]
可见一模一样。

这样,三个Lie代数之间的关系就清楚了
\[
	\mathfrak{su}(2)\cong\mathfrak{so}(3),\quad \mathfrak{su}(2)_\cc\cong \mathfrak{sl}(2,\mathbb{C}) \cong\mathfrak{sl}(2,\rr)_\cc.
\]

\para 我们考虑复半单Lie代数任意的不可约表示,因为$\{H^i\}$是可以同时对角化的(这里可以先假设还是非简并的),所以总可以找到一组基$|\lambda\rangle$使得$H^i|\lambda\rangle=\lambda^i|\lambda\rangle$,则矢量$(\lambda^1,\cdots,\lambda^r)$被称为权。权通过$\lambda(\mu_iH^i)=\mu_i\lambda^i$,则我们可以将权看做$\mathfrak{h}^*$中的一个线性函数。因此,在伴随表示里面,权就是根。同样可以类比定义
\[
	(h_\lambda,h)_K=\lambda(h),\quad h|\lambda\rangle=\lambda(h)|\lambda\rangle,
\]
以及
\[
	\langle \lambda,\mu\rangle=(h_\lambda,h_\mu)_K,
\]
这是正定非退化的一个内积。

利用$[H^i,E^\alpha]=\alpha^iE^\alpha$,我们有
\[
	H^iE^\alpha|\lambda\rangle=[H^i,E^\alpha]|\lambda\rangle+E^\alpha H^i|\lambda\rangle=(\lambda^i+\alpha^i)E^\alpha |\lambda\rangle.
\]
如果$E^\alpha |\lambda\rangle$不为零,那么他一定正比于$|\lambda+\alpha\rangle$,这就是为什么我们称呼$E^\alpha$为阶梯算符,不妨记$E^\alpha |\lambda\rangle\sim |\lambda+\alpha\rangle$表示他们至多差一个非零标量。按照上一节里面的知识,我们知道,如果$\alpha\in \Delta$,则$-\alpha\in\Delta$,所以$E^{-\alpha} |\lambda\rangle\sim |\lambda-\alpha\rangle$.

所以我们可以进行有限归纳了,因为我们考虑的总是有限维的表示,所以对任意的$\alpha\in \Delta$,存在正整数$p$和$q$使得
\[
	(E^{-\alpha})^p |\lambda\rangle\sim |\lambda-p\alpha\rangle =0,
\]
\[
	(E^{\alpha})^q |\lambda\rangle\sim |\lambda+q\alpha\rangle =0.
\]

\para 如果我们选$E_\alpha\in\lag_\alpha$, $F_\alpha\in\lag_{-\alpha}$满足$(E_\alpha,F_{\alpha})_K=2/\langle \alpha,\alpha \rangle$,再选
\[
	H_\alpha=\frac{1}{\langle \alpha,\alpha \rangle}h_\alpha
\]
那么计算对易关系
\[
	\begin{split}
	&[H_\alpha,E_\alpha]=\frac{1}{\langle \alpha,\alpha \rangle}[h_\alpha,E_\alpha]=\frac{1}{\langle \alpha,\alpha \rangle}\langle \alpha,\alpha \rangle E_\alpha=E_\alpha,\\
	&[H_\alpha,F_\alpha]=\frac{1}{\langle \alpha,\alpha \rangle}[h_\alpha,F_\alpha]=\frac{1}{\langle \alpha,\alpha \rangle}\langle \alpha,-\alpha \rangle F_\alpha=-F_\alpha,\\
	&[E_\alpha,F_\alpha]=\frac{2}{\langle \alpha,\alpha \rangle}h_\alpha=2H_\alpha.
	\end{split}
\]
总结一下就是
\[
	[H_\alpha,E_\alpha]=E_\alpha,\quad[H_\alpha,F_\alpha]=-F_\alpha,\quad[E_\alpha,F_\alpha]=2H_\alpha.
\]
这其实就是$\mathfrak{sl}(2,\cc)$的对易关系,所以呢,这三个基$E_\alpha$, $F_\alpha$, $H_\alpha$生成了一个同构于$\mathfrak{sl}(2,\cc)$的子代数,记作$\mathfrak{s}^\alpha$.

\para 现在来看$\mathfrak{sl}(2,\cc)$的有限维不可约表示,每一个Lie代数的元素$a$都变成了有限维矢量空间$V$上面的线性映射$\pi(a)$,复数域的代数完备性可以推知$\pi(H)$有一个特征值,即$H|v\rangle=\lambda |v\rangle$.

使用前面的讨论,不断使用$E_\alpha$就可以得到一族本征值和本征矢量,而且存在一个$N\geq 0$使得
\[
	E^N|v\rangle\neq 0,\quad E^{N+1}|v\rangle =0,
\]
这就是说存在一个$|0\rangle$使得
\[
	H|0\rangle =\lambda_M |0\rangle,\quad E|0\rangle =0.
\]
$\lambda_M$是$H$的最大的本征值,下面依旧以$\lambda$记$\lambda_M$.

我们再定义$|k\rangle=F^k|0\rangle$,那么$H|k\rangle=(\lambda-k) |k\rangle$,当然这也不可能无限地进行下去,就是说存在一个$m\in \mathbb{N}$使得$k\leq m$满足$|k\rangle\neq 0$但$|m+1\rangle=0$. 使用归纳法和对易关系$[E,F]=2H$可以算得
\[
E|k\rangle=(2k\lambda -k(k-1)) |k-1\rangle\quad (k>0).
\]
假如$|m+1\rangle =0$,那么
\[
0=E|m+1\rangle=(2(m+1)\lambda -m(m+1)) |m\rangle =(m+1)(2\lambda-m)|m\rangle.
\]
这就是说$2\lambda=m$. 那么$H$的最大本征值为正整数或者半整数$\lambda=m/2$,其他的本征值可以通过$\lambda-n=m/2-n$得到,所以也是整数或者半整数。这是一个$m+1$维的不可约表示,适当将$|0\rangle$重新编号成$|-m/2\rangle$这就推出了:

\pro $\mathfrak{sl}(2,\cc)$的$2j+1$维不可约表示中$H$的本征值或者是整数或者是半整数,且可以对角化为$\mathrm{diag}(-j,-j+1,\cdots,j-1,j)$。

利用这个子代数,甚至我们可以证明下面这一个结构性命题:

\pro 如果$\alpha$是一个根,对任意的非零标量$k\in \cc$,除了$k=0$或者$\pm 1$,$k\alpha$都不是一个根。此外$\dim \lag_{\alpha}=1$. 

证明的思路来自于$\mathfrak{s}^\alpha$在$\mathfrak{h} \oplus \bigoplus_k \lag_{k\alpha}$的伴随表示,这里就不说了。

至于$\dim \lag_{\alpha}=1$,这意味着本征方程$H^i|\alpha\rangle = \alpha^i|\alpha\rangle$的解并不存在简并。因此,最一般的Killing形式的计算可以如下进行
\[
	(h,h')_K=\sum_{\gamma \in \Delta}\gamma(h)\gamma(h'),
\]
特别地,我们有
\[
	\langle \alpha,\beta \rangle=(h_\alpha,h_\beta)_K=\sum_{\gamma \in \Delta}\gamma(h_\alpha)\gamma(h_\beta)=\sum_{\gamma \in \Delta}\langle \alpha,\gamma\rangle\langle \gamma,\beta\rangle.
\]

% \para 设$\alpha$是一个根,而$\ker\alpha=\{H\in \mathfrak{h}\,:\, H|\alpha\rangle=\alpha(H)=0\}$,由于$(h_\alpha,h)_K=\alpha(h)$又$H_\alpha$正比于$h_\alpha$,所以$\ker\alpha=\{H\in \mathfrak{h}\,:\, (H_\alpha,H)_K=0\}$. 令$(\ker\alpha)^\perp$是$\ker\alpha$在$\mathfrak{h}$的正交补,内积即Killing形式。% 如果$\dim \mathfrak{h}=r$,则$\dim \ker\alpha=r-1$以及$\dim (\ker\alpha)^\perp=1$.

% 考虑$\mathfrak{s}^\alpha$通过伴随表示作用在子代数$\mathfrak{m}=(\ker \alpha)^\perp \oplus \bigoplus_k \lag_{k\alpha}$. 由于$(\ker \alpha)^\perp$是一维的,以及$H_\alpha$是一个非零元。

% 我们要证明$\mathfrak{m}$在$\mathfrak{s}^\alpha$的伴随作用下不变,则$(\mathfrak{s}^\alpha,\mathfrak{m})$构成一个表示。首先证明他在$\ad(H_\alpha)$作用下不变,这是因为$(\ker \alpha)^\perp$以及$\lag_{k\alpha}$都是$H_\alpha$的本征空间。设$X\in \lag_\beta$,则$\ad(E_\alpha)X\in \lag_{\alpha+\beta}$,如果$\alpha+\beta$非零,则$\lag_{\alpha+\beta}=0$或者$\alpha+\beta=k\alpha$. 如果$\alpha+\beta=0$,则$\ad(E_\alpha)X\in (\ker \alpha)^\perp$,因此$\mathfrak{m}$在$\ad(E_\alpha)$作用下不变。同理$F_\alpha$. 

% 现在考虑$X\in \mathfrak{g}_{k\alpha}$,我们有$\ad(H_\alpha)X = \beta(H_\alpha)X = k\alpha(H_\alpha)X =kX $,由于这是$\mathfrak{sl}(2,\cc)$的一个表示,因此$k$必须是一个整数或者半整数。

% \para 如果$\alpha$是根,则$2\alpha$不是根。

% 这来自于半单Lie代数的有限维表示总可以分解为不可约表示。所以我们可以把$\mathfrak{m}$分解为几个不变子空间,分别对应一些不可约表示。如果$\lag_{2\alpha}$非零,则$X\in\lag_{2\alpha}$都满足$\ad(H_\alpha)X=2\alpha(H_\alpha)X=2X$,即本征值为$2$. 这个本征值为2的表示应该对应这上面的某个不变子空间$U_k$。对于这个表示,$0$也是$H_\alpha$在$U_k$的本征值(我们都知道,此时$H$的本征值能取$(-2,1,0,1,2)$这五个值)。这意味着我们有一个非零的$Y\in U_k$是的$\ad(H_\alpha)Y=0$,这就是说$Y\in (\ker \alpha)^\perp\subet \mathfrak{h}$,因为$\mathfrak{m}$是$(\ker \alpha)^\perp$和$\ad(H_\alpha)$的非零本征值的本征空间。然而,我们已经知道$(\ker \alpha)^\perp$是一维的,所以$\mathfrak{s}^\alpha$和$U_k$没有非零交集,矛盾,因为$\mathfrak{m}$是$\mathfrak{s}^\alpha$和其他不变子空间的直和。

\para 设有一个Lie代数的不可约有限维表示,则他限制在子代数$\mathfrak{s}^\alpha$上的表示也是不可约有限维表示。令$\lambda$是这个表示的一个权,而$\alpha$是一个根,以及$\mathfrak{s}^\alpha$的表示的维度为$2j+1$。对$|\lambda\rangle$,他的本征值为
\[
	H_\alpha|\lambda\rangle = \frac{1}{\langle \alpha,\alpha \rangle}h_\alpha|\lambda\rangle=\frac{\langle\alpha,\lambda\rangle}{\langle \alpha,\alpha \rangle}|\lambda\rangle.
\]

利用$p$次$E_\alpha$和$q$次的$F_\alpha$,我们分别可以到达最高和最低的本征值,即
\[
	j =\frac{\langle\alpha,\lambda\rangle}{\langle \alpha,\alpha \rangle}-p,\quad -j=\frac{\langle\alpha,\lambda\rangle}{\langle \alpha,\alpha \rangle}+q,
\]
换而言之
\[
	2\frac{\langle\alpha,\lambda\rangle}{\langle \alpha,\alpha\rangle}=-(q-p)
\]
是一个正整数。特别地,对任意的$\alpha$, $\beta\in \Delta$,$2\langle \alpha,\beta \rangle/\langle \alpha,\alpha \rangle$是一个整数。

如果把$\langle \star,\star \rangle$当成内积,那么$\langle \alpha,\lambda \rangle/\langle \alpha,\alpha \rangle$就是$\lambda$往$\alpha$方向的投影,可以看到投影只能是整数和半整数。

\para 考虑子代数$\mathfrak{k}=\bigoplus_j\lag_{\beta+j\alpha}$,其中$\beta\neq \pm \alpha$,每一个非零的加项都是一维的,以及$(\mathfrak{s}^\alpha,\mathfrak{k})$构成一个表示。对于$\beta+j\alpha\neq 0$的$j$,他们对应$H_\alpha|\beta+j\alpha\rangle =(\beta(H_\alpha)+j)|\beta+j\alpha\rangle$的本征空间,由于每一个权都差$1$,所以这是一个不可约有限维表示。因此存在两个自然数$p$和$q$使得
\[
	\beta-p\alpha,\beta-(p-1)\alpha,\cdots,\beta+q\alpha
\]
是一列根,在这列根之中,$\beta$是正数第$p$个根,那么倒数第$p$个应该也是一个根,即$\beta+(q-p)\alpha$也是一个根,即
\[
	\beta-2\frac{\langle \alpha,\beta \rangle}{\langle \alpha,\alpha \rangle}\alpha
\]
是一个根。几何上来看,$\alpha$确定了一个法平面,那么$\beta-2\alpha\langle \alpha,\beta \rangle/\langle \alpha,\alpha \rangle$就是$\beta$对着这个法平面反射得到的矢量。

\section{单根与Cartan矩阵}

\para 前面谈到了分解$\lag=\mathfrak{h}\oplus \lag'$,如果$\lag$的维度是$n$而$\mathfrak{h}$的维度是$r$,则$\lag'$的维度是$n-r$,对于$\lag'$,我们知道,他是由阶梯算符$\{E^\alpha\}$张成的,这就意味着根的数量其实是$n-r$个。由于$\Delta$张成$\mathfrak{h}^*$,所以$n-r\geq\dim \mathfrak{h}^*=\dim \mathfrak{h}=r$. 所以$\Delta\subset \mathfrak{h}^*$的元素个数大于等于$\mathfrak{h}^*$的维度,因此$\Delta$的元素们多半是线性相关的。固定$\mathfrak{h}^*$的一组基$\{\beta_1,\cdots,\beta_r\}$,任意一个根可以表示为$\alpha=n^i_\alpha\beta_i$.

我们将根分类如下:如果$(n^1_\alpha,\cdots,n^l_\alpha)$的第一个非零系数是正(负)的,就将其分类为$\alpha\in \Delta_+$($\alpha\in \Delta_-$),称作正根(负根)。由于$\alpha\in \Delta$总有$-\alpha\in \Delta$,则$\Delta_-=-\Delta_+$. 如果一个正根没法写成两个正根的和,则称其为一个单根。单根是一定存在的,考察有限集$\{n^1_\alpha\}_{\alpha\in \Delta_+}$,他里面最小的那个$n^1_\alpha$对应的就是第一个单根,记作$\alpha_1$。去掉其他正根里面的$\alpha_1$分量,我们在这些正根里面可以找到最小的那个$n^1_\alpha$对应的就是第二个单根,如是进行下去(至多有限步),就找到了一族单根。

\para 可以断言,任意两个单根的差不是根。设$\alpha$和$\beta$是单根,以及$\alpha-\beta$是根。如果$\alpha-\beta$是正根,则$\alpha=\beta+(\alpha-\beta)$,反之,如果$\alpha-\beta$是负根,则$\beta=\alpha+(\beta-\alpha)$,都矛盾。

同时可以断言,任意根都是单根的整系数线性组合。这是因为所有正根都是单根之和,而每一个负根都可以由正根得到。因此,单根是一族足够合适的矢量族用来展开根。因为$\mathfrak{h}^*$被$\Delta$张成,所以也被单根张成,故而单根的个数不可能少于$r$. 

\para 定义$\alpha^\vee=2\alpha/\langle \alpha,\alpha \rangle$,称为根$\alpha$的伴随根,从上一节,可以看到任意的$\langle \alpha^\vee,\beta\rangle$都是整数。对于单根族$\{\alpha_i\}$,定义矩阵$A_{ij}=\langle \alpha_i,\alpha^\vee_j\rangle$,称为Cartran矩阵。毫无疑问,这是一个整数矩阵,他的对角元素都是$2$.

从Schwarz不等式,$\langle \alpha_i,\alpha_j\rangle^2\leq \langle \alpha_i,\alpha_i\rangle\langle \alpha_j,\alpha_j\rangle$,所以如果$i\neq j$,则
\[
	A_{ij}A_{ji}=4\frac{\langle \alpha_i,\alpha_j\rangle\langle \alpha_j,\alpha_i\rangle}{\langle \alpha_i,\alpha_i\rangle\langle \alpha_j,\alpha_j\rangle}<4.
\]
因为$\alpha_i-\alpha_j$不是一个根,所以$E^{-\alpha_j}|\alpha_i\rangle =0$,这就意味着在
\[
	\langle \lambda,\alpha^\vee\rangle =2\frac{\langle\alpha,\lambda\rangle}{\langle \alpha,\alpha\rangle}=-(p-q)
\]
中的$q=0$,因此$A_{ij}\leq 0$当$i\neq j$。综上,$A_{ij}=0$, $-1$, $-2$, $-3$,且如果$A_{ij}\neq 0$,则至少有一个$A_{ij}$或者$A_{ij}$中有一个是$-1$,否则$A_{ij}A_{ji}>4$.

\para $\Delta$中的最大根$\theta$是很关键的,他写成单根的整系数线性组合$\sum_i a_i\alpha_i$,其中$\sum_i m_i$最大。所有的根都可以用最大根减去一些单根得到。系数$\{a_i\}$被称为最大根的标记,而如果用单根的伴随来展开,就有伴随标记$\{a^\vee_i\}$,通过
\[
	\theta=\sum_i a_i\alpha_i=\sum_i a^\vee_i\alpha^\vee_i
\]
联系,即$a_i=2a_i^\vee/\langle\alpha_i,\alpha_i\rangle$.

\para 现在,我们可以完全描述整个Lie代数了,对每一个单根$\alpha_i$,选取相应的$e^i=E_{\alpha_i}$, $f^{i}=F_{\alpha_i}$以及$h^i=H_{\alpha_i}$,全部的交换关系为
\[
	[h^i,h^j]=0,\quad [h^i,e^j]=A_{ji}e^j,\quad [h^i,f^j]=-A_{ji}f^j,\quad [e^i,f^j]=-\delta_{ij}h^j
\]

所以Cartran矩阵正表示了全部的对易关系中的系数。我们可以用图示来表示Cartran矩阵,对每一个单根$\alpha_i$,我们赋予一个点$i$,在每一对$i$和$j$之中,我们连接$A_{ij}A_{ji}$根线,所以对于根系的分类,就变成了对于这样一种图的分类,这种图称为Dynkin图。

\para 由于单根张成$\mathfrak{h}^*$,所以我们也可以用它来展开权。类似于所有的根都可以有单根整系数线性组合而成,我们希望找到一组基,他们可以用来整系数线性组合成所有的权,注意到
\[
	\langle\lambda,\alpha^\vee\rangle=2\frac{\langle\alpha,\lambda\rangle}{\langle \alpha,\alpha\rangle}=-(q-p)
\]
是一个整数,所以我们选取$\{\alpha_i^\vee\}$的对偶基$\{\omega_i\}$来展开,其中$\langle \omega_i,\alpha_j^\vee\rangle=\delta_{ij}$,这样,任意的权$\lambda$都可以如下展开
\[
	\lambda=\sum_i \langle \lambda,\alpha_i^\vee\rangle \omega_i=\sum_i \lambda_i \omega_i,
\]
其中每一个$\lambda_i$都是整数,被称为Dynkin标记,而$\omega_i$被称为基本权。Dynkin标记其实是$h^i$对权$|\lambda \rangle$的本征值,即
\[
	h^i|\lambda \rangle=\lambda(h^i)|\lambda \rangle=(\lambda,a_i^\vee)|\lambda \rangle=\lambda_i|\lambda \rangle.
\]

对于Cartan矩阵,$A_{ij}=\langle \alpha_i,\alpha^\vee_j\rangle$,所以我们有
\[
	\alpha_i=\sum_j A_{ij}\omega_j,
\]
因此$A$的第$i$行就是单根$\alpha_i$的Dynkin标记。

\para 考虑两个权$\lambda$和$\mu$的内积
\[
	\langle \lambda,\mu\rangle=\sum_{i,j}\lambda_i\mu_j\langle \omega_i,\omega_j\rangle=\sum_{i,j}\lambda_i\mu_j F_{ij}.
\]
其中的系数$F_{ij}$可以如下计算:注意到
\[
	\omega_i=\sum_j F_{ij} \alpha_j^\vee,\quad \alpha_i^\vee=\frac{2}{\langle \alpha_i,\alpha_i\rangle}\sum_j A_{ij}\omega_j,
\]
所以$F_{ij}$就是如下矩阵
\[
	F_{ij}=(A^{-1})_{ij}\frac{\langle \alpha_j,\alpha_j\rangle}{2}.
\]

\section{Weyl群}

令$H_\alpha|\beta\rangle = m|\beta\rangle$,则由
\[
	H_\alpha|\beta\rangle = \frac{1}{\langle \alpha,\alpha \rangle}h_\alpha|\beta\rangle=\frac{\langle\alpha,\beta\rangle}{\langle \alpha,\alpha \rangle}|\beta\rangle,
\]
我们可以断言
\[
	\langle\beta,\alpha^\vee\rangle=2\frac{\langle\alpha,\beta\rangle}{\langle \alpha,\alpha\rangle}=2m.
\]

如果$m\neq 0$,则存在另一个态$|\beta+l\alpha\rangle$,他的本征值是$-m$,那么
\[
	-\langle \alpha^\vee,\beta\rangle=-2m=\langle \alpha^\vee,\beta+l\alpha\rangle=\langle \alpha^\vee,\beta\rangle+\langle \alpha^\vee,l\alpha\rangle=\langle \alpha^\vee,\beta\rangle+2l,
\]
由此解出$\langle \alpha^\vee,\beta\rangle=-l$,所以如果$\beta$是一个根,则$\beta+l\alpha=\beta-\langle \alpha^\vee,\beta\rangle\alpha$也是一个根。这是我们已经知道的结论。

\para 定义变换
\[
	s_\alpha:\beta\mapsto \beta-\langle \alpha^\vee,\beta\rangle\alpha.
\]
所有这样的变换构成一个群,我们称为Weyl群。

考虑单根$\alpha_i$,我们简记$s_i=s_{\alpha_i}$,可以断言,任意Weyl群的元素都可以分解为$w=s_is_j\cdots s_k$,几何上来看这是比较清楚的,但是具体证明不算太容易。

\para 简单的计算我们就得到了$s_i^2=1$以及$s_is_j=s_js_i$如果$A_{ij}=0$,以及更有用一点的
\[
	s_i\alpha_j=\alpha_j-A_{ji}\alpha_i.
\]

\section{$\mathfrak{su}(n)_\cc$以及Young图}

\para 记$E_{ij}$是$n\times n$矩阵,只有$(i,j)$位置的矩阵元是$1$,其他都是零。一个有用的公式是
\[
	E_{ij}AE_{kl}=A_{jk}E_{il}.
\]

\para $\mathfrak{su}(n)$是所有迹零的满足$A=A^\dag$的所有复矩阵$A$构成的代数。这是一个实Lie代数,因为如果$A$满足$A=A^\dag$,则$iA$一般不满足$iA=(iA)^\dag$. 他的生成元可以选成
\[
	E_{ij}+E_{ji},\quad iE_{ij}-iE_{ji},\quad E_{11}-E_{ii},
\]
所以这个Lie代数的维数是$(n^2-n)+(n-1)=n^2-1$.

\para $\mathfrak{su}(n)_\cc$是$\mathfrak{su}(n)$的复化,复维度依旧是$n^2-1$,他的Cartan子代数$\mathfrak{h}$即全部的矩阵
\[
	\sum_{i=1}^n a_i E_{ii} \text{ ,where } \sum_{i=1}^n  a_i,
\]
他的复维度是$n-1$,因为他被$\{E_{11}-E_{ii}\,:\, 2\leq i \leq n\}$生成。

我们记$\epsilon_i \in \mathfrak{h}_*$为如下的线性函数
\[
	 \epsilon_i\left(\sum_{i=1}^n a_i E_{ii}\right)=a_i,
\]
则$\epsilon_i-\epsilon_1$构成了$\mathfrak{h}_*$的一组基。

通过计算
\[
	\left [\sum_{i=1}^n a_i E_{ii},E_{jk}\right ]=\sum_{i=1}^n  a_i(\delta_{ij}E_{ik}-\delta_{ik}E_{ji})=(a_j-a_k)E_{jk},
\]
可以得知$E_{ij}$构成了全部的升降算符,而对应于$E_{ij}$的根就是$\epsilon_i-\epsilon_j$,单根可以选作
\[
	\alpha_i= \epsilon_i-\epsilon_{i+1},
\]
同样是$n-1$个。

考虑任意的一个权$\lambda$,我们对$\alpha_i^\vee$将其展开,得到了Dynkin标记
\[
	\lambda=(\lambda_1,\cdots,\lambda_{n-1}),
\]
其中$\lambda_i$都是整数。我们考虑另一个整数组$\{l_1,\cdots,l_{n-1}\}$,其中$\lambda_i=\sum_{j=i}^{n-1}\lambda_j$。这两种表示是显然等价的。但是一般来说我们还会略去$\{l_1,\cdots,l_{n-1}\}$中的零。这种简略的表示其实还是等价于Dynkin标记的。

比如对于$\mathfrak{su}(5)_\cc$,我们有
\[
	\lambda=(2,0,2,0)=\{4,2,2\},
\]
我们将其用如下图表示
\[
	\yng(4,2,2)
\]
他被称为Young图。
\end{document}