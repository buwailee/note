\chapter{Curve}

\section{Cartier Divisor}

The Cartier divisor is related with the meromorphic function.

\begin{defi}
Let $X$ be a locally Noetherian scheme, and let
\[
	\Ass(\mathcal O_X) := \{ x\in X\,:\,
		\mathfrak m_x\in \Ass_{\mathscr O_{X,x}}(\mathscr O_{X,x})
	\}.
\]
The points of $\Ass(\mathcal O_X)$ are called the \textbf{associated points} 
of $X$. For any open subset $U\subset X$, $\Ass(\mathcal O_U)=U\cap \Ass(\mathcal O_X)$.
If $X=\spec R$ is an affine scheme, $\Ass(O_X)=\Ass(R)$.
The generic points of $X$ (\emph{i.e.} generic points of its irreducible components, or 
point $x\in X$ such that if $x\in \overline{\{y\}}$, then $y=x$) 
are associated points of $X$, associated points other than
generic points are call the \textbf{embedded points} of $X$.
\end{defi} 

%Suppose $\mathfrak p_x=\operatorname{ann}(r_x)$ is a prime ideal of $R$, 
%then $x$ is an embedded point of $\spec R$. For any prime ideal 
%$\mathfrak p_x\setminus \mathfrak p_y\neq \varnothing$, then 
%there exists $f\in \mathfrak p_x$ such that $f\not\in \mathfrak p_y$ and $r_xf=0$
%in $R_{y}$, so that $r_x=0 \in k(y)$. 


\begin{supply}[primary decomposition]
Here we review some facts on primary decomposition. Suppose $\mathfrak p$ is a prime
ideal of a ring $R$, a ideal $\mathfrak q$ is called a $\mathfrak p$-primary 
if $\sqrt{\mathfrak q}=\mathfrak p$ and for any $f$, $g\in R$ with $fg\in \mathfrak q$
but $f\not\in \mathfrak p$, we have $g\in \mathfrak q$, \textit{i.e.}
localization map $R/\mathfrak q\to (R/\mathfrak q)_{\mathfrak p}$ is monomorphism.

Any ideal of a Noetherian ring has primary decompositions. Suppose $\mathfrak a$ is a 
proper ideal of $R$, it can be written as an intersection $\mathfrak a=\cap_{i=1}^n 
\mathfrak q_i$, where $\mathfrak q_i$ is $\mathfrak p_i$-primary ideal. Let 
$P=\{\mathfrak p_i\}$, then 
\begin{compactenum}
\item $\Ass(R/\mathfrak a)\subset P$.
\item If the decomposition is minimal, \emph{i.e.} $n$ is minimal, then 
	$\mathfrak p_i\neq \mathfrak p_j$ iff $i\neq j$ and $\Ass(R/\mathfrak a)=P$. 
	Moreover, if $\mathfrak p_i$ is the minimal prime ideal containing $\mathfrak a$,
	then $\mathfrak q_i=\alpha_i^{-1}(\mathfrak a_{\mathfrak p_i})$, where 
	$\alpha_i:R\to R_{\mathfrak p_i}$ is the localization map, so it's unique.
\end{compactenum}
A prime ideal $\mathfrak p$ can occur in $P$ if and only if the length of the largest 
ideal of finite length in the ring $R_{\mathfrak p}/\mathfrak a_{\mathfrak p}$ (called
multiplicity of $\mathfrak p$ in $R/\mathfrak a$) is nonzero.
In particular, $R_{\mathfrak p}/\mathfrak a_{\mathfrak p}$ is Artinian iff $\mathfrak p$
is a minimal prime ideal containing $\mathfrak a$.

Geometrically, for a Noetherian affine scheme $\spec R$, each closed subscheme 
$V(\mathfrak a)$ has primary decompositions, it can be viewed as a union of 
primary closed subschemes $V(\mathfrak q_i)$. As topological spaces,
$|V(\mathfrak q_i)|=|V(\mathfrak p_i)|$, so the set-theoretically maximal 
components are unique, called \textbf{isolated components}, corresponding to
the minimal prime ideals containing $\mathfrak a$ (so they are in $\Ass(R/\mathfrak a)$).
The other components are called \textbf{embedded components}. If we apply this definition
to local scheme $\spec \mathscr O_{X,x}$, we'll get above terminologies.
\end{supply}


\begin{lem}
Let $X$ be a locally Noetherian scheme, $U$ an open subset of $X$, and 
$i:U\to X$ the canonical inclusion. Then the canonical homomorphism 
$\mathcal O_X\to i_*\mathcal O_U$ is injective iff $\Ass(\mathcal O_X)\subset U$.
\end{lem}

If $\Ass(\mathcal O_X)\subset U$, then $\overline U=X$, \textit{i.e.} $U$ is dense 
in $X$.  Conversely, if there's no embedded point in $\Ass(\mathcal O_X)$, 
$\overline U=X$ means that $\Ass(\mathcal O_X)\subset U$.

\begin{proof}
It's a local lemma, so we assume that $X=\spec R$, and then wo only need prove 
that $R\to \mathcal O_X(U)$ is injective iff $\Ass(R)\subset U$.
First we suppose that $\Ass(R)\subset U$. Let $r\in R$ s.t. $r|_U=0$. If $r\neq 0$,
there exists a nonzero $\mathfrak p=\operatorname{ann}(rs)\in \Ass(\langle r\rangle)
\subset \Ass(R)\subset U$
because $\langle r\rangle$ is a nonzero $R$-module. Since $r=0 \in R_{\mathfrak p}$,
there exists $t\not\in \mathfrak p$ s.t. $rt=0\in R$. However, it means that
$t\in \operatorname{ann}(rs)=\mathfrak p$. Therefore, $r=0\in R$.

Conversely, suppose that $R\to \mathcal O_X(U)$ is injective and there exists a 
prime ideal $\mathfrak p_y = \operatorname{ann}(r)\not\in U$. For any point $x\in U$,
$\mathfrak p_x\not\in \overline{\{y\}}=V(\mathfrak p_y)$, i.e. 
$\mathfrak p_y \not\subset \mathfrak p_x$, so there exists a $s\in \mathfrak p_y$ 
such that $s_x$ is a unit of $R_x\subset \mathcal O_{X,x}$. Now $r_x=0$ from 
$(sr)_x=s_xr_x=0$. Therefore, $r|_U=0$.
\end{proof}

\begin{defi}
Suppose $R$ is a ring, $r\in R$ is regular if it's not a zero-divisor, 
or equivalently $\operatorname{ann}(r)=0$. Let's denote $\operatorname{Reg}(R)$
the set of regular elements of $R$ and define $\operatorname{Frac}(R)$ by
$(\operatorname{Reg}(R))^{-1}R$.
\end{defi}

Now we can construct two new presheaives $\mathcal R_X$ and $\mathcal K_X$ for a 
scheme $X$ by 
\[
	\mathcal R_X(U)=\{a\in \mathcal O_X(U)\,:\, a_x\in 
	\operatorname{Reg}(\mathcal O_{X,x})\text{ for all $x\in U$}\},
	\quad \mathcal{PK}_X(U)= (\mathcal R_X(U))^{-1}\mathcal O_X(U)
\]
It's clear that $\mathcal R_X$ is a sheaf,  
$\mathcal R_X(U)\subset \operatorname{Reg}(\mathcal O_X(U))$ for open set $U$, 
and then $\mathcal R_{X,x} \subset \operatorname{Reg}(\mathcal O_{X,x})$ for any $x$.
In fact, if a section $s\not\in \operatorname{Reg}(\mathcal O_X(U))$, then there exists 
a non-zero section $t\in \mathcal O_X(U)$ such that $st=0$, and a point $x_0\in U$ 
such that $t_{x_0}\neq 0$, so that $s_{x_0}t_{x_0}=0$ means that $s_{x_{0}}\not\in 
\operatorname{Reg}(\mathcal O_{X,x_0})$ and then $s\not\in \mathcal R_X(U)$. Moreover,
$\mathcal R_X(U)=\operatorname{Reg}(\mathcal O_X(U))$ for an affine open set $U$ since
$s_{\mathfrak p}$ is regular in $R_{\mathfrak p}$ if $s$ in regular in $R$.
However, $\mathcal{PK}_X$ does not need to be a sheaf, let's denote its 
associated sheaf by $\mathcal K_X$.

\begin{lem}
Let $X$ be a locally Noetherian scheme, then 
$\mathcal R_{X,x}=\operatorname{Reg}(\mathcal O_{X,x})$ 
for any $x\in X$. Therefore, $\mathcal K_{X,x}=\mathcal R_{X,x}^{-1}\mathcal O_{X,x}
=\operatorname{Frac}(\mathcal O_{X,x})$.
\end{lem}

\begin{proof}
Let $a_x=(a,U)$ is a stalk in $\operatorname{Reg}(\mathcal O_{X,x})$ with
$U$ affine and $\mathcal O_X(U)$ Noetherian. 
Since $\mathcal O_X(U)$ is Noetherian, $\operatorname{ann}(a)$
is finitely gengerated whose generators are denoted by $\{a_i\,:\,0\leq i\leq n\}$.
Since $a_x\in \operatorname{Reg}(\mathcal O_{X,x})$, 
$\operatorname{ann}(a)\mathcal O_{X,x}=0$ or $(a_i)_x\mathcal O_{X,x}=0$. There exists
affine open subset $V_i$ such that $a_i|_{V_i}\mathcal O_X(V_i)=0$ for each $i$, so on
$V=\cap_{i=1}^n V_i$, $\operatorname{ann}(a|_V)\mathcal O_X(V)=0$. Hence 
$\operatorname{ann}(a|_V)=0$ or 
$a|_V\in R(\mathcal O_X(V))=\mathcal R_X(V)$, and then $a_x\in \mathcal R_{X,x}$,
which completes the proof.
\end{proof}

\begin{pro}
Let $X$ be a locally Noetherian scheme, $U$ an open subset of $X$ such that 
$\Ass(\mathcal O_X)\subset U$ and 
$i:U\to X$ the canonical inclusion. Then the canonical homomorphism 
$\mathcal K_X\to i_*\mathcal K_U$ is an isomorphism.
\end{pro}

\begin{proof}
	See [Qing Liu, Proposition 7.1.15].
\end{proof}

\begin{defi}
Let $X$ be a scheme, we denote the group $H^0(X,\mathcal K^*_X/
\mathcal O_X^*)$ by $\Div(X)$ whose elements are called \textbf{Cartier divisors}
on $X$, where $A^*$ is the set of invertible elements of a ring $A$.
The image of $f\in H^0(X,\mathcal K^*)$ in $\Div(X)$ is called a \textbf{principle}
Cartier divisor and is denoted by $\operatorname{div}(f)$. Moreover, a divisor
$D$ is called \textbf{effective} if it's in the image of 
$H^0(X,\mathcal O_X\cap \mathcal K^*)\to \Div(X)$, denoted by $D\geq 0$
and the image is denoted by $\Div_+(X)$.
\end{defi}

By definition, we can represent a Cartier divisor $D$ by a system $\{(U_i,f_i)\}_i$, 
where $\{U_i\}$ is an open covering of $X$ and $\{f_i\}$ are compatible sections of 
$\mathcal K^*_X$. We can further require that $f_i$ is the quotient of two regular 
elements of $\mathcal O_X(U_i)$, and the compatible condition is 
$f_i|_{U_i\cap U_j}\in f_j|_{U_i\cap U_j}\mathcal O_X(U_i\cap U_j)^*$.
Two such system $\{U_i,f_i\}$ and $\{V_j,g_j\}$ are equivalent if 
$f_i$ and $g_j$ differ by a multiplicative factor in $\mathcal O_X(U_i\cap V_j)^*$.
$D\geq 0$ iff it's represented by $\{(U_i,f_i)\}$ where $f_i\in \mathcal O_X(U_i)$.
Suppose that $D_1=\{(U_i,f_i)\}_i$ and $D_2=\{(V_j,g_j)\}_j$, then the group 
structure of $\Div(X)$ is represented by 
\[
	D_1+D_2=\{(U_i\cap V_j,f_ig_j)\}_{i,j}.
\]
A Cartier divisor can be related to a invertible sheaf on $X$: for a Cartier divisor 
$D=\{(U_i,f_i)\}_i$, we associate a subsheaf $\mathcal O_X(D)\subset \mathcal K_X$ 
defined by $\mathcal O_X(D)|_{U_i}=f_i^{-1}\mathcal O_X|_{U_i}$

\begin{lem}
Suppose $R$ is a Noetherian local ring of dimension $1$, and let $f$, $g$ are regular
elements, then $R/\langle f\rangle$ is of finite length, and 
\[
	\operatorname{length}(R/\langle fg\rangle) =
	\operatorname{length}(R/\langle f\rangle) +
	\operatorname{length}(R/\langle g\rangle).
\]
\end{lem}

\begin{proof}
Since $f$ is regular, there's no minimal prime $\mathfrak p$ of $R$ containing it,
and then $R/\langle f\rangle$ is of dimension $0$, or it's Artinian and of finite
length. Now, consider the short exact sequence
\[
	0\to \langle g\rangle /\langle fg\rangle \to 
	R/\langle fg\rangle\to R/\langle g\rangle \to 0,
\]
and note that the $R$-module morphism 
\[
	R/\langle f\rangle \to\langle g\rangle/\langle fg\rangle
\]
given by $r\mapsto gr$ is an isomorphism because $g$ is regular, then we get the proof.
\end{proof}

We can extend $f\mapsto \operatorname{length}(R/\langle f\rangle)$ to a group 
homomorphism $\operatorname{Frac}(R)^*/R^*\to \mathbb Z$ by the above lemma, 
denoted by $\operatorname{mult}_R$. Note that $\operatorname{mult}_R(f/g)
=\operatorname{mult}_R(f)-\operatorname{mult}_R(g)$ for two regular elements $f$ and 
$g$.

Let $X$ be a locally Noetherian scheme, and $D\in \Div(X)$ is a Cartier divisor.
For any point $x\in X$ of codimenison $1$, \textit{i.e.} $\dim \mathcal O_{X,x}=1$,
the stalk of $D$ at $x$ belongs to $(\mathcal K^*_X/\mathcal O_X^*)_x=
\operatorname{Frac} (\mathcal O_{X,x})^*/\mathcal O_{X,x}^*$, and then we define
\[
	\operatorname{mult}_x(D):= \operatorname{mult}_{\mathcal O_{X,x}}(D_x).
\]
