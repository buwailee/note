\chapter{Basics}

\section{Homology of Complexes}

In this section, we will recall some basic definitions in
homological algebra.

\begin{definition}
	Suppose $A$ is an abelian category, a complex $K^\bullet$
	is a series of objects $K^i\in A$ and morphisms $d^i_K:K^{i-1}\to K^i$ such that $d_K^{i+1}d_K^i=0$.
\end{definition}

It's natural to visualize a complex $K^\bullet$ by a diagram
\[
	\cdots  \xrightarrow{d_K^{i-2}} K^{i-1} \xrightarrow{d_K^{i-1}}K^i \xrightarrow{d_K^{i}}
	K^{i+1}\xrightarrow{d^{i+1}}\cdots.
\]
If the label of $d_K^{i}$ is manifest in the context, 
we will write it as $d$ for short. For example, $d^2=0$ is the 
shorthand of $d_K^{i+1}d_K^i=0$.

\begin{definition}
	Suppose $K^\bullet$, $L^\bullet$ are two complexes in $A$, then
	$f=\{f^i:K^i\to L^i\}_i$ is called a morphism between 
	$K^\bullet$ and $L^\bullet$ if $f^{i+1}d^i_K=d^i_Lf^i$ for all $i$.
\end{definition}

The morphism $f$ between two complexes $K^\bullet$ and $L^\bullet$  
is usually written as $f:K^\bullet\to L^\bullet$.
It can be visualized by the following commutative diagram
\[
	\xymatrix{
		\cdots \ar[r]& K^{i-1}\ar[r]^{d^{i-1}_K}\ar[d]^{f^{i-1}}&K^i\ar[r]^{d^{i}_K}\ar[d]^{f^{i}}&K^{i+1}\ar[r]^{d^{i+1}_K}\ar[d]^{f^{i+1}}&\cdots\\
		\cdots \ar[r]& L^{i-1}\ar[r]^{d^{i-1}_L}&L^i\ar[r]^{d^{i}_L}&L^{i+1}\ar[r]^{d^{i+1}_L}&\cdots
	}
\]

Complexes on an abelian category $A$ 
and morphisms of complexes form a category. Let's denote it by 
$\operatorname{Kom}(A)$. In applications, it's often useful to consider
complexes with various finitness conditions. In particular, let 
\[
\begin{aligned}
	&\operatorname{Kom}^+(A):K^i=0\quad \text{for}\quad i< i_0(K^\bullet),\\
	&\operatorname{Kom}^-(A):K^i=0\quad \text{for}\quad i> i_0(K^\bullet),\\
	&\operatorname{Kom}^b(A)=\operatorname{Kom}^+(A)\cap \operatorname{Kom}^-(A),\\
\end{aligned}
\]
where $i_0(K^\bullet)$ is a integer. $\operatorname{Kom}^+(A)$, 
$\operatorname{Kom}^-(A)$ and $\operatorname{Kom}^b(A)$ are clearly
subcategories of $\operatorname{Kom}(A)$.

\begin{definition}
	Suppose $K^\bullet \in \operatorname{Kom}(A)$ is a complex, then 
	the shifted complex $K[n]^\bullet$ by a integer $n$
	is defined by
	\[
		(K[n])^i=K^{n+i},\quad d_{K[n]}^i=(-1)^n d_K^{n+i}.
	\]
	It's also natural to define $f[n]:K[n]\to L[n]$ by 
	$(f[n])^i=f^{n+i}$.
\end{definition}

There 
\[
	T^n:K^\bullet \mapsto K[n]^\bullet, \quad 
	T^n:f\mapsto f[n]
\]
is a functor $T^n:\operatorname{Kom}(A)\to \operatorname{Kom}(A)$, which
is clearly a equivalence. Because of this equivalence, one can set 
$i_0(K^\bullet)=0$ for a single $K^\bullet \in \operatorname{Kom}^+(A)$.

\begin{definition}
Suppose $K^\bullet$ is a complex in an abelian category $A$, then for 
$K^{i-1}\xrightarrow{d^{i-1}}K^i\xrightarrow{d^i}K^{i+1}$, there exist
morphisms $a$ and $b$ such that the following diagram is commutative
\[
	\begin{xy}
		\xymatrix{
			&&\cdot \ar[rrd]^-{b}&&\\
			K^{i-1} \ar[rr]^-{d^{i-1}} \ar[rrd]_{a}&& K^i \ar[rr]^-{d^i}\ar[u]^-{\coker d^{i-1}}&&K^{i+1}\\
			&&\cdot \ar[u]_-{\ker d^i}&&
		}
	\end{xy}
\]
Then we could define the homology object by 
$H^i(K^\bullet):=\coker(a)=\ker(b)$.
\end{definition}

In fact, since $d^id^{i-1}=0$ and the universal property of kernal of $d^i$,
there exists a morphism $a$ such that $d^{i-1}=\ker (d^i)a$. The existance
of $b$ is similarly form the universal property of cokernal of $d^{i-1}$.
The most non-trivial part in the above definition is that objects
$\coker(a)$ and $\ker(b)$ are the same (or more formally, they are naturally
isomorphism). To prove it, one can consider the following diagram firstly
\[
	\begin{xy}
		\xymatrix{
		&&\cdot \ar[rrd]^-{b'}&&\\
		\im(d^{i-1}) \ar[rr]^-{d^{i-1}} \ar[rrd]_-{a'}&& K^i \ar[rr]^-{d^i}\ar[u]^-{\coker d^{i-1}}&&\operatorname{coim}(d^{i}) \\
		&&\cdot \ar[u]_-{\ker d^i}&&
		}
	\end{xy}
\]
and then prove that 
\[
	\ker b'=\ker b= \im (\coker d^{i-1}\ker d^i),\quad  \coker a'=\coker a
	=\operatorname{coim} (\coker d^{i-1}\ker d^i).
\]
It's an easy exercise so that we omit it.

If $A$ is the category of modules, then
\[
	H^i(K^\bullet)=\ker d^i/\im d^{i-1},
\]
it's the classical definition.

\begin{pro}
	Suppose $f:K^\bullet \to L^\bullet$ is a morphism between complexes,
	then it induces morphisms $\hat f$, $\bar f$ and $H^\bullet(f)$ 
	such that the following diagrams are commutative
	\[
		\xymatrix{
			\ker d_K^i\ar[r]\ar@{-->}[d]^{\hat{f}^i}&K^i\ar[r]
			\ar[d]^{f^i}&\coker d^{i-1}_K\ar@{-->}[d]^{\bar{f}^i}
			\\
			\ker d_L^i\ar[r]&L^i\ar[r]&\coker d^{i-1}_L
		}
		\quad 
		\xymatrix{
			\ker d_K^i\ar[rr]^{\coim \varphi^i_K}\ar@{-->}[d]^{\hat{f}^i}&&H^i(K^\bullet)\ar[r]^{\im\varphi^i_K}
			\ar[d]^{H^i(f)}&\coker d^{i-1}_K\ar@{-->}[d]^{\bar{f}^i}\\
			\ker d_L^i\ar[rr]^{\coim \varphi^i_L}&&H^i(L^\bullet)\ar[r]^{\im\varphi^i_L}&\coker d^{i-1}_L
		}
	\]
	where $\varphi^i=\coker d^{i-1}\ker d^i$.
\end{pro}

\begin{proof}
	It's an easy exercise about universal properties. Since that
	$f^{i+1}d_K^i=d_L^if^i$, then $d_L^if^i\ker d_K^i=0$, and by
	the universal properties of $\ker d_L^i$, there exists a 
	morphism $\hat{f}^i$ such that 
	\[
		f^i\ker d_K^i=\ker d_L^i\hat{f}^i.
	\]
	The same method works for the existance of $\bar f^i$. 
	For the second diagram, since that
	\[
	\begin{aligned}
		\coker \varphi^i_L \bar f^i \varphi^i_K
		&= \coker \varphi^i_L \bar f^i \varphi^i_K\\
		&= \coker \varphi^i_L \varphi^i_L \hat f^i \\
		&= 0,
	\end{aligned}
	\]
	$\coker \varphi^i_L \bar f^i \im \varphi^i_K$ also vanishes. In fact,
	\[
		\coker \varphi^i_L \bar f^i \im \varphi^i_K\operatorname{coim}\varphi^i_K=\coker \varphi^i_L \bar f^i \varphi^i_K=0
	\]
	and $\operatorname{coim}\varphi^i_K$ is an epimorphism.
	Therefore, the universal property of 
	$\ker \coker \varphi_K^i=\im \varphi_K^i$ gives a
	morphism $H^i(f)$ such that
	\[
		\bar f^i \im \varphi^i_K=\im \varphi_L^i H^i(f).
	\]
	It remains to show that $\coim \varphi_L^i\hat f^i= H^i(f)\coim \varphi_K^i$, which follows from that
	\[
		\im \varphi_L^i\coim \varphi_L^i\hat f^i=\varphi^i_L\hat f^i=\bar f^i \varphi^i_K=
		\im \varphi_L^i H^i(f)\coim \varphi^i_K
	\]
	and $\im \varphi_L^i$ is a monomorphism.
\end{proof}

In the the category of modules, suppose $[a]\in H^i(K)$, then $H^i(f)$ is
defined by $H^i(f)[a]=[f^i(a)]\in H^i(L)$.

Now we have defined 
\[
	H^n :K^\bullet\mapsto H^n(K^\bullet)\quad \text{and}\quad
	H^n :f\mapsto H^n(f),
\]
it's really easy to check that $H^n$ is a functor for any $n$, 
which is called the homology functor.

\begin{definition}
	Suppose $f$, $g:K^\bullet\to L^\bullet$ are morphisms of two complexes,
	if there exists a series morphisms $s^i:K^i\to L^{i-1}$ such that
	\[
		f^i-g^i=s^{i+1}d_K^i+d_L^{i-1}s^i,
	\]
	then $f$ and $g$ are said to be homotopic and written as $f\sim g$.
\end{definition}

It's easy to see that $f\sim g$ is equivalent to $f-g\sim 0$.

\begin{lem}\label{lem:1.7}
	Suppose $f$, $g:K^\bullet\to L^\bullet$ are morphisms of two complexes.
	If $f\sim g$, then $H^\bullet(f)=H^\bullet(g)$.
\end{lem}

\begin{proof}
	It's fine to reduce to the case that $g=0$ and then $H^\bullet(g)=0$. We will
	prove that
	\[
		\bar f^i \varphi_K^i = \im \varphi_L^i H^i(f) \operatorname{coim} \varphi_K^i=0.
	\]
	In fact,
	\[
		\begin{aligned}
		\bar f^i \varphi_K^i=\bar f^i \coker d_K^{i-1}\ker d_{K}^i 
		&=\coker d_L^{i-1} f^i\ker d_{K}^i\\
		&=\coker d_L^{i-1}(s^{i+1}d_K^i+d_L^{i-1}s^i)\ker d_{K}^i\\
		&=\coker d_L^{i-1}s^{i+1}d_K^i\ker d_{K}^i+\coker d_L^{i-1}d_L^{i-1}s^i\ker d_{K}^i\\
		&=0.
		\end{aligned}
	\]
	Therefore, $H^i(f) = 0$ because $\im \varphi_L^i$ is a monomorphism
	and $\operatorname{coim} \varphi_K^i$ is an epimorphism.
\end{proof}

\begin{definition}
	A morphism $f:K^\bullet\to L^\bullet$ of complexes in an abelian
	category is said to be a quasi-isomorphism if the corresponding 
	homology morphism $H^n(f):H^n(K^\bullet)\to H^n(L^\bullet)$
	is an isomorphism for any $n$.
\end{definition}

\begin{pro}\label{pro:1.9}
	A short exact sequence of complexes $0\to K^\bullet\xrightarrow{f}L^\bullet\xrightarrow{g}M^\bullet\to 0$ induce a exact sequence 
	of homology
	\[
		\cdots\to H^n(K^\bullet)\xrightarrow{H^n(f)}H^n(L^\bullet)\xrightarrow{H^n(g)}H^n(M^\bullet)\xrightarrow{\partial^n}H^{n+1}(K^\bullet)\xrightarrow{H^{n+1}(f)}H^{n+1}(L^\bullet)\to\cdots.
	\]
\end{pro}

In the category of modules, $\partial^n:H^n(M^\bullet)\to H^{n+1}(K^\bullet)$ can be calculated by
\[
	\partial^n([c])=\left[(f^{n+1})^{-1}\left(\dd^n_L\left((g^n)^{-1}(c)\right)\right)\right].
\]

\begin{proof}
	According to the snake lemma, there's a exact sequence
	\[
		0\to \ker d_K^n\xrightarrow{\hat f^n} \ker d_L^n
		\xrightarrow{\hat g^n}
		\ker d_M^n \xrightarrow{\delta^n}
		\coker d_K^n\xrightarrow{\bar f^{n+1}} \coker d_L^n\xrightarrow{\bar g^{n+1}} \coker d_M^n \to 0
	\]
	for any $n$. The left part of this proof is an easy exercise.
	One could use the Mitchell embedding theorem to prove it in the category
	of modules only.
\end{proof}

% \[
% 	\xymatrix{
% 	&0\ar[d]&0\ar[d]&0\ar[d]&\\
% 	\cdots\ar[r]&K^{n-1}\ar[r]^{\dd_K^{n-1}}\ar[d]^{f^{n-1}}&K^n\ar[r]^{\dd_K^n}\ar[d]^{f^{n}}&K^{n+1}\ar[r]\ar[d]^{f^{n+1}}&\cdots\\
% 	\cdots\ar[r]&L^{n-1}\ar[r]^{\dd_L^{n-1}}\ar[d]^{g^{n-1}}&L^n\ar[r]^{\dd_L^n}\ar[d]^{g^{n}}&L^{n+1}\ar[r]\ar[d]^{g^{n+1}}&\cdots\\
% 	\cdots\ar[r]&M^{n-1}\ar[r]^{\dd_L^{n-1}}\ar[d]&M^n\ar[r]^{\dd_L^n}\ar[d]\ar[ruu]&M^{n+1}\ar[r]\ar[d]&\cdots\\
% 	&0&0&0&
% 	}
% \]

\begin{definition}[Cone]
Let $f:K^\bullet\to L^\bullet$ be a morphism of complexes,
the cone of $f$ is the following complex $\operatorname{Cone}(f)$:
\[
	\operatorname{Cone}(f)^i=K[1]^i\oplus L^i=K^{i+1}\oplus L^i,\quad
	d_{\operatorname{Cone}(f)}=\begin{pmatrix}
		d_{K[1]}&0\\
		f[1]&d_L
	\end{pmatrix}.
\]
\end{definition}
One easily verifies that
\[
	d_{\operatorname{Cone}(f)}^2=\begin{pmatrix}
		d_{K[1]}&0\\
		f[1]&d_L
	\end{pmatrix}\begin{pmatrix}
		d_{K[1]}&0\\
		f[1]&d_L
	\end{pmatrix}
	=\begin{pmatrix}
		0&0\\
		f[1]d_{K[1]}+d_Lf[1]&0
	\end{pmatrix}
	=0,
\]
because that
\[
	(f[1]d_{K[1]}+d_Lf[1])^i =
	-f^{i+2}d_K^{i+1}+d_L^{i+1} f^{i+1}=0.
\]

The cone of a morphism has a geometric background: 
the complex of singular chains of the topological cone is homotopy equivalent to the cone of the induced map of singular chains, i.e. for a continuous map $\varphi:X\to Y$
\[
	\operatorname{Complex}(\operatorname{cone}(\varphi)) 
	\sim \operatorname{Cone}(\varphi_*),
\]
where $\operatorname{cone}(\varphi)$ the topological cone and 
$\operatorname{Complex}(*)$ is the complex of singular chains of a topological space.

% There's a natural morphism between $K[1]^\bullet$ and
% $\operatorname{Cone}(f)^\bullet$
% \[
% 	\eta^i:K[1]^i\to \operatorname{Cone}(f)^i=K[1]^i\oplus L^i
% \]
% defined by the natural morphism of coproduct.

\begin{definition}[Cyclinder]
	Let $f:K^\bullet\to L^\bullet$ be a morphism of complexes,
	the cyclinder of $f$ is cone of the natural morphism of biproduct
	\[
		\delta(f)[-1]:\operatorname{Cone}(f)[-1] \to K^\bullet
	\]
	i.e. it's the following complex $\operatorname{Cyl}(f)$:
	\[
		\operatorname{Cyl}(f)^i=\operatorname{Cone}(f)^i\oplus K^{i} ,\quad
		d_{\operatorname{Cyl}(f)}=\begin{pmatrix}
			d_{\operatorname{Cone}(f)}&0\\
			\delta(f) & d_{K}
		\end{pmatrix}.
	\]
\end{definition}

In algebraic topology, the mapping cyclinder $\operatorname{cyl}(\varphi)$ of
a continuous map $\varphi:X\to Y$ is the colimit of the diagram
\[
	X\times [0,1] \xleftarrow{i} X \xrightarrow{\varphi} Y,
\]
where $i:X\hookrightarrow X\times [0,1]$ is the inclusion defined by $x\mapsto (x,0)$. 
The universal universal of mapping cylinder:
for any space $Z$ and mapping $g_1:X\times [0,1]\to Z$, $g_2:Y\to Z$ such that 
$g_1(x,0)=g_2(f(x))$ for all $x\in X$, there is a unique $k:\operatorname{cyl}(\varphi)\to Z$,
such that the following diagram is commutative
\[
	\xymatrix{
		Z &  & \\
		  & \operatorname{cyl}(\varphi)\ar@{-->}[ul]_k & Y\ar[l]\ar@/_/[llu]_{g_2}\\
		  & X\times [0,1]\ar[u]\ar@/^/[luu]^{g_1} & X\ar[u]\ar[l]
	}
\]
It's very natural to consider the space $X\times [0,1]$ because of the definition
of homotopy\footnote{A homotopy between $f$, $g:X\to Y$ is a map $F:X\times [0,1]\to Y$
such that $F(x,0)=f(x)$ and $F(x,1)=g(x)$ for any $x\in X$.}. 
% Suppose $F$ is a homological
% between $f$ and $g$, then the universal property of $\operatorname{cyl}(f)$
% \[
% 	\xymatrix{
% 		Y &  & \\
% 		  & \operatorname{cyl}(f)\ar@{-->}[ul]_{h} & Y\ar[l]\ar@/_/[llu]_{\id_Y}\\
% 		  & X\times [0,1]\ar[u]\ar@/^/[luu]^{F} & X\ar[u]_f\ar[l]
% 	}
% \]
% gives us a map $h:\operatorname{cyl}(f)\to Y$.
One of the most important properties of mapping cyclinder is the homotopy equivalence between $Y$
and $\operatorname{cyl}(\varphi)$ induced by the natural map $Y\to \operatorname{cyl}(\varphi)$.
In fact its homotopy inverse can be chosen a deformation retraction. In particular every continuous function factors as a map into its mapping cylinder followed by a deformation retraction.

In the abstract theory of complexes, we will see the same properties of a cyclinder of a morphism.
In fact, we could find a object $I$ in any abelian category like an closed interval $[0,1]$ in the 
category of topological spaces, then the cyclinder of a morphism $f:K^\bullet \to L^\bullet$ 
is just the colimit of the following diagram 
\[
	K^\bullet\otimes I^\bullet \xleftarrow{i}K^\bullet \xrightarrow{f}L^\bullet.
\]

\begin{pro}\label{pro:1.12}
	For any morphism $f:K^\bullet \to L^\bullet$ there exists the following commutative
	diagram with exact rows:
	\begin{equation}\label{tri}
		\xymatrix{
			& 0\ar[r] & L^\bullet \ar[r]^-{\bar\pi}\ar[d]^\alpha & \Cone(f) \ar[r]^-{\delta(f)}\ar@{=}[d]
			& K[1]^\bullet \ar[r] & 0\\
			0 \ar[r] & K^\bullet \ar[r]^-{\bar f}\ar@{=}[d] & \operatorname{Cyl}(f) \ar[r]^-\pi\ar[d]^\beta 
			&\Cone(f) \ar[r] & 0 & \\
			& K^\bullet \ar[r]^f & L^\bullet & & &
		}
	\end{equation}
	where
	\[
		\begin{aligned}
		&\bar f = (0,0,\id_{K^i}),\quad\bar\pi^i=(0,\id_{L^i}),\quad \pi^i=(p^i_{K[1]},p^i_{L})\\
		&\alpha^i = (0,\id_{L^i},0),\quad \beta^i= p^i_{L} + f^i p^i_K,\quad \delta(f)=q_{K[1]}
		\end{aligned}
	\]
	and $p_{K[1]}$, $p_{L}$ and $p_{K}$ is the canonical morphism of the product 
	$\operatorname{Cyl}(f)=K[1]^\bullet\oplus L^\bullet\oplus K^\bullet$,
	$q_{K[1]}$ and $q_L$ is the canonical morphism of the product of $\Cone(f)=K[1]^\bullet\oplus L^\bullet$,
	then $\beta\alpha=\id_L$ and $\alpha\beta\sim \id_{\operatorname{Cyl}(f)}$.
	Therefore, $\alpha$ and $\beta$ are quasi-isomorphisms between $L^\bullet$ and $\operatorname{Cyl}(f)$.
\end{pro}

\begin{proof}
	It's direct.
\end{proof}

\begin{lem}\label{lem:1.13}
	Suppose $f:K^\bullet\to L^\bullet$ is a morphism, then $f\delta(f)[-1]\sim 0$.
\end{lem}

\begin{proof}
	Suppose that 
	\[
		p^i:(\Cone(f)[-1])^i=K^i\oplus L^{i-1}\to L^{i-1}
	\]
	is the canonical projection of the product. Now we can check that 
	\[
		\begin{aligned}
			p^{i+1}d^i_{\Cone(f)[-1]}+d^{i-1}_{L}p^i
			&=-p^{i+1}d^{i-1}_{\Cone(l)}+d^{i-1}_{L}p^i\\
			&=-p^{i+1}\begin{pmatrix}
				-d^{i}_{K}&0\\
				f^i&d_{L}^{i-1}
			\end{pmatrix}+d^{i-1}_{L}p^i\\
			&=-f^i\delta(f)^{i-1}-d_{L}^{i-1}p^i+d^{i-1}_{L}p^i\\
			&=-f^i\delta(f)^{i-1},
		\end{aligned}
	\]
	which imples that $f\delta(f)[-1] \sim 0$.
\end{proof}

\begin{pro}\label{pro:1.13}
	Suppose $0\to K^\bullet \xrightarrow{f}L^\bullet \xrightarrow{g}M^\bullet\to 0$ is a short exact sequence of complexes, then it is quasi-isomorphic to the middle row of eq.\eqref{tri}, i.e. there exists morphisms $\alpha$, $\beta$, $\gamma$ such that the following diagram
	is commutative,
	\[
		\xymatrix{
		0 \ar@{->}[r] & K^\bullet \ar@{->}[r]^{f} & L^\bullet \ar@{->}[r]^{g} & M^\bullet \ar@{->}[r] & 0 \\
		0 \ar@{->}[r] & K^\bullet \ar@{->}[r]^{\bar f} \ar@{->}[u]_{\alpha} & \operatorname{Cyl}(f) \ar@{->}[r]^{\pi} \ar@{->}[u]_{\beta} & \operatorname{Cone}(f) \ar@{->}[r] \ar@{->}[u]_{\gamma} & 0
		}
	\]
	where $\bar f$ and $\pi$ is defined in Proposition \ref{pro:1.12}.
\end{pro}

\begin{proof}
	We use the same notation in the Proposition \ref{pro:1.12}. Define that
	\[
		\alpha=\id,\quad \beta^i= p^i_{L} + f^i p^i_K,\quad \gamma^i=gq_L^i.
	\]
	It's easy to show that $\gamma$ is a morphism. Since that 
	\[
		\gamma\pi=gq_L\pi=gp_L,\quad g\beta=gp_L+gfp_K=gp_L,
	\]
	the diagram is commutative.

	As has showed in the Proposition \ref{pro:1.12}, $\beta$ is a quasi-isomorphism, so we only need to show that $\gamma$ is a quasi-isomorphism. Firstly, notice that $\gamma=gq_L$ is a epimorphism since $g$ and $q_L$ are both epimorphic. Consider the short exact sequence
	\[
		0\to \ker \gamma\to \Cone(f)\xrightarrow{f} M^\bullet\to 0,
	\]
	it imples the long exact sequence of homology
	\[
		H^n(\ker \gamma)\to H^n(\Cone(f))\xrightarrow{H^n(\gamma)}H^n(M)\to H^{n+1}(\ker \gamma),
	\]
	we need to prove that $H^i(\ker \gamma)=0$ for all $i$. It's clear that $\ker \gamma$ is the following complex:
	\[
		K[1]^\bullet \oplus \ker g=K[1]^\bullet \oplus \im f,
	\]
	and 
	\[
		d=(d_{K[1]}p_{K[1]},K[1]+d_{K}p_K).
	\]
	It's not hard to check that $\chi d+d \chi =\id$ for $\chi^i:K^{i+1}\oplus K^i\to K^i\oplus K^{i-1}$ defined by $\chi=(p_K,0)$. Therefore $H^i(\ker \gamma)=0$ for all $i$ by Lemma \ref{lem:1.7}.
\end{proof}

Therefore, all homological information of $0\to K^\bullet \xrightarrow{f}L^\bullet \xrightarrow{g}M^\bullet\to 0$ is contained in another short exact sequence $0\to K^\bullet \xrightarrow{\bar f}\operatorname{Cyl}(f) \xrightarrow{\pi}\Cone(f)\to 0$. For induced long exact sequence of homology, we can say even more.

\begin{lem}\label{lem:1.14}
	In eq.\eqref{tri}, the short exact in the middle row induce a long exact sequence
	\[
		\cdots\to H^n(K^\bullet)\xrightarrow{H^n(\bar f)}H^n(\Cyl(f))\xrightarrow{H^n(\pi)}H^n(\Cone(f))\xrightarrow{\partial^n}H^{n+1}(K^\bullet)\to\cdots,
	\]
	then $\partial^n=H^n(\delta(f))$.
\end{lem}

\begin{proof}
	It's direct.
\end{proof}

\begin{coro}
	$\Cone(f)$ is acyclic (i.e. exact) iff that $f$ is a quasi-isomorphism.
\end{coro}

\begin{proof}
	If $H^n(\Cone(f))=0$ for all $n$, then the long exact sequence above tells us that $H^n(\bar f)$ is an isomorphism for all $n$. Thus, $H^n(f)=H^n(\beta)H^n(\bar f)$ is an isomorphism for all $n$ because $\beta$ is a quasi-isomorphism.

	Conversely, since $f$ is a quasi-isomorphism, $H^n(\bar f)=(H^n(\beta))^{-1}H^n(f)$ is an isomorphism for all $n$. Therefore, 
	\[
		\ker H^n(\pi)=\im H^n(\bar f)=H^n(\Cyl(f)),\quad 
		\im H^{n-1}(\delta)=\ker H^n(\bar f)=0,
	\]
	and then $H^n(\pi)=0=H^{n}(\delta)$ for all $n$. Thus 
	\[
		H^n(\Cone(f))=\ker H^n(\delta)=\im H^n(\pi)=0
	\]
	for all $n$.
\end{proof}

\begin{definition}
	A triangle is a diagram of the form
	\[
		K^\bullet \xrightarrow{u}L^\bullet\xrightarrow{v}M^\bullet 
		\xrightarrow{w}K[1]^\bullet.
	\]
	A morphism of triangles is a commutative diagram of the form
	\[
		\xymatrix{
		K^\bullet \ar@{->}[r]^{u} \ar@{->}[d]^{f} & L^\bullet \ar@{->}[r]^{v} \ar@{->}[d]^{g} & M^\bullet \ar@{->}[r]^{w} \ar@{->}[d]^{h} & K[1]^\bullet \ar@{->}[d]^{{f[1]}} \\
		K_1^\bullet \ar@{->}[r]^{u_1} & L_1^\bullet \ar@{->}[r]^{v_1} & M_1^\bullet \ar@{->}[r]^{w_1} & K_1[1]^\bullet
		}
	\]
	A triangle is said to be \textit{distinguished} if it is isomorphic to the middle row 
	\[
		K^\bullet \xrightarrow{\bar f}\operatorname{Cyl}(f) \xrightarrow{\pi}\Cone(f)\xrightarrow{\delta(f)}K[1]^\bullet
	\]
	of some diagram of the form eq.\eqref{tri}.
\end{definition}

\begin{pro}
	A distinguished triangle $K^\bullet \xrightarrow{u}L^\bullet\xrightarrow{v}M^\bullet 
	\xrightarrow{w}K[1]^\bullet$ induces a long exact sequence
	\[
		\cdots\to H^n(K^\bullet)\xrightarrow{H^n(u)}H^n(L^\bullet)\xrightarrow{H^n(v)}H^n(M^\bullet)\xrightarrow{H^n(w)}H^{n+1}(K^\bullet)\xrightarrow{H^{n+1}(u)}H^{n+1}(L^\bullet)\to\cdots.
	\]
\end{pro}

\begin{proof}
	It's the tautology of Lemma \ref{lem:1.14}.
\end{proof}

Notice that a distinguished triangle does not need to come from a short exact sequence of complexes like in the Proposition \ref{pro:1.13}, so the last proposition is a generalization of Proposition \ref{pro:1.9}.

\section{Derived Category and Derived Functor}

\begin{definition}
	Let $A$ be an abelian category, $\operatorname{Kom}(A)$ the 
	category of complexes over $A$. There exists a category 
	$D(A)$ and a functor $Q:\operatorname{Kom}(A)\to D(A)$
	with the following properties:
	\begin{enumerate}
		\item $Q(f)$ is an isomorphism for any quasi-isomorphism $f$.
		\item Any functor $F:\operatorname{Kom}(A)\to C$ transforming
			quasi-isomorphisms into isomorphisms can be uniquely
			factorized through $D(A)$, i.e. there exists a unique 
			functor $G:D(A)\to C$ such that $F=GQ$.
	\end{enumerate}
	The category $D(A)$ is called the derived category of the abelian
	category $A$. We can similarly define the derived category $D^+(A)$, $D^-(A)$ and 
	$D^b(A)$, they are subcategories of $D(A)$.
\end{definition}

The existance of derived category can be carried out by introducing 
the formal inverses of the equivalence class of quasi-isomorphisms 
(modulo homotopy equivalence), which is called the 
localization of a category. We will not go into details of this proof.
Instead, we will describe the category $D(A)$ directly.

\begin{definition}[Homotopy Category]
	Suppose $A$ be an abelian category. The homotopy category $K(A)$ is defined as follows:
	\[
		\begin{aligned}
			\text{Obj $K(A)$} &= \text{Obj $\operatorname{Kom}(A)$}\\
			\text{Mor $K(A)$} &= \text{Mor $\operatorname{Kom}(A)$ modulo homotopy equivalence}
		\end{aligned}
	\]
	By $K^+(A)$, $K^-(A)$, $K^b(A)$ we denote full subcategories of $K(A)$ formed by complexes satisfying the corresponding boundness conditions.
\end{definition}

\begin{lem}\label{lem:1.21}
	Suppose $f:K^\bullet \to L^\bullet$ is a quasi-isomorphism in $K(A)$ and $g:M^\bullet \to L^\bullet$ is a morphism, then there exist a complex $N^\bullet$, morphism $h:N^\bullet\to K^\bullet$ and a quasi-isomorphism $k:N^\bullet \to M^\bullet$ such that $gk=fh$, i.e. the following diagram is commutative.
	\[
		\xymatrix{
		N^\bullet \ar@{-->}[rr]^{k} \ar@{-->}[d]_{h} &  & M^\bullet \ar@{->}[d]^{g} \\
		K^\bullet \ar@{->}[rr]^{f} &  & L^\bullet
		}
	\]
\end{lem}

\begin{lem}\label{lem:1.22}
	Suppose $f$, $g$ are morphisms from $K^\bullet$ to $L^\bullet$, then existence of quasi-isomorphism $s$ with $sf\sim sg$ is equivalent to the existance of quasi-isomorphism $t$ with $ft\sim gt$.
\end{lem}

\begin{proof}
	We can assume without loss of generality that $g=0$. Suppose $\{h^i:K^i\to M^{i-1}\}$ is a homotopy between $sf$ and $0$. Now consider the following diagram in $K(A)$
	\[
		\xymatrix{
\Cone(l)[-1] \ar@{->}[rr]^-{\delta(l)[-1]}\ar[d]^0 && K^\bullet \ar@{->}[d]^{f} \ar@{->}[lld]_{l} &  \\
\Cone(s)[-1] \ar@{->}[rr]^-{{\delta(s)[-1]}} && L^\bullet \ar@{->}[r]^{s} & M^\bullet
}
	\]
	where $l:K^\bullet \to \Cone(s)[-1]=L^\bullet \oplus M^\bullet[-1]$ is defined by $l^i=(f^i,-h^i)$. It's clear that $l$ is a morphism, and $f=\delta(s)[-1]l$ is clear from definitions. Now $l\delta(l)[-1]\sim 0$ (from Lemma \ref{lem:1.13}) imples that $l\delta(l)[-1] = 0$ in $K(A)$ and the commutativitiy of the diagram above. Finally, we know that 
	\[
		\Cone(\delta(l)[-1])=\Cyl(l) \sim \Cone(s)[-1]
	\]
	from Proposition \ref{pro:1.12}, and $\Cone(s)[-1]$ is exact from Lemma \ref{lem:1.14} and the fact that $s$ is a quasi-isomorphism, so $\delta(l)[-1]$ is a quasi-isomorphism from Lemma \ref{lem:1.14} again. Since $f\delta(l)[-1]=0$, it is the wanted quasi-isomorphism. In the same way we can prove that $ft\sim 0$ imples the existance of a quasi-isomorphism $s$ such that $sf\sim 0$.
\end{proof}

\begin{para}[A Realization of Derived Category]
	Suppose $A$ is an abelian category, $K(A)$ is its homology category. Objects of derived category $D(A)$ are objects of $K(A)$. 
	\begin{compactenum}[\quad (1)]
		\item A morphism $X\to Y$ in $D(A)$ is the equivalence class of ``roofs'', i.e. a diagram $(s,f)$ in $K(A)$ of the form $X\xleftarrow{s}X'\xrightarrow{f}Y$, where $s$ is a quasi-isomorphism and $f$ is a morphism.
		Two roofs are equivalent, $(s,f)\sim (t,g)$, iff there exists a third roof forming a commutative diagram of the form
		\[
			\xymatrix{
				&  & X''' \ar@{->}[rd]^{h} \ar@{->}[ld]_{r} &  &  \\
				& X' \ar@{->}[ld]_{s} \ar@{->}[rrrd]^{f} &  & X'' \ar@{->}[rd]^{g} \ar@{->}[llld]_{t} &  \\
				X &  &  &  & Y
			}
		\]
		\item The composition of morphisms represented by the roofs $(s,f)$ and $(t,g)$ is the class of the roof $(st',gf')$ obtained using Lemma \ref{lem:1.21}:
		\[
			\xymatrix{
			 &  & X'' \ar@{-->}[ld]_{t'} \ar@{-->}[rd]^{f'} &  &  \\
			 & X' \ar@{->}[ld]_{s} \ar@{->}[rd]^{f} &  & Y' \ar@{->}[ld]_{t} \ar@{->}[rd]^{g} &  \\
			X &  & Y &  & Z
			}
		\]
		One can check that the composition is well defined, and the identity morphism $\id:X\to X$ is a class of the roof $(\id_X,\id_X)$.
		\item $D(A)$ is a additive category. Suppose $\varphi$, $\varphi':X\to Y$ are morphisms in $D(A)$, they are represented by roofs $(s,f)$ and $(s',f')$: 
		\[
			\xymatrix{
				& Z \ar@{->}[ld]_{s} \ar@{->}[rd]^{f} &  \\
				X &  & Y
				}
			\quad
			\xymatrix{
				& Z' \ar@{->}[ld]_{s'} \ar@{->}[rd]^{f'} &  \\
				X &  & Y
			}
		\]
		Then we can use Lemma \ref{lem:1.21} to find a commutative diagram
		\[
			\xymatrix{
				U \ar@{-->}[rr]^{r} \ar@{-->}[d]_{r} &  & Z' \ar@{->}[d]^{s'} \\
				Z \ar@{->}[rr]^{s} &  & X
				}
		\]
		where $r$, $r'$, $s$ and $s'$ are quasi-isomorphisms, and get two new roofs 
		\[
			\xymatrix{
				& U \ar@{->}[ld]_{t} \ar@{->}[rd]^{fr} &  \\
				X &  & Y
				}
			\quad 
			\xymatrix{
				& U \ar@{->}[ld]_{t} \ar@{->}[rd]^{f'r'} &  \\
				X &  & Y
				}
		\]
		to represent $\varphi$ and $\varphi'$, where $t=sr=s'r'$. Finally, we define that $\varphi+\varphi'$ is the morphism represented by the roof $(t,fr+f'r')$. 
	\item We can embed a morphism $f$ in $K(A)$ into $D(A)$ by setting the corresponding morphism is represented by the roof $(\id,f)$. Therefore, a morphism $f$ in $K(A)$ is zero morphism in $D(A)$ iff there exists quasi-isomorphism $s$ or $t$ such that $sf=0$ or $ft=0$ in $K(A)$. (They are equivalent because Lemma \ref{lem:1.22}.)
	\end{compactenum}
\end{para}

\begin{para}[Cyclic Complex]
	A complex $K^\bullet$ is cyclic if $d_K^n=0$ for any $n$. 
	Cyclic complexes form a subcategory $\operatorname{Kom}_0(A)
	\subset \operatorname{Kom}(A)$. Now let's consider the 
	homology functor
	\[
		h:\operatorname{Kom}(A)\to \operatorname{Kom}_0(A)
	\]
	defined by 
	\[
		h((K^\bullet,d_K))=(H^\bullet(K),0),\quad 
		h(f)=H^\bullet(f).
	\]
	It transforms quasi-isomorphisms to isomorphisms. Therefore,
	there exists a functor 
	\[
		k:D(A)\to \operatorname{Kom}_0(A).
	\]
\end{para}

\begin{pro}
	Suppose $A$ is an abelian category, then 
	the functor $D(A)\to \operatorname{Kom}_0(A)$ defined above is 
	an equivalence iff $A$ is semisimple, i.e. any exact triple 
	in $A$ splits.
\end{pro}
