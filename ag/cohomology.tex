\chapter{上同调}

\section{同调代数}

在这节中,我们将回忆一些同调代数的基本概念。同调代数的历史,可以追溯到
Hilbert对多项式系数的线性齐次方程组的求解上。他考虑了如下方程组
\[
    \sum_{j=1}^ma_{ij}x_j=0,\quad i=1,\dots,n,\quad 
    a_{ij}\in k[t_1,\dots,t_r],
\]
其中$k$是一个域,这可以理解为求一个$k[t_1,\dots,t_r]$-模的同态
\[
    \epsilon:E\to 0
\]
的核$\ker \epsilon$. 
于是有限生成$k[t_1,\dots,t_r]$-模也不一定是自由的,所以即使找到了一组
$\ker \epsilon$的生成元,他们之间也是存在关系的。找到一组生成元可以理解为,
找到了一个自由$k[t_1,\dots,t_r]$-模$F^0$,以及一个满同态
\[
    d^{-1}:F^0\to \ker \epsilon,
\]
我们可以进一步将其嵌入到$E$里,即写作$d^{-1}:F^0\to E$.
而理解生成元的关系,就是要计算$\ker(d^{-1})$. 为了计算这个核,还是找一组
生成元,然后去理解他的关系,于是,我们可以写出序列
\[
    \cdots\xrightarrow{d^{-n-2}}F^{-n}\xrightarrow{d^{-n-1}}\cdots \xrightarrow{d^{-3}}F^{-1}\xrightarrow{d^{-2}}F^0\xrightarrow{d^{-1}}E\xrightarrow{d^0:=\epsilon} 0,
\]
不难检查$d^{-k}d^{-k-1}=0$. 这就构成了所谓的一组链复形。Hilbert证明了,
对特定的环,这样的链复形会终止,即存在一个$n$使得,$F^{k}=0$
对所有的$k<-n$. 对$k[t_1,\dots,t_r]$,$n$最大是$r$,特别地,如果环
是一个域,则$F^{-1}=0$,这就回到了线性代数中熟悉的情况。

Hilbert的这个想法,一方面诱导了链复形的定义,另一方面也展现了利用一个链复形
来研究某个对象或态射的思路。在上面中,我们看到所有的$F^{n}$都是自由摸,
这样的链复形
\[
    \cdots\xrightarrow{d^{-n-2}}F^{-n}\xrightarrow{d^{-n-1}}\cdots \xrightarrow{d^{-3}}F^{-1}\xrightarrow{d^{-2}}F^0\xrightarrow{d^{-1}}E
\]
被称为$E$的一个自由消解。类似地,还可以定义所谓的投射、内射消解,
而同调代数的基本工具之一,就是用消解来获取一些对象或态射的基本信息。



\subsection{链复形}

\begin{para}[链复形]
	设 $A$ 是一个Abel范畴,链复形 $K^\bullet$ 是一族 $K^i\in A$ 以及同态 
    $d^i_K:K^{i-1}\to K^i$ 使得 $d_K^{i+1}d_K^i=0$.
\end{para}

链复形常被简称为复形。我们可以将复形$K^\bullet$用图
\[
	\cdots  \xrightarrow{d_K^{i-2}} K^{i-1} \xrightarrow{d_K^{i-1}}K^i \xrightarrow{d_K^{i}}
	K^{i+1}\xrightarrow{d^{i+1}}\cdots
\]
来表示。当$d_K^{i}$的指标自明时,我们可以简写为$d$. 比方说,$d^2=0$
是$d_K^{i+1}d_K^i=0$的简写。

\begin{para}[链复形的态射]
	设$K^\bullet$, $L^\bullet$是两个Abel范畴$A$中的复形,态射族
    $f=\{f^i:K^i\to L^i\}_i$被称为链复形$K^\bullet$和$L^\bullet$间的态射,
    如果对任意的$i$都有$f^{i+1}d^i_K=d^i_Lf^i$.
\end{para}

态射$f$常被写作$f:K^\bullet\to L^\bullet$,且可以用交换图表示
\[
	\xymatrix{
		\cdots \ar[r]& K^{i-1}\ar[r]^{d^{i-1}_K}\ar[d]^{f^{i-1}}&K^i\ar[r]^{d^{i}_K}\ar[d]^{f^{i}}&K^{i+1}\ar[r]^{d^{i+1}_K}\ar[d]^{f^{i+1}}&\cdots\\
		\cdots \ar[r]& L^{i-1}\ar[r]^{d^{i-1}_L}&L^i\ar[r]^{d^{i}_L}&L^{i+1}\ar[r]^{d^{i+1}_L}&\cdots
	}
\]

Abel范畴$A$上的复形和态射构成了一个范畴,我们将其记作$\operatorname{Kom}(A)$.
在实际应用中,我们经常考虑带有各种有限性条件的复形。比如,我们可以定义
\[
\begin{aligned}
	&\operatorname{Kom}^+(A):K^i=0\quad \text{for}\quad i< i_0(K^\bullet),\\
	&\operatorname{Kom}^-(A):K^i=0\quad \text{for}\quad i> i_0(K^\bullet),\\
	&\operatorname{Kom}^b(A)=\operatorname{Kom}^+(A)\cap \operatorname{Kom}^-(A),\\
\end{aligned}
\]
其中$i_0(K^\bullet)$是一个整数。显然,$\operatorname{Kom}^+(A)$, 
$\operatorname{Kom}^-(A)$和$\operatorname{Kom}^b(A)$ 都是 
$\operatorname{Kom}(A)$ 的子范畴。

\begin{para}[平移]
	设$K^\bullet \in \operatorname{Kom}(A)$是一个复形,则将其平移$n$
    位的复形$K[n]^\bullet$定义为
	\[
		(K[n])^i=K^{n+i},\quad d_{K[n]}^i=(-1)^n d_K^{n+i}.
	\]
	同样,可以自然定义 $f[n]:K[n]\to L[n]$ 通过 
	$(f[n])^i=f^{n+i}$.
\end{para}

于是,
\[
	T^n:K^\bullet \mapsto K[n]^\bullet, \quad 
	T^n:f\mapsto f[n]
\]
构成了一个函子$T^n:\operatorname{Kom}(A)\to \operatorname{Kom}(A)$,
并且显然是一个范畴等价。由于这种等价,我们常可以对单个
$K^\bullet \in \operatorname{Kom}^+(A)$ 设 $i_0(K^\bullet)=0$.

\begin{para}[双重链复形]
双重链复形$K^{\bullet,\bullet}$是有两个指标的链复形,他有两个微分算符
$d:K^{m,n}\to K^{m+1,n}$和$\delta:K^{m,n}\to K^{m,n+1}$,满足
$d^2=0$, $\delta^2=0$ 以及 $d\delta+\delta d=0$. 对一个双重链复形$K^{\bullet,\bullet}$,
我们可以定义一个链复形$|K|^\bullet$,
\[
    |K|^p=\bigoplus_{m+n=p}K^{m,n},\quad d=d+\delta,
\]
不难看到$d^2=(d+\delta)^2=d^2+d\delta+\delta d+\delta^2=0$. $|K|^\bullet$常被
称为双重链复形$K^{\bullet,\bullet}$的整体化。
\end{para}

\begin{para}[映射锥]
    设$f:K^\bullet\to L^\bullet$是一个态射,我们可以定义一个双重链复形
    $C(f)^{\bullet,\bullet}$:
    \[
        C(f)^{i,-1}=K^i,\quad C(f)^{i,0}=L^i,\quad 
        C(f)^{i,j}=0\,\,\text{for $j>0$},\quad 
        \delta^i=(-1)^i f_i:C(f)^{i,-1}\to C(f)^{i,0},
    \]
    不难看到$d\delta +\delta d=0$在这里即态射定义中的$fd_K=d_Lf$.
    他的整体化$|C(f)|^\bullet$被称为$f$的映射锥,记作$\operatorname{Cone}(f)$.
\end{para}

映射锥的名字来自于拓扑,并不指这里的$f$是一个映射。
我们也可以直接写出映射锥$\operatorname{Cone}(f)$的定义:
\[
    (\operatorname{Cone}(f))^i=K[1]^i\oplus L^i=K^{i+1}\oplus L^i,\quad
    d_{\operatorname{Cone}(f)}=\begin{pmatrix}
        d_{K[1]}&0\\
        f[1]&d_L
    \end{pmatrix}.
\]
容易验证
\[
	d_{\operatorname{Cone}(f)}^2=\begin{pmatrix}
		d_{K[1]}&0\\
		f[1]&d_L
	\end{pmatrix}\begin{pmatrix}
		d_{K[1]}&0\\
		f[1]&d_L
	\end{pmatrix}
	=\begin{pmatrix}
		0&0\\
		f[1]d_{K[1]}+d_Lf[1]&0
	\end{pmatrix}
	=0,
\]
因为
\[
	(f[1]d_{K[1]}+d_Lf[1])^i =
	-f^{i+2}d_K^{i+1}+d_L^{i+1} f^{i+1}=0.
\]
利用Abel范畴的双积,我们不难看到,存在$K[1]^\bullet$, $L^\bullet$和
$\operatorname{Cone}(f)^\bullet$之间的典范态射。特别地,我们有一个
短正和列
\[
    0\to L^\bullet\to \operatorname{Cone}(f)^\bullet\to K[1]^\bullet \to 0,
\]
其中的态射都由双积给出。

\begin{para}[映射柱]
	令$f:K^\bullet\to L^\bullet$为一个态射,$f$的态射柱即双积的态射
	\[
		\delta(f)[-1]:\operatorname{Cone}(f)[-1]^\bullet \to K^\bullet
	\]
    的映射锥,换言之,即如下复形$\operatorname{Cyl}(f)$:
	\[
		\operatorname{Cyl}(f)^i=\operatorname{Cone}(f)^i\oplus K^{i} ,\quad
		d_{\operatorname{Cyl}(f)}=\begin{pmatrix}
			d_{\operatorname{Cone}(f)}&0\\
			\delta(f) & d_{K}
		\end{pmatrix}.
	\]
\end{para}

% 在代数拓扑中,一个连续映射$\varphi:X\to Y$的映射柱 $\operatorname{cyl}(\varphi)$ 
% 是如下图的余极限
% \[
% 	X\times [0,1] \xleftarrow{i} X \xrightarrow{\varphi} Y,
% \]
% 其中$i:X\hookrightarrow X\times [0,1]$是典范含入$x\mapsto (x,0)$. 
% 则映射柱的泛性质为,对任意的空间$Z$和连续映射$g_1:X\times [0,1]\to Z$, $g_2:Y\to Z$,
% 若$g_1(x,0)=g_2(f(x))$对任意的$x$都成立,则存在唯一的$k:\operatorname{cyl}(\varphi)\to Z$,
% 使得如下图是交换的
% \[
% 	\xymatrix{
% 		Z &  & \\
% 		  & \operatorname{cyl}(\varphi)\ar@{-->}[ul]_k & Y\ar[l]\ar@/_/[llu]_{g_2}\\
% 		  & X\times [0,1]\ar[u]\ar@/^/[luu]^{g_1} & X\ar[u]\ar[l]
% 	}
% \]
% 这里考虑$X\times [0,1]$是自然的,这就来自于同伦的定义。这里,
% 典范映射$Y\to \operatorname{cyl}(\varphi)$还将给出$Y$和$\operatorname{cyl}(\varphi)$
% 之间的同伦等价。特别地,任何连续映射可以通过这个映射分解为其与一个形变收缩的复合。
% %every continuous function factors as a map into its mapping cylinder followed by a deformation retraction.

% 在同调代数中,我们也将看到映射锥有类似的性质。实际上,我们可以在任何的Abel范畴中,
% 找到一个类似于拓扑空间范畴中闭区间$[0,1]$的对象$I$,使得
% 态射$f:K^\bullet \to L^\bullet$ 的映射锥即如下图的余极限
% \[
% 	K^\bullet\otimes I^\bullet \xleftarrow{i}K^\bullet \xrightarrow{f}L^\bullet.
% \]


\begin{para}[上同调]
设$K^\bullet$是一个复形,则对
$K^{i-1}\xrightarrow{d^{i-1}}K^i\xrightarrow{d^i}K^{i+1}$存在态射$a$ 和 $b$
使得如下图是交换的,
\[
	\begin{xy}
		\xymatrix{
			&&\cdot \ar[rrd]^-{b}&&\\
			K^{i-1} \ar[rr]^-{d^{i-1}} \ar[rrd]_{a}&& K^i \ar[rr]^-{d^i}\ar[u]^-{\coker d^{i-1}}&&K^{i+1}\\
			&&\cdot \ar[u]_-{\ker d^i}&&
		}
	\end{xy}
\]
且在$\coker(a):\cdot\to P$和$\ker(b):r\to \cdot Q$中$P$和$Q$自然同构
(当然也可以取成一样的),于是我们定义$H^i(K^\bullet):=P$.
\end{para}

定义的自洽性不过是简单的范畴论验证。如果$A$是模范畴,则回归到经典的定义:
\[
	H^i(K^\bullet)=\ker d^i/\im d^{i-1}.
\]

\begin{pro}
	设 $f:K^\bullet \to L^\bullet$ 是一个链复形态射,则其诱导了
    态射$\hat f$, $\bar f$ 以及 $H^\bullet(f)$ 使得如下图交换
	\[
		\xymatrix{
			\ker d_K^i\ar[r]\ar@{-->}[d]^{\hat{f}^i}&K^i\ar[r]
			\ar[d]^{f^i}&\coker d^{i-1}_K\ar@{-->}[d]^{\bar{f}^i}
			\\
			\ker d_L^i\ar[r]&L^i\ar[r]&\coker d^{i-1}_L
		}
		\quad 
		\xymatrix{
			\ker d_K^i\ar[rr]^{\coim \varphi^i_K}\ar@{-->}[d]^{\hat{f}^i}&&H^i(K^\bullet)\ar[r]^{\im\varphi^i_K}
			\ar[d]^{H^i(f)}&\coker d^{i-1}_K\ar@{-->}[d]^{\bar{f}^i}\\
			\ker d_L^i\ar[rr]^{\coim \varphi^i_L}&&H^i(L^\bullet)\ar[r]^{\im\varphi^i_L}&\coker d^{i-1}_L
		}
	\]
	其中 $\varphi^i=\coker d^{i-1}\ker d^i$.
\end{pro}

\begin{proof}
	还是一个泛性质的简单练习。由于
	$f^{i+1}d_K^i=d_L^if^i$, 则 $d_L^if^i\ker d_K^i=0$, 再从 
    $\ker d_L^i$的泛性质,存在态射 $\hat{f}^i$ 使得
	\[
		f^i\ker d_K^i=\ker d_L^i\hat{f}^i.
	\]
	类似可以得到 $\bar f^i$ 的存在性。 对第二幅图,由于
	\[
	\begin{aligned}
		\coker \varphi^i_L \bar f^i \varphi^i_K
		&= \coker \varphi^i_L \bar f^i \varphi^i_K\\
		&= \coker \varphi^i_L \varphi^i_L \hat f^i \\
		&= 0,
	\end{aligned}
	\]
	$\coker \varphi^i_L \bar f^i \im \varphi^i_K$ 也为零。
    实际上,
	\[
		\coker \varphi^i_L \bar f^i \im \varphi^i_K\operatorname{coim}\varphi^i_K=\coker \varphi^i_L \bar f^i \varphi^i_K=0
	\]
	以及 $\operatorname{coim}\varphi^i_K$ 是一个满态射。
    因此,泛性质
	$\ker \coker \varphi_K^i=\im \varphi_K^i$ 给出态射 $H^i(f)$ 使得
	\[
		\bar f^i \im \varphi^i_K=\im \varphi_L^i H^i(f).
	\]
	最后还需说明 $\coim \varphi_L^i\hat f^i= H^i(f)\coim \varphi_K^i$,
    这来自于
	\[
		\im \varphi_L^i\coim \varphi_L^i\hat f^i=\varphi^i_L\hat f^i=\bar f^i \varphi^i_K=
		\im \varphi_L^i H^i(f)\coim \varphi^i_K
	\]
	以及 $\im \varphi_L^i$ 是单态射。
\end{proof}

如今我们已经定义了,
\[
	H^n :K^\bullet\mapsto H^n(K^\bullet)\quad \text{and}\quad
	H^n :f\mapsto H^n(f),
\]
不难看到 $H^n$ 对任意的$n$都构成了一个函子,被称为上同调函子。

\begin{para}[链同伦]
	若 $f$, $g:K^\bullet\to L^\bullet$是两个态射,如果存在
	一系列态射 $s^i:K^i\to L^{i-1}$ 使得
	\[
		f^i-g^i=s^{i+1}d_K^i+d_L^{i-1}s^i,
	\]
	则$f$和$g$被称为(链)同伦的,记作 $f\sim g$.
\end{para}

不难看到,$f\sim g$等价于$f-g\sim 0$.

\begin{lem}\label{lem:1.7}
	若$f$, $g:K^\bullet\to L^\bullet$是同伦的,则
	$H^\bullet(f)=H^\bullet(g)$.
\end{lem}

\begin{proof}
	不妨设$g=0$以及$H^\bullet(g)=0$. 我们将证明
	\[
		\bar f^i \varphi_K^i = \im \varphi_L^i H^i(f) \operatorname{coim} \varphi_K^i=0.
	\]
	实际上,
	\[
		\begin{aligned}
		\bar f^i \varphi_K^i=\bar f^i \coker d_K^{i-1}\ker d_{K}^i 
		&=\coker d_L^{i-1} f^i\ker d_{K}^i\\
		&=\coker d_L^{i-1}(s^{i+1}d_K^i+d_L^{i-1}s^i)\ker d_{K}^i\\
		&=\coker d_L^{i-1}s^{i+1}d_K^i\ker d_{K}^i+\coker d_L^{i-1}d_L^{i-1}s^i\ker d_{K}^i\\
		&=0.
		\end{aligned}
	\]
	因为 $\im \varphi_L^i$ 是单的,而
	$\operatorname{coim} \varphi_K^i$ 是满的,故$H^i(f) = 0$.
\end{proof}

\begin{para}[拟同构]
	态射$f:K^\bullet\to L^\bullet$ 被称为拟同构的,如果存在
     $H^n(f):H^n(K^\bullet)\to H^n(L^\bullet)$ 对任意$n$都是同构。
\end{para}

\begin{pro}\label{pro:1.9}
	链复形的短正合列 $0\to K^\bullet\xrightarrow{f}L^\bullet\xrightarrow{g}M^\bullet\to 0$ 
    诱导了一个上同调的正合列
	\[
		\cdots\to H^n(K^\bullet)\xrightarrow{H^n(f)}H^n(L^\bullet)\xrightarrow{H^n(g)}H^n(M^\bullet)\xrightarrow{\partial^n}H^{n+1}(K^\bullet)\xrightarrow{H^{n+1}(f)}H^{n+1}(L^\bullet)\to\cdots.
	\]
\end{pro}

在模范畴, $\partial^n:H^n(M^\bullet)\to H^{n+1}(K^\bullet)$ 可以如下计算
\[
	\partial^n([c])=\left[(f^{n+1})^{-1}\left(\dd^n_L\left((g^n)^{-1}(c)\right)\right)\right].
\]

\begin{proof}
	从蛇引理,对任意的$n$,存在正合列
	\[
		0\to \ker d_K^n\xrightarrow{\hat f^n} \ker d_L^n
		\xrightarrow{\hat g^n}
		\ker d_M^n \xrightarrow{\delta^n}
		\coker d_K^n\xrightarrow{\bar f^{n+1}} \coker d_L^n\xrightarrow{\bar g^{n+1}} \coker d_M^n \to 0.
	\]
	剩下部分是容易的。
\end{proof}

% \[
% 	\xymatrix{
% 	&0\ar[d]&0\ar[d]&0\ar[d]&\\
% 	\cdots\ar[r]&K^{n-1}\ar[r]^{\dd_K^{n-1}}\ar[d]^{f^{n-1}}&K^n\ar[r]^{\dd_K^n}\ar[d]^{f^{n}}&K^{n+1}\ar[r]\ar[d]^{f^{n+1}}&\cdots\\
% 	\cdots\ar[r]&L^{n-1}\ar[r]^{\dd_L^{n-1}}\ar[d]^{g^{n-1}}&L^n\ar[r]^{\dd_L^n}\ar[d]^{g^{n}}&L^{n+1}\ar[r]\ar[d]^{g^{n+1}}&\cdots\\
% 	\cdots\ar[r]&M^{n-1}\ar[r]^{\dd_L^{n-1}}\ar[d]&M^n\ar[r]^{\dd_L^n}\ar[d]\ar[ruu]&M^{n+1}\ar[r]\ar[d]&\cdots\\
% 	&0&0&0&
% 	}
% \]


\begin{pro}\label{pro:1.12}
	对态射$f:K^\bullet \to L^\bullet$,存在如下的交换图,其行是正和的:
	\begin{equation}\label{tri}
		\xymatrix{
			& 0\ar[r] & L^\bullet \ar[r]^-{\bar\pi}\ar[d]^\alpha & \Cone(f)^\bullet \ar[r]^-{\delta(f)}\ar@{=}[d]
			& K[1]^\bullet \ar[r] & 0\\
			0 \ar[r] & K^\bullet \ar[r]^-{\bar f}\ar@{=}[d] & \operatorname{Cyl}(f)^\bullet \ar[r]^-\pi\ar[d]^\beta 
			&\Cone(f)^\bullet \ar[r] & 0 & \\
			& K^\bullet \ar[r]^f & L^\bullet & & &
		}
	\end{equation}
	其中
	\[
		\begin{aligned}
		&\bar f = (0,0,\id_{K^i}),\quad\bar\pi^i=(0,\id_{L^i}),\quad \pi^i=(p^i_{K[1]},p^i_{L})\\
		&\alpha^i = (0,\id_{L^i},0),\quad \beta^i= p^i_{L} + f^i p^i_K,\quad \delta(f)=q_{K[1]}
		\end{aligned}
	\]
	以及$p_{K[1]}$, $p_{L}$, $p_{K}$, $q_{K[1]}$和$q_L$都是双积给出的典范态射,
    于是$\beta\alpha=\id_L$以及$\alpha\beta\sim \id_{\operatorname{Cyl}(f)}$.
	因此,$\alpha$和$\beta$是$L^\bullet$和$\operatorname{Cyl}(f)^\bullet$之间的拟同构。
\end{pro}

在代数拓扑中,$f:X\to Y$的映射锥和$Y$同伦等价。

\begin{proof}
	证明是直接的。
\end{proof}

\begin{lem}\label{lem:1.13}
	设$f:K^\bullet\to L^\bullet$是一个态射,则$f\delta(f)[-1]\sim 0$.
\end{lem}

\begin{proof}
	设
	\[
		p^i:(\Cone(f)[-1])^i=K^i\oplus L^{i-1}\to L^{i-1}
	\]
	是积的典范态射,现在我们有
	\[
		\begin{aligned}
			p^{i+1}d^i_{\Cone(f)[-1]}+d^{i-1}_{L}p^i
			&=-p^{i+1}d^{i-1}_{\Cone(l)}+d^{i-1}_{L}p^i\\
			&=-p^{i+1}\begin{pmatrix}
				-d^{i}_{K}&0\\
				f^i&d_{L}^{i-1}
			\end{pmatrix}+d^{i-1}_{L}p^i\\
			&=-f^i\delta(f)^{i-1}-d_{L}^{i-1}p^i+d^{i-1}_{L}p^i\\
			&=-f^i\delta(f)^{i-1},
		\end{aligned}
	\]
	于是$f\delta(f)[-1] \sim 0$.
\end{proof}

\begin{pro}\label{pro:1.13}
	设$0\to K^\bullet \xrightarrow{f}L^\bullet \xrightarrow{g}M^\bullet\to 0$
    是一个短正合列,则其拟同构于\eqref{tri}式中的中间那行,即存在态射
    $\alpha$, $\beta$, $\gamma$ 使得如下图交换
	\[
		\xymatrix{
		0 \ar@{->}[r] & K^\bullet \ar@{->}[r]^{f} & L^\bullet \ar@{->}[r]^{g} & M^\bullet \ar@{->}[r] & 0 \\
		0 \ar@{->}[r] & K^\bullet \ar@{->}[r]^{\bar f} \ar@{->}[u]_{\alpha} & \operatorname{Cyl}(f) \ar@{->}[r]^{\pi} \ar@{->}[u]_{\beta} & \operatorname{Cone}(f) \ar@{->}[r] \ar@{->}[u]_{\gamma} & 0
		}
	\]
	其中$\bar f$ 和 $\pi$ 定义于 Proposition \ref{pro:1.12}.
\end{pro}

\begin{proof}
	我们还是使用 Proposition \ref{pro:1.12} 的记号。取
	\[
		\alpha=\id,\quad \beta^i= p^i_{L} + f^i p^i_K,\quad \gamma^i=gq_L^i.
	\]
	容易看到$\gamma$是一个态射,且 
	\[
		\gamma\pi=gq_L\pi=gp_L,\quad g\beta=gp_L+gfp_K=gp_L,
	\]
	故上图交换。

	如 Proposition \ref{pro:1.12} 所示, $\beta$ 是拟同构,我们只需要
    证明$\gamma$是拟同构。首先,注意到$\gamma=gq_L$
    是满的,因为$g$和$q_L$都是满的。考虑短正和列
	\[
		0\to \ker \gamma\to \Cone(f)\xrightarrow{f} M^\bullet\to 0,
	\]
	他诱导了长正和列
	\[
		H^n(\ker \gamma)\to H^n(\Cone(f))\xrightarrow{H^n(\gamma)}H^n(M)\to H^{n+1}(\ker \gamma),
	\]
	我需要证明$H^i(\ker \gamma)=0$对任何$i$都对。
    容易看到,$\ker \gamma$就是如下复形
	\[
		K[1]^\bullet \oplus \ker g=K[1]^\bullet \oplus \im f,
	\]
	以及
	\[
		d=(d_{K[1]}p_{K[1]},K[1]+d_{K}p_K).
	\]
	所以不难检查由$\chi=(p_K,0)$定义的$\chi^i:K^{i+1}\oplus K^i\to K^i\oplus K^{i-1}$
    满足$\chi d+d \chi =\id$,这是一个链同伦。因此,对任何的$i$都有$H^i(\ker \gamma)=0$.
\end{proof}

因此,所有$0\to K^\bullet \xrightarrow{f}L^\bullet \xrightarrow{g}M^\bullet\to 0$的
同伦信息都可以从另一个短正合列$0\to K^\bullet \xrightarrow{\bar f}\operatorname{Cyl}(f) \xrightarrow{\pi}\Cone(f)\to 0$得到。对其诱导的长正和列,我们可以说更多。

\begin{lem}\label{lem:1.14}
	我们有长正和列
	\[
		\cdots\to H^n(K^\bullet)\xrightarrow{H^n(\bar f)}H^n(\Cyl(f))\xrightarrow{H^n(\pi)}H^n(\Cone(f))\xrightarrow{\partial^n}H^{n+1}(K^\bullet)\to\cdots,
	\]
	其中$\partial^n=H^n(\delta(f))$.
\end{lem}

\begin{proof}
	直接的。
\end{proof}

\begin{coro}
	链复形$\Cone(f)^\bullet$是正和的当且仅当$f$是拟同构。
\end{coro}

\begin{proof}
	如果 $H^n(\Cone(f)^\bullet)=0$,则上面的长正和列给出$H^n(\bar f)$是同构,
    因此$H^n(f)=H^n(\beta)H^n(\bar f)$也是同构,因为
    $\beta$是拟同构。

	反过来,如果$f$是拟同构,则$H^n(\bar f)=(H^n(\beta))^{-1}H^n(f)$是同构。
    因此,
	\[
		\ker H^n(\pi)=\im H^n(\bar f)=H^n(\Cyl(f)),\quad 
		\im H^{n-1}(\delta)=\ker H^n(\bar f)=0,
	\]
	故$H^n(\pi)=0=H^{n}(\delta)$,以及 
	\[
		H^n(\Cone(f))=\ker H^n(\delta)=\im H^n(\pi)=0.\qedhere
	\]
\end{proof}

\begin{para}[三角]
	一个三角是指如下的一幅图
	\[
		K^\bullet \xrightarrow{u}L^\bullet\xrightarrow{v}M^\bullet 
		\xrightarrow{w}K[1]^\bullet,
	\]
	他们之间的态射被定义为如下的交换图
	\[
		\xymatrix{
		K^\bullet \ar@{->}[r]^{u} \ar@{->}[d]^{f} & L^\bullet \ar@{->}[r]^{v} \ar@{->}[d]^{g} & M^\bullet \ar@{->}[r]^{w} \ar@{->}[d]^{h} & K[1]^\bullet \ar@{->}[d]^{{f[1]}} \\
		K_1^\bullet \ar@{->}[r]^{u_1} & L_1^\bullet \ar@{->}[r]^{v_1} & M_1^\bullet \ar@{->}[r]^{w_1} & K_1[1]^\bullet
		}
	\]
	一个三角被称为是可分的,如果他同构于某个
	\[
		K^\bullet \xrightarrow{\bar f}\operatorname{Cyl}(f)^\bullet \xrightarrow{\pi}\Cone(f)^\bullet\xrightarrow{\delta(f)}K[1]^\bullet.
	\]
\end{para}

\begin{pro}
	可分三角$K^\bullet \xrightarrow{u}L^\bullet\xrightarrow{v}M^\bullet 
	\xrightarrow{w}K[1]^\bullet$诱导了长正和列
    \[
		\cdots\to H^n(K^\bullet)\xrightarrow{H^n(u)}H^n(L^\bullet)\xrightarrow{H^n(v)}H^n(M^\bullet)\xrightarrow{H^n(w)}H^{n+1}(K^\bullet)\xrightarrow{H^{n+1}(u)}H^{n+1}(L^\bullet)\to\cdots.
	\]
\end{pro}

\subsection{导出范畴和导出函子}

\begin{para}
    令$A$是一个Abel范畴,$\operatorname{Kom}(A)$是$A$上的链复形范畴。
    则存在一个范畴$D(A)$以及函子$Q:\operatorname{Kom}(A)\to D(A)$
	使得
	\begin{enumerate}
		\item 对任何拟同构$f$,$Q(f)$都是同构;
		\item 任何将拟同构变成同构的函子$F:\operatorname{Kom}(A)\to C$
            可以唯一地经由$D(A)$分解,即存在唯一的函子$G:D(A)\to C$
            使得$F=GQ$.
	\end{enumerate}
	范畴$D(A)$被称为$A$的导出范畴。我们可以类似地定义适当的有限性链复形范畴
    的相应导出范畴$D^+(A)$, $D^-(A)$以及$D^b(A)$,他们都是$D(A)$的完全子范畴。
\end{para}

导出范畴的存在性可以通过引入拟同构的等价类(模同伦等价的意义上)的形式逆来证明。
这个构造被称为范畴的局部化。我们这里不会给出这个构造的细节,而是直接描述导出范畴。

\begin{para}[同伦范畴]
	同伦范畴$K(A)$由如下的资料定义:
	\[
		\begin{aligned}
			\text{Obj $K(A)$} &= \text{Obj $\operatorname{Kom}(A)$}\\
			\text{Mor $K(A)$} &= \text{Mor $\operatorname{Kom}(A)$ 模去同伦等价}
		\end{aligned}
	\]
	类似地,我们可以定义子范畴$K^+(A)$, $K^-(A)$, $K^b(A)$.
\end{para}

\begin{lem}
    如果$f,g:K^\bullet\to L^\bullet$是同伦的,则在导出范畴$D(A)$中有$Q(f)=Q(g)$.
\end{lem}

\begin{lem}\label{lem:1.21}
	若$f:K^\bullet \to L^\bullet$是$K(A)$中的一个拟同构,而$g:M^\bullet \to L^\bullet$
    是一个态射,则存在一个复形$N^\bullet$,态射$h:N^\bullet\to K^\bullet$和拟同构
    $k:N^\bullet \to M^\bullet$使得$gk=fh$,即使得如下图交换
	\[
		\xymatrix{
		N^\bullet \ar@{-->}[rr]^{k} \ar@{-->}[d]_{h} &  & M^\bullet \ar@{->}[d]^{g} \\
		K^\bullet \ar@{->}[rr]^{f} &  & L^\bullet
		}
	\]
\end{lem}

\begin{lem}\label{lem:1.22}
    假如$f$, $g$都是$K^\bullet$ to $L^\bullet$的态射,则存在拟同构$s$使得$sf\sim sg$当且仅当存在拟同构$t$使得$ft\sim gt$.
\end{lem}

\begin{proof}
	不失一般性,我们可以假设$g=0$. 假设$\{h^i:K^i\to M^{i-1}\}$是$sf$和$0$之间的同伦。
    考虑$K(A)$中的如下图
	\[
		\xymatrix{
\Cone(l)[-1]^\bullet \ar@{->}[rr]^-{\delta(l)[-1]}\ar[d]^0 && K^\bullet \ar@{->}[d]^{f} \ar@{->}[lld]_{l} &  \\
\Cone(s)[-1]^\bullet \ar@{->}[rr]^-{{\delta(s)[-1]}} && L^\bullet \ar@{->}[r]^{s} & M^\bullet
}
	\]
	其中$l:K^\bullet \to \Cone(s)[-1]^\bullet=L^\bullet \oplus M[-1]^\bullet$由$l^i=(f^i,-h^i)$
    所定义。显然, $l$是一个态射,而从定义,$f=\delta(s)[-1]l$. 现在 $l\delta(l)[-1]\sim 0$ 
    告诉我们在$K(A)$中有$l\delta(l)[-1] = 0$以及上图是交换的。最后,我们从
    Proposition \ref{pro:1.12} 知道
	\[
		\Cone(\delta(l)[-1])^\bullet=\Cyl(l)^\bullet \sim \Cone(s)[-1]^\bullet,
	\]
	以及  Lemma \ref{lem:1.14} 告诉我们 $\Cone(s)[-1]$ 是正和的,
    由于 $s$ 是拟同构,则 $\delta(l)[-1]$ 也是拟同构。因为 $f\delta(l)[-1]=0$,
    这就是想要的拟同构$t$. 另一个方向也是类似的。
\end{proof}

\begin{para}[导出范畴的一种实现]
	若$A$是一个Abel范畴,而$K(A)$是同伦范畴。导出范畴$D(A)$的对象即$K(A)$的对象。
	\begin{compactenum}[\quad (1)]
		\item $D(A)$的态射$X\to Y$为“根图”的等价类,所谓“根图”
        为$K(A)$中的图$(s,f)$形如$X\xleftarrow{s}X'\xrightarrow{f}Y$,其中$s$
        是一个拟同构,而$f$是一个同构。我们可以将其理解为$f/s$. 两个
		根图是等价的,$(s,f)\sim (t,g)$,当且仅当存在第三个根图使得如下图交换
		\[
			\xymatrix{
				&  & X''' \ar@{->}[rd]^{h} \ar@{->}[ld]_{r} &  &  \\
				& X' \ar@{->}[ld]_{s} \ar@{->}[rrrd]^{f} &  & X'' \ar@{->}[rd]^{g} \ar@{->}[llld]_{t} &  \\
				X &  &  &  & Y
			}
		\]
        我们可以将其理解为$f/s=g/t$当且仅当他们可以通分为$fr/sr=gh/th$.
		\item 两个根图$(s,f)$和$(t,g)$的复合通过“通分”得到的,利用Lemma \ref{lem:1.21}
		我们可以构造出如下图
		\[
			\xymatrix{
			 &  & X'' \ar@{-->}[ld]_{t'} \ar@{-->}[rd]^{f'} &  &  \\
			 & X' \ar@{->}[ld]_{s} \ar@{->}[rd]^{f} &  & Y' \ar@{->}[ld]_{t} \ar@{->}[rd]^{g} &  \\
			X &  & Y &  & Z
			}
		\]
        而$(st',gf')$对应的等价类就是$(s,f)$和$(t,g)$的复合。不难检查这是良定的。
        以及$D(A)$的恒等态射$\id:X\to X$为根图$(\id_X,\id_X)$的等价类。
		\item $D(A)$是加性范畴。假设$\varphi$, $\varphi':X\to Y$为$D(A)$中的态射,
        他们由根图$(s,f)$和$(s',f')$所表示:
		\[
			\xymatrix{
				& Z \ar@{->}[ld]_{s} \ar@{->}[rd]^{f} &  \\
				X &  & Y
				}
			\quad
			\xymatrix{
				& Z' \ar@{->}[ld]_{s'} \ar@{->}[rd]^{f'} &  \\
				X &  & Y
			}
		\]
		于是,利用 Lemma \ref{lem:1.21} 可以找到交换图
		\[
			\xymatrix{
				U \ar@{-->}[rr]^{r'} \ar@{-->}[d]_{r} &  & Z' \ar@{->}[d]^{s'} \\
				Z \ar@{->}[rr]^{s} &  & X
				}
		\]
		其中 $r$, $r'$, $s$ 和 $s'$ 是拟同构,进而我们可以得到两个新的根图
		\[
			\xymatrix{
				& U \ar@{->}[ld]_{t} \ar@{->}[rd]^{fr} &  \\
				X &  & Y
				}
			\quad 
			\xymatrix{
				& U \ar@{->}[ld]_{t} \ar@{->}[rd]^{f'r'} &  \\
				X &  & Y
				}
		\]
		来表示$\varphi$ 和 $\varphi'$,其中 $t=sr=s'r'$. 最后,我们定义 $\varphi+\varphi'$ 
        为根图 $(t,fr+f'r')$ 所表示的态射。
	\item 我们可以将$K(A)$的态射$f$嵌入到$D(A)$中,通过将其表示为根图$(\id,f)$. 
    因此,$K(A)$中的态射$f$到$D(A)$中是一个零态射当且仅当存在拟同构 $s$ 或 $t$ 
    使得$sf=0$ 或 $ft=0$. (由 Lemma \ref{lem:1.22},这俩是等价的。)
	\end{compactenum}
\end{para}

% \begin{para}[Cyclic Complex]
% 	A complex $K^\bullet$ is cyclic if $d_K^n=0$ for any $n$. 
% 	Cyclic complexes form a subcategory $\operatorname{Kom}_0(A)
% 	\subset \operatorname{Kom}(A)$. Now let's consider the 
% 	homology functor
% 	\[
% 		h:\operatorname{Kom}(A)\to \operatorname{Kom}_0(A)
% 	\]
% 	defined by 
% 	\[
% 		h((K^\bullet,d_K))=(H^\bullet(K),0),\quad 
% 		h(f)=H^\bullet(f).
% 	\]
% 	It transforms quasi-isomorphisms to isomorphisms. Therefore,
% 	there exists a functor 
% 	\[
% 		k:D(A)\to \operatorname{Kom}_0(A).
% 	\]
% \end{para}

% \begin{pro}
% 	Suppose $A$ is an abelian category, then 
% 	the functor $D(A)\to \operatorname{Kom}_0(A)$ defined above is 
% 	an equivalence iff $A$ is semisimple, i.e. any exact triple 
% 	in $A$ splits.
% \end{pro}

下面我们将讨论将对象嵌入到某个复形中。

\begin{para}
我们称Abel范畴$A$上的$0$-复形是指形如$\cdots \to 0\to X\to 0\to \cdots$
的复形,即只有一个非零的$X$位于$0$的位置。我们记这个复形为$X[0]$.
类似地,我们可以定义$i$-复形,即只有一个非零的$X$位于$i$的位置,记作
$X[-i]$.

我们称Abel范畴$A$上的复形$K^\bullet$是一个$H^0$-复形,如果$H^i(K^\bullet)=0$
对任何$i\neq 0$都成立。不难看到$H^0$-复形是$0$-复形的推广。
可以证明,$H^0$-复形构成了一个范畴,他是一个$D(A)$的一个完全子范畴,
并且实际上等价于$A$. 我们当然可以类似地定义$H^i$-复形,这不过就是平移,
所以也等价于$A$.
\end{para}

\begin{para}[$\operatorname{Ext}$群]
    $\operatorname{Ext}_A^i(X,Y):=\Hom_{D(A)}(X[0],Y[i])$. 由于$D(A)$
    是加性范畴,所以这是一个Abel群。
\end{para}

通过适当平移,我们容易看到$\operatorname{Ext}_A^i(X,Y)=\Hom_{D(A)}(X[k],Y[k+i])$
对任意的整数$k$都成立。于是$\operatorname{Ext}_A^i(X,Y)$与$\operatorname{Ext}_A^j(Y,Z)$
之间的复合,就是$\Hom_{D(A)}(X[k],Y[k+i])$和$\Hom_{D(A)}(Y[k+i],Y[k+i+j])$之间的
复合,即$\Hom_{D(A)}(X[k],Z[k+i+j])$,即
\[
    \operatorname{Ext}_A^i(X,Y)\times  \operatorname{Ext}_A^j(Y,Z)
    \to \operatorname{Ext}_A^{i+j}(X,Z).
\]

\begin{para}[内射消解]
    一个对象$X$的内射嵌入是一个正和列$0\to X\to I$,也即一个单射$X\to I$,其中$I$是一个
    内射对象。如果$A$中的每一个对象都有一个内射嵌入,则称$A$有足够多的内射对象。

    设$X$是一个对象,正和列
    \[
        0\to X\to I^0\to I^1\to I^2\to \cdots
    \]
    被称为$X$的一个内射消解,如果其中$I^i$都是内射对象。
    不难用内射对象的泛性质得到,同一个$X$的不同内射消解都是同伦的。

    如果$A$有足够多的内射对象,则对任意的$X$,他都存在内射消解。
    实际上,我们从正和列$0\to X\xrightarrow{\epsilon} I^0$
    开始,考虑态射$\coker\epsilon:I^0\to Y$,由于$Y$也有内射嵌入,所以取
    $0\to Y\to I^1$,取$d^0$为复合$I^0\to Y\to I^1$,因为$Y\to I^1$是单的,所以
    $\ker d^0=\ker \coker \epsilon=\im \epsilon$,这样我们得到了正和列
    \[
        0\to X\xrightarrow{\epsilon} I^0\xrightarrow{d^0} I^0,
    \]
    对$d^0$重复上面的操作,我们可以得到一个正和列
    \[
        0\to X\to I^0\to I^1\to I^2\to \cdots.
    \]
\end{para}

\begin{thm}
令$I$为Abel范畴$A$中所有的内射对象所构成的完全子范畴,则同伦范畴$K^+(I)$
到导出范畴$D^+(A)$的自然函子$K^+(I)\to D^+(A)$给出了$K^+(I)$与$D^+(A)$的一个
完全子范畴的等价。同事,如果$A$有足够多的内射对象,则上述函子给出了
$K^+(I)$与$D^+(A)$等价。
\end{thm}

\begin{para}[导出函子]
设$F:A\to B$是一个加性的左正和函子,则其导出函子是函子
$RF:D^+(A)\to D^+(B)$及一个自然变换$\epsilon_F:Q_B\circ K^+(F)
\to RF\circ Q_A$,
\[
    \xymatrix{
        K^+(A) \ar[r]^{K^+(F)} \ar[d]_{Q_A} & K^+(B) \ar[d]^{Q_B}
        \ar@{=>}[ld] \\
        D^+(A) \ar[r]_{RF} & D^+(B)
    }
\]
满足以下泛性质:对任意的函子$G:D^+(A)\to D^+(B)$及自然变换
$\epsilon:Q_B\circ K^+(F)\to G\circ Q_A$,都存在唯一的自然变换
$\eta:RF\to G$使得如下图交换
\[
    \xymatrix{
        & Q_B\circ K^+(F) \ar[ld]_{\epsilon_F} \ar[rd]^{\epsilon} & \\
        RF\circ Q_A \ar[rr]_{\eta \circ Q_A} & & G\circ Q_A
    }
\]
类似地,我们可以对右正和函子给出相应的定义。
\end{para}

导出函子是可以复合的,即如果$F:A\to B$和$G:B\to C$都是左正和函子,
那个$GF$也是左正和的,且有同构$E:R(GF)\to R(G)R(F)$. 描述这个同构
往往是复杂的,为此我们需要一个新的工具,叫做谱序列。

\begin{para}[谱序列]
令$A$为一个Abel范畴,则$A$中的一个谱序列是资料$E=(E_r^{p,q},E^n)$以及
微分算符
\[
	d_r^{p,q}:E_r^{p,q}\to E_r^{p+r,q-r+1},
\]
其中$r\geq 1$,而其他指标都是整数。微分算符当然需要满足$d_r^2=0$,
即$d_r^{p+r,q-r+1}d_r^{p,q}=0$. 于是,对给定的$r$,$E_r$构成了一个
复形,于是我们可以定义$H^{p,q}(E_r):=\ker d_r^{p,q}/\im d_r^{p+r,q-r+1}$.
此外,我们还有同构$\alpha_r^{p,q}:H^{p,q}(E_r)\to E_{r+1}^{p,q}$,即
而$E_{r+1}$是$E_r$的(上)同调。
\end{para}