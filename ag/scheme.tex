% !TEX root = main.tex

\chapter{层与仿射概形}
\ThisULCornerWallPaper{1}{../Pictures/7.png}

我们假设本文所出现的环都是含幺交换环,即有一个乘法单位元的交换环,并且,我们所谓的环同态,他将单位元映到单位元。特别地,我们约定,零环即$R=\{0\}$,其中乘法单位元和加法单位元相等,即$0=1$。在环(即我们这里所谓的环)范畴中,零环的地位大致相当于集合范畴中的空集。所以下面我们在提到环的时候,经常是默认该环不是零环。此外,我们下面谈及的模也都是含幺交换环上的模,所以不分左右双边。

\section{层与赋环空间}

\begin{para}
一个拓扑空间$X$能被看成一个范畴,对象取作他的所有开集,而态射取作
\[
	X(U,V)=\begin{cases}
	\bigl\{i_{VU}:U\hookrightarrow V\bigr\}&\text{, if }U\subset V\text{;}\\
	\varnothing&\text{, otherwise}.
	\end{cases}
\]

现在,任取$X$的一个开覆盖$\mathscr{U}=\{U_i\,:\, i\in I\}$,为后续的方便起见,我们记$\overline{\mathscr{U}}$为在$\mathscr{U}$中添入所有$\mathscr{U}$中元素的有限交后得到的新的开覆盖,这使得开覆盖是有限交封闭的。显然,依然通过
\[
	\overline{\mathscr{U}}(U,V)=\begin{cases}
	\bigl\{i_{VU}:U\hookrightarrow V\bigr\}&\text{, if }U\subset V\text{;}\\
	\varnothing&\text{, otherwise}.
	\end{cases}
\]
$\overline{\mathscr{U}}$也是一个范畴。再方便起见,我们可以假设开覆盖是有限交封闭的。
\end{para}

\begin{para}
对于任意的范畴$K$,一个拓扑空间$X$,我们称呼一个取值在$K$中的\idx{预层}$\calf$(简称为$K$-预层)是一个$X\to K$的反变函子。特别地,设$V\subset U$是两个开集,记态射$\calf(i_{UV}):\calf(U)\to \calf(V)$为$\rho^U_{V}$,称为限制态射。如果$K$是集合范畴的子范畴,则$\calf(U)$中的元素$s$被称为$U$上的一个\idx{截面},而$s|_V:=\rho^U_V(s)$被称为截面$s$在$V$上的限制。

设$\calf$是一个$X$上的$K$-预层,任取一个开集$U$以及一个$U$的(有限交封闭的)开覆盖$\mathscr{U}$. 显然,将$\calf$限制在$\mathscr{U}$上也将得到一个反变函子$\calf_\mathscr{U}:\mathscr{U}\to K$. 如果$\calf_\mathscr{U}$-图的极限是$\calf(U)$,即
\[
	{\varprojlim}_{V\in \mathscr{U}}\calf(V)=\calf(U),
\]
则称预层$\calf$为一个\idx{层}。上面这个条件被称为层公理。如果需要明确取值的范畴是$K$,则称为$K$-层。
\end{para}

层公理可以用极限的泛性质写出来。设$A\in K$是范畴$K$里面的一个对象,而$\calf$是$X$上的一个预层,$U$是$X$的任意开子集, 而$\{U_\alpha\}_{\alpha\in I}$是$U$的任意开覆盖,如果存在态射族$\{\varphi_\alpha:A\to \calf(U_\alpha)\}_{\alpha\in I}$使得下图交换(画图时省去了限制映射)
\[
	\xymatrix{
		A\ar[r]^{\varphi_{\beta}}\ar[d]_{\varphi_{\alpha}}&\calf(U_\beta)\ar[d]\\
		\calf(U_\alpha)\ar[r]&\calf(U_{\alpha}\cap U_{\beta})
	}
\]
则存在唯一的态射$\varphi:A\to \calf(U)$使得下图交换的时候,
\[
	\xymatrix{
		A\ar@/_/[ddr]_{\varphi_{\alpha}}\ar@/^/[drr]^{\varphi_{\beta}}\ar@{-->}[dr]|{\varphi}&& \\
		&\calf(U)\ar[r]\ar[d]&\calf(U_\beta)\ar[d]\\
		&\calf(U_\alpha)\ar[r]&\calf(U_{\alpha}\cap U_{\beta})
	}
\]

如果$K$是集合范畴或者交换群范畴,则它可以容纳任意的积,很容易发现上面关于层的定义和我们熟知的层的定义是等价的,即局部相容的截面可以唯一地拼成一个整体的截面\footnote{具体来说,就是给定$U$的一个开覆盖$\{U_\alpha\}$,如果存在一族$\{s_\alpha\in U_\alpha\}$使得$s_\alpha|_{U_\alpha\cap U_\beta}=s_\beta|_{U_\alpha\cap U_\beta}$,则存在唯一的$s\in \calf(U)$使得$s|_{U_\alpha}=s_\alpha$.}. 这点只要将$A$取作$\prod_{\alpha\in I}\calf(U_\alpha)$即可。

\begin{para}
两个层的例子,流形$M$上,每一个开集$U$上的光滑函数的全体构成一个环,记作$\calf(U)$,则这是一个预层,$U$上的截面就是$U$上的光滑函数。这显然是一个层,因为光滑性是局部概念,如果两个光滑函数限制在同一个开集上是相同的,那么我们自然可以拼成一个唯一的更大的光滑函数,一族也是类似的。

另一个例子来自代数簇,设$X$是一个代数簇,他的每一个开集$U$都是一个子代数簇,自然有相应的正则函数环$\calf(U)$,则这是一个预层,$U$上的截面就是$U$上的正则函数。这依然是一个层,因为正则性还是局部概念,可以拼接的理由同上。
\end{para}

\begin{para}
所有$X$上的预层构成一个范畴,态射就是所谓的“函子间态射”,或者叫做\idx{自然变换}:设$\calf$和$\calg$是$X$上的两个预层,设$\varphi(U)$是使得如下交换图交换的一族态射,
\[
	\xymatrix{
		\calf(U)\ar[rr]^{\varphi(U)} \ar[d]_{\rho^U_V}&&\calg(U) \ar[d]^{\pi^U_V}\\
		\calf(V)\ar[rr]^{\varphi(V)}&&\calg(V)
	}
\]
则称呼$\varphi$是$\calf$到$\calg$的一个态射,记做$\varphi:\calf\to\calg$. 层之间的态射就是作为预层的态射。很自然地,如果已知一个预层态射$\varphi:\calf\to\calg$,即使$\calf$是一个层,也不能断言$\calg$是一个层。
\end{para}

\begin{para}[截面函子]
设$\calf$是$X$上的一个预层,$U\subset X$是一个开集,定义$\Gamma(U,\calf)=\calf(U)$. 特别地,如果$U=X$,则记$\Gamma(\calf)=\Gamma(U,\calf)$. 现考虑$X$上的另一个预层$\calg$以及态射$u:\calf\to\calg$,定义$\Gamma(U,u)=u(U)$. 此时$\Gamma(U,-)$就构成了一个函子。如果固定$\calf$,考虑$V\subset U$,则我们还有限制映射$\Gamma(U,\calf)\to\Gamma(V,\calf)$,于是$\Gamma(-,\calf)$也是一个函子。由于预层态射与限制映射相容,因此$\Gamma(\star,-)$是一个双函子。
\end{para}

\begin{para}[茎]
设$\calf$是$X$上的一个预层,如果范畴$K$能容纳余极限,那么定义$\calf$在$p\in X$处的\idx{茎}$\calf_p$为$\varinjlim \calf(U)$,其中$U$跑遍所有$p$的邻域,这些领域通过包含构成一个显然的direct system. 

如果$K$是集合范畴的子范畴,$s\in \calf(U)$是一个界面,而$p\in U$,记$s$在典范态射$\calf(U)\to \calf_p$下的像为$s_p$. 它被称为界面$s$在$p$处的\idx{芽}。

对于交换群范畴(我们知道这是余完备的),直接写出芽的具体构造是比较方便的,$\calf_p$中的元素(芽)是这样的等价类$\langle U,f\rangle$的全体,其中$f$是$U$上的截面,而$U$是$p$的邻域。等价关系定义为$\langle U,f\rangle\sim \langle V,g\rangle$,当且仅当存在一个$p$的邻域$W\subset U\cap V$使得$f|_W=g|_W$. 

对$X$上的交换群层$\calf$,所有使得$\calf_x\neq 0$的$x$构成了一个$X$的子集,我们称之为$\calf$的支集,记作$\mathrm{Supp}(\calf)$.
\end{para}

利用余极限的泛性质,我们发现,预层间的态射$\varphi:\calf\to\calg$如下图在局部诱导了$\varphi_p:\calf_p\to\calg_p$.
\[
	\xymatrix{
		&&\calf(U)\ar[dl]^{\rho^U_{p}}\ar[dd]^{\rho^U_{V}}\ar@/_/[lld]_{{\rho'}^U_{p}\circ\varphi(U)} \\
		\calg_p&\calf_p\ar@{-->}[l]_(0.4){\varphi_p}&\\
		&&\calf(V)\ar[ul]_{\rho^V_{p}}\ar@/^/[llu]^{{\rho'}^V_{p}\circ\varphi(V)}
	}
\]

下面一个引理从交换群层的芽的构造来看是平凡的。

\begin{lem}\label{lem:1}
假设$\calf$是一个$X$上的交换群层,且$s$, $t\in \calf(U)$是两个截面。如果任取$p\in U$,都有$s_p=t_p$,则$s=t$. 此外,设$\calg$是$X$上的另一个交换群层,$\psi$, $\varphi:\calf\to\calg$都是层之间的态射,则$\psi|_U=\varphi|_U$当且仅当$\psi_p=\varphi_p$对任意的$p\in U$都成立。
\end{lem}

% \begin{proof}
% 不失一般性,可以假设$U=X$. 由交换群层上芽的构造,我们知道,如果$s_p=t_p$,则存在$p$的一个邻域$U_p$使得$s|_{U_p}=t|_{U_p}$. 遍历$p$,$\{U_p\}$是$U$的一个开覆盖,由层上截面的黏合,我们就得到了$s=t$.

% 此外,如果$\psi_p=\varphi_p$对任意的$p\in X$都成立,我们要证明,任取开集$U$,有$\psi(U)=\varphi(U)$. 任取$s\in \calf(U)$,由于$(\psi(U)(s))_p=\psi_p(s_p)=\varphi_p(s_p)=(\varphi(U)(s))_p$对所有$p\in U$都成立,所以$\psi(U)(s)=\varphi(U)(s)$. 由于$s$是任意的,所以$\psi(U)=\varphi(U)$. 反过来是显然的。
% \end{proof}

\begin{para}
考虑$\mathfrak{B}$是$X$的拓扑基,则同拓扑空间一样,我们取对象为基中的元素,态射为内含映射,则自然他也构成一个范畴。类似地,我们可以定义$\mathfrak{B}$上取值在$K$的预层为$\mathfrak{B}$到$K$的一个反变函子,层公理也可以一模一样搬过来。

如果范畴$K$可以容纳极限,而$\calf$是$\mathfrak{B}$上的一个预层,则我们可以通过$\calf'(U)=\varprojlim \calf(V)$定义出$X$上的一个预层$\calf'$,其中$V$跑遍集合$\{V\in \mathfrak{B}\,:\, V\subset U\}$,而这个集合通过包含构成一个显然的投影系。
\end{para}

\begin{pro}\label{psgl}
如果范畴$K$可以容纳极限,且$\calf$是$\mathfrak{B}$上的一个层,$\mathfrak{B}$上的层诱导的预层$\calf'$也是一个层。
\end{pro}

证明完全是泛性质的练习,见[EGA, Chap 0, 3.2]. 作为推论:在拓扑基的每一个开集$U_\alpha$上都有一个截面$s_\alpha$,任取指标$(\alpha,\beta)$,都存在$\gamma$使得$U_\gamma\subset U_\alpha\cap U_\beta$且$s_\alpha|_{U_\gamma}=s_\beta|_{U_\gamma}$,则存在唯一的整体截面$s$使得$s_\alpha=s|_{U_\alpha}$.

\begin{para}[层的限制]
对于$X$上的$K$-层$\calf$,很自然可以定义层$\calf$在$U$上的限制$\calf|_U$,这是一个$U$上的预层,对于$U$中的开集,有$\calf|_U(V)=\calf(V)$,限制态射直接从$\calf$那里继承过来。显然这也是一个层。
\end{para}

现在我们可以讨论层的黏合。

\begin{pro}
设$\{U_\alpha\}_{\alpha \in I}$是$X$的一个开覆盖,如果分别在每一个$U_\alpha$上有一个层$\calf_\alpha$,而且对于任意的指标组$(\alpha,\beta)$,都有一个同构$\theta_{\alpha\beta}:\calf_\alpha|_{U_\alpha\cap U_\beta}\to \calf_\beta|_{U_\alpha\cap U_\beta}$,满足:$\theta_{\alpha\alpha}=\id$, $\theta_{\alpha\beta}=\theta_{\beta\alpha}^{-1}$,以及对于任意的指标组$(\alpha,\beta,\gamma)$成立$\theta'_{\alpha\gamma}=\theta'_{\alpha\beta}\circ \theta'_{\beta\gamma}$,其中$\theta'_{\star\star}$是态射$\theta_{\star\star}$在$U_\alpha\cap U_\beta \cap U_\gamma$上的限制。则存在一个层$\calf$和一族同构$\eta_\alpha:\calf|_{U_\alpha}\to \calf_\alpha$.
\end{pro}

\begin{proof}
选取包含在某个开覆盖的开集的全体,这是全空间的一个拓扑基,命题中的黏合条件自然地给出层的条件。%详见[EGA, Chap 0, 3.3].
\end{proof}

\begin{para}[顺像]\label{sx}
设$f:X\to Y$是拓扑空间之间的连续映射,而$\calf$是$X$上的一个预层,则通过$f_*\calf(U)=\calf(f^{-1}(U))$我们可以定义出$Y$上的一个预层$f_*\calf$,他被称为$\calf$关于$f$的\idx{顺像}。 由于原象是保持包含关系的,即如果$V\subset U$,则$f^{-1}(V)\subset f^{-1}(U)$,所以$f_*\calf$上的限制映射就很自然地写作$(f_*\rho)^U_V=\rho^{f^{-1}(U)}_{f^{-1}(V)}$. 不难检验,当$\calf$是$X$上的一个层的时候,$f_*\calf$是$Y$上的一个层。

设$\varphi:\calf\to\calg$是$X$上两个预层之间的态射,则$f$自然诱导了一个新的态射$f_*\varphi:f_*\calf\to f_*\calg$,具体写出来就是$(f_*\varphi)(U)=\varphi(f^{-1}(U))$,通过自然变换$\varphi$可以验证$f_*\varphi$是一个自然变换。考虑两个连续映射的复合,很自然地可以验证$(f g)_*=f_* g_*$,这来自于关于原象的等式$(fg)^{-1}(U)=g^{-1}(f^{-1}(U))$. 

设$u:\mathcal{F}\to\mathcal{G}$是一个$X$上预层间的态射,则$f_*u(U)=u(f^{-1}(U))$就是一个$f_*\mathcal{F}\to f_*\mathcal{G}$之间的态射。并且,设$v:\mathcal{G}\to\mathcal{H}$是令一个$X$上预层间的态射,成立$f_*(v\circ u)=f_*v\circ f_* u$. 于是$f_*$就构成了一个$X$上预层范畴到$Y$上预层范畴之间的协变函子。并且,由于$f_*$将层变成层,所以$f_*$也诱导了一个$X$上层范畴到$Y$上层范畴之间的协变函子。
\end{para}

% \begin{lem}\label{lem:hanrudan}
% 设$u$, $v$是$Y$上的预层态射,而$i:Y\to X$是一个子空间的含入映射,则$i_*u=i_*v$将给出$u=v$.
% \end{lem}

% \begin{proof}
% 任取$Y$中的开集$V$,则必然存在一个$X$中的开集$U$使得$V=U\cap Y$,所以$i^{-1}(U)=V$. 于是$(i_{*}u)(U)=(i_{*}v)(U)$将给出$u(V)=v(V)$. 又$V$是任意的,所以$u=v$.
% \end{proof}

现在,我们考虑$(f_*\calf)_{f(p)}$与$\calf_p$之间的联系。设$\rho$是$\calf$上的限制映射族,从交换图
\[
	\xymatrix{
		&&&f_*\calf(U)=\calf(f^{-1}(U))\ar[dl]^{(f_*\rho)_{p}^U}\ar[dd]^{(f_*\rho)^U_{V}}\ar@/_/[llld]_-{\rho_{p}^{f^{-1}(U)}} \\
		\calf_p&&(f_*\calf)_{f(p)}\ar@{-->}[ll]_-{\quad (f_*^\calf)_p}&\\
		&&&f_*\calf(V)=\calf(f^{-1}(V))\ar[ul]_{(f_*\rho)_{p}^V}\ar@/^/[lllu]^-{\rho_{p}^{f^{-1}(V)}}
	}
\]
我们可以得到一个唯一的典范态射
\[
(f_*^\calf)_p:(f_*\calf)_{f(p)}\to \calf_p,
\]
这个态射的性质一般情况下并不清楚,即使在交换群范畴,它可能既不是单的也不是满的。

\begin{para}[赋环空间]
所谓的\idx{赋环空间}(ringed space),就是一个拓扑空间$X$,和拓扑空间上的交换环层$\mathcal{O}_X$构成的资料$(X,\mathcal{O}_X)$. 流形和上面的光滑函数构成一个赋环空间,同样,代数簇和上面的正则函数构成一个赋环空间。所谓的\idx{局部赋环空间},就是说任意一点$p\in X$处的茎$\mathcal{O}_p=\varinjlim \mathcal{O}_X(U)$是局部环。
\end{para}

流形的例子是一个局部赋环空间,因为在$p$附近的光滑函数,如果在$p$处不为零,则存在一个邻域是可逆的,故而,在$p$处为零的光滑函数构成的等价类是唯一的极大理想。代数簇的例子还是一个局部赋环空间,因为代数簇局部是仿射的,所以我们可以假设这是一个仿射簇,对于仿射簇而言,$X$中的点$p$(如果是非代数闭的情况,那就是一条Galois轨道)一一对应着坐标环$A(X)$中的一个极大理想$\mm_p$(这来自于Hilbert零点定理),而$\mathcal{O}_p$正好同构于$A(X)$关于$\mm_p$的局部化,所以是一个局部环。

\begin{para}[赋环空间之间的态射]
赋环空间之间的态射如下定义:设有赋环空间$(X,\mathcal{O}_X)$和$(Y,\mathcal{O}_Y)$,以及连续映射$\psi:X\to Y$和$Y$上的交换环层态射$\theta:\mathcal{O}_Y\to \psi_*\mathcal{O}_X$,这样的一个二元组$\Psi=(\psi,\theta)$构成了赋环空间范畴的态射,记做$\Psi:(X,\mathcal{O}_X)\to (Y,\mathcal{O}_Y)$.
\end{para}

要确切地理解是一个态射,则需要验证复合,假设$\Phi=(\psi',\theta'):(Y,\mathcal{O}_Y)\to (Z,\mathcal{O}_Z)$是另一个态射,在$\Phi\circ \Psi$中,连续映射的复合不用多说,对于层的复合,则如图所示
\[
	\mathcal{O}_Z\xrightarrow{\theta'} \psi'_*\mathcal{O}_Y \xrightarrow{\psi'_*\theta} \psi'_*\psi_*\mathcal{O}_X,
\]
写成$(\psi'_*\theta)\circ \theta'$的形式。

\begin{para}[逆像]
有了顺像,我们现在考虑\idx{逆像}。设$f:X\to Y$是一个连续映射,而$\calf$是$X$上的一个层,$\calg$是$Y$上的一个预层。$f_*$是$X$上层范畴到$Y$上预层范畴的一个函子,如果它存在左伴随函子$f^*$,则$f^*\calg$被称为预层$\calg$关于$f$的逆像。换而言之,$\Hom_X(f^*\calg,\calf)$和$\Hom_Y(\calg,f_*\calf)$对层$\calf$与预层$\calg$是双函子同构。

设$u:\calg\to f_*\calf$,则在$\Hom_X(f^*\calg,\calf)$中对应的态射我们记作$u^\#$. 对应地,设$v:f^*\calg\to \calf$,在双函子同构下,在$\Hom_Y(\calg,f_*\calf)$中对应的态射我们记作$v^\flat$. 在音乐上,$\#$对应升调,而$\flat$对应降调,于是在这里分别对应提升和下降。前者读作sharp,后者读作flat. 
% 
% 逆像的复合类似于逆的复合。考虑两个连续映射$f:X\to Y$, $g:Y\to Z$,则从$(gf)_*=g_*f_*$,我们有函子性同构
% \[
% 	\Hom_Z((gf)^*\calg,\calf)\cong \Hom_Z(\calg,(gf)_*\calf)=\Hom_Z(\calg,g_*f_*\calf)\cong \Hom_Y(g^*\calg,f_*\calf)\cong \Hom_X(f^*g^*\calg,\calf),
% \]
% 所以$(gf)^*\cong f^*g^*$,这里的同构是函子的同构,即存在互逆的自然变换。
\end{para}

恒等态射的提升和下降在伴随函子理论中非常重要,他们可以反过来唯一确定伴随函子。我们下面稍微复习一下。

\begin{para}[伴随函子]\label{adjoint_functor}
设$L:D\to C$和$R:C\to D$为两个函子,如果存在双函子同构
\[
	C(L(-),\star)\cong D(-,R(\star)),
\]
则称$(L,R)$为一对伴随函子,$L$是$R$的左伴随,$R$是$L$的右伴随。遵从上面的习惯,我们将上式右边到左边称为提升,左边到右边称为下降,分别用$\#$与$\flat$表示。

现在,设$x\in D$,考虑恒等态射$\id_{L(x)}$的下降$\id_{L(x)}^\flat$,记作$L_x$. 同样,设$y\in C$,将恒等态射$\id_{R(y)}$的提升$\id_{R(y)}^\#$记作$R^y$. 改变$x$和$y$,我们就有了一对自然变换$L_-:\id_D\to RL$和$R^-:LR\to \id_C$. 前者经常被叫做单位(unit),后者叫做上单位(counit). 

上单位满足如下泛性质,任取$f:L(x)\to y$,都存在唯一的态射$f^\flat:x\to R(y)$使得分解
\[
	f:L(x)\xrightarrow{L(f^\flat)}LR(y)\xrightarrow{R^y}y
\]
成立,即$f=R^yL(f^\flat)$. 对单位也类似,对$g:x\to R(y)$有唯一分解$g=R(g^\#)L_x$.

现在反过来,给定函子$L:D\to C$,若任取$x\in C$,都存在一个$R(x)\in D$以及一个态射$\epsilon^x:L(R(x))\to x$使得任取态射$f:L(y)\to x$,都存在唯一的态射$g:y\to R(x)$使得分解
\[
	f:L(y)\xrightarrow{L(g)}L(R(x))\xrightarrow{\epsilon^x}x
\]
成立,则$L$是一个左伴随函子。它的右伴随函子$R$可以如下定义:对对象,$R:x\mapsto R(x)$,对态射$f:x\to x'$,考虑复合$f\epsilon^x:L(R(x))\to x'$,则泛性质给出$R(f):R(x)\to R(x')$. 不难检查这是$L$的左伴随函子,且$R^x=\epsilon^x$. 对偶地,我们也可以从单位出发去构造一对伴随函子。
%
% 对偶地,右伴随函子也有泛性质:设$R: C\to  D$是一个函子,使得任取$X\in  D$,都存在一个$L(X)\in  C$以及一个态射$\eta_X:X\to R(L(X))$使得任取态射$f:X\to R(Y)$,都存在唯一的态射$g:L(X)\to Y$使得分解
% \[
% 	f:X\xrightarrow{\eta_X}R(L(X))\xrightarrow{R(g)}R(Y)
% \]
% 成立,则$R$被称为一个右伴随函子。同样,从泛性质,不难构造其左伴随函子。
\end{para}

现在回到我们的语境。考虑$\id_{f_*\calf}:f_*\calf\to f_*\calf$,按上面的记号,有\footnote{
	注意,这里出现了$f_*^\calf$,与我们在上面看到的$f_*$在一点处的典范态射$(f_*^\calf)_p:(f_*\calf)_{f(p)}\to \calf_p$有着相同符号,下面会看到,他们就是一样的。实际上,至少对取值在交换群范畴的预层来说,$(f^*\calg)_p=\calg_{f(p)}$,所以这里出现的$f_*^\calf$也有$(f_*^\calf)_p:(f_*\calf)_{f(p)}\to \calf_p$. 首先可以看到定义域和值域相同。
}
\[
	f_*^\calf:=\left(\id_{f_*\calf}\right)^\#: f^*f_*\calf \to \calf, \quad 
	f^*_\calg:=\left(\id_{f^*\calg}\right)^\flat : \calg \to f_*f^*\calg.
\]
并且,我们可以有如下唯一分解
\[
	u^\# : f^*\calg \xrightarrow{f^*(u)} f^*f_*\calf \xrightarrow{f_*^\calf}\calf,\quad 
	v^\flat : \calg \xrightarrow{f^*_\calg} f_*f^*\calg\xrightarrow{f_*(v)} f_*\calf.
\]

最后,构造左伴随函子$f^*$,即要求构造层$f^*\mathcal F$以及预层态射$f^*_{\mathcal G}:\mathcal G\to f_*f^*\mathcal G$使得,任取态射$u:\calg \to f_*\calf$,都存在唯一分解
\[
	u : \calg \xrightarrow{f^*_\calg} f_*f^*\calg\xrightarrow{f_*(u^\#)} f_*\calf,
\]
其中$u^\#:f^*\mathcal G\to \calf$.

% \begin{lem}
% 	设有层的态射$v:\mathcal G'\to \mathcal G$以及$u:\mathcal G\to f_*\calf$,则$uv:\mathcal G'\to f_*\calf$,且$(uv)^\#=u^\#(f^*v)$.
% \end{lem}

% \begin{proof}
% 实际上,从典范分解$u^\#=f_\calf^\#(f^*u)$,所以
% \[
% 	u^\#(f^*v)=f_\calf^\#(f^*u)(f^*v)=f_\calf^\#(f^*(uv))=(uv)^\#.
% \]
% \end{proof}

% \begin{pro}
% \label{pro:1}对于一些特殊情况,逆像层总是存在的。考虑一个内含映射的例子,设$U$是$X$的开子集,相应的内含映射为$i:U\hookrightarrow X$. 对于$X$上的$K$-层$\calf$,$\calf|_U$即其在$U$上的逆像层$i^*\calf$.
% \end{pro}

% \begin{proof}
% 检查泛性质即可。现在还缺一个典范态射$\calf\to i_*i^*\calf$,由于我们假设了$i^*\calf=\calf|_U$,所以
% \[
% 	i_*i^*\calf(V)=i^*\calf(i^{-1}(V))=\calf|_U(U\cap V)=\calf(U\cap V).
% \]
% 那么限制映射族$\rho^V_{U\cap V}:\calf(V)\to \calf(U\cap V)$可以构成层之间的态射$\calf\to i_*i^*\calf$,我们将其记做$i^\flat$,我们希望这就是典范态射。剩下的不过是检查$(\calf|_U,i^\flat)$所需要满足的泛性质,即任取$u:\calf\to i_*\calg$,都有唯一的态射$u^\#:i^*\calf\to \calg$使得分解$u:\calf\xrightarrow{i^\flat} i_*i^*\calf \xrightarrow{i_*u^\#} i_*\calg$成立。

% 首先,我们来证明分解的唯一性。假设存在$v$, $w:i^*\calf\to \calg$使得分解$u=(i_*v)i^\flat=(i_*w)i^\flat$成立。设$V$是$U$的开子集,此时$i^\flat(V)=\id_{\calf(V)}$,所以$(i_*v)(V)=(i_*w)(V)$. 同时再注意到,$u(V)=u(i^{-1}(V))=i_*u(V)$和$v(V)=v(i^{-1}(V))=i_*v(V)$,所以我们有$v(V)=u(V)$成立,由于$u$, $v$都是$U$上的层之间的态射,而$V$可以取遍$U$任意的开子集,所以$v=u$成立。

% 然后证明存在性,对于任意的态射$u:\calf\to i_*\calg$,他具有形式$u(V):\calf(V)\to i_*\calg(V)=\calg(U\cap V)$. 遍历$U$中的开集$V$,我们定义$u^\#(V)=u(V):\calf(V)\to \calg(U\cap V)=\calg(V)$,此时由于$i^*\calf(V)=\calf|_U(V)=\calf(V)$,所以我们定义出了一族态射$u^\#(V):i^*\calf(V)=\calf(V)\to \calg(V)$,自然也得到了态射$u^\#:i^*\calf\to \calg$. 现在,需要检查分解$u:\calf\xrightarrow{i^\flat} i_*i^*\calf \xrightarrow{i_*u^\#} i_*\calg$成立,这个分解其实说白了就是,已知如下交换图(图中的限制映射都略去了)成立,
% \[
% 	\xymatrix{
% 		\calf(V)\ar[r]^-{u(V)} \ar[d]&\calg(U\cap V) \ar[d]\\
% 		\calf(W)\ar[r]^-{u(W)}&\calg(U\cap W)
% 	}
% \]
% 其中$W\subset V$是$X$中的开集,要验证定出来的$u^\#$使得如下交换图成立。
% \[
% 	\xymatrix{
% 		\calf(V)\ar[r] \ar[d]&\ar[rr]^{u^\#(U\cap V)}\calf(U\cap V) \ar[d]&&\calg(U\cap V) \ar[d]\\
% 		\calf(W)\ar[r]&\ar[rr]^{u^\#(U\cap W)}\calf(U\cap W)&&\calg(U\cap W)
% 	}
% \]
% 这个并不困难,左边一个矩形的交换性来自于限制映射的复合,这是自然的,右边一个矩形的交换性来自于$u^\#$的定义和上面一个交换图,所以最后要检验的不过是横向的那个分解,即对于任意的开集$V$,分解$u(V):\calf(V)\to \calf(U\cap V) \xrightarrow{u^\#(U\cap V)} \calg(U\cap V)$成立。现考虑如下交换图,
% \[
% 	\xymatrix{
% 		\calf(V)\ar[rr]^-{u(V)} \ar[d]&&\calg(U\cap V) \ar@{=}[d]\\
% 		\calf(U\cap V)\ar[rr]^{u(U\cap V)}&&\calg(U\cap V)
% 	}
% \]
% 这就得出了$u(V)=u(U\cap V)\rho^V_{U\cap V}$,而$u^\#(U\cap V)=u(U\cap V)$,所以$u(V)=u^\#(U\cap V)\rho^V_{U\cap V}$,即交换图横向的分解成立。
% \end{proof}

\begin{para}[伴随层]
设$1_X:X\to X$是恒同映射,对于$X$上的预层$\calf$,如果层$(1_X)^*\calf$存在,则他称为$\calf$的\idx{伴随层},有时候也被记做$\calf^+$. 对于任意的层$\calf'$和态射$u:\calf\to\calf'$,总有唯一分解$u:\calf\xrightarrow{(1_X)^*_\calf} \calf^+ \to \calf'$成立。
\end{para}

注意到双函子同构$\Hom_X ((1_X)^* \mathcal{F},\mathcal{G})\cong \Hom_X(\mathcal{F},\mathcal{G})$,前者是$X$上层范畴之间的态射集,后者是$X$上预层范畴的态射集。所以$(1_X)^*$是层范畴的自由对象,即层范畴到预层范畴的遗忘函子的左伴随函子。若$\calf$是一个层,则$\Hom_X(\calf,\calg)$同时是预层与层的态射集,由Yoneda引理,$\calf\cong (1_X)^* \mathcal{F}$. 因此,层的伴随层总存在,且同构于他本身。

\begin{lem}
如果范畴$K$可以容纳极限,且对$Y$上的预层$\mathcal{F}$存在伴随层,则对任意连续映射$f :X\to Y$,逆像层$f^*\mathcal{F}$存在,且同构于层$U\mapsto \varprojlim_{V\supset f(U)} \mathcal{F}^+(V)$,其中$V$跑遍所有包含$f(U)$的开集。
\end{lem}

\begin{proof}
	我们先验证$U\mapsto \varprojlim_{V\supset f(U)} \mathcal{F}^+(V)$是一个层。设$U$是一个开集,而$U_\alpha\subset U$也是一个开集。对使得$f(U_\alpha)\subset W$的开集$W$,我们都可以找到一个开集$W'$包含$W$且使得$f(U)\subset W'$. 此时典范投影$\varprojlim_{V\supset f(U)} \mathcal{F}^+(V)\to \mathcal{F}^+(W')$和限制态射$\mathcal{F}^+(W')\to \mathcal{F}^+(W)$复合给出了态射$\varprojlim_{V\supset f(U)} \mathcal{F}^+(V)\to \mathcal{F}^+(W)$. 由限制态射的相容性与极限的泛性质,我们得到了新的限制态射
	\[
	{\varprojlim}_{V\supset f(U)} \mathcal{F}^+(V)\to {\varprojlim}_{W\supset f(U_\alpha)} \mathcal{F}^+(W).
	\]
	所以这确实是一个预层,暂时记作$\mathcal{H}$.

	现在设$\mathscr{U}$是$U$的一个开覆盖,对下列交换图,我们要证明存在唯一的态射$\varphi:A\to \mathcal{H}(U)$
	% 若存在一个对象$A$使得下列交换图成立
	% \[
	% \xymatrix{
	% 	A\ar[r]^{\varphi_{\alpha}}\ar[d]_{\varphi_{\beta}}&\mathcal{H}(U_\beta)\ar[d]\\
	% 	\mathcal{H}(U_\alpha)\ar[r]&\mathcal{H}(U_{\alpha}\cap U_{\beta})
	% }
	% \]
	% 则我们只需证明存在唯一的态射$\varphi:A\to \mathcal{H}(U)$使得下图交换的时候,
	\[
		\xymatrix{
			A\ar@/_/[ddr]_{\varphi_{\alpha}}\ar@/^/[drr]^{\varphi_{\beta}}\ar@{-->}[dr]|{\varphi}&& \\
			&\mathcal{H}(U)\ar[r]\ar[d]&\mathcal{H}(U_\beta)\ar[d]\\
			&\mathcal{H}(U_\alpha)\ar[r]&\mathcal{H}(U_{\alpha}\cap U_{\beta})
		}
	\]
	由极限$\mathcal{H}(U)=\varprojlim_{V\supset f(U)} \mathcal{F}^+(V)$的泛性质,实际上只需要知道一族态射$A\to \mathcal{F}^+(V)$即可,其中$V\supset f(U)$. 而这直接来自于$\mathcal{F}^+$是一个层以及上面的交换图。

	任取$u:f^*\mathcal{F}\to\calg$,任取$Y$中的开集$V$,考虑开集$U=f^{-1}(V)$. 由于$f(U)=V$就是一个开集,所有包含$f(U)$的开集拥有极小元$V$,因此$f^*\mathcal{F}(f^{-1}(V))=\mathcal{F}^+(V)$.
	则
	\[
	u(f^{-1}(V)):\mathcal{F}^+(V)\to \calg(f^{-1}(V)),
	\]
	给出了态射$u':\mathcal{F}^+\to f_*\calg$. 最后,再复合上典范态射$(1_X)^*_\calf:\calf \to \mathcal{F}^+$. 我们定义$u^\flat=u'\circ (1_X)^*_\calf: \mathcal{F}\to f_*\calg$.

	反过来。由于$\calg$是一个层,所以$f_*\calg$也是一个层,根据证明最开始的推理,
	\[
		f^*f_*\calg(U)=\varprojlim_{V\supset f(U)} (f_*\calg)^+(V)=\varprojlim_{V\supset f(U)} f_*\calg(V)=\varprojlim_{V\supset f(U)} \calg(f^{-1}(V))
	\]
	也是一个层。由于包含关系$U\subset f^{-1}(f(U))\subset f^{-1}(V)$,
	我们有一族限制态射$\mathcal{G}(f^{-1}(V))\to \mathcal{G}(U)$. 于是有映射
	\[
	\varprojlim_{V\supset f(U)} \calg(f^{-1}(V))\to \calg(f^{-1}(W))\to \mathcal{G}(U),
	\]
	其中$W$是任意的开集$W\supset f(U)$. 由限制映射的相容性,我们可以得知,这个映射的定义与$W$的选取无关。因此,我们就得到了态射$f^*_\calg:f^*f_*\calg\to \mathcal{G}$. 任取$v:\mathcal{F}\to f_*\mathcal{G}$,我们有$f^*(v):f^*\mathcal{F}\to f^*f_*\mathcal{G}$(这直接来自极限的泛性质),然后复合上面的态射$f^*_\calg$,就得到了$v^\#=f^*_\calg\circ f^*(v):f^*\mathcal{F}\to \mathcal{G}$.

	最后,无外乎检查一些泛性质。由构造,这是直接的。
\end{proof}

% 从这个引理,Proposition \ref{pro:1} 就是显然的了。事实上,由于包含$i(V)=U\cap V$的任意开集中有极小元$U\cap V$,所以极限$\varprojlim_W \mathcal{F}^+(W)$总存在且就等于$\mathcal{F}^+(U\cap V)$. 并且,由于$\mathcal{F}$是一个层,所以$\mathcal{F}^+=\mathcal{F}$也存在。

\begin{coro}
	设$U$是$X$的开子集,相应的内含映射为$i:U\hookrightarrow X$. 对于$X$上的$K$-层$\calf$,$\calf|_U$即其在$U$上的逆像层$i^*\calf$. 
\end{coro}

\begin{proof}
	事实上,由于包含$i(V)=U\cap V$的任意开集中有极小元$U\cap V$,所以极限$\varprojlim_W \mathcal{F}^+(W)$总存在且就等于$\mathcal{F}^+(U\cap V)$. 并且,由于$\mathcal{F}$是一个层,所以$\mathcal{F}^+=\mathcal{F}$也存在。
\end{proof}

\begin{para}[赋环空间的典范含入态射]
将层的限制的知识应用到赋环空间,我们就有了赋环空间的态射$(i,i^\flat):(U,\mathcal{O}_U)\to (X,\mathcal{O}_X)$,其中$\mathcal{O}_U=\mathcal{O}_X|_U$,他被称为赋环空间的典范含入态射。赋环空间$(U,\oo_U)$时常被称为赋环空间$(X,\oo_X)$的开子赋环空间。

设$\Phi:(X,\mathcal{O}_X)\to (Y,\mathcal{O}_Y)$是一个态射,他在$U\subset X$上的限制记做$\Phi|_U:=\Phi\circ (i,i^\flat)$,如果$\Phi=(\psi,\theta)$,则$\Phi|_U=\bigl(\psi|_U, \psi_*(i^\flat) \circ \theta\bigr):(U,\mathcal{O}_U)\to (Y,\mathcal{O}_Y)$,层之间的态射具体写出来即
\[
	\mathcal{O}_Y\xrightarrow{\theta} \psi_*\mathcal{O}_X \xrightarrow{\psi_*(i^\flat)} \psi_*i_*\mathcal{O}_U=\bigl(\psi|_U\bigr)_*\mathcal{O}_U,
\]
其中$\psi_*(i^\flat)(V)=i^\flat\bigl(\psi^{-1}(V)\bigr)=\rho^{\psi^{-1}(V)}_{U\cap \psi^{-1}(V)}$,设$s\in \mathcal{O}_Y(V)$是一个截面,那么表现为
\[
	\psi_*(i^\flat)(V)\circ \theta(V)(s)=\theta(V)(s)|_{U\cap \psi^{-1}(V)}.
\]
我们记$\Phi|_U=(\psi|_U,\theta|_U)$,其中$\theta|_U=\psi_*(i^\flat)\theta$,对截面$s\in \mathcal{O}_Y(V)$,它成立
\[
	\theta|_U(V):s\mapsto \theta(V)(s)|_{\psi|_U^{-1}(V)}\in \oo_X(\psi|_U^{-1}(V)).
\]

很容易验证,如果$V$又是$U$的一个开子集,则$(\Phi|_U)|_V=\Phi|_V$.
\end{para}

\begin{pro}\label{pro:1.20}
设$(\psi,\theta):(X,\oo_X)\to (Y,\oo_Y)$是一个赋环空间态射,如果存在一个$Y$的开子集$U$使得$\psi(X)\subset U$,则存在唯一的态射$\Phi:(X,\oo_X)\to (U,\oo_U)$使得分解$(\psi,\theta)=(i,i^\flat)\Phi$成立,其中$(i,i^\flat):(U,\oo_U)\hookrightarrow (Y,\oo_Y)$是典范含入。
\end{pro}

\begin{proof}
由于$\psi(X)\subset U$,所以我们可以通过$\psi'(x)=\psi(x)$可以定义唯一的连续映射$\psi':X\to U$使得分解$\psi=i\psi'$成立。现在,考虑典范同构
\[
	\Hom_{U}(\oo_U,\psi'_*\oo_X)=\Hom_{U}(i^*\oo_Y,\psi'_*\oo_X)\cong \Hom_{Y}(\oo_Y,i_*\psi'_*\oo_X)=\Hom_{Y}(\oo_Y,\psi_*\oo_X).
\]
所以对每一个$\theta:\oo_Y\to \psi_*\oo_X$,都对应唯一的$\theta':\oo_U\to \psi'_*\oo_X$. 注意分解$\theta=(i_*\theta')i^\flat$不外乎是$i^*\oo_Y$的泛性质而已,此时$\Phi=(\psi',\theta')$就是我们所求的唯一态射。
\end{proof}

\begin{coro}\label{coro:1.21}
开子赋环空间的典范含入是一个单态射。
\end{coro}

有了限制的知识,就可以讨论赋环空间间态射的黏合。

\begin{pro}
设$(X,\mathcal{O}_X)$是一个赋环空间,并且$\{U_\alpha\}_{\alpha\in I}$是一个$X$的开覆盖。如果我们有一族态射$\Phi_\alpha:(U_{\alpha},\mathcal{O}_{U_{\alpha}})\to (Y,\mathcal{O}_Y)$,对于任意的两个指标$(\alpha,\beta)$,成立$\Phi_\alpha|_{U_{\alpha}\cap U_{\beta}}=\Phi_\beta|_{U_{\alpha}\cap U_{\beta}}$,则存在唯一的态射$\Phi:(X,\mathcal{O}_{X})\to (Y,\mathcal{O}_Y)$使得$\Phi|_{U_{\alpha}}=\Phi_\alpha$.
\end{pro}

\begin{proof} 连续函数在开覆盖上的黏合是显然的,对于层的态射那部分,用的技术无外乎层上截面的黏合,这是很琐碎的检验,这里略去了。\end{proof}

下面一个命题是类似的,但我们给一个证明。

\begin{pro}\label{rsnh}
设$(X,\mathcal{O}_X)$是一个赋环空间,$\mathfrak{B}$是$X$的一个拓扑基。如果任取$U_\alpha\in \mathfrak{B}$都给一个态射$\Phi_\alpha:(U_{\alpha},\mathcal{O}_{U_{\alpha}})\to (Y,\mathcal{O}_Y)$,且任取两个指标$(\alpha,\beta)$,存在一个指标$\gamma$使得$U_\gamma\subset U_\alpha\cap U_\beta$且$\Phi_\alpha|_{U_\gamma}=\Phi_\beta|_{U_\gamma}$,则存在唯一的态射$\Phi:(X,\mathcal{O}_{X})\to (Y,\mathcal{O}_Y)$使得$\Phi|_{U_{\alpha}}=\Phi_\alpha$.
\end{pro}

\begin{proof}
设$\Phi_\alpha=(\psi_\alpha,\theta_a)$,这里的$\psi_\alpha$的黏合是显然的,得到$\psi$. 现在我们考虑$\theta_\alpha:\oo_Y\to (\psi_\alpha)_*\oo_{U_\alpha}$. 任取$V\subset Y$,我们有
\[
	\theta_\alpha(V):\oo_Y(V)\to (\psi_\alpha)_*\oo_{U_\alpha}(V)=\Gamma(\psi_\alpha^{-1}(V),\oo_{U_\alpha})=\Gamma(U_\alpha\cap\psi^{-1}(V),\oo_{X}),
\]
任取$s\in \oo_Y(V)$,我们得到了一族截面$t_\alpha=\theta_\alpha(V)(s)\in\Gamma(U_\alpha\cap\psi^{-1}(V),\oo_{X})$. 黏合条件告诉我们,任取两个指标$(\alpha,\beta)$,存在一个指标$\gamma$使得$U_\gamma\subset U_\alpha\cap U_\beta$且$t_\alpha|_{U_\gamma}=t_\beta|_{U_\gamma}$. 于是,层公理(见Proposition \ref{psgl})告诉我们存在$t\in\Gamma(\psi^{-1}(V),\oo_{X})$使得$t|_{U_\alpha}=t_\alpha$. 于是$s\mapsto t$就给出了态射$\theta(V):\oo_Y(V)\to \Gamma(\psi^{-1}(V),\oo_{X})=\psi_*\oo_X(V)$. 此即所证。
\end{proof}

\begin{pro}[赋环空间的黏合]\label{rsn}
设$(X_\alpha,\oo_\alpha)$是一族赋环空间,并且给定任意一个二元组$(\alpha,\beta)$,都有一个$X_\alpha$的开集$U_{\alpha\beta}$,$X_\beta$的开集$U_{\beta\alpha}$,以及赋环空间同构$\varphi_{\alpha\beta}:(U_{\alpha\beta},\oo_\alpha|_{U_{\alpha\beta}})\to (U_{\beta\alpha},\oo_\beta|_{U_{\beta\alpha}})$,满足:
\begin{compactenum}[~~~1.]
\item $U_{\alpha\alpha}=X_\alpha$,此时$\varphi_{\alpha\alpha}$是恒同态射。
\item 任取三元组$(\alpha,\beta,\gamma)$,若以$\varphi'_{\alpha\beta}$记$\varphi_{\alpha\beta}$在$U_{\alpha\beta}\cap U_{\alpha\mu}$上的限制,则它应是一个同构,满足$\varphi'_{\alpha\gamma}=\varphi'_{\alpha\beta}\varphi'_{\beta\gamma}$.
\end{compactenum}

将$\{X_\alpha\}$利用同构$\{\varphi_{\alpha\beta}\}$沿着$\{U_{\alpha\beta}\}$黏成一个拓扑空间$X$,此时$X_\alpha$可以等同于$X$的开子集$X'_\alpha$,而$U_{\alpha\beta}$等同于$X'_\alpha\cap X'_\beta$. 于是,利用层的黏合,存在一个赋环空间$(X,\oo)$还有一族同构$\varphi_\alpha:(X|_{U'_\alpha},\oo|_{U'_\alpha})\to (X_\alpha,\oo_\alpha)$.
\end{pro}

最后我们给一个比较一般情况的逆像的存在性的证明。

\begin{thm}\label{proinverse}
设$\psi:X\to Y$是一个连续映射,$\calf$是$X$上的一个$K$-层,而$\calg$是$Y$上的一个$K$-预层。如果范畴$K$是交换群范畴,则$\calg$关于$\psi$的逆像层$\psi^*\calg$存在。
\end{thm}

由于交换群作为$\zz$-模,对任意族都一定存在极限. 所以从上面的引理,逆像的存在性就归结到了伴随层的存在性上面。不过下面我们还是直接证明这个命题。

\begin{proof} 
	证明是直接的构造。交换群范畴,我们知道预层的茎总是存在的。所以我们这样定义$\psi^*\calg$:对$X$中的开集$U$, 令$\psi^*\calg(U)$是函数$s:U\to \bigoplus_{x\in U}\calg_{\psi(x)}$的集合(若只是集合范畴,考虑不交并即可),这些函数需要满足
	\begin{itemize}
		\item 对任意的$x\in U$, $s(x)\in \calg_{\psi(x)}$,以及

		\item 对任意的$x\in U$,总存在$\psi(x)$的一个开邻域$V$,$x$的一个邻域$W\subset \psi^{-1}(V)\cap U$和一个元素$t\in \calg(V)$,使得当$p\in W$的时候总有$s(p)=t_{\psi(p)}$.
	\end{itemize}
	$\psi^*\calg(U)$上的加法直接定义为$(s+t)(x)=s(x)+t(x)$,因此$\psi^*\calg(U)$是一个交换群。如果$V$是$U$的开子集,那么函数限制$i^U_V:\psi^*\calg(U)\to \psi^*\calg(V)$就是所需要的预层的限制映射,记$i^U_V(s)=s|_V$,因此$\psi^*\calg$是一个预层。同时$\psi^*\calg$也是一个层,这直接来自于函数的粘合。

	此外,成立典范同构$(\psi^*\calg)_x\cong \calg_{\psi(x)}$. 实际上,$s\mapsto s(x)$给出了态射$\psi^*\calg(U)\to \calg_{\psi(x)}$,他与限制映射相容,所以由余极限的泛性质,存在唯一的态射$\theta:(\psi^*\calg)_x\to \calg_{\psi(x)}$满足$\theta(s_x)=s(x)$. 它显然是单射,因为如果$s_x=t_x$,那么存在一个领域$U$使得$s|_U=t|_U$,所以$s(x)=s|_U(x)=t|_U(x)=t(x)$. 同时,它也是满射,任取$t_{\psi(x)}=(V,t)\in \calg_{\psi(x)}$,其中$V$是$\psi(x)$的一个邻域,则取$x$附近的一个邻域$U$使得$\psi(U)\subset V$,在$\psi^*\calg(U)$上,我们定义$s:p\mapsto t_{\psi(p)}$. 于是$t_{\psi(x)}=s(x)$. 因此,我们一般会通过$s_x=s(x)$来等同$(\psi^*\calg)_x$和$\calg_{\psi(x)}$. 下面我们默认这种等同。

	现在设$U$是$Y$的一个开集,对于$s\in \calg(U)$, 通过$s^+(x)=s_{\psi(x)}$可以定义一个截面
	\[
		s^+\in \psi^*\calg(\psi^{-1}(U))=\psi_*\psi^*\calg(U),
	\]
	其中$x\in \psi^{-1}(U)$. 于是,$\psi^*_{\calg}(U):s\mapsto s^+$定义了态射$\psi^*_{\calg}(U):\calg(U)\to \psi_*\psi^*\calg(U)$. 下面我们检验$\psi^*_{\calg}$是一个预层之间的态射,即一个自然变换。

	对任意的$s\in \calg(U)$,因为
	\[
		(\psi_* i)^U_V\bigl(\psi^*_{\calg}(U)(s)\bigr)=i^{\psi^{-1}(U)}_{\psi^{-1}(V)}\left(\{x\mapsto s_{\psi(x)}\,:\, x\in \psi^{-1}(U)\}\right)=\{x\mapsto s_{\psi(x)}\,:\, x\in \psi^{-1}(V)\}=(s|_V)^+,
	\]
	以及
	\[
		\psi^*_{\calg}(V)\bigl(\rho^U_V(s)\bigr)=\psi^*_{\calg}(V)(s|_V)=(s|_V)^+.
	\]
	所以$\psi^*_{\calg}$是一个预层之间的态射。

	最后就是要检验泛性质,即对于任意的态射$u:\calg \to \psi_*\calf$,都可以找到唯一的态射$u^\#:\psi^*\calg \to \calf$成立分解
	\[
		u:\calg\xrightarrow{\psi^*_{\calg}} \psi_*\psi^*\calg \xrightarrow{\psi_*u^\#} \psi_*\calf.
	\]

	固定$U$是$X$的开子集以及$s\in \psi^*\calg(U)$,在每一点$x\in U$附近,我们可以找到一个开子集$W_x$, 一个$\psi(x)$附近的开子集$V_x$使得$W_x\subset \psi^{-1}(V_x)\cap U$以及一个$t(x)\in \calg(V_x)$使得$p\in W_x$的时候$s(p)=t(x)_{\psi(p)}$恒成立。遍历$x\in U$,$\{W_x\}_{x\in U}$是$U$的一个开覆盖,由于$\calf$是一个层,可以把$\bigl(u(V_x)(t(x))\bigr)|_{W_x}$拼成$U$上的一个截面$s'$. 实际上,利用Lemma \ref{lem:1},我们只要证明$\bigl(u(V_x)(t(x))\bigr)|_{W_x}$的芽不依赖选取的$x$即可。任取$p\in W_x$,我们有
	\[
	\bigl(u(V_x)(t(x))\bigr)_p=\psi_p\left(u(V_x)(t(x))_{\psi(p)}\right)=\psi_p\circ u_{\psi(p)}(t(x)_{\psi(p)})=\psi_p\circ u_{\psi(p)}(s(p)).
	\]
	于是,定义$u^\#(U):s\mapsto s'$就得到了层的态射$u^\#:\psi^*\calg \to \calf$的完整定义,显然这由$u$唯一确定。具体到茎间,即
	\[
	u^\#_x(s_x)=(\psi_*^{\calf})_x u_{\psi(x)}(s(x))=(\psi_*^{\calf})_x u_{\psi(x)}(s_x),
	\]
	这里$(\psi_*^{\calf})_x:(\psi_*\calf)_{\psi(x)}\to \calf_x$是典范态射。所以$u^\#_x=(\psi_*^{\calf})_x u_{\psi(x)}$. 

	证明的最后一步就是检验分解是否成立。任取$r\in \mathcal{G}(V)$,则$u(V)(r)$和$\psi_*u^\#(V)\circ \psi^*_\calg(V)(r)$都是$\mathcal{F}$在$\psi^{-1}(V)$上的截面,任取$x\in \psi^{-1}(V)$,它们局部可以写作
	\[
	(u(V)(r))_x=(\psi_*^{\calf})_x (u(V)(r))_{\psi(x)}=(\psi_*^{\calf})_x u_{\psi(x)}(r_{\psi(x)})=u^\#_x(r_{\psi(x)})
	\]
	和
	\[
	\left(\psi_*u^\#(V)\circ \psi^*_\calg(V)(r)\right)_x=\bigl(u^\#(\psi^{-1}(V))(r^+)\bigr)_x=u^\#_x(r^+_x)=u^\#_x(r_{\psi(x)}).
	\]
	所以这两个截面相同。又由于$r$和$x$是任意的,分解$u(V)=\psi_*u^\#(V)\circ \psi^*_\calg(V)$成立。
\end{proof}

证明之中的一些映射是值得注意的。由于$\psi^*_{\calg}$是预层之间的态射,所以任取$x\in X$,它诱导了
\[
	(\psi^*_{\calg})_{\psi(x)}:\calg_{\psi(x)}\to (\psi_*\psi^*\calg)_{\psi(x)}.
\]
结合典范态射$(\psi_*^{\psi^*\calg})_x:(\psi_*\psi^*\calg)_{\psi(x)}\to (\psi^*\calg)_{x}$,我们可以得到典范态射
\[
	(\psi_*^{\psi^*\calg})_x(\psi^*_{\calg})_{\psi(x)}:\calg_{\psi(x)}\to (\psi^*\calg)_{x},
\]
按照我们之前的论断,$\calg_{\psi(x)}$和$(\psi^*\calg)_{x}$之间存在典范同构,实际上,它就是$(\psi_*^{\psi^*\calg})_x(\psi^*_{\calg})_{\psi(x)}$. 为说明这点,只需说明在前面默认的等同下这个态射是恒等即可。任取$r_{\psi(x)}=(V,r)\in \calg_{\psi(x)}$,其中$V$是$\psi(x)$的一个邻域,我们有
\[
	(\psi^*_{\calg})_{\psi(x)}r_{\psi(x)}=(\psi^*_{\calg}(V)(r))_{\psi(x)}=(r^+)_{\psi(x)},
\]
其中$r^+\in \psi_*\psi^*\calg(U)$. 此时
\[
	(\psi_*^{\psi^*\calg})_x(\psi^*_{\calg})_{\psi(x)}(r_{\psi(x)})=(\psi_*^{\psi^*\calg})_x((r^+)_{\psi(x)})=r^+_x=r^+(x)=r_{\psi(x)},
\]
此即所证。

此外,不难注意到,在我们的构造中
\[
	(f^\calf_*)_x=(\id_{f_*\calf})^\#_x:(f^*f_*\calf)_x=(f_*\calf)_{f(x)}\to \calf_x
\]
就是典范态射$(f^\calf_*)_x:(f_*\calf)_{f(x)}\to \calf_x$.

\begin{pro}
设$f:\calf\to\calg$是两个$Y$上的交换群层之间的态射,而$\psi:X\to Y$是一个连续映射。于是,局部地,态射$\psi^*f:\psi^*\calf\to \psi^*\calg$写作$(\psi^* f)_x=f_{\psi(x)}$.
\end{pro}

\begin{proof}
由如下双函子同构,
\[
	\xymatrix{
		\Hom_X(\psi^*\calg,\mathcal{H})\ar[rr] \ar[d]_{(\psi^*f)^*}&&\Hom_Y(\calg,\psi_*\mathcal{H}) \ar[d]^{f^*}\\
		\Hom_X(\psi^*\calf,\mathcal{H})\ar[rr]&&\Hom_Y(\calf,\psi_*\mathcal{H})
	}
\]
任取态射$u:\calg\to \psi_*\mathcal{H}$,应当成立
\[
	u^\#\circ \psi^* f=(uf)^\#.
\]
特别地,取$u=\psi_\calg^*$,则应有
\[
	\psi^* f=(\psi_\calg^* f)^\#.
\]
所以
\[
	(\psi^* f)_x=(\psi_\calg^* f)^\#_x =(\psi_*^{\psi^*\calg})_x (\psi_\calg^* f)_{\psi(x)}=(\psi_*^{\psi^*\calg})_x (\psi_\calg^*)_{\psi(x)} f_{\psi(x)}=f_{\psi(x)}.
\]
这里应用了上面我们约定的等同。
\end{proof}

\begin{pro}[逆像函子的复合]
考虑两个连续映射$f:X\to Y$, $g:Y\to Z$,则Theorem \ref{proinverse} 构造的逆像函子满足$(gf)^*=f^*g^*$.
\end{pro}

\begin{proof}
% 已经知道,从$(gf)_*=g_*f_*$,我们有函子性同构
% \[
% 	\Hom_Z((gf)^*\calg,\calf)\cong \Hom_Z(\calg,(gf)_*\calf)=\Hom_Z(\calg,g_*f_*\calf)\cong \Hom_Y(g^*\calg,f_*\calf)\cong \Hom_X(f^*g^*\calg,\calf),
% \]
% 所以$(gf)^*\cong f^*g^*$,这里的同构是函子的同构,即存在互逆的自然变换,所以我们只要检验这些个自然变换是恒同即可。
任取$x\in X$以及$Z$上的层$\mathcal H$,在我们的构造中,上述自然变换诱导了等同
\[
	((gf)^*\mathcal H)_x=\mathcal H_{g(f(x))}=(g^* \mathcal H)_{f(x)}=(f^*(g^*\mathcal H))_x,
\]
此外,任取$Z$上的层之间的态射$u$,自然变换诱导了等同
\[
	((gf)^* u)_x= u_{g(f(x))}=(g^* u)_{f(x)}=(f^*(g^* u))_x,
\]
由于$x$是任意的,所以$(gf)^*\mathcal H=f^*(g^*\mathcal H)$以及$(gf)^* u=f^*g^* u$.
\end{proof}

从另一个角度来看,给定一对伴随函子,则我们可以唯一确定态射集之间的同构,这就意味着,在我们的构造中,双函子同构的复合
\[
	\Hom_Z(\calg,g_*f_*\calf)\cong \Hom_Y(g^*\calg,f_*\calf)\cong \Hom_X(f^*g^*\calg,\calf)
\]
与双函子同构
\[
	\Hom_Z(\calg,(gf)_*\calf)\cong \Hom_Z((gf)^*\calg,\calf)
\]
是相同的。特别地,考虑$\id_{(gf)_*\calf}\in \Hom_Z((gf)_*\calf,(gf)_*\calf)$,则应该有
\[
	(gf)^*_{\calf} = f_{\calf}^* f^*(g_{f_*\calf}^*).
\]

\begin{para}
	作为上面命题及其推论的应用,考虑两个赋环空间的态射$(f,\varphi):X\to Y$和$(g,\psi):Y\to Z$,层之间的复合已知写作
	\[
		\oo_Z\xrightarrow{\psi} g_*\oo_Y\xrightarrow{g_*\varphi} g_*f_*\oo_X,
	\]
	有时候,我们需要考虑$((g_*\varphi)\psi)^\#:(gf)^*\oo_Z\to \oo_X$. 此时,注意到我们有另一种方式
	\[
		\varphi^\#(f^*\psi^\#):(gf)^*\oo_Z=f^*g^*\oo_Z\xrightarrow{f^*\psi^\#}f^*\oo_Y\xrightarrow{\varphi^\#}\oo_X
	\]
	来得到$f^*g^*\oo_Z\to \oo_X$的态射,所以上一个命题的推论告诉我们$((g_*\varphi)\psi)^\#=\varphi^\#(f^*\psi^\#)$. 

	当然,我们也可以没那么聪明地,考虑典范分解
	\[
		((g_*\varphi)\psi)^\#=(gf)^*_{\oo_X} (gf)^*((g_*\varphi)\psi)=(gf)^*_{\oo_X} (f^*g^*g_*\varphi)((gf)^*\psi).
	\]
	首先,注意到$(gf)^*_{\oo_X}$可以由上一个命题的推论分解为两个态射$(gf)^*_{\oo_X}=f_{\oo_X}^* f^*(g_{f_*\oo_X}^*)$,所以
	\[
		((g_*\varphi)\psi)^\#=f_{\oo_X}^* f^* \left(g_{f_*\oo_X}^* g^*g_*\varphi\right)((gf)^*\psi)=f_{\oo_X}^* f^* \left(g_{f_*\oo_X}^* g^*g_*\varphi g^*\psi\right).
	\]
	接着,考虑交换图
	\[
		\xymatrix{
			g^*g_*\oo_Y \ar[d]_{g^*g_*\varphi}\ar[rr]^-{g_{\oo_Y}^*}&&\oo_Y\ar[d]^\varphi\\
			g^*g_*f_*\oo_X\ar[rr]^-{g_{f_*\oo_X}^*}&&f_*\oo_X
		}
	\]
	即$\varphi g_{\oo_Y}^*=g_{f_*\oo_X}^*(g^*g_*\varphi)$给出了
	\[
		((g_*\varphi)\psi)^\#=f_{\oo_X}^* f^* \left(\varphi g_{\oo_Y}^* g^*\psi\right)=f_{\oo_X}^* f^* \left(\varphi \psi^\#\right)=f_{\oo_X}^* f^* \varphi f^*(\psi^\#)=\varphi^\# f^*(\psi^\#).
	\]
	这就是我们相当笨重但是却很扎实的证明。
\end{para}

将其应用到茎上,我们立即得到
\[
	((g_*\varphi)\psi)^\#_x=\varphi^\#_x(f^*\psi^\#)_x=\varphi^\#_x \psi^\#_{f(x)},
\]
此即命题:

\begin{pro}\label{pro:morcom}
	设有两个赋环空间的态射$(f,\varphi):X\to Y$和$(g,\psi):Y\to Z$,层之间的复合已知写作
	\[
		\theta:\oo_Z\xrightarrow{\psi} g_*\oo_Y\xrightarrow{g_*\varphi} g_*f_*\oo_X,
	\]
	则$\theta^\#_x:(\oo_Z)_{g(f(x))}\to (\oo_X)_x$可以分解为
	\[
		\theta^\#_x:(\oo_Z)_{g(f(x))}\xrightarrow{\psi^\#_{f(x)}}(\oo_Y)_{f(x)}\xrightarrow{\varphi^\#_x} (\oo_X)_x.
	\]
\end{pro}

后面我们在研究概形态射的复合的时候,这个命题可能是有益的。

\begin{para}[一些总结]
	让我们限制取值在交换群范畴的层和预层,总结一下上面这些乱七八糟的等式或者关系。设$f:X\to Y$, $g:Y\to Z$是连续映射,$\calf$, $\calg$, $\mathcal H$是预层或层,看具体语境。
	\begin{compactenum}
		\item 定义
		\[
			f_*^\calf:=\left(\id_{f_*\calf}\right)^\#: f^*f_*\calf \to \calf, \quad 
			f^*_\calg:=\left(\id_{f^*\calg}\right)^\flat : \calg \to f_*f^*\calg.
		\]
		他们使得态射$u:\calg\to f_*\calf$和$v:f^*\calg\to\calf$的提升和下降可以写为
		\[
			u^\# =f_*^\calf f^*(u),\quad v^\flat = f_*(v) f^*_\calg.
		\]
		\item 局部地,对(预)层有$(f^*\calf)_x=\calf_{f(x)}$,对态射有$(f^*u)_x=u_{f(x)}$,而提升写作$v^\#_x=(f_*^{\calf})_x v_{f(x)}$.
		\item $(f_*^\calf)_x:(f_*\calf)_{f(x)}\to \calf_x$就是$f_*:\calf\to f_*\calf$在点$x$由余极限的泛性质诱导的典范同态。
		\item 若等同$\calg_{f(x)}$和$(f^*\calg)_{x}$,则以下典范同构是恒等态射
		\[
			(f_*^{f^*\calg})_x(f^*_{\calg})_{f(x)}:\calg_{f(x)}\to (f^*\calg)_{x}.
		\]
		\item 设$u:\calf\to \calg$而$v:\calg\to f_*\mathcal H$是层同态,则$(vu)^\# = u^\# f^*u$.
		\item 两个连续映射的复合下,有
		\[
			(gf)^*_{\calf} = f_{\calf}^* f^*(g_{f_*\calf}^*).
		\]
		\item 设有两个赋环空间的态射$(f,\varphi):X\to Y$和$(g,\psi):Y\to Z$,他们的复合的态射的层部分为$\theta = g_*\varphi \psi$,则提升$\theta^\#_x:(\oo_Z)_{g(f(x))}\to (\oo_X)_x$局部可以分解为$\theta^\#_x=\varphi^\#_x\psi^\#_{f(x)}$.
	\end{compactenum}
\end{para}

\begin{para}[交换群层范畴]
$X$上的交换群层构成一个范畴,态射就是层之间的态射。并且,这是一个准加性范畴,常值层$0$(即任取开集$U$,$0(U)=0$)是它的零对象。

由于极限与极限是可以交换的,所以一族交换群层的极限依然是一个层。特别地,我们可以断言,任意有限极限是存在的。所以$X$上的交换群层构成一个加性范畴。

特别地,由于核是一个极限,所以对
态射$u:\calf\to\calg$,预层$U\mapsto \ker(u(U))$是一个层,我们记作$\ker u$. 结合典范含入
\[
	i(U):\ker(u(U))\hookrightarrow \calf(U),
\]
它们构成了$u$的核。因此,$X$上的交换群层范畴存在核。

在茎上,我们有$(\ker u)_x=\ker(u_x)$. 实际上,任取$(U,s)\in (\ker u)_x$,由于此时$s\in \ker u(U)$即$u(U)(s)=0$,所以$u_x(U,s)=(U,u(U)(s))=0$. 反之,任取$(U,s)\in \ker(u_x)$,我们应有$(U,u(U)(s))=0$,所以存在一个$V\subset U$使得$u(V)(s|_V)=0$,此即说明$(U,s)=(V,s|_V)\in (\ker u)_x$.

对余核,预层$U\mapsto \coker(u(U))$不一定是一个层,暂时将这个预层记作$\mathcal{K}$,所以我们转而考虑其伴随层$\mathcal{K}^+$. 可以检验,$\mathcal{K}^+$就是$u$的余核。实际上,对每一个$X$的开集$U$,我们考虑如下交换图
\[
	\xymatrix{
		\calf(U)\ar@<0.3ex>[r]^-{u(U)} \ar@<-0.3ex>[r]_-{0}&\calg(U)\ar[r]\ar[dr]_-{v(U)}&\coker(u(U))=\mathcal{K}(U)\ar[d]\ar[r]^-{i(U)}&\mathcal{K}^+(U)\ar@{-->}[ld]\\
		&&\mathcal{H}(U)&
	}
\]
图中略去了显然的典范态射的符号,而$i:\mathcal{K}\to \mathcal{K}^+$是预层到伴随层的典范态射,虚线直接来自于伴随层的泛性质。所以$\mathcal{K}^+$正是$u$的余核,我们记作$\coker u$. 于是,$X$上的交换群层范畴存在余核。
\end{para}

\begin{lem}
在茎上,我们有$(\coker u)_x\cong \coker(u_x)$. 
\end{lem}

一般而言,我们会直接等同$(\coker u)_x$和$\coker(u_x)$,这个并不会有什么问题,因为在茎是一个余极限,本身就容许一个同构。此时$(\coker u)_x=\coker(u_x)$.

\begin{proof}
预层与伴随层的茎是相同的,所以我们只需考虑预层$U\mapsto \coker(u(U))$. 首先记$\pi(U):\calg(U)\to \coker (u(U))$为典范同态,$\pi:\calg\to \coker u$是一个预层同态,以及$p_x:\calg_x\to \coker(u_x)$是$u_x:\calf_x\to\calg_x$对应的典范同态。

现在考虑如下交换图
\[
	\xymatrix{
	\calf(U)\ar[r]^{u(U)}\ar[d]^{\mu^U_x}&\calg(U)\ar[r]^-{\pi(U)}\ar[d]^{\nu^U_x}&\coker (u(U))\ar@{-->}[d]^{\alpha(U)}\\
	\calf_x\ar[r]^{u_x}&\calg_x\ar[r]^-{p_x}&\coker (u_x)
	}
\]
其中$\mu$, $\nu$是对应的限制映射。由于$0=0\mu^U_x=p_xu_x\mu^U_x=p_x\nu^U_xu(U)$,由$\coker(u(U))$的泛性质,我们有唯一分解$p_x\nu^U_x=\alpha(U)\pi(U)$,此即图中$\alpha(U)$的由来。并且,它与限制映射相容,这是分解的唯一性保证的。再由余极限的泛性质,我们得到了$\alpha_x:(\coker u)_x\to \coker (u_x)$使得$\alpha(U)=\alpha_x\pi^U_x$. 我们下面证明$\alpha_x$是一个同构。

首先,证明$\alpha_x$是一个单同态,技术上就是追图。任取$s_x\in (\coker u)_x$使得$\alpha_x(s_x)=0$. 设$s_x=\langle U,s\rangle$,其中$s\in \coker (u(U))$,于是$0=\alpha_x(s_x)=\alpha_x \pi_x^U(s)=\alpha(U)(s)$. 由于$\pi(U)$是满同态,故存在$t\in \calg(U)$使得$s=\pi(U)(t)$. 由上面的交换图,我们有
\[
	0=\alpha(U)\circ \pi(U)(t)=p_x\nu_x^U(t)=p_x(t_x).
\]
从交换群中熟知的构造$\coker(u_x)=\calg_x/\im u_x$,所以$t_x\in \im u_x$,即存在一个$r_x\in \calf_x$使得$t_x=u_x(r_x)$. 因此存在$U$的开子集$V$以及$r_x=\langle V,r\rangle$使得$t_x=\nu_x^V\circ u(V)(r)$,所以存在一个$V$的开子集$W$使得$t|_W=u(V)(r)|_W=u(W)(r|_W)$. 综上
\[
	s|_W=\pi(W)(t|_W)=\pi(W)\circ u(W)(r|_W)=0,
\]
这就给出了$s_x=0$.

然后证明$\alpha_x$是满同态。任取$a\in \coker(u_x)$,由于$p_x$是满同态,所以存在$b_x\in\calg_x$使得$a=p_x(b_x)$. 设$b_x=\langle U,b\rangle$,所以$\alpha(U)\circ \pi(U)(b)=a$,又因为$\alpha(U)=\alpha_x\pi^U_x$,所以$a=\alpha_x\left(\pi^U_x\circ \pi(U)(b)\right)$. 因此$\alpha_x$是一个满同态。
\end{proof}

稍微提一句,在等同下$(\coker u)_x=\coker(u_x)$,上述证明中就会有等同$\pi_x=p_x$. 因此,如果将余核的典范态射依然记作$\coker$的话,我们还是有$\coker(u_x)=(\coker u)_x$,它们都是$u_x:\calf_x\to\calg_x$的余核,而$\coker u$是$u:\calf \to \calg$的余核。对$\ker$这点也是类似的。

\begin{para}[商层]\label{qsheaf}
设$\calf$, $\calg$是$X$上的交换群层,如果对每一个$U$,$\calg(U)$是$\calf(U)$的子群,且典范含入$i(U):\calg(U)\to\calf(U)$是一个层的态射,则称$\calg$是$\calf$的子层。在模范畴中,商定义为典范含入的余核。现在我们依然如此定义,定义$\calf/\calg$为层$\coker(i)$,具体来说,$\calf/\calg$是预层$U\mapsto \calf(U)/\calg(U)$的伴随层。它被称为商层。在茎上,我们有$(\calf/\calg)_x=\calf_x/\calg_x$.

就像在任何加性范畴中那样,我们依然会用$\ker u$和$\coker u$来记典范态射。此时,定义$\im u=\ker\coker u$,从核与余核的知识,这当然是一个层。实际上,它就是$U\mapsto \im(u(U))$的伴随层,利用Lemma \ref{lem:1},只要说明它们拥有相同的茎即可。注意到预层和其伴随层的茎相同,所以一个是$x\mapsto \im u_x$,另一个是$x\mapsto \ker\coker u_x=\im u_x$,它们相同。
\end{para}

最后,我们考虑单态射和满态射。所谓单态射,就是说态射是左可消的,所谓满态射,就是说态射是右可消的。在准加性范畴中,$u$是单态射当且仅当$\ker u=0$,$u$是满态射当且仅当$\coker u=0$. 

\begin{lem}
设$\calf$, $\calg$是两个$X$上的交换群层,则$u:\calf\to\calg$是一个单(满)态射当且仅当$u_x:\calf_x\to\calg_x$是单(满)同态对每一个$x\in X$都成立。
\end{lem}

所以,如果$u_x$处处是同构,则$u$也是同构。因为在Abel范畴中,既单又满即同构。

\begin{proof}
单的部分直接来自于Lemma \ref{lem:1}. 所以我们假设$u:\calf\to\calg$是一个满态射,或者说$\coker(u)=0$. 首先,由于伴随层的茎不变,所以$\coker u_x=(\coker u)_x=0$,因此$u_x$是满同态。

反过来,给定任意态射$v:\calg\to\mathcal{H}$,如果$vu=0$,则其等价于$(vu)_x=v_xu_x=0$. 由于$u_x$处处是满的,所以$v_x=0$. 利用Lemma \ref{lem:1},我们得到了$v=0$. 所以$u$是一个满态射。
\end{proof}

\begin{para}
下面我们说明,单态射是一个核,而满态射是一个余核。因此$X$上的交换群层构成一个Abel范畴。

不难检验,如果$u:\calf\to\calg$是一个单态射,则$u=\ker(\coker(u))$. 检验这点并不困难,利用Lemma \ref{lem:1},只要它们在每一点上的茎相同即可,而
\[
	\ker(\coker(u))_x=\ker(\coker(u)_x)=\ker(\coker(u_x))=u_x,
\]
因为$u_x$是单的。

类似地,如果$u:\calf\to \calg$是一个满射,则$u=\coker (\ker(u))$. 实际上
\[
	\coker (\ker(u))_x=\coker(\ker(u)_x)=\coker(\ker(u_x))=u_x,
\]
因为$u_x$是满的。
\end{para}

\begin{pro}
在$X$上的交换群层的Abel范畴和$Y$上的交换群层的Abel范畴间,函子$\psi^*$是一个正合函子。
\end{pro}

\begin{proof}
首先它当然是右正和的,因为左伴随。所以我们只要证明他将单态射变成单态射即可。考虑$f:\calf \to \calg$是一个单态射,所以$f_{\psi(x)}:\calf_{\psi(x)} \to \calg_{\psi(x)}$都是单的。在等同$(\psi^* f)_x=f_{\psi(x)}$下,他给出了
\[
	(\psi^* f)_x=f_{\psi(x)}:(\psi^*\calf)_{x} \to (\psi^*\calg)_{x}
\]
都是单的,而这又等价于$\psi^* f$是单的。
\end{proof}

% \para 设$w:\calg_1\to \calg_2$是$Y$上的一个$K$-预层态射,如果$\calg_2$和$\calg_2$关于$\psi:X\to Y$的逆像是存在的,则复合$\rho_{\calg_2}\circ w:\calg_1\to \psi_*\psi^*\calg_2$,由逆像的泛性质,我们可以找到一个态射$v:\psi^*\calg_1\to \psi^*\calg_2$来分解上面的复合态射,我们将其记做$\psi^*(w)$,他将$Y$上的$K$-预层态射,变成了$X$上的$K$-层态射。这里不加验证,$\psi^*(w_1)\circ \psi^*(w_2)=\psi^*(w_1\circ w_2)$,所以这是一个协变函子。

\section{模层}

\begin{para}[模层]
设$(X,\mathcal{O}_X)$是一个赋环空间,我们称$\calf$是一个$\mathcal{O}_X$-\idx{模层},如果对每一个开集$U$,$\calf(U)$都是一个$\mathcal{O}_X(U)$-模,并且$\calf$和$\mathcal{O}_X$的限制映射是相容的,即:设$V\subset U$是$X$的开子集,记$\calf$的限制映射为$\pi^U_V$,相应地$\mathcal{O}_X$的限制映射为$\rho^U_V$,那么对于标量乘法而成的截面$a\cdot s\in\calf(U)$,则$\pi^U_V(a\cdot s)=\rho^U_V(a)\cdot \pi^U_V(s)$. 类似地,可以定义$\mathcal{O}_X$-\idx{代数层}:环层$\calf$被称为$\mathcal{O}_X$-代数层,如果他是一个$\mathcal{O}_X$-模层。
\end{para}

我们知道$\mathcal{O}_X(U)$自己作为$\mathcal{O}_X(U)$-模的子模就是$\mathcal{O}_X(U)$的理想,所以一个$\mathcal{O}_X$的$\mathcal{O}_X$-子模层,被称为$\mathcal{O}_X$-\idx{理想层}。

\para 考虑任意的小范畴$I$,以及一系列$K$-层$\{\calf_i\}_{i\in I}$,由于极限的极限还是一个极限,所以可以看到预层$U\mapsto \varprojlim_{i\in I}\calf_i(U)$依然是一个$K$-层。他被称作\idx{极限层},记做$\varprojlim_{i\in I}\calf_i$. 放到模范畴中,考虑一系列$\mathcal{O}_X$-模层$\{\calf_i\}_{i\in I}$,则$\prod_{i\in I}\calf_i$依然是一个$\mathcal{O}_X$-模层,他被称为\idx{直积层}。

但对于直和,箭头反过来了导致预层$U\mapsto \bigoplus_{i\in I}\calf_i(U)$就不一定是一个层。所以我们定义\idx{直和层}为上述预层的伴随层,记做$\bigoplus_{i\in I}\calf_i$. 但对于有限指标集$I$,直和和直积等价,所以$U\mapsto \bigoplus_{i\in I}\calf_i(U)$此时就是一个层。

\begin{para}
考虑一个$\mathcal{O}_X$-模层同态$u:\mathcal{O}_X\to \calf$,这个同态完全由截面$s=u(X)(1)\in \calf(X)$决定,其他的任意截面$t\in \mathcal{O}_X(U)$,则$u(U)(t)=t\cdot (s|_U)$.

类似地,考虑直和层$\bigoplus_{i\in I}\mathcal{O}_X=\mathcal{O}_X^{(I)}$,对每一个$i\in I$,都有典范含入$h_i:\mathcal{O}_X\to \mathcal{O}_X^{(I)}$. 设$\calf$是一个$\mathcal{O}_X$-模层,而$u:\mathcal{O}_X^{(I)}\to \calf$是一个$\mathcal{O}_X$-模层同态,现在考虑复合$u\circ h_i:\mathcal{O}_X\to \calf$,他被截面$s_i=(u\circ h_i)(X)(1)$唯一确定,故$u$被$\{s_i\}_{i\in I}$唯一确定,对应着$u(U)\bigl(\bigoplus_i a_i\bigr)=\sum_i a_i\cdot (s_i|_U)$.
\end{para}

% 现在,我们来看所谓的有限生成模等概念如何推广到模层上面来。回忆一下,$R$-模$M$称被某个集合$E\subset M$生成,就是说,$M$是所有$s\in E$生成的自由模$R^{(E)}$的商模,即如下正和列成立$R^{(E)}\to M \to 0$. 所谓的有限生成,即是指存在一个有限集$E$生成$M$。

% 一个$\mathcal{O}_X$-模层$\calf$被称作被截面$\{s_i\}_{i\in I}$生成,就是说由截面$\{s_i\}_{i\in I}$定义的同态$\mathcal{O}_X^{(I)}\to \calf$是满射。类似地,$\calf$被称为\idx{有限生成}$\mathcal{O}_X$-模层,如果存在一族有限截面$\{s_i\}_{i\in I}$生成他。

% 换句话说,一个$\mathcal{O}_X$模层是有限生成的,如果存在如下正和列
% \[
% 	\bigoplus_{i=1}^n\mathcal{O}_X\to \calf\to 0,
% \]
% 其中$n$是一个正整数。

\begin{para}
设$u:\calf\to \calg$是$\mathcal{O}_X$-模层同态,他要使得$u(U)$是$\mathcal{O}_X(U)$-模$\calf(U)$和$\calg(U)$之间的同态。完全类似交换群层范畴,我们可以定义$U\to \ker(u(U))$, $U\to \im(u(U))$, $U\to \coker(u(U))$的伴随层为相应的$\mathcal{O}_X$-模层$\ker(u)$, $\im(u)$以及$\coker(u)$. $u$被称为满的,就是说$\im(u)=\calg$或$\coker u=0$,$u$被称为单的$\ker(u)=0$. 
\end{para}

和交换群层上类似,$\ker(u_x)=(\ker u)_x$, $\coker(u_x)=(\coker u)_x$, $\im(u_x)=(\im u)_x$. 同样,$\mathcal{O}_X$-模层构成一个Abel范畴,态射集记作$\Hom_{\oo_X}$,我们将在其中讨论正和列。下面是一个实用而简单的引理。

\begin{lem}
设$\calf\to \calg \to \mathcal{H}$是$\mathcal{O}_X$-模层的一个列,它是正和列当且仅当在每条茎上$\calf_x\to\calg_x\to \mathcal{H}_x$是正和列。
\end{lem}

\begin{para}[拟凝聚层]
一个$\mathcal{O}_X$-模层$\calf$被称作\idx{拟凝聚}的,就是说,对任意的$x\in X$,存在他的一个开邻域$U$使得$\calf|_U$同构于某个形如$\mathcal{O}_X^{(I)}|_U\to \mathcal{O}_X^{(J)}|_U$的同态的余核,此处$I$, $J$是任意的指标集。如果一个$\mathcal{O}_X$-代数层作为$\mathcal{O}_X$-模层是拟凝聚的,则他被称为拟凝聚$\mathcal{O}_X$-代数层。

换句话说,对拟凝聚$\mathcal{O}_X$-模层$\calf$,在每一点附近都有一个开邻域$U$使得如下正和列\footnote{顺带一提,对于任何一个$R$-模$M$而言,正合列
\[
	R^{(I)}\to R^{(J)}\to M\to 0
\]
总是存在的。实际上,取$J=M$,同态取为$i_M:1_m\mapsto m$. 由同构基本定理,我们有短正合列
\[
	0\to \ker i_M\hookrightarrow R^{(M)} \xrightarrow{i_M} M \to 0.
\]
现在取$I=\ker i_M$,将$i_{\ker i_M}$复合上$\ker i_M\hookrightarrow R^{(M)}$,我们就得到了
\[
	R^{(\ker i)}\to R^{(M)} \xrightarrow{i_M} M \to 0.
\]
他显然是正和的。}成立
\[
	\mathcal{O}_X^{(I)}|_U\to \mathcal{O}_X^{(J)}|_U\to \calf|_U\to 0.
\]
考虑显然的正和列$0\to \mathcal{O}_X^{(I)}\to \mathcal{O}_X^{(I)} \to 0$,所以$\mathcal{O}_X^{(I)}$本身就是拟凝聚$\mathcal{O}_X$-模层。
\end{para}

% 后面我们会看到,每一个环都有一个自然的拓扑空间,在这个拓扑空间上有一个环层,这俩构成一个局部赋环空间。对于一个$R$-模$M$,因为$R$已经诱导了一个局部赋环空间,我们在上面还将搞出一个模层,称为$M$的伴随层。所谓的拟凝聚模层就是局部是某个模的伴随层。这点可以类比流形,局部是欧几里得的。

\begin{para}[有限型模层]
称一个$\oo_X$-模层$\calf$是有限型模层,如果任取$x\in X$,都存在一个开集$U$和正整数$n_x$以及如下的正合列
\[
	\mathcal{O}^{n_x}_X|_U\to \calf|_U\to 0.
\]
显然,这是一个局部条件。
\end{para}

\begin{para}[凝聚层]
称一个$\oo_X$-模层$\calf$是凝聚的,如果$\calf$是一个有限型$\oo_X$-模层,且对任意开集$U\subset X$、任意正整数$n$以及任意同态
\[
	u:\mathcal{O}^{n}_X|_U\to \calf|_U,
\]
$\ker u$都是有限型模层。显然,凝聚层是一个拟凝聚层。
\end{para}

\begin{para}
设$\mathcal{F}$和$\mathcal{G}$都是$\mathcal{O}_X$-模层,我们定义$\mathcal{F}\otimes_{\mathcal{O}_X}\mathcal{G}$是预层$U\mapsto \mathcal{F}(U)\otimes_{\mathcal{O}_X(U)}\mathcal{G}(U)$的伴随层。如果张量积的下标是清楚的,我们一般会略去它。

类似模范畴中那样,我们有典范同构$\oo_X\otimes \calf\cong \calf$, $\calf\otimes \calg\cong \calg\otimes \calf$, $(\calf\otimes \calg) \otimes \mathcal{H}\cong \calf\otimes (\calg \otimes \mathcal{H})$,且这些同构都是函子性的。为了证明它,我们可以在每个开集上找到与限制映射相容的模同态,这族模同态构成了一个模层之间的态射,为证明这是同构,只需它在每一点处的茎上诱导的是同构就行。为此,我们需要下面这个命题,有了它,一切都是显然的了。
\end{para}

\begin{pro}
$(\mathcal{F}\otimes\mathcal{G})_x=\mathcal{F}_x\otimes\mathcal{G}_x$.
\end{pro}

\begin{proof}
我们直接验证$\mathcal{F}_x\otimes\mathcal{G}_x=\varinjlim_{U\ni x} \mathcal{F}(U)\otimes\mathcal{G}(U)$,即$\mathcal{F}_x\otimes\mathcal{G}_x$满足余积的泛性质。记$\rho^U_V$为预层$U\mapsto \mathcal{F}(U)\otimes \mathcal{G}(U)$的限制映射,它将$\sum_{i=1}^n s_i\otimes t_i$变成$\sum_{i=1}^n s_i|_V\otimes t_i|_V$. 记态射$\rho^U_x:\mathcal{F}(U)\otimes \mathcal{G}(U)\to \mathcal{F}_x\otimes\mathcal{G}_x$,它将$\sum_{i=1}^n s_i\otimes t_i$变成$\sum_{i=1}^n (s_i)_x\otimes (t_i)_x$.

如果存在对象$A$以及态射族$\varphi(U):\mathcal{F}(U)\otimes \mathcal{G}(U)\to A$使得,如果$V$是$U$的开子集,则$\varphi(U)=\varphi(V)\rho^U_V$. 我们需要定义一个$\varphi_x:\mathcal{F}_x\otimes\mathcal{G}_x\to A$使得$\varphi(U)=\varphi_x \rho^U_x$对每一个开集$U$都成立,并验证这是唯一满足要求的态射。此即泛性质。

任取$\sum_{i=1}^n (s_i)_x\otimes (t_i)_x$,其中$(s_i)_x=\langle s_i,U_i\rangle$, $(t_i)_x=\langle t_i,V_i\rangle$,任取开集$U\subset \bigcap_{i=1}^n U_i\cap V_i$,定义
\[
	\varphi_x\left(\sum_{i=1}^n (s_i)_x\otimes (t_i)_x\right)=\varphi(U)\left(\sum_{i=1}^n s_i|_U\otimes t_i|_U\right),
\]
由于$\varphi(U)=\varphi(V)\rho^U_V$,所以这个定义是良定的,不依赖于选取的$U$.

现在来检验分解,给定开集$U$,任取$\sum_{i=1}^n s_i\otimes t_i\in \mathcal{F}(U)\otimes \mathcal{G}(U)$,从构造有
\[
	\varphi_x\circ \rho^U_x\left(\sum_{i=1}^n s_i\otimes t_i\right)=\varphi_x\left(\sum_{i=1}^n (s_i)_x\otimes (t_i)_x\right)=\varphi(U)\left(\sum_{i=1}^n s_i\otimes t_i\right),
\]
所以$\varphi(U)=\varphi_x \rho^U_x$成立。

最后,$\varphi_x$的唯一性来自于所有$\mathcal{F}_x\otimes\mathcal{G}_x$的元素都可以写成某个$\rho_x^U$的像($U$并不固定),故分解$\varphi(U)=\varphi_x \rho^U_x$告诉我们$\varphi_x$由所有的$\varphi(U)$确定,此即唯一性。再具体一些,给定$a\in \mathcal{F}_x\otimes\mathcal{G}_x$,一定存在一个$b$和$V$使得$a=\rho^V_x(b)$,此时$\varphi_x(a)=\varphi(V)(b)$. 遍历$a$,我们就得到了$\varphi_x$的唯一性。
\end{proof}

\begin{para}[顺像]
设$(X,\oo_X)$和$(Y,\oo_Y)$是两个赋环空间,而$\Psi=(\psi,\theta):(X,\oo_X)\to (Y,\oo_Y)$是赋环空间之间的态射。现在给一个$X$上的$\oo_X$-模层$\calf$,将$\calf$只看成一个交换群层,我们有顺像$\psi_*\calf$. 由于$\psi_*\calf(U)=\calf(\psi^{-1}(U))$与$\psi_*\oo_X(U)=\oo_X(\psi^{-1}(U))$相容,所以$\psi_*\calf$具有$\psi_*\oo_X$-模层结构。此外,由于存在态射$\theta:\oo_Y\to \psi_*\oo_X$,所以$\psi_*\calf$具有$\oo_Y$-模层结构,我们将这个$\oo_Y$-模层记作$\Psi_*\calf$,称为$\calf$在$\Psi$下的顺像。

设$u:\calf\to\calg$是一个$\oo_X$-模层态射,于是$\psi_*u:\psi_*\calf\to\psi_*\calf$是一个$\psi_*\oo_X$-模层态射,同时也就是一个$\oo_Y$-模层态射,我们将这个态射记作$\Psi_*u$,称为态射$u$的顺像。结合上述两点,我们可以知道,顺像$\Psi_*$构成一个函子。
\end{para}

\begin{para}[张量积的顺像]
设$(X,\oo_X)$和$(Y,\oo_Y)$是两个赋环空间,而$\Psi=(\psi,\theta):(X,\oo_X)\to (Y,\oo_Y)$是赋环空间之间的态射。再设$\calf$, $\calg$都是$\oo_X$-模层。任取$Y$的开集$U$,我们有预层到其伴随层的典范态射
\[
	\calf(\psi^{-1}(U))\otimes_{\oo_X(\psi^{-1}(U))} \calg(\psi^{-1}(U))\to \calf\otimes \calg(\psi^{-1}(U)),
\]
他对$\oo_X(\psi^{-1}(U))=\psi_*\oo_X(U)$双线性的,所以也对$\oo_Y$是双线性的,张量积的泛性质将给出态射
\[
	\Psi_*\calf (U)\otimes_{\oo_Y(U)}\Psi_*\calg (U)\to \Psi_*(\calf\otimes \calg) (U).
\]
他与限制映射相容,由伴随层的泛性质,所以存在典范态射
\[
	\Psi_*(\calf)\otimes_{\oo_Y}\Psi_*(\calg) \to \Psi_*(\calf\otimes_{\oo_X} \calg).
\]
\end{para}

\begin{para}[逆像]
设$(X,\oo_X)$和$(Y,\oo_Y)$是两个赋环空间,而$\Psi=(\psi,\theta):(X,\oo_X)\to (Y,\oo_Y)$是����环空间之间的态射。逆像我们这里依然定义为顺像的左伴随函子,即
\[
	\Hom_{\oo_X}(\Psi^*\calf,\calg)\cong \Hom_{\oo_Y}(\calf,\Psi_*\calg)
\]
对$\calf$, $\calg$都是函子性同构。

逆像的存在可以如下构造:首先对$\theta:\oo_Y\to \psi_*\oo_X$,由交换环层的只是,我们有对应的$\theta^\#:\psi^*\oo_Y\to \oo_X$,因此$\oo_X$具有一个$\psi^*\oo_Y$-模层结构,我们将其记作$(\oo_X)_{[\theta]}$.

此外,对$\oo_Y$-模层$\calf$,作为交换群层,存在交换群层$\psi^*\calf$. 它实际上具有一个$\psi^*\oo_Y$-模层结构,任取$s\in \psi^*\calf(U)$以及$r\in \psi^*\oo_Y(U)$,从构造(见Theorem \ref{proinverse}),我们可以如下定义$rs\in \psi^*\calf(U)$
\[
	rs(x)=r(x)s(x)\in \calf_{\psi(x)}.
\]
它确实处于$\psi^*\calf(U)$中:对$s$和$r$,存在一个$\psi(x)$的邻域$V$,$x$的一个邻域$W\subset \psi^{-1}(V)\cap U$以及$a\in \oo_Y(V)$, $b\in \calf(V)$使得任取$x\in W$都有$r(x)=a_{\psi(x)}$, $s(x)=b_{\psi(x)}$. 此时,$rs(x)=r(x)s(x)=a_{\psi(x)}b_{\psi(x)}=(ab)_{\psi(x)}$.

综上,$\psi^*\calf$和$(\oo_X)_{[\theta]}$都是$\psi^*\oo_Y$-模层。最后,我们定义
\[
	\Psi^*\calf=\psi^*(\calf)\otimes_{\psi^*\oo_Y} (\oo_X)_{[\theta]},
\]
它当然是$(\oo_X)_{[\theta]}$-模层,也自然是一个$\oo_X$-模层。

为定义一个函子,我们还需要构造其在态射上的表现。设$u:\calf_1\to\calf_2$是一个$\oo_Y$-模层态射,很容易看到,$\psi^*u :\psi^*\calf_1\to\psi^*\calf_2$是一个$\psi^*\oo_Y$-模层态射。最后,定义$\Psi^*u=\psi^*u\otimes 1$,他就给出了态射$\Psi^*\calf_1\to \Psi^*\calf_2$.
\end{para}

\begin{pro}
在茎上,$(\Psi^*\calf)_x=\calf_{\psi(x)}\otimes_{(\oo_Y)_{\psi(x)}} (\oo_X)_{x}$.
\end{pro}

\begin{proof}
它直接来自于
\[
	(\Psi^*\calf)_x=(\psi^*(\calf)\otimes_{\psi^*\oo_Y} (\oo_X)_{[\theta]})_x=(\psi^*\calf)_x\otimes_{(\oo_Y)_{\psi(x)}}((\oo_X)_{[\theta]})_x=\calf_{\psi(x)}\otimes_{(\oo_Y)_{\psi(x)}}(\oo_X)_x,
\]
注意,此时$(\oo_X)_x$具有来自于$(\theta^\#)_x:(\oo_Y)_{\psi(x)}\to (\oo_X)_x$的$(\oo_Y)_{\psi(x)}$-模结构。
\end{proof}

\begin{para}[张量积的逆像]
考虑两个$\oo_Y$-模层$\calf$和$\calg$,以及态射$\Psi=(\psi,\theta):(X,\oo_X)\to (Y,\oo_Y)$. 由于$\psi^*\calf$和$\psi^*\calg$都具有$\psi^*\oo_Y$-模层结构,所以存在$\psi^*\calf\otimes_{\psi^*\oo_Y}\psi^*\calg$.

固定$X$的子集$U$,我们可以给出一个映射
\[
	\varphi(U):\psi^*\calf(U)\times \psi^*\calg(U) \to \psi^*(\calf\otimes \calg)(U)
\]
通过
\[
	\varphi(U)(s,t)(p)=s(p)\otimes t(p).
\]
这当然是一个双线性映射,且与限制映射相容。于是,张量积泛性质给出了一个典范态射
\[
	\psi^*\calf\otimes_{\psi^*\oo_Y} \psi^*\calg \to \psi^*(\calf\otimes_{\oo_Y} \calg).
\]
这是一个同构,因为在每条茎上它都是同构。

与$(\oo_X)_{[\theta]}$取张量积,我们就得到了典范同构
\[
	\Psi^*\calf\otimes_{\oo_X} \Psi^*\calg \to \Psi^*(\calf\otimes_{\oo_Y} \calg)
\]
\end{para}

\section{site与层}

\begin{para}[site]
	site某种程度上是拓扑空间的推广,首先来自于对开覆盖的推广。设$C$是一个范畴,$U$是一个对象,称一族指向$U$的态射$\mathscr U = \{U_i\to U\}_{i\in I}$为$U$的一个覆盖,其中$I$是一个集合。如果一个$C$中覆盖的集合$\mathscr T$满足
	\begin{compactenum}
		\item 若$V\to U$是一个同构,则$\{V\to U\}\in \mathscr T$.
		\item 如果$\{U_i\to U\}_{i\in I}$, $\{U_{ij}\to U_i\}_{j\in J_i} \in \mathscr T$,则$\{U_{ij}\to U\}_{j\in J_i,i\in I}\in \mathscr T$.
		\item 如果$\{U_i\to U\}_{i\in I}\in \mathscr T$且$V\to U$是一个态射,则纤维积$U_i\times_U V$存在且$\{U_i\times_U V\to U\}_{i\in I}\in \mathscr T$.
	\end{compactenum}
	则称$\mathscr T$是一个Grothendieck拓扑,$(C,\mathscr T)$为一个site. 同拓扑空间,我们一般略去$\mathscr T$,只称$C$为一个site. 而给定一个site $C$后,我们用$\operatorname{Cov}(C)$来记其Grothendieck拓扑。
\end{para}

特别地,若$C$就是拓扑空间定义的范畴,则上述site的定义中的纤维积即开集之交,显然,此时site就回到拓扑。

\begin{para}[预层]
	设$C$和$K$是范畴,则反变函子$C\to K$被称为$C$上的一个$K$-预层。这里,我们将预层推广到了任意的范畴上,同时,现在开始我们一般将$K$取成集合范畴。
\end{para}

\begin{para}[层]
	设$\calf$是site $C$上的集合预层,若任取覆盖$\{U_i\to U\}_{i\in I}\in \operatorname{Cov}(C)$,考虑所有纤维积$U_i\times_U U_j$的典范态射(以及相应的对象,再加上恒等态射等)构成的小范畴$\mathscr I$,都有
	\[
	{\varprojlim}_{U_i \in \mathscr{I}}\calf(U_i)=\calf(U),
	\]
	则称$\mathcal F$是一个层。
	
	若用交换图来表示,设$A$是一个集合,而$\calf$是$C$上的一个预层,任取覆盖$\{U_i\to U\}_{i\in I}\in \operatorname{Cov}(C)$,如果存在态射族$\{\varphi_i:A\to \calf(U_i)\}_{i\in I}$使得下图对任意的$i$, $j\in I$都交换(画图时省去了限制映射)
	\[
		\xymatrix{
			A\ar[r]^{\varphi_{j}}\ar[d]_{\varphi_{i}}&\calf(U_j)\ar[d]\\
			\calf(U_i)\ar[r]&\calf(U_{i}\times_U U_{j})
		}
	\]
	则存在唯一的态射$\varphi:A\to \calf(U)$使得下图对任意的$i$, $j\in I$都交换
	\[
		\xymatrix{
			A\ar@/_/[ddr]_{\varphi_{i}}\ar@/^/[drr]^{\varphi_{j}}\ar@{-->}[dr]|{\varphi}&& \\
			&\calf(U)\ar[r]\ar[d]&\calf(U_j)\ar[d]\\
			&\calf(U_i)\ar[r]&\calf(U_{i}\times_U U_{j})
		}
	\]
	这和拓扑空间上的层的定义几乎一模一样,我们只是将“交”换成了纤维积。
\end{para}

\begin{para}[连续函子]
	设$C$和$D$都是site,而$f:C\to D$是一个函子,若对$\operatorname{Cov}(C)$中的任意覆盖$\{U_i\to U\}_{i\in I}$,$\{f(U_i)\to f(U)\}_{i\in I}\in \operatorname{Cov}(D)$,且对$C$中任意态射$V\to U$,态射$f(V\times_U U_i)\to f(V)\times_{f(U)}f(U_i)$都是同构,则称$f$是一个连续函子。
\end{para}

连续函子是拓扑空间间连续映射的原像的类似物。

\begin{para}[推前与拉回]
	设$C$和$D$都是site,而$f:C\to D$是一个连续函子,而$\mathcal F$是$C$上的(预)层,我们定义推前$f_*\calf(U)=\calf(f(U))$,则$f_*\calf$是一个$D$上的(预)层。而拉回$f^*$是$f_*$的左伴随函子。
\end{para}

\begin{para}[拓扑斯]
	设$C$是一个site,拓扑斯是$C$上的(集合)层范畴$\operatorname{Sh}(C)$. 设$C$和$D$都是site,则对应有两个拓扑斯$\operatorname{Sh}(C)$和$\operatorname{Sh}(D)$,则拓扑斯的态射$f:\operatorname{Sh}(D)\to \operatorname{Sh}(C)$由一对伴随函子$f_*:\operatorname{Sh}(D)\to \operatorname{Sh}(C)$和$f^*:\operatorname{Sh}(C)\to \operatorname{Sh}(D)$给出,存在双函子同构
	\[
		\operatorname{Sh}(D)(f^*\mathcal G,\mathcal F)\cong \operatorname{Sh}(C)(\mathcal G,f_*\mathcal F),
	\]
	此外$f^*$是左正和的(与有限极限可交换)。拓扑斯同态$f$和$g$的复合$fg$具体写作$(fg)_*=f_* g_*$与$(fg)^*=g^*f^*$.
\end{para}

site之间的连续函子一定诱导一个拓扑斯的态射,但拓扑斯的态射不一定是由site之间的连续函子诱导的。

\vspace{2cm}

以后再写,这方面最好的参考是SGA 4与The Stacks project的第七章,许多我们在拓扑空间上的层的构造都被推广到site上的层了。

The étale topology is a Grothendieck topology on the category of schemes which has properties similar to the Euclidean topology. 它远比Zariski topology更好。

\section{仿射概形与伴随层}

\para 设$R$是一个交换环,所有$R$的素理想构成一个集合,记作$\spec(R)$,称为$R$的素谱。如果$R$是零环,则$R$没有素理想,所以$\spec(R)=\varnothing$,我们一般对此没什么兴趣。所以,我们默认$R$不是零环,但零环确实是在交换环范畴中的,就像集合范畴中有空集一样。

我们一般以$\pp$来表示素理想,但后面会看到$\spec(R)$有一个自然的拓扑结构,所以里面的点按照拓扑空间的习惯一般是记作拉丁字母$x$或者$p$等来表示这是一个点。所以下面我们以$\pp_x$来表示$x\in \spec(R)$对应的素理想。类似的,以$R_x$来表示局部化$R_{\pp_x}$,一个环局部化后是一个局部环,只有一个极大理想,以$\mm_x$来标记$R_x$的极大理想,同时记域$k(x):=R_x/\mm_x$.

\para 现在设$f\in R$是环中的任意元素,则经过如下自然的映射
\[
	f\mapsto f(x):R\to R_x\to k(x)
\]
得到的$f(x)$成为环元$f$在$x$处的值,其中第一个映射将$f$映作$f/1$,而第二个映射为商映射。显然$f(x)=0$等价于$f\in \pp_x$,或者$(f)\subset \pp_x$.

可以看到,$f(x)$并不是真正意义上的一个函数,因为在每一点处的$k(x)$都是不同的。但是,每一个域都有一个加法零元,所以谈论$f(x)$的零点还是有意义的。

我们可以在层上进行类似的操作。比方说$M$是一个光滑流形,他上面有一个光滑函数层$\mathcal{O}$,在一点$x$处,有局部环$\mathcal{O}_x$,他的极大理想$\mm_x$就是那些在$x$附近为零的光滑函数构成的等价类,此时任意$\mathcal{O}$的截面就是一个光滑函数$f$,而作为函数值的$f(x)$和通过$\mathcal{O}(M)\to \mathcal{O}_x\to \mathcal{O}_x/\mm_x\cong \rr$得到的$f(x)$是相同的。

所以,想要$\spec(R)$获得一个层结构,首先$R$中的元素作为截面要表现得像一个连续函数,这将给出$\spec(R)$的拓扑结构,然后局部环$R_x$应该表现得和$\mathcal{O}_x$类似,这将给出$\spec(R)$的一个层结构。

\begin{para}[素谱的拓扑结构]
定义$D(f):=\{x\in \spec(R)\,:\, f(x)\neq 0\}$,我们下面的目标是让$f(x)$尽可能看上去像一个连续函数,我们要证明$D(f)$满足开集公理。等价地,我们要验证$V(f):=\{x\in \spec(R)\,:\, f(x)=0\}=\{\pp\in \spec(R)\,:\, (f)\subset \pp\}$满足闭集公理,或者更一般地,我们可以验证$V(E):=\{\pp\in \spec(R)\,:\, E\subset \pp\}$满足闭集公理,其中$E$是$R$的任意子集。
\end{para}

\begin{pro}\label{pro:3.4}
$V(E)$满足闭集公理,所以$\spec(R)$以$V(E)$为闭集是一个拓扑空间,且如下命题成立:
\begin{compactenum}
\item $D(f)$是开集,且它们构成了$\spec(R)$的一个拓扑基。
\item 环的素谱满足$T_0$分离公理。
\item 环的素谱是紧的。
\end{compactenum}
\end{pro}

\begin{proof}
显然的是$V(1)=\varnothing$以及$V(0)=\spec(R)$,剩下的我们需要检验$\bigcap_{i\in I}V(E_i)=V(\bigcup_{i\in I}E_i)$. 设$\pp \in \bigcap_{i\in I}V(E_i)$,则$\pp\in V(E_i)$对$i\in I$都成立,这等价于$E_i\subset \pp$对$i\in I$都成立,而这又等价于$\bigcup_{i\in I}E_i \subset \pp$,因此得证。

关于这个拓扑空间的命题,我们逐个证明。首先,设$U$是任意的开集,则其写作$U=\spec(R)-V(E)$的形式,而
\[
\spec(R)-V(E)=\spec(R)-V\bigl(\bigcup_{f\in E} f\bigr)=\spec(R)-\bigcap_{f\in E}V(f)=\bigcup_{f\in E}D(f),
\]
所以$U=\bigcup_{f\in E}D(f)$.

接着,实际上,设$\pp_x$和$\pp_y$分别是两个素理想,则或者有$\pp_x\not\subset \pp_y$或者有$\pp_y\not\subset \pp_x$,因此有$y\not\in \overline{\{x\}}$或者$x\not\in \overline{\{y\}}$. 用开集描述就是,$y\in \spec(A)-\overline{\{x\}}$或者$x\in \spec(A)-\overline{\{y\}}$,这就使得$T_0$分离公理成立。

结合第一点,设$\{D(f_\alpha)\,:\, \alpha\in A\}$是$\spec R$的一个开覆盖,这里$A$是一个指标集。我们需要证明
存在一个有限子集$B$,使得$\{D(f_\alpha)\,:\, \alpha\in B\}$也是一个开覆盖。考虑$\bigcap_{\alpha\in A}V(f_\alpha)=V(\{f_\alpha\}_{\alpha\in A})=\varnothing$,于是$\{f_\alpha\}_{\alpha\in A}$生成了整个环$R$,特别地,生成了$1$,于是存在有限个$f$可以线性组合出$1$,即$\sum_{\alpha\in B} r_\alpha f_\alpha=1$,这样我们也就找到了这个有限子集。
\end{proof}

作为推论,如果$\overline{\{x\}}=\overline{\{y\}}$,则$x=y$.

\begin{pro}
关于$D(f)$,我们也有如下两个简单的性质:
\begin{compactenum}[~~~1.]
\item $D(fg)=D(f)\cap D(g)$.
\item 如果$a\in R$是一个可逆元,则$D(af)=D(f)$.
\end{compactenum}
\end{pro}

第二点在分式环的时候很常用,比如在$\spec R_f$中,$f/1$和$1/f$都是可逆元,故$D(g/f^n)=D(g/1)$.

\begin{proof}
第一点,任取$x\in D(fg)$,则$(fg)(x)=f(x)g(x)\neq 0\in k(x)$. 由于$k(x)$是一个域,所以$x\in D(fg)$当且仅当$f(x)\neq 0$且$g(x)\neq 0$,即当且仅当$x\in D(f)\cap D(g)$.

第二点,任取$x\in D(af)$即$(af)(x)=a(x)f(x)\neq 0$,由于$a\in R$可逆,所以$a(x)\neq 0$对任意的$x$都成立,故$(af)(x)\neq 0$等价于$f(x)\neq 0$等价于$x\in D(f)$.
\end{proof}

记$\pp(Y)=\{f\in A\,:\, f(Y)=0\}$,即在$Y$上消失的所有环元构成的理想,利用这个记号,显然的是$\pp(\{x\})=\pp_x$. 把$f(Y)=0$的意义详细写出来,则$\pp(Y)=\bigcap_{y\in Y}\pp_y$. 于是,$\pp(V(E))=\sqrt{(E)}$,这来自于$\sqrt{(E)}$是所有包含$(E)$的素理想的交。这是传统的Hilbert零点定理的对应。

\begin{pro}如下命题成立:
\begin{compactenum}[~~~1.]
\item $V(E)=V((E))=V(\sqrt{(E)})$;
\item 若$E\subset F$,则$V(F)\subset V(E)$;
\item $\bigcap_{i\in I}V(E_i)=V(\bigcup_{i\in I}E_i)$;
\item 若$X\subset Y$,则$\pp(Y) \subset \pp(X)$;
\item $\pp(V(E))=\sqrt{(E)}$;
\item $V(\pp(Y))=\overline{Y}$.
\end{compactenum}
\end{pro}

于是,由1, 2, 4, 5和6,可知$\pp$与$V$构成了一对Galois联络,于是它们的像构成一一对应。即$\spec R$中的闭集一一对应于$R$中的根式理想。

\begin{proof}
% 对$R$的任意子集$E$,显然有$V(E)=V((E))$,其中$(E)=ER$是$E$生成的理想,由于$\sqrt{(E)}$是所有包含$(E)$的素理想的交,所以有$V(E)=V((E))=V(\sqrt{(E)})$. 再比如,如果$E\subset F$,则$V(F)\subset V(E)$.

除了最后一点其他都是容易的。由于$\pp(Y)=\bigcap_{y\in Y}\pp_y$,所以$\pp(Y) \subset \pp_y$,其中$y\in Y$,所以$Y\subset V(\pp(Y))$. 剩下我们要证明,$V(\pp(Y))$就是包含$Y$的最小闭集。考虑$Y\subset V(E)$,对于任意的$f\in E$以及$y\in Y$,都有$f(y)=0$,从而$E\subset \pp(Y)$,进而$V(\pp(Y))\subset V(E)$. 
\end{proof}

设$x\in \spec(A)$,$\overline{\{x\}}=V(\pp(\{x\}))=V(\pp_x)$,则$\{x\}$是闭集的充分必要条件就是$V(\pp_x)=\{x\}=\{\pp_x\}$,或者说$\pp_x$是一个极大理想。所以极大理想对应着素谱中的闭点,类似于坐标环的极大理想对应着仿射簇中的点一样。

\begin{para}[素谱之间的映射]
设$\varphi:R_1\to R_2$是一个环同态,由于素理想在$\varphi$的原像也是素理想,所以$\varphi$通过$\spec(R_2)\ni x\mapsto \varphi^{-1}(\pp_x)\in \spec(R_1)$诱导了素谱之间的映射$^a\varphi:\spec(R_2)\to \spec(R_1)$. 这是一个连续映射,这个我们只要检验$^a\varphi^{-1}(V(f))$是一个闭集就可以了,事实上,更一般的命题是:$^a\varphi^{-1}(V(E))=V(\varphi(E))$,老实按照定义检验即可。有时也把$^a\varphi$记作$\spec \varphi$. 由于$^a(\varphi\circ \psi)={^a\psi}\circ{^a\varphi}$,所以$\spec$是环范畴到拓扑空间范畴的反变函子。

局部来看,我们用$\varphi^x$来表示$\varphi$在商环$R_1/\varphi^{-1}(\pp_x)\to R_2/\pp_x$诱导出的同态,以及用同样的符号来表示诱导的域同态$\varphi^x:k({^a}\varphi(x))\to k(x)$. 根据定义,对任意的$f\in R_1$,以及$x\in \spec(R_2)$,有等式
\[
	\varphi^x(f({^a}\varphi(x)))=\varphi^x\left(f \text{ mod } \varphi^{-1}(\pp_x)\right)=\varphi(f)\text{ mod } \pp_x=\varphi(f)(x).
\]
\end{para}

\begin{pro}\label{pro:3.8}
如下命题成立:
\begin{compactenum}[~~~1.]
\item $^a\varphi^{-1}(V(E))=V(\varphi(E))$;
\item $^a\varphi^{-1}(D(f))=D(\varphi(f))$;
\item $\varphi^x(f({^a}\varphi(x)))=\varphi(f)(x)$;
\item $\overline{{^a}\varphi(V(E))}=V(\varphi^{-1}(E))$.
\end{compactenum}
\end{pro}

\begin{proof}
除了最后一点,我们都是清楚的。由于根式理想的原象还是根式理想,所以可以干脆假设$E=\mathfrak{a}$是一个根式理想。设$Y=V(\mathfrak{a})$以及$\mathfrak{a}'=\pp({^a}\varphi(Y))$,则$\overline{{^a}\varphi(Y)}=V(\mathfrak{a}')$. 我们先证明$V(\varphi^{-1}(\mathfrak{a}))\subset V(\mathfrak{a}')=\overline{{^a}\varphi(Y)}$. 而这又等价于去证明$\mathfrak{a}'\subset \varphi^{-1}(\mathfrak{a})$. 由定义,$f'\in \mathfrak{a}'$等价于$f'({^a\varphi}(Y))=0$,因此对任意的$y\in Y$成立$\varphi(f')(y)=\varphi^y(f'(y))=0$,所以$\varphi(f')\in \pp(Y)=\pp(V(\mathfrak{a}))=\sqrt{\mathfrak{a}}=\mathfrak{a}$. 于是$f'\in \varphi^{-1}(\mathfrak{a})$.

再证明反过来的包含$\overline{{^a}\varphi(V(E))}\subset V(\varphi^{-1}(E))$. 设$\varphi^{-1}(\pp)\in {^a}\varphi(V(E))$,其中$\pp\in V(E)$或者$E\subset \pp$,则$\varphi^{-1}(E)\subset \varphi^{-1}(\pp)$,于是$\varphi^{-1}(\pp)\in V(\varphi^{-1}(E))$或者说${^a}\varphi(V(E))\subset V(\varphi^{-1}(E))$. 由于$V(\varphi^{-1}(E))$是一个闭集,所以$\overline{{^a}\varphi(V(E))}\subset V(\varphi^{-1}(E))$.
\end{proof}

\begin{lem}\label{lem:3.9}
设$\varphi:A'\to A$是一个环同态,对任意的$a\in A$,总存在$a'\in A'$和$A$中的可逆元$h$使得$\varphi(a')=ha$,那么$^a\varphi$是$\spec(A)$到$\im (^a\varphi)$的同胚。
\end{lem}

\begin{proof} 先对每一个$A$的子集$E$,我们证明总存在$A'$中的一个子集$E'$满足$V(E)=V(\varphi(E'))$即可。固定$E$,对每一个$a\in E$,取一个$a'\in A$和一个可逆元$h$使得$\varphi(a')=ha$,注意到此时$D(\varphi(a'))=D(a)$或者$V(\varphi(a'))=V(a)$,所以这些$a'$组成的集合$E'$就满足我们的需求。

首先证明$^a\varphi$是一个单射。假设$^a\varphi(x)={}^a\varphi(y)$但$x\neq y$. 由$T_0$分离性,可以假设$y\not\in \overline{\{x\}}$,选择$V(E)=\overline{\{x\}}$,于是存在$E'$使得$x\in V(E)=V(\varphi(E'))$,所以${}^a\varphi(y)={}^a\varphi(x)\in V(E')$. 因为$y\not\in \overline{\{x\}}=V(E)=V(\varphi(E'))={}^a\varphi^{-1}(V(E'))$,所以${}^a\varphi(y)\not\in V(E')$,矛盾。因此$^a\varphi$是一个单射。设$\psi$是它在$\im (^a\varphi)$上的函数逆,下面我们证明这是一个连续映射。

任取$E$,存在$E'$使得$V(E)=V(\varphi(E'))={}^a\varphi^{-1}(V(E'))$,于是
\[
	\psi^{-1}(V(E))={}^a\varphi(V(E))={}^a\varphi({}^a\varphi^{-1}(V(E')))=V(E'),
\]
任意闭集关于$\psi$的原像是闭集,故$\psi$连续。进而$^a\varphi$是$\spec(A)$到$\im (^a\varphi)$的同胚。
\end{proof}

如果$\varphi$是满射,按照上一个命题,$^a\varphi$是$\spec(A)$到$\im (^a\varphi)$的同胚。设$\mathfrak{a}$是$R$的一个理想,则商映射$\varphi:R\to R/\mathfrak{a}$诱导的连续映射$^a\varphi:\spec(R/\mathfrak{a})\to \spec(R)$给出了$\spec(R/\mathfrak{a})$与$\spec(R)$的闭子集$V(\mathfrak{a})$之间的同胚。特别地,$\spec(R/\sqrt{0})$和$\spec(R)$同胚。

设$S$是$A$的一个乘性子集,我们由典范的同态$i:A\to S^{-1}A$,且,对任意的$a/s\in S^{-1}A$,我们都有$a'=a$使得$i(a')=a/1=s(a/s)$,所以引理条件满足。此时${}^ai:\spec(S^{-1}A)\to \im ({}^ai)$是一个同胚,由$S^{-1}A$的素理想的结构,它的素理想一一对应于$A$中与$S$不相交的素理想,所以$\im ({}^ai)$即$\spec A$中满足$\pp_x\cap S=\varnothing$的$x$构成的子空间。

特别地,任取$f\in R$,考虑$\spec(R_f)$. 它应与$\spec R$中所有不包含$f$的素理想构成的集合同胚,也即$D(f)$. 

\begin{pro}
$\spec R$不可约当且仅当$\sqrt{0}$是一个素理想。$\spec(R)$的不可约闭子集一一对应着$R$的素理想,特别地,$\spec(R)$的不可约分支对应着$R$的极小素理想。
\end{pro}

\begin{proof}
由于$\spec(R)\cong \spec(R/\sqrt{0})$,所以如果$\sqrt{0}$是素理想,可以假设$R$是一个整环。现在假设$\spec R$是可约的,则存在两个真闭子集$V_1$和$V_2$使得$\spec R=V_1\cup V_2$. 于是$\pp(V_1)\cap \pp(V_2)=\pp(V_1\cup V_2)=\pp (\spec R)=0$,但由于$V_1$, $V_2$是真闭子集,所以$\pp(V_1)$和$\pp(V_2)$都不为零,因此$R$不是整环。

反过来,由于$\spec(R/\sqrt{0})$和$\spec R$的同胚,可以假设$\sqrt{0}=0$. 如果$\spec R$不是整环,则存在非零的$f$, $g$使得$fg=0$,于是$\spec R=V(0)=V(fg)=V(f)\cup V(g)$. 但由于$f$, $g$非零,且$\sqrt{0}=0$,所以$V(f)$与$V(g)$是$\spec R$的真闭子集。进而$\spec R$可约。
\end{proof}

设$V(\mathfrak{a})$是$\spec R$的一个不可约闭子集,由于$V(\mathfrak{a})$与$\spec (R/\mathfrak{a})$同胚,所以$\spec (R/\mathfrak{a})$是不可约的。因此,$\sqrt{\mathfrak{a}}$是极小的素理想,故是$V(\mathfrak{a})$的一个一般点。由于$\spec R$是$T_0$的,所以这个一般点也是唯一的一般点。于是$\spec R$中的任意不可约闭子集都具有唯一的一般点。或者说,$\spec R$中所有的不可约闭子集都具有$\overline{\{x\}}$的形式,其一般点是$x$. 

反过来,给定$x\in\spec R$,$\overline{\{x\}}$是不可约闭子集,所以$x\mapsto \overline{\{x\}}$是$\spec R$上的点到$\spec R$中不可约闭子集之间的一一对应。

\begin{para}
由于在拓扑空间的一组基上给出层的公理,就可以得到一个层,所以我们通过$D(f)\mapsto R_f$来定义$\spec(R)$的一个预层结构$\mathcal{O}_R$,下面会证明这是一个环层,称为结构层或者说伴随层。类似地,如果有一个$R$-模$M$,我们可以通过$D(f)\mapsto M_f$来定义出一个$\mathcal{O}_R$-模层$\widetilde{M}$. 特别地,$\mathcal{O}_R(\spec(R))=R_1=R$. 素谱连同其上的结构层构成一个赋环空间,我们依旧记做$\spec(R)$,如有必要,他的结构层写作$\mathcal{O}_R$.

由于$x\in \spec(R)$附近的邻域$D(f)$关于包含于是共尾(cofinal)的,所以茎$(\widetilde{M})_x=M_x$,故而这正符合我们上面对于层结构的需求。所以这是一个局部赋环空间。
\end{para}

为了描述层结构,我们需要限制映射,这直接来自于下面的引理。

\begin{lem}
设$R$是一个环,而$M$是一个$R$-模。以下命题成立:
\begin{compactenum}[~~~1.]
\item 设$S$是$R$的一个乘性子集,而$S'$是$S$中元素的所有因子构成的新的乘性子集,则$(S')^{-1}M$和$S^{-1}M$典范同构。所以一般我们将其看成等同的,即$(S')^{-1}M=S^{-1}M$. 
\item 存在典范同构$(M_f)_g\cong M_{fg}$,我们也将其看成是等同的。
\item 如果$D(g)\subset D(f)$,则$M_{fg}=M_{g}$.
\item 设$n$是一个正整数,则$D(f^n)=D(f)$且$M_{f}=M_{f^n}$,其中$n$是一个正整数。
%\item 对一个有限环元族$\{f_\alpha\}$和一个给定的正整数$n$,$\{f_\alpha^n\}$生成单位理想当且仅当$\{f_\alpha\}$生成单位理想。
\end{compactenum}
\end{lem}

\begin{proof}
一条一条来证明:
\begin{compactenum}[~~~1.]
\item 注意到$S\subset S'$,所以存在一个典范同态$i:S^{-1}M\to (S')^{-1}M$,他将$m/s$变成$m/s$. 现在设$i(m/s)=0$,因此存在一个$s'\in S'$使得$s'm=0$. 由于$s'$是$S$中某个元$t$的因子,而因此存在$t\in S$使得$tm=0$,或者说$m/s=0$. 所以$i$是一个单同态。反之,任取$m/s'\in S^{-1}M$,因为$s'$是某个$t\in S$的因子,写作$t=s'r$. 所以$m/s'=rm/t=i(rm/t)$. 因此$i$是一个满射。

\item 由于$(M_f)_g=S^{-1}M_f$,其中$S=\{g/1$, $g^2/1$, $\dots\}$. 我们通过$(m/f^i)/(g^j/1)\mapsto f^j g^i m/(fg)^{i+j}$定义$i:(M_f)_g\to M_{fg}$. 它显然是单的,因为$(m/f^i)/(g^j/1)=0\in (M_f)_g$等价于存在$k$和$l$使得$g^kf^l m=0$,而$f^j g^i m/(fg)^{i+j}=0\in M_{fg}$告诉我们存在$n$使得$(fg)^nf^j g^i m=0$. 它也是一个满同态,因为任取$m/(fg)^i\in M_{fg}$,我们可以写成$(m/f^i)/(g^i/1)$的像。所以这是一个同构。

\item 记$S_f$为$\{1$, $f$, $f^{2}$, $\dots\}$全部因子构成的乘性子集。首先,由于$D(g)\subset D(f)$,等价地,$V(g)\supset V(f)$,因之,$\sqrt{(g)}=\pp(V(g))\subset \pp(V(f))=\sqrt{(f)}$. 所以,存在
一个正整数$n$和一个$r\in R$使得$g^n=rf$. 现在,注意$\{1$, $g$, $g^{2}$, $\dots\}$包含$g^{n+1}=rfg$,所以$S'_g$中包含$fg$,故$S_{fg}\subset S_g$. 反之,由于$g$是$fg$的一个因子,所以$S_g\subset S_{fg}$. 结合此二者,$S_g=S_{fg}$,故$M_g=M_{fg}$.

\item 注意到$D(f^n)=D(f)\cap D(f^{n-1})=\cdots=D(f)$,然后反复应用上面这个命题即可。\qedhere
\end{compactenum}
\end{proof}

所以,如果$D(g)\subset D(f)$,我们定义限制映射$\rho^{D(f)}_{D(g)}$为典范同态$M_f\to (M_f)_g=M_{fg}=M_g$. 具体到元素上,即$m/f^n\mapsto mg^n/(fg)^n$. 特别地,当$M=R$的时候,我们有
\[
	\rho^{D(f)}_{D(g)}(r_1/f^{n_1} r_2/f^{n_2})=\rho^{D(f)}_{D(g)}(r_1r_2/f^{n_1+n_2})=r_1r_2 g^{n_1+n_2}/(fg)^{n_1+n_2},
\]
另一方面
\[
	\rho^{D(f)}_{D(g)}(r_1/f^{n_1})\rho^{D(f)}_{D(g)}(r_2/f^{n_2})=(r_1g^{n_1}/(fg)^{n_1})(r_2g^{n_2}/(fg)^{n_2})=r_1r_2 g^{n_1+n_2}/(fg)^{n_1+n_2},
\]
所以$\rho^{D(f)}_{D(g)}$是一个环同态。

\begin{lem}\label{lem:3.12}
给定环$R$以及一个环元$f\in R$. 设$\{f_i\}_{i\in I}$是$R$的一族元素,则$\{D(f_i)\}_{i\in I}$是$D(f)$的一个开覆盖当且仅当$\{f_i/1\}$生成了$R_f$. 此外,$D(f)$是预紧拓扑空间。
\end{lem}

特别地,如果$f=1$,则$\{D(f_i)\}_{i\in I}$是$D(1)=\spec R$的一个开覆盖当且仅当$\{f_i\}_{i\in I}$生成了单位理想。此外,我们还得到$D(1)=\spec R$是预紧的。

\begin{proof}
任取$x\in D(f)$即$f(x)\neq 0$,由于$\{D(f_i)\}_{i\in I}$是$D(f)$的一个开覆盖,存在一个$i$使得$x\in D(f_i)$即$f_i(x)\neq 0$. 所以,如果存在一个素理想$\pp_x$使得$f_i(x)=0$对$i\in I$都成立,则$x\not\in D(f)$或等价地$x\in V(f)$. 或者说,对素理想$\pp$,$f_i\in \pp$对所有$i\in I$都成立可以推出$f\in \pp$.

现在考虑$\{f_i/1\}_{i \in I}$在$R_f$中生成的理想$\langle f_i/1\,:\,i \in I \rangle$. 如果$\langle f_i/1\,:\,i \in I \rangle$不是单位理想,因此存在一个素理想$\mm$使得$\langle f_i/1\,:\,i \in I \rangle\subset \mm$. 设$i:R\to R_f$为典范同态,则我们有$R$中理想的包含关系
\[
	\langle f_i\,:\,i \in I \rangle \subset i^{-1}\left(\langle f_i/1\,:\,i \in I \rangle\right)\subset i^{-1}(\mm),
\]
其中$i^{-1}(\mm)$是一个与乘性子集$\{1$, $f$, $f^2$, $\dots\}$不相交的素理想。但由于素理想$i^{-1}(\mm)$包含所有$f_i$,所以也包含$f$,矛盾。

反过来,如果$\{f_i/1\}_{i \in I}$生成了$R_f$,但存在一个$x\in D(f)$使得对所有$i\in I$都有$x\not\in D(f_i)$. 或者说,存在一个素理想$\pp_x$使得$f\not\in \pp_x$但是$\langle f_i\,:\,i \in I \rangle\subset \pp_x$. 现在考虑$\langle f_i\,:\,i \in I \rangle\subset \pp_x$在$i$下的像,由于$f\not\in \pp_x$,所以$i(\pp_x)$必然是$R_f$中的一个素理想,它包含了所有的$f_i/1$,这与$\{f_i/1\}_{i \in I}$生成了$R_f$矛盾。

最后,任取$D(f)$的开覆盖,我们都可以加细成$D(f)=\bigcup_i D(f_i)$的形式,所以只需考虑这种覆盖即可。由于$\{f_i/1\}$生成$R_f$,所以$1\in R_f$可以由有限个$f_i/1$线性组合而成,故这些有限个$D(f_i)$构成了$D(f)$的一个有限子覆盖。
\end{proof}

现在可以来证明$\widetilde{M}$是一个层了。

\begin{thm}
设$R$是一个环,则预层$D(f)\mapsto M_f$是$\spec R$上的一个交换群层,记作$\widetilde{M}$. 特别地,当$M=R$的时候,$\widetilde{R}$是一个环层,记作$\mathcal{O}_R$. 对于任意的$R$-模$M$,$\widetilde{M}$是一个$\mathcal{O}_R$-模层。
\end{thm}

\begin{proof}
设开集$D(f)$被$\{D(f_i)\subset D(f)\}$所覆盖,给定$f_i$,设$m_1$, $m_2\in M_f=\widetilde{M}(D(f))$限制在$D(f_i)$上都相同,我们需要证明$m_1=m_2\in M_f$. 此即截面的唯一性。

由于$D(f)$是预紧的,所以可以假设覆盖是有限的。于是存在整数$N_i$使得$(f_i^{N_i}/1)(m_1-m_2)=0$. 由于覆盖是有限的,所以可以找一个足够大的$N$使得$(f_i^N/1)(m_1-m_2)=0$对任意的$i$都成立。由于$\{D(f_i)=D(f_i^N)\}$是$D(f)$的一个开覆盖,所以$\{f_i^N/1\}$生成了$R_f$,即有$1=\sum_i r_i(f_i^N/1)$,其中$r_i\in R_f$. 因此
\[
	m_1-m_2=\sum_i r_i(f_i^N/1)(m_1-m_2)=0.
\]

此外,给定$D(f)$的一个开覆盖$\{D(f_i)\}$,其中$D(f_i)\subset D(f)$. 如果存在一族截面$\{m_i\in M_{f_i}\}$使得$m_i$和$m_j$限制在$D(f_i)\cap D(f_j)=D(f_if_j)$上时是相等的,我们需要说明存在$m\in M_f$使得$m_i=\rho^{D(f)}_{D(f_i)}(m)$. 此即局部相容的截面可以拼起来。

我们先选一个有限子覆盖$\{f_i\,:\,i\in J\}$. 在这些$M_{f_i}$中,因为$m_i$和$m_j$在$M_{f_if_j}$中相同,所以有
\[
	(f_if_j/1)^{n_{ij}} m_j = (f_if_j/1)^{n_{ij}} m_i.
\]
由于是有限覆盖,所以可以将$n_{ij}$都选成一个数$N_1$. 此外,存在$(f_i^{n_i}/1)m_i=h_i\in M_f$,由于是有限子覆盖,所以可以将所有$n_i$都选成相同的$N>N_1$. 此时,
\[
	(f_i/1)^N h_j=(f_if_j/1)^Nm_j=(f_if_j/1)^Nm_i=(f_j/1)^N h_i.
\]

因为是有限子覆盖,这些$f_i^N/1$生成了$M_f$,所以存在$r_i\in R_f$使得$1=\sum_i r_i (f_i^N/1)$. 定义$m=\sum_i r_ih_i$,我们有
\[
	(f_i/1)^N m =\sum_j r_j (f_i/1)^N h_j =\sum_j r_j (f_j/1)^N h_i= h_i = (f_i/1)^N m_i.
\]
所以$m$限制在$D(f_i)$上就是$m_i$.

最后回到原覆盖,设$f_\alpha$不在我们选取的有限子覆盖中。由于$\{D(f_\alpha f_i)\,:\, i\in J\}$构成了$D(\alpha)$的一个有限开覆盖。从上面的推导,$m$和$m_\alpha$限制到每个$D(f_\alpha f_i)$上都是相同的,都等于$m_i$限制在$D(f_\alpha f_i)=D(f_\alpha)\cap D(f_i)$上得到的截面,所以由一开始证明的唯一性,$m$限制在$D(f_\alpha)$将得到$m_\alpha$.

由于$\widetilde{R}$的限制映射是环同态,所以我们实际上已经得到了一个环层$\mathcal{O}_R$. 在任意的$D(f)$上,交换群层$\widetilde{M}(D(f))=M_f$是一个自然的$\oo_R(D(f))=R_f$-模,且这个模结构与限制映射相容。最后,对于任意的开集$U$,我们需要证明,$\widetilde{M}(U)$是一个$\oo_R(U)$-模。

我们知道
\[
	\oo_R(U)={\varprojlim}_{D(f)\subset U} \oo_R(D(f)),\quad \widetilde{M}(U)={\varprojlim}_{D(f)\subset U} \widetilde{M}(D(f)).
\]
或者具体写出构造
\[
	{\varprojlim}_{D(f)\subset U} \oo_R(D(f))=\left\{
	(r_f)\in \prod_{D(f)\subset U}\oo_R(D(f))\,:\,\text{如果$D(g)\subset D(f)\subset U$,则成立$r_g=\rho^{D(f)}_{D(g)}r_f$}
	\right\},
\]
\[
	{\varprojlim}_{D(f)\subset U} \widetilde{M}(D(f))=\left\{
	(m_f)\in \prod_{D(f)\subset U}\widetilde{M}(D(f))\,:\,\text{如果$D(g)\subset D(f)\subset U$,则成立$m_g=\rho^{D(f)}_{D(g)}m_f$}
	\right\},
\]
于是,我们可以在$\widetilde{M}(U)$上如下定义一个$\oo_R(U)$-模结构
\[
	(r_f)(m_f)=(r_fm_f).
\]
他显然与限制映射相容。因此$\widetilde{M}$是一个$\mathcal{O}_R$-模层。
\end{proof}

\begin{pro}
成立典范的层同构$\widetilde{M_f}\cong \widetilde{M}|_{D(f)}$. 
\end{pro}

\begin{proof}
首先,$D(f)$作为拓扑空间同胚于$\spec(R_f)$,他将$D(g)\subset D(f)$映射为$\spec(R_f)$中的$D(g/1)$,反过来,任取$D(h/f^n)\subset \spec(R_f)$,它在$D(f)$中对应的开集为$D(fh)=D(f)\cap D(h)$. 它们显然互逆。

其次,任取$D(g/f^n)\subset \spec(R_f)$,它上面的模是$(M_f)_g=M_{fg}$. 同样,考虑$D(g/f^n)$在$D(f)$中对应的开集,即$D(fg)\subset D(f)$,它上面的模是$M_{fg}$. 于是,模$M_{fg}$的恒等映射就定义了一族同构$\widetilde{M_f}(D(g/f^n))\to \widetilde{M}|_{D(f)}(D(fg))$,这族同构显然与限制映射相容,所以是一个层的同构。
\end{proof}

\begin{pro}所有$R$-模的伴随层实际上构成一个范畴,这里我们给出这个范畴的一些结论:
\begin{compactenum}[~~~1.]
\item $\Hom_R(M,N)\cong \Hom_{\oo_R}(\widetilde{M},\widetilde{N})$. 将$\psi\in\Hom_R(M,N)$在$\Hom_{\oo_R}(\widetilde{M},\widetilde{N})$中的像记作$\widetilde{\psi}$. 特别地,Yoneda引理告诉我们,$M=0$实际与$\widetilde{M}=0$等价。
\item $M\mapsto \widetilde{M}$, $\psi\mapsto \widetilde{\psi}$构成了一个正合函子。
\item 设$\{M_i\}$是一族$R$-模,则$M=\varinjlim_{i}M_i$的伴随层为$\varinjlim_{i} \widetilde{M_i}$.
\item 设$M$, $N$是两个$R$-模,则$M\otimes N$伴随层为$\widetilde{M}\otimes_{\oo_R}\widetilde{N}$.
\item 设$\psi:M\to N$是一个$R$-模同态,则$\ker \psi$, $\im \psi$和$\coker \psi$的伴随层分别是$\ker \widetilde{\psi}$, $\im \widetilde{\psi}$和$\coker \widetilde{\psi}$. 因此,$\psi$是单同态、满同态、同构当且仅当$\widetilde{\psi}$是单态射、满态射、同构。
\end{compactenum}
\end{pro}

\begin{proof}
第一点,任取$\varphi:\Hom_{\oo_R}(\widetilde{M},\widetilde{N})$,则$\varphi(\spec R):\widetilde{M}(\spec R)\to \widetilde{N}(\spec R)$就是从$M$到$N$的$R$-模同态。反过来,任取$R$-模同态$\psi:M\to N$,我们需要对每一个$D(f)$定义
\[
	\widetilde{\psi}(D(f)):\widetilde{M}(D(f))\to \widetilde{N}(D(f)),
\]
注意到
\[
	\widetilde{M}(D(f))=M_f=R_f\otimes_R M,\quad \widetilde{N}(D(f))=N_f=R_f\otimes_R N,
\]
所以我们定义$\widetilde{\psi}(D(f))=\id_{R_f}\otimes_R \psi=\psi_f$. 不难检验,这是一个$\oo_R$-模层同态,并且$\psi\mapsto \widetilde{\psi}$和$\varphi\mapsto \varphi(\spec R)$互逆。

对于第二点,考虑正合列$M_1\to M_2\to M_3$,对应地有列$\widetilde{M_1}\to \widetilde{M_2}\to \widetilde{M_3}$. 我们知道,一个模层的列是正和的,当且仅当在每条茎上是正和的,所以这里只要考察
\[
	\left(\widetilde{M_1}\right)_x\to \left(\widetilde{M_2}\right)_x\to \left(\widetilde{M_3}\right)_x
\]
或者等价的
\[
	(M_1)_{\pp_x}\to (M_2)_{\pp_x}\to (M_3)_{\pp_x}
\]
即可。由于局部化函子总是正和的,所以上述列都是正合列。

第三点,对每一个$D(f)$,我们有
\[
	\widetilde{M}(D(f))=\left({\varinjlim}_{i}M_i\right)_f,
\]
同时,由于局部化函子作为张量函子,是左伴随函子,与余极限可交换,所以
\[
	\widetilde{M}(D(f))={\varinjlim}_{i}(M_i)_f={\varinjlim}_{i}\widetilde{M_i}(D(f)),
\]
复合上预层$U\mapsto {\varinjlim}_{i}\widetilde{M_f}(U)$到其伴随层${\varinjlim}_{i}\widetilde{M_f}$的典范态射,我们就得到了预层同态$\widetilde{M}\to {\varinjlim}_{i}\widetilde{M_f}$. 现在回到每条茎上,这个态射诱导了茎之间的同构,因为局部化函子作为张量函子,是左伴随函子,与余极限可交换,所以上述预层同态是一个同构。

第四点,对每一个$D(f)$,我们有
\[
	\widetilde{M\otimes_R N}(D(f))=(M\otimes_R N)_f=M_f\otimes_{R_f}N_f=\widetilde{M}(D(f))\otimes_{\oo_R(D(f))}\widetilde{N}(D(f)),
\]
复合上预层$U\mapsto \widetilde{M}(U)\otimes_{\oo_R(U)}\widetilde{N}(U)$到其伴随层的典范态射,我们就得到了预层之间的态射$\widetilde{M\otimes_R N}\to \widetilde{M}\otimes_{\oo_R}\widetilde{N}$. 回到每条茎上,上述态射诱导了以下同构(等同)
\[
	(\widetilde{M}\otimes_{\oo_R}\widetilde{N})_x=\widetilde{M}_x\otimes_{(\oo_R)_x}\widetilde{N}_x=M_{x}\otimes_{R_x}N_x=(M\otimes_R N)_x,
\]
所以预层之间的态射是同构。

最后一点考察正合列
\[
	0\to \ker \psi \to M \to \im \psi\to 0
\]
和
\[
	0\to \im \psi \to N \to \coker \psi\to 0
\]
即可。
\end{proof}

\begin{pro}[素谱上的拟凝聚层]\label{niningju}
设$U$是$\spec R$的一个拟紧开集,$\calf$是一个$\oo_X|_U$-模层,则以下条件等价:
\begin{compactenum}[~~~(1).]
\item 存在一个$R$-模$M$使得$\calf$同构于$\widetilde{M}|_U$.
\item 对$U$的任意形如$\{U_i=D(f_i)\}$的有限覆盖,每一个$\calf|_{U_i}$同构于一个$R_{f_i}$-模$M_i$的伴随层$\widetilde{M_i}$.
\item $\calf$是一个拟凝聚层。
\item 以下两条件同时满足:
\begin{compactenum}[\hspace{-1em}1.]
\item 任取$D(f)\subset U$以及$s\in \calf(D(f))$,都存在一个整数$n\geq 0$以及$U$上的一个截面$t$使得$t|_{D(f)}=f^ns$.
\item 任取$D(f)\subset U$,如果$t\in \calf(U)$限制在$D(f)$上为零,则存在一个整数$n\geq 0$使得$f^nt=0$.
\end{compactenum}
\end{compactenum}
\end{pro}

我们不再证明这个命题,详细见[EGA, Chap 1, 1.4]. 上面命题中的第一点告诉我们,在模层中,拟凝聚模层是更类似于概形的东西。

同时,我们指出,对任意的$R$-模$M$,$\widetilde{M}$是一个拟凝聚$\oo_R$-模层。实际上,由于任取$M$,一定存在正和列
\[
	R^{(I)}\to R^{(J)}\to M\to 0.
\]
由于函子$M\mapsto \widetilde{M}$是正合函子且与余极限可交换,所以存在正合列
\[
	(\oo_R)^{(I)}\to (\oo_R)^{(J)}\to \widetilde{M}\to 0,
\]
这就告诉我们$\widetilde{M}$是一个拟凝聚$\oo_R$-模层。

作为推论,由于$\spec R$本身就是拟紧的,所以$\spec R$上的拟凝聚$\oo_R$-模层都具有$\widetilde{M}$的形式,且$\spec R$的任意拟紧开集上的拟凝聚层都是由某个$\oo_R$-模层$\widetilde{M}$限制得到的。

\begin{para}[仿射概形]
一个赋环空间$(X,\oo_X)$如果同构于关于某个环$R$的赋环空间$(\spec R,\oo_R)$,则称$(X,\oo_X)$是一个仿射概形。一般而言,我们会把赋环空间$(\spec R,\oo_R)$直接记作$\spec R$.

% 如果$(X,\oo_X)$是一个仿射概形,则存在环$S$使得$(X,\oo_X)\cong (\spec S,\oo_S)$. 于是,它们的整体截面是同构的,即$\oo_X(X)\cong \oo_S(\spec S)=S$. 反之,记$(X,\oo_X)$的整体截面是$R$,如果$(X,\oo_X)$是仿射概形,则应该有$(X,\oo_X)\cong (\spec R,\oo_R)$. 为了证明这点,由于$(X,\oo_X)\cong (\spec S,\oo_S)$,所以我们只需要证明,如果$S\cong R$,则$(\spec S,\oo_S)\cong (\spec R,\oo_R)$. 
\end{para}

\begin{para}[仿射概形之间的态射]
设$\varphi:S\to R$是一个环同态,对应地,我们有连续映射${}^a\varphi:\spec R\to \spec S$. 但是,一个仿射概形是一个赋环空间,所以我们还需要给出层之间的态射
\[
	\widetilde{\varphi}:\oo_S\to {}^a\varphi_*\oo_R.
\]

首先,注意到${}^a\varphi^{-1}(V(E))=V(\varphi(E))$,见Proposition \ref{pro:3.8}. 对$E=\{f\in S\}$的情况,我们有${}^a\varphi^{-1}(D(f))=D(\varphi(f))$. 由于$\oo_S(D(f))=S_f$,而$\oo_R(D(\varphi(f)))=R_{\varphi(f)}$. 在这两个环之间,存在典范同态
\[
	\varphi_f:S_f\to R_{\varphi(f)},
\]
它将$s/f^n\in S_f$映射为$\varphi(s)/\varphi(f)^n$. 它也可以看作
\[
	\varphi_f:S(D(f))\to \oo_R(D(\varphi(f)))=\oo_R({}^a\varphi^{-1}(D(f)))={}^a\varphi_*\oo_R(D(f)).
\]
现在取$D(g)\subset D(f)$,注意到,在$S_g=S_{fg}$, $R_{\varphi(g)}=R_{\varphi(fg)}$的等同下,$\varphi_g$也等同于$\varphi_{fg}$,所以交换图
\[
	\xymatrix{
	S_f\ar[d]\ar[rr]^-{\varphi_f}&&R_{\varphi(f)}\ar[d]\\
	S_{fg}\ar[rr]^-{\varphi_{fg}}&&R_{\varphi(fg)}
	}
\]
就告诉我们,$\varphi_f$满足相容条件。由于所有形如$D(f)$的开集构成了$\spec S$的一个拓扑基,因此$\widetilde{\varphi}(D(f))=\varphi_f$就给出了层同态
\[
	\widetilde{\varphi}:\oo_S\to {}^a\varphi_*\oo_R.
\]

综上,我们将$\Phi=({}^a\varphi,\widetilde{\varphi}):\spec R\to \spec S$定义为仿射概形之间的态射。不难发现,如果$\varphi:S\to R$是一个同构,则$\Phi=({}^a\varphi,\widetilde{\varphi})$也是一个同构,这就回答了我们前面的问题。

一般地,假设存在同构$\Phi:(X,\oo_X)\to \spec R$, $\Theta:(Y,\oo_Y)\to \spec S$,如果
\[
	\Theta\Psi\Phi^{-1}:\spec R\to \spec S
\]
具有形式$({}^a\varphi,\widetilde{\varphi})$,其中$\varphi:S\to R$是一个环同态,则称赋环空间态射$\Psi:(X,\oo_X)\to (Y,\oo_Y)$是一个仿射概形态射。当然,后面我们会给一个更加自然的定义,它的表述仅仅依赖于$\Psi$.
\end{para}

\begin{para}
回忆层之间的顺像在茎上的表现(见 \ref{sx}),设$x\in \spec R$,存在一个典范同态
\[
	{}^a\varphi_x:({}^a\varphi_*\oo_R)_{{}^a\varphi(x)}\to (\oo_R)_{x}.
\]
取$(U,s)\in ({}^a\varphi_*\oo_R)_{{}^a\varphi(x)}$,其中$s\in ({}^a\varphi_*\oo_R)(U)=\oo_R({}^a\varphi^{-1}(U))$,而${}^a\varphi_x((U,s))$可以看作$s$在典范限制映射
\[
	\oo_R({}^a\varphi^{-1}(U))\to (\oo_R)_x
\]
下的像。换句话说,
\[
	{}^a\varphi_x:(U,s)\mapsto (\varphi^{-1}(U),s).
\]

同时,对层之间的态射$\widetilde{\varphi}:\oo_S\to {}^a\varphi_*\oo_R$,由于环层范畴的逆像层总是存在的,所以存在(见Theorem \ref{proinverse})
\[
	\widetilde{\varphi}^\#:{}^a\varphi^*\oo_S\to \oo_R.
\]
在茎上,它表现为
\[
	\widetilde{\varphi}^\#_x={}^a\varphi_x \widetilde{\varphi}_{{}^a\varphi(x)}:({}^a\varphi^*\oo_S)_x=(\oo_S)_{{}^a\varphi(x)}\to (\oo_R)_x.
\]
不难发现,在等同
\[
	(\oo_S)_{{}^a\varphi(x)}=S_{{}^a\varphi(x)},\quad (\oo_R)_x=R_x
\]
下,$\widetilde{\varphi}^\#_x$就是环同态$\varphi_x:S_{{}^a\varphi(x)}\to R_x$,它将$s/t$映射为$\varphi(s)/\varphi(t)$,其中$t\not\in \pp_{{}^a\varphi(x)}=\varphi^{-1}(\pp_x)$. 实际上,取$a_{{}^a\varphi(x)}\in (\oo_S)_{{}^a\varphi(x)}$,在$S_{{}^a\varphi(x)}$它写作$s/t$,此时$D(t)$是${}^a\varphi(x)$的开邻域以及$a_{{}^a\varphi(x)}=\langle D(t),s/t\rangle$,注意到$\widetilde{\varphi}(D(t))=\varphi_t$,所以
\[
	\widetilde{\varphi}^\#_x(a_{{}^a\varphi(x)})={}^a\varphi_x \circ \widetilde{\varphi}_{{}^a\varphi(x)}(a_{{}^a\varphi(x)})=\langle {}^a\varphi^{-1}(D(t)),\varphi_t(s/t)\rangle =\langle D(\varphi(t)),\varphi(s)/\varphi(t)\rangle.
\]
\end{para}

\begin{pro}
设赋环空间$(X,\oo_X)$, $(Y,\oo_Y)$是仿射概形,则赋环空间态射$\Psi=(\psi,\theta):(X,\oo_X)\to (Y,\oo_Y)$是仿射概形态射,当且仅当,任取$x\in X$,$\theta_x^\#:(\oo_Y)_{\psi(x)}\to (\oo_X)_x$是一个局部同态,即将$(\oo_Y)_{\psi(x)}$的极大理想映入$(\oo_X)_x$的极大理想中。
\end{pro}

\begin{proof}
可以直接假设$(X,\oo_X)=(\spec R,\oo_R)$, $(Y,\oo_Y)=(\spec S,\oo_S)$. 如果$\Psi=({}^a\varphi,\widetilde{\varphi})$,则此时${}^a\varphi^\#_x=\varphi_x:S_{{}^a\varphi(x)}\to R_x$,他自然是一个局部同态。

反过来,假设$\theta_x^\#:(\oo_S)_{\psi(x)}\to (\oo_R)_x$是一个局部同态。我们先定义
\[
	\varphi=\theta(\spec S):\oo_S(\spec S)=S\to \psi_*\oo_R(\spec S)=R.
\]
由于$\theta_x^\#:(\oo_S)_{\psi(x)}\to (\oo_R)_x$是局部同态,所以它诱导了域的单同态$\theta^x:k(\psi(x))\to k(x)$,取整体截面$\langle \spec S,f\rangle$以及$f(\psi(x))\in k(\psi(x))$,如下等式成立(可参考Proposition \ref{pro:3.8} 前面的推导)
\[
	\theta^x(f(\psi(x)))=\theta_x^\#(\langle \spec S,f\rangle)\,\, \text{mod}\,\, \mm_x=\langle \spec R,\varphi(f)\rangle \,\, \text{mod}\,\, \mm_x = \varphi(f)(x).
\]
因此$f(\psi(x))=0$等价于$\varphi(f)(x)=0$. 同时,注意到等式$\varphi^x(f({}^a\varphi(x)))=\varphi(f)(x)$(见Proposition \ref{pro:3.8}),所以$f(\psi(x))=0$等价于$f({}^a\varphi(x))=0$. 结合这两点,我们有$\pp_{\psi(x)}=\pp_{{}^a\varphi(x)}$,所以$\psi(x)={}^a\varphi(x)$. 由于$x$是任意的,所以作为映射,$\psi={}^a\varphi$.

最后我们只要证明$\theta=\widetilde{\varphi}$. 注意到交换图
\[
	\xymatrix{
	S\ar[d]\ar[rr]^-{\varphi}&&R\ar[d]\\
	S_{f}\ar[rr]^-{\theta(D(f))}&&R_{\varphi(f)}
	}
\]
任取$s\in S$,我们应该有等式$\theta(D(f))(s/1)=\varphi(s)/1$. 当$s=f$的时候,利用$\theta(D(f))$是一个环同态,他将逆变成逆,所以$\theta(D(f))(1/f)=1/\varphi(f)$,依然是利用$\theta(D(f))$是一个环同态,
\[
	\theta(D(f))(s/f^n)=\theta(D(f))(s/1)(\theta(D(f))(1/f))^n=\varphi(s)/\varphi(f)^n.
\]
因此$\theta(D(f))=\varphi_f=\widetilde{\varphi}(D(f))$. 由于$D(f)$构成$\spec S$的拓扑基,于是希望的$\theta=\widetilde{\varphi}$就得到了。
\end{proof}

由这个命题,仿射概形之间的态射$\Psi=(\psi,\theta):(X,\oo_X)\to (Y,\oo_Y)$可以定义为:任取$x\in X$,$\theta_x^\#:(\oo_Y)_{\psi(x)}\to (\oo_X)_x$是一个局部同态。这就诱导我们给出下面的定义。

\begin{para}
设$(X,\oo_X)$是赋环空间,如果任取$x\in X$,$(\oo_X)_x$都是一个局部环,则称$(X,\oo_X)$是一个局部赋环空间。显然,仿射概形都是局部赋环空间。设$(X,\oo_X)$, $(Y,\oo_Y)$都是局部赋环空间,如果赋环空间态射$\Psi=(\psi,\theta):(X,\oo_X)\to (Y,\oo_Y)$对任意的$x\in X$,$\theta_x^\#:(\oo_Y)_{\psi(x)}\to (\oo_X)_x$是一个局部同态,则称$\Psi$是一个局部态射。
\end{para}

所以,上一个命题就是在说:仿射概形之间的局部赋环空间态射$\Psi=(\psi,\theta):(X,\oo_X)\to (Y,\oo_Y)$是局部态射当且仅当它是一个仿射概形态射,且该局部态射完全由$\theta$的整体截面$\theta(\oo_X)$决定。换而言之,如果$\Psi=(\psi,\theta)$和$\Psi'=(\psi',\theta')$都是仿射概形之间的局部态射,则$\Psi=\Psi'$当且仅当$\theta(\oo_X)=\theta'(\oo_X)$.

\begin{thm}
仿射概形范畴等价于交换环范畴的对偶范畴。
\end{thm}

\begin{proof}
函子由$R\mapsto \spec R$以及$\varphi\mapsto ({}^a\varphi,\widetilde{\varphi})$定义。按定义,一个仿射概形同构于某个$\spec R$,所以我们只需证明
\[
	\Hom(S,R)\cong \Hom(\spec R,\spec S).
\]
为此,只需定义$f:\varphi\mapsto ({}^a\varphi,\widetilde{\varphi})$的逆即可。任取$\Phi=(\psi,\theta):(X,\oo_X)\to (Y,\oo_Y)$,我们定义$g:\Phi\mapsto \theta(\oo_X)$. 首先,$g(f(\varphi))=\widetilde{\varphi}(\oo_R)=\varphi$是简单的,其次,由于
\[
f(g(\Psi))=\left({}^a\theta(\oo_X),\widetilde{\theta(\oo_X)}\right)
\]
是一个局部态射,它完全由$\widetilde{\theta(\oo_X)}$的整体截面$\widetilde{\theta(\oo_X)}(\oo_X)=\theta(\oo_X)$决定,因此$f(g(\Psi))=\Psi$. 此即所证。
\end{proof}

这个定理用起来还是很方便的,比如$\zz$是交换环范畴的始对象,则$\spec \zz$是仿射概形范畴的终对象。再比如,如果$\varphi:S\to R$是满同态,则$({}^a\varphi,\widetilde{\varphi})$是一个单态射。

传统的代数几何,是域$k$上的仿射簇范畴与$k$上的有限生成约态$k$-代数范畴的对偶范畴等价。现代代数几何的这个对应的地位即是扩展了经典的对应。

\chapter{概形}
\section{概形}

\begin{para}[概形]
设$(X,\mathcal{O}_X)$是一个赋环空间,如果对于任意的$x\in X$,他都一个开邻域$U$使得$(U,\mathcal{O}_X|_U)$同构于一个仿射概形$\spec{R}$,则称呼$(X,\mathcal{O}_X)$是一个概形。为方便起见,我们会简单称$X$为一个概形,而需要特别指明$X$的拓扑空间结构的时候,会用$|X|$来记$X$的支集,或者叫底空间。

很容易看到任意的概形是局部赋环空间,因为仿射概形是局部赋环空间。概形构成一个范畴,概形间的态射是局部赋环空间之间的局部态射。
\end{para}

给定$f:X\to Y$是一个概形间的态射,如果$f(x)=y$,则称点$y$位于$x$上。由于存在局部同态$(\oo_Y)_y\to (\oo_X)_x$,所以它也给出了域扩张$k(y)\to k(x)$. 

\begin{para}[概形的拓扑]
设$X$是一个概形,对$X$的开集,如果$U$同构于一个仿射概形,则称$U$是一个仿射开集。特别地,当$X=\spec R$的时候,所有形如$D(f)$的开集都是仿射开集,因为$D(f)$同构于$\spec R_f$. 于是,所有仿射开集构成了概形的拓扑基,因为概形局部是仿射概形,而仿射概形的所有仿射开集构成了其拓扑基。因此,设$U$是$X$的一个开集,则$(U,\oo_X|_U)$也是一个概形,他被称为$X$的一个开子概形,更一般的子概形概念我们会在后面详细阐述,这里先按下不表。

现在考虑概形的分离性,它和仿射概形一样,都是$T_0$的。设$x$, $y$是概形上的不同点,如果存在仿射开集包含一点但不包含另一点,则我们已经得证,否则$x$, $y$都是同一个仿射开集中,此时利用仿射概形的结论即可。

对于$T_0$空间,一般点如果存在则唯一。在仿射概形中,我们知道$x\mapsto \overline{\{x\}}$是仿射概形中点与不可约闭子集的一一对应,换句话说,每一个不可约闭子集都存在唯一的一般点。来到概形$X$上,我们任取一个不可约闭子集$Y$,取$y\in Y$以及一个包含$y$的一个仿射开集$U$,则$Y\cap U$是$U$中的不可约闭子集,由仿射概形的结论,存在$x\in Y\cap U$使得$Y\cap U=\overline{\{x\}}\cap U$. 此外,$Y\cap U$当然也是$Y$中的稠密开集,所以,
\[
	Y=\overline{Y\cap U}=\overline{\overline{\{x\}}\cap U}\subset \overline{\{x\}}\cap \overline{U},
\]
故$Y\subset \overline{\{x\}}$. 反过来,由于$x\in Y$,所以$\overline{\{x\}}\subset Y$. 综上,$Y$存在一般点$x$,且由于概形是$T_0$空间,所以也唯一。

如果概形只看拓扑空间是不可约的,则称概形是不可约的。如果只看拓扑空间是连通的,则称概形是连通的。

如果概形间的态射只看连续映射是开(闭,单,满)映射,则称这是一个开(闭,单,满)态射。注意,这里的单射和满射不是指monomorphism和epimorphism,而是称injective morphism和surjective morphism,后文会避免直接称呼一个满射,而会用“态射是满的”来替代。设$\Psi=(\psi,\theta):X\to Y$是概形间态射,如果$\im\psi = Y$,则称态射$\Psi$是笼罩的(dominant)。
\end{para}

\begin{pro}\label{pro:3.1.3}
设$X$是一个概形,记$S=\Gamma(X,\oo_X)$,则存在一个典范同构
\[
	\Hom(X,\spec R)\cong \Hom(R,S),
\]
上边左侧是概形态射,右侧是环同态。
\end{pro}

\begin{proof}
	任取$\Phi=(\varphi,\theta):X\to \spec R$,则$\theta:\oo_R\to \varphi_*\oo_X$在$\spec R$上的截面就给出了同态$R\to S$. 反过来,给定同态$\psi:R\to S=\Gamma(X,\oo_X)$,我们的核心工具是Proposition \ref{rsnh}. 找一个$X$的仿射开覆盖$\{U_\alpha\}$,这限制映射复合$\psi$给出了一个一族同态
	\[
	\left\{\psi_\alpha:R\to \Gamma(U_\alpha,\oo_X)\right\},
	\]
	利用交换环范畴的对偶范畴和仿射概形范畴等价,我们可以找到对应的
	\[
	\left\{\Psi_\alpha:\spec\Gamma(U_\alpha,\oo_X)\to \spec R\right\},
	\]
	由于$U_\alpha$都是仿射开集,则$\spec\Gamma(U_\alpha,\oo_X)$可以典范等同于$U_\alpha$. 因此,我们给出了一族赋环空间之间的局部态射$\left\{\Psi_\alpha: U_\alpha \to \spec R\right\}$. 它们符合相容条件,所以Proposition \ref{rsnh} 告诉我们可以粘成唯一一个$\Psi:X\to \spec R$. 他是局部态射,所以也就是概形同态。由唯一性,上述两个态射的构造是互逆的。
\end{proof}

由于任意的环都有唯一到$\zz$的同态,所以,任取概形$X$,我们都有唯一的概形态射$X\to \spec \zz$. 因此,$\spec \zz$是概形范畴的终对象。

\begin{para}[概形的黏合]
概形的黏合就是赋环空间的黏合,见Proposition \ref{rsn},当然,由于仿射开集构成了概形的拓扑基,所以概形黏合起来得到的还是概形。在这意义上,我们可以说,每一个概形都是由仿射概形黏合起来得到的。
\end{para}

\begin{para}[局部概形]
所谓的局部概形就是局部环的谱。设$R$是一个局部环,则$\spec R$只有一个闭点$a$(闭点对应$R$的极大理想),任取其他点$b$,我们都有$a\in\overline{\{b\}}$.

任取概形$X$以及一点$x\in X$,我们称局部概形$\spec\left((\oo_{X})_x\right)$为$X$在$x$处的局部概形。任取$x$的一个仿射开集$U=\spec R$,我们可以将$(\oo_{X})_x$典范等同于$R_x$,于是典范同态$R\to R_x$就给出了态射$\spec\left((\oo_{X})_x\right)\to U$,然后复合上典范含入$U\hookrightarrow X$,我们就得到了典范态射$\spec\left((\oo_{X})_x\right)\to X$. 方便起见,我们记$\spec\left((\oo_{X})_x\right)$为$X_x$,当然这和$(\oo_X)_x$是不同的。

上述构造似乎与我们选取的仿射开集$U$有关,但事实并非如此。如果$V$是另一个包含$x$的仿射开集,则存在包含$x$的仿射开集$W\subset U\cap V$,所以我们可以假设$V\subset U$. 此时我们只需考虑仿射概形的情况,而限制映射的相容性就给出了结论。

下面我们要说明典范同态的像在$X$中是如何的,为此首先记$S_x=\left\{y\in X\,:\, x\in \overline{\{y\}}\right\}$. 不难发现,$S_x=\bigcap_{U\ni x}U$,其中$U$跑遍$x$的开邻域。实际上,如果$x\in \overline{\{y\}}$,则$U\cap \overline{\{y\}}$作为$\overline{\{y\}}$中的开集\footnote{一个空间如果有一般点,则其中任意开集都包含这个一般点。}必然包含$y$,所以$y\in U$,这给出了$S_x\subset \bigcap_{U\ni x}U$. 反过来,如果$y\not\in \bigcap_{U\ni x}U$,换言之存在$x$的一个开邻域$U$使得$y\not\in U$,即$X-U$是包含$y$的一个闭集,因此$\overline{\{y\}}\cap U=\varnothing$,这给出$x\not\in \overline{\{y\}}$. 所以$\bigcap_{U\ni x}U\subset S_x$.

现在考虑典范态射$(\psi,\theta):X_x\to X$. 显然,$\psi^{-1}(x)$是$X_x$中唯一的闭点,于是任取$a\in X_x$,我们有
\[
	x=\psi(\psi^{-1}(x))\in \psi\left(\overline{\{a\}}\right)\subset \overline{\{\psi(a)\}}.
\]
所以$\im \psi\subset S_x$. 更进一步,我们其实还有$\im \psi=S_x$. 由于$S_x$包含于任意一个$x$的仿射开邻域中,所以可以假设概形$X=\spec R$,而这时$\psi={}^a i$,其中$i:R\to R_x$是典范同态,此时$\im({}^a i)$即$\spec R$中满足$\pp_y\subset \pp_x$的$y$构成的子集(见Lemma \ref{lem:3.9} 的推论),由于$\pp_y\subset \pp_x$等价于$x\in \overline{\{y\}}$,所以$\im({}^a i)$即$S_x$. Lemma \ref{lem:3.9} 同时还告诉我们,$\psi$是从$X_x$到$\im\psi$的一个同胚。并且,由于$X$中所有不可约闭子集都具有形式$\overline{\{y\}}$,因此$X_x$中的点与$X$中所有包含$x$的不可约闭子集之间有一个一一对应。

最后,取$z=\psi(y)$,则$\theta_y^\#:(\oo_X)_z\to ((\oo_X)_x)_y$是一个同构。实际上,只需考虑$X=\spec R$, $\theta=\widetilde{i}$的情况,此时上面的同态即$i_y:R_z\to (R_x)_y$. 这个是显然的同构。
\end{para}

总结一下上面的讨论:

\begin{pro}
$X_x$作为集合是所有包含$x$的开集的交,等价地,$X_x$中的点由所有包含$x$的不可约闭子集的一般点构成,因此$X_x$中的点一一对应于$X$中所有包含$x$的不可约闭子集。此外,典范态射$X_x\to X$是一个概形范畴的单态射。
\end{pro}

如果$x$是一般点,则$X_x$只有一个点即$x$,即$(\oo_X)_x$只有一个素理想,当然也只有一个极大理想。反之亦然。

顺带一提,一般来说$X_x$不是$X$的子概形,因为他可能不是局部闭的,但如果$x$是闭点,则$X_x$是闭子概形。具体定义见子概形章节。

\begin{pro}\label{pro:4.6}
设$\spec A$是一个局部概形,闭点记作$a$,又设$X$是一个概形,则任意态射$u=(\psi,\theta):\spec A\to X$都可以经由典范态射$i_{\psi(a)}:X_{\psi(a)}\to Y$唯一分解,即唯一存在态射$v:\spec A\to X_{\psi(a)}$使得$u=i_{\psi(a)}v$. 
\end{pro}

\begin{proof}
任取$b\in \spec A$,总是有$b\in\overline{\{a\}}$,所以$\psi(b)\in \overline{\{\psi(a)\}}$. 换言之,我们有$\im \psi\subset S_{\psi(a)}$. 由于$S_{\psi(a)}$包含于$\psi(a)$的任意开邻域中,所以也包含于$\psi(a)$的任意仿射开邻域中。所以不妨假设$X=\spec B$,从而$u=({}^a\varphi,\widetilde{\varphi})$,而$\varphi:B\to A$是一个环同态。

由于$\varphi^{-1}(\pp_a)=\pp_{\psi(a)}$,所以$B-\pp_{\psi(a)}$的任意元素在$\varphi$下的像总是局部环$A$中的可逆元,所以由分式环的泛性质,存在唯一的环同态$\sigma:(B-\pp_{\psi(a)})^{-1}B=B_{\psi(a)}\to A$使得分解$\varphi=\sigma i$成立,其中$i:B\to B_{\psi(a)}$是分式环的典范同态。由于交换环范畴的对偶范畴与仿射概形范畴是等价范畴,所以在仿射概形范畴范畴,我们也有唯一分解$u=i_{\psi(a)}v$,其中$u$对应环同态$\varphi$,$i_{\psi(a)}$对应环同态$i$,而$v$对应环同态$\sigma$.
\end{proof}

\begin{para}[$S$-概形]
设$S$是一个概形,而$X$是另一个概形,一个$S$-概形是一个二元组$(X,\psi)$,其中$\psi:X\to S$是一个概形态射。在仿射概形的情况下,一个$\spec S$-概形$\spec R$不过是在说$S$是一个$R$-代数(因为箭头是反向的)。所以类似于称呼$R$-代数,我们在称呼$S$-概形的时候,经常也略去定义态射$\psi$. 特别地,当我们讨论$\spec A$-范畴的时候,我们也称为一个$A$-范畴。

两个$S$-概形$X$, $Y$间的态射$\varphi$被称为$S$-态射,如果满足交换图
\[
	\xymatrix{
	X\ar[rr]^\varphi \ar[rd]&&Y\ar[ld]\\
	&S&
	}
\]
所有$S$-概形和$S$-态射构成一个范畴,态射集记作$\Hom_S$. 这是非常自然的定义,类似于$R$-代数范畴。

由于概形范畴有终对象$\spec \zz$,所以,每一个概形都可以看成一个$\spec \zz$-概形,或者按照前面介绍过的习惯,也被称为$\zz$-范畴,这是任何环都可以看成$\zz$-代数的一个类比。
\end{para}

\begin{para}[$T$-值点]
设$X$是一个概形,而$T$是任意概形,记$X(T)=\Hom(T,X)$. 于是$X$构成一个概形范畴到集合范畴的反变函子,而$X(T)$中的元素被称为概形$X$的$T$-值点。现在来到$S$-概形范畴,我们可以同样定义$X(T)_S=\Hom_S(T,X)$. 如果不会造成误解,我们有时会略去它。此外,如果$T=\spec A$,则我们也会称$X(T)$中的元素为概形$X$的$A$-值点,如果不会发生误解。

将$X$看成这样一个函子的理由一部分来自于Yoneda引理,如果$X$和$Y$作为反变函子同构,则$X$和$Y$作为概形也同构。
\end{para}

\begin{para}[几何点]
设$A$是一个局部环,而$X$是一个概形,我们考虑$X$的$A$-值点。由Proposition \ref{pro:4.6},$X(A)$中的元素一一对应于某个局部同态$(\oo_X)_x\to A$,其中$x\in X$,这个$x$被称为$A$-值点的位置,或者说这个$A$-值点位于$x$. 进而,我们定义$X$的在某个域$k$上取值的点为一个在$k$上取值的几何点。

给出一个在$k$上取值的几何点,等价于给出它的位置$x$以及一个域扩张$k(x)\to k$. 这是因为局部同态$(\oo_X)_x\to k$将$(\oo_X)_x$的极大理想映成了$0$(作为$k$的极大理想),所以给出局部同态$(\oo_X)_x\to k$也等价于给出域扩张$k(x)\to k$. 

现在设$X$, $\spec k$都是$S$-概形,其中$k$是一个域。由上所述,$\spec k$是一个$S$-概形等价于给定$s\in S$以及一个域扩张$k(s)\to k$. 考虑一个几何点$\psi\in X(k)_S$,它等价于给出$x\in X$以及一个域扩张$k(x)\to k$. 由分解
\[
	\spec k \xrightarrow{\psi} X\xrightarrow{f} S,
\]
其中$f$是$X$作为$S$-概形的定义态射,我们应有$s=f(x)$以及分解
\[
	k(s)\to k(x)\to k.
\]
此时,我们称几何点$\psi$是一个位于$s$上取值于$k$中的点。
\end{para}

关于域扩张,我们可能以后需要下面的引理,这是域扩张的基本常识,但是方便起见,我们还是列出并给予证明。它依赖于选择引理,因为要保证极大理想(素理想)的存在性。

\begin{lem}\label{composite_ext}
设$\{u_i:k\to k_i\}$是有限个$k$的扩张,其中$k_i$当然都是域,于是存在一个域$K$以及域扩张$\{v_i:k_i\to K\}$和$w:k\to K$使得$w=v_iu_i$对所有的$i$都成立。
\end{lem}

\begin{proof}
考虑张量积
\[
	R=k_1\otimes_k \cdots \otimes_k k_n,
\]
这是一个$k$-代数,一般来说不是一个域也不是一个整环。选一个$R$的极大理想$\mm$,考虑商域$K=R/\mm$,我们有自然的同态$\pi:R\to K$. 现在记$p_i$为$k_i$到$R$的典范态射,则
\[
	v_i=\pi p_i:k_i\to K
\]
就满足$v_iu_i=v_ju_j$. 实际上,任取$a\in k$,由张量积的性质
\[
	1\otimes_k \cdots\otimes_k u_i(a)\otimes_k \cdots\otimes_k 1=1\otimes_k \cdots\otimes_k u_j(a)\otimes_k \cdots\otimes_k 1,
\]
所以$p_iu_i(a)=p_ju_j(a)$. 
\end{proof}

\section{概形的纤维积}

\begin{para}[纤维积]
设$X$, $Y$是$S$-概形,类似任何范畴中纤维积的定义,我们定义$X\times_S Y$是图$X\to S\leftarrow Y$的极限。用泛性质表述的话,即如下交换图
\[
\begin{xy}
	\xymatrix{
		A\ar@/_/[rdd]_{p_2}\ar@/^/[drr]^{p_1}\ar@{-->}[dr]&&\\
		&X\times_S Y\ar[d]^{\pi_2}\ar[r]_-{\pi_1}&X\ar[d]\\
		&Y\ar[r]&S
	}
\end{xy}
\]
类似于任何范畴的情况,如果存在终对象,则我们可以由纤维积构造积。具体到概形范畴,即$X\times Y=X\times_{\spec \zz} Y$.

一般而言,如果$S=\spec R$,则$X\times_S Y$也会记作$X\times_R Y$. 所以$X\times Y=X\times_{\zz} Y$.
\end{para}

概形的纤维积的存在性先按下不表,不过对一个特殊的情况,我们可以直接写出纤维积:
\[
	\spec A\times_{\spec R} \spec B=\spec(A\otimes_R B).
\]
利用交换环范畴的对偶与仿射概形范畴的等价,直接检查泛性质就行。

此外,如果记$|X|$为概形$X$的底空间(或称支集),则由$|X|\times_{|S|}|Y|$的泛性质,我们得到了典范态射$|X\times_S Y|\to |X|\times_{|S|}|Y|$. 这个典范态射一般不是单射,但后面会证明,这是一个满射。当然,这个典范态射不是同胚的理由是不难猜的,因为不是所有连续映射$|X|\to |S|$都可以写成某个概形间态射$X\to S$的连续函数部分。实际上,对任意的域$k$,$|\spec k|$都是单点集。然后我们考虑$f:|\spec \mathbb{Q}|\to |\spec \rr|$,他将左侧唯一的点映成右侧唯一的点,自然是连续的。如果他可以写成概形间一个态射$\spec \mathbb{Q}\to \spec \rr$的连续函数部分,即$f={}^a\psi$,而$\psi:\rr\to \mathbb{Q}$是一个域同态。它是一个单射且$\psi(1)=1$,由环同态与环乘法的性质,我们有$\psi$还是满的,因此$\psi:\rr\to \mathbb{Q}$是一个同构,显然不可能,毕竟两边基数都不同。

\begin{pro}\label{pro:3.2.2}
纤维积具有如下一些性质,它们并不依赖于概形范畴:
\begin{compactenum}[~~~1.]
\item 设$X$是一个$S$-概形,则存在典范等同$X\times_S S=S\times_S X=X$,这个等同是函子性。同时,在这个等同下,设$f:X\to S$和$g:Y\to S$是两个$S$-概形,则$f\times_S 1_Y:X\times_S Y\to S\times_S Y=Y$和$1_X\times_S g:X\times_S Y\to X\times_S S=X$就是典范投影。
\item 设$X$, $S'$是一个$S$-概形,$S''$是一个$S'$概形,则可以典范等同$(X\times_S S')\times_{S'}S''=X\times_S S''$,这个等同是函子性。
\item 存在典范等同$(X\times_S Y)\times_S Z=X\times_S (Y\times_S Z)$,这个等同是函子性的。
\item 若$f:S'\to S$是一个单态射,而$X$, $Y$都是$S'$-概形,则通过$f$可将$X$, $Y$看成$S$-概形,则成立典范等同$X\times_{S'}Y=X\times_S Y$.
\end{compactenum}
\end{pro}

\begin{proof}
前三点都是范畴论里面的标准结论,不再证明。至于第四点,我们来证明$X\times_{S'} Y$满足$X\times_S Y$的泛性质。

设$X$和$Y$的典范态射为$\varphi:X\to S'$和$\psi:Y\to S'$,任取$u:T\to X$和$v:T\to Y$满足$f\varphi u=f\psi v$,由于$f$是单态射,所以$\varphi u =\psi v$,进而从$X\times_{S'} Y$的泛性质,存在唯一的态射$(u,v):T\to X\times_{S'} Y$使得$u=\pi_X (u,v)$以及$u=\pi_Y (u,v)$,其中$\pi_X$, $\pi_Y$是典范投影。此即所需。
\end{proof}

\begin{para}[基概形扩张]
设$X$, $S'$都是$S$-概形,其中$S'$作为$S$-概形的定义态射为$\varphi:S'\to S$,则$X\times_S S'$在$\pi_2:X\times_S S'\to S'$下构成一个$S'$-概形。这个过程被称为基概形扩张,方便起见,我们又是会记$X\times_S S'$为$X_{(S')}$或者$X_{(\varphi)}$. 设$f:X\to Y$是$S$-概形态射,我们存在$f\times_S \id_{S'}:X\times_S S'\to Y\times_S S'$,这是一个$S'$-概形扩张。方便起见,我们记作$f_{(S')}$或者$f_{(\varphi)}$. 
\end{para}

\begin{pro}
存在函子性等同$(X\times_S Y)_{(S')}=X_{(S')}\times_{S'} Y_{(S')}$. 因此,基概形扩张与纤维积是可交换的。
\end{pro}

\begin{proof}
考虑
\[
	(X\times_S S')\times_{S'}Y_{(S')}=X\times_{S}Y_{(S')}=X\times_{S}(Y\times_S S')=(X\times_{S} Y)\times_S S',
\]
所以$(X\times_S Y)_{(S')}=X_{(S')}\times_{S'} Y_{(S')}$.
\end{proof}

\begin{para}[原像]
基概形扩张有时候又被称为原像。这不是没有原因的,回到集合范畴,如果$Y\subset S$以及$\psi:X\to S$,则$X\times_S Y$不过就是$Y$关于$\psi$的原像$\psi^{-1}(Y)\subset X$. 

在这里,如果$X$是一个$S$-概形,而$\varphi:S'\to S$是另一个$S$-概形,则我们记$\varphi^{-1}(X)=X_{(\varphi)}$,这是一个$S'$-概形,称为$S$-概形$X$在$\varphi:S'\to S$下的原像。
\end{para}

作为原像的例子,我们有下面的命题。

\begin{pro}\label{preimage}
设$U$是$S$的一个开子概形,$X$是一个概形,而$\Psi=(\psi,\theta):X\to S$是一个概形同态。利用典范同态$U\hookrightarrow S$,则$U$可以看成一个$S$-概形,此时纤维积$U\times_S X=\Psi^{-1}(U)$存在,他就是$X$的开子概形$\psi^{-1}(U)$.
\end{pro}

所以,此时$\Psi^{-1}(U)$就是经典意义下的原像。方便起见,我们时常会直接用连续映射$\psi$来指代$\Psi$.

\begin{proof}
直接检查开子概形$\psi^{-1}(U)$是否满足纤维积$U\times_S X$的泛性质即可,取交换图
\[
\begin{xy}
	\xymatrix{
		Z\ar@/_/[rdd]_{g}\ar@/^/[drr]^{f}\ar@{-->}[dr]&&\\
		&\psi^{-1}(U)\ar[r]_-{p}\ar[d]^{i_{\psi^{-1}(U)}}&U\ar[d]\\
		&X\ar[r]_-\Psi & S
	}
\end{xy}
\]
其中$i_{\psi^{-1}(U)}$是典范含入,而$p$来自于Proposition \ref{pro:1.20},满足唯一分解$\Psi|_{\psi^{-1}(U)}=i_Up$,其中$i_U:U\to S$是典范含入。

现在设存在$f:Z\to U$以及$g:Z\to X$使得上面的交换图成立,即$i_U f=\Psi g$. 设$g=(\varphi,\sigma)$,分解$i_U f=\Psi g$的连续函数部分给出了$\im(\psi\varphi)\subset U$,因此$\im \varphi\subset \psi^{-1}(U)$. 利用Proposition \ref{pro:1.20},我们得到了唯一的态射$g':Z\to\psi^{-1}(U)$使得分解$g=i_{\psi^{-1}(U)}g'$成立。

最后,只要检查$g'$满足上述交换图即可。由于
\[
	i_U f = \Psi g = \Psi i_{\psi^{-1}(U)}g'=\Psi|_{\psi^{-1}(U)}g'=i_Upg',
\]
由于$i_U$是单态射,是左可消的,此即$f=pg'$.
\end{proof}

\begin{coro}
设存在纤维积$X\times_S Y$,$p_X:X\times_S Y\to X$是其典范投影态射,再设$U$是$X$的一个开子概形,则概形$Y\times_S U=(X\times_S Y)\times_X U$就是$X\times_S Y$的开子概形$p_X^{-1}(U)$.
\end{coro}

% \begin{proof}
% 直接检查$\pi_X^{-1}(U)$是否满足纤维积$Y\times_S U$的泛性质即可,取交换图
% \[
% \begin{xy}
% 	\xymatrix{
% 		Z\ar@/_/[rdd]_{g}\ar@/^/[drr]^{f}\ar@{-->}[dr]&&\\
% 		&\pi_X^{-1}(U)\ar[r]_-{q_X}\ar[d]^{q_Y}&U\ar[d]\\
% 		&Y\ar[r]&S
% 	}
% \end{xy}
% \]
% 其中$q_X$, $q_Y$分别是$p_X$, $p_Y$限制在${\pi_X^{-1}(U)}$得到的态射,特别是$q_X$来自于Proposition \ref{pro:1.20},满足$p_X|_{\pi_X^{-1}(U)}=i_Uq_X$,其中
% $i_U:U\to X$是典范含入。

% 对$i_Uf:Z\to X$与$g:Z\to Y$,由纤维积的泛性质,我们有唯一的态射$h=(\psi,\sigma):Z\to X\times_S Y$,使得分解$i_Uf=p_X h$和$g=p_Yh$成立。分解$if=p_X h$的连续映射部分给出$\im(\pi_X\psi)\subset U$,所以$\im \psi\subset \pi_X^{-1}(U)$. 利用Proposition \ref{pro:1.20},我们得到了唯一的态射$h':Z\to\pi_X^{-1}(U)$使得分解$h=i_{\pi_X^{-1}(U)}h'$成立,其中$i_{\pi_X^{-1}(U)}:\pi_X^{-1}(U)\to X\times_S Y$是典范含入。

% 最后,只要检查$h'$满足上述交换图即可。这来自于
% \[
% 	g=p_Yh=p_Y i_{\pi_X^{-1}(U)}h'=p_Y|_{\pi_X^{-1}(U)}h'=q_Yh',
% \]
% 以及
% \[
% 	i_U f = p_X h = p_X i_{\pi_X^{-1}(U)}h'=p_X|_{\pi_X^{-1}(U)}h'=i_Uq_Xh',
% \]
% 由于$i_U$是单态射,所以也是左可消的,此即$f=q_Xh'$.
% \end{proof}

\begin{coro}\label{coro:3.2.8}
设存在纤维积$X\times_S Y$,再设$U$是$X$的开子概形而$V$是$Y$的开子概形,则$U\times_S V$就是$X\times_S Y$的开子概形$p_X^{-1}(U)\cap p_Y^{-1}(V)$. 因此,如果两个$S$-概形有纤维积,则它们的任意开子概形(当然都是$S$-概形)之间也存在纤维积。
\end{coro}

\begin{proof}
首先注意到,将$p_Y$限制在$p_X^{-1}(U)$上得到了态射$p_X^{-1}(U)\to Y$,连同典范含入$V\hookrightarrow Y$,我们考虑纤维积$p_X^{-1}(U)\times_Y V$,由Proposition \ref{preimage},它存在且作为概形等于$p_X^{-1}(U)$的开子概形
\[
	\left(p_Y|_{p_X^{-1}(U)}\right)^{-1}(V)=p_X^{-1}(U)\cap p_Y^{-1}(V),
\]
$p_X^{-1}(U)\cap p_Y^{-1}(V)$是$p_X^{-1}(U)$的开子概形,当然也就是$X\times_S Y$的开子概形。同时,上个推论告诉我们
\[
	p_X^{-1}(U)\times_Y V=(U\times_S Y)\times_Y V=U\times_S V,
\]
因此$U\times_S V=p_X^{-1}(U)\cap p_Y^{-1}(V)$. 
\end{proof}

% 这个推论很管用,有了它,我们局部计算概形的纤维积才成了可能。当然,现在的问题是我们完全不知道纤维积的典范投影$p_X$和$p_Y$如何表现。

\begin{thm}[纤维积的存在性]\label{thm:1.5.9}
概形范畴存在纤维积。
\end{thm}

\begin{lem}\label{lem:3.2.10}
设$X$, $Y$都是$S$-概形,而$\{X_\alpha\}$是$X$的一个开覆盖,如果$X_\alpha\times_S Y$都存在,则$X\times_S Y$存在。
\end{lem}

\begin{proof}
设$p_\alpha:X_\alpha\times_S Y\to X_\alpha$是典范投影态射。由于$X_{\alpha\beta}=X_{\alpha}\cap X_{\beta}$是$X_{\alpha}$和$X_{\beta}$的开子集,此时纤维积$X_{\alpha\beta}\times_S Y$由上面的推论是存在的,且就是$X_\alpha\times_S Y$的开子概形$p_\alpha^{-1}(X_{\alpha\beta})$,同样也就是$X_\beta\times_S Y$的开子概形$p_\beta^{-1}(X_{\alpha\beta})$. 所以,由纤维积在同构意义下的唯一性,我们有了概形的同构
\[
	\varphi_{\alpha\beta}:p_\alpha^{-1}(X_{\alpha\beta})\to p_\beta^{-1}(X_{\alpha\beta}),
\]
所以我们通过等同$p_\alpha^{-1}(X_{\alpha\beta})$和$p_\beta^{-1}(X_{\alpha\beta})$可以将$X_\alpha\times_S Y$和$X_\beta\times_S Y$黏起来。更一般地,不难检查黏合条件(不同黏法黏出同一样东西)成立,这样将所有$\{X_\alpha\times_S Y\}$黏成的概形,可以检查就是$X\times_S Y$,典范态射是将所有$p_\alpha$黏合起来得到的。
\end{proof}

\begin{lem}\label{lem:3.2.11}
设$\varphi:X\to S$, $\psi:Y\to S$都是$S$-概形,而$U$是$S$的一个开子集,则$\varphi^{-1}(U)\to U$和$\psi^{-1}(U)\to U$的纤维积$\varphi^{-1}(U)\times_U \psi^{-1}(U)$就是$\varphi^{-1}(U)\times_S Y$或$X\times_S \psi^{-1}(U)$.
\end{lem}

\begin{proof}
实际上,我们只需要检查交换图
\[
\begin{xy}
	\xymatrix{
		Z\ar@/_/[rdd]_{g}\ar@/^/[drrr]^{f}\ar@{-->}[dr]^h&&&\\
		&A\ar[rr]_-{i_{\psi^{-1}(U)}p_2}\ar[d]^{p_1}&&Y\ar[d]^\psi\\
		&\varphi^{-1}(U)\ar[rr]^-{\varphi'}&&S
	}
\end{xy}
\]
其中$\varphi'$满足唯一分解$\varphi=i_{\varphi^{-1}(U)}\varphi'$. 由于$\psi f=\varphi' g$,所以$\im(\psi f)\subset U$,或者$\im f\subset \psi^{-1}(U)$,因此,我们可以唯一分解$f=i_{\psi^{-1}(U)}f'$,其中$f':Z\to \psi^{-1}(U)$. 现在考虑$A=\varphi^{-1}(U)\times_{U}\psi^{-1}(U)$的泛性质,存在唯一的态射$h:Z\to A$,使得$g=p_1h$且$f'=p_2h$,所以$f=i_{\psi^{-1}(U)}f'=i_{\psi^{-1}(U)}p_2h$. 这就检验了上述交换图。
\end{proof}

\begin{proof}[The proof of Theorem \ref{thm:1.5.9} ]
设$X$, $Y$都是$S$-概形,定义态射分别是$\varphi:X\to S$和$\psi:Y\to S$. 先设$S$, $Y$是仿射概形,找一个$X$的仿射开覆盖,由于两个仿射概形在仿射概形上的纤维积是存在的,所以Lemma \ref{lem:3.2.10} 告诉我们,此时$X\times_S Y$存在,即一个概形与一个仿射概形在仿射概形上的纤维积是存在的。现在回到$Y$是一般的概形,由于此时$X$与$Y$的任意仿射开集在$S$上的纤维积是存在的,再一次利用Lemma \ref{lem:3.2.10},就得到了此时$X\times_S Y$存在。

现在设$\{S_\alpha\}$是$S$的一个仿射开覆盖,Lemma \ref{lem:3.2.11} 告诉我们,$\varphi^{-1}(S_\alpha)\to S_\alpha$和$\psi^{-1}(S_\alpha)\to S_\alpha$的纤维积$A_\alpha=\varphi^{-1}(S_\alpha)\times_{S_\alpha}\psi^{-1}(S_\alpha)$就是$\varphi^{-1}(S_\alpha)\times_S Y$. 由于$S_\alpha$是仿射的,所以$A_\alpha=\varphi^{-1}(S_\alpha)\times_S Y$存在。由于$\bigcup_\alpha\varphi^{-1}(S_\alpha)=\varphi^{-1}\left(\bigcup_\alpha S_\alpha\right)=\varphi^{-1}(S)=X$,所以$\varphi^{-1}(S_\alpha)$是$X$的一个开覆盖,由Lemma \ref{lem:3.2.10},$X\times_S Y$存在。
\end{proof}

\begin{para}[纤维积下的几何点]
设$X$, $Y$都是$S$-概形,而$T$是任意概形,由纤维积的泛性质,我们有如下集合的典范等同
\[
	(X\times_S Y)(T)=X(T)\times_{S(T)}Y(T).
\]
这个等式实际上在任意的范畴都是成立的,其实这不过是泛性质的再表述而已。

现设$f:X\to S$和$g:Y\to S$是定义态射,然后取$T=k$是一个域。给定$\varphi:\spec k\to X$和$\psi:\spec k\to Y$满足$f\varphi=g\psi$,则由泛性质(或者说上面的集合典范等同),存在$\varphi\times_S\psi:\spec k\to X\times_S Y$使得分解
\[
	\varphi=p_X(\varphi\times_S\psi),\quad \psi=p_Y(\varphi\times_S\psi),
\]
其中$p_X$和$p_Y$都是典范投影。

回忆一些结论和记号习惯,给定概形间态射$\varphi:\spec k\to X$就是在说给定$\varphi(a)$和一个域扩张$k(\varphi(a))\to k$. 还有,对$S$-概形$X$,定义态射为$f:X\to S$,如果$f(x)=s$,则称$x\in X$位于$s\in S$上。

所以上面说了那么一大串,可以重新表为:给定位于同一点$s\in S$上的点$x\in X$和$y\in Y$,以及域扩张$k(x)\to k$和$k(y)\to k$,使得以下复合
\[
	k(s)\to k(x)\to k \quad \text{and} \quad k(s)\to k(y)\to k
\]
得到的是同一个域扩张$k(s)\to k$,则$(x,y)\in X\times_S Y$. 
\end{para}

我们可以将这个判别法拓展到有限纤维积上,这并没有增加任何困难。从这个小应用,我们可以大概对几何点有所感受,粗略地说,几何点$\spec k\to X$代表了$X$上的一个点。

下面的命题是上面这个判别法的直接应用,命题虽然没有提到域扩张或者几何点,但是技术上是利用了它们,将点看成一个态射。

\begin{pro}\label{pro.2.12}
	设$\{X_i\,:\,1\leq i\leq n\}$都是$S$-概形,对每一个指标$i$,设$x_i\in X_i$,则$(x_1$, $\dots$, $x_n)\in X_1\times_S \cdots\times_S X_n$当且仅当每一个$x_i$都位于同一个点$s\in S$上。
\end{pro}

\begin{proof}
设$X_i$的定义态射为$f_i$,而$X_1\times_S \cdots\times_S X_n$的定义态射为$f$. 如果$x=(x_1$, $\dots$, $x_n)\in X_1\times_S \cdots\times_S X_n$,则$f_i(x_i)=f(x)$,它们都位于同一点$f(x)$上。反过来,此时$f_i(x_i)=f_j(x_j)=s\in S$,考虑$f_i$自然诱导的域扩张$k(f_i(x_i))=k(s)\to k(x_i)$,从Lemma \ref{composite_ext},存在域扩张$k(s)\to k$和$k(x_i)\to k$使得分解
\[
	k(s)\to k(x_i)\to k
\]
对每一个指标$i$都成立。所以,上面那个判别法给出了$(x_1$, $\dots$, $x_n)\in X_1\times_S \cdots\times_S X_n$.
\end{proof}

作为推论,我们可以断言,典范连续映射$|X\times_S Y|\to |X|\times_{|S|}|Y|$是满的。但是,一般而言,这并不是单的,换而言之,$X\times_S Y$中可能存在不同的点但是有着相同的投影!比方说,我们考虑两个域扩张$k\to k_1$和$k\to k_2$,则$\spec(k_1)\times_{\spec k} \spec(k_2)=\spec(k_1\otimes_k k_2)$. 此时$k_1\otimes_k k_2$一般来说并不是一个域,它可能具有多个素理想,但是它们的投影必然是唯一的,即$|\spec(k_1)|\times_{|\spec k|} |\spec(k_2)|=|\spec(k_1)|\times |\spec(k_2)|$只有一个点。

同时,这也说明了,不是所有$X\times_S Y$中的点都可以写成$(x,y)$的形式。

\begin{pro}
	设$f:\spec A\to \spec B$是一个仿射概形间的态射,再令$p\in \spec B$为像$f(\spec A)$的Zariksi闭包
	中的一个点,则存在$q\in f(\spec A)$使得$p\in \overline{\{q\}}$.
\end{pro}

一般来说,任取$\spec R$中的子集$X$,他的闭包中的点$p$并不可以找到$q\in X$使得$p\in \overline{\{q\}}$.

\begin{proof}
用$f(\spec A)$的Zariski闭包替代$\spec B$,我们可以假设$f(\spec A)$稠密。
此时,任取$p\in \spec B$,我们找包含于$p$的一个极小素理想
$q\subset p$,可以看到$p\in \overline{\{q\}}$. 从Zorn引理,
我们知道这样的极小素理想存在,因为此时素理想递减链的交集也是素的。下面验证$q\in f(\spec A)$,
即找到$q'\in \spec A$使得$f(q')=q$.

我们的主要思路是考虑如下的交换图,
\[
\xymatrix{
	B\ar[r]\ar[d]& B_q\ar[d]\\
	A\ar[r]&A\otimes_B B_{q}
}
\]
如果沿着$B\to B_q\to A\otimes_B B_{q}$,我们可以找到$A\otimes_B B_{q}$中的某个素理想$p'$一路原像
回到$B$的时候是$q$,则$p'$在另一条路径中$A$的原像就是我们要找的$q'$. 这里很重要的一点是,我们需要说明
$A\otimes_B B_{q}\neq 0$,就是他至少存在一个素理想。为此,我们首先看到$h:B\to A$是个单射,这是因为
$f(\spec A)$稠密,而他又包含于$V(\ker h)$,所以后者等于整个$\spec B$,继而$h:B\to A$是一个单射。
由于局部化把保持单射,所以$B_q\to A\otimes_B B_{q}$是一个单射。

现在注意到,$B_q$只有一个素理想,就是$q$的局部化,他当然在局部化下的原像就是$q$.
最后对于$B_{q}\to A\otimes_B B_q$,任取$A\otimes_B B_q$中的素理想$p'$,他在域$B_q$中的原像作为素理想
只能是$q$的局部化。
\end{proof}

从证明过程来说,我们找了一个点$q$,以及乘积$X=\spec A\times_{B}(\spec B)_q$,存在$q'\in f^{-1}(p)$使得$(q',0)\in X$.

\begin{pro}\label{pro.2.13}
考虑如下两个命题:
\begin{compactenum}[~~~(1)]
\item 若$f$和$g$是两个$S$-态射,它们都具有性质$\mathsf{P}$,则$f\times_S g$也具有性质$\mathsf{P}$.
\item 若$f:X\to Y$是一个$S$-态射,具有性质$\mathsf{P}$,则对任意的$S$-概形$S'$,经过基概形扩张后,$f_{(S')}:X_{(S')}\to Y_{(S')}$也具有$\mathsf{P}$.
\end{compactenum}
我们成立如下论断:
\begin{compactenum}[~~(i)]
\item 如果对任意的$X$,$1_X$都具有性质$\mathsf{P}$,则(1)可以推出(2).
\item 如果任意两个具有性质$\mathsf{P}$的态射复合依然具有$\mathsf{P}$,则(2)可以推出(1).
\end{compactenum}
\end{pro}

\begin{proof}
因为$f_{(S')}=f\times_S 1_{(S')}$,因此(i)是自然的。设$f:X\to X'$和$g:Y\to Y'$都是$S$-态射,成立复合关系
\[
	f\times_S g=(1_{X}\times_S g)(f\times_S 1_Y),
\]
因此(ii)也是自然的。
\end{proof}

我们时常要关注一个态射在任意的基概形扩张下的表现,有些性质是良好的,在基概形扩张下表现良好,但有些性质就不那些好。比方说,如果一个态射是满的,则它在任意基概形扩张后依然是满的。但是,一个单的态射在基概形扩张后就不一定还是单的。对应地,我们有下面这样一个结论。

\begin{pro}\label{pro.2.14}
设$f:X\to Y$是一个概形间态射,以下命题等价:
\begin{compactenum}[~~~(a)]
\item 对任意的域$k$,映射$f_*:X(k)\to Y(k)$总是单的。
\item 对任意的代数闭域$k$,映射$f_*:X(k)\to Y(k)$总是单的。
\item $f$是单的,且在任意基概形扩张后就还是单的。
\item $f$是单的,且对每一个$x\in X$,典范诱导的域扩张$k(f(x))\to k(x)$都是纯不可分扩张。
\end{compactenum}
\end{pro}

\begin{proof}
具体见[EGA, Chap 1, 3.5],后面我们会用到$(a)\Rightarrow (b)\Rightarrow (c)$,所以只证明这部分,其中$(a)\Rightarrow (b)$是显然的。假设对任意的代数闭域$k$,映射$f_*:X(k)\to Y(k)$总是单的。

首先我们证明$f$是单的。取$x_1$, $x_2\in X$使得$f(x_1)=f(x_2)=y$. 利用Lemma \ref{composite_ext},可以找到代数闭域$k$以及相应扩张$i:k(y)\to k$, $j_1:k(x_1)\to k$, $j_2:k(x_2)\to k$使得分解成立
\[
	i:k(y)\to k(x_1)\xrightarrow{j_1} k,\quad i:k(y)\to k(x_2)\xrightarrow{j_2} k.
\]
由于扩张$j_i:k(x_i)\to k$等价于给出态射$u_i:\spec k\to X$且$\im u_i=\{x_i\}$,此时$i$的两个分解等价于$f_*(u_1)=f_*(u_2)$,由假设可知,$u_1=u_2$,所以$x_1=x_2$.

其次,设$f:X\to Y$, $g:W\to Z$是两个$S$-态射。如果对任意的代数闭域$k$,映射$f_*:X(k)\to Y(k)$和$g_*:W(k)\to Z(k)$总是单的,我们只需证明,
\[
	(f\times_S g)_*:(X\times_S W)(k)\to (Y\times_S Z)(k)
\]
也总是单的。这是因为,由前面已证明的部分,$f\times_S g$是单的,结合$1$总是单的,Proposition \ref{pro.2.13} 就给出了结论。

现在,注意到
\[
	(X\times_S W)(k)=X(k)\times_{S(k)}W(k),\quad (Y\times_S Z)(k)=Y(k)\times_{S(k)}Z(k),
\]
所以$(f\times_S g)_*$的作用就是
\[
	(f\times_S g)_*:(u,v)\mapsto (f_*(u),g_*(v)).
\]
显然是单的。
\end{proof}

下面我们考虑纤维,在此之前,我们首先有如下观察:设$y\in Y$是概形上的一个点,则$\spec (k(y))$有自然的$Y$-概形结构。实际上,首先商同态$\pi:(\oo_{Y})_y\to k(y)$给出了概形同态$\widetilde{\pi}:\spec(k(y))\to \spec ((\oo_{Y})_y)=Y_y$. 复合上典范同态$i_y:Y_y\to Y$,我们就得到了$i_y\widetilde{\pi}:\spec(k(y))\to Y$. 当然,我们也知道,给定这样一个态射,等价于给定$y$和域扩张$\id_{k(y)}:k(y)\to k(y)$. 显然,$i_y\widetilde{\pi}:\spec(k(y))\to Y$是一个单态射,因为$i_y:Y_y\to Y$是单态射,而$\pi$在交换环范畴是满态射,所以$\widetilde{\pi}$在概形范畴是单态射。我们记$j_y=i_y\widetilde{\pi}$.

\begin{pro}[纤维]\label{fiber}
设$f:X\to Y$是一个态射,而$y\in Y$是任意一点,则作为拓扑空间,纤维积$X\times_Y\spec (k(y))$同胚于$f^{-1}(y)$. 所以,我们可以将$X\times_Y\spec (k(y))$定义为概形意义上的纤维,同样记做$f^{-1}(y)$.
\end{pro}

之所以考虑$X\times_Y\spec (k(y))$而不是$X\times_Y Y_y$的理由很简单,因为$Y_y$有可能不止一个点,而$\spec (k(y))$只有一个点。同时,也因为这个原因,如果将$(\oo_Y)_y$中$\{\mm_y^n\}$取作一个拓扑基得到一个拓扑,选开理想$\mathfrak{a}$满足$\mathfrak{a}^n\to 0$(这被称为定义理想),则$\spec((\oo_Y)_y/\mathfrak{a})$其实也只有一个点,且$\spec((\oo_Y)_y/\mathfrak{a})$也同胚于$f^{-1}(y)$. 不过我们就不处理那么一般的情况了。

\begin{proof}
考虑基概形扩张$(j_y)_{(X)}:X\times_Y \spec (k(y))\to X\times_Y Y=X$,不难发现,他就是典范投影$p_X:X\times_Y \spec (k(y))\to X$. 由于$j_y$是单态射,所以对任意域$k$,它诱导了单映射$(j_y)_*:(\spec (k(y)))(k)\to Y(k)$,由Proposition \ref{pro.2.14},$j_y$在任意基概形扩张后依然是单的,故典范投影$p_X:X\times_Y \spec (k(y))\to X$是单的。

设$\spec(k(y))$的唯一点为$a$,取$z\in X\times_Y \spec (k(y))$,由于$fp_X=p_Yj_y$,所以$f(p_X(z))=j_y(p_Y(z))=j_y(a)=y$,故$\im p_X(z)\subset f^{-1}(y)$. 同时,利用Proposition \ref{pro.2.12},则$(x,a)\in X\times_Y\spec (k(y))$当且仅当$x$和$a$都位于同一点$y$上,即当且仅当$f(x)=y$. 所以典范投影$p_X:X\times_Y \spec (k(y))\to f^{-1}(y)$还是一个满射。

综上,$p_X:X\times_Y \spec (k(y))\to f^{-1}(y)$是一个双射。最后,我们只要检验同胚。由于所有形如$f^{-1}(y)\cap U$的开集构成了$f^{-1}(y)$的一组拓扑基,其中$U$是$X$中的一个仿射开集,则我们可以在$p_X^{-1}(f^{-1}(y)\cap U)=p_X^{-1}(U)=U \times_Y \spec (k(y))$上考虑问题,只要验证这是一个局部同胚,从$p_X$是一个双射,则可以得到$p_X$是一个同胚。

我们可以假设$Y$也是仿射的,事实上,选一个$y$的仿射邻域$V$,我们有
\[
	X\times_Y \spec (k(y))=(X\times_Y V)\times_V \spec (k(y))=f^{-1}(V)\times_V \spec (k(y)).
\]
此时,记$U=\spec R$而$Y=\spec S$,则
\[
	U \times_Y \spec (k(y))=\spec (R\otimes_S k(y)),
\]
其中$k(y)$是$S/\pp_y$的分式域。态射$p_X|_{U \times_Y \spec (k(y))}$此时对应于同态$\psi=1\otimes \pi:R=R\otimes_S S\to R\otimes_S k(y)$,其中$\pi:S\to k(y)$是商同态$S\to S/\pp_y$复合上含入$S/\pp_y\hookrightarrow k(y)$.

为证明局部同胚,我们下面试图利用Lemma \ref{lem:3.9}. 注意到任取$\sum_i r_i \otimes \pi(s_i)/\pi(t_i)\in R\otimes_S k(y)$,其中$r_i\in R$而$s_i$, $t_i\in S$,记$t=\prod_i t_i$以及$u_i=\prod_{j\neq i}t_j$,注意到此时
\[
	\sum_i r_i \otimes \pi(s_i)/\pi(t_i)=\sum_i r_is_i \otimes \pi(u_i)/\pi(t)=\sum_i r_is_iu_i \otimes 1/\pi(t).
\]
所以
\[
	\sum_i r_i \otimes \pi(s_i)/\pi(t_i)=\psi\left(\sum_i r_is_iu_i\right)\bigl(1\otimes \pi(t)\bigr)^{-1}.
\]
由Lemma \ref{lem:3.9},$p_X$就是局部同胚,因而也是同胚。
\end{proof}

\section{子概型}

对仿射概形$\spec R$,每一个$R$的理想$I$都可以定义一个$\spec R$上的$\oo_R$的理想层$\widetilde{I}$,这是一个拟凝聚理想层。反过来,任取$\spec R$的一个拟凝聚理想层,由Proposition \ref{niningju} 可知,它必然是$\widetilde{I}$的形式。因为这个原因,我们后面对概形也将主要关注拟凝聚理想层。但是,注意到,即使是仿射概形$\spec R$,并不是所有理想层都是拟凝聚的。

拟凝聚是局部的性质,所以利用Proposition \ref{niningju},我们给出概形$X$上模层$\calf$拟凝聚的如下定义:存在$X$的一个仿射开覆盖$\{U_\alpha\}$,使得在每一个$U_\alpha$上,都存在一个$\oo_X(U_\alpha)$-模$M_\alpha$使得$\calf|_{U_\alpha}=\widetilde{M_\alpha}$.

\begin{pro}
设$\mathscr{I}$是概形$X$的一个拟凝聚理想层,则$\oo_X/\mathscr{I}$的支集$Y$是闭集。
\end{pro}

\begin{proof}
设$Y$是$\oo_X/\mathscr{I}$的支集。注意到给定一个开覆盖$\{U_i\}$,$Y$是闭集当且仅当$U_i\cap Y$在每一个$U_i$中是闭集,所以我们只需在每一个仿射开集$U$中考察。

不妨记$U=\spec R$,则$I=\mathscr{I}(U)$是$R$的一个理想,由于$\mathscr{I}$是拟凝聚的,所以$\widetilde{I}=\mathscr{I}|_U$,于是任取$x\in U$,我们都有$\mathscr{I}_x=I_x$. 此时,$(\oo_X/\mathscr{I})_x=(\oo_X)_x/I_x=R_x/I_x$. 于是,$Y\cap U$是所有使得$R_x/I_x\neq 0$的$x\in U$的集合,即闭集$V(I)\subset \spec R$. 实际上,如果素理想$\pp_x$不包含$I$,则存在$a\in I$但$a\not\in \pp_x$,此时对$R_x/I_x=(R/I)_{\pp_x}$,因为$a$不在$\pp_x$中,它的像$\bar{a}=0$在$(R/I)_{\pp_x}$中是可逆的,所以$R_x/I_x$只可能为零。
\end{proof}

\begin{lem}
记号同上一个命题,设开集$U$, $V\subset X$满足$U\cap Y=V\cap Y$,则$\Gamma(U,\oo_X/\mathscr{I})=\Gamma(V,\oo_X/\mathscr{I})$.
\end{lem}

\begin{proof}
定义开集$W=X-Y$,不妨假设$V$是$U$的开子集。于是,我们有一个自然的限制态射
\[
	\rho^U_V:\Gamma(U,\oo_X/\mathscr{I})\to\Gamma(V,\oo_X/\mathscr{I}),
\]
首先它是单射。任取$s\in \Gamma(U,\oo_X/\mathscr{I})$,如果$s|_V=0$,则$s_x=0$对$x\in V$都成立,由于任取$x\in U-Y$都有$s_x=0$,以及$U-Y=U-Y\cap U=U-Y\cap V$,所以$s_x=0$对任意$x\in U$都成立,故$s=0$. $\rho^U_V$还是一个满射,注意,$W\cup V$是$U$的一个开覆盖,$0\in \Gamma(W,\oo_X/\mathscr{I})$和$s\in \Gamma(V,\oo_X/\mathscr{I})$在$V\cap W=V-Y$上都是零,所以可以拼出$W\cup V$上的截面,限制在$U$上就得到了$\rho^U_V$的一个原像。
\end{proof}

利用这个引理,我们可以在拓扑空间$Y$上定义一个结构层$\oo_Y$如下:任取$Y$的一个开集$V$,都存在$X$的一个开集$U$使得$V=Y\cap U$,定义$\Gamma(V,\oo_Y)=\Gamma(U,\oo_X/\mathscr{I})$,由上述引理,这个定义是良定的。此时,$(Y,\oo_Y)$也是一个概形。实际上,任取$x\in Y$,都存在仿射开集$U\subset X$包含它,而此时$U\cap Y$就是$(Y,\oo_Y)$的包含$x$的一个仿射开集。

\begin{lem}
设$\mathscr{I}$和$\mathscr{K}$是$X$的两个拟凝聚理想层,如果$\oo_X/\mathscr{I}$和$\oo_X/\mathscr{K}$的支集都是$Y$,且在$Y$上诱导了相同的概形$(Y,\oo_Y)$,则$\mathscr{I}=\mathscr{K}$.
\end{lem}

\begin{proof}
如果概形$Y$是由两个拟凝聚理想层$\mathscr{I}$和$\mathscr{K}$诱导的,则任取开集$U\subset X$,我们有
\[
	\Gamma(U,\oo_X/\mathscr{I})=\Gamma(U\cap Y,\oo_Y)=\Gamma(U,\oo_X/\mathscr{K})
\]
故$\oo_X/\mathscr{I}=\oo_X/\mathscr{K}$,也因此$\mathscr{I}=\mathscr{K}$.
\end{proof}

\begin{para}[子概形]
设$X$是一个概形,$U$是$X$的一个开集,则$U$自然地构成一个概形,此时$U$被称为$X$的一个开子概形。设$\mathscr{I}$是$X$的一个拟凝聚理想层,设$Y=\mathrm{Supp}(\oo_X/\mathscr{I})$. 我们从上面已经知道$Y$上面有一个概形结构,他被称为$X$的一个闭子概形。从上面的引理可知,概形$X$的闭子概形一一对应于它的拟凝聚理想层。一般地,如果概形$Z$是概形$X$的开子概形的闭子概形,则称$Z$是$X$的一个子概形。

在仿射概形$\spec R$,它的闭子概形一一对应于$R$的理想$\mathfrak a$,实际上,它的闭子概形就是闭子集$V(\mathfrak a)$,连同上面的层结构,可以将其等同于概形$\spec(R/\mathfrak a)$. 特别地,如果$\mathfrak a$是一个极大理想,则闭子概形$V(\mathfrak a)$是一个闭点$x$构成的集合,可以等同于概形$\spec(k(x))$.
\end{para}

\begin{para}[子概形的典范含入]
开子概形的典范含入我们已经非常清楚了,现在我们要对任意子概形进行定义。由于子概形都是开子概形的闭子概形,所以我们只要再对闭子概形定义出典范含入即可。

设$Y$是$X$的闭子概形,对应的拟凝聚理想层为$\mathscr I$,而作为拓扑空间的含入映射记作$\psi$. 我们可以定义自然的环层态射$\pi:\oo_X\to \psi_*\oo_Y$如下:任取$X$中的开集$U$,由于
\[
	\Gamma(U,\psi_*\oo_Y)=\Gamma(\psi^{-1}(U),\oo_Y)=\Gamma(U\cap Y,\oo_Y)=\Gamma(U,\oo_X/\mathscr I),
\]
所以$\psi_*\oo_Y=\oo_X/\mathscr I$. 再回忆 \ref{qsheaf},$\oo_X/\mathscr I$是典范含入$i(U):\mathscr I(U)\hookrightarrow \oo_X(U)$的余核,所以有典范态射$\pi:\oo_X\to \oo_X/\mathscr I$,此即所需。

所以,我们定义$(\psi,\pi):Y\to X$为闭子概形$Y$到$X$的典范含入。我们下面想要说明$(\psi,\pi)$是一个单态射,取$(\varphi_1,\theta_1)$, $(\varphi_2,\theta_2):Z\to Y$都是从概形$Z$到$Y$的一个态射,且$(\psi,\pi)(\varphi_1,\theta_1)=(\psi,\pi)(\varphi_1,\theta_1)$,换而言之,
\[
	\psi\varphi_1=\psi\varphi_2,\quad \psi_*(\theta_1)\pi=\psi_*(\theta_2)\pi,
\]
由于余核的典范态射都是满态射,而单射也是拓扑空间范畴的单态射,所以
\[
	\varphi_1=\varphi_2,\quad \psi_*(\theta_1)=\psi_*(\theta_2).
\]
再利用$\psi$是一个拓扑空间的含入映射(这意味着$Y$中的开集都可以写成某个$X$中开集在含入映射下的原像),$\psi_*(\theta_1)=\psi_*(\theta_2)$给出$\theta_1=\theta_2$. 所以闭子概形的典范含入态射是一个单态射。

很早以前就证明过,开子概形的典范含入也是单态射,见Corollary \ref{coro:1.21}. 由于子概形的典范含入是开子概形和闭子概形典范含入的复合,所以也是单态射。
\end{para}

\begin{para}[典范含入的局部刻画]
对于任意的子概形$(i,\pi):W\hookrightarrow X$,由于他是一个开子概形的闭子概形,所以存在一个开子概形$W'$以及一个上面的理想层$\mathscr I_W$使得$(\oo_W)_w=(\oo_{W'})_w/(\mathscr I_W)_w=(\oo_{X})_w/(\mathscr I_W)_w$,此时
\[
	\pi_w^\#:(\oo_X)_w\to (\oo_{X})_w/(\mathscr I_W)_w
\]
为商同态,显然这是一个局部同态。所以,典范含入是一个概形间的态射。
\end{para}

\begin{pro}[子概形的刻画]
$Y$是$X$的子概形,当且仅当以下两条成立:
\begin{compactenum}[~~~1.]
\item 拓扑上,$Y$是局部闭的,即可以写成$X$中一个开集与一个闭集的交。
\item 记$U=X-(\overline{Y}-Y):=X-\partial Y$,即$U$是$Y$边界\footnote{有些作者将边界定义为$\overline{Y}-Y^\circ$,这样$\partial Y$一定是一个闭集,但我们这里并不总是如此。}的补集,此时$Y$是$U$的闭子概形。
\end{compactenum}
\end{pro}

\begin{proof}
	设$Y$是一个子概形,如果他是开子概形$V$的闭子概形,那么拓扑上,我们应该有$\overline{Y}\cap V=Y$,所以$Y$是局部闭的。此外,集合上来看$U=X-(\overline{Y}-Y)$正是那些以$Y$为闭子集的开集中最大的那个,事实上,首先$U=V\cup (X-\overline{Y})$是一个开集,其次$Y=\overline{Y}\cap U$,故$Y$在$U$中是闭的。所以,如果$Y$是开子概形$V$的闭子概形,则$V\subset U$,利用典范含入,则$Y$也是开子概形$U$的闭子概形。
\end{proof}

\begin{pro}[子概形的局部刻画]\label{pro:3.3.8}
设$Y$是一个赋环空间,作为拓扑空间,$Y$是概形$X$的子空间。如果$X$中存在$Y$的一个开覆盖$\{U_\alpha\}$使得对每一个$U_\alpha$,$Y$的开子赋环空间$Y\cap U_\alpha$是$X$的开子概形$U_\alpha$的闭子概形,则$Y$是$X$的子概形。反之亦然。
\end{pro}

\begin{proof}
	因为$Y\cap U_\alpha$都是$U_\alpha$中的闭集,所以$Y$在$\{U_\alpha\}$的并中是闭的(闭的局部性质),即$Y$是一个局部闭的子空间。于是$U=X-\partial Y$是包含$Y$为闭子集的最大开集,可以$U$代替$X$,进而假设$Y$是一个闭集。
	
	由于$Y\cap U_\alpha$是$X$的开子概形$U_\alpha$的闭子概形,所以存在一个$U_\alpha$上的拟凝聚理想层$\mathscr I_\alpha$定义了闭子概形$Y\cap U_\alpha$. 我们试图将这些拟凝聚理想层粘起来得到$X$上的理想层$\mathscr I$,黏合条件是直接的,但是$\bigcup_\alpha U_\alpha$并不一定是$X$,所以还需补充那些与$Y$不交的开集上的表现。如果开集$V$与$Y$不交,定义$\mathscr I|_V=X|_V$. 这样,我们就在$X$上定义出了一个理想层$\mathscr I$,它由拟凝聚理想层拼成,而拟凝聚是局部性条件,所以$\mathscr I$也是拟凝聚的,它定义了一个$Y$的闭子概形结构。同时,因为在每个$Y$的局部与原赋环空间的层结构相同,所以两个赋环空间是相同的,即$Y$是一个闭子概形。

	反过来,假设$Y$是$X$的一个子概形,设$U$是那些以$Y$为闭子集的开集中最大的那个,则$Y$可以由$U$上的一个拟凝聚理想层$\mathscr I$定义。取$Y$的一个开覆盖$\{U_\alpha\}$,如果每个$U_\alpha$都是$U$的开子集,则此时$U_\alpha\cap Y$是$U_\alpha$中的闭子集,当然也是$Y$的开子集。作为$Y$的开子概形,$U_\alpha\cap Y$由$\mathscr I|_{U_\alpha}$定义,因此它也就是$U_\alpha$的一个闭子概形。
\end{proof}

我们下面要推广Proposition \ref{pro:1.20} 到子概形的典范含入上。Proposition \ref{pro:1.20} 告诉我们,如果$Y$是$X$的开子概形,且$Z\to X$拓扑空间的像在$Y$中,则存在分解$Z\to Y\to X$. %因此,从某种角度上来说,我们要定义概形态射的像。

\begin{para}
设$Y$是$X$的一个子概形,$f:Z\to X$是一个概形间态射,如果存在分解
\[
	f:Z\xrightarrow{g} Y\hookrightarrow X,
\]
则称$f$被概形$Y$的含入态射所遮盖。由于典范含入是单态射,所以对于给定的$f$和子概形$Y$,分解总是唯一的。
\end{para}

\begin{pro}\label{pro:3.3.11}
态射$(f,\theta):Z\to X$被含入$(i,\pi):Y\hookrightarrow X$所遮盖,当且仅当,$f(Z)\subset Y$且对任意的$z\in Z$,同态$\theta^\#_z:(\oo_{X})_{f(z)}\to (\oo_{Z})_z$总可以分解为
\[
	\theta^\#_z:(\oo_{X})_{f(z)}\xrightarrow{\pi^\#_{f(z)}} (\oo_Y)_{f(z)}\to (\oo_{Z})_z.
\]
\end{pro}

\begin{proof}
必要性显然。对于充分性,如果用$U=X-\partial Y$来代替$X$,我们可以假设$Y$是一个闭子概形,所以$Y$由一个拟凝聚理想层$\mathscr{I}$定义,局部地,典范含入诱导了
\[
	(\oo_{X})_{y}\to (\oo_Y)_{y}=(\oo_{X}/\mathscr{I})_y=(\oo_{X})_{y}/\mathscr{I}_y.
\]
令$f=(\psi,\theta)$,并设$\mathscr{J}$是$\theta^\#:\psi^*\oo_X\to \oo_Z$的核,他是$\psi^*\oo_X$的一个理想层。由假设,成立分解
\[
	\theta^\#_z:(\oo_X)_{f(z)}\to (\oo_{X})_{f(z)}/\mathscr{I}_{f(z)}\to (\oo_Z)_z,
\]
所以
\[
	(\psi^*\mathscr I)_z= \mathscr I_{f(z)}\subset \ker(\theta^\#_z)=\mathscr J_z,
\]
即$\psi^* \mathscr I\subset \mathscr J$. 进而,利用余核的泛性质,$\theta^\#$可以唯一分解为
\[
	\theta^\#:\psi^*(\oo_X)\to \psi^*(\oo_X)/\psi^*(\mathscr I)=\psi^*(\oo_X/\mathscr I)\xrightarrow{\omega} \oo_Z
\]
其中第一个态射是商的典范态射,等式是来自于左伴随函子$\psi^*$与余核可交换。最后,由于$\psi(Z)\subset Y$,所以连续映射$\psi:Z\to X$将给出连续映射$Z\to Y$,记作$\bar\psi$. 不难看到,此时$\bar \psi^*(\oo_Y)=\psi^*(\oo_X/\mathscr J)$,此外,$\omega:\bar \psi^*(\oo_Y)\to \oo_Z$显然是一个局部态射(这来自于商环的理想结构以及$\theta^\#_z$是一个局部同态),所以$(\bar\psi,\omega^\flat):Z\to Y$就是我们所需的分解,其中$\omega^\flat:\oo_Y\to \bar\psi_*\oo_Z$是$\bar\psi_*$和$\bar\psi^*$互为伴随下$\omega$的典范对应态射。
\end{proof}

\section{浸入}

以一个关于概形间单态射的常用引理开始。

\begin{lem}\label{lem:3.4.1}
	令$(\varphi,\theta):X\to Y$是赋环空间之间的态射,如果$\varphi$是单的且局部同态$\theta^\#_x:(\oo_Y)_{\varphi(x)}\to (\oo_X)_x$是满的,则$(\varphi,\theta)$是单态射。
\end{lem}

\begin{proof}
	令$(\psi_1,\eta_1)$, $(\psi_2,\eta_2):Z\to X$为两个赋环空间之间的态射,且$(\varphi,\theta)(\psi_1,\eta_1)=(\varphi,\theta)(\psi_2,\eta_2)$. 由于$\varphi$是单的,所以$\psi_1=\psi_2=\psi$. 此外,由Proposition \ref{pro:morcom},层之间的态射满足$(\eta_1)^\#_z\theta^\#_{\psi(z)}=(\eta_2)^\#_z\theta^\#_{\psi(z)}$,所以$(\eta_1)^\#_z=(\eta_2)^\#_z$对任意的$z\in Z$都成立,即$(\eta_1)^\#=(\eta_1)^\#$,故$\eta_1=\eta_2$. 
\end{proof}

上一节中定义的子概形从某种角度来说并不算一个特别好的概念,因为子概形的子概形不能简单地看作原概形的子概形。虽然,其可能确实同构于原概形的一个子概形。

类似于流形范畴,那里的子流形不仅仅要考虑一个流形,还有考虑流形之间的映射,即考虑浸入映射而不只是子流形。而我们这里类似,比起子概形,浸入是一个更容易操作的概念。

\begin{para}[浸入]
	设$X$是一个概形,概形态射$f:Y\to X$被称为一个(闭、开)浸入,如果它可以分解为$f:Y\xrightarrow{g}Z\xrightarrow{i} X$,其中$g$是一个同构,而$Z$是$X$的一个(闭、开)子概形,$i$是一个典范含入。
\end{para}

由于含入是单态射,如果浸入也是单态射。下一个引理告诉我们,浸入分解中出现的子概形和同构是由浸入本身唯一确定的。

\begin{lem}
	设$X$是一个概形,$i:Y\to X$和$i':Y'\to X$都是子概形的典范含入,如果存在分解$i':Y'\xrightarrow{g}Y\xrightarrow{i}X$,其中$g$是一个同构,则$Y'=Y$且$i=i'$.
\end{lem}

\begin{proof}
	首先,任取$x\in Y'$,则作为连续映射,我们有$x=i'(x)=i(g(x))=g(x)$,所以$g$作为连续映射只能为恒等映射。进而,只需将$X$换成包含$Y=Y'$为闭集的最大开集,我们可以进一步假设$Y$和$Y'$都是$X$的闭子概形,对应的拟凝聚理想层为$\mathscr I$和$\mathscr I'$. 我们下面证明$\mathscr I=\mathscr I'$. 这样$Y'$和$Y$是同一个闭子概形。

	任取$x\not\in |Y|=|Y'|$,有$\mathscr I'_x=(\oo_X)_x=\mathscr I_x$. 而取$x\in |Y|=|Y'|$,我们有分解$(i')^\#_x=g^\#_x i^\#_x$. 由闭子概形的构造,$i^\#_x:(\oo_X)_x\to (\oo_X)_x/\mathscr I_x$和$(i')^\#_x:(\oo_X)_x\to (\oo_X)_x/\mathscr I'_x$都是商同态。任取$s_x\in \mathscr I_x$,分解告诉我们$(i')^\#_x(s_x)=0$当且仅当$i^\#_x(s_x)=0$,所以$\mathscr I'_x = \mathscr I_x$处处成立,即$\mathscr I=\mathscr I'$. 
\end{proof}

如果浸入$f$存在两个分解$f=ig=i'g'$,其中$g$, $g'$为同构,而$i$, $i'$为含入,于是$i=i'g'g^{-1}$,由上一引理,$i=i'$. 此外,由于$ig=ig'$以及含入是一个单态射左可消,所以$g=g'$. 

\begin{lem}
设$X$是一个仿射概形,$Y$是$X$的一个闭子概形,而$Z$是$Y$的一个闭子概形,含入分别为$i$, $j$,则$ij:Z\to X$是一个闭浸入。
\end{lem}

\begin{proof}
	设$X=\spec R$,由于$Y$具有形式$V(\mathfrak a)$,同构于$\spec (R/\mathfrak a)$. 由于$Z$是$Y$的子概形,所以同构于$\spec (R/\mathfrak a)$的子概形$V(\mathfrak b)$,其中$\mathfrak b$是$R/\mathfrak a$的一个理想,也同构于$\spec ((R/\mathfrak a)/\mathfrak b)$. 此时,存在$X$的一个闭子概形$Z'=V(\mathfrak b')$,其中$\mathfrak b'$是$\mathfrak b$在$R$中的原像,使得$Z$同构于$Z'$,因为环$(R/\mathfrak a)/\mathfrak b$和$R/\mathfrak b'$显然同构。此时,注意到,$R\to (R/\mathfrak a)/\mathfrak b$可以分解为商同态$R\to R/\mathfrak b'$的复合与我们的同构的复合,所以$ij$是一个闭浸入。
\end{proof}

\begin{pro}
	设$X$是一个概形,$Y$是$X$的一个(闭、开)子概形,而$Z$是$Y$的一个(闭、开)子概形,含入分别为$i$, $j$,则$ij:Z\to X$是一个(闭、开)浸入。
\end{pro}

\begin{proof}
首先注意到,开子概形的开子概形还是开子概形,所以,两个开子概形的含入复合还是开子概形的含入。这就意味着,我们可以假设$Y$是$X$的闭子概形。并且,我们已经证明了如果$i$, $j$都是开子概形的含入,则$ij$是开浸入。

利用Proposition \ref{pro:3.3.8},我们可以将问题归结到仿射情况。找一个$X$中所有与$Y$相交非空的仿射开集构成的族$\{U_\alpha\}$,显然,$\{U_\alpha\cap Y\}$构成$Y$的一个开覆盖。由于$Z$是$Y$的一个子概形,找$Y$中包含$Z$为闭子集的最大开集$V$. 如果$U_\alpha\cap Y\subset V$中,则$U_\alpha\cap Z$是$U_\alpha\cap Y$的一个闭子概形,而$U_\alpha\cap Y$是$U_\alpha$的闭子概形。由仿射的情况,$(ij)|_{U_\alpha\cap Z}$是一个闭浸入,可以唯一分解为
\[
	(ij)|_{U_\alpha\cap Z}:U_\alpha\cap Z\xrightarrow{g_\alpha} Z'_\alpha \hookrightarrow U_\alpha,
\]
我们可以将概形$Z'_\alpha$黏成一个$|Z|$上的概形$Z'$,这是因为$Z'_\alpha$由$(ij)|_{U_\alpha\cap Z}$唯一确定。此时,上一引理给出,$Z'\cap U_\alpha=Z'_\alpha$是$U_\alpha$的闭子概形,Proposition \ref{pro:3.3.8} 给出$Z'$是$\bigcup_\alpha U_\alpha$的闭子概形,进而是$X$的子概形。类似地,那些同构则黏合成同构$g:Z\to Z'$使得分解$ij:Z\xrightarrow{g}Z'\hookrightarrow X$成立。
\end{proof}

如果通过上面命题证明中的同构$g$,将闭子概形的闭子概形典范等同于一个闭子概形,这就使得子概形的子概形是一个子概形,从而补救了原本子概形概念的不足。但下面我们不采用这种方式。

\begin{pro}
	两个(开、闭)浸入复合还是一个(开、闭)浸入。
\end{pro}

\begin{proof}
	设$p:X\to Y$,$q:Y\to Z$是两个(开、闭)浸入,分别可以分解为
	\[
		p:X\xrightarrow{g}P\hookrightarrow Y,\quad q:Y\xrightarrow{h}Q\hookrightarrow Z,
	\]
	则复合具有分解
	\[
		qp:X\xrightarrow{g}P\hookrightarrow Y\xrightarrow{h}Q\hookrightarrow Z,
	\]
	所以,我们只需研究$P\xrightarrow{i} Y\xrightarrow{h}Q$,其中$i$是一个典范含入,而$h$是一个同构。利用同构$h$,$P$同构于$Q$的一个子概形$h(P)$,使得$hi$可以分解为同构$h_P:P\to h(P)$与含入$j:h(P)\hookrightarrow Q$的复合。所以,分解有
	\[
		qp:X\xrightarrow{g}P\xrightarrow{h_P} h(P)\hookrightarrow Q\hookrightarrow Z,
	\]
	由于两个(开、闭)子概形含入复合是一个(开、闭)浸入,所以$h(P)\hookrightarrow Q\hookrightarrow Z$可以分解为$h(P)\xrightarrow{k}O\hookrightarrow Z$,所以分解
	\[
		qp:X\xrightarrow{kh_Pg}O\hookrightarrow Z
	\]
	告诉我们$qp$也是一个(开、闭)浸入。
\end{proof}

下面我们用浸入的语言重新定义子概形。

\begin{para}[子概形]
	设$i:Y\to X$和$j:Z\to X$是两个浸入,如果存在同构$g:Y\to Z$使得$i=jg$成立,则称浸入$i$和浸入$j$等价。显然,这是一个等价关系。现在,我们定义,(闭、开)子概形是(闭、开)浸入的等价类。而我们原本定义的子概形被称为典范子概形,对应典范含入。从浸入分解的唯一性,一个等价类唯一确定了一个典范子概形。

	虽然典范子概形的典范子概形不一定是一个典范子概形,但上一个命题告诉我们,子概形的子概形还是一个子概形。所以,这里定义的子概形的概念要远比典范子概形要自然。
\end{para}

\begin{para}[子概形的偏序关系]
	一个概形的子概形之间存在一个偏序关系:设$Y$和$Z$都是$X$的子概形,如果存在浸入$p:Z\to Y$,则我们定义$Z\leq Y$. 为验证这是一个合理的定义,我们需要如下引理。
\end{para}

\begin{lem}
如果$Y$和$Z$都是$X$的子概形,则含入$i:Z\hookrightarrow X$被含入$j:Y\hookrightarrow X$所遮盖当且仅当存在浸入$p:Z\to Y$.
\end{lem}

\begin{proof}
从Proposition \ref{pro:3.3.11} 显然。
\end{proof}

本节的最后,我们可以(开、闭)浸入的局部刻画。其中,开浸入的比较简单,而(闭)浸入的刻画则复杂得多,但很神奇的是,它几乎就是Lemma \ref{lem:3.4.1} 的条件。

\begin{pro}
态射$f=(\psi,\theta):Y\to X$是一个开浸入当且仅当$\psi$是$Y$到$X$的某个开集的同胚,且$\theta^\#_y:(\oo_X)_{\psi(y)}\to (\oo_Y)_y$对所有的$y\in Y$都是同构。
\end{pro}

\begin{proof}
必要性显然。充分性,条件将给出$\theta^\#:\psi^*\oo_X\to \oo_Y$是一个层同构。此外,$\psi$可以分解为$\psi=i\bar\psi$,其中$i$是开子概形$U$的典范含入而$\bar\psi:Y\to U$是一个同胚,所以$\psi^*\oo_X=\bar\psi^* i^*\oo_X=\bar\psi^*(\oo_X|_U)$,所层同构$\theta^\#:\bar\psi^*(\oo_X|_U)\to \oo_Y$将给出层同构$\oo_X|_U\to \bar\psi_*\oo_Y$,连同同胚$\bar\psi$这就给出了一个$Y$到$\oo_X|_U$的概形同构。
\end{proof}

\begin{pro}\label{pro:3.4.11}
态射$f=(\psi,\theta):Y\to X$是一个浸入(闭浸入)当且仅当$\psi$是$Y$到$X$的某个局部闭集(闭集)的同胚,且$\theta^\#_y:(\oo_X)_{\psi(y)}\to (\oo_Y)_y$对所有的$y\in Y$都是满同态。
\end{pro}

\begin{proof}
由于子概形都是开子概形的闭子概形,从开浸入的命题,我们只需证明闭浸入的情况。对于闭浸入的情况,必要性是显然的。为证明充分性,我们首先假设命题对$X$是仿射概形而$Z=\psi(Y)$在$X$中是闭的情况是对的。

那么,对于一般情况,设$U$是$X$的一个仿射开集,使得$U\cap \psi(Y)$在$U$中是非空闭集,将$f$限制到概形$\psi^{-1}(U)$上,从特殊情况可知$f$的限制$f_U:\psi^{-1}(U)\to U$是一个闭浸入。所以,可以分解为$f_U=j_Ug_U$,其中$g_U:\psi^{-1}(U)\to Z_U$是一个同构,而$j_U:Z_U\hookrightarrow U$是一个典范含入。现在,我们想要把局部地拼成整体的,所以需要检查黏合条件。取另一个仿射开集$V$,不失一般性,可以假设$V\subset U$,则$f_V$是一个闭浸入,存在分解$f_V=j_Vg_V$. 同时,由于$f_V$是$f_U$在$\psi^{-1}(V)$上的限制,设$g_U(\psi^{-1}(V))=Z'_V$,则
\[
	f_V=j_U|_{Z'_V}g_U|_{\psi^{-1}(V)},
\]
这里$Z'_V$当然是$Z_U$的一个开集,是闭子概形的开子概形,所以$j_U|_{Z'_V}$是一个浸入,可以分解为$j_U|_{Z'_V}=ih$,所以
\[
	f_V=ihg_U|_{\psi^{-1}(V)},
\]
其中$i$是一个子概形的典范含入,从浸入分解的唯一性,$i=j_V$,再利用一次浸入分解的唯一性,$j_V$或子概形$Z_V$由$j_U|_{Z'_V}$唯一确定。类似地,$g_V$也由$j_U|_{Z'_V}$唯一确定,这些就是黏合条件。此时,将$Z_U$以及$g_U$黏合起来,我们有闭子概形$j:Z\hookrightarrow X$以及同构$g:Y\to Z$使得分解$f=jg$成立,所以$f$是一个闭浸入。

最后,回到特殊情况的证明。首先,$X$是仿射概形而$Z=\psi(Y)$在$X$中是闭的。注意$\psi$是到一个$Y$到$X$中闭集$Z$的同胚,所以$(\psi_*^{\oo_Y})_{\psi(x)}:(\psi_*\oo_Y)_{\psi(y)}\to (\oo_Y)_y$对任意的$y\in Y$都是同构,但对$x\not\in \bar Z=Z$,有$(\psi_*\oo_Y)_{x}=\{0\}$. 实际上,取$x$的一个邻域$U$域$Z$不交,则$\psi_*\oo_Y(U)=\oo_Y(\psi^{-1}(U))=\oo_Y(\varnothing)=\{0\}$. 所以,$\psi_* \oo_Y$的支集就是$Z$,我们下面证明$\psi_*\oo_Y$是一个拟凝聚$\oo_X$-模层,于是$Z$上就有一个$\psi_*\oo_Y$定义的闭子概形结构,而他通过$f$与$Y$同构,这就完成了证明。

取$x\in Z$,记$y=\psi^{-1}(x)$. 设$V$是$y$在$Y$中的一个仿射开邻域,$\psi(V)$在$Z$中也是开的,故存在$X$的开集$U$使得$\psi(V)=U\cap Z$,且使得$(\psi_*\oo_Y)|_U=((\psi|_V)_*(\oo_Y|_V))|_U$. 现在$f$在$(V,\oo_Y|_V)$上的限制是一个仿射概形到仿射概形$X$的态射,必然具有形式$({}^a\varphi,\tilde\varphi)$,其中$\varphi:\Gamma(X,\oo_X)\to \Gamma(V,\oo_Y)$是一个环同态,所以$(\psi|_V)_*(\oo_Y|_V)$是一个拟凝聚$\oo_X$-模层。\notprove
\end{proof}

\begin{coro}\label{coro:3.4.12}
设$f:X\to Y$是一个概形态射,$\{V_\lambda\}$是$Y$的一个开覆盖,则$f$是一个闭浸入的充分必要条件是,$f$在每一个开子概形$f^{-1}(V_\lambda)$上的限制都是到$V_\lambda$中的闭浸入。
\end{coro}

\begin{proof}
层态射的局部条件此时是显然的,再注意到,闭集也是局部性质,所以命题直接得证。
\end{proof}

\begin{coro}
设$f:X\to Y$是一个概形态射,$\{V_\lambda\}$是$f(X)$的一个开覆盖,则$f$是一个(开)浸入的充分必要条件是,$f$在每一个开子概形$f^{-1}(V_\lambda)$上的限制都是到$V_\lambda$中的(开)浸入。
\end{coro}

\begin{proof}
层态射的局部条件依然是显然的,拓扑上,从局部到整体,我们可以得到$f(X)$在$\bigcup_\lambda V_\lambda$中是局部闭的(开的),进而在$Y$中也是局部闭的(开的)。
\end{proof}

\begin{pro}
设$\alpha:X'\to X$和$\beta:Y'\to Y$都是$S$-概形态射,如果它们都是(开、闭)浸入,则$\alpha\times_S\beta$也是。特别地,如果$\alpha$, $\beta$都是子概形的典范含入,则$\alpha\times_S \beta$将拓扑空间$X'\times_S Y'$等同于$\pi_X^{-1}(X')\cap \pi_Y^{-1}(Y')$,这里$\pi_X$, $\pi_Y$是$X\times_S Y$的典范投影。
\end{pro}

由于恒同总是一个开、闭浸入,则上述命题告诉我们,如果$S$-概形态射$f:X\to Y$是一个(开、闭)浸入,则对任何基扩展$S'\to S$,$f_{(S')}$也是(开、闭)浸入。

\begin{proof}
从浸入的定义,我们可以考虑$\alpha$和$\beta$都是子概形典范含入的情况。此时,如果都是开子概形,则Corollary \ref{coro:3.2.8} 立刻给出了结论。因为子概形都是开子概形的闭子概形,所以我们只需证明$X'$和$Y'$都是闭子概形的情况。

利用浸入是单态射,浸入的局部检验,经过一些简单的检查,我们可以直接将问题约化到$X$, $Y$, $S$都是仿射概形的情况。设$X$, $Y$, $S$的环分别为$B$, $C$, $A$,由于$X'$和$Y'$都是闭子概形,设它们分别对应于$B$和$C$的商代数$B'$和$C'$,此时$\alpha=\widetilde{\rho}$, $\beta=\widetilde{\sigma}$,其中$\rho$, $\sigma$都是典范商同态。于是,$X\times_S Y=\spec(B\otimes_A C)$以及$X'\times_S Y'=\spec(B'\otimes_A C')$,而
\[
	\alpha\times_S\beta=\widetilde{\rho\otimes_A\sigma}.
\]
由于$\rho\otimes_A\sigma$是满的,所以$\alpha\times_S\beta$是一个闭浸入。此外,$\ker(\rho\otimes \sigma)=u(\ker \rho)+v(\ker \sigma)$,这里$u$, $v$分别是$B$, $C$到$B\otimes C$的典范同态,而对应的$\widetilde{u}$, $\widetilde{v}$就是典范投影。因此,我们有拓扑空间的等式
\[
	V(\ker(\rho\otimes \sigma))=V(u(\ker\rho))\cap V(v(\ker\sigma))=\widetilde{u}^{-1}\bigl(V(\ker\rho)\bigr)\cap \widetilde{v}^{-1}\bigl(V(\ker \sigma)\bigr),
\]
第一项是对应于闭浸入$\alpha\times_S \beta$的闭子概形,
此即所证。
\end{proof}

\begin{pro}[子概形的原像]
设$f:X\to Y$是一个概形间态射,$i:Y'\hookrightarrow Y$的(闭、开)子概形,则投影$p_X:X\times_Y Y'\to X$是一个(闭、开)浸入,它唯一确定了$X$的一个(闭、开)典范子概形$X'$,$X'$作为拓扑空间同胚于$f^{-1}(Y')$.
\end{pro}

粗略地讲,子概形的原像也是子概形。特别地,考虑闭点$y\in Y$,它本身可以看成一个闭子概形$\spec (k(y))$,它关于$f$的原像正是我们在Proposition \ref{fiber} 引入的纤维$f^{-1}(y)$,它可以看成$X$的一个闭子概形。

\begin{proof}
如果把$X$和$X\times_Y Y$等同,不难看到$p_X=1_X\times_Y i$,则上一个命题直接给出了结论。
\end{proof}

\begin{pro}
	设$f:X\to Y$是一个概形间态射,$Y'$是$Y$的一个闭子概形,对应的拟凝聚理想层为$\mathscr I$,则$X$的闭子概形$f^{-1}(Y')$由拟凝聚理想层$(f^*\mathscr I)\oo_X$定义。
\end{pro}

\begin{proof}
	对$X$和$Y$都是局部的,所以考虑仿射的情况即可。此时,利用上面的命题,纤维积在代数范畴变成张量积,所以,注意到,如果$A$是一个$B$-代数,且$I$是$B$的一个理想,则$A\otimes_B (B/I)=A/(IA)$.
\end{proof}

\begin{para}[闭像与闭包]
	设$f:X\to Y$是概形态射,$f$的闭像是$Y$的一个闭子概形$Z$,使得如果$f$能被$Y$的闭子概形$Z'\hookrightarrow Y$所遮盖,则$Z$是$Z'$的闭子概形。换句话说,能遮盖$f$的最小闭子概形。特别地,考虑子概形的典范含入$i:U\hookrightarrow Y$,则$i$的闭像被称为$U$的闭包。
\end{para}

闭像一般来说不一定存在,但下面命题告诉我们,在一定情况下,闭像是存在的。

\begin{pro}
	设$f:X\to Y$是概形态射,如果$f_*\oo_X$是拟凝聚$\oo_Y$-模层,则$f$的闭像存在。
\end{pro}

\section{约态概形}

\begin{pro}
	设$X$是一个概形,$\mathcal A$是一个拟凝聚$\oo_X$-代数层,则存在唯一一个拟凝聚$\oo_X$-模层$\mathcal N$使得任取$x\in X$,$\mathcal N_x$都是$\mathcal A_x$的幂零根。如果$X$是仿射概形,则$\mathcal A=\widetilde{A}$,其中$A$是一个$\Gamma(X)$-代数,此时$\mathcal N=\widetilde{N}$,其中$N$是$A$的幂零根。
\end{pro}

\begin{proof}
	显然是局部命题,所以我们只需证明仿射概形情形即可。已知$\widetilde{N}$是一个拟凝聚$\oo_X$-模层,再从显然的唯一性,只需检验$\widetilde{N}_x=N_x$就是$A_x$的幂零根即可,即验证局部化和取幂零根可交换。记$A_x$的幂零根为$\sqrt{0}$,则显然$N_x\subset \sqrt{0}$. 此外,任取$r/s\in \sqrt{0}$,存在一个$n$使得$r^n/s^n=0\in A_x$,换言之,存在一个$t\not\in \pp_x$使得$tr^n=0\in A$,于是$(tr)^n=0$继而$r/s=tr/(ts)\in N_x$,这就给出反方向的包含。
\end{proof}

\begin{para}[幂零根层]
	特别地,如果$X$是概形,则$\oo_X$是一个拟凝聚$\oo_X$-代数层,所以上面的命题告诉我们有一个拟凝聚理想层$\mathcal N$,他被称为$X$的幂零根层,我们将其记作$\mathcal N_X$.
\end{para}

\begin{para}[约态概形]
	设$X$是一个概形,如果任取$x$,$(\oo_X)_x$都是约态环,即幂零根为零,则称$X$是一个约态概形。
\end{para}

显然,约态概形是一个局部概念,所以我们只需考察仿射概形的情况。设$X=\spec R$,则所有的$R_x$都是约态的当且仅当$R$是约态的,所以仿射概形约态当且仅当其坐标环是约态环。

对每一个概形$X$,幂零根层$\mathcal N_X$是拟凝聚理想层,对应一个$X$的闭子概形$X_{\text{red}}$,从定义,不难看到$X_{\text{red}}$是一个约态概形,他被称为概形$X$的伴随约态概形。此外,$X_{\text{red}}$的底空间还是$X$,这是因为非零环模掉幂零根后不为零。再从幂零根是所有素理想的交这个刻画,不难看到$X_{\text{red}}$是所有以$X$为底空间的$X$的子概形中最小的那个。

\begin{para}[态射的约态化、函子性]
	设$f=(\psi,\theta):X\to Y$是一个概形态射,则$\theta^\#_x:(\oo_Y)_{f(x)}\to (\oo_X)_x$是一个局部同态。此外,由于环同态将幂零元映成幂零元,所以$\theta^\#_x$自然诱导了一个环同态$\omega_x:(\oo_Y)_{f(x)}/(\mathcal N_Y)_{f(x)}\to (\oo_X)_{x}/(\mathcal N_X)_{x}$,不难检验这还是一个局部同态,继而给出态射
	\[
		\omega:\psi^*(\oo_Y/\mathcal N_Y)\to \oo_X/\mathcal N_X,
	\]
	此时$(\psi,\omega^\flat):X_{\text{red}}\to Y_{\text{red}}$就是一个概形态射。我们将其记作$f_{\text{red}}$,称为$f$的伴随约态态射。显然,$(gf)_{\text{red}}=g_{\text{red}}f_{\text{red}}$,所以$X\mapsto X_{\text{red}}$连同$f\mapsto f_{\text{red}}$构成了一个协变函子。
\end{para}
