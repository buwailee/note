% !TEX root = main.tex

\chapter{概形}

\section{仿射概形与伴随层}

\para 设$R$是一个交换环,所有$R$的素理想构成一个集合,记作$\spec(R)$,称为$R$的素谱。如果$R$是零环,则$R$没有素理想,所以$\spec(R)=\varnothing$,我们一般对此没什么兴趣。所以,我们默认$R$不是零环,但零环确实是在交换环范畴中的,就像集合范畴中有空集一样。

我们一般以$\pp$来表示素理想,但后面会看到$\spec(R)$有一个自然的拓扑结构,所以里面的点按照拓扑空间的习惯一般是记作拉丁字母$x$或者$p$等来表示这是一个点。所以下面我们以$\pp_x$来表示$x\in \spec(R)$对应的素理想。类似的,以$R_x$来表示局部化$R_{\pp_x}$,一个环局部化后是一个局部环,只有一个极大理想,以$\mm_x$来标记$R_x$的极大理想,同时记域$k(x):=R_x/\mm_x$.

\para 现在设$f\in R$是环中的任意元素,则经过如下自然的映射
\[
	f\mapsto f(x):R\to R_x\to k(x)
\]
得到的$f(x)$成为环元$f$在$x$处的值,其中第一个映射将$f$映作$f/1$,而第二个映射为商映射。显然$f(x)=0$等价于$f\in \pp_x$,或者$(f)\subset \pp_x$.

可以看到,$f(x)$并不是真正意义上的一个函数,因为在每一点处的$k(x)$都是不同的。但是,每一个域都有一个加法零元,所以谈论$f(x)$的零点还是有意义的。

我们可以在层上进行类似的操作。比方说$M$是一个光滑流形,他上面有一个光滑函数层$\mathcal{O}$,在一点$x$处,有局部环$\mathcal{O}_x$,他的极大理想$\mm_x$就是那些在$x$附近为零的光滑函数构成的等价类,此时任意$\mathcal{O}$的截面就是一个光滑函数$f$,而作为函数值的$f(x)$和通过$\mathcal{O}(M)\to \mathcal{O}_x\to \mathcal{O}_x/\mm_x\cong \rr$得到的$f(x)$是相同的。

所以,想要$\spec(R)$获得一个层结构,首先$R$中的元素作为截面要表现得像一个连续函数,这将给出$\spec(R)$的拓扑结构,然后局部环$R_x$应该表现得和$\mathcal{O}_x$类似,这将给出$\spec(R)$的一个层结构。

\begin{para}[素谱的拓扑结构]
定义$D(f):=\{x\in \spec(R)\,:\, f(x)\neq 0\}$,我们下面的目标是让$f(x)$尽可能看上去像一个连续函数,我们要证明$D(f)$满足开集公理。等价地,我们要验证$V(f):=\{x\in \spec(R)\,:\, f(x)=0\}=\{\pp\in \spec(R)\,:\, (f)\subset \pp\}$满足闭集公理,或者更一般地,我们可以验证$V(E):=\{\pp\in \spec(R)\,:\, E\subset \pp\}$满足闭集公理,其中$E$是$R$的任意子集。
\end{para}

\begin{pro}\label{pro:3.4}
$V(E)$满足闭集公理,所以$\spec(R)$以$V(E)$为闭集是一个拓扑空间,且如下命题成立:
\begin{compactenum}
\item $D(f)$是开集,且它们构成了$\spec(R)$的一个拓扑基。
\item 环的素谱满足$T_0$分离公理。
\item 环的素谱是拟紧的。
\end{compactenum}
\end{pro}

\begin{proof}
显然的是$V(1)=\varnothing$以及$V(0)=\spec(R)$,剩下的我们需要检验$\bigcap_{i\in I}V(E_i)=V(\bigcup_{i\in I}E_i)$. 设$\pp \in \bigcap_{i\in I}V(E_i)$,则$\pp\in V(E_i)$对$i\in I$都成立,这等价于$E_i\subset \pp$对$i\in I$都成立,而这又等价于$\bigcup_{i\in I}E_i \subset \pp$,因此得证。

关于这个拓扑空间的命题,我们逐个证明。首先,设$U$是任意的开集,则其写作$U=\spec(R)-V(E)$的形式,而
\[
\spec(R)-V(E)=\spec(R)-V\bigl(\bigcup_{f\in E} f\bigr)=\spec(R)-\bigcap_{f\in E}V(f)=\bigcup_{f\in E}D(f),
\]
所以$U=\bigcup_{f\in E}D(f)$.

接着,实际上,设$\pp_x$和$\pp_y$分别是两个素理想,则或者有$\pp_x\not\subset \pp_y$或者有$\pp_y\not\subset \pp_x$,因此有$y\not\in \overline{\{x\}}$或者$x\not\in \overline{\{y\}}$. 用开集描述就是,$y\in \spec(A)-\overline{\{x\}}$或者$x\in \spec(A)-\overline{\{y\}}$,这就使得$T_0$分离公理成立。

结合第一点,设$\{D(f_\alpha)\,:\, \alpha\in A\}$是$\spec R$的一个开覆盖,这里$A$是一个指标集。我们需要证明
存在一个有限子集$B$,使得$\{D(f_\alpha)\,:\, \alpha\in B\}$也是一个开覆盖。考虑$\bigcap_{\alpha\in A}V(f_\alpha)=V(\{f_\alpha\}_{\alpha\in A})=\varnothing$,于是$\{f_\alpha\}_{\alpha\in A}$生成了整个环$R$,特别地,生成了$1$,于是存在有限个$f$可以线性组合出$1$,即$\sum_{\alpha\in B} r_\alpha f_\alpha=1$,这样我们也就找到了这个有限子集。
\end{proof}

作为推论,如果$\overline{\{x\}}=\overline{\{y\}}$,则$x=y$.

\begin{pro}
关于$D(f)$,我们也有如下两个简单的性质:
\begin{compactenum}[~~~1.]
\item $D(fg)=D(f)\cap D(g)$.
\item 如果$a\in R$是一个可逆元,则$D(af)=D(f)$.
\end{compactenum}
\end{pro}

第二点在分式环的时候很常用,比如在$\spec R_f$中,$f/1$和$1/f$都是可逆元,故$D(g/f^n)=D(g/1)$.

\begin{proof}
第一点,任取$x\in D(fg)$,则$(fg)(x)=f(x)g(x)\neq 0\in k(x)$. 由于$k(x)$是一个域,所以$x\in D(fg)$当且仅当$f(x)\neq 0$且$g(x)\neq 0$,即当且仅当$x\in D(f)\cap D(g)$.

第二点,任取$x\in D(af)$即$(af)(x)=a(x)f(x)\neq 0$,由于$a\in R$可逆,所以$a(x)\neq 0$对任意的$x$都成立,故$(af)(x)\neq 0$等价于$f(x)\neq 0$等价于$x\in D(f)$.
\end{proof}

记$\pp(Y)=\{f\in A\,:\, f(Y)=0\}$,即在$Y$上消失的所有环元构成的理想,利用这个记号,显然的是$\pp(\{x\})=\pp_x$. 把$f(Y)=0$的意义详细写出来,则$\pp(Y)=\bigcap_{y\in Y}\pp_y$. 于是,$\pp(V(E))=\sqrt{(E)}$,这来自于$\sqrt{(E)}$是所有包含$(E)$的素理想的交。这是传统的Hilbert零点定理的对应。

\begin{pro}如下命题成立:
\begin{compactenum}[~~~1.]
\item $V(E)=V((E))=V(\sqrt{(E)})$;
\item 若$E\subset F$,则$V(F)\subset V(E)$;
\item $\bigcap_{i\in I}V(E_i)=V(\bigcup_{i\in I}E_i)$;
\item $V(EF)=V(E)\cup V(F)$;
\item 若$X\subset Y$,则$\pp(Y) \subset \pp(X)$;
\item $\pp(V(E))=\sqrt{(E)}$;
\item $V(\pp(Y))=\overline{Y}$.
\end{compactenum}
\end{pro}

于是,由1, 2, 5, 6和7,可知$\pp$与$V$构成了一对Galois联络,于是它们的像构成一一对应。即$\spec R$中的闭集一一对应于$R$中的根式理想。

\begin{proof}
% 对$R$的任意子集$E$,显然有$V(E)=V((E))$,其中$(E)=ER$是$E$生成的理想,由于$\sqrt{(E)}$是所有包含$(E)$的素理想的交,所以有$V(E)=V((E))=V(\sqrt{(E)})$. 再比如,如果$E\subset F$,则$V(F)\subset V(E)$.

除了最后一点其他都是容易的。由于$\pp(Y)=\bigcap_{y\in Y}\pp_y$,所以$\pp(Y) \subset \pp_y$,其中$y\in Y$,所以$Y\subset V(\pp(Y))$. 剩下我们要证明,$V(\pp(Y))$就是包含$Y$的最小闭集。考虑$Y\subset V(E)$,对于任意的$f\in E$以及$y\in Y$,都有$f(y)=0$,从而$E\subset \pp(Y)$,进而$V(\pp(Y))\subset V(E)$. 
\end{proof}

设$x\in \spec(A)$,$\overline{\{x\}}=V(\pp(\{x\}))=V(\pp_x)$,则$\{x\}$是闭集的充分必要条件就是$V(\pp_x)=\{x\}=\{\pp_x\}$,或者说$\pp_x$是一个极大理想。所以极大理想对应着素谱中的闭点,类似于坐标环的极大理想对应着仿射簇中的点一样。

\begin{para}[素谱之间的映射]
设$\varphi:R_1\to R_2$是一个环同态,由于素理想在$\varphi$的原像也是素理想,所以$\varphi$通过$\spec(R_2)\ni x\mapsto \varphi^{-1}(\pp_x)\in \spec(R_1)$诱导了素谱之间的映射$^a\varphi:\spec(R_2)\to \spec(R_1)$. 这是一个连续映射,这个我们只要检验$^a\varphi^{-1}(V(f))$是一个闭集就可以了,事实上,更一般的命题是:$^a\varphi^{-1}(V(E))=V(\varphi(E))$,老实按照定义检验即可。有时也把$^a\varphi$记作$\spec \varphi$. 由于$^a(\varphi\circ \psi)={^a\psi}\circ{^a\varphi}$,所以$\spec$是环范畴到拓扑空间范畴的反变函子。

局部来看,我们用$\varphi^x$来表示$\varphi$在商环$R_1/\varphi^{-1}(\pp_x)\to R_2/\pp_x$诱导出的同态,以及用同样的符号来表示诱导的域同态$\varphi^x:k({^a}\varphi(x))\to k(x)$. 根据定义,对任意的$f\in R_1$,以及$x\in \spec(R_2)$,有等式
\[
	\varphi^x(f({^a}\varphi(x)))=\varphi^x\left(f \text{ mod } \varphi^{-1}(\pp_x)\right)=\varphi(f)\text{ mod } \pp_x=\varphi(f)(x).
\]
\end{para}

\begin{pro}\label{pro:3.8}
如下命题成立:
\begin{compactenum}[~~~1.]
\item $^a\varphi^{-1}(V(E))=V(\varphi(E))$;
\item $^a\varphi^{-1}(D(f))=D(\varphi(f))$;
\item $\varphi^x(f({^a}\varphi(x)))=\varphi(f)(x)$;
\item $\overline{{^a}\varphi(V(E))}=V(\varphi^{-1}(E))$.
\end{compactenum}
\end{pro}

\begin{proof}
除了最后一点,我们都是清楚的。由于根式理想的原象还是根式理想,所以可以干脆假设$E=\mathfrak{a}$是一个根式理想。设$Y=V(\mathfrak{a})$以及$\mathfrak{a}'=\pp({^a}\varphi(Y))$,则$\overline{{^a}\varphi(Y)}=V(\mathfrak{a}')$. 我们先证明$V(\varphi^{-1}(\mathfrak{a}))\subset V(\mathfrak{a}')=\overline{{^a}\varphi(Y)}$. 而这又等价于去证明$\mathfrak{a}'\subset \varphi^{-1}(\mathfrak{a})$. 由定义,$f'\in \mathfrak{a}'$等价于$f'({^a\varphi}(Y))=0$,因此对任意的$y\in Y$成立$\varphi(f')(y)=\varphi^y(f'(y))=0$,所以$\varphi(f')\in \pp(Y)=\pp(V(\mathfrak{a}))=\sqrt{\mathfrak{a}}=\mathfrak{a}$. 于是$f'\in \varphi^{-1}(\mathfrak{a})$.

再证明反过来的包含$\overline{{^a}\varphi(V(E))}\subset V(\varphi^{-1}(E))$. 设$\varphi^{-1}(\pp)\in {^a}\varphi(V(E))$,其中$\pp\in V(E)$或者$E\subset \pp$,则$\varphi^{-1}(E)\subset \varphi^{-1}(\pp)$,于是$\varphi^{-1}(\pp)\in V(\varphi^{-1}(E))$或者说${^a}\varphi(V(E))\subset V(\varphi^{-1}(E))$. 由于$V(\varphi^{-1}(E))$是一个闭集,所以$\overline{{^a}\varphi(V(E))}\subset V(\varphi^{-1}(E))$.
\end{proof}

\begin{lem}\label{lem:3.9}
设$\varphi:A'\to A$是一个环同态,对任意的$a\in A$,总存在$a'\in A'$和$A$中的可逆元$h$使得$\varphi(a')=ha$,那么$^a\varphi$是$\spec(A)$到$\im (^a\varphi)$的同胚。
\end{lem}

\begin{proof} 先对每一个$A$的子集$E$,我们证明总存在$A'$中的一个子集$E'$满足$V(E)=V(\varphi(E'))$即可。固定$E$,对每一个$a\in E$,取一个$a'\in A$和一个可逆元$h$使得$\varphi(a')=ha$,注意到此时$D(\varphi(a'))=D(a)$或者$V(\varphi(a'))=V(a)$,所以这些$a'$组成的集合$E'$就满足我们的需求。

首先证明$^a\varphi$是一个单射。假设$^a\varphi(x)={}^a\varphi(y)$但$x\neq y$. 由$T_0$分离性,可以假设$y\not\in \overline{\{x\}}$,选择$V(E)=\overline{\{x\}}$,于是存在$E'$使得$x\in V(E)=V(\varphi(E'))$,所以${}^a\varphi(y)={}^a\varphi(x)\in V(E')$. 因为$y\not\in \overline{\{x\}}=V(E)=V(\varphi(E'))={}^a\varphi^{-1}(V(E'))$,所以${}^a\varphi(y)\not\in V(E')$,矛盾。因此$^a\varphi$是一个单射。设$\psi$是它在$\im (^a\varphi)$上的函数逆,下面我们证明这是一个连续映射。

任取$E$,存在$E'$使得$V(E)=V(\varphi(E'))={}^a\varphi^{-1}(V(E'))$,于是
\[
	\psi^{-1}(V(E))={}^a\varphi(V(E))={}^a\varphi({}^a\varphi^{-1}(V(E')))=V(E'),
\]
任意闭集关于$\psi$的原像是闭集,故$\psi$连续。进而$^a\varphi$是$\spec(A)$到$\im (^a\varphi)$的同胚。
\end{proof}

如果$\varphi$是满射,按照上一个命题,$^a\varphi$是$\spec(A)$到$\im (^a\varphi)$的同胚。设$\mathfrak{a}$是$R$的一个理想,则商映射$\varphi:R\to R/\mathfrak{a}$诱导的连续映射$^a\varphi:\spec(R/\mathfrak{a})\to \spec(R)$给出了$\spec(R/\mathfrak{a})$与$\spec(R)$的闭子集$V(\mathfrak{a})$之间的同胚。特别地,$\spec(R/\sqrt{0})$和$\spec(R)$同胚。

设$S$是$A$的一个乘性子集,我们由典范的同态$i:A\to S^{-1}A$,且,对任意的$a/s\in S^{-1}A$,我们都有$a'=a$使得$i(a')=a/1=s(a/s)$,所以引理条件满足。此时${}^ai:\spec(S^{-1}A)\to \im ({}^ai)$是一个同胚,由$S^{-1}A$的素理想的结构,它的素理想一一对应于$A$中与$S$不相交的素理想,所以$\im ({}^ai)$即$\spec A$中满足$\pp_x\cap S=\varnothing$的$x$构成的子空间。

特别地,任取$f\in R$,考虑$\spec(R_f)$. 它应与$\spec R$中所有不包含$f$的素理想构成的集合同胚,也即$D(f)$. 

\begin{pro}
$\spec R$不可约当且仅当$\sqrt{0}$是一个素理想。$\spec(R)$的不可约闭子集一一对应着$R$的素理想,特别地,$\spec(R)$的不可约分支对应着$R$的极小素理想。
\end{pro}

\begin{proof}
由于$\spec(R)\cong \spec(R/\sqrt{0})$,所以如果$\sqrt{0}$是素理想,可以假设$R$是一个整环。现在假设$\spec R$是可约的,则存在两个真闭子集$V_1$和$V_2$使得$\spec R=V_1\cup V_2$. 于是$\pp(V_1)\cap \pp(V_2)=\pp(V_1\cup V_2)=\pp (\spec R)=0$,但由于$V_1$, $V_2$是真闭子集,所以$\pp(V_1)$和$\pp(V_2)$都不为零,因此$R$不是整环。

反过来,由于$\spec(R/\sqrt{0})$和$\spec R$的同胚,可以假设$\sqrt{0}=0$. 如果$\spec R$不是整环,则存在非零的$f$, $g$使得$fg=0$,于是$\spec R=V(0)=V(fg)=V(f)\cup V(g)$. 但由于$f$, $g$非零,且$\sqrt{0}=0$,所以$V(f)$与$V(g)$是$\spec R$的真闭子集。进而$\spec R$可约。
\end{proof}

设$V(\mathfrak{a})$是$\spec R$的一个不可约闭子集,由于$V(\mathfrak{a})$与$\spec (R/\mathfrak{a})$同胚,所以$\spec (R/\mathfrak{a})$是不可约的。因此,$\sqrt{\mathfrak{a}}$是极小的素理想,故是$V(\mathfrak{a})$的一个一般点。由于$\spec R$是$T_0$的,所以这个一般点也是唯一的一般点。于是$\spec R$中的任意不可约闭子集都具有唯一的一般点。或者说,$\spec R$中所有的不可约闭子集都具有$\overline{\{x\}}$的形式,其一般点是$x$. 

反过来,给定$x\in\spec R$,$\overline{\{x\}}$是不可约闭子集,所以$x\mapsto \overline{\{x\}}$是$\spec R$上的点到$\spec R$中不可约闭子集之间的一一对应。

\begin{para}
由于在拓扑空间的一组基上给出层的公理,就可以得到一个层,所以我们通过$D(f)\mapsto R_f$来定义$\spec(R)$的一个预层结构$\mathcal{O}_R$,下面会证明这是一个环层,称为结构层或者说伴随层。类似地,如果有一个$R$-模$M$,我们可以通过$D(f)\mapsto M_f$来定义出一个$\mathcal{O}_R$-模层$\widetilde{M}$. 特别地,$\mathcal{O}_R(\spec(R))=R_1=R$. 素谱连同其上的结构层构成一个赋环空间,我们依旧记做$\spec(R)$,如有必要,他的结构层写作$\mathcal{O}_R$.

由于$x\in \spec(R)$附近的邻域$D(f)$关于包含于是共尾(cofinal)的,所以茎$(\widetilde{M})_x=M_x$,故而这正符合我们上面对于层结构的需求。所以这是一个局部赋环空间。
\end{para}

为了描述层结构,我们需要限制映射,这直接来自于下面的引理。

\begin{lem}
设$R$是一个环,而$M$是一个$R$-模。以下命题成立:
\begin{compactenum}[~~~1.]
\item 设$S$是$R$的一个乘性子集,而$S'$是$S$中元素的所有因子构成的新的乘性子集,则$(S')^{-1}M$和$S^{-1}M$典范同构。所以一般我们将其看成等同的,即$(S')^{-1}M=S^{-1}M$. 
\item 存在典范同构$(M_f)_g\cong M_{fg}$,我们也将其看成是等同的。
\item 如果$D(g)\subset D(f)$,则$M_{fg}=M_{g}$.
\item 设$n$是一个正整数,则$D(f^n)=D(f)$且$M_{f}=M_{f^n}$,其中$n$是一个正整数。
%\item 对一个有限环元族$\{f_\alpha\}$和一个给定的正整数$n$,$\{f_\alpha^n\}$生成单位理想当且仅当$\{f_\alpha\}$生成单位理想。
\end{compactenum}
\end{lem}

\begin{proof}
一条一条来证明:
\begin{compactenum}[~~~1.]
\item 注意到$S\subset S'$,所以存在一个典范同态$i:S^{-1}M\to (S')^{-1}M$,他将$m/s$变成$m/s$. 现在设$i(m/s)=0$,因此存在一个$s'\in S'$使得$s'm=0$. 由于$s'$是$S$中某个元$t$的因子,而因此存在$t\in S$使得$tm=0$,或者说$m/s=0$. 所以$i$是一个单同态。反之,任取$m/s'\in S^{-1}M$,因为$s'$是某个$t\in S$的因子,写作$t=s'r$. 所以$m/s'=rm/t=i(rm/t)$. 因此$i$是一个满射。

\item 由于$(M_f)_g=S^{-1}M_f$,其中$S=\{g/1$, $g^2/1$, $\dots\}$. 我们通过$(m/f^i)/(g^j/1)\mapsto f^j g^i m/(fg)^{i+j}$定义$i:(M_f)_g\to M_{fg}$. 它显然是单的,因为$(m/f^i)/(g^j/1)=0\in (M_f)_g$等价于存在$k$和$l$使得$g^kf^l m=0$,而$f^j g^i m/(fg)^{i+j}=0\in M_{fg}$告诉我们存在$n$使得$(fg)^nf^j g^i m=0$. 它也是一个满同态,因为任取$m/(fg)^i\in M_{fg}$,我们可以写成$(m/f^i)/(g^i/1)$的像。所以这是一个同构。

\item 记$S_f$为$\{1$, $f$, $f^{2}$, $\dots\}$全部因子构成的乘性子集。首先,由于$D(g)\subset D(f)$,等价地,$V(g)\supset V(f)$,因之,$\sqrt{(g)}=\pp(V(g))\subset \pp(V(f))=\sqrt{(f)}$. 所以,存在
一个正整数$n$和一个$r\in R$使得$g^n=rf$. 现在,注意$\{1$, $g$, $g^{2}$, $\dots\}$包含$g^{n+1}=rfg$,所以$S'_g$中包含$fg$,故$S_{fg}\subset S_g$. 反之,由于$g$是$fg$的一个因子,所以$S_g\subset S_{fg}$. 结合此二者,$S_g=S_{fg}$,故$M_g=M_{fg}$.

\item 注意到$D(f^n)=D(f)\cap D(f^{n-1})=\cdots=D(f)$,然后反复应用上面这个命题即可。\qedhere
\end{compactenum}
\end{proof}

所以,如果$D(g)\subset D(f)$,我们定义限制映射$\rho^{D(f)}_{D(g)}$为典范同态$M_f\to (M_f)_g=M_{fg}=M_g$. 具体到元素上,即$m/f^n\mapsto mg^n/(fg)^n$. 特别地,当$M=R$的时候,我们有
\[
	\rho^{D(f)}_{D(g)}(r_1/f^{n_1} r_2/f^{n_2})=\rho^{D(f)}_{D(g)}(r_1r_2/f^{n_1+n_2})=r_1r_2 g^{n_1+n_2}/(fg)^{n_1+n_2},
\]
另一方面
\[
	\rho^{D(f)}_{D(g)}(r_1/f^{n_1})\rho^{D(f)}_{D(g)}(r_2/f^{n_2})=(r_1g^{n_1}/(fg)^{n_1})(r_2g^{n_2}/(fg)^{n_2})=r_1r_2 g^{n_1+n_2}/(fg)^{n_1+n_2},
\]
所以$\rho^{D(f)}_{D(g)}$是一个环同态。

\begin{lem}\label{lem:3.12}
给定环$R$以及一个环元$f\in R$. 设$\{f_i\}_{i\in I}$是$R$的一族元素,则$\{D(f_i)\}_{i\in I}$是$D(f)$的一个开覆盖当且仅当$\{f_i/1\}$生成了$R_f$. 此外,$D(f)$是拟紧拓扑空间。
\end{lem}

特别地,如果$f=1$,则$\{D(f_i)\}_{i\in I}$是$D(1)=\spec R$的一个开覆盖当且仅当$\{f_i\}_{i\in I}$生成了单位理想。此外,我们还得到$D(1)=\spec R$是拟紧的。

\begin{proof}
任取$x\in D(f)$即$f(x)\neq 0$,由于$\{D(f_i)\}_{i\in I}$是$D(f)$的一个开覆盖,存在一个$i$使得$x\in D(f_i)$即$f_i(x)\neq 0$. 所以,如果存在一个素理想$\pp_x$使得$f_i(x)=0$对$i\in I$都成立,则$x\not\in D(f)$或等价地$x\in V(f)$. 或者说,对素理想$\pp$,$f_i\in \pp$对所有$i\in I$都成立可以推出$f\in \pp$.

现在考虑$\{f_i/1\}_{i \in I}$在$R_f$中生成的理想$\langle f_i/1\,:\,i \in I \rangle$. 如果$\langle f_i/1\,:\,i \in I \rangle$不是单位理想,因此存在一个素理想$\mm$使得$\langle f_i/1\,:\,i \in I \rangle\subset \mm$. 设$i:R\to R_f$为典范同态,则我们有$R$中理想的包含关系
\[
	\langle f_i\,:\,i \in I \rangle \subset i^{-1}\left(\langle f_i/1\,:\,i \in I \rangle\right)\subset i^{-1}(\mm),
\]
其中$i^{-1}(\mm)$是一个与乘性子集$\{1$, $f$, $f^2$, $\dots\}$不相交的素理想。但由于素理想$i^{-1}(\mm)$包含所有$f_i$,所以也包含$f$,矛盾。

反过来,如果$\{f_i/1\}_{i \in I}$生成了$R_f$,但存在一个$x\in D(f)$使得对所有$i\in I$都有$x\not\in D(f_i)$. 或者说,存在一个素理想$\pp_x$使得$f\not\in \pp_x$但是$\langle f_i\,:\,i \in I \rangle\subset \pp_x$. 现在考虑$\langle f_i\,:\,i \in I \rangle\subset \pp_x$在$i$下的像,由于$f\not\in \pp_x$,所以$i(\pp_x)$必然是$R_f$中的一个素理想,它包含了所有的$f_i/1$,这与$\{f_i/1\}_{i \in I}$生成了$R_f$矛盾。

最后,任取$D(f)$的开覆盖,我们都可以加细成$D(f)=\bigcup_i D(f_i)$的形式,所以只需考虑这种覆盖即可。由于$\{f_i/1\}$生成$R_f$,所以$1\in R_f$可以由有限个$f_i/1$线性组合而成,故这些有限个$D(f_i)$构成了$D(f)$的一个有限子覆盖。
\end{proof}

现在可以来证明$\widetilde{M}$是一个层了。

\begin{thm}
设$R$是一个环,则预层$D(f)\mapsto M_f$是$\spec R$上的一个交换群层,记作$\widetilde{M}$. 特别地,当$M=R$的时候,$\widetilde{R}$是一个环层,记作$\mathcal{O}_R$. 对于任意的$R$-模$M$,$\widetilde{M}$是一个$\mathcal{O}_R$-模层。
\end{thm}

\begin{proof}
设开集$D(f)$被$\{D(f_i)\subset D(f)\}$所覆盖,给定$f_i$,设$m_1$, $m_2\in M_f=\widetilde{M}(D(f))$限制在$D(f_i)$上都相同,我们需要证明$m_1=m_2\in M_f$. 此即截面的唯一性。

由于$D(f)$是预紧的,所以可以假设覆盖是有限的。于是存在整数$N_i$使得$(f_i^{N_i}/1)(m_1-m_2)=0$. 由于覆盖是有限的,所以可以找一个足够大的$N$使得$(f_i^N/1)(m_1-m_2)=0$对任意的$i$都成立。由于$\{D(f_i)=D(f_i^N)\}$是$D(f)$的一个开覆盖,所以$\{f_i^N/1\}$生成了$R_f$,即有$1=\sum_i r_i(f_i^N/1)$,其中$r_i\in R_f$. 因此
\[
	m_1-m_2=\sum_i r_i(f_i^N/1)(m_1-m_2)=0.
\]

此外,给定$D(f)$的一个开覆盖$\{D(f_i)\}$,其中$D(f_i)\subset D(f)$. 如果存在一族截面$\{m_i\in M_{f_i}\}$使得$m_i$和$m_j$限制在$D(f_i)\cap D(f_j)=D(f_if_j)$上时是相等的,我们需要说明存在$m\in M_f$使得$m_i=\rho^{D(f)}_{D(f_i)}(m)$. 此即局部相容的截面可以拼起来。

我们先选一个有限子覆盖$\{f_i\,:\,i\in J\}$. 在这些$M_{f_i}$中,因为$m_i$和$m_j$在$M_{f_if_j}$中相同,所以有
\[
	(f_if_j/1)^{n_{ij}} m_j = (f_if_j/1)^{n_{ij}} m_i.
\]
由于是有限覆盖,所以可以将$n_{ij}$都选成一个数$N_1$. 此外,存在$(f_i^{n_i}/1)m_i=h_i\in M_f$,由于是有限子覆盖,所以可以将所有$n_i$都选成相同的$N>N_1$. 此时,
\[
	(f_i/1)^N h_j=(f_if_j/1)^Nm_j=(f_if_j/1)^Nm_i=(f_j/1)^N h_i.
\]

因为是有限子覆盖,这些$f_i^N/1$生成了$M_f$,所以存在$r_i\in R_f$使得$1=\sum_i r_i (f_i^N/1)$. 定义$m=\sum_i r_ih_i$,我们有
\[
	(f_i/1)^N m =\sum_j r_j (f_i/1)^N h_j =\sum_j r_j (f_j/1)^N h_i= h_i = (f_i/1)^N m_i.
\]
所以$m$限制在$D(f_i)$上就是$m_i$.

最后回到原覆盖,设$f_\alpha$不在我们选取的有限子覆盖中。由于$\{D(f_\alpha f_i)\,:\, i\in J\}$构成了$D(\alpha)$的一个有限开覆盖。从上面的推导,$m$和$m_\alpha$限制到每个$D(f_\alpha f_i)$上都是相同的,都等于$m_i$限制在$D(f_\alpha f_i)=D(f_\alpha)\cap D(f_i)$上得到的截面,所以由一开始证明的唯一性,$m$限制在$D(f_\alpha)$将得到$m_\alpha$.

由于$\widetilde{R}$的限制映射是环同态,所以我们实际上已经得到了一个环层$\mathcal{O}_R$. 在任意的$D(f)$上,交换群层$\widetilde{M}(D(f))=M_f$是一个自然的$\oo_R(D(f))=R_f$-模,且这个模结构与限制映射相容。最后,对于任意的开集$U$,我们需要证明,$\widetilde{M}(U)$是一个$\oo_R(U)$-模。

我们知道
\[
	\oo_R(U)={\varprojlim}_{D(f)\subset U} \oo_R(D(f)),\quad \widetilde{M}(U)={\varprojlim}_{D(f)\subset U} \widetilde{M}(D(f)).
\]
或者具体写出构造
\[
	{\varprojlim}_{D(f)\subset U} \oo_R(D(f))=\left\{
	(r_f)\in \prod_{D(f)\subset U}\oo_R(D(f))\,:\,\text{如果$D(g)\subset D(f)\subset U$,则成立$r_g=\rho^{D(f)}_{D(g)}r_f$}
	\right\},
\]
\[
	{\varprojlim}_{D(f)\subset U} \widetilde{M}(D(f))=\left\{
	(m_f)\in \prod_{D(f)\subset U}\widetilde{M}(D(f))\,:\,\text{如果$D(g)\subset D(f)\subset U$,则成立$m_g=\rho^{D(f)}_{D(g)}m_f$}
	\right\},
\]
于是,我们可以在$\widetilde{M}(U)$上如下定义一个$\oo_R(U)$-模结构
\[
	(r_f)(m_f)=(r_fm_f).
\]
他显然与限制映射相容。因此$\widetilde{M}$是一个$\mathcal{O}_R$-模层。
\end{proof}

\begin{pro}
成立典范的层同构$\widetilde{M_f}\cong \widetilde{M}|_{D(f)}$. 
\end{pro}

\begin{proof}
首先,$D(f)$作为拓扑空间同胚于$\spec(R_f)$,他将$D(g)\subset D(f)$映射为$\spec(R_f)$中的$D(g/1)$,反过来,任取$D(h/f^n)\subset \spec(R_f)$,它在$D(f)$中对应的开集为$D(fh)=D(f)\cap D(h)$. 它们显然互逆。

其次,任取$D(g/f^n)\subset \spec(R_f)$,它上面的模是$(M_f)_g=M_{fg}$. 同样,考虑$D(g/f^n)$在$D(f)$中对应的开集,即$D(fg)\subset D(f)$,它上面的模是$M_{fg}$. 于是,模$M_{fg}$的恒等映射就定义了一族同构$\widetilde{M_f}(D(g/f^n))\to \widetilde{M}|_{D(f)}(D(fg))$,这族同构显然与限制映射相容,所以是一个层的同构。
\end{proof}

\begin{pro}所有$R$-模的伴随层实际上构成一个范畴,这里我们给出这个范畴的一些结论:
\begin{compactenum}[~~~1.]
\item $\Hom_R(M,N)\cong \Hom_{\oo_R}(\widetilde{M},\widetilde{N})$. 将$\psi\in\Hom_R(M,N)$在$\Hom_{\oo_R}(\widetilde{M},\widetilde{N})$中的像记作$\widetilde{\psi}$. 特别地,Yoneda引理告诉我们,$M=0$实际与$\widetilde{M}=0$等价。
\item $M\mapsto \widetilde{M}$, $\psi\mapsto \widetilde{\psi}$构成了一个正合函子。
\item 设$\{M_i\}$是一族$R$-模,则$M=\varinjlim_{i}M_i$的伴随层为$\varinjlim_{i} \widetilde{M_i}$.
\item 设$M$, $N$是两个$R$-模,则$M\otimes N$伴随层为$\widetilde{M}\otimes_{\oo_R}\widetilde{N}$.
\item 设$\psi:M\to N$是一个$R$-模同态,则$\ker \psi$, $\im \psi$和$\coker \psi$的伴随层分别是$\ker \widetilde{\psi}$, $\im \widetilde{\psi}$和$\coker \widetilde{\psi}$. 因此,$\psi$是单同态、满同态、同构当且仅当$\widetilde{\psi}$是单态射、满态射、同构。
\end{compactenum}
\end{pro}

\begin{proof}
第一点,任取$\varphi:\Hom_{\oo_R}(\widetilde{M},\widetilde{N})$,则$\varphi(\spec R):\widetilde{M}(\spec R)\to \widetilde{N}(\spec R)$就是从$M$到$N$的$R$-模同态。反过来,任取$R$-模同态$\psi:M\to N$,我们需要对每一个$D(f)$定义
\[
	\widetilde{\psi}(D(f)):\widetilde{M}(D(f))\to \widetilde{N}(D(f)),
\]
注意到
\[
	\widetilde{M}(D(f))=M_f=R_f\otimes_R M,\quad \widetilde{N}(D(f))=N_f=R_f\otimes_R N,
\]
所以我们定义$\widetilde{\psi}(D(f))=\id_{R_f}\otimes_R \psi=\psi_f$. 不难检验,这是一个$\oo_R$-模层同态,并且$\psi\mapsto \widetilde{\psi}$和$\varphi\mapsto \varphi(\spec R)$互逆。

对于第二点,考虑正合列$M_1\to M_2\to M_3$,对应地有列$\widetilde{M_1}\to \widetilde{M_2}\to \widetilde{M_3}$. 我们知道,一个模层的列是正和的,当且仅当在每条茎上是正和的,所以这里只要考察
\[
	\left(\widetilde{M_1}\right)_x\to \left(\widetilde{M_2}\right)_x\to \left(\widetilde{M_3}\right)_x
\]
或者等价的
\[
	(M_1)_{\pp_x}\to (M_2)_{\pp_x}\to (M_3)_{\pp_x}
\]
即可。由于局部化函子总是正和的,所以上述列都是正合列。

第三点,对每一个$D(f)$,我们有
\[
	\widetilde{M}(D(f))=\left({\varinjlim}_{i}M_i\right)_f,
\]
同时,由于局部化函子作为张量函子,是左伴随函子,与余极限可交换,所以
\[
	\widetilde{M}(D(f))={\varinjlim}_{i}(M_i)_f={\varinjlim}_{i}\widetilde{M_i}(D(f)),
\]
复合上预层$U\mapsto {\varinjlim}_{i}\widetilde{M_f}(U)$到其伴随层${\varinjlim}_{i}\widetilde{M_f}$的典范态射,我们就得到了预层同态$\widetilde{M}\to {\varinjlim}_{i}\widetilde{M_f}$. 现在回到每条茎上,这个态射诱导了茎之间的同构,因为局部化函子作为张量函子,是左伴随函子,与余极限可交换,所以上述预层同态是一个同构。

第四点,对每一个$D(f)$,我们有
\[
	\widetilde{M\otimes_R N}(D(f))=(M\otimes_R N)_f=M_f\otimes_{R_f}N_f=\widetilde{M}(D(f))\otimes_{\oo_R(D(f))}\widetilde{N}(D(f)),
\]
复合上预层$U\mapsto \widetilde{M}(U)\otimes_{\oo_R(U)}\widetilde{N}(U)$到其伴随层的典范态射,我们就得到了预层之间的态射$\widetilde{M\otimes_R N}\to \widetilde{M}\otimes_{\oo_R}\widetilde{N}$. 回到每条茎上,上述态射诱导了以下同构(等同)
\[
	(\widetilde{M}\otimes_{\oo_R}\widetilde{N})_x=\widetilde{M}_x\otimes_{(\oo_R)_x}\widetilde{N}_x=M_{x}\otimes_{R_x}N_x=(M\otimes_R N)_x,
\]
所以预层之间的态射是同构。

最后一点考察正合列
\[
	0\to \ker \psi \to M \to \im \psi\to 0
\]
和
\[
	0\to \im \psi \to N \to \coker \psi\to 0
\]
即可。
\end{proof}

\begin{pro}[素谱上的拟凝聚层]\label{niningju}
设$U$是$\spec R$的一个拟紧开集,$\calf$是一个$\oo_X|_U$-模层,则以下条件等价:
\begin{compactenum}[~~~(1).]
\item 存在一个$R$-模$M$使得$\calf$同构于$\widetilde{M}|_U$.
\item 对$U$的任意形如$\{U_i=D(f_i)\}$的有限覆盖,每一个$\calf|_{U_i}$同构于一个$R_{f_i}$-模$M_i$的伴随层$\widetilde{M_i}$.
\item $\calf$是一个拟凝聚层。
\item 以下两条件同时满足:
\begin{compactenum}[\hspace{-1em}1.]
\item 任取$D(f)\subset U$以及$s\in \calf(D(f))$,都存在一个整数$n\geq 0$以及$U$上的一个截面$t$使得$t|_{D(f)}=f^ns$.
\item 任取$D(f)\subset U$,如果$t\in \calf(U)$限制在$D(f)$上为零,则存在一个整数$n\geq 0$使得$f^nt=0$.
\end{compactenum}
\end{compactenum}
\end{pro}

我们不再证明这个命题,详细见[EGA, Chap 1, 1.4]. 上面命题中的第一点告诉我们,在模层中,拟凝聚模层是更类似于概形的东西。

同时,我们指出,对任意的$R$-模$M$,$\widetilde{M}$是一个拟凝聚$\oo_R$-模层。实际上,由于任取$M$,一定存在正和列
\[
	R^{(I)}\to R^{(J)}\to M\to 0.
\]
由于函子$M\mapsto \widetilde{M}$是正合函子且与余极限可交换,所以存在正合列
\[
	(\oo_R)^{(I)}\to (\oo_R)^{(J)}\to \widetilde{M}\to 0,
\]
这就告诉我们$\widetilde{M}$是一个拟凝聚$\oo_R$-模层。

作为推论,由于$\spec R$本身就是拟紧的,所以$\spec R$上的拟凝聚$\oo_R$-模层都具有$\widetilde{M}$的形式,且$\spec R$的任意拟紧开集上的拟凝聚层都是由某个$\oo_R$-模层$\widetilde{M}$限制得到的。

\begin{para}[仿射概形]
一个赋环空间$(X,\oo_X)$如果同构于关于某个环$R$的赋环空间$(\spec R,\oo_R)$,则称$(X,\oo_X)$是一个仿射概形。一般而言,我们会把赋环空间$(\spec R,\oo_R)$直接记作$\spec R$.

% 如果$(X,\oo_X)$是一个仿射概形,则存在环$S$使得$(X,\oo_X)\cong (\spec S,\oo_S)$. 于是,它们的整体截面是同构的,即$\oo_X(X)\cong \oo_S(\spec S)=S$. 反之,记$(X,\oo_X)$的整体截面是$R$,如果$(X,\oo_X)$是仿射概形,则应该有$(X,\oo_X)\cong (\spec R,\oo_R)$. 为了证明这点,由于$(X,\oo_X)\cong (\spec S,\oo_S)$,所以我们只需要证明,如果$S\cong R$,则$(\spec S,\oo_S)\cong (\spec R,\oo_R)$. 
\end{para}

\begin{para}[仿射概形之间的态射]
设$\varphi:S\to R$是一个环同态,对应地,我们有连续映射${}^a\varphi:\spec R\to \spec S$. 但是,一个仿射概形是一个赋环空间,所以我们还需要给出层之间的态射
\[
	\widetilde{\varphi}:\oo_S\to {}^a\varphi_*\oo_R.
\]

首先,注意到${}^a\varphi^{-1}(V(E))=V(\varphi(E))$,见Proposition \ref{pro:3.8}. 对$E=\{f\in S\}$的情况,我们有${}^a\varphi^{-1}(D(f))=D(\varphi(f))$. 由于$\oo_S(D(f))=S_f$,而$\oo_R(D(\varphi(f)))=R_{\varphi(f)}$. 在这两个环之间,存在典范同态
\[
	\varphi_f:S_f\to R_{\varphi(f)},
\]
它将$s/f^n\in S_f$映射为$\varphi(s)/\varphi(f)^n$. 它也可以看作
\[
	\varphi_f:S(D(f))\to \oo_R(D(\varphi(f)))=\oo_R({}^a\varphi^{-1}(D(f)))={}^a\varphi_*\oo_R(D(f)).
\]
现在取$D(g)\subset D(f)$,注意到,在$S_g=S_{fg}$, $R_{\varphi(g)}=R_{\varphi(fg)}$的等同下,$\varphi_g$也等同于$\varphi_{fg}$,所以交换图
\[
	\xymatrix{
	S_f\ar[d]\ar[rr]^-{\varphi_f}&&R_{\varphi(f)}\ar[d]\\
	S_{fg}\ar[rr]^-{\varphi_{fg}}&&R_{\varphi(fg)}
	}
\]
就告诉我们,$\varphi_f$满足相容条件。由于所有形如$D(f)$的开集构成了$\spec S$的一个拓扑基,因此$\widetilde{\varphi}(D(f))=\varphi_f$就给出了层同态
\[
	\widetilde{\varphi}:\oo_S\to {}^a\varphi_*\oo_R.
\]

综上,我们将$\Phi=({}^a\varphi,\widetilde{\varphi}):\spec R\to \spec S$定义为仿射概形之间的态射。不难发现,如果$\varphi:S\to R$是一个同构,则$\Phi=({}^a\varphi,\widetilde{\varphi})$也是一个同构,这就回答了我们前面的问题。

一般地,假设存在同构$\Phi:(X,\oo_X)\to \spec R$, $\Theta:(Y,\oo_Y)\to \spec S$,如果
\[
	\Theta\Psi\Phi^{-1}:\spec R\to \spec S
\]
具有形式$({}^a\varphi,\widetilde{\varphi})$,其中$\varphi:S\to R$是一个环同态,则称赋环空间态射$\Psi:(X,\oo_X)\to (Y,\oo_Y)$是一个仿射概形态射。当然,后面我们会给一个更加自然的定义,它的表述仅仅依赖于$\Psi$.
\end{para}

\begin{para}
回忆层之间的顺像在茎上的表现(见 \ref{sx}),设$x\in \spec R$,存在一个典范同态
\[
	{}^a\varphi_x:({}^a\varphi_*\oo_R)_{{}^a\varphi(x)}\to (\oo_R)_{x}.
\]
取$(U,s)\in ({}^a\varphi_*\oo_R)_{{}^a\varphi(x)}$,其中$s\in ({}^a\varphi_*\oo_R)(U)=\oo_R({}^a\varphi^{-1}(U))$,而${}^a\varphi_x((U,s))$可以看作$s$在典范限制映射
\[
	\oo_R({}^a\varphi^{-1}(U))\to (\oo_R)_x
\]
下的像。换句话说,
\[
	{}^a\varphi_x:(U,s)\mapsto (\varphi^{-1}(U),s).
\]

同时,对层之间的态射$\widetilde{\varphi}:\oo_S\to {}^a\varphi_*\oo_R$,由于环层范畴的逆像层总是存在的,所以存在(见Theorem \ref{proinverse})
\[
	\widetilde{\varphi}^\#:{}^a\varphi^*\oo_S\to \oo_R.
\]
在茎上,它表现为
\[
	\widetilde{\varphi}^\#_x={}^a\varphi_x \widetilde{\varphi}_{{}^a\varphi(x)}:({}^a\varphi^*\oo_S)_x=(\oo_S)_{{}^a\varphi(x)}\to (\oo_R)_x.
\]
不难发现,在等同
\[
	(\oo_S)_{{}^a\varphi(x)}=S_{{}^a\varphi(x)},\quad (\oo_R)_x=R_x
\]
下,$\widetilde{\varphi}^\#_x$就是环同态$\varphi_x:S_{{}^a\varphi(x)}\to R_x$,它将$s/t$映射为$\varphi(s)/\varphi(t)$,其中$t\not\in \pp_{{}^a\varphi(x)}=\varphi^{-1}(\pp_x)$. 实际上,取$a_{{}^a\varphi(x)}\in (\oo_S)_{{}^a\varphi(x)}$,在$S_{{}^a\varphi(x)}$它写作$s/t$,此时$D(t)$是${}^a\varphi(x)$的开邻域以及$a_{{}^a\varphi(x)}=\langle D(t),s/t\rangle$,注意到$\widetilde{\varphi}(D(t))=\varphi_t$,所以
\[
	\widetilde{\varphi}^\#_x(a_{{}^a\varphi(x)})={}^a\varphi_x \circ \widetilde{\varphi}_{{}^a\varphi(x)}(a_{{}^a\varphi(x)})=\langle {}^a\varphi^{-1}(D(t)),\varphi_t(s/t)\rangle =\langle D(\varphi(t)),\varphi(s)/\varphi(t)\rangle.
\]
\end{para}

\begin{pro}
若赋环空间$(X,\oo_X)$, $(Y,\oo_Y)$是仿射概形,则态射$\Psi=(\psi,\theta):(X,\oo_X)\to (Y,\oo_Y)$是仿射概形态射当且仅当,任取$x\in X$,$\theta_x^\#:(\oo_Y)_{\psi(x)}\to (\oo_X)_x$是一个局部同态,即将$(\oo_Y)_{\psi(x)}$的极大理想映入$(\oo_X)_x$的极大理想中。
\end{pro}

\begin{proof}
可以直接假设$(X,\oo_X)=(\spec R,\oo_R)$, $(Y,\oo_Y)=(\spec S,\oo_S)$. 如果$\Psi=({}^a\varphi,\widetilde{\varphi})$,则此时${}^a\varphi^\#_x=\varphi_x:S_{{}^a\varphi(x)}\to R_x$,他自然是一个局部同态。

反过来,假设$\theta_x^\#:(\oo_S)_{\psi(x)}\to (\oo_R)_x$是一个局部同态。我们先定义
\[
	\varphi=\theta(\spec S):\oo_S(\spec S)=S\to \psi_*\oo_R(\spec S)=R.
\]
由于$\theta_x^\#:(\oo_S)_{\psi(x)}\to (\oo_R)_x$是局部同态,所以它诱导了域的单同态$\theta^x:k(\psi(x))\to k(x)$,取整体截面$\langle \spec S,f\rangle$以及$f(\psi(x))\in k(\psi(x))$,如下等式成立(可参考Proposition \ref{pro:3.8} 前面的推导)
\[
	\theta^x(f(\psi(x)))=\theta_x^\#(\langle \spec S,f\rangle)\,\, \text{mod}\,\, \mm_x=\langle \spec R,\varphi(f)\rangle \,\, \text{mod}\,\, \mm_x = \varphi(f)(x).
\]
因此$f(\psi(x))=0$等价于$\varphi(f)(x)=0$. 同时,注意到等式$\varphi^x(f({}^a\varphi(x)))=\varphi(f)(x)$(见Proposition \ref{pro:3.8}),所以$f(\psi(x))=0$等价于$f({}^a\varphi(x))=0$. 结合这两点,我们有$\pp_{\psi(x)}=\pp_{{}^a\varphi(x)}$,所以$\psi(x)={}^a\varphi(x)$. 由于$x$是任意的,所以作为映射,$\psi={}^a\varphi$.

最后我们只要证明$\theta=\widetilde{\varphi}$. 注意到交换图
\[
	\xymatrix{
	S\ar[d]\ar[rr]^-{\varphi}&&R\ar[d]\\
	S_{f}\ar[rr]^-{\theta(D(f))}&&R_{\varphi(f)}
	}
\]
任取$s\in S$,我们应该有等式$\theta(D(f))(s/1)=\varphi(s)/1$. 当$s=f$的时候,利用$\theta(D(f))$是一个环同态,他将逆变成逆,所以$\theta(D(f))(1/f)=1/\varphi(f)$,依然是利用$\theta(D(f))$是一个环同态,
\[
	\theta(D(f))(s/f^n)=\theta(D(f))(s/1)(\theta(D(f))(1/f))^n=\varphi(s)/\varphi(f)^n.
\]
因此$\theta(D(f))=\varphi_f=\widetilde{\varphi}(D(f))$. 由于$D(f)$构成$\spec S$的拓扑基,于是希望的$\theta=\widetilde{\varphi}$就得到了。
\end{proof}

由这个命题,仿射概形之间的态射$\Psi=(\psi,\theta):(X,\oo_X)\to (Y,\oo_Y)$可以定义为:任取$x\in X$,$\theta_x^\#:(\oo_Y)_{\psi(x)}\to (\oo_X)_x$是一个局部同态。这就诱导我们给出下面的定义。

\begin{para}
设$(X,\oo_X)$是赋环空间,如果任取$x\in X$,$(\oo_X)_x$都是一个局部环,则称$(X,\oo_X)$是一个局部赋环空间。显然,仿射概形都是局部赋环空间。设$(X,\oo_X)$, $(Y,\oo_Y)$都是局部赋环空间,如果赋环空间态射$\Psi=(\psi,\theta):(X,\oo_X)\to (Y,\oo_Y)$对任意的$x\in X$,$\theta_x^\#:(\oo_Y)_{\psi(x)}\to (\oo_X)_x$是一个局部同态,则称$\Psi$是一个局部态射。
\end{para}

所以,上一个命题就是在说:仿射概形之间的局部赋环空间态射$\Psi=(\psi,\theta):(X,\oo_X)\to (Y,\oo_Y)$是局部态射当且仅当它是一个仿射概形态射,且该局部态射完全由$\theta$的整体截面$\theta(\oo_X)$决定。换而言之,如果$\Psi=(\psi,\theta)$和$\Psi'=(\psi',\theta')$都是仿射概形之间的局部态射,则$\Psi=\Psi'$当且仅当$\theta(\oo_X)=\theta'(\oo_X)$.

\begin{thm}
仿射概形范畴等价于交换环范畴的对偶范畴。
\end{thm}

\begin{proof}
函子由$R\mapsto \spec R$以及$\varphi\mapsto ({}^a\varphi,\widetilde{\varphi})$定义。按定义,一个仿射概形同构于某个$\spec R$,所以我们只需证明
\[
	\Hom(S,R)\cong \Hom(\spec R,\spec S).
\]
为此,只需定义$f:\varphi\mapsto ({}^a\varphi,\widetilde{\varphi})$的逆即可。任取$\Phi=(\psi,\theta):(X,\oo_X)\to (Y,\oo_Y)$,我们定义$g:\Phi\mapsto \theta(\oo_X)$. 首先,$g(f(\varphi))=\widetilde{\varphi}(\oo_R)=\varphi$是简单的,其次,由于
\[
f(g(\Psi))=\left({}^a\theta(\oo_X),\widetilde{\theta(\oo_X)}\right)
\]
是一个局部态射,它完全由$\widetilde{\theta(\oo_X)}$的整体截面$\widetilde{\theta(\oo_X)}(\oo_X)=\theta(\oo_X)$决定,因此$f(g(\Psi))=\Psi$. 此即所证。
\end{proof}

这个定理用起来还是很方便的,比如$\zz$是交换环范畴的始对象,则$\spec \zz$是仿射概形范畴的终对象。再比如,如果$\varphi:S\to R$是满同态,则$({}^a\varphi,\widetilde{\varphi})$是一个单态射。

传统的代数几何,是域$k$上的仿射簇范畴与$k$上的有限生成约态$k$-代数范畴的对偶范畴等价。现代代数几何的这个对应的地位即是扩展了经典的对应。


\section{概形}

\begin{para}[概形]
设$(X,\mathcal{O}_X)$是一个赋环空间,如果对于任意的$x\in X$,他都一个开邻域$U$使得$(U,\mathcal{O}_X|_U)$同构于一个仿射概形$\spec{R}$,则称呼$(X,\mathcal{O}_X)$是一个概形。为方便起见,我们会简单称$X$为一个概形,而需要特别指明$X$的拓扑空间结构的时候,会用$|X|$来记$X$的支集,或者叫底空间。

很容易看到任意的概形是局部赋环空间,因为仿射概形是局部赋环空间。概形构成一个范畴,概形间的态射是局部赋环空间之间的局部态射。
\end{para}

给定$f:X\to Y$是一个概形间的态射,如果$f(x)=y$,则称点$y$位于$x$上。由于存在局部同态$(\oo_Y)_y\to (\oo_X)_x$,所以它也给出了域扩张$k(y)\to k(x)$. 

\begin{para}[概形的拓扑]
设$X$是一个概形,对$X$的开集,如果$U$同构于一个仿射概形,则称$U$是一个仿射开集。特别地,当$X=\spec R$的时候,所有形如$D(f)$的开集都是仿射开集,因为$D(f)$同构于$\spec R_f$. 于是,所有仿射开集构成了概形的拓扑基,因为概形局部是仿射概形,而仿射概形的所有仿射开集构成了其拓扑基。因此,设$U$是$X$的一个开集,则$(U,\oo_X|_U)$也是一个概形,他被称为$X$的一个开子概形,更一般的子概形概念我们会在后面详细阐述,这里先按下不表。

现在考虑概形的分离性,它和仿射概形一样,都是$T_0$的。设$x$, $y$是概形上的不同点,如果存在仿射开集包含一点但不包含另一点,则我们已经得证,否则$x$, $y$都是同一个仿射开集中,此时利用仿射概形的结论即可。

对于$T_0$空间,一般点如果存在则唯一。在仿射概形中,我们知道$x\mapsto \overline{\{x\}}$是仿射概形中点与不可约闭子集的一一对应,换句话说,每一个不可约闭子集都存在唯一的一般点。来到概形$X$上,我们任取一个不可约闭子集$Y$,取$y\in Y$以及一个包含$y$的一个仿射开集$U$,则$Y\cap U$是$U$中的不可约闭子集,由仿射概形的结论,存在$x\in Y\cap U$使得$Y\cap U=\overline{\{x\}}\cap U$. 此外,$Y\cap U$当然也是$Y$中的稠密开集,所以,
\[
	Y=\overline{Y\cap U}=\overline{\overline{\{x\}}\cap U}\subset \overline{\{x\}}\cap \overline{U},
\]
故$Y\subset \overline{\{x\}}$. 反过来,由于$x\in Y$,所以$\overline{\{x\}}\subset Y$. 综上,$Y$存在一般点$x$,且由于概形是$T_0$空间,所以也唯一。

如果概形只看拓扑空间是不可约(连通,拟紧)的,则称概形是不可约(连通,拟紧)的。如果概形间的态射只看连续映射是开(闭,单,满)映射,则称这是一个开(闭,单,满)态射。注意,这里的单射和满射不是指monomorphism和epimorphism,而是称injective morphism和surjective morphism,后文会避免直接称呼一个满射,而会用“态射是满的”来替代。设$\Psi=(\psi,\theta):X\to Y$是概形间态射,如果$\im\psi = Y$,则称态射$\Psi$是笼罩的(dominant)。
\end{para}

\begin{para}[$S$-概形]
	设$S$是一个概形,而$X$是另一个概形,一个$S$-概形是一个二元组$(X,\psi)$,其中$\psi:X\to S$是一个概形态射。在仿射概形的情况下,一个$\spec S$-概形$\spec R$不过是在说$S$是一个$R$-代数(因为箭头是反向的)。所以类似于称呼$R$-代数,我们在称呼$S$-概形的时候,经常也略去定义态射$\psi$. 特别地,当我们讨论$\spec A$-范畴的时候,我们也称为一个$A$-范畴。
	
	两个$S$-概形$X$, $Y$间的态射$\varphi$被称为$S$-态射,如果满足交换图
	\[
		\xymatrix{
		X\ar[rr]^\varphi \ar[rd]&&Y\ar[ld]\\
		&S&
		}
	\]
	所有$S$-概形和$S$-态射构成一个范畴,态射集记作$\Hom_S$. 这是非常自然的定义,类似于$R$-代数范畴。
	
	由于概形范畴有终对象$\spec \zz$,所以,每一个概形都可以看成一个$\spec \zz$-概形,或者按照前面介绍过的习惯,也被称为$\zz$-范畴,这是任何环都可以看成$\zz$-代数的一个类比。
\end{para}

\begin{pro}\label{pro:3.1.3}
设$X$是一个概形,则存在一个典范同构
\[
	\Hom(X,\spec R)\cong \Hom(R,\Gamma(X,\oo_X)),
\]
上边左侧是概形态射,右侧是环同态。
\end{pro}

\begin{proof}
	任取$\Phi=(\varphi,\theta):X\to \spec R$,则$\theta:\oo_R\to \varphi_*\oo_X$在$\spec R$上的截面就给出了同态$R\to S$. 反过来,给定同态$\psi:R\to S=\Gamma(X,\oo_X)$,我们的核心工具是Proposition \ref{rsnh}. 找一个$X$的仿射开覆盖$\{U_\alpha\}$,这限制映射复合$\psi$给出了一个一族同态
	\[
	\left\{\psi_\alpha:R\to \Gamma(U_\alpha,\oo_X)\right\},
	\]
	利用交换环范畴的对偶范畴和仿射概形范畴等价,我们可以找到对应的
	\[
	\left\{\Psi_\alpha:\spec\Gamma(U_\alpha,\oo_X)\to \spec R\right\},
	\]
	由于$U_\alpha$都是仿射开集,则$\spec\Gamma(U_\alpha,\oo_X)$可以典范等同于$U_\alpha$. 因此,我们给出了一族赋环空间之间的局部态射$\left\{\Psi_\alpha: U_\alpha \to \spec R\right\}$. 它们符合相容条件,所以Proposition \ref{rsnh} 告诉我们可以粘成唯一一个$\Psi:X\to \spec R$. 他是局部态射,所以也就是概形同态。由唯一性,上述两个态射的构造是互逆的。
\end{proof}

由于任意的环都有唯一到$\zz$的同态,所以,任取概形$X$,我们都有唯一的概形态射$X\to \spec \zz$. 因此,$\spec \zz$是概形范畴的终对象。

\begin{pro}
	设$X$是一个概形,则存在一个典范同构
	\[
		\Hom(X,\mathbb A_{\zz}^1)\cong \Gamma(X,\oo_X),
	\]
	其中$\mathbb A_{\spec \zz}^1:=\spec \zz[t]$是一维仿射空间。
\end{pro}

利用基概形扩张,我们可以看到,如果$X$是一个$S$-概形,则
$\Hom_S(X,\mathbb A_{S}^1)\cong \Gamma(X,\oo_X)$,其中
$\mathbb A_{S}^1:=\mathbb A_{\spec \zz}^1\times_\zz S$是$S$上的一维仿射空间,
如果$S=\spec R$是仿射概形,则$\mathbb A_{\spec R}^1=\spec R[t]$,
从下面的证明可以看到,我们可以容易推广到这个较简单的情况。

\begin{proof}
	任取$R$-代数$S$,我们都有$R$-代数同构
	\[
		S\cong \Hom_R(R[t],S),
	\]
	来自于元素$s\in S$,我们定义右侧的态射为$\varphi_s:t\mapsto s$.
	反过来,任取$\varphi\in \Hom_R(R[t],S)$,根据代数同态的性质,只需要
	知道$\varphi(t)\in S$就可以知道整个$\varphi$. 最后,我们将其应用到
	$S=\Gamma(X,\mathcal O_X)$并利用Proposition \ref{pro:3.1.3} 即可。
\end{proof}

\begin{para}[概形的黏合]
概形的黏合就是赋环空间的黏合,见Proposition \ref{rsn},当然,由于仿射开集构成了概形的拓扑基,所以概形黏合起来得到的还是概形。在这意义上,我们可以说,每一个概形都是由仿射概形黏合起来得到的。
\end{para}

著名的反例之一就是具有双重原点的直线,将两个直线$\spec(k[x])$挖掉零点的部分粘合起来,这样依然可以得到一个概形,但是他并不是仿射概形。

\begin{para}[射影空间]
考虑$R$是一个环,我们可以定义$n$-维射影空间$\mathbb P_R^n$如下:他由$n+1$个仿射空间$\mathbb A_R^n$粘合而成,记作
\[
	U_i=\spec R\biggl[\frac{x_0}{x_i},\dots,\frac{x_n}{x_i}\biggr],
\]
其中我们略去了$x_i/x_i$. 我们可以将上面$\spec$中的环看作
\[
	R[x_0,x_0^{-1},x_1,x_1^{-1},\dots,x_n,x_n^{-1}]
\]
的子环。子环$U_i$可以看作$x_i\neq 0$定义的开集,粘合就可以看作如下典范的等同:
\[
	U_i\cap U_j=\spec R\biggl[\frac{x_0}{x_i},\dots,\frac{x_n}{x_i}\biggr]_{x_j/x_i} = \spec R\biggl[\frac{x_0}{x_j},\dots,\frac{x_n}{x_j}\biggr]_{x_i/x_j}=U_j\cap U_i.
\]

不难看到,对$k\geq 1$,我们都有$\Gamma(\bigcup_{i=0}^k U_i,\mathcal O_{\mathbb P_R^n})=R$. 实际上,
我们只需对$U_0\cup U_1$证明即可。设$s$是一个$U_0\cup U_1$上的截面,他在$U_0$上的限制
可以写成一个多项式$s_0(x_1/x_0,\dots,x_n/x_0)$,而在$U_1$上的限制可以写成一个多项式
$s_1(x_0/x_1,\dots,x_n/x_1)$,由于$s_0$和$s_1$在$U_0\cap U_1$上的限制相同,
所以
\[
	s_0(x_1/x_0,\dots,x_n/x_0)=s_1(x_0/x_1,\dots,x_n/x_1)\in R[x_0,x_0^{-1},x_1,x_1^{-1},\dots,x_n,x_n^{-1}],
\]
此时,
\[
	s_0=\sum_a c_a(x_1/x_0)x_0^{-\sum_ia_i}\prod_{i>1} x_i^{a_i},\quad 
	s_1=\sum_a d_a(x_0/x_1)x_1^{-\sum_ia_i}\prod_{i>1} x_i^{a_i},
\]
其中$c_a$和$d_b$都是一元多项式,比较系数,立马得到
\[
	c_a(x_1/x_0)(x_1/x_0)^{\sum_ia_i}=d_a(x_0/x_1),
\]
再次比较系数,只可能$\sum_i a_i=0$且$c_a=d_a$是一个常数,即$s\in R$. 特别地,我们有
$\Gamma(\mathbb P_R^n,\mathcal O_{\mathbb P_R^n})=R$.

\end{para}

\begin{para}[局部概形]
所谓的局部概形就是局部环的谱。设$R$是一个局部环,则$\spec R$只有一个闭点$a$(闭点对应$R$的极大理想),任取其他点$b$,我们都有$a\in\overline{\{b\}}$.

任取概形$X$以及一点$x\in X$,我们称局部概形$\spec\left((\oo_{X})_x\right)$为$X$在$x$处的局部概形。任取$x$的一个仿射开集$U=\spec R$,我们可以将$(\oo_{X})_x$典范等同于$R_x$,于是典范同态$R\to R_x$就给出了态射$\spec\left((\oo_{X})_x\right)\to U$,然后复合上典范含入$U\hookrightarrow X$,我们就得到了典范态射$\spec\left((\oo_{X})_x\right)\to X$. 方便起见,我们记$\spec\left((\oo_{X})_x\right)$为$X_x$,当然这和$(\oo_X)_x$是不同的。

上述构造似乎与我们选取的仿射开集$U$有关,但事实并非如此。如果$V$是另一个包含$x$的仿射开集,则存在包含$x$的仿射开集$W\subset U\cap V$,所以我们可以假设$V\subset U$. 此时我们只需考虑仿射概形的情况,而限制映射的相容性就给出了结论。

下面我们要说明典范同态的像在$X$中是如何的,为此首先记$S_x=\left\{y\in X\,:\, x\in \overline{\{y\}}\right\}$. 不难发现,$S_x=\bigcap_{U\ni x}U$,其中$U$跑遍$x$的开邻域。实际上,如果$x\in \overline{\{y\}}$,则$U\cap \overline{\{y\}}$作为$\overline{\{y\}}$中的开集\footnote{一个空间如果有一般点,则其中任意开集都包含这个一般点。}必然包含$y$,所以$y\in U$,这给出了$S_x\subset \bigcap_{U\ni x}U$. 反过来,如果$y\not\in \bigcap_{U\ni x}U$,换言之存在$x$的一个开邻域$U$使得$y\not\in U$,即$X-U$是包含$y$的一个闭集,因此$\overline{\{y\}}\cap U=\varnothing$,这给出$x\not\in \overline{\{y\}}$. 所以$\bigcap_{U\ni x}U\subset S_x$.

现在考虑典范态射$(\psi,\theta):X_x\to X$. 显然,$\psi^{-1}(x)$是$X_x$中唯一的闭点,于是任取$a\in X_x$,我们有
\[
	x=\psi(\psi^{-1}(x))\in \psi\left(\overline{\{a\}}\right)\subset \overline{\{\psi(a)\}}.
\]
所以$\im \psi\subset S_x$. 更进一步,我们其实还有$\im \psi=S_x$. 由于$S_x$包含于任意一个$x$的仿射开邻域中,所以可以假设概形$X=\spec R$,而这时$\psi={}^a i$,其中$i:R\to R_x$是典范同态,此时$\im({}^a i)$即$\spec R$中满足$\pp_y\subset \pp_x$的$y$构成的子集(见Lemma \ref{lem:3.9} 的推论),由于$\pp_y\subset \pp_x$等价于$x\in \overline{\{y\}}$,所以$\im({}^a i)$即$S_x$. Lemma \ref{lem:3.9} 同时还告诉我们,$\psi$是从$X_x$到$\im\psi$的一个同胚。并且,由于$X$中所有不可约闭子集都具有形式$\overline{\{y\}}$,因此$X_x$中的点与$X$中所有包含$x$的不可约闭子集之间有一个一一对应。

最后,取$z=\psi(y)$,则$\theta_y^\#:(\oo_X)_z\to ((\oo_X)_x)_y$是一个同构。实际上,只需考虑$X=\spec R$, $\theta=\widetilde{i}$的情况,此时上面的同态即$i_y:R_z\to (R_x)_y$. 这个是显然的同构。
\end{para}

总结一下上面的讨论:

\begin{pro}
$X_x$作为集合是所有包含$x$的开集的交,等价地,$X_x$中的点由所有包含$x$的不可约闭子集的一般点构成,因此$X_x$中的点一一对应于$X$中所有包含$x$的不可约闭子集。此外,典范态射$X_x\to X$是一个概形范畴的单态射。
\end{pro}

如果$x$是一个不可约分支的一般点,则$X_x$只有一个点即$x$,即$(\oo_X)_x$只有一个素理想,当然也只有一个极大理想。反之亦然。

顺带一提,一般来说$X_x$不是$X$的子概形,因为他可能不是局部闭的,但如果$x$是闭点,则$X_x$是闭子概形。具体定义见子概形章节。

\begin{pro}\label{pro:4.6}
设$\spec A$是一个局部概形,闭点记作$a$,又设$X$是一个概形,则任意态射$u=(\psi,\theta):\spec A\to X$都可以经由典范态射$i_{\psi(a)}:X_{\psi(a)}\to Y$唯一分解,即唯一存在态射$v:\spec A\to X_{\psi(a)}$使得$u=i_{\psi(a)}v$. 
\end{pro}

\begin{proof}
任取$b\in \spec A$,总是有$a\in\overline{\{b\}}$,所以$\psi(a)\in \overline{\{\psi(b)\}}$. 换言之,我们有$\im \psi\subset S_{\psi(a)}$. 由于$S_{\psi(a)}$包含于$\psi(a)$的任意开邻域中,所以也包含于$\psi(a)$的任意仿射开邻域中。所以不妨假设$X=\spec B$,从而$u=({}^a\varphi,\widetilde{\varphi})$,而$\varphi:B\to A$是一个环同态。

由于$\varphi^{-1}(\pp_a)=\pp_{\psi(a)}$,所以$B-\pp_{\psi(a)}$的任意元素在$\varphi$下的像总是局部环$A$中的可逆元,所以由分式环的泛性质,存在唯一的环同态$\sigma:(B-\pp_{\psi(a)})^{-1}B=B_{\psi(a)}\to A$使得分解$\varphi=\sigma i$成立,其中$i:B\to B_{\psi(a)}$是分式环的典范同态。由于交换环范畴的对偶范畴与仿射概形范畴是等价范畴,所以在仿射概形范畴范畴,我们也有唯一分解$u=i_{\psi(a)}v$,其中$u$对应环同态$\varphi$,$i_{\psi(a)}$对应环同态$i$,而$v$对应环同态$\sigma$.
\end{proof}

现在设$x\in \overline{\{y\}}$(或$\overline{\{x\}}\subset \overline{\{y\}}$),换言之,$y\in S_x$,故所有包含$x$的开子集也都包含$y$,故$S_y\subset S_x$.  此时,取一个仿射开邻域包含$S_x$,我们可以假设$X=\spec R$,且$\pp_y\subset \pp_x$,则$X_x=\spec R_x$和$X_y=\spec R_y$. 由于乘性子集的包含关系$(R-\pp_x) \subset (R-\pp_y)$,同态$R\to R_{y}$将所有的$(R-\pp_x)$的元素都变成可逆的,进而分式环的泛性质给出了分解$R\to R_x\to R_y$,所以我们给出了概形间态射$X_y\to X_x\to X$. 他们之间连续映射在$X$中即单射$S_y\hookrightarrow S_x$. 进一步,认识到$(R_{x})_y=R_y$,这就告诉我们$(X_x)_y=X_y$. 所以,粗略来说,$\overline{\{x\}}$越大,$X_x$越小。

\begin{para}[$T$-值点]
设$X$是一个概形,而$T$是任意概形,记$X(T)=\Hom(T,X)$. 于是$X$构成一个概形范畴到集合范畴的反变函子,而$X(T)$中的元素被称为概形$X$的$T$-值点。现在来到$S$-概形范畴,我们可以同样定义$X(T)_S=\Hom_S(T,X)$. 如果不会造成误解,我们有时会略去它。此外,如果$T=\spec A$,则我们也会称$X(T)$中的元素为概形$X$的$A$-值点,如果不会发生误解。

将$X$看成这样一个函子的理由一部分来自于Yoneda引理,如果$X$和$Y$作为反变函子同构,则$X$和$Y$作为概形也同构。
\end{para}

\begin{para}[几何点]
设$A$是一个局部环,而$X$是一个概形,我们考虑$X$的$A$-值点。由Proposition \ref{pro:4.6},$X(A)$中的元素一一对应于某个局部同态$(\oo_X)_x\to A$,其中$x\in X$,这个$x$被称为$A$-值点的位置,或者说这个$A$-值点位于$x$. 进而,我们定义$X$的在某个域$k$上取值的点为一个在$k$上取值的几何点。

给出一个在$k$上取值的几何点,等价于给出它的位置$x$以及一个域扩张$k(x)\to k$. 这是因为局部同态$(\oo_X)_x\to k$将$(\oo_X)_x$的极大理想映成了$0$(作为$k$的极大理想),所以给出局部同态$(\oo_X)_x\to k$也等价于给出域扩张$k(x)\to k$. 

现在设$X$, $\spec k$都是$S$-概形,其中$k$是一个域。由上所述,$\spec k$是一个$S$-概形等价于给定$s\in S$以及一个域扩张$k(s)\to k$. 考虑一个几何点$\psi\in X(k)_S$,它等价于给出$x\in X$以及一个域扩张$k(x)\to k$. 由分解
\[
	\spec k \xrightarrow{\psi} X\xrightarrow{f} S,
\]
其中$f$是$X$作为$S$-概形的定义态射,我们应有$s=f(x)$以及分解
\[
	k(s)\to k(x)\to k.
\]
此时,我们称几何点$\psi$是一个位于$s$上取值于$k$中的点。
\end{para}

关于域扩张,我们可能以后需要下面的引理,这是域扩张的基本常识,但是方便起见,我们还是列出并给予证明。它依赖于选择引理,因为要保证极大理想(素理想)的存在性。

\begin{lem}\label{composite_ext}
设$\{u_i:k\to k_i\}$是有限个$k$的扩张,其中$k_i$当然都是域,于是存在一个域$K$以及域扩张$\{v_i:k_i\to K\}$和$w:k\to K$使得$w=v_iu_i$对所有的$i$都成立。
\end{lem}

\begin{proof}
考虑张量积
\[
	R=k_1\otimes_k \cdots \otimes_k k_n,
\]
这是一个$k$-代数,一般来说不是一个域也不是一个整环。选一个$R$的极大理想$\mm$,考虑商域$K=R/\mm$,我们有自然的同态$\pi:R\to K$. 现在记$p_i$为$k_i$到$R$的典范态射,则
\[
	v_i=\pi p_i:k_i\to K
\]
就满足$v_iu_i=v_ju_j$. 实际上,任取$a\in k$,由张量积的性质
\[
	1\otimes_k \cdots\otimes_k u_i(a)\otimes_k \cdots\otimes_k 1=1\otimes_k \cdots\otimes_k u_j(a)\otimes_k \cdots\otimes_k 1,
\]
所以$p_iu_i(a)=p_ju_j(a)$. 
\end{proof}

\begin{pro}
	设$K$是一个域,如果$f:\spec K\to X$是一个位于$x$的$K$-值点,
	而$y\in \overline{\{x\}}$是$\{x\}$的闭包中与$x$不同的点,
	则存在$K$的一个赋值环$R\subset K$和位于$y$上的$R$-值点$g:\spec R\to X$使得
	分解$f:\spec K\to \spec R\xrightarrow{g} X$成立。
\end{pro}
	
注意到,$g$将$\spec R$的闭点映射到$y$,而将$\spec R$的一般点(对应零理想)映射到$x$,
因为分解中的第一个映射将$\spec K$的闭点(对应零理想)映射到了$\spec R$的一般点
(对应零理想)。
	
\begin{proof}
	如果存在这样的分解,那么$g$还可以进一步分解为
	\[
		g:\spec R\to X_y=\spec (\mathcal O_X)_y\to X,
	\]
	由于$X_y$拓扑上是$S_y=\left\{z\in X\,:\, y\in \overline{\{z\}}\right\}$,
	所以$x\in X_y$. 我们可以进一步在$(\mathcal O_X)_y$中模掉$x$对应的素理想,
	对应的局部整环记作$A$. 此时,我们可以用$\spec A$来代替$X$,此时$x$是其一般点
	而$y$是其闭点。这样,问题就变成了一个交换代数问题,即,若$A$是一个局部整环
	而$K$是一个包含他的域,则存在$K$的一个赋值环$R$,使得$A\subset R\subset K$,
	且$R$笼罩$A$,即对$R$和$A$的极大理想$\mathfrak m_R$, $\mathfrak m_A$成立
	$\mathfrak m_A=A\cap \mathfrak m_R$. 若$K$恰好是$A$的分式域,则命题是显然的,
	因为赋值环是局部整环按照笼罩关系构成的极大元,所以只要找个包含$A$的链中的
	极大元,那就是想要的赋值环。更一般的情况的证明可见 Ulrich \& Torsten 
	的Proposition 15.6和15.7.
\end{proof}

利用这个命题,容易看到,$\overline{\{x\}}$中的每个点都对应于$k(x)$的一个赋值环。


\section{概形的纤维积}

\begin{para}[纤维积]
设$X$, $Y$是$S$-概形,类似任何范畴中纤维积的定义,我们定义$X\times_S Y$是图$X\to S\leftarrow Y$的极限。用泛性质表述的话,即如下交换图
\[
\begin{xy}
	\xymatrix{
		A\ar@/_/[rdd]_{p_2}\ar@/^/[drr]^{p_1}\ar@{-->}[dr]&&\\
		&X\times_S Y\ar[d]^{\pi_2}\ar[r]_-{\pi_1}&X\ar[d]\\
		&Y\ar[r]&S
	}
\end{xy}
\]
类似于任何范畴的情况,如果存在终对象,则我们可以由纤维积构造积。具体到概形范畴,即$X\times Y=X\times_{\spec \zz} Y$.

一般而言,如果$S=\spec R$,则$X\times_S Y$也会记作$X\times_R Y$. 所以$X\times Y=X\times_{\zz} Y$.
\end{para}

概形的纤维积的存在性先按下不表,不过对一个特殊的情况,我们可以直接写出纤维积:
\[
	\spec A\times_{\spec R} \spec B=\spec(A\otimes_R B).
\]
利用交换环范畴的对偶与仿射概形范畴的等价,直接检查泛性质就行。

此外,如果记$|X|$为概形$X$的底空间(或称支集),则由$|X|\times_{|S|}|Y|$的泛性质,我们得到了典范态射$|X\times_S Y|\to |X|\times_{|S|}|Y|$. 这个典范态射一般不是单射,但后面会证明,这是一个满射。当然,这个典范态射不是同胚的理由是不难猜的,因为不是所有连续映射$|X|\to |S|$都可以写成某个概形间态射$X\to S$的连续函数部分。实际上,对任意的域$k$,$|\spec k|$都是单点集。然后我们考虑$f:|\spec \mathbb{Q}|\to |\spec \rr|$,他将左侧唯一的点映成右侧唯一的点,自然是连续的。如果他可以写成概形间一个态射$\spec \mathbb{Q}\to \spec \rr$的连续函数部分,即$f={}^a\psi$,而$\psi:\rr\to \mathbb{Q}$是一个域同态。它是一个单射且$\psi(1)=1$,由环同态与环乘法的性质,我们有$\psi$还是满的,因此$\psi:\rr\to \mathbb{Q}$是一个同构,显然不可能,毕竟两边基数都不同。

\begin{pro}\label{pro:3.2.2}
纤维积具有如下一些性质,它们并不依赖于概形范畴:
\begin{compactenum}[~~~1.]
\item 设$X$是一个$S$-概形,则存在典范等同$X\times_S S=S\times_S X=X$,这个等同是函子性。同时,在这个等同下,设$f:X\to S$和$g:Y\to S$是两个$S$-概形,则$f\times_S 1_Y:X\times_S Y\to S\times_S Y=Y$和$1_X\times_S g:X\times_S Y\to X\times_S S=X$就是典范投影。
\item 设$X$, $S'$是一个$S$-概形,$S''$是一个$S'$概形,则可以典范等同$(X\times_S S')\times_{S'}S''=X\times_S S''$,这个等同是函子性。
\item 存在典范等同$(X\times_S Y)\times_S Z=X\times_S (Y\times_S Z)$,这个等同是函子性的。
\item 若$f:S'\to S$是一个单态射,而$X$, $Y$都是$S'$-概形,则通过$f$可将$X$, $Y$看成$S$-概形,则成立典范等同$X\times_{S'}Y=X\times_S Y$.
\end{compactenum}
\end{pro}

\begin{proof}
前三点都是范畴论里面的标准结论,不再证明。至于第四点,我们来证明$X\times_{S'} Y$满足$X\times_S Y$的泛性质。

设$X$和$Y$的典范态射为$\varphi:X\to S'$和$\psi:Y\to S'$,任取$u:T\to X$和$v:T\to Y$满足$f\varphi u=f\psi v$,由于$f$是单态射,所以$\varphi u =\psi v$,进而从$X\times_{S'} Y$的泛性质,存在唯一的态射$(u,v):T\to X\times_{S'} Y$使得$u=\pi_X (u,v)$以及$u=\pi_Y (u,v)$,其中$\pi_X$, $\pi_Y$是典范投影。此即所需。
\end{proof}

\begin{para}[基概形扩张]
设$X$, $S'$都是$S$-概形,其中$S'$作为$S$-概形的定义态射为$\varphi:S'\to S$,则$X\times_S S'$在$\pi_2:X\times_S S'\to S'$下构成一个$S'$-概形。这个过程被称为基概形扩张,方便起见,我们又是会记$X\times_S S'$为$X_{(S')}$或者$X_{(\varphi)}$. 设$f:X\to Y$是$S$-概形态射,我们存在$f\times_S \id_{S'}:X\times_S S'\to Y\times_S S'$,这是一个$S'$-概形扩张。方便起见,我们记作$f_{(S')}$或者$f_{(\varphi)}$. 

显然,如果$T$是一个$S'$-概形,则复合上$S'\to S$,$T$也可以看出一个$S$-概形,
而这构成了一个遗忘函子,则基概形扩张可以看成这个遗忘函子的右伴随,
即存在双函子同构
\[
	\Hom_{S'}(T,X_{(S')})\cong \Hom_S(T,X).
\]
证明不过是纤维积的泛性质:
\[
\begin{xy}
	\xymatrix{
		T\ar@/_/[rdd]\ar@/^/[drr]\ar@{-->}[dr]&&\\
		&X\times_S S'\ar[d]\ar[r]&X\ar[d]\\
		&S'\ar[r]&S
	}
\end{xy}
\]
\end{para}

作为直接的应用,我们已经知道概形$X$的整体截面意义对应到了$X$到一维仿射概形的
态射,即
\[
	\Gamma(X,\mathcal O_X)\cong \Hom(X,\mathbb A_{\mathbb Z}^1),
\]
如果进一步$X$是一个$S$-概形,则
\[
	\Gamma(X,\mathcal O_X)\cong \Hom_S(X,\mathbb A_{S}^1),
\]
其中$\mathbb A_{S}^1$是$S$上的一维仿射空间,即
$\mathbb A_{S}^1:=\mathbb A_{\mathbb Z}^1\times_\zz S$. 
特别地,如果我们把$S$取成$\spec \mathbb C$,则$\mathbb A_{\mathbb C}^1$
基本可以看出复平面本身,所以$\mathbb C$-概形$X$的整体截面差不多就是
$X$上的复值函数的集合。

\begin{pro}
存在函子性等同$(X\times_S Y)_{(S')}=X_{(S')}\times_{S'} Y_{(S')}$. 因此,基概形扩张与纤维积是可交换的。所以,整体截面可以看出$X$上的$\mathbb A_{S}^1$-值函数的
集合。
\end{pro}

\begin{proof}
考虑
\[
	(X\times_S S')\times_{S'}Y_{(S')}=X\times_{S}Y_{(S')}=X\times_{S}(Y\times_S S')=(X\times_{S} Y)\times_S S',
\]
所以$(X\times_S Y)_{(S')}=X_{(S')}\times_{S'} Y_{(S')}$.
\end{proof}

\begin{para}[原像]
基概形扩张有时候又被称为原像。这不是没有原因的,回到集合范畴,如果$Y\subset S$以及$\psi:X\to S$,则$X\times_S Y$不过就是$Y$关于$\psi$的原像$\psi^{-1}(Y)\subset X$. 

在这里,如果$X$是一个$S$-概形,而$\varphi:S'\to S$是另一个$S$-概形,则我们记$\varphi^{-1}(X)=X_{(\varphi)}$,这是一个$S'$-概形,称为$S$-概形$X$在$\varphi:S'\to S$下的原像。
\end{para}

作为原像的例子,我们有下面的命题。

\begin{pro}\label{preimage}
设$U$是$S$的一个开子概形,$X$是一个概形,而$\Psi=(\psi,\theta):X\to S$是一个概形同态。利用典范同态$U\hookrightarrow S$,则$U$可以看成一个$S$-概形,此时纤维积$U\times_S X=\Psi^{-1}(U)$存在,他就是$X$的开子概形$\psi^{-1}(U)$.
\end{pro}

所以,此时$\Psi^{-1}(U)$就是经典意义下的原像。方便起见,我们时常会直接用连续映射$\psi$来指代$\Psi$.

\begin{proof}
直接检查开子概形$\psi^{-1}(U)$是否满足纤维积$U\times_S X$的泛性质即可,取交换图
\[
\begin{xy}
	\xymatrix{
		Z\ar@/_/[rdd]_{g}\ar@/^/[drr]^{f}\ar@{-->}[dr]&&\\
		&\psi^{-1}(U)\ar[r]_-{p}\ar[d]^{i_{\psi^{-1}(U)}}&U\ar[d]\\
		&X\ar[r]_-\Psi & S
	}
\end{xy}
\]
其中$i_{\psi^{-1}(U)}$是典范含入,而$p$来自于Proposition \ref{pro:1.20},满足唯一分解$\Psi|_{\psi^{-1}(U)}=i_Up$,其中$i_U:U\to S$是典范含入。

现在设存在$f:Z\to U$以及$g:Z\to X$使得上面的交换图成立,即$i_U f=\Psi g$. 设$g=(\varphi,\sigma)$,分解$i_U f=\Psi g$的连续函数部分给出了$\im(\psi\varphi)\subset U$,因此$\im \varphi\subset \psi^{-1}(U)$. 利用Proposition \ref{pro:1.20},我们得到了唯一的态射$g':Z\to\psi^{-1}(U)$使得分解$g=i_{\psi^{-1}(U)}g'$成立。

最后,只要检查$g'$满足上述交换图即可。由于
\[
	i_U f = \Psi g = \Psi i_{\psi^{-1}(U)}g'=\Psi|_{\psi^{-1}(U)}g'=i_Upg',
\]
由于$i_U$是单态射,是左可消的,此即$f=pg'$.
\end{proof}

\begin{coro}
设存在纤维积$X\times_S Y$,$p_X:X\times_S Y\to X$是其典范投影态射,再设$U$是$X$的一个开子概形,则概形$Y\times_S U=(X\times_S Y)\times_X U$就是$X\times_S Y$的开子概形$p_X^{-1}(U)$.
\end{coro}

% \begin{proof}
% 直接检查$\pi_X^{-1}(U)$是否满足纤维积$Y\times_S U$的泛性质即可,取交换图
% \[
% \begin{xy}
% 	\xymatrix{
% 		Z\ar@/_/[rdd]_{g}\ar@/^/[drr]^{f}\ar@{-->}[dr]&&\\
% 		&\pi_X^{-1}(U)\ar[r]_-{q_X}\ar[d]^{q_Y}&U\ar[d]\\
% 		&Y\ar[r]&S
% 	}
% \end{xy}
% \]
% 其中$q_X$, $q_Y$分别是$p_X$, $p_Y$限制在${\pi_X^{-1}(U)}$得到的态射,特别是$q_X$来自于Proposition \ref{pro:1.20},满足$p_X|_{\pi_X^{-1}(U)}=i_Uq_X$,其中
% $i_U:U\to X$是典范含入。

% 对$i_Uf:Z\to X$与$g:Z\to Y$,由纤维积的泛性质,我们有唯一的态射$h=(\psi,\sigma):Z\to X\times_S Y$,使得分解$i_Uf=p_X h$和$g=p_Yh$成立。分解$if=p_X h$的连续映射部分给出$\im(\pi_X\psi)\subset U$,所以$\im \psi\subset \pi_X^{-1}(U)$. 利用Proposition \ref{pro:1.20},我们得到了唯一的态射$h':Z\to\pi_X^{-1}(U)$使得分解$h=i_{\pi_X^{-1}(U)}h'$成立,其中$i_{\pi_X^{-1}(U)}:\pi_X^{-1}(U)\to X\times_S Y$是典范含入。

% 最后,只要检查$h'$满足上述交换图即可。这来自于
% \[
% 	g=p_Yh=p_Y i_{\pi_X^{-1}(U)}h'=p_Y|_{\pi_X^{-1}(U)}h'=q_Yh',
% \]
% 以及
% \[
% 	i_U f = p_X h = p_X i_{\pi_X^{-1}(U)}h'=p_X|_{\pi_X^{-1}(U)}h'=i_Uq_Xh',
% \]
% 由于$i_U$是单态射,所以也是左可消的,此即$f=q_Xh'$.
% \end{proof}

\begin{coro}\label{coro:3.2.8}
设存在纤维积$X\times_S Y$,再设$U$是$X$的开子概形而$V$是$Y$的开子概形,则$U\times_S V$就是$X\times_S Y$的开子概形$p_X^{-1}(U)\cap p_Y^{-1}(V)$. 因此,如果两个$S$-概形有纤维积,则它们的任意开子概形(当然都是$S$-概形)之间也存在纤维积。
\end{coro}

\begin{proof}
首先注意到,将$p_Y$限制在$p_X^{-1}(U)$上得到了态射$p_X^{-1}(U)\to Y$,连同典范含入$V\hookrightarrow Y$,我们考虑纤维积$p_X^{-1}(U)\times_Y V$,由Proposition \ref{preimage},它存在且作为概形等于$p_X^{-1}(U)$的开子概形
\[
	\left(p_Y|_{p_X^{-1}(U)}\right)^{-1}(V)=p_X^{-1}(U)\cap p_Y^{-1}(V),
\]
$p_X^{-1}(U)\cap p_Y^{-1}(V)$是$p_X^{-1}(U)$的开子概形,当然也就是$X\times_S Y$的开子概形。同时,上个推论告诉我们
\[
	p_X^{-1}(U)\times_Y V=(U\times_S Y)\times_Y V=U\times_S V,
\]
因此$U\times_S V=p_X^{-1}(U)\cap p_Y^{-1}(V)$. 
\end{proof}

% 这个推论很管用,有了它,我们局部计算概形的纤维积才成了可能。当然,现在的问题是我们完全不知道纤维积的典范投影$p_X$和$p_Y$如何表现。

\begin{thm}[纤维积的存在性]\label{thm:1.5.9}
概形范畴存在纤维积。
\end{thm}

\begin{lem}\label{lem:3.2.10}
设$X$, $Y$都是$S$-概形,而$\{X_\alpha\}$是$X$的一个开覆盖,如果$X_\alpha\times_S Y$都存在,则$X\times_S Y$存在。
\end{lem}

\begin{proof}
设$p_\alpha:X_\alpha\times_S Y\to X_\alpha$是典范投影态射。由于$X_{\alpha\beta}=X_{\alpha}\cap X_{\beta}$是$X_{\alpha}$和$X_{\beta}$的开子集,此时纤维积$X_{\alpha\beta}\times_S Y$由上面的推论是存在的,且就是$X_\alpha\times_S Y$的开子概形$p_\alpha^{-1}(X_{\alpha\beta})$,同样也就是$X_\beta\times_S Y$的开子概形$p_\beta^{-1}(X_{\alpha\beta})$. 所以,由纤维积在同构意义下的唯一性,我们有了概形的同构
\[
	\varphi_{\alpha\beta}:p_\alpha^{-1}(X_{\alpha\beta})\to p_\beta^{-1}(X_{\alpha\beta}),
\]
所以我们通过等同$p_\alpha^{-1}(X_{\alpha\beta})$和$p_\beta^{-1}(X_{\alpha\beta})$可以将$X_\alpha\times_S Y$和$X_\beta\times_S Y$黏起来。更一般地,不难检查黏合条件(不同黏法黏出同一样东西)成立,这样将所有$\{X_\alpha\times_S Y\}$黏成的概形,可以检查就是$X\times_S Y$,典范态射是将所有$p_\alpha$黏合起来得到的。
\end{proof}

\begin{lem}\label{lem:3.2.11}
设$\varphi:X\to S$, $\psi:Y\to S$都是$S$-概形,而$U$是$S$的一个开子集,则$\varphi^{-1}(U)\to U$和$\psi^{-1}(U)\to U$的纤维积$\varphi^{-1}(U)\times_U \psi^{-1}(U)$就是$\varphi^{-1}(U)\times_S Y$或$X\times_S \psi^{-1}(U)$.
\end{lem}

\begin{proof}
实际上,我们只需要检查交换图
\[
\begin{xy}
	\xymatrix{
		Z\ar@/_/[rdd]_{g}\ar@/^/[drrr]^{f}\ar@{-->}[dr]^h&&&\\
		&A\ar[rr]_-{i_{\psi^{-1}(U)}p_2}\ar[d]^{p_1}&&Y\ar[d]^\psi\\
		&\varphi^{-1}(U)\ar[rr]^-{\varphi'}&&S
	}
\end{xy}
\]
其中$\varphi'$满足唯一分解$\varphi=i_{\varphi^{-1}(U)}\varphi'$. 由于$\psi f=\varphi' g$,所以$\im(\psi f)\subset U$,或者$\im f\subset \psi^{-1}(U)$,因此,我们可以唯一分解$f=i_{\psi^{-1}(U)}f'$,其中$f':Z\to \psi^{-1}(U)$. 现在考虑$A=\varphi^{-1}(U)\times_{U}\psi^{-1}(U)$的泛性质,存在唯一的态射$h:Z\to A$,使得$g=p_1h$且$f'=p_2h$,所以$f=i_{\psi^{-1}(U)}f'=i_{\psi^{-1}(U)}p_2h$. 这就检验了上述交换图。
\end{proof}

\begin{proof}[The proof of Theorem \ref{thm:1.5.9} ]
设$X$, $Y$都是$S$-概形,定义态射分别是$\varphi:X\to S$和$\psi:Y\to S$. 先设$S$, $Y$是仿射概形,找一个$X$的仿射开覆盖,由于两个仿射概形在仿射概形上的纤维积是存在的,所以Lemma \ref{lem:3.2.10} 告诉我们,此时$X\times_S Y$存在,即一个概形与一个仿射概形在仿射概形上的纤维积是存在的。现在回到$Y$是一般的概形,由于此时$X$与$Y$的任意仿射开集在$S$上的纤维积是存在的,再一次利用Lemma \ref{lem:3.2.10},就得到了此时$X\times_S Y$存在。

现在设$\{S_\alpha\}$是$S$的一个仿射开覆盖,Lemma \ref{lem:3.2.11} 告诉我们,$\varphi^{-1}(S_\alpha)\to S_\alpha$和$\psi^{-1}(S_\alpha)\to S_\alpha$的纤维积$A_\alpha=\varphi^{-1}(S_\alpha)\times_{S_\alpha}\psi^{-1}(S_\alpha)$就是$\varphi^{-1}(S_\alpha)\times_S Y$. 由于$S_\alpha$是仿射的,所以$A_\alpha=\varphi^{-1}(S_\alpha)\times_S Y$存在。由于$\bigcup_\alpha\varphi^{-1}(S_\alpha)=\varphi^{-1}\left(\bigcup_\alpha S_\alpha\right)=\varphi^{-1}(S)=X$,所以$\varphi^{-1}(S_\alpha)$是$X$的一个开覆盖,由Lemma \ref{lem:3.2.10},$X\times_S Y$存在。
\end{proof}

\begin{para}[纤维积下的几何点]
设$X$, $Y$都是$S$-概形,而$T$是任意概形,由纤维积的泛性质,我们有如下集合的典范等同
\[
	(X\times_S Y)(T)=X(T)\times_{S(T)}Y(T).
\]
这个等式实际上在任意的范畴都是成立的,其实这不过是泛性质的再表述而已。

现设$f:X\to S$和$g:Y\to S$是定义态射,然后取$T=k$是一个域。给定$\varphi:\spec k\to X$和$\psi:\spec k\to Y$满足$f\varphi=g\psi$,则由泛性质(或者说上面的集合典范等同),存在$\varphi\times_S\psi:\spec k\to X\times_S Y$使得分解
\[
	\varphi=p_X(\varphi\times_S\psi),\quad \psi=p_Y(\varphi\times_S\psi),
\]
其中$p_X$和$p_Y$都是典范投影。

回忆一些结论和记号习惯,给定概形间态射$\varphi:\spec k\to X$就是在说给定$\varphi(a)$和一个域扩张$k(\varphi(a))\to k$. 还有,对$S$-概形$X$,定义态射为$f:X\to S$,如果$f(x)=s$,则称$x\in X$位于$s\in S$上。

所以上面说了那么一大串,可以重新表为:给定位于同一点$s\in S$上的点$x\in X$和$y\in Y$,以及域扩张$k(x)\to k$和$k(y)\to k$,使得以下复合
\[
	k(s)\to k(x)\to k \quad \text{and} \quad k(s)\to k(y)\to k
\]
得到的是同一个域扩张$k(s)\to k$,则$(x,y)\in X\times_S Y$. 
\end{para}

我们可以将这个判别法拓展到有限纤维积上,这并没有增加任何困难。从这个小应用,我们可以大概对几何点有所感受,粗略地说,几何点$\spec k\to X$代表了$X$上的一个点。

下面的命题是上面这个判别法的直接应用,命题虽然没有提到域扩张或者几何点,但是技术上是利用了它们,将点看成一个态射。

\begin{pro}\label{pro.2.12}
	设$\{X_i\,:\,1\leq i\leq n\}$都是$S$-概形,对每一个指标$i$,设$x_i\in X_i$,则$(x_1$, $\dots$, $x_n)\in X_1\times_S \cdots\times_S X_n$当且仅当每一个$x_i$都位于同一个点$s\in S$上。
\end{pro}

\begin{proof}
设$X_i$的定义态射为$f_i$,而$X_1\times_S \cdots\times_S X_n$的定义态射为$f$. 如果$x=(x_1$, $\dots$, $x_n)\in X_1\times_S \cdots\times_S X_n$,则$f_i(x_i)=f(x)$,它们都位于同一点$f(x)$上。反过来,此时$f_i(x_i)=f_j(x_j)=s\in S$,考虑$f_i$自然诱导的域扩张$k(f_i(x_i))=k(s)\to k(x_i)$,从Lemma \ref{composite_ext},存在域扩张$k(s)\to k$和$k(x_i)\to k$使得分解
\[
	k(s)\to k(x_i)\to k
\]
对每一个指标$i$都成立。所以,上面那个判别法给出了$(x_1$, $\dots$, $x_n)\in X_1\times_S \cdots\times_S X_n$.
\end{proof}

作为推论,我们可以断言,典范连续映射$|X\times_S Y|\to |X|\times_{|S|}|Y|$是满的。但是,一般而言,这并不是单的,换而言之,$X\times_S Y$中可能存在不同的点但是有着相同的投影!比方说,我们考虑两个域扩张$k\to k_1$和$k\to k_2$,则$\spec(k_1)\times_{\spec k} \spec(k_2)=\spec(k_1\otimes_k k_2)$. 此时$k_1\otimes_k k_2$一般来说并不是一个域,它可能具有多个素理想,但是它们的投影必然是唯一的,即$|\spec(k_1)|\times_{|\spec k|} |\spec(k_2)|=|\spec(k_1)|\times |\spec(k_2)|$只有一个点。

同时,这也说明了,不是所有$X\times_S Y$中的点都可以写成$(x,y)$的形式。

\begin{pro}
	设$f:\spec A\to \spec B$是一个仿射概形间的态射,再令$p\in \spec B$为像$f(\spec A)$的Zariksi闭包
	中的一个点,则存在$q\in f(\spec A)$使得$p\in \overline{\{q\}}$(叫做$p$是$q$的特化,或$q$是$p$的一般化)。
\end{pro}

这个命题也可以翻译到纤维积上:任取$p\in \overline{f(\spec A)}$,他的局部概形$(\spec B)_p$与$\spec A$的纤维积
$\spec A\times_B (\spec B)_p$非空。注意到,连续函数部分就是说$S_p \cap f(\spec A)$非空,而
$S_p=\{q\in \spec B\,:\,p\in \overline{\{q\}}\}$.

一般来说,任取$\spec R$中的子集$X$,他的闭包中的点$p$并不可以找到$q\in X$使得$p\in \overline{\{q\}}$.

\begin{proof}
用$f(\spec A)$的Zariski闭包替代$\spec B$,我们可以假设$f(\spec A)$稠密。
此时,任取$p\in \spec B$,我们找包含于$p$的一个极小素理想
$q\subset p$,可以看到$p\in \overline{\{q\}}$. 几何角度来说,就是找了个$p$所在的不可约分支,$q$是其一般点。
从Zorn引理,我们知道这样的极小素理想存在,因为此时素理想递减链的交集也是素的。
下面验证$q\in f(\spec A)$,即找到$q'\in \spec A$使得$f(q')=q$.

我们的主要思路是考虑如下的交换图,
\[
\xymatrix{
	B\ar[r]\ar[d]& B_q\ar[d]\\
	A\ar[r]&A\otimes_B B_{q}
}\qquad 
\xymatrix{
	\spec B& \spec B_q\ar[l]\\
	\spec A\ar[u]&\spec A\times_B \spec B_{q}\ar[u]\ar[l]
}
\]
如果沿着$B\to B_q\to A\otimes_B B_{q}$,我们可以找到$A\otimes_B B_{q}$中的某个素理想$p'$一路原像
回到$B$的时候是$q$,则$p'$在另一条路径中$A$的原像就是我们要找的$q'$. 这里很重要的一点是,我们需要说明
$A\otimes_B B_{q}\neq 0$,就是他至少存在一个素理想。为此,我们首先看到$h:B\to A$是个单射,这是因为
$f(\spec A)$稠密,而他又包含于$V(\ker h)$,所以后者等于整个$\spec B$.
由于局部化保持单射,所以$B_q\to A\otimes_B B_{q}$是一个单射。

现在注意到,$B_q$只有一个素理想,就是$q$的局部化,他当然在局部化下的原像就是$q$.
最后对于$B_{q}\to A\otimes_B B_q$,任取$A\otimes_B B_q$中的素理想$p'$,他在域$B_q$中的原像作为素理想
只能是$q$的局部化。
\end{proof}

\begin{pro}\label{pro.2.13}
考虑如下两个命题:
\begin{compactenum}[~~~(1)]
\item 若$f$和$g$是两个$S$-态射,它们都具有性质$\mathsf{P}$,则$f\times_S g$也具有性质$\mathsf{P}$.
\item 若$f:X\to Y$是一个$S$-态射,具有性质$\mathsf{P}$,则对任意的$S$-概形$S'$,经过基概形扩张后,$f_{(S')}:X_{(S')}\to Y_{(S')}$也具有$\mathsf{P}$.
\end{compactenum}
我们成立如下论断:
\begin{compactenum}[~~(i)]
\item 如果对任意的$X$,$1_X$都具有性质$\mathsf{P}$,则(1)可以推出(2).
\item 如果任意两个具有性质$\mathsf{P}$的态射复合依然具有$\mathsf{P}$,则(2)可以推出(1).
\end{compactenum}
\end{pro}

\begin{proof}
因为$f_{(S')}=f\times_S 1_{(S')}$,因此(i)是自然的。设$f:X\to X'$和$g:Y\to Y'$都是$S$-态射,成立复合关系
\[
	f\times_S g=(1_{X}\times_S g)(f\times_S 1_Y),
\]
因此(ii)也是自然的。
\end{proof}

我们时常要关注一个态射在任意的基概形扩张下的表现,有些性质是良好的,在基概形扩张下表现良好,但有些性质就不那些好。比方说,如果一个态射是满的,则它在任意基概形扩张后依然是满的。但是,一个单的态射在基概形扩张后就不一定还是单的。对应地,我们有下面这样一个结论。

\begin{pro}\label{pro.2.14}
设$f:X\to Y$是一个概形间态射,以下命题等价:
\begin{compactenum}[~~~(a)]
\item 对任意的域$k$,映射$f_*:X(k)\to Y(k)$总是单的。
\item 对任意的代数闭域$k$,映射$f_*:X(k)\to Y(k)$总是单的。
\item $f$是单的,且在任意基概形扩张后就还是单的。
\item $f$是单的,且对每一个$x\in X$,典范诱导的域扩张$k(f(x))\to k(x)$都是纯不可分扩张。
\end{compactenum}
\end{pro}

\begin{proof}
具体见[EGA, Chap 1, 3.5],后面我们会用到$(a)\Rightarrow (b)\Rightarrow (c)$,所以只证明这部分,其中$(a)\Rightarrow (b)$是显然的。假设对任意的代数闭域$k$,映射$f_*:X(k)\to Y(k)$总是单的。

首先我们证明$f$是单的。取$x_1$, $x_2\in X$使得$f(x_1)=f(x_2)=y$. 利用Lemma \ref{composite_ext},可以找到代数闭域$k$以及相应扩张$i:k(y)\to k$, $j_1:k(x_1)\to k$, $j_2:k(x_2)\to k$使得分解成立
\[
	i:k(y)\to k(x_1)\xrightarrow{j_1} k,\quad i:k(y)\to k(x_2)\xrightarrow{j_2} k.
\]
由于扩张$j_i:k(x_i)\to k$等价于给出态射$u_i:\spec k\to X$且$\im u_i=\{x_i\}$,此时$i$的两个分解等价于$f_*(u_1)=f_*(u_2)$,由假设可知,$u_1=u_2$,所以$x_1=x_2$.

其次,设$f:X\to Y$, $g:W\to Z$是两个$S$-态射。如果对任意的代数闭域$k$,映射$f_*:X(k)\to Y(k)$和$g_*:W(k)\to Z(k)$总是单的,我们只需证明,
\[
	(f\times_S g)_*:(X\times_S W)(k)\to (Y\times_S Z)(k)
\]
也总是单的。这是因为,由前面已证明的部分,$f\times_S g$是单的,结合$1$总是单的,Proposition \ref{pro.2.13} 就给出了结论。

现在,注意到
\[
	(X\times_S W)(k)=X(k)\times_{S(k)}W(k),\quad (Y\times_S Z)(k)=Y(k)\times_{S(k)}Z(k),
\]
所以$(f\times_S g)_*$的作用就是
\[
	(f\times_S g)_*:(u,v)\mapsto (f_*(u),g_*(v)).
\]
显然是单的。
\end{proof}

下面我们考虑纤维,在此之前,我们首先有如下观察:设$y\in Y$是概形上的一个点,则$\spec (k(y))$有自然的$Y$-概形结构。实际上,首先商同态$\pi:(\oo_{Y})_y\to k(y)$给出了概形同态$\widetilde{\pi}:\spec(k(y))\to \spec ((\oo_{Y})_y)=Y_y$. 复合上典范同态$i_y:Y_y\to Y$,我们就得到了$i_y\widetilde{\pi}:\spec(k(y))\to Y$. 当然,我们也知道,给定这样一个态射,等价于给定$y$和域扩张$\id_{k(y)}:k(y)\to k(y)$. 显然,$i_y\widetilde{\pi}:\spec(k(y))\to Y$是一个单态射,因为$i_y:Y_y\to Y$是单态射,而$\pi$在交换环范畴是满态射,所以$\widetilde{\pi}$在概形范畴是单态射。我们记$j_y=i_y\widetilde{\pi}$.

\begin{pro}[纤维]\label{fiber}
设$f:X\to Y$是一个态射,而$y\in Y$是任意一点,则作为拓扑空间,纤维积$X\times_Y\spec (k(y))$同胚于$f^{-1}(y)$. 所以,我们可以将$X\times_Y\spec (k(y))$定义为概形意义上的纤维,同样记做$f^{-1}(y)$.
\end{pro}

之所以考虑$X\times_Y\spec (k(y))$而不是$X\times_Y Y_y$的理由很简单,因为$Y_y$有可能不止一个点,而$\spec (k(y))$只有一个点。同时,也因为这个原因,如果将$(\oo_Y)_y$中$\{\mm_y^n\}$取作一个拓扑基得到一个拓扑,选开理想$\mathfrak{a}$满足$\mathfrak{a}^n\to 0$(这被称为定义理想),则$\spec((\oo_Y)_y/\mathfrak{a})$其实也只有一个点,且$\spec((\oo_Y)_y/\mathfrak{a})$也同胚于$f^{-1}(y)$. 不过我们就不处理那么一般的情况了。

\begin{proof}
考虑基概形扩张$(j_y)_{(X)}:X\times_Y \spec (k(y))\to X\times_Y Y=X$,不难发现,他就是典范投影$p_X:X\times_Y \spec (k(y))\to X$. 由于$j_y$是单态射,所以对任意域$k$,它诱导了单映射$(j_y)_*:(\spec (k(y)))(k)\to Y(k)$,由Proposition \ref{pro.2.14},$j_y$在任意基概形扩张后依然是单的,故典范投影$p_X:X\times_Y \spec (k(y))\to X$是单的。

设$\spec(k(y))$的唯一点为$a$,取$z\in X\times_Y \spec (k(y))$,由于$fp_X=p_Yj_y$,所以$f(p_X(z))=j_y(p_Y(z))=j_y(a)=y$,故$\im p_X(z)\subset f^{-1}(y)$. 同时,利用Proposition \ref{pro.2.12},则$(x,a)\in X\times_Y\spec (k(y))$当且仅当$x$和$a$都位于同一点$y$上,即当且仅当$f(x)=y$. 所以典范投影$p_X:X\times_Y \spec (k(y))\to f^{-1}(y)$还是一个满射。

综上,$p_X:X\times_Y \spec (k(y))\to f^{-1}(y)$是一个双射。最后,我们只要检验同胚。由于所有形如$f^{-1}(y)\cap U$的开集构成了$f^{-1}(y)$的一组拓扑基,其中$U$是$X$中的一个仿射开集,则我们可以在$p_X^{-1}(f^{-1}(y)\cap U)=p_X^{-1}(U)=U \times_Y \spec (k(y))$上考虑问题,只要验证这是一个局部同胚,从$p_X$是一个双射,则可以得到$p_X$是一个同胚。

我们可以假设$Y$也是仿射的,事实上,选一个$y$的仿射邻域$V$,我们有
\[
	X\times_Y \spec (k(y))=(X\times_Y V)\times_V \spec (k(y))=f^{-1}(V)\times_V \spec (k(y)).
\]
此时,记$U=\spec R$而$Y=\spec S$,则
\[
	U \times_Y \spec (k(y))=\spec (R\otimes_S k(y)),
\]
其中$k(y)$是$S/\pp_y$的分式域。态射$p_X|_{U \times_Y \spec (k(y))}$此时对应于同态$\psi=1\otimes \pi:R=R\otimes_S S\to R\otimes_S k(y)$,其中$\pi:S\to k(y)$是商同态$S\to S/\pp_y$复合上含入$S/\pp_y\hookrightarrow k(y)$.

为证明局部同胚,我们下面试图利用Lemma \ref{lem:3.9}. 注意到任取$\sum_i r_i \otimes \pi(s_i)/\pi(t_i)\in R\otimes_S k(y)$,其中$r_i\in R$而$s_i$, $t_i\in S$,记$t=\prod_i t_i$以及$u_i=\prod_{j\neq i}t_j$,注意到此时
\[
	\sum_i r_i \otimes \pi(s_i)/\pi(t_i)=\sum_i r_is_i \otimes \pi(u_i)/\pi(t)=\sum_i r_is_iu_i \otimes 1/\pi(t).
\]
所以
\[
	\sum_i r_i \otimes \pi(s_i)/\pi(t_i)=\psi\left(\sum_i r_is_iu_i\right)\bigl(1\otimes \pi(t)\bigr)^{-1}.
\]
由Lemma \ref{lem:3.9},$p_X$就是局部同胚,因而也是同胚。
\end{proof}

\section{子概型}

对仿射概形$\spec R$,每一个$R$的理想$I$都可以定义一个$\spec R$上的$\oo_R$的理想层$\widetilde{I}$,这是一个拟凝聚理想层。反过来,任取$\spec R$的一个拟凝聚理想层,由Proposition \ref{niningju} 可知,它必然是$\widetilde{I}$的形式。因为这个原因,我们后面对概形也将主要关注拟凝聚理想层。但是,注意到,即使是仿射概形$\spec R$,并不是所有理想层都是拟凝聚的。

拟凝聚是局部的性质,所以利用Proposition \ref{niningju},我们给出概形$X$上模层$\calf$拟凝聚的如下定义:存在$X$的一个仿射开覆盖$\{U_\alpha\}$,使得在每一个$U_\alpha$上,都存在一个$\oo_X(U_\alpha)$-模$M_\alpha$使得$\calf|_{U_\alpha}=\widetilde{M_\alpha}$.

\begin{pro}
设$\mathscr{I}$是概形$X$的一个拟凝聚理想层,则$\oo_X/\mathscr{I}$的支集$Y$是闭集。
\end{pro}

\begin{proof}
设$Y$是$\oo_X/\mathscr{I}$的支集。注意到给定一个开覆盖$\{U_i\}$,$Y$是闭集当且仅当$U_i\cap Y$在每一个$U_i$中是闭集,所以我们只需在每一个仿射开集$U$中考察。

不妨记$U=\spec R$,则$I=\mathscr{I}(U)$是$R$的一个理想,由于$\mathscr{I}$是拟凝聚的,所以$\widetilde{I}=\mathscr{I}|_U$,于是任取$x\in U$,我们都有$\mathscr{I}_x=I_x$. 此时,$(\oo_X/\mathscr{I})_x=(\oo_X)_x/I_x=R_x/I_x$. 于是,$Y\cap U$是所有使得$R_x/I_x\neq 0$的$x\in U$的集合,即闭集$V(I)\subset \spec R$. 实际上,如果素理想$\pp_x$不包含$I$,则存在$a\in I$但$a\not\in \pp_x$,此时对$R_x/I_x=(R/I)_{\pp_x}$,因为$a$不在$\pp_x$中,它的像$\bar{a}=0$在$(R/I)_{\pp_x}$中是可逆的,所以$R_x/I_x$只可能为零。
\end{proof}

\begin{lem}
记号同上一个命题,设开集$U$, $V\subset X$满足$U\cap Y=V\cap Y$,则$\Gamma(U,\oo_X/\mathscr{I})=\Gamma(V,\oo_X/\mathscr{I})$.
\end{lem}

\begin{proof}
定义开集$W=X-Y$,不妨假设$V$是$U$的开子集。于是,我们有一个自然的限制态射
\[
	\rho^U_V:\Gamma(U,\oo_X/\mathscr{I})\to\Gamma(V,\oo_X/\mathscr{I}),
\]
首先它是单射。任取$s\in \Gamma(U,\oo_X/\mathscr{I})$,如果$s|_V=0$,则$s_x=0$对$x\in V$都成立,由于任取$x\in U-Y$都有$s_x=0$,以及$U-Y=U-Y\cap U=U-Y\cap V$,所以$s_x=0$对任意$x\in U$都成立,故$s=0$. $\rho^U_V$还是一个满射,注意,$W\cup V$是$U$的一个开覆盖,$0\in \Gamma(W,\oo_X/\mathscr{I})$和$s\in \Gamma(V,\oo_X/\mathscr{I})$在$V\cap W=V-Y$上都是零,所以可以拼出$W\cup V$上的截面,限制在$U$上就得到了$\rho^U_V$的一个原像。
\end{proof}

利用这个引理,我们可以在拓扑空间$Y$上定义一个结构层$\oo_Y$如下:任取$Y$的一个开集$V$,都存在$X$的一个开集$U$使得$V=Y\cap U$,定义$\Gamma(V,\oo_Y)=\Gamma(U,\oo_X/\mathscr{I})$,由上述引理,这个定义是良定的。此时,$(Y,\oo_Y)$也是一个概形。实际上,任取$x\in Y$,都存在仿射开集$U\subset X$包含它,而此时$U\cap Y$就是$(Y,\oo_Y)$的包含$x$的一个仿射开集。

\begin{lem}
设$\mathscr{I}$和$\mathscr{K}$是$X$的两个拟凝聚理想层,如果$\oo_X/\mathscr{I}$和$\oo_X/\mathscr{K}$的支集都是$Y$,且在$Y$上诱导了相同的概形$(Y,\oo_Y)$,则$\mathscr{I}=\mathscr{K}$.
\end{lem}

\begin{proof}
如果概形$Y$是由两个拟凝聚理想层$\mathscr{I}$和$\mathscr{K}$诱导的,则任取开集$U\subset X$,我们有
\[
	\Gamma(U,\oo_X/\mathscr{I})=\Gamma(U\cap Y,\oo_Y)=\Gamma(U,\oo_X/\mathscr{K})
\]
故$\oo_X/\mathscr{I}=\oo_X/\mathscr{K}$,也因此$\mathscr{I}=\mathscr{K}$.
\end{proof}

\begin{para}[子概形]
设$X$是一个概形,$U$是$X$的一个开集,则$U$自然地构成一个概形,此时$U$被称为$X$的一个开子概形。设$\mathscr{I}$是$X$的一个拟凝聚理想层,设$Y=\mathrm{Supp}(\oo_X/\mathscr{I})$. 我们从上面已经知道$Y$上面有一个概形结构,他被称为$X$的一个闭子概形。从上面的引理可知,概形$X$的闭子概形一一对应于它的拟凝聚理想层。一般地,如果概形$Z$是概形$X$的开子概形的闭子概形,则称$Z$是$X$的一个子概形。

在仿射概形$\spec R$,它的闭子概形一一对应于$R$的理想$\mathfrak a$,实际上,它的闭子概形就是闭子集$V(\mathfrak a)$,连同上面的层结构,可以将其等同于概形$\spec(R/\mathfrak a)$. 特别地,如果$\mathfrak a$是一个极大理想,则闭子概形$V(\mathfrak a)$是一个闭点$x$构成的集合,可以等同于概形$\spec(k(x))$.
\end{para}

\begin{para}[子概形的典范含入]
开子概形的典范含入我们已经非常清楚了,现在我们要对任意子概形进行定义。由于子概形都是开子概形的闭子概形,所以我们只要再对闭子概形定义出典范含入即可。

设$Y$是$X$的闭子概形,对应的拟凝聚理想层为$\mathscr I$,而作为拓扑空间的含入映射记作$\psi$. 我们可以定义自然的环层态射$\pi:\oo_X\to \psi_*\oo_Y$如下:任取$X$中的开集$U$,由于
\[
	\Gamma(U,\psi_*\oo_Y)=\Gamma(\psi^{-1}(U),\oo_Y)=\Gamma(U\cap Y,\oo_Y)=\Gamma(U,\oo_X/\mathscr I),
\]
所以$\psi_*\oo_Y=\oo_X/\mathscr I$. 再回忆 \ref{qsheaf},$\oo_X/\mathscr I$是典范含入$i(U):\mathscr I(U)\hookrightarrow \oo_X(U)$的余核,所以有典范态射$\pi:\oo_X\to \oo_X/\mathscr I$,此即所需。

所以,我们定义$(\psi,\pi):Y\to X$为闭子概形$Y$到$X$的典范含入。我们下面想要说明$(\psi,\pi)$是一个单态射,取$(\varphi_1,\theta_1)$, $(\varphi_2,\theta_2):Z\to Y$都是从概形$Z$到$Y$的一个态射,且$(\psi,\pi)(\varphi_1,\theta_1)=(\psi,\pi)(\varphi_1,\theta_1)$,换而言之,
\[
	\psi\varphi_1=\psi\varphi_2,\quad \psi_*(\theta_1)\pi=\psi_*(\theta_2)\pi,
\]
由于余核的典范态射都是满态射,而单射也是拓扑空间范畴的单态射,所以
\[
	\varphi_1=\varphi_2,\quad \psi_*(\theta_1)=\psi_*(\theta_2).
\]
再利用$\psi$是一个拓扑空间的含入映射(这意味着$Y$中的开集都可以写成某个$X$中开集在含入映射下的原像),$\psi_*(\theta_1)=\psi_*(\theta_2)$给出$\theta_1=\theta_2$. 所以闭子概形的典范含入态射是一个单态射。

很早以前就证明过,开子概形的典范含入也是单态射,见Corollary \ref{coro:1.21}. 由于子概形的典范含入是开子概形和闭子概形典范含入的复合,所以也是单态射。
\end{para}

\begin{para}[典范含入的局部刻画]
对于任意的子概形$(i,\pi):W\hookrightarrow X$,由于他是一个开子概形的闭子概形,所以存在一个开子概形$W'$以及一个上面的理想层$\mathscr I_W$使得$(\oo_W)_w=(\oo_{W'})_w/(\mathscr I_W)_w=(\oo_{X})_w/(\mathscr I_W)_w$,此时
\[
	\pi_w^\#:(\oo_X)_w\to (\oo_{X})_w/(\mathscr I_W)_w
\]
为商同态,显然这是一个局部同态。所以,典范含入是一个概形间的态射。
\end{para}

\begin{pro}[子概形的刻画]
$Y$是$X$的子概形,当且仅当以下两条成立:
\begin{compactenum}[~~~1.]
\item 拓扑上,$Y$是局部闭的,即可以写成$X$中一个开集与一个闭集的交。
\item 记$U=X-(\overline{Y}-Y):=X-\partial Y$,即$U$是$Y$边界\footnote{有些作者将边界定义为$\overline{Y}-Y^\circ$,这样$\partial Y$一定是一个闭集,但我们这里并不总是如此。}的补集,此时$Y$是$U$的闭子概形。
\end{compactenum}
\end{pro}

\begin{proof}
	设$Y$是一个子概形,如果他是开子概形$V$的闭子概形,那么拓扑上,我们应该有$\overline{Y}\cap V=Y$,所以$Y$是局部闭的。此外,集合上来看$U=X-(\overline{Y}-Y)$正是那些以$Y$为闭子集的开集中最大的那个,事实上,首先$U=V\cup (X-\overline{Y})$是一个开集,其次$Y=\overline{Y}\cap U$,故$Y$在$U$中是闭的。所以,如果$Y$是开子概形$V$的闭子概形,则$V\subset U$,利用典范含入,则$Y$也是开子概形$U$的闭子概形。
\end{proof}

\begin{pro}[子概形的局部刻画]\label{pro:3.3.8}
设$Y$是一个赋环空间,作为拓扑空间,$Y$是概形$X$的子空间。如果$X$中存在$Y$的一个开覆盖$\{U_\alpha\}$使得对每一个$U_\alpha$,$Y$的开子赋环空间$Y\cap U_\alpha$是$X$的开子概形$U_\alpha$的闭子概形,则$Y$是$X$的子概形。反之亦然。
\end{pro}

\begin{proof}
	因为$Y\cap U_\alpha$都是$U_\alpha$中的闭集,所以$Y$在$\{U_\alpha\}$的并中是闭的(闭的局部性质),即$Y$是一个局部闭的子空间。于是$U=X-\partial Y$是包含$Y$为闭子集的最大开集,可以$U$代替$X$,进而假设$Y$是一个闭集。
	
	由于$Y\cap U_\alpha$是$X$的开子概形$U_\alpha$的闭子概形,所以存在一个$U_\alpha$上的拟凝聚理想层$\mathscr I_\alpha$定义了闭子概形$Y\cap U_\alpha$. 我们试图将这些拟凝聚理想层粘起来得到$X$上的理想层$\mathscr I$,黏合条件是直接的,但是$\bigcup_\alpha U_\alpha$并不一定是$X$,所以还需补充那些与$Y$不交的开集上的表现。如果开集$V$与$Y$不交,定义$\mathscr I|_V=X|_V$. 这样,我们就在$X$上定义出了一个理想层$\mathscr I$,它由拟凝聚理想层拼成,而拟凝聚是局部性条件,所以$\mathscr I$也是拟凝聚的,它定义了一个$Y$的闭子概形结构。同时,因为在每个$Y$的局部与原赋环空间的层结构相同,所以两个赋环空间是相同的,即$Y$是一个闭子概形。

	反过来,假设$Y$是$X$的一个子概形,设$U$是那些以$Y$为闭子集的开集中最大的那个,则$Y$可以由$U$上的一个拟凝聚理想层$\mathscr I$定义。取$Y$的一个开覆盖$\{U_\alpha\}$,如果每个$U_\alpha$都是$U$的开子集,则此时$U_\alpha\cap Y$是$U_\alpha$中的闭子集,当然也是$Y$的开子集。作为$Y$的开子概形,$U_\alpha\cap Y$由$\mathscr I|_{U_\alpha}$定义,因此它也就是$U_\alpha$的一个闭子概形。
\end{proof}

我们下面要推广Proposition \ref{pro:1.20} 到子概形的典范含入上。Proposition \ref{pro:1.20} 告诉我们,如果$Y$是$X$的开子概形,且$Z\to X$拓扑空间的像在$Y$中,则存在分解$Z\to Y\to X$. %因此,从某种角度上来说,我们要定义概形态射的像。

\begin{para}
设$Y$是$X$的一个子概形,$f:Z\to X$是一个概形间态射,如果存在分解
\[
	f:Z\xrightarrow{g} Y\hookrightarrow X,
\]
则称$f$被概形$Y$的含入态射所遮盖。由于典范含入是单态射,所以对于给定的$f$和子概形$Y$,分解总是唯一的。
\end{para}

\begin{pro}\label{pro:3.3.11}
态射$(f,\theta):Z\to X$被含入$(i,\pi):Y\hookrightarrow X$所遮盖,当且仅当,$f(Z)\subset Y$且对任意的$z\in Z$,同态$\theta^\#_z:(\oo_{X})_{f(z)}\to (\oo_{Z})_z$总可以分解为
\[
	\theta^\#_z:(\oo_{X})_{f(z)}\xrightarrow{\pi^\#_{f(z)}} (\oo_Y)_{f(z)}\to (\oo_{Z})_z.
\]
\end{pro}

\begin{proof}
必要性显然。对于充分性,如果用$U=X-\partial Y$来代替$X$,我们可以假设$Y$是一个闭子概形,所以$Y$由一个拟凝聚理想层$\mathscr{I}$定义,局部地,典范含入诱导了
\[
	(\oo_{X})_{y}\to (\oo_Y)_{y}=(\oo_{X}/\mathscr{I})_y=(\oo_{X})_{y}/\mathscr{I}_y.
\]
令$f=(\psi,\theta)$,并设$\mathscr{J}$是$\theta^\#:\psi^*\oo_X\to \oo_Z$的核,他是$\psi^*\oo_X$的一个理想层。由假设,成立分解
\[
	\theta^\#_z:(\oo_X)_{f(z)}\to (\oo_{X})_{f(z)}/\mathscr{I}_{f(z)}\to (\oo_Z)_z,
\]
所以
\[
	(\psi^*\mathscr I)_z= \mathscr I_{f(z)}\subset \ker(\theta^\#_z)=\mathscr J_z,
\]
即$\psi^* \mathscr I\subset \mathscr J$. 进而,利用余核的泛性质,$\theta^\#$可以唯一分解为
\[
	\theta^\#:\psi^*(\oo_X)\to \psi^*(\oo_X)/\psi^*(\mathscr I)=\psi^*(\oo_X/\mathscr I)\xrightarrow{\omega} \oo_Z
\]
其中第一个态射是商的典范态射,等式是来自于左伴随函子$\psi^*$与余核可交换。最后,由于$\psi(Z)\subset Y$,所以连续映射$\psi:Z\to X$将给出连续映射$Z\to Y$,记作$\bar\psi$. 不难看到,此时$\bar \psi^*(\oo_Y)=\psi^*(\oo_X/\mathscr J)$,此外,$\omega:\bar \psi^*(\oo_Y)\to \oo_Z$显然是一个局部态射(这来自于商环的理想结构以及$\theta^\#_z$是一个局部同态),所以$(\bar\psi,\omega^\flat):Z\to Y$就是我们所需的分解,其中$\omega^\flat:\oo_Y\to \bar\psi_*\oo_Z$是$\bar\psi_*$和$\bar\psi^*$互为伴随下$\omega$的典范对应态射。
\end{proof}

\section{浸入}

以一个关于概形间单态射的常用引理开始。

\begin{lem}\label{lem:3.4.1}
	令$(\varphi,\theta):X\to Y$是赋环空间之间的态射,如果$\varphi$是单的且局部同态$\theta^\#_x:(\oo_Y)_{\varphi(x)}\to (\oo_X)_x$是满的,则$(\varphi,\theta)$是单态射。
\end{lem}

\begin{proof}
	令$(\psi_1,\eta_1)$, $(\psi_2,\eta_2):Z\to X$为两个赋环空间之间的态射,且$(\varphi,\theta)(\psi_1,\eta_1)=(\varphi,\theta)(\psi_2,\eta_2)$. 由于$\varphi$是单的,所以$\psi_1=\psi_2=\psi$. 此外,由Proposition \ref{pro:morcom},层之间的态射满足$(\eta_1)^\#_z\theta^\#_{\psi(z)}=(\eta_2)^\#_z\theta^\#_{\psi(z)}$,所以$(\eta_1)^\#_z=(\eta_2)^\#_z$对任意的$z\in Z$都成立,即$(\eta_1)^\#=(\eta_2)^\#$,故$\eta_1=\eta_2$. 
\end{proof}

上一节中定义的子概形从某种角度来说并不算一个特别好的概念,因为子概形的子概形不能简单地看作原概形的子概形。虽然,其可能确实同构于原概形的一个子概形。

类似于流形范畴,那里的子流形不仅仅要考虑一个流形,还有考虑流形之间的映射,即考虑浸入映射而不只是子流形。而我们这里类似,比起子概形,浸入是一个更容易操作的概念。

\begin{para}[浸入]
	设$X$是一个概形,概形态射$f:Y\to X$被称为一个(闭、开)浸入,如果它可以分解为$f:Y\xrightarrow{g}Z\xrightarrow{i} X$,其中$g$是一个同构,而$Z$是$X$的一个(闭、开)子概形,$i$是一个典范含入。
\end{para}

由于含入是单态射,如果浸入也是单态射。下一个引理告诉我们,浸入分解中出现的子概形和同构是由浸入本身唯一确定的。

\begin{lem}
	设$X$是一个概形,$i:Y\to X$和$i':Y'\to X$都是子概形的典范含入,如果存在分解$i':Y'\xrightarrow{g}Y\xrightarrow{i}X$,其中$g$是一个同构,则$Y'=Y$且$i=i'$.
\end{lem}

\begin{proof}
	首先,任取$x\in Y'$,则作为连续映射,我们有$x=i'(x)=i(g(x))=g(x)$,所以$g$作为连续映射只能为恒等映射。进而,只需将$X$换成包含$Y=Y'$为闭集的最大开集,我们可以进一步假设$Y$和$Y'$都是$X$的闭子概形,对应的拟凝聚理想层为$\mathscr I$和$\mathscr I'$. 我们下面证明$\mathscr I=\mathscr I'$. 这样$Y'$和$Y$是同一个闭子概形。

	任取$x\not\in |Y|=|Y'|$,有$\mathscr I'_x=(\oo_X)_x=\mathscr I_x$. 而取$x\in |Y|=|Y'|$,我们有分解$(i')^\#_x=g^\#_x i^\#_x$. 由闭子概形的构造,$i^\#_x:(\oo_X)_x\to (\oo_X)_x/\mathscr I_x$和$(i')^\#_x:(\oo_X)_x\to (\oo_X)_x/\mathscr I'_x$都是商同态。任取$s_x\in \mathscr I_x$,分解告诉我们$(i')^\#_x(s_x)=0$当且仅当$i^\#_x(s_x)=0$,所以$\mathscr I'_x = \mathscr I_x$处处成立,即$\mathscr I=\mathscr I'$. 
\end{proof}

如果浸入$f$存在两个分解$f=ig=i'g'$,其中$g$, $g'$为同构,而$i$, $i'$为含入,于是$i=i'g'g^{-1}$,由上一引理,$i=i'$. 此外,由于$ig=ig'$以及含入是一个单态射左可消,所以$g=g'$. 

\begin{lem}
设$X$是一个仿射概形,$Y$是$X$的一个闭子概形,而$Z$是$Y$的一个闭子概形,含入分别为$i$, $j$,则$ij:Z\to X$是一个闭浸入。
\end{lem}

\begin{proof}
	设$X=\spec R$,由于$Y$具有形式$V(\mathfrak a)$,同构于$\spec (R/\mathfrak a)$. 由于$Z$是$Y$的子概形,所以同构于$\spec (R/\mathfrak a)$的子概形$V(\mathfrak b)$,其中$\mathfrak b$是$R/\mathfrak a$的一个理想,也同构于$\spec ((R/\mathfrak a)/\mathfrak b)$. 此时,存在$X$的一个闭子概形$Z'=V(\mathfrak b')$,其中$\mathfrak b'$是$\mathfrak b$在$R$中的原像,使得$Z$同构于$Z'$,因为环$(R/\mathfrak a)/\mathfrak b$和$R/\mathfrak b'$显然同构。此时,注意到,$R\to (R/\mathfrak a)/\mathfrak b$可以分解为商同态$R\to R/\mathfrak b'$的复合与我们的同构的复合,所以$ij$是一个闭浸入。
\end{proof}

\begin{pro}
	设$X$是一个概形,$Y$是$X$的一个(闭、开)子概形,而$Z$是$Y$的一个(闭、开)子概形,含入分别为$i$, $j$,则$ij:Z\to X$是一个(闭、开)浸入。
\end{pro}

\begin{proof}
首先注意到,开子概形的开子概形还是开子概形,所以,两个开子概形的含入复合还是开子概形的含入。这就意味着,我们可以假设$Y$是$X$的闭子概形。并且,我们已经证明了如果$i$, $j$都是开子概形的含入,则$ij$是开浸入。

利用Proposition \ref{pro:3.3.8},我们可以将问题归结到仿射情况。找一个$X$中所有与$Y$相交非空的仿射开集构成的族$\{U_\alpha\}$,显然,$\{U_\alpha\cap Y\}$构成$Y$的一个开覆盖。由于$Z$是$Y$的一个子概形,找$Y$中包含$Z$为闭子集的最大开集$V$. 如果$U_\alpha\cap Y\subset V$中,则$U_\alpha\cap Z$是$U_\alpha\cap Y$的一个闭子概形,而$U_\alpha\cap Y$是$U_\alpha$的闭子概形。由仿射的情况,$(ij)|_{U_\alpha\cap Z}$是一个闭浸入,可以唯一分解为
\[
	(ij)|_{U_\alpha\cap Z}:U_\alpha\cap Z\xrightarrow{g_\alpha} Z'_\alpha \hookrightarrow U_\alpha,
\]
我们可以将概形$Z'_\alpha$黏成一个$|Z|$上的概形$Z'$,这是因为$Z'_\alpha$由$(ij)|_{U_\alpha\cap Z}$唯一确定。此时,上一引理给出,$Z'\cap U_\alpha=Z'_\alpha$是$U_\alpha$的闭子概形,Proposition \ref{pro:3.3.8} 给出$Z'$是$\bigcup_\alpha U_\alpha$的闭子概形,进而是$X$的子概形。类似地,那些同构则黏合成同构$g:Z\to Z'$使得分解$ij:Z\xrightarrow{g}Z'\hookrightarrow X$成立。
\end{proof}

如果通过上面命题证明中的同构$g$,将闭子概形的闭子概形典范等同于一个闭子概形,这就使得子概形的子概形是一个子概形,从而补救了原本子概形概念的不足。但下面我们不采用这种方式。

\begin{pro}
	两个(开、闭)浸入复合还是一个(开、闭)浸入。
\end{pro}

\begin{proof}
	设$p:X\to Y$,$q:Y\to Z$是两个(开、闭)浸入,分别可以分解为
	\[
		p:X\xrightarrow{g}P\hookrightarrow Y,\quad q:Y\xrightarrow{h}Q\hookrightarrow Z,
	\]
	则复合具有分解
	\[
		qp:X\xrightarrow{g}P\hookrightarrow Y\xrightarrow{h}Q\hookrightarrow Z,
	\]
	所以,我们只需研究$P\xrightarrow{i} Y\xrightarrow{h}Q$,其中$i$是一个典范含入,而$h$是一个同构。利用同构$h$,$P$同构于$Q$的一个子概形$h(P)$,使得$hi$可以分解为同构$h_P:P\to h(P)$与含入$j:h(P)\hookrightarrow Q$的复合。所以,分解有
	\[
		qp:X\xrightarrow{g}P\xrightarrow{h_P} h(P)\hookrightarrow Q\hookrightarrow Z,
	\]
	由于两个(开、闭)子概形含入复合是一个(开、闭)浸入,所以$h(P)\hookrightarrow Q\hookrightarrow Z$可以分解为$h(P)\xrightarrow{k}O\hookrightarrow Z$,所以分解
	\[
		qp:X\xrightarrow{kh_Pg}O\hookrightarrow Z
	\]
	告诉我们$qp$也是一个(开、闭)浸入。
\end{proof}

下面我们用浸入的语言重新定义子概形。

\begin{para}[子概形]
	设$i:Y\to X$和$j:Z\to X$是两个浸入,如果存在同构$g:Y\to Z$使得$i=jg$成立,则称浸入$i$和浸入$j$等价。显然,这是一个等价关系。现在,我们定义,(闭、开)子概形是(闭、开)浸入的等价类。而我们原本定义的子概形被称为典范子概形,对应典范含入。从浸入分解的唯一性,一个等价类唯一确定了一个典范子概形。

	虽然典范子概形的典范子概形不一定是一个典范子概形,但上一个命题告诉我们,子概形的子概形还是一个子概形。所以,这里定义的子概形的概念要远比典范子概形要自然。
\end{para}

\begin{para}[子概形的偏序关系]
	一个概形的子概形之间存在一个偏序关系:设$Y$和$Z$都是$X$的子概形,如果存在浸入$p:Z\to Y$,则我们定义$Z\leq Y$. 为验证这是一个合理的定义,我们需要如下引理。
\end{para}

\begin{lem}
如果$Y$和$Z$都是$X$的子概形,则含入$i:Z\hookrightarrow X$被含入$j:Y\hookrightarrow X$所遮盖当且仅当存在浸入$p:Z\to Y$.
\end{lem}

\begin{proof}
从Proposition \ref{pro:3.3.11} 显然。
\end{proof}

本节的最后,我们可以(开、闭)浸入的局部刻画。其中,开浸入的比较简单,而(闭)浸入的刻画则复杂得多,但很神奇的是,它几乎就是Lemma \ref{lem:3.4.1} 的条件。

\begin{pro}
态射$f=(\psi,\theta):Y\to X$是一个开浸入当且仅当$\psi$是$Y$到$X$的某个开集的同胚,且$\theta^\#_y:(\oo_X)_{\psi(y)}\to (\oo_Y)_y$对所有的$y\in Y$都是同构。
\end{pro}

\begin{proof}
必要性显然。充分性,条件将给出$\theta^\#:\psi^*\oo_X\to \oo_Y$是一个层同构。此外,$\psi$可以分解为$\psi=i\bar\psi$,其中$i$是开子概形$U$的典范含入而$\bar\psi:Y\to U$是一个同胚,所以$\psi^*\oo_X=\bar\psi^* i^*\oo_X=\bar\psi^*(\oo_X|_U)$,所层同构$\theta^\#:\bar\psi^*(\oo_X|_U)\to \oo_Y$将给出层同构$\oo_X|_U\to \bar\psi_*\oo_Y$,连同同胚$\bar\psi$这就给出了一个$Y$到$\oo_X|_U$的概形同构。
\end{proof}

\begin{pro}\label{pro:3.4.11}
态射$f=(\psi,\theta):Y\to X$是一个浸入(闭浸入)当且仅当$\psi$是$Y$到$X$的某个局部闭集(闭集)的同胚,且$\theta^\#_y:(\oo_X)_{\psi(y)}\to (\oo_Y)_y$对所有的$y\in Y$都是满同态。
\end{pro}

\begin{proof}
由于子概形都是开子概形的闭子概形,从开浸入的命题,我们只需证明闭浸入的情况。对于闭浸入的情况,必要性是显然的。为证明充分性,我们首先假设命题对$X$是仿射概形而$Z=\psi(Y)$在$X$中是闭的情况是对的。

那么,对于一般情况,设$U$是$X$的一个仿射开集,使得$U\cap \psi(Y)$在$U$中是非空闭集,将$f$限制到概形$\psi^{-1}(U)$上,从特殊情况可知$f$的限制$f_U:\psi^{-1}(U)\to U$是一个闭浸入。所以,可以分解为$f_U=j_Ug_U$,其中$g_U:\psi^{-1}(U)\to Z_U$是一个同构,而$j_U:Z_U\hookrightarrow U$是一个典范含入。现在,我们想要把局部地拼成整体的,所以需要检查黏合条件。取另一个仿射开集$V$,不失一般性,可以假设$V\subset U$,则$f_V$是一个闭浸入,存在分解$f_V=j_Vg_V$. 同时,由于$f_V$是$f_U$在$\psi^{-1}(V)$上的限制,设$g_U(\psi^{-1}(V))=Z'_V$,则
\[
	f_V=j_U|_{Z'_V}g_U|_{\psi^{-1}(V)},
\]
这里$Z'_V$当然是$Z_U$的一个开集,是闭子概形的开子概形,所以$j_U|_{Z'_V}$是一个浸入,可以分解为$j_U|_{Z'_V}=ih$,所以
\[
	f_V=ihg_U|_{\psi^{-1}(V)},
\]
其中$i$是一个子概形的典范含入,从浸入分解的唯一性,$i=j_V$,再利用一次浸入分解的唯一性,$j_V$或子概形$Z_V$由$j_U|_{Z'_V}$唯一确定。类似地,$g_V$也由$j_U|_{Z'_V}$唯一确定,这些就是黏合条件。此时,将$Z_U$以及$g_U$黏合起来,我们有闭子概形$j:Z\hookrightarrow X$以及同构$g:Y\to Z$使得分解$f=jg$成立,所以$f$是一个闭浸入。

最后,回到特殊情况的证明。首先,$X$是仿射概形而$Z=\psi(Y)$在$X$中是闭的。注意$\psi$是到一个$Y$到$X$中闭集$Z$的同胚,所以$(\psi_*^{\oo_Y})_{\psi(x)}:(\psi_*\oo_Y)_{\psi(y)}\to (\oo_Y)_y$对任意的$y\in Y$都是同构,但对$x\not\in \bar Z=Z$,有$(\psi_*\oo_Y)_{x}=\{0\}$. 实际上,取$x$的一个邻域$U$与$Z$不交,则$\psi_*\oo_Y(U)=\oo_Y(\psi^{-1}(U))=\oo_Y(\varnothing)=\{0\}$. 所以,$\psi_* \oo_Y$的支集就是$Z$,我们下面证明$\psi_*\oo_Y$是一个拟凝聚$\oo_X$-模层,于是$Z$上就有一个$\psi_*\oo_Y$定义的闭子概形结构,而他通过$f$与$Y$同构,这就完成了证明。

取$x\in Z$,记$y=\psi^{-1}(x)$. 设$V$是$y$在$Y$中的一个仿射开邻域,$\psi(V)$在$Z$中也是开的,故存在$X$的开集$U$使得$\psi(V)=U\cap Z$,且使得$(\psi_*\oo_Y)|_U=((\psi|_V)_*(\oo_Y|_V))|_U$. 现在$f$在$(V,\oo_Y|_V)$上的限制是一个仿射概形到仿射概形$X$的态射,必然具有形式$({}^a\varphi,\tilde\varphi)$,其中$\varphi:\Gamma(X,\oo_X)\to \Gamma(V,\oo_Y)$是一个环同态,所以$(\psi|_V)_*(\oo_Y|_V)$是一个拟凝聚$\oo_X$-模层。而拟凝聚性是局部的,所以$\psi_*\oo_Y$是一个拟凝聚$\oo_X$-模层。同时,由于$\theta^\#_y$是满的,所以我们也知道$\mathcal O_X\to f_*(\mathcal O_Y)$是满的。再$X=\spec R$是仿射概形,所以$f_*(\mathcal O_Y)$同构于$\mathcal O_X/\mathcal I$,其中$\mathcal I$是$R$的一个理想$I$给出的,所以$\theta$是由这个同构复合上典范的商同态给出的。
\end{proof}

\begin{coro}\label{coro:3.4.12}
设$f:X\to Y$是一个概形态射,$\{V_\lambda\}$是$Y$的一个开覆盖,则$f$是一个闭浸入的充分必要条件是,$f$在每一个开子概形$f^{-1}(V_\lambda)$上的限制都是到$V_\lambda$中的闭浸入。
\end{coro}

\begin{proof}
层态射的局部条件此时是显然的,再注意到,闭集也是局部性质,所以命题直接得证。
\end{proof}

\begin{coro}
设$f:X\to Y$是一个概形态射,$\{V_\lambda\}$是$f(X)$的一个开覆盖,则$f$是一个(开)浸入的充分必要条件是,$f$在每一个开子概形$f^{-1}(V_\lambda)$上的限制都是到$V_\lambda$中的(开)浸入。
\end{coro}

\begin{proof}
层态射的局部条件依然是显然的,拓扑上,从局部到整体,我们可以得到$f(X)$在$\bigcup_\lambda V_\lambda$中是局部闭的(开的),进而在$Y$中也是局部闭的(开的)。
\end{proof}

\begin{pro}
设$\alpha:X'\to X$和$\beta:Y'\to Y$都是$S$-概形态射,如果它们都是(开、闭)浸入,则$\alpha\times_S\beta$也是。特别地,如果$\alpha$, $\beta$都是子概形的典范含入,则$\alpha\times_S \beta$将拓扑空间$X'\times_S Y'$等同于$\pi_X^{-1}(X')\cap \pi_Y^{-1}(Y')$,这里$\pi_X$, $\pi_Y$是$X\times_S Y$的典范投影。
\end{pro}

由于恒同总是一个开、闭浸入,则上述命题告诉我们,如果$S$-概形态射$f:X\to Y$是一个(开、闭)浸入,则对任何基扩展$S'\to S$,$f_{(S')}$也是(开、闭)浸入。

\begin{proof}
从浸入的定义,我们可以考虑$\alpha$和$\beta$都是子概形典范含入的情况。此时,如果都是开子概形,则Corollary \ref{coro:3.2.8} 立刻给出了结论。因为子概形都是开子概形的闭子概形,所以我们只需证明$X'$和$Y'$都是闭子概形的情况。

利用浸入是单态射,浸入的局部检验,经过一些简单的检查,我们可以直接将问题约化到$X$, $Y$, $S$都是仿射概形的情况。设$X$, $Y$, $S$的环分别为$B$, $C$, $A$,由于$X'$和$Y'$都是闭子概形,设它们分别对应于$B$和$C$的商代数$B'$和$C'$,此时$\alpha=\widetilde{\rho}$, $\beta=\widetilde{\sigma}$,其中$\rho$, $\sigma$都是典范商同态。于是,$X\times_S Y=\spec(B\otimes_A C)$以及$X'\times_S Y'=\spec(B'\otimes_A C')$,而
\[
	\alpha\times_S\beta=\widetilde{\rho\otimes_A\sigma}.
\]
由于$\rho\otimes_A\sigma$是满的,所以$\alpha\times_S\beta$是一个闭浸入。此外,$\ker(\rho\otimes \sigma)=u(\ker \rho)+v(\ker \sigma)$,这里$u$, $v$分别是$B$, $C$到$B\otimes C$的典范同态,而对应的$\widetilde{u}$, $\widetilde{v}$就是典范投影。因此,我们有拓扑空间的等式
\[
	V(\ker(\rho\otimes \sigma))=V(u(\ker\rho))\cap V(v(\ker\sigma))=\widetilde{u}^{-1}\bigl(V(\ker\rho)\bigr)\cap \widetilde{v}^{-1}\bigl(V(\ker \sigma)\bigr),
\]
第一项是对应于闭浸入$\alpha\times_S \beta$的闭子概形,
此即所证。
\end{proof}

\begin{pro}[子概形的原像]
设$f:X\to Y$是一个概形间态射,$i:Y'\hookrightarrow Y$的(闭、开)子概形,则投影$p_X:X\times_Y Y'\to X$是一个(闭、开)浸入,它唯一确定了$X$的一个(闭、开)典范子概形$X'$,$X'$作为拓扑空间同胚于$f^{-1}(Y')$.
\end{pro}

粗略地讲,子概形的原像也是子概形。特别地,考虑闭点$y\in Y$,它本身可以看成一个闭子概形$\spec (k(y))$,它关于$f$的原像正是我们在Proposition \ref{fiber} 引入的纤维$f^{-1}(y)$,它可以看成$X$的一个闭子概形。

\begin{proof}
如果把$X$和$X\times_Y Y$等同,不难看到$p_X=1_X\times_Y i$,则上一个命题直接给出了结论。
\end{proof}

\begin{pro}
	设$f:X\to Y$是一个概形间态射,$Y'$是$Y$的一个闭子概形,对应的拟凝聚理想层为$\mathscr I$,则$X$的闭子概形$f^{-1}(Y')$由拟凝聚理想层$(f^*\mathscr I)\oo_X$定义。
\end{pro}

\begin{proof}
	对$X$和$Y$都是局部的,所以考虑仿射的情况即可。此时,利用上面的命题,纤维积在代数范畴变成张量积,所以,注意到,如果$A$是一个$B$-代数,且$I$是$B$的一个理想,则$A\otimes_B (B/I)=A/(IA)$.
\end{proof}

\begin{para}[闭像与闭包]
	设$f:X\to Y$是概形态射,$f$的闭像(也叫概形学式的像)是$Y$的一个闭子概形$Z$,使得如果$f$能被$Y$的闭子概形$Z'\hookrightarrow Y$所遮盖,则$Z$是$Z'$的闭子概形。换句话说,能遮盖$f$的最小闭子概形,我们将其记作$\im(f)$. 特别地,考虑子概形的典范含入$i:U\hookrightarrow Y$,则$i$的闭像被称为$U$的闭包(也叫概形学式的闭包)。
\end{para}

\begin{pro}
	概形态射$f:X\to Y$的闭像是存在且唯一。
\end{pro}

\begin{proof}
	唯一性显然。现在,考虑能将$f$遮盖的$Y$的所有闭子概形$Z\hookrightarrow Y$对应的拟凝聚理想层的和$\mathscr I_f=\sum_Z \mathscr I_Z$,这显然是一个拟凝聚理想层,因为拟凝聚层的直和和商都是拟凝聚的,这个拟凝聚理想层对应的闭子概形就是$f$的闭像。
\end{proof}

容易看到$\im(f)$的底空间包含$f(X)$的Zariksi闭包,但一般来说,它们不相等。

\begin{pro}
若$f:X\to Y$是拟紧态射,则$f^\flat:\mathcal O_Y\to f_*\mathcal O_X$的核是一个拟凝聚理想层,它定义了$f$的闭像,且$\im(f)$的底空间等于$f(X)$的Zariksi闭包。
\end{pro}

\begin{proof}
	拟紧态射的定义见下一章。由于是局部的,所以我们可以设$Y$是仿射概形,
	进而$X$是拟紧的,所以我们可以选一个有限仿射开覆盖$X=\bigcup_{i=1}^n U_i$.
	此时,存在嵌入$\mathcal O_X\hookrightarrow \bigoplus_i(i_{U_i})_*\mathcal O_{U_i}$,
	复合上$f_*$给出
	$f_*\mathcal O_X\hookrightarrow \bigoplus_i(fi_{U_i})_*\mathcal O_{U_i}$,
	由于$fi_{U_i}$是仿射概形间态射,所以$(fi_{U_i})_*\mathcal O_{U_i}$是拟凝聚的,
	进而$\bigoplus_i(fi_{U_i})_*\mathcal O_{U_i}$也是,所以
	$\ker f^\flat$是拟凝聚$\mathcal O_Y$-模态射$\mathcal O_Y\to \bigoplus_i(fi_{U_i})_*\mathcal O_{U_i}$的核,也是拟凝聚的。

	容易看到,$\ker f^\flat$已经是最大的拟凝聚理想层了,所以给出了最小的闭子概形。
	而且从支集的定义可知,$\operatorname{Supp}(\mathcal O_Y/\ker f^\flat)=\overline{f(X)}$,
	故$\im(f)$的底空间就是$\overline{f(X)}$.
\end{proof}


\section{约态概形}

\begin{pro}
	设$X$是一个概形,$\mathcal A$是一个拟凝聚$\oo_X$-代数层,则存在唯一一个拟凝聚$\oo_X$-模层$\mathcal N$使得任取$x\in X$,$\mathcal N_x$都是$\mathcal A_x$的幂零根。如果$X$是仿射概形,则$\mathcal A=\widetilde{A}$,其中$A$是一个$\Gamma(X)$-代数,此时$\mathcal N=\widetilde{N}$,其中$N$是$A$的幂零根。
\end{pro}

\begin{proof}
	显然是局部命题,所以我们只需证明仿射概形情形即可。已知$\widetilde{N}$是一个拟凝聚$\oo_X$-模层,再从显然的唯一性,只需检验$\widetilde{N}_x=N_x$就是$A_x$的幂零根即可,即验证局部化和取幂零根可交换。记$A_x$的幂零根为$\sqrt{0}$,则显然$N_x\subset \sqrt{0}$. 此外,任取$r/s\in \sqrt{0}$,存在一个$n$使得$r^n/s^n=0\in A_x$,换言之,存在一个$t\not\in \pp_x$使得$tr^n=0\in A$,于是$(tr)^n=0$继而$r/s=tr/(ts)\in N_x$,这就给出反方向的包含。
\end{proof}

\begin{para}[幂零根层]
	特别地,如果$X$是概形,则$\oo_X$是一个拟凝聚$\oo_X$-代数层,所以上面的命题告诉我们有一个拟凝聚理想层$\mathcal N$,他被称为$X$的幂零根层,我们将其记作$\mathcal N_X$.
\end{para}

\begin{para}[约态概形]
	设$X$是一个概形,如果任取$x$,$(\oo_X)_x$都是约态环,即幂零根为零,则称$X$是一个约态概形。
	如果还是不可约的,则称其为一个整概形。
\end{para}

显然,约态概形是一个局部概念,所以我们只需考察仿射概形的情况。设$X=\spec R$,则所有的$R_x$都是约态的当且仅当$R$是约态的,所以仿射概形约态当且仅当其坐标环是约态环。

对每一个概形$X$,幂零根层$\mathcal N_X$是拟凝聚理想层,对应一个$X$的闭子概形$X_{\text{red}}$,从定义,不难看到$X_{\text{red}}$是一个约态概形,他被称为概形$X$的伴随约态概形。此外,$X_{\text{red}}$的底空间还是$X$,这是因为非零环模掉幂零根后不为零。再从幂零根是所有素理想的交这个刻画,不难看到$X_{\text{red}}$是所有以$X$为底空间的$X$的子概形中最小的那个。

\begin{pro}
	一个非空概形$X$是约态的(整的)当且仅当任取开集$U$,环$\Gamma(U,\oo_X)$都是约态的(整的)。
	对整概形$X$,任取$x$,$(\oo_X)_x$都是整环。
\end{pro}

但是,如果从$(\oo_X)_x$都是整环推出$X$是整概形,这个方向是不对的。

\begin{proof}
	任取$f\in (\oo_X)_x$,如果他是幂零的,则取其一个代表元,$(U,\bar f)$,显然,适当缩小$U$,
	可以看到$\bar f$也是幂零的。反过来,如果$f\in \Gamma(U,\oo_X)$是幂零的,则其局部化后依然是幂零的。注意到任何不可约空间的开子集也是不可约的,所以对整的命题,只需证明$\Gamma(X,\oo_X)$是一个整环。设$f$, $g\in \Gamma(X,\oo_X)$,如果$fg=0$,则$X$为两个闭集$V(f)$和$V(g)$的并。由不可约性,我们可以假设$X=V(f)$. 我们要证明$f=0$,这是局部的,所以可以假设$X$是仿射概形。由于$f$在所有的素理想中,也即$\sqrt{0}$中,但从约态性,$\sqrt{0}=\{0\}$,所以$f=0$. 反之,
	我们要证明如果环$\Gamma(U,\oo_X)$都是整的,则$X$是不可约的。考虑两个非空开集$U_1$,$U_2\subset X$,如果他们相交为空(他们的补就构成了交为整个开集的两个非空闭集),则$\Gamma(U_1\cup U_2,\oo_X)=\Gamma(U_1,\oo_X)\times \Gamma(U_2,\oo_X)$. 右边显然具有非零零因子,比如$(0,a)(b,0)=0$,所以矛盾。
\end{proof}

一个仿射概形是整的当且仅当其坐标环是整环。

\begin{para}[态射的约态化、函子性]
	设$f=(\psi,\theta):X\to Y$是一个概形态射,则$\theta^\#_x:(\oo_Y)_{f(x)}\to (\oo_X)_x$是一个局部同态。此外,由于环同态将幂零元映成幂零元,所以$\theta^\#_x$自然诱导了一个环同态$\omega_x:(\oo_Y)_{f(x)}/(\mathcal N_Y)_{f(x)}\to (\oo_X)_{x}/(\mathcal N_X)_{x}$,不难检验这还是一个局部同态,继而给出态射
	\[
		\omega:\psi^*(\oo_Y/\mathcal N_Y)\to \oo_X/\mathcal N_X,
	\]
	此时$(\psi,\omega^\flat):X_{\text{red}}\to Y_{\text{red}}$就是一个概形态射。我们将其记作$f_{\text{red}}$,称为$f$的伴随约态态射。显然,$(gf)_{\text{red}}=g_{\text{red}}f_{\text{red}}$,所以$X\mapsto X_{\text{red}}$连同$f\mapsto f_{\text{red}}$构成了一个协变函子。
\end{para}


