\chapter{层}
\ThisULCornerWallPaper{1}{../Pictures/7.png}

我们假设本文所出现的环都是含幺交换环,即有一个乘法单位元的交换环,并且,我们所谓的环同态,他将单位元映到单位元。特别地,我们约定,零环即$R=\{0\}$,其中乘法单位元和加法单位元相等,即$0=1$。在环(即我们这里所谓的环)范畴中,零环的地位大致相当于集合范畴中的空集。所以下面我们在提到环的时候,通常默认该环不是零环。此外,我们下面谈及的模也都是含幺交换环上的模,所以不分左右双边。

\section{层与赋环空间}

\begin{para}
一个拓扑空间$X$能被看成一个范畴,对象取作他的所有开集,而态射取作
\[
	X(U,V)=\begin{cases}
	\bigl\{i_{VU}:U\hookrightarrow V\bigr\}&\text{, if }U\subset V\text{;}\\
	\varnothing&\text{, otherwise}.
	\end{cases}
\]

% 现在,任取$X$的一个开覆盖$\mathscr{U}=\{U_i\,:\, i\in I\}$,为后续的方便起见,我们记$\overline{\mathscr{U}}$为在$\mathscr{U}$中添入所有$\mathscr{U}$中元素的有限交后得到的新的开覆盖,这使得开覆盖是有限交封闭的。显然,依然通过
% \[
% 	\overline{\mathscr{U}}(U,V)=\begin{cases}
% 	\bigl\{i_{VU}:U\hookrightarrow V\bigr\}&\text{, if }U\subset V\text{;}\\
% 	\varnothing&\text{, otherwise}.
% 	\end{cases}
% \]
% $\overline{\mathscr{U}}$也是一个范畴。再方便起见,我们可以假设开覆盖是有限交封闭的。

对于任意的范畴$K$,一个拓扑空间$X$,我们称呼一个取值在$K$中的\idx{预层}$\calf$(简称为$K$-预层)是一个$X\to K$的反变函子。特别地,设$V\subset U$是两个开集,记态射$\calf(i_{UV}):\calf(U)\to \calf(V)$为$\rho^U_{V}$,称为限制态射。如果$K$是集合范畴的子范畴,则$\calf(U)$中的元素$s$被称为$U$上的一个\idx{截面},而$s|_V:=\rho^U_V(s)$被称为截面$s$在$V$上的限制。

设$\calf$是一个$X$上的$K$-预层,$A\in K$是范畴$K$里面的一个对象,而$\calf$是$X$上的一个预层,$U$是$X$的任意开子集, 而$\mathscr{U}=\{U_\alpha\}_{\alpha\in I}$是$U$的任意开覆盖,如果存在态射族$\{\varphi_\alpha:A\to \calf(U_\alpha)\}_{\alpha\in I}$使得下图交换(画图时省去了限制映射)
\[
	\xymatrix{
		A\ar[r]^{\varphi_{\beta}}\ar[d]_{\varphi_{\alpha}}&\calf(U_\beta)\ar[d]\\
		\calf(U_\alpha)\ar[r]&\calf(U_{\alpha}\cap U_{\beta})
	}
\]
则存在唯一的态射$\varphi:A\to \calf(U)$使得下图交换的时候,
\[
	\xymatrix{
		A\ar@/_/[ddr]_{\varphi_{\alpha}}\ar@/^/[drr]^{\varphi_{\beta}}\ar@{-->}[dr]|{\varphi}&& \\
		&\calf(U)\ar[r]\ar[d]&\calf(U_\beta)\ar[d]\\
		&\calf(U_\alpha)\ar[r]&\calf(U_{\alpha}\cap U_{\beta})
	}
\]
则称预层$\calf$为一个\idx{层}。如果需要明确取值的范畴是$K$,则称为$K$-层。这个条件被称为层公理,他告诉我们,$\mathcal F(U)$是一类开覆盖上特殊的图$U_\alpha \supset U_\alpha\cap U_\beta \subset U_\beta$的极限,进一步地,我们可以断言
\[
	{\varprojlim}_{V\subset U}\calf(V)=\calf(U).
\]
这里图更多的了,但我们可以先利用$\calf(U)$是特殊的图的极限,然后验证其他的态射也满足。

如果$K$是集合范畴或者交换群范畴,则它可以容纳任意的积,很容易发现上面关于层的定义和我们熟知的层的定义是等价的,即局部相容的截面可以唯一地拼成一个整体的截面\footnote{具体来说,就是给定$U$的一个开覆盖$\{U_\alpha\}$,如果存在一族$\{s_\alpha\in U_\alpha\}$使得$s_\alpha|_{U_\alpha\cap U_\beta}=s_\beta|_{U_\alpha\cap U_\beta}$,则存在唯一的$s\in \calf(U)$使得$s|_{U_\alpha}=s_\alpha$.}. 这点只要将$A$取作$\prod_{\alpha\in I}\calf(U_\alpha)$即可。
\end{para}

\begin{para}
两个层的例子:流形$M$上,每一个开集$U$上的光滑函数的全体构成一个环,记作$\calf(U)$,则这是一个预层,$U$上的截面就是$U$上的光滑函数。这显然是一个层,因为光滑性是局部概念,如果两个光滑函数限制在同一个开集上是相同的,那么我们自然可以拼成一个唯一的更大的光滑函数,一族也是类似的。

另一个例子来自代数簇,设$X$是一个代数簇,他的每一个开集$U$都是一个子代数簇,自然有相应的正则函数环$\calf(U)$,则这是一个预层,$U$上的截面就是$U$上的正则函数。这依然是一个层,因为正则性还是局部概念,可以拼接的理由同上。
\end{para}

\begin{para}
所有$X$上的预层构成一个范畴,态射就是所谓的“函子间态射”,或者叫做\idx{自然变换}:设$\calf$和$\calg$是$X$上的两个预层,设$\varphi(U)$是使得如下交换图交换的一族态射,
\[
	\xymatrix{
		\calf(U)\ar[rr]^{\varphi(U)} \ar[d]_{\rho^U_V}&&\calg(U) \ar[d]^{\pi^U_V}\\
		\calf(V)\ar[rr]^{\varphi(V)}&&\calg(V)
	}
\]
则称呼$\varphi$是$\calf$到$\calg$的一个态射,记做$\varphi:\calf\to\calg$. 层之间的态射就是作为预层的态射。很自然地,如果已知一个预层态射$\varphi:\calf\to\calg$,即使$\calf$是一个层,也不能断言$\calg$是一个层。
\end{para}

\begin{para}[截面函子]
设$\calf$是$X$上的一个预层,$U\subset X$是一个开集,定义$\Gamma(U,\calf)=\calf(U)$. 特别地,如果$U=X$,则记$\Gamma(\calf)=\Gamma(U,\calf)$. 现考虑$X$上的另一个预层$\calg$以及态射$u:\calf\to\calg$,定义$\Gamma(U,u)=u(U)$. 此时$\Gamma(U,-)$就构成了一个函子。如果固定$\calf$,考虑$V\subset U$,则我们还有限制映射$\Gamma(U,\calf)\to\Gamma(V,\calf)$,于是$\Gamma(-,\calf)$也是一个函子。由于预层态射与限制映射相容,因此$\Gamma(\star,-)$是一个双函子。
\end{para}

\begin{para}[茎]
设$\calf$是$X$上的一个预层,如果范畴$K$能容纳余极限,那么定义$\calf$在$p\in X$处的\idx{茎}$\calf_p$为$\varinjlim \calf(U)$,其中$U$跑遍所有$p$的邻域。

如果$K$是集合范畴的子范畴,$s\in \calf(U)$是一个截面,而$p\in U$,记$s$在典范态射$\calf(U)\to \calf_p$下的像为$s_p$. 它被称为截面$s$在$p$处的\idx{芽}。

对于交换群范畴(我们知道这是余完备的),直接写出芽的具体构造是比较方便的,$\calf_p$中的元素(芽)是这样的等价类$\langle U,f\rangle$的全体,其中$f$是$U$上的截面,而$U$是$p$的邻域。等价关系定义为$\langle U,f\rangle\sim \langle V,g\rangle$,当且仅当存在一个$p$的邻域$W\subset U\cap V$使得$f|_W=g|_W$. 

对$X$上的交换群层$\calf$,所有使得$\calf_x\neq 0$的$x$构成了一个$X$的子集,我们称之为$\calf$的支集,记作$\mathrm{Supp}(\calf)$.
\end{para}

利用余极限的泛性质,我们发现,预层间的态射$\varphi:\calf\to\calg$如下图在局部诱导了$\varphi_p:\calf_p\to\calg_p$.
\[
	\xymatrix{
		&&\calf(U)\ar[dl]^{\rho^U_{p}}\ar[dd]^{\rho^U_{V}}\ar@/_/[lld]_{{\rho'}^U_{p}\circ\varphi(U)} \\
		\calg_p&\calf_p\ar@{-->}[l]_(0.4){\varphi_p}&\\
		&&\calf(V)\ar[ul]_{\rho^V_{p}}\ar@/^/[llu]^{{\rho'}^V_{p}\circ\varphi(V)}
	}
\]

下面一个引理从交换群层的芽的构造来看是平凡的。

\begin{lem}\label{lem:1}
假设$\calf$是一个$X$上的交换群层,且$s$, $t\in \calf(U)$是两个截面。如果任取$p\in U$,都有$s_p=t_p$,则$s=t$. 此外,设$\calg$是$X$上的另一个交换群层,$\psi$, $\varphi:\calf\to\calg$都是层之间的态射,则$\psi|_U=\varphi|_U$当且仅当$\psi_p=\varphi_p$对任意的$p\in U$都成立。
\end{lem}

% \begin{proof}
% 不失一般性,可以假设$U=X$. 由交换群层上芽的构造,我们知道,如果$s_p=t_p$,则存在$p$的一个邻域$U_p$使得$s|_{U_p}=t|_{U_p}$. 遍历$p$,$\{U_p\}$是$U$的一个开覆盖,由层上截面的黏合,我们就得到了$s=t$.

% 此外,如果$\psi_p=\varphi_p$对任意的$p\in X$都成立,我们要证明,任取开集$U$,有$\psi(U)=\varphi(U)$. 任取$s\in \calf(U)$,由于$(\psi(U)(s))_p=\psi_p(s_p)=\varphi_p(s_p)=(\varphi(U)(s))_p$对所有$p\in U$都成立,所以$\psi(U)(s)=\varphi(U)(s)$. 由于$s$是任意的,所以$\psi(U)=\varphi(U)$. 反过来是显然的。
% \end{proof}

\begin{para}
考虑$\mathfrak{B}$是$X$的拓扑基,则同拓扑空间一样,我们取对象为基中的元素,态射为内含映射,则自然他也构成一个范畴。类似地,我们可以定义$\mathfrak{B}$上取值在$K$的预层为$\mathfrak{B}$到$K$的一个反变函子,层公理也可以一模一样搬过来。

如果范畴$K$可以容纳极限,而$\calf$是$\mathfrak{B}$上的一个预层,则我们可以通过$\calf'(U)=\varprojlim \calf(V)$定义出$X$上的一个预层$\calf'$,其中$V$跑遍集合$\{V\in \mathfrak{B}\,:\, V\subset U\}$,而这个集合通过包含构成一个显然的投影系。
\end{para}

\begin{pro}\label{psgl}
如果范畴$K$可以容纳极限,且$\calf$是$\mathfrak{B}$上的一个层,$\mathfrak{B}$上的层诱导的预层$\calf'$也是一个层。
\end{pro}

证明完全是泛性质的练习,见[EGA, Chap 0, 3.2]. 作为推论:在拓扑基的每一个开集$U_\alpha$上都有一个截面$s_\alpha$,任取指标$(\alpha,\beta)$,都存在$\gamma$使得$U_\gamma\subset U_\alpha\cap U_\beta$且$s_\alpha|_{U_\gamma}=s_\beta|_{U_\gamma}$,则存在唯一的整体截面$s$使得$s_\alpha=s|_{U_\alpha}$.

\begin{para}[层的限制]
对于$X$上的$K$-层$\calf$,很自然可以定义层$\calf$在$U$上的限制$\calf|_U$,这是一个$U$上的预层,对于$U$中的开集,有$\calf|_U(V)=\calf(V)$,限制态射直接从$\calf$那里继承过来。显然这也是一个层。
\end{para}

现在我们可以讨论层的黏合。

\begin{pro}
设$\{U_\alpha\}_{\alpha \in I}$是$X$的一个开覆盖,如果分别在每一个$U_\alpha$上有一个层$\calf_\alpha$,而且对于任意的指标组$(\alpha,\beta)$,都有一个同构$\theta_{\alpha\beta}:\calf_\alpha|_{U_\alpha\cap U_\beta}\to \calf_\beta|_{U_\alpha\cap U_\beta}$,满足:$\theta_{\alpha\alpha}=\id$, $\theta_{\alpha\beta}=\theta_{\beta\alpha}^{-1}$,以及对于任意的指标组$(\alpha,\beta,\gamma)$成立$\theta'_{\alpha\gamma}=\theta'_{\alpha\beta}\circ \theta'_{\beta\gamma}$,其中$\theta'_{\star\star}$是态射$\theta_{\star\star}$在$U_\alpha\cap U_\beta \cap U_\gamma$上的限制。则存在一个层$\calf$和一族同构$\eta_\alpha:\calf|_{U_\alpha}\to \calf_\alpha$.
\end{pro}

\begin{proof}
选取包含在某个开覆盖的开集的全体,这是全空间的一个拓扑基,命题中的黏合条件自然地给出层的条件。%详见[EGA, Chap 0, 3.3].
\end{proof}

\begin{para}[顺像]\label{sx}
设$f:X\to Y$是拓扑空间之间的连续映射,而$\calf$是$X$上的一个预层,则通过$f_*\calf(U)=\calf(f^{-1}(U))$我们可以定义出$Y$上的一个预层$f_*\calf$,他被称为$\calf$关于$f$的\idx{顺像}。 由于原象是保持包含关系的,即如果$V\subset U$,则$f^{-1}(V)\subset f^{-1}(U)$,所以$f_*\calf$上的限制映射就很自然地写作$(f_*\rho)^U_V=\rho^{f^{-1}(U)}_{f^{-1}(V)}$. 不难检验,当$\calf$是$X$上的一个层的时候,$f_*\calf$是$Y$上的一个层。

设$\varphi:\calf\to\calg$是$X$上两个预层之间的态射,则$f$自然诱导了一个新的态射$f_*\varphi:f_*\calf\to f_*\calg$,具体写出来就是$(f_*\varphi)(U)=\varphi(f^{-1}(U))$,通过自然变换$\varphi$可以验证$f_*\varphi$是一个自然变换。考虑两个连续映射的复合,很自然地可以验证$(f g)_*=f_* g_*$,这来自于关于原象的等式$(fg)^{-1}(U)=g^{-1}(f^{-1}(U))$. 

设$u:\mathcal{F}\to\mathcal{G}$是一个$X$上预层间的态射,则$f_*u(U)=u(f^{-1}(U))$就是一个$f_*\mathcal{F}\to f_*\mathcal{G}$之间的态射。并且,设$v:\mathcal{G}\to\mathcal{H}$是令一个$X$上预层间的态射,成立$f_*(v\circ u)=f_*v\circ f_* u$. 于是$f_*$就构成了一个$X$上预层范畴到$Y$上预层范畴之间的协变函子。并且,由于$f_*$将层变成层,所以$f_*$也诱导了一个$X$上层范畴到$Y$上层范畴之间的协变函子。
\end{para}

% \begin{lem}\label{lem:hanrudan}
% 设$u$, $v$是$Y$上的预层态射,而$i:Y\to X$是一个子空间的含入映射,则$i_*u=i_*v$将给出$u=v$.
% \end{lem}

% \begin{proof}
% 任取$Y$中的开集$V$,则必然存在一个$X$中的开集$U$使得$V=U\cap Y$,所以$i^{-1}(U)=V$. 于是$(i_{*}u)(U)=(i_{*}v)(U)$将给出$u(V)=v(V)$. 又$V$是任意的,所以$u=v$.
% \end{proof}

现在,我们考虑$(f_*\calf)_{f(p)}$与$\calf_p$之间的联系。设$\rho$是$\calf$上的限制映射族,从交换图
\[
	\xymatrix{
		&&&f_*\calf(U)=\calf(f^{-1}(U))\ar[dl]^{(f_*\rho)_{p}^U}\ar[dd]^{(f_*\rho)^U_{V}}\ar@/_/[llld]_-{\rho_{p}^{f^{-1}(U)}} \\
		\calf_p&&(f_*\calf)_{f(p)}\ar@{-->}[ll]_-{\quad (f_*^\calf)_p}&\\
		&&&f_*\calf(V)=\calf(f^{-1}(V))\ar[ul]_{(f_*\rho)_{p}^V}\ar@/^/[lllu]^-{\rho_{p}^{f^{-1}(V)}}
	}
\]
我们可以得到一个唯一的典范态射
\[
(f_*^\calf)_p:(f_*\calf)_{f(p)}\to \calf_p,
\]
这个态射的性质一般情况下并不清楚,即使在交换群范畴,它可能既不是单的也不是满的。

\begin{para}[赋环空间]
所谓的\idx{赋环空间}(ringed space),就是一个拓扑空间$X$,和拓扑空间上的交换环层$\mathcal{O}_X$构成的资料$(X,\mathcal{O}_X)$. 流形和上面的光滑函数构成一个赋环空间,同样,代数簇和上面的正则函数构成一个赋环空间。所谓的\idx{局部赋环空间},就是说任意一点$p\in X$处的茎$\mathcal{O}_p=\varinjlim \mathcal{O}_X(U)$是局部环。
\end{para}

流形的例子是一个局部赋环空间,因为在$p$附近的光滑函数,如果在$p$处不为零,则存在一个邻域是可逆的,故而,在$p$处为零的光滑函数构成的等价类是唯一的极大理想。代数簇的例子还是一个局部赋环空间,因为代数簇局部是仿射的,所以我们可以假设这是一个仿射簇,对于仿射簇而言,$X$中的点$p$(如果是非代数闭的情况,那就是一条Galois轨道)一一对应着坐标环$A(X)$中的一个极大理想$\mm_p$(这来自于Hilbert零点定理),而$\mathcal{O}_p$正好同构于$A(X)$关于$\mm_p$的局部化,所以是一个局部环。

\begin{para}[赋环空间之间的态射]
赋环空间之间的态射如下定义:设有赋环空间$(X,\mathcal{O}_X)$和$(Y,\mathcal{O}_Y)$,以及连续映射$\psi:X\to Y$和$Y$上的交换环层态射$\theta:\mathcal{O}_Y\to \psi_*\mathcal{O}_X$,这样的一个二元组$\Psi=(\psi,\theta)$构成了赋环空间范畴的态射,记做$\Psi:(X,\mathcal{O}_X)\to (Y,\mathcal{O}_Y)$.
\end{para}

要确切地理解这是一个态射,则需要验证复合,假设$\Phi=(\psi',\theta'):(Y,\mathcal{O}_Y)\to (Z,\mathcal{O}_Z)$是另一个态射,在$\Phi\circ \Psi$中,连续映射的复合不用多说,对于层的复合,则如图所示
\[
	\mathcal{O}_Z\xrightarrow{\theta'} \psi'_*\mathcal{O}_Y \xrightarrow{\psi'_*\theta} \psi'_*\psi_*\mathcal{O}_X,
\]
写成$(\psi'_*\theta)\circ \theta'$的形式。

\begin{para}[逆像]
有了顺像,我们现在考虑\idx{逆像}。设$f:X\to Y$是一个连续映射,而$\calf$是$X$上的一个层,$\calg$是$Y$上的一个预层。$f_*$是$X$上层范畴到$Y$上预层范畴的一个函子,如果它存在左伴随函子$f^*$,则$f^*\calg$被称为预层$\calg$关于$f$的逆像。换而言之,$\Hom_X(f^*\calg,\calf)$和$\Hom_Y(\calg,f_*\calf)$对层$\calf$与预层$\calg$是双函子同构。

设$u:\calg\to f_*\calf$,则在$\Hom_X(f^*\calg,\calf)$中对应的态射我们记作$u^\#$. 对应地,设$v:f^*\calg\to \calf$,在双函子同构下,在$\Hom_Y(\calg,f_*\calf)$中对应的态射我们记作$v^\flat$. 在音乐上,$\#$对应升调,而$\flat$对应降调,于是在这里分别对应提升和下降。前者读作sharp,后者读作flat. 
% 
% 逆像的复合类似于逆的复合。考虑两个连续映射$f:X\to Y$, $g:Y\to Z$,则从$(gf)_*=g_*f_*$,我们有函子性同构
% \[
% 	\Hom_Z((gf)^*\calg,\calf)\cong \Hom_Z(\calg,(gf)_*\calf)=\Hom_Z(\calg,g_*f_*\calf)\cong \Hom_Y(g^*\calg,f_*\calf)\cong \Hom_X(f^*g^*\calg,\calf),
% \]
% 所以$(gf)^*\cong f^*g^*$,这里的同构是函子的同构,即存在互逆的自然变换。
\end{para}

恒等态射的提升和下降在伴随函子理论中非常重要,他们可以反过来唯一确定伴随函子。我们下面稍微复习一下。

\begin{para}[伴随函子]\label{adjoint_functor}
设$L:D\to C$和$R:C\to D$为两个函子,如果存在双函子同构
\[
	C(L(-),\star)\cong D(-,R(\star)),
\]
则称$(L,R)$为一对伴随函子,$L$是$R$的左伴随,$R$是$L$的右伴随。遵从上面的习惯,我们将上式右边到左边称为提升,左边到右边称为下降,分别用$\#$与$\flat$表示。

现在,设$x\in D$,考虑恒等态射$\id_{L(x)}$的下降$\id_{L(x)}^\flat$,记作$L_x$. 同样,设$y\in C$,将恒等态射$\id_{R(y)}$的提升$\id_{R(y)}^\#$记作$R^y$. 改变$x$和$y$,我们就有了一对自然变换$L_-:\id_D\to RL$和$R^-:LR\to \id_C$. 前者经常被叫做单位(unit),后者叫做上单位(counit). 

上单位满足如下泛性质,任取$f:L(x)\to y$,都存在唯一的态射$f^\flat:x\to R(y)$使得分解
\[
	f:L(x)\xrightarrow{L(f^\flat)}LR(y)\xrightarrow{R^y}y
\]
成立,即$f=R^yL(f^\flat)$. 对单位也类似,对$g:x\to R(y)$有唯一分解$g=R(g^\#)L_x$.

现在反过来,给定函子$L:D\to C$,若任取$x\in C$,都存在一个$R(x)\in D$以及一个态射$\epsilon^x:L(R(x))\to x$使得任取态射$f:L(y)\to x$,都存在唯一的态射$g:y\to R(x)$使得分解
\[
	f:L(y)\xrightarrow{L(g)}L(R(x))\xrightarrow{\epsilon^x}x
\]
成立,则$L$是一个左伴随函子。它的右伴随函子$R$可以如下定义:对对象,$R:x\mapsto R(x)$,对态射$f:x\to x'$,考虑复合$f\epsilon^x:L(R(x))\to x'$,则泛性质给出$R(f):R(x)\to R(x')$. 不难检查这是$L$的左伴随函子,且$R^x=\epsilon^x$. 对偶地,我们也可以从单位出发去构造一对伴随函子。
%
% 对偶地,右伴随函子也有泛性质:设$R: C\to  D$是一个函子,使得任取$X\in  D$,都存在一个$L(X)\in  C$以及一个态射$\eta_X:X\to R(L(X))$使得任取态射$f:X\to R(Y)$,都存在唯一的态射$g:L(X)\to Y$使得分解
% \[
% 	f:X\xrightarrow{\eta_X}R(L(X))\xrightarrow{R(g)}R(Y)
% \]
% 成立,则$R$被称为一个右伴随函子。同样,从泛性质,不难构造其左伴随函子。
\end{para}

现在回到我们的语境。考虑$\id_{f_*\calf}:f_*\calf\to f_*\calf$,按上面的记号,有\footnote{
	注意,这里出现了$f_*^\calf$,与我们在上面看到的$f_*$在一点处的典范态射$(f_*^\calf)_p:(f_*\calf)_{f(p)}\to \calf_p$有着相同符号,下面会看到,他们就是一样的。实际上,至少对取值在交换群范畴的预层来说,$(f^*\calg)_p=\calg_{f(p)}$,所以这里出现的$f_*^\calf$也有$(f_*^\calf)_p:(f_*\calf)_{f(p)}\to \calf_p$. 首先可以看到定义域和值域相同。
}
\[
	f_*^\calf:=\left(\id_{f_*\calf}\right)^\#: f^*f_*\calf \to \calf, \quad 
	f^*_\calg:=\left(\id_{f^*\calg}\right)^\flat : \calg \to f_*f^*\calg.
\]
并且,我们可以有如下唯一分解
\[
	u^\# : f^*\calg \xrightarrow{f^*(u)} f^*f_*\calf \xrightarrow{f_*^\calf}\calf,\quad 
	v^\flat : \calg \xrightarrow{f^*_\calg} f_*f^*\calg\xrightarrow{f_*(v)} f_*\calf.
\]

最后,构造左伴随函子$f^*$,即要求构造层$f^*\mathcal F$以及预层态射$f^*_{\mathcal G}:\mathcal G\to f_*f^*\mathcal G$使得,任取态射$u:\calg \to f_*\calf$,都存在唯一分解
\[
	u : \calg \xrightarrow{f^*_\calg} f_*f^*\calg\xrightarrow{f_*(u^\#)} f_*\calf,
\]
其中$u^\#:f^*\mathcal G\to \calf$.

% \begin{lem}
% 	设有层的态射$v:\mathcal G'\to \mathcal G$以及$u:\mathcal G\to f_*\calf$,则$uv:\mathcal G'\to f_*\calf$,且$(uv)^\#=u^\#(f^*v)$.
% \end{lem}

% \begin{proof}
% 实际上,从典范分解$u^\#=f_\calf^\#(f^*u)$,所以
% \[
% 	u^\#(f^*v)=f_\calf^\#(f^*u)(f^*v)=f_\calf^\#(f^*(uv))=(uv)^\#.
% \]
% \end{proof}

% \begin{pro}
% \label{pro:1}对于一些特殊情况,逆像层总是存在的。考虑一个内含映射的例子,设$U$是$X$的开子集,相应的内含映射为$i:U\hookrightarrow X$. 对于$X$上的$K$-层$\calf$,$\calf|_U$即其在$U$上的逆像层$i^*\calf$.
% \end{pro}

% \begin{proof}
% 检查泛性质即可。现在还缺一个典范态射$\calf\to i_*i^*\calf$,由于我们假设了$i^*\calf=\calf|_U$,所以
% \[
% 	i_*i^*\calf(V)=i^*\calf(i^{-1}(V))=\calf|_U(U\cap V)=\calf(U\cap V).
% \]
% 那么限制映射族$\rho^V_{U\cap V}:\calf(V)\to \calf(U\cap V)$可以构成层之间的态射$\calf\to i_*i^*\calf$,我们将其记做$i^\flat$,我们希望这就是典范态射。剩下的不过是检查$(\calf|_U,i^\flat)$所需要满足的泛性质,即任取$u:\calf\to i_*\calg$,都有唯一的态射$u^\#:i^*\calf\to \calg$使得分解$u:\calf\xrightarrow{i^\flat} i_*i^*\calf \xrightarrow{i_*u^\#} i_*\calg$成立。

% 首先,我们来证明分解的唯一性。假设存在$v$, $w:i^*\calf\to \calg$使得分解$u=(i_*v)i^\flat=(i_*w)i^\flat$成立。设$V$是$U$的开子集,此时$i^\flat(V)=\id_{\calf(V)}$,所以$(i_*v)(V)=(i_*w)(V)$. 同时再注意到,$u(V)=u(i^{-1}(V))=i_*u(V)$和$v(V)=v(i^{-1}(V))=i_*v(V)$,所以我们有$v(V)=u(V)$成立,由于$u$, $v$都是$U$上的层之间的态射,而$V$可以取遍$U$任意的开子集,所以$v=u$成立。

% 然后证明存在性,对于任意的态射$u:\calf\to i_*\calg$,他具有形式$u(V):\calf(V)\to i_*\calg(V)=\calg(U\cap V)$. 遍历$U$中的开集$V$,我们定义$u^\#(V)=u(V):\calf(V)\to \calg(U\cap V)=\calg(V)$,此时由于$i^*\calf(V)=\calf|_U(V)=\calf(V)$,所以我们定义出了一族态射$u^\#(V):i^*\calf(V)=\calf(V)\to \calg(V)$,自然也得到了态射$u^\#:i^*\calf\to \calg$. 现在,需要检查分解$u:\calf\xrightarrow{i^\flat} i_*i^*\calf \xrightarrow{i_*u^\#} i_*\calg$成立,这个分解其实说白了就是,已知如下交换图(图中的限制映射都略去了)成立,
% \[
% 	\xymatrix{
% 		\calf(V)\ar[r]^-{u(V)} \ar[d]&\calg(U\cap V) \ar[d]\\
% 		\calf(W)\ar[r]^-{u(W)}&\calg(U\cap W)
% 	}
% \]
% 其中$W\subset V$是$X$中的开集,要验证定出来的$u^\#$使得如下交换图成立。
% \[
% 	\xymatrix{
% 		\calf(V)\ar[r] \ar[d]&\ar[rr]^{u^\#(U\cap V)}\calf(U\cap V) \ar[d]&&\calg(U\cap V) \ar[d]\\
% 		\calf(W)\ar[r]&\ar[rr]^{u^\#(U\cap W)}\calf(U\cap W)&&\calg(U\cap W)
% 	}
% \]
% 这个并不困难,左边一个矩形的交换性来自于限制映射的复合,这是自然的,右边一个矩形的交换性来自于$u^\#$的定义和上面一个交换图,所以最后要检验的不过是横向的那个分解,即对于任意的开集$V$,分解$u(V):\calf(V)\to \calf(U\cap V) \xrightarrow{u^\#(U\cap V)} \calg(U\cap V)$成立。现考虑如下交换图,
% \[
% 	\xymatrix{
% 		\calf(V)\ar[rr]^-{u(V)} \ar[d]&&\calg(U\cap V) \ar@{=}[d]\\
% 		\calf(U\cap V)\ar[rr]^{u(U\cap V)}&&\calg(U\cap V)
% 	}
% \]
% 这就得出了$u(V)=u(U\cap V)\rho^V_{U\cap V}$,而$u^\#(U\cap V)=u(U\cap V)$,所以$u(V)=u^\#(U\cap V)\rho^V_{U\cap V}$,即交换图横向的分解成立。
% \end{proof}

\begin{para}[伴随层]
设$1_X:X\to X$是恒同映射,对于$X$上的预层$\calf$,如果层$(1_X)^*\calf$存在,则他称为$\calf$的\idx{伴随层},有时候也被记做$\calf^+$. 对于任意的层$\calf'$和态射$u:\calf\to\calf'$,总有唯一分解$u:\calf\xrightarrow{(1_X)^*_\calf} \calf^+ \to \calf'$成立。
\end{para}

注意到双函子同构$\Hom_X ((1_X)^* \mathcal{F},\mathcal{G})\cong \Hom_X(\mathcal{F},\mathcal{G})$,前者是$X$上层范畴之间的态射集,后者是$X$上预层范畴的态射集。所以$(1_X)^*$是层范畴的自由对象,即层范畴到预层范畴的遗忘函子的左伴随函子。若$\calf$是一个层,则$\Hom_X(\calf,\calg)$同时是预层与层的态射集,由Yoneda引理,$\calf\cong (1_X)^* \mathcal{F}$. 因此,层的伴随层总存在,且同构于他本身。

\begin{lem}
如果范畴$K$可以容纳极限,且对$Y$上的预层$\mathcal{F}$存在伴随层,则对任意连续映射$f :X\to Y$,逆像层$f^*\mathcal{F}$存在,且同构于层$U\mapsto \varprojlim_{V\supset f(U)} \mathcal{F}^+(V)$,其中$V$跑遍所有包含$f(U)$的开集。
\end{lem}

\begin{proof}
	我们先验证$U\mapsto \varprojlim_{V\supset f(U)} \mathcal{F}^+(V)$是一个层。设$U$是一个开集,而$U_\alpha\subset U$也是一个开集。对使得$f(U_\alpha)\subset W$的开集$W$,我们都可以找到一个开集$W'$包含$W$且使得$f(U)\subset W'$. 此时典范投影$\varprojlim_{V\supset f(U)} \mathcal{F}^+(V)\to \mathcal{F}^+(W')$和限制态射$\mathcal{F}^+(W')\to \mathcal{F}^+(W)$复合给出了态射$\varprojlim_{V\supset f(U)} \mathcal{F}^+(V)\to \mathcal{F}^+(W)$. 由限制态射的相容性与极限的泛性质,我们得到了新的限制态射
	\[
	{\varprojlim}_{V\supset f(U)} \mathcal{F}^+(V)\to {\varprojlim}_{W\supset f(U_\alpha)} \mathcal{F}^+(W).
	\]
	所以这确实是一个预层,暂时记作$\mathcal{H}$.

	现在设$\mathscr{U}$是$U$的一个开覆盖,对下列交换图,我们要证明存在唯一的态射$\varphi:A\to \mathcal{H}(U)$
	% 若存在一个对象$A$使得下列交换图成立
	% \[
	% \xymatrix{
	% 	A\ar[r]^{\varphi_{\alpha}}\ar[d]_{\varphi_{\beta}}&\mathcal{H}(U_\beta)\ar[d]\\
	% 	\mathcal{H}(U_\alpha)\ar[r]&\mathcal{H}(U_{\alpha}\cap U_{\beta})
	% }
	% \]
	% 则我们只需证明存在唯一的态射$\varphi:A\to \mathcal{H}(U)$使得下图交换的时候,
	\[
		\xymatrix{
			A\ar@/_/[ddr]_{\varphi_{\alpha}}\ar@/^/[drr]^{\varphi_{\beta}}\ar@{-->}[dr]|{\varphi}&& \\
			&\mathcal{H}(U)\ar[r]\ar[d]&\mathcal{H}(U_\beta)\ar[d]\\
			&\mathcal{H}(U_\alpha)\ar[r]&\mathcal{H}(U_{\alpha}\cap U_{\beta})
		}
	\]
	由极限$\mathcal{H}(U)=\varprojlim_{V\supset f(U)} \mathcal{F}^+(V)$的泛性质,实际上只需要知道一族态射$A\to \mathcal{F}^+(V)$即可,其中$V\supset f(U)$. 而这直接来自于$\mathcal{F}^+$是一个层以及上面的交换图。

	任取$u:f^*\mathcal{F}\to\calg$,任取$Y$中的开集$V$,考虑开集$U=f^{-1}(V)$. 由于$f(U)=V$就是一个开集,所有包含$f(U)$的开集拥有极小元$V$,因此$f^*\mathcal{F}(f^{-1}(V))=\mathcal{F}^+(V)$.
	则
	\[
	u(f^{-1}(V)):\mathcal{F}^+(V)\to \calg(f^{-1}(V)),
	\]
	给出了态射$u':\mathcal{F}^+\to f_*\calg$. 最后,再复合上典范态射$(1_X)^*_\calf:\calf \to \mathcal{F}^+$. 我们定义$u^\flat=u'\circ (1_X)^*_\calf: \mathcal{F}\to f_*\calg$.

	反过来。由于$\calg$是一个层,所以$f_*\calg$也是一个层,根据证明最开始的推理,
	\[
		f^*f_*\calg(U)=\varprojlim_{V\supset f(U)} (f_*\calg)^+(V)=\varprojlim_{V\supset f(U)} f_*\calg(V)=\varprojlim_{V\supset f(U)} \calg(f^{-1}(V))
	\]
	也是一个层。由于包含关系$U\subset f^{-1}(f(U))\subset f^{-1}(V)$,
	我们有一族限制态射$\mathcal{G}(f^{-1}(V))\to \mathcal{G}(U)$. 于是有映射
	\[
	\varprojlim_{V\supset f(U)} \calg(f^{-1}(V))\to \calg(f^{-1}(W))\to \mathcal{G}(U),
	\]
	其中$W$是任意的开集$W\supset f(U)$. 由限制映射的相容性,我们可以得知,这个映射的定义与$W$的选取无关。因此,我们就得到了态射$f^*_\calg:f^*f_*\calg\to \mathcal{G}$. 任取$v:\mathcal{F}\to f_*\mathcal{G}$,我们有$f^*(v):f^*\mathcal{F}\to f^*f_*\mathcal{G}$(这直接来自极限的泛性质),然后复合上面的态射$f^*_\calg$,就得到了$v^\#=f^*_\calg\circ f^*(v):f^*\mathcal{F}\to \mathcal{G}$.

	最后,无外乎检查一些泛性质。由构造,这是直接的。
\end{proof}

% 从这个引理,Proposition \ref{pro:1} 就是显然的了。事实上,由于包含$i(V)=U\cap V$的任意开集中有极小元$U\cap V$,所以极限$\varprojlim_W \mathcal{F}^+(W)$总存在且就等于$\mathcal{F}^+(U\cap V)$. 并且,由于$\mathcal{F}$是一个层,所以$\mathcal{F}^+=\mathcal{F}$也存在。

\begin{coro}
	设$U$是$X$的开子集,相应的内含映射为$i:U\hookrightarrow X$. 对于$X$上的$K$-层$\calf$,$\calf|_U$即其在$U$上的逆像层$i^*\calf$. 
\end{coro}

\begin{proof}
	事实上,由于包含$i(V)=U\cap V$的任意开集中有极小元$U\cap V$,所以极限$\varprojlim_W \mathcal{F}^+(W)$总存在且就等于$\mathcal{F}^+(U\cap V)$. 并且,由于$\mathcal{F}$是一个层,所以$\mathcal{F}^+=\mathcal{F}$也存在。
\end{proof}

\begin{para}[赋环空间的典范含入态射]
将层的限制的知识应用到赋环空间,我们就有了赋环空间的态射$(i,i^\flat):(U,\mathcal{O}_U)\to (X,\mathcal{O}_X)$,其中$\mathcal{O}_U=\mathcal{O}_X|_U$,他被称为赋环空间的典范含入态射。赋环空间$(U,\oo_U)$时常被称为赋环空间$(X,\oo_X)$的开子赋环空间。

设$\Phi:(X,\mathcal{O}_X)\to (Y,\mathcal{O}_Y)$是一个态射,他在$U\subset X$上的限制记做$\Phi|_U:=\Phi\circ (i,i^\flat)$,如果$\Phi=(\psi,\theta)$,则$\Phi|_U=\bigl(\psi|_U, \psi_*(i^\flat) \circ \theta\bigr):(U,\mathcal{O}_U)\to (Y,\mathcal{O}_Y)$,层之间的态射具体写出来即
\[
	\mathcal{O}_Y\xrightarrow{\theta} \psi_*\mathcal{O}_X \xrightarrow{\psi_*(i^\flat)} \psi_*i_*\mathcal{O}_U=\bigl(\psi|_U\bigr)_*\mathcal{O}_U,
\]
其中$\psi_*(i^\flat)(V)=i^\flat\bigl(\psi^{-1}(V)\bigr)=\rho^{\psi^{-1}(V)}_{U\cap \psi^{-1}(V)}$,设$s\in \mathcal{O}_Y(V)$是一个截面,那么表现为
\[
	\psi_*(i^\flat)(V)\circ \theta(V)(s)=\theta(V)(s)|_{U\cap \psi^{-1}(V)}.
\]
我们记$\Phi|_U=(\psi|_U,\theta|_U)$,其中$\theta|_U=\psi_*(i^\flat)\theta$,对截面$s\in \mathcal{O}_Y(V)$,它成立
\[
	\theta|_U(V):s\mapsto \theta(V)(s)|_{\psi|_U^{-1}(V)}\in \oo_X(\psi|_U^{-1}(V)).
\]

很容易验证,如果$V$又是$U$的一个开子集,则$(\Phi|_U)|_V=\Phi|_V$.
\end{para}

\begin{pro}\label{pro:1.20}
设$(\psi,\theta):(X,\oo_X)\to (Y,\oo_Y)$是一个赋环空间态射,如果存在一个$Y$的开子集$U$使得$\psi(X)\subset U$,则存在唯一的态射$\Phi:(X,\oo_X)\to (U,\oo_U)$使得分解$(\psi,\theta)=(i,i^\flat)\Phi$成立,其中$(i,i^\flat):(U,\oo_U)\hookrightarrow (Y,\oo_Y)$是典范含入。
\end{pro}

\begin{proof}
由于$\psi(X)\subset U$,所以我们可以通过$\psi'(x)=\psi(x)$可以定义唯一的连续映射$\psi':X\to U$使得分解$\psi=i\psi'$成立。现在,考虑典范同构
\[
	\Hom_{U}(\oo_U,\psi'_*\oo_X)=\Hom_{U}(i^*\oo_Y,\psi'_*\oo_X)\cong \Hom_{Y}(\oo_Y,i_*\psi'_*\oo_X)=\Hom_{Y}(\oo_Y,\psi_*\oo_X).
\]
所以对每一个$\theta:\oo_Y\to \psi_*\oo_X$,都对应唯一的$\theta':\oo_U\to \psi'_*\oo_X$. 注意分解$\theta=(i_*\theta')i^\flat$不外乎是$i^*\oo_Y$的泛性质而已,此时$\Phi=(\psi',\theta')$就是我们所求的唯一态射。
\end{proof}

\begin{coro}\label{coro:1.21}
开子赋环空间的典范含入是一个单态射。
\end{coro}

有了限制的知识,就可以讨论赋环空间间态射的黏合。

\begin{pro}
设$(X,\mathcal{O}_X)$是一个赋环空间,并且$\{U_\alpha\}_{\alpha\in I}$是一个$X$的开覆盖。如果我们有一族态射$\Phi_\alpha:(U_{\alpha},\mathcal{O}_{U_{\alpha}})\to (Y,\mathcal{O}_Y)$,对于任意的两个指标$(\alpha,\beta)$,成立$\Phi_\alpha|_{U_{\alpha}\cap U_{\beta}}=\Phi_\beta|_{U_{\alpha}\cap U_{\beta}}$,则存在唯一的态射$\Phi:(X,\mathcal{O}_{X})\to (Y,\mathcal{O}_Y)$使得$\Phi|_{U_{\alpha}}=\Phi_\alpha$.
\end{pro}

\begin{proof} 连续函数在开覆盖上的黏合是显然的,对于层的态射那部分,用的技术无外乎层上截面的黏合,这是很琐碎的检验,这里略去了。\end{proof}

下面一个命题是类似的,但我们给一个证明。

\begin{pro}\label{rsnh}
设$(X,\mathcal{O}_X)$是一个赋环空间,$\mathfrak{B}$是$X$的一个拓扑基。如果任取$U_\alpha\in \mathfrak{B}$都给一个态射$\Phi_\alpha:(U_{\alpha},\mathcal{O}_{U_{\alpha}})\to (Y,\mathcal{O}_Y)$,且任取两个指标$(\alpha,\beta)$,存在一个指标$\gamma$使得$U_\gamma\subset U_\alpha\cap U_\beta$且$\Phi_\alpha|_{U_\gamma}=\Phi_\beta|_{U_\gamma}$,则存在唯一的态射$\Phi:(X,\mathcal{O}_{X})\to (Y,\mathcal{O}_Y)$使得$\Phi|_{U_{\alpha}}=\Phi_\alpha$.
\end{pro}

\begin{proof}
设$\Phi_\alpha=(\psi_\alpha,\theta_a)$,这里的$\psi_\alpha$的黏合是显然的,得到$\psi$. 现在我们考虑$\theta_\alpha:\oo_Y\to (\psi_\alpha)_*\oo_{U_\alpha}$. 任取$V\subset Y$,我们有
\[
	\theta_\alpha(V):\oo_Y(V)\to (\psi_\alpha)_*\oo_{U_\alpha}(V)=\Gamma(\psi_\alpha^{-1}(V),\oo_{U_\alpha})=\Gamma(U_\alpha\cap\psi^{-1}(V),\oo_{X}),
\]
任取$s\in \oo_Y(V)$,我们得到了一族截面$t_\alpha=\theta_\alpha(V)(s)\in\Gamma(U_\alpha\cap\psi^{-1}(V),\oo_{X})$. 黏合条件告诉我们,任取两个指标$(\alpha,\beta)$,存在一个指标$\gamma$使得$U_\gamma\subset U_\alpha\cap U_\beta$且$t_\alpha|_{U_\gamma}=t_\beta|_{U_\gamma}$. 于是,层公理(见Proposition \ref{psgl})告诉我们存在$t\in\Gamma(\psi^{-1}(V),\oo_{X})$使得$t|_{U_\alpha}=t_\alpha$. 于是$s\mapsto t$就给出了态射$\theta(V):\oo_Y(V)\to \Gamma(\psi^{-1}(V),\oo_{X})=\psi_*\oo_X(V)$. 此即所证。
\end{proof}

\begin{pro}[赋环空间的黏合]\label{rsn}
设$(X_\alpha,\oo_\alpha)$是一族赋环空间,并且给定任意一个二元组$(\alpha,\beta)$,都有一个$X_\alpha$的开集$U_{\alpha\beta}$,$X_\beta$的开集$U_{\beta\alpha}$,以及赋环空间同构$\varphi_{\alpha\beta}:(U_{\alpha\beta},\oo_\alpha|_{U_{\alpha\beta}})\to (U_{\beta\alpha},\oo_\beta|_{U_{\beta\alpha}})$,满足:
\begin{compactenum}[~~~1.]
\item $U_{\alpha\alpha}=X_\alpha$,此时$\varphi_{\alpha\alpha}$是恒同态射。
\item 任取三元组$(\alpha,\beta,\gamma)$,若以$\varphi'_{\alpha\beta}$记$\varphi_{\alpha\beta}$在$U_{\alpha\beta}\cap U_{\alpha\mu}$上的限制,则它应是一个同构,满足$\varphi'_{\alpha\gamma}=\varphi'_{\alpha\beta}\varphi'_{\beta\gamma}$.
\end{compactenum}

将$\{X_\alpha\}$利用同构$\{\varphi_{\alpha\beta}\}$沿着$\{U_{\alpha\beta}\}$黏成一个拓扑空间$X$,此时$X_\alpha$可以等同于$X$的开子集$X'_\alpha$,而$U_{\alpha\beta}$等同于$X'_\alpha\cap X'_\beta$. 于是,利用层的黏合,存在一个赋环空间$(X,\oo)$还有一族同构$\varphi_\alpha:(X|_{U'_\alpha},\oo|_{U'_\alpha})\to (X_\alpha,\oo_\alpha)$.
\end{pro}

最后我们给一个比较一般情况的逆像的存在性的证明。

\begin{thm}\label{proinverse}
设$\psi:X\to Y$是一个连续映射,$\calf$是$X$上的一个$K$-层,而$\calg$是$Y$上的一个$K$-预层。如果范畴$K$是交换群范畴,则$\calg$关于$\psi$的逆像层$\psi^*\calg$存在。
\end{thm}

由于交换群作为$\zz$-模,对任意族都一定存在极限. 所以从上面的引理,逆像的存在性就归结到了伴随层的存在性上面。不过下面我们还是直接证明这个命题。

\begin{proof} 
	在交换群范畴,我们知道预层的茎总是存在的。所以我们如下构造$\psi^*\calg$:对$X$中的开集$U$, 令$\psi^*\calg(U)$是函数$s:U\to \bigoplus_{x\in U}\calg_{\psi(x)}$的集合(若只是集合范畴,考虑不交并即可),这些函数需要满足
	\begin{itemize}
		\item 对任意的$x\in U$, $s(x)\in \calg_{\psi(x)}$,以及

		\item 对任意的$x\in U$,总存在$\psi(x)$的一个开邻域$V$,$x$的一个邻域$W\subset \psi^{-1}(V)\cap U$和一个元素$t\in \calg(V)$,使得当$p\in W$的时候总有$s(p)=t_{\psi(p)}$.
	\end{itemize}
	$\psi^*\calg(U)$上的加法直接定义为$(s+t)(x)=s(x)+t(x)$,因此$\psi^*\calg(U)$是一个交换群。如果$V$是$U$的开子集,那么函数限制$i^U_V:\psi^*\calg(U)\to \psi^*\calg(V)$就是所需要的预层的限制映射,记$i^U_V(s)=s|_V$,因此$\psi^*\calg$是一个预层。同时$\psi^*\calg$也是一个层,这直接来自于函数可以粘合。

	此外,成立典范同构$(\psi^*\calg)_x\cong \calg_{\psi(x)}$. 实际上,$s\mapsto s(x)$给出了态射$\psi^*\calg(U)\to \calg_{\psi(x)}$,他与限制映射相容,所以由余极限的泛性质,存在唯一的态射$\theta:(\psi^*\calg)_x\to \calg_{\psi(x)}$满足$\theta(s_x)=s(x)$. 它显然是单射,因为如果$s_x=t_x$,那么存在一个领域$U$使得$s|_U=t|_U$,所以$s(x)=s|_U(x)=t|_U(x)=t(x)$. 同时,它也是满射,任取$t_{\psi(x)}=(V,t)\in \calg_{\psi(x)}$,其中$V$是$\psi(x)$的一个邻域,则取$x$附近的一个邻域$U$使得$U\subset \psi^{-1}(V)$,在$\psi^*\calg(U)$上,我们定义$s:p\mapsto t_{\psi(p)}$. 于是$t_{\psi(x)}=s(x)$. 因此,我们一般会通过$s_x=s(x)$来等同$(\psi^*\calg)_x$和$\calg_{\psi(x)}$. 下面我们默认这种等同。

	现在设$U$是$Y$的一个开集,对于$s\in \calg(U)$, 通过$s^+(x)=s_{\psi(x)}$可以定义一个截面
	\[
		s^+\in \psi^*\calg(\psi^{-1}(U))=\psi_*\psi^*\calg(U),
	\]
	其中$x\in \psi^{-1}(U)$. 于是,$\psi^*_{\calg}(U):s\mapsto s^+$定义了态射$\psi^*_{\calg}(U):\calg(U)\to \psi_*\psi^*\calg(U)$. 容易检查,$\psi^*_{\calg}$是一个预层之间的态射,即一个自然变换。即,任取$s\in \calg(U)$,都有
	\[(\psi_* i)^U_V\bigl(\psi^*_{\calg}(U)(s)\bigr)=(s|_V)^+=\psi^*_{\calg}(V)\bigl(\rho^U_V(s)\bigr).
	\]

	% 对任意的$s\in \calg(U)$,因为
	% \[
	% 	(\psi_* i)^U_V\bigl(\psi^*_{\calg}(U)(s)\bigr)=i^{\psi^{-1}(U)}_{\psi^{-1}(V)}\left(\{x\mapsto s_{\psi(x)}\,:\, x\in \psi^{-1}(U)\}\right)=\{x\mapsto s_{\psi(x)}\,:\, x\in \psi^{-1}(V)\}=(s|_V)^+,
	% \]
	% 以及
	% \[
	% 	\psi^*_{\calg}(V)\bigl(\rho^U_V(s)\bigr)=\psi^*_{\calg}(V)(s|_V)=(s|_V)^+.
	% \]
	% 所以$\psi^*_{\calg}$是一个预层之间的态射。

	最后就是要检验泛性质,即对于任意的态射$u:\calg \to \psi_*\calf$,都可以找到唯一的态射$u^\#:\psi^*\calg \to \calf$成立分解
	\[
		u:\calg\xrightarrow{\psi^*_{\calg}} \psi_*\psi^*\calg \xrightarrow{\psi_*u^\#} \psi_*\calf.
	\]

	固定$U$是$X$的开子集以及$s\in \psi^*\calg(U)$,在每一点$x\in U$附近,我们可以找到一个开子集$W_x$, 一个$\psi(x)$附近的开子集$V_x$使得$W_x\subset \psi^{-1}(V_x)\cap U$以及一个$t(x)\in \calg(V_x)$使得$p\in W_x$的时候$s(p)=t(x)_{\psi(p)}$恒成立。遍历$x\in U$,$\{W_x\}_{x\in U}$是$U$的一个开覆盖,由于$\calf$是一个层,可以把$\bigl(u(V_x)(t(x))\bigr)|_{W_x}$拼成$U$上的一个截面$s'$. 实际上,利用Lemma \ref{lem:1},我们只要证明$\bigl(u(V_x)(t(x))\bigr)|_{W_x}$的芽不依赖选取的$x$即可。任取$p\in W_x$,我们有
	\[
	\bigl(u(V_x)(t(x))\bigr)_p=\psi_p\left(u(V_x)(t(x))_{\psi(p)}\right)=\psi_p\circ u_{\psi(p)}(t(x)_{\psi(p)})=\psi_p\circ u_{\psi(p)}(s(p)).
	\]
	于是,定义$u^\#(U):s\mapsto s'$就得到了层的态射$u^\#:\psi^*\calg \to \calf$的完整定义,显然这由$u$唯一确定。具体到茎间,即
	\[
	u^\#_x(s_x)=(\psi_*^{\calf})_x u_{\psi(x)}(s(x))=(\psi_*^{\calf})_x u_{\psi(x)}(s_x),
	\]
	这里$(\psi_*^{\calf})_x:(\psi_*\calf)_{\psi(x)}\to \calf_x$是典范态射。所以$u^\#_x=(\psi_*^{\calf})_x u_{\psi(x)}$. 

	证明的最后一步就是检验分解是否成立。任取$r\in \mathcal{G}(V)$,则$u(V)(r)$和$\psi_*u^\#(V)\circ \psi^*_\calg(V)(r)$都是$\mathcal{F}$在$\psi^{-1}(V)$上的截面,任取$x\in \psi^{-1}(V)$,它们局部可以写作
	\[
	(u(V)(r))_x=(\psi_*^{\calf})_x (u(V)(r))_{\psi(x)}=(\psi_*^{\calf})_x u_{\psi(x)}(r_{\psi(x)})=u^\#_x(r_{\psi(x)})
	\]
	和
	\[
	\left(\psi_*u^\#(V)\circ \psi^*_\calg(V)(r)\right)_x=\bigl(u^\#(\psi^{-1}(V))(r^+)\bigr)_x=u^\#_x(r^+_x)=u^\#_x(r_{\psi(x)}).
	\]
	所以这两个截面相同。又由于$r$和$x$是任意的,分解$u(V)=\psi_*u^\#(V)\circ \psi^*_\calg(V)$成立。
\end{proof}

证明之中的一些映射是值得注意的。由于$\psi^*_{\calg}$是预层之间的态射,所以任取$x\in X$,它诱导了
\[
	(\psi^*_{\calg})_{\psi(x)}:\calg_{\psi(x)}\to (\psi_*\psi^*\calg)_{\psi(x)}.
\]
结合典范态射$(\psi_*^{\psi^*\calg})_x:(\psi_*\psi^*\calg)_{\psi(x)}\to (\psi^*\calg)_{x}$,我们可以得到典范态射
\[
	(\psi_*^{\psi^*\calg})_x(\psi^*_{\calg})_{\psi(x)}:\calg_{\psi(x)}\to (\psi^*\calg)_{x},
\]
而我们在证明中等同了$\calg_{\psi(x)}$和$(\psi^*\calg)_{x}$,此时,$(\psi_*^{\psi^*\calg})_x(\psi^*_{\calg})_{\psi(x)}$就是恒等。任取$r_{\psi(x)}=(V,r)\in \calg_{\psi(x)}$,其中$V$是$\psi(x)$的一个邻域,我们有
\[
	(\psi^*_{\calg})_{\psi(x)}r_{\psi(x)}=(\psi^*_{\calg}(V)(r))_{\psi(x)}=(r^+)_{\psi(x)},
\]
其中$r^+\in \psi_*\psi^*\calg(U)$. 此时
\[
	(\psi_*^{\psi^*\calg})_x(\psi^*_{\calg})_{\psi(x)}(r_{\psi(x)})=(\psi_*^{\psi^*\calg})_x((r^+)_{\psi(x)})=r^+_x=r^+(x)=r_{\psi(x)},
\]
此即所证。

此外,不难注意到,在我们的构造中
\[
	(f^\calf_*)_x=(\id_{f_*\calf})^\#_x:(f^*f_*\calf)_x=(f_*\calf)_{f(x)}\to \calf_x
\]
就是典范态射$(f^\calf_*)_x:(f_*\calf)_{f(x)}\to \calf_x$.

\begin{pro}
设$f:\calf\to\calg$是两个$Y$上的交换群层之间的态射,而$\psi:X\to Y$是一个连续映射,则,局部地,态射$\psi^*f:\psi^*\calf\to \psi^*\calg$写作$(\psi^* f)_x=f_{\psi(x)}$.
\end{pro}

\begin{proof}
由如下双函子同构,
\[
	\xymatrix{
		\Hom_X(\psi^*\calg,\mathcal{H})\ar[rr] \ar[d]_{(\psi^*f)^*}&&\Hom_Y(\calg,\psi_*\mathcal{H}) \ar[d]^{f^*}\\
		\Hom_X(\psi^*\calf,\mathcal{H})\ar[rr]&&\Hom_Y(\calf,\psi_*\mathcal{H})
	}
\]
任取态射$u:\calg\to \psi_*\mathcal{H}$,应当成立
\[
	u^\#\circ \psi^* f=(uf)^\#.
\]
特别地,取$u=\psi_\calg^*$,则应有
\[
	\psi^* f=(\psi_\calg^* f)^\#.
\]
所以
\[
	(\psi^* f)_x=(\psi_\calg^* f)^\#_x =(\psi_*^{\psi^*\calg})_x (\psi_\calg^* f)_{\psi(x)}=(\psi_*^{\psi^*\calg})_x (\psi_\calg^*)_{\psi(x)} f_{\psi(x)}=f_{\psi(x)}.
\]
这里应用了上面我们约定的等同。
\end{proof}

\begin{pro}[逆像函子的复合]
考虑两个连续映射$f:X\to Y$, $g:Y\to Z$,则Theorem \ref{proinverse} 构造的逆像函子满足$(gf)^*=f^*g^*$.
\end{pro}

\begin{proof}
% 已经知道,从$(gf)_*=g_*f_*$,我们有函子性同构
% \[
% 	\Hom_Z((gf)^*\calg,\calf)\cong \Hom_Z(\calg,(gf)_*\calf)=\Hom_Z(\calg,g_*f_*\calf)\cong \Hom_Y(g^*\calg,f_*\calf)\cong \Hom_X(f^*g^*\calg,\calf),
% \]
% 所以$(gf)^*\cong f^*g^*$,这里的同构是函子的同构,即存在互逆的自然变换,所以我们只要检验这些个自然变换是恒同即可。
任取$x\in X$以及$Z$上的层$\mathcal H$,在我们的构造中,上述自然变换诱导了等同
\[
	((gf)^*\mathcal H)_x=\mathcal H_{g(f(x))}=(g^* \mathcal H)_{f(x)}=(f^*(g^*\mathcal H))_x,
\]
此外,任取$Z$上的层之间的态射$u$,自然变换诱导了等同
\[
	((gf)^* u)_x= u_{g(f(x))}=(g^* u)_{f(x)}=(f^*(g^* u))_x,
\]
由于$x$是任意的,所以$(gf)^*\mathcal H=f^*(g^*\mathcal H)$以及$(gf)^* u=f^*g^* u$.
\end{proof}

从另一个角度来看,给定一对伴随函子,则我们可以唯一确定态射集之间的同构,这就意味着,在我们的构造中,双函子同构的复合
\[
	\Hom_Z(\calg,g_*f_*\calf)\cong \Hom_Y(g^*\calg,f_*\calf)\cong \Hom_X(f^*g^*\calg,\calf)
\]
与双函子同构
\[
	\Hom_Z(\calg,(gf)_*\calf)\cong \Hom_Z((gf)^*\calg,\calf)
\]
是相同的。特别地,考虑$\id_{(gf)_*\calf}\in \Hom_Z((gf)_*\calf,(gf)_*\calf)$,则应该有
\[
	(gf)^*_{\calf} = f_{\calf}^* g_{f_*\calf}^*.
\]

\begin{para}
	作为上面命题及其推论的应用,考虑两个赋环空间的态射$(f,\varphi):X\to Y$和$(g,\psi):Y\to Z$,层之间的复合已知写作
	\[
		\oo_Z\xrightarrow{\psi} g_*\oo_Y\xrightarrow{g_*\varphi} g_*f_*\oo_X,
	\]
	有时候,我们需要考虑$((g_*\varphi)\psi)^\#:(gf)^*\oo_Z\to \oo_X$. 此时,注意到我们有另一种方式
	\[
		\varphi^\#(f^*\psi^\#):(gf)^*\oo_Z=f^*g^*\oo_Z\xrightarrow{f^*\psi^\#}f^*\oo_Y\xrightarrow{\varphi^\#}\oo_X
	\]
	来得到$f^*g^*\oo_Z\to \oo_X$的态射,所以上一个命题的推论告诉我们$((g_*\varphi)\psi)^\#=\varphi^\#(f^*\psi^\#)$. 

	% 当然,我们也可以没那么聪明地,考虑典范分解
	% \[
	% 	((g_*\varphi)\psi)^\#=(gf)^*_{\oo_X} (gf)^*((g_*\varphi)\psi)=(gf)^*_{\oo_X} (f^*g^*g_*\varphi)((gf)^*\psi).
	% \]
	% 首先,注意到$(gf)^*_{\oo_X}$可以由上一个命题的推论分解为两个态射$(gf)^*_{\oo_X}=f_{\oo_X}^* f^*(g_{f_*\oo_X}^*)$,所以
	% \[
	% 	((g_*\varphi)\psi)^\#=f_{\oo_X}^* f^* \left(g_{f_*\oo_X}^* g^*g_*\varphi\right)((gf)^*\psi)=f_{\oo_X}^* f^* \left(g_{f_*\oo_X}^* g^*g_*\varphi g^*\psi\right).
	% \]
	% 接着,考虑交换图
	% \[
	% 	\xymatrix{
	% 		g^*g_*\oo_Y \ar[d]_{g^*g_*\varphi}\ar[rr]^-{g_{\oo_Y}^*}&&\oo_Y\ar[d]^\varphi\\
	% 		g^*g_*f_*\oo_X\ar[rr]^-{g_{f_*\oo_X}^*}&&f_*\oo_X
	% 	}
	% \]
	% 即$\varphi g_{\oo_Y}^*=g_{f_*\oo_X}^*(g^*g_*\varphi)$给出了
	% \[
	% 	((g_*\varphi)\psi)^\#=f_{\oo_X}^* f^* \left(\varphi g_{\oo_Y}^* g^*\psi\right)=f_{\oo_X}^* f^* \left(\varphi \psi^\#\right)=f_{\oo_X}^* f^* \varphi f^*(\psi^\#)=\varphi^\# f^*(\psi^\#).
	% \]
	% 这就是我们相当笨重但是却很扎实的证明。
\end{para}

将其应用到茎上,我们立即得到
\[
	((g_*\varphi)\psi)^\#_x=\varphi^\#_x(f^*\psi^\#)_x=\varphi^\#_x \psi^\#_{f(x)},
\]
此即命题:

\begin{pro}\label{pro:morcom}
	设有两个赋环空间的态射$(f,\varphi):X\to Y$和$(g,\psi):Y\to Z$,层之间的复合已知写作
	\[
		\theta:\oo_Z\xrightarrow{\psi} g_*\oo_Y\xrightarrow{g_*\varphi} g_*f_*\oo_X,
	\]
	则$\theta^\#_x:(\oo_Z)_{g(f(x))}\to (\oo_X)_x$可以分解为
	\[
		\theta^\#_x:(\oo_Z)_{g(f(x))}\xrightarrow{\psi^\#_{f(x)}}(\oo_Y)_{f(x)}\xrightarrow{\varphi^\#_x} (\oo_X)_x.
	\]
\end{pro}

后面我们在研究概形态射的复合的时候,这个命题可能是有益的。

\begin{para}[一些总结]
	让我们限制取值在交换群范畴的层和预层,总结一下上面这些乱七八糟的等式或者关系。设$f:X\to Y$, $g:Y\to Z$是连续映射,$\calf$, $\calg$, $\mathcal H$是预层或层,看具体语境。
	\begin{compactenum}
		\item 定义
		\[
			f_*^\calf:=\left(\id_{f_*\calf}\right)^\#: f^*f_*\calf \to \calf, \quad 
			f^*_\calg:=\left(\id_{f^*\calg}\right)^\flat : \calg \to f_*f^*\calg.
		\]
		他们使得态射$u:\calg\to f_*\calf$和$v:f^*\calg\to\calf$的提升和下降可以写为
		\[
			u^\# =f_*^\calf f^*(u),\quad v^\flat = f_*(v) f^*_\calg.
		\]
		\item 局部地,对(预)层有$(f^*\calf)_x=\calf_{f(x)}$,对态射有$(f^*u)_x=u_{f(x)}$,而提升写作$v^\#_x=(f_*^{\calf})_x v_{f(x)}$.
		\item $(f_*^\calf)_x:(f_*\calf)_{f(x)}\to \calf_x$就是$f_*:\calf\to f_*\calf$在点$x$由余极限的泛性质诱导的典范同态。
		\item 若等同$\calg_{f(x)}$和$(f^*\calg)_{x}$,则以下典范同构是恒等态射
		\[
			(f_*^{f^*\calg})_x(f^*_{\calg})_{f(x)}:\calg_{f(x)}\to (f^*\calg)_{x}.
		\]
		\item 设$u:\calf\to \calg$而$v:\calg\to f_*\mathcal H$是层同态,则$(vu)^\# = u^\# f^*u$.
		\item 两个连续映射的复合下,有$(gf)^*=f^*g^*$.
		% \[
		% 	(gf)^*_{\calf} = f_{\calf}^* f^*(g_{f_*\calf}^*).
		% \]
		\item 设有两个赋环空间的态射$(f,\varphi):X\to Y$和$(g,\psi):Y\to Z$,他们的复合的态射的层部分为$\theta = g_*\varphi \psi$,则提升$\theta^\#_x:(\oo_Z)_{g(f(x))}\to (\oo_X)_x$局部可以分解为$\theta^\#_x=\varphi^\#_x\psi^\#_{f(x)}$.
	\end{compactenum}
\end{para}

\begin{para}[交换群层范畴]
$X$上的交换群层构成一个范畴,态射就是层之间的态射。并且,这是一个准加性范畴,常值层$0$(即任取开集$U$,$0(U)=0$)是它的零对象。
由于极限与极限是可以交换的,所以一族交换群层的极限依然是一个层。特别地,我们可以断言,任意有限极限是存在的。所以$X$上的交换群层构成一个加性范畴。
特别地,由于核是一个极限,所以对
态射$u:\calf\to\calg$,预层$U\mapsto \ker(u(U))$是一个层,我们记作$\ker u$. 结合典范含入
\[
	i(U):\ker(u(U))\hookrightarrow \calf(U),
\]
它们构成了$u$的核。因此,$X$上的交换群层范畴存在核。

在茎上,我们有$(\ker u)_x=\ker(u_x)$. 实际上,任取$(U,s)\in (\ker u)_x$,由于此时$s\in \ker u(U)$即$u(U)(s)=0$,所以$u_x(U,s)=(U,u(U)(s))=0$. 反之,任取$(U,s)\in \ker(u_x)$,我们应有$(U,u(U)(s))=0$,所以存在一个$V\subset U$使得$u(V)(s|_V)=0$,此即说明$(U,s)=(V,s|_V)\in (\ker u)_x$.

对余核,预层$U\mapsto \coker(u(U))$不一定是一个层,暂时将这个预层记作$\mathcal{K}$,所以我们转而考虑其伴随层$\mathcal{K}^+$. 可以检验,$\mathcal{K}^+$就是$u$的余核。实际上,对每一个$X$的开集$U$,我们考虑如下交换图
\[
	\xymatrix{
		\calf(U)\ar@<0.3ex>[r]^-{u(U)} \ar@<-0.3ex>[r]_-{0}&\calg(U)\ar[r]\ar[dr]_-{v(U)}&\coker(u(U))=\mathcal{K}(U)\ar[d]\ar[r]^-{i(U)}&\mathcal{K}^+(U)\ar@{-->}[ld]\\
		&&\mathcal{H}(U)&
	}
\]
图中略去了显然的典范态射的符号,而$i:\mathcal{K}\to \mathcal{K}^+$是预层到伴随层的典范态射,虚线直接来自于伴随层的泛性质。所以$\mathcal{K}^+$正是$u$的余核,我们记作$\coker u$. 于是,$X$上的交换群层范畴存在余核。
\end{para}

\begin{lem}
在茎上,我们有$(\coker u)_x\cong \coker(u_x)$. 
\end{lem}

一般而言,我们会直接等同$(\coker u)_x$和$\coker(u_x)$,这个并不会有什么问题,因为在茎是一个余极限,本身就容许一个同构。此时$(\coker u)_x=\coker(u_x)$.

\begin{proof}
预层与伴随层的茎是相同的,所以我们只需考虑预层$U\mapsto \coker(u(U))$. 首先记$\pi(U):\calg(U)\to \coker (u(U))$为典范同态,$\pi:\calg\to \coker u$是一个预层同态,以及$p_x:\calg_x\to \coker(u_x)$是$u_x:\calf_x\to\calg_x$对应的典范同态。

现在考虑如下交换图
\[
	\xymatrix{
	\calf(U)\ar[r]^{u(U)}\ar[d]^{\mu^U_x}&\calg(U)\ar[r]^-{\pi(U)}\ar[d]^{\nu^U_x}&\coker (u(U))\ar@{-->}[d]^{\alpha(U)}\\
	\calf_x\ar[r]^{u_x}&\calg_x\ar[r]^-{p_x}&\coker (u_x)
	}
\]
其中$\mu$, $\nu$是对应的限制映射。由于$0=0\mu^U_x=p_xu_x\mu^U_x=p_x\nu^U_xu(U)$,由$\coker(u(U))$的泛性质,我们有唯一分解$p_x\nu^U_x=\alpha(U)\pi(U)$,此即图中$\alpha(U)$的由来。并且,它与限制映射相容,这是分解的唯一性保证的。再由余极限的泛性质,我们得到了$\alpha_x:(\coker u)_x\to \coker (u_x)$使得$\alpha(U)=\alpha_x\pi^U_x$. 我们下面证明$\alpha_x$是一个同构。

首先,证明$\alpha_x$是一个单同态,技术上就是追图。任取$s_x\in (\coker u)_x$使得$\alpha_x(s_x)=0$. 设$s_x=\langle U,s\rangle$,其中$s\in \coker (u(U))$,于是$0=\alpha_x(s_x)=\alpha_x \pi_x^U(s)=\alpha(U)(s)$. 由于$\pi(U)$是满同态,故存在$t\in \calg(U)$使得$s=\pi(U)(t)$. 由上面的交换图,我们有
\[
	0=\alpha(U)\circ \pi(U)(t)=p_x\nu_x^U(t)=p_x(t_x).
\]
从交换群中熟知的构造$\coker(u_x)=\calg_x/\im u_x$,所以$t_x\in \im u_x$,即存在一个$r_x\in \calf_x$使得$t_x=u_x(r_x)$. 因此存在$U$的开子集$V$以及$r_x=\langle V,r\rangle$使得$t_x=\nu_x^V\circ u(V)(r)$,所以存在一个$V$的开子集$W$使得$t|_W=u(V)(r)|_W=u(W)(r|_W)$. 综上
\[
	s|_W=\pi(W)(t|_W)=\pi(W)\circ u(W)(r|_W)=0,
\]
这就给出了$s_x=0$.

然后证明$\alpha_x$是满同态。任取$a\in \coker(u_x)$,由于$p_x$是满同态,所以存在$b_x\in\calg_x$使得$a=p_x(b_x)$. 设$b_x=\langle U,b\rangle$,所以$\alpha(U)\circ \pi(U)(b)=a$,又因为$\alpha(U)=\alpha_x\pi^U_x$,所以$a=\alpha_x\left(\pi^U_x\circ \pi(U)(b)\right)$. 因此$\alpha_x$是一个满同态。
\end{proof}

稍微提一句,在等同下$(\coker u)_x=\coker(u_x)$,上述证明中就会有等同$\pi_x=p_x$. 因此,如果将余核的典范态射依然记作$\coker$的话,我们还是有$\coker(u_x)=(\coker u)_x$,它们都是$u_x:\calf_x\to\calg_x$的余核,而$\coker u$是$u:\calf \to \calg$的余核。对$\ker$这点也是类似的。

\begin{para}[商层]\label{qsheaf}
设$\calf$, $\calg$是$X$上的交换群层,如果对每一个$U$,$\calg(U)$是$\calf(U)$的子群,且典范含入$i(U):\calg(U)\to\calf(U)$是一个层的态射,则称$\calg$是$\calf$的子层。在模范畴中,商定义为典范含入的余核。现在我们依然如此定义,定义$\calf/\calg$为层$\coker(i)$,具体来说,$\calf/\calg$是预层$U\mapsto \calf(U)/\calg(U)$的伴随层。它被称为商层。在茎上,我们有$(\calf/\calg)_x=\calf_x/\calg_x$.

就像在任何加性范畴中那样,我们依然会用$\ker u$和$\coker u$来记典范态射。此时,定义$\im u=\ker\coker u$,从核与余核的知识,这当然是一个层。实际上,它就是$U\mapsto \im(u(U))$的伴随层,利用Lemma \ref{lem:1},只要说明它们拥有相同的茎即可。注意到预层和其伴随层的茎相同,所以一个是$x\mapsto \im u_x$,另一个是$x\mapsto \ker\coker u_x=\im u_x$,它们相同。
\end{para}

最后,我们考虑单态射和满态射。所谓单态射,就是说态射是左可消的,所谓满态射,就是说态射是右可消的。在准加性范畴中,$u$是单态射当且仅当$\ker u=0$,$u$是满态射当且仅当$\coker u=0$. 

\begin{lem}
设$\calf$, $\calg$是两个$X$上的交换群层,则$u:\calf\to\calg$是一个单(满)态射当且仅当$u_x:\calf_x\to\calg_x$是单(满)同态对每一个$x\in X$都成立。
\end{lem}

所以,如果$u_x$处处是同构,则$u$也是同构。因为在Abel范畴中,既单又满即同构。

\begin{proof}
单的部分直接来自于Lemma \ref{lem:1}. 所以我们假设$u:\calf\to\calg$是一个满态射,或者说$\coker(u)=0$. 首先,由于伴随层的茎不变,所以$\coker u_x=(\coker u)_x=0$,因此$u_x$是满同态。

反过来,给定任意态射$v:\calg\to\mathcal{H}$,如果$vu=0$,则其等价于$(vu)_x=v_xu_x=0$. 由于$u_x$处处是满的,所以$v_x=0$. 利用Lemma \ref{lem:1},我们得到了$v=0$. 所以$u$是一个满态射。
\end{proof}

\begin{para}
下面我们说明,单态射是一个核,而满态射是一个余核。因此$X$上的交换群层构成一个Abel范畴。

不难检验,如果$u:\calf\to\calg$是一个单态射,则$u=\ker(\coker(u))$. 检验这点并不困难,利用Lemma \ref{lem:1},只要它们在每一点上的茎相同即可,而
\[
	\ker(\coker(u))_x=\ker(\coker(u)_x)=\ker(\coker(u_x))=u_x,
\]
因为$u_x$是单的。

类似地,如果$u:\calf\to \calg$是一个满射,则$u=\coker (\ker(u))$. 实际上
\[
	\coker (\ker(u))_x=\coker(\ker(u)_x)=\coker(\ker(u_x))=u_x,
\]
因为$u_x$是满的。
\end{para}

\begin{pro}
在$X$上的交换群层的Abel范畴和$Y$上的交换群层的Abel范畴间,函子$\psi^*$是一个正合函子。
\end{pro}

\begin{proof}
首先它当然是右正和的,因为左伴随。所以我们只要证明他将单态射变成单态射即可。考虑$f:\calf \to \calg$是一个单态射,所以$f_{\psi(x)}:\calf_{\psi(x)} \to \calg_{\psi(x)}$都是单的。在等同$(\psi^* f)_x=f_{\psi(x)}$下,他给出了
\[
	(\psi^* f)_x=f_{\psi(x)}:(\psi^*\calf)_{x} \to (\psi^*\calg)_{x}
\]
都是单的,而这又等价于$\psi^* f$是单的。
\end{proof}

% \para 设$w:\calg_1\to \calg_2$是$Y$上的一个$K$-预层态射,如果$\calg_2$和$\calg_2$关于$\psi:X\to Y$的逆像是存在的,则复合$\rho_{\calg_2}\circ w:\calg_1\to \psi_*\psi^*\calg_2$,由逆像的泛性质,我们可以找到一个态射$v:\psi^*\calg_1\to \psi^*\calg_2$来分解上面的复合态射,我们将其记做$\psi^*(w)$,他将$Y$上的$K$-预层态射,变成了$X$上的$K$-层态射。这里不加验证,$\psi^*(w_1)\circ \psi^*(w_2)=\psi^*(w_1\circ w_2)$,所以这是一个协变函子。

\section{模层}

\begin{para}[模层]
设$(X,\mathcal{O}_X)$是一个赋环空间,我们称$\calf$是一个$\mathcal{O}_X$-\idx{模层},如果对每一个开集$U$,$\calf(U)$都是一个$\mathcal{O}_X(U)$-模,并且$\calf$和$\mathcal{O}_X$的限制映射是相容的,即:设$V\subset U$是$X$的开子集,记$\calf$的限制映射为$\pi^U_V$,相应地$\mathcal{O}_X$的限制映射为$\rho^U_V$,那么对于标量乘法而成的截面$a\cdot s\in\calf(U)$,则$\pi^U_V(a\cdot s)=\rho^U_V(a)\cdot \pi^U_V(s)$. 类似地,可以定义$\mathcal{O}_X$-\idx{代数层}:环层$\calf$被称为$\mathcal{O}_X$-代数层,如果他是一个$\mathcal{O}_X$-模层。
\end{para}

我们知道$\mathcal{O}_X(U)$自己作为$\mathcal{O}_X(U)$-模的子模就是$\mathcal{O}_X(U)$的理想,所以一个$\mathcal{O}_X$的$\mathcal{O}_X$-子模层,被称为$\mathcal{O}_X$-\idx{理想层}。

\para 考虑任意的小范畴$I$,以及一系列$K$-层$\{\calf_i\}_{i\in I}$,由于极限的极限还是一个极限,所以可以看到预层$U\mapsto \varprojlim_{i\in I}\calf_i(U)$依然是一个$K$-层。他被称作\idx{极限层},记做$\varprojlim_{i\in I}\calf_i$. 放到模范畴中,考虑一系列$\mathcal{O}_X$-模层$\{\calf_i\}_{i\in I}$,则$\prod_{i\in I}\calf_i$依然是一个$\mathcal{O}_X$-模层,他被称为\idx{直积层}。

但对于直和,箭头反过来了导致预层$U\mapsto \bigoplus_{i\in I}\calf_i(U)$就不一定是一个层。所以我们定义\idx{直和层}为上述预层的伴随层,记做$\bigoplus_{i\in I}\calf_i$. 但对于有限指标集$I$,直和和直积等价,所以$U\mapsto \bigoplus_{i\in I}\calf_i(U)$此时就是一个层。

\begin{para}
考虑一个$\mathcal{O}_X$-模层同态$u:\mathcal{O}_X\to \calf$,这个同态完全由截面$s=u(X)(1)\in \calf(X)$决定,其他的任意截面$t\in \mathcal{O}_X(U)$,则$u(U)(t)=t\cdot (s|_U)$.

类似地,考虑直和层$\bigoplus_{i\in I}\mathcal{O}_X=\mathcal{O}_X^{(I)}$,对每一个$i\in I$,都有典范含入$h_i:\mathcal{O}_X\to \mathcal{O}_X^{(I)}$. 设$\calf$是一个$\mathcal{O}_X$-模层,而$u:\mathcal{O}_X^{(I)}\to \calf$是一个$\mathcal{O}_X$-模层同态,现在考虑复合$u\circ h_i:\mathcal{O}_X\to \calf$,他被截面$s_i=(u\circ h_i)(X)(1)$唯一确定,故$u$被$\{s_i\}_{i\in I}$唯一确定,对应着$u(U)\bigl(\bigoplus_i a_i\bigr)=\sum_i a_i\cdot (s_i|_U)$.
\end{para}

% 现在,我们来看所谓的有限生成模等概念如何推广到模层上面来。回忆一下,$R$-模$M$称被某个集合$E\subset M$生成,就是说,$M$是所有$s\in E$生成的自由模$R^{(E)}$的商模,即如下正和列成立$R^{(E)}\to M \to 0$. 所谓的有限生成,即是指存在一个有限集$E$生成$M$。

% 一个$\mathcal{O}_X$-模层$\calf$被称作被截面$\{s_i\}_{i\in I}$生成,就是说由截面$\{s_i\}_{i\in I}$定义的同态$\mathcal{O}_X^{(I)}\to \calf$是满射。类似地,$\calf$被称为\idx{有限生成}$\mathcal{O}_X$-模层,如果存在一族有限截面$\{s_i\}_{i\in I}$生成他。

% 换句话说,一个$\mathcal{O}_X$模层是有限生成的,如果存在如下正和列
% \[
% 	\bigoplus_{i=1}^n\mathcal{O}_X\to \calf\to 0,
% \]
% 其中$n$是一个正整数。

\begin{para}
设$u:\calf\to \calg$是$\mathcal{O}_X$-模层同态,他要使得$u(U)$是$\mathcal{O}_X(U)$-模$\calf(U)$和$\calg(U)$之间的同态。完全类似交换群层范畴,我们可以定义$U\to \ker(u(U))$, $U\to \im(u(U))$, $U\to \coker(u(U))$的伴随层为相应的$\mathcal{O}_X$-模层$\ker(u)$, $\im(u)$以及$\coker(u)$. $u$被称为满的,就是说$\im(u)=\calg$或$\coker u=0$,$u$被称为单的$\ker(u)=0$. 
\end{para}

和交换群层上类似,$\ker(u_x)=(\ker u)_x$, $\coker(u_x)=(\coker u)_x$, $\im(u_x)=(\im u)_x$. 同样,$\mathcal{O}_X$-模层构成一个Abel范畴,态射集记作$\Hom_{\oo_X}$,我们将在其中讨论正和列。下面是一个实用而简单的引理。

\begin{lem}
设$\calf\to \calg \to \mathcal{H}$是$\mathcal{O}_X$-模层的一个列,它是正和列当且仅当在每条茎上$\calf_x\to\calg_x\to \mathcal{H}_x$是正和列。
\end{lem}

\begin{para}[拟凝聚层]
一个$\mathcal{O}_X$-模层$\calf$被称作\idx{拟凝聚}的,就是说,对任意的$x\in X$,存在他的一个开邻域$U$使得$\calf|_U$同构于某个形如$\mathcal{O}_X^{(I)}|_U\to \mathcal{O}_X^{(J)}|_U$的同态的余核,此处$I$, $J$是任意的指标集。如果一个$\mathcal{O}_X$-代数层作为$\mathcal{O}_X$-模层是拟凝聚的,则他被称为拟凝聚$\mathcal{O}_X$-代数层。

换句话说,对拟凝聚$\mathcal{O}_X$-模层$\calf$,在每一点附近都有一个开邻域$U$使得如下正和列\footnote{顺带一提,对于任何一个$R$-模$M$而言,正合列
\[
	R^{(I)}\to R^{(J)}\to M\to 0
\]
总是存在的。实际上,取$J=M$,同态取为$i_M:1_m\mapsto m$. 由同构基本定理,我们有短正合列
\[
	0\to \ker i_M\hookrightarrow R^{(M)} \xrightarrow{i_M} M \to 0.
\]
现在取$I=\ker i_M$,将$i_{\ker i_M}$复合上$\ker i_M\hookrightarrow R^{(M)}$,我们就得到了
\[
	R^{(\ker i)}\to R^{(M)} \xrightarrow{i_M} M \to 0.
\]
他显然是正和的。}成立
\[
	\mathcal{O}_X^{(I)}|_U\to \mathcal{O}_X^{(J)}|_U\to \calf|_U\to 0.
\]
这里的$I$和$J$可以依赖于点。

考虑显然的正和列$0\to \mathcal{O}_X^{(I)}\to \mathcal{O}_X^{(I)} \to 0$,所以$\mathcal{O}_X^{(I)}$本身就是拟凝聚$\mathcal{O}_X$-模层。
\end{para}

% 后面我们会看到,每一个环都有一个自然的拓扑空间,在这个拓扑空间上有一个环层,这俩构成一个局部赋环空间。对于一个$R$-模$M$,因为$R$已经诱导了一个局部赋环空间,我们在上面还将搞出一个模层,称为$M$的伴随层。所谓的拟凝聚模层就是局部是某个模的伴随层。这点可以类比流形,局部是欧几里得的。

\begin{para}[有限型模层]
称一个$\oo_X$-模层$\calf$是有限型模层,如果任取$x\in X$,都存在一个开集$U$和正整数$n_x$以及如下的正合列
\[
	\mathcal{O}^{n_x}_X|_U\to \calf|_U\to 0.
\]
显然,这是一个局部条件。
\end{para}

\begin{para}[凝聚层]
称一个$\oo_X$-模层$\calf$是凝聚的,如果$\calf$是一个有限型$\oo_X$-模层,且对任意开集$U\subset X$、任意正整数$n$以及任意同态
\[
	u:\mathcal{O}^{n}_X|_U\to \calf|_U,
\]
$\ker u$都是有限型模层。显然,凝聚层是一个拟凝聚层。
\end{para}

\begin{para}
设$\mathcal{F}$和$\mathcal{G}$都是$\mathcal{O}_X$-模层,我们定义$\mathcal{F}\otimes_{\mathcal{O}_X}\mathcal{G}$是预层$U\mapsto \mathcal{F}(U)\otimes_{\mathcal{O}_X(U)}\mathcal{G}(U)$的伴随层。如果张量积的下标是清楚的,我们一般会略去它。

类似模范畴中那样,我们有典范同构$\oo_X\otimes \calf\cong \calf$, $\calf\otimes \calg\cong \calg\otimes \calf$, $(\calf\otimes \calg) \otimes \mathcal{H}\cong \calf\otimes (\calg \otimes \mathcal{H})$,且这些同构都是函子性的。为了证明它,我们可以在每个开集上找到与限制映射相容的模同态,这族模同态构成了一个模层之间的态射,为证明这是同构,只需它在每一点处的茎上诱导的是同构就行。为此,我们需要下面这个命题,有了它,一切都是显然的了。
\end{para}

\begin{pro}
$(\mathcal{F}\otimes\mathcal{G})_x=\mathcal{F}_x\otimes\mathcal{G}_x$.
\end{pro}

\begin{proof}
我们直接验证$\mathcal{F}_x\otimes\mathcal{G}_x=\varinjlim_{U\ni x} \mathcal{F}(U)\otimes\mathcal{G}(U)$,即$\mathcal{F}_x\otimes\mathcal{G}_x$满足余积的泛性质。记$\rho^U_V$为预层$U\mapsto \mathcal{F}(U)\otimes \mathcal{G}(U)$的限制映射,它将$\sum_{i=1}^n s_i\otimes t_i$变成$\sum_{i=1}^n s_i|_V\otimes t_i|_V$. 记态射$\rho^U_x:\mathcal{F}(U)\otimes \mathcal{G}(U)\to \mathcal{F}_x\otimes\mathcal{G}_x$,它将$\sum_{i=1}^n s_i\otimes t_i$变成$\sum_{i=1}^n (s_i)_x\otimes (t_i)_x$.

如果存在对象$A$以及态射族$\varphi(U):\mathcal{F}(U)\otimes \mathcal{G}(U)\to A$使得,如果$V$是$U$的开子集,则$\varphi(U)=\varphi(V)\rho^U_V$. 我们需要定义一个$\varphi_x:\mathcal{F}_x\otimes\mathcal{G}_x\to A$使得$\varphi(U)=\varphi_x \rho^U_x$对每一个开集$U$都成立,并验证这是唯一满足要求的态射。此即泛性质。

任取$\sum_{i=1}^n (s_i)_x\otimes (t_i)_x$,其中$(s_i)_x=\langle s_i,U_i\rangle$, $(t_i)_x=\langle t_i,V_i\rangle$,任取开集$U\subset \bigcap_{i=1}^n U_i\cap V_i$,定义
\[
	\varphi_x\left(\sum_{i=1}^n (s_i)_x\otimes (t_i)_x\right)=\varphi(U)\left(\sum_{i=1}^n s_i|_U\otimes t_i|_U\right),
\]
由于$\varphi(U)=\varphi(V)\rho^U_V$,所以这个定义是良定的,不依赖于选取的$U$.

现在来检验分解,给定开集$U$,任取$\sum_{i=1}^n s_i\otimes t_i\in \mathcal{F}(U)\otimes \mathcal{G}(U)$,从构造有
\[
	\varphi_x\circ \rho^U_x\left(\sum_{i=1}^n s_i\otimes t_i\right)=\varphi_x\left(\sum_{i=1}^n (s_i)_x\otimes (t_i)_x\right)=\varphi(U)\left(\sum_{i=1}^n s_i\otimes t_i\right),
\]
所以$\varphi(U)=\varphi_x \rho^U_x$成立。

最后,$\varphi_x$的唯一性来自于所有$\mathcal{F}_x\otimes\mathcal{G}_x$的元素都可以写成某个$\rho_x^U$的像($U$并不固定),故分解$\varphi(U)=\varphi_x \rho^U_x$告诉我们$\varphi_x$由所有的$\varphi(U)$确定,此即唯一性。再具体一些,给定$a\in \mathcal{F}_x\otimes\mathcal{G}_x$,一定存在一个$b$和$V$使得$a=\rho^V_x(b)$,此时$\varphi_x(a)=\varphi(V)(b)$. 遍历$a$,我们就得到了$\varphi_x$的唯一性。
\end{proof}

\begin{para}[顺像]
设$(X,\oo_X)$和$(Y,\oo_Y)$是两个赋环空间,而$\Psi=(\psi,\theta):(X,\oo_X)\to (Y,\oo_Y)$是赋环空间之间的态射。现在给一个$X$上的$\oo_X$-模层$\calf$,将$\calf$只看成一个交换群层,我们有顺像$\psi_*\calf$. 由于$\psi_*\calf(U)=\calf(\psi^{-1}(U))$与$\psi_*\oo_X(U)=\oo_X(\psi^{-1}(U))$相容,所以$\psi_*\calf$具有$\psi_*\oo_X$-模层结构。此外,由于存在态射$\theta:\oo_Y\to \psi_*\oo_X$,所以$\psi_*\calf$具有$\oo_Y$-模层结构,我们将这个$\oo_Y$-模层记作$\Psi_*\calf$,称为$\calf$在$\Psi$下的顺像。

设$u:\calf\to\calg$是一个$\oo_X$-模层态射,于是$\psi_*u:\psi_*\calf\to\psi_*\calf$是一个$\psi_*\oo_X$-模层态射,同时也就是一个$\oo_Y$-模层态射,我们将这个态射记作$\Psi_*u$,称为态射$u$的顺像。结合上述两点,我们可以知道,顺像$\Psi_*$构成一个函子。
\end{para}

\begin{para}[张量积的顺像]
设$(X,\oo_X)$和$(Y,\oo_Y)$是两个赋环空间,而$\Psi=(\psi,\theta):(X,\oo_X)\to (Y,\oo_Y)$是赋环空间之间的态射。再设$\calf$, $\calg$都是$\oo_X$-模层。任取$Y$的开集$U$,我们有预层到其伴随层的典范态射
\[
	\calf(\psi^{-1}(U))\otimes_{\oo_X(\psi^{-1}(U))} \calg(\psi^{-1}(U))\to \calf\otimes \calg(\psi^{-1}(U)),
\]
他对$\oo_X(\psi^{-1}(U))=\psi_*\oo_X(U)$双线性的,所以也对$\oo_Y$是双线性的,张量积的泛性质将给出态射
\[
	\Psi_*\calf (U)\otimes_{\oo_Y(U)}\Psi_*\calg (U)\to \Psi_*(\calf\otimes \calg) (U).
\]
他与限制映射相容,由伴随层的泛性质,所以存在典范态射
\[
	\Psi_*(\calf)\otimes_{\oo_Y}\Psi_*(\calg) \to \Psi_*(\calf\otimes_{\oo_X} \calg).
\]
\end{para}

\begin{para}[逆像]
设$(X,\oo_X)$和$(Y,\oo_Y)$是两个赋环空间,而$\Psi=(\psi,\theta):(X,\oo_X)\to (Y,\oo_Y)$是����环空间之间的态射。逆像我们这里依然定义为顺像的左伴随函子,即
\[
	\Hom_{\oo_X}(\Psi^*\calf,\calg)\cong \Hom_{\oo_Y}(\calf,\Psi_*\calg)
\]
对$\calf$, $\calg$都是函子性同构。

逆像的存在可以如下构造:首先对$\theta:\oo_Y\to \psi_*\oo_X$,由交换环层的只是,我们有对应的$\theta^\#:\psi^*\oo_Y\to \oo_X$,因此$\oo_X$具有一个$\psi^*\oo_Y$-模层结构,我们将其记作$(\oo_X)_{[\theta]}$.

此外,对$\oo_Y$-模层$\calf$,作为交换群层,存在交换群层$\psi^*\calf$. 它实际上具有一个$\psi^*\oo_Y$-模层结构,任取$s\in \psi^*\calf(U)$以及$r\in \psi^*\oo_Y(U)$,从构造(见Theorem \ref{proinverse}),我们可以如下定义$rs\in \psi^*\calf(U)$
\[
	rs(x)=r(x)s(x)\in \calf_{\psi(x)}.
\]
它确实处于$\psi^*\calf(U)$中:对$s$和$r$,存在一个$\psi(x)$的邻域$V$,$x$的一个邻域$W\subset \psi^{-1}(V)\cap U$以及$a\in \oo_Y(V)$, $b\in \calf(V)$使得任取$x\in W$都有$r(x)=a_{\psi(x)}$, $s(x)=b_{\psi(x)}$. 此时,$rs(x)=r(x)s(x)=a_{\psi(x)}b_{\psi(x)}=(ab)_{\psi(x)}$.

综上,$\psi^*\calf$和$(\oo_X)_{[\theta]}$都是$\psi^*\oo_Y$-模层。最后,我们定义
\[
	\Psi^*\calf=\psi^*(\calf)\otimes_{\psi^*\oo_Y} (\oo_X)_{[\theta]},
\]
它当然是$(\oo_X)_{[\theta]}$-模层,也自然是一个$\oo_X$-模层。

为定义一个函子,我们还需要构造其在态射上的表现。设$u:\calf_1\to\calf_2$是一个$\oo_Y$-模层态射,很容易看到,$\psi^*u :\psi^*\calf_1\to\psi^*\calf_2$是一个$\psi^*\oo_Y$-模层态射。最后,定义$\Psi^*u=\psi^*u\otimes 1$,他就给出了态射$\Psi^*\calf_1\to \Psi^*\calf_2$.
\end{para}

\begin{pro}
在茎上,$(\Psi^*\calf)_x=\calf_{\psi(x)}\otimes_{(\oo_Y)_{\psi(x)}} (\oo_X)_{x}$.
\end{pro}

\begin{proof}
它直接来自于
\[
	(\Psi^*\calf)_x=(\psi^*(\calf)\otimes_{\psi^*\oo_Y} (\oo_X)_{[\theta]})_x=(\psi^*\calf)_x\otimes_{(\oo_Y)_{\psi(x)}}((\oo_X)_{[\theta]})_x=\calf_{\psi(x)}\otimes_{(\oo_Y)_{\psi(x)}}(\oo_X)_x,
\]
注意,此时$(\oo_X)_x$具有来自于$(\theta^\#)_x:(\oo_Y)_{\psi(x)}\to (\oo_X)_x$的$(\oo_Y)_{\psi(x)}$-模结构。
\end{proof}

\begin{para}[张量积的逆像]
考虑两个$\oo_Y$-模层$\calf$和$\calg$,以及态射$\Psi=(\psi,\theta):(X,\oo_X)\to (Y,\oo_Y)$. 由于$\psi^*\calf$和$\psi^*\calg$都具有$\psi^*\oo_Y$-模层结构,所以存在$\psi^*\calf\otimes_{\psi^*\oo_Y}\psi^*\calg$.

固定$X$的子集$U$,我们可以通过$\varphi(U)(s,t)(p)=s(p)\otimes t(p)$
给出一个映射
\[
	\varphi(U):\psi^*\calf(U)\times \psi^*\calg(U) \to \psi^*(\calf\otimes \calg)(U).
\]
这当然是一个双线性映射,且与限制映射相容。于是,张量积泛性质给出了一个典范态射
\[
	\psi^*\calf\otimes_{\psi^*\oo_Y} \psi^*\calg \to \psi^*(\calf\otimes_{\oo_Y} \calg).
\]
这是一个同构,因为在每条茎上它都是同构。进一步与$(\oo_X)_{[\theta]}$取张量积,我们就得到了典范同构
\[
	\Psi^*\calf\otimes_{\oo_X} \Psi^*\calg \to \Psi^*(\calf\otimes_{\oo_Y} \calg)
\]
\end{para}

\section{site与层}

\begin{para}[site]
	site某种程度上是拓扑空间的推广,首先来自于对开覆盖的推广。设$C$是一个范畴,$U$是一个对象,称一族指向$U$的态射$\mathscr U = \{U_i\to U\}_{i\in I}$为$U$的一个覆盖,其中$I$是一个集合。如果一个$C$中覆盖的集合$\mathscr T$满足
	\begin{compactenum}
		\item 若$V\to U$是一个同构,则$\{V\to U\}\in \mathscr T$.
		\item 如果$\{U_i\to U\}_{i\in I}$, $\{U_{ij}\to U_i\}_{j\in J_i} \in \mathscr T$,则$\{U_{ij}\to U\}_{j\in J_i,i\in I}\in \mathscr T$.
		\item 如果$\{U_i\to U\}_{i\in I}\in \mathscr T$且$V\to U$是一个态射,则纤维积$U_i\times_U V$存在且$\{U_i\times_U V\to U\}_{i\in I}\in \mathscr T$.
	\end{compactenum}
	则称$\mathscr T$是一个Grothendieck拓扑,$(C,\mathscr T)$为一个site. 同拓扑空间,我们一般略去$\mathscr T$,只称$C$为一个site. 而给定一个site $C$后,我们用$\operatorname{Cov}(C)$来记其Grothendieck拓扑。
\end{para}

特别地,若$C$就是拓扑空间定义的范畴,则上述site的定义中的纤维积即开集之交,显然,此时site就回到拓扑。

\begin{para}[预层]
	设$C$和$K$是范畴,则反变函子$C\to K$被称为$C$上的一个$K$-预层。这里,我们将预层推广到了任意的范畴上,同时,现在开始我们一般将$K$取成集合范畴。
\end{para}

\begin{para}[层]
	设$\calf$是site $C$上的集合预层,若任取覆盖$\{U_i\to U\}_{i\in I}\in \operatorname{Cov}(C)$,考虑所有纤维积$U_i\times_U U_j$的典范态射(以及相应的对象,再加上恒等态射等)构成的小范畴$\mathscr I$,都有
	\[
	{\varprojlim}_{U_i \in \mathscr{I}}\calf(U_i)=\calf(U),
	\]
	则称$\mathcal F$是一个层。
	
	若用交换图来表示,设$A$是一个集合,而$\calf$是$C$上的一个预层,任取覆盖$\{U_i\to U\}_{i\in I}\in \operatorname{Cov}(C)$,如果存在态射族$\{\varphi_i:A\to \calf(U_i)\}_{i\in I}$使得下图对任意的$i$, $j\in I$都交换(画图时省去了限制映射)
	\[
		\xymatrix{
			A\ar[r]^{\varphi_{j}}\ar[d]_{\varphi_{i}}&\calf(U_j)\ar[d]\\
			\calf(U_i)\ar[r]&\calf(U_{i}\times_U U_{j})
		}
	\]
	则存在唯一的态射$\varphi:A\to \calf(U)$使得下图对任意的$i$, $j\in I$都交换
	\[
		\xymatrix{
			A\ar@/_/[ddr]_{\varphi_{i}}\ar@/^/[drr]^{\varphi_{j}}\ar@{-->}[dr]|{\varphi}&& \\
			&\calf(U)\ar[r]\ar[d]&\calf(U_j)\ar[d]\\
			&\calf(U_i)\ar[r]&\calf(U_{i}\times_U U_{j})
		}
	\]
	这和拓扑空间上的层的定义几乎一模一样,我们只是将“交”换成了纤维积。
\end{para}

\begin{para}[连续函子]
	设$C$和$D$都是site,而$f:C\to D$是一个函子,若对$\operatorname{Cov}(C)$中的任意覆盖$\{U_i\to U\}_{i\in I}$,$\{f(U_i)\to f(U)\}_{i\in I}\in \operatorname{Cov}(D)$,且对$C$中任意态射$V\to U$,态射$f(V\times_U U_i)\to f(V)\times_{f(U)}f(U_i)$都是同构,则称$f$是一个连续函子。
\end{para}

连续函子是拓扑空间间连续映射的原像的类似物。

\begin{para}[推前与拉回]
	设$C$和$D$都是site,而$f:C\to D$是一个连续函子,而$\mathcal F$是$C$上的(预)层,我们定义推前$f_*\calf(U)=\calf(f(U))$,则$f_*\calf$是一个$D$上的(预)层。而拉回$f^*$是$f_*$的左伴随函子。
\end{para}

\begin{para}[拓扑斯]
	设$C$是一个site,拓扑斯是$C$上的(集合)层范畴$\operatorname{Sh}(C)$. 设$C$和$D$都是site,则对应有两个拓扑斯$\operatorname{Sh}(C)$和$\operatorname{Sh}(D)$,则拓扑斯的态射$f:\operatorname{Sh}(D)\to \operatorname{Sh}(C)$由一对伴随函子$f_*:\operatorname{Sh}(D)\to \operatorname{Sh}(C)$和$f^*:\operatorname{Sh}(C)\to \operatorname{Sh}(D)$给出,存在双函子同构
	\[
		\operatorname{Sh}(D)(f^*\mathcal G,\mathcal F)\cong \operatorname{Sh}(C)(\mathcal G,f_*\mathcal F),
	\]
	此外$f^*$是左正和的(与有限极限可交换)。拓扑斯同态$f$和$g$的复合$fg$具体写作$(fg)_*=f_* g_*$与$(fg)^*=g^*f^*$.
\end{para}

site之间的连续函子一定诱导一个拓扑斯的态射,但拓扑斯的态射不一定是由site之间的连续函子诱导的。

\vspace{2cm}

以后再写,这方面最好的参考是SGA 4与The Stacks project的第七章,许多我们在拓扑空间上的层的构造都被推广到site上的层了。

The étale topology is a Grothendieck topology on the category of schemes which has properties similar to the Euclidean topology. 它远比Zariski topology更好。