% !TEX root = main_eng.tex
\renewcommand\chapterimg{Pictures/7.png}
\chapter{Some Definitions}

Throughtout this article, the word ``ring'' will mean a communtative ring with an identity element, and the word ``module'' will mean a double side moulde over a ring.

\section{Sheafs and Ringed Spaces}

\para A topological space $X$ can be seen as a categroy, whose objects are the open subsets of $X$ and the morphisms are defined by
\[
	\Hom_{X}(U,V)=\begin{cases}
	\bigl\{i^U_V:U\hookrightarrow V\bigr\}&\text{, if }U\subset V\text{;}\\
	\varnothing&\text{, otherwise}.
	\end{cases}
\]

\para Suppose $K$ is a category and $X$ is a topological space, a $K$ valued presheaf $\calf$ (simply $K$-presheaf) is a contravariant functor frome $X$ to $K$. As a matter of terminology, if $\calf$ is a presheaf on $X$, we refer to $\calf(U)$ as the sections of the presheaf $\calf$ over the open set $U$, and we sometimes use the notation $\Gamma(\calf,U)$ to denote the object $\calf(U)$. We call the maps $\rho^U_V:=\calf(i^U_V)$ restriction maps, and we sometimes write $s|_V$ instead of $\rho^U_V(s)$, if $s\in \calf(U)$.

\para Suppose $\calf$ is a $K$-presheaf on $X$. The condition that

\textit {`` For any open subset $U$ and its open cover $\{U_\alpha\}_{\alpha\in I}$, if there exists a family of morphisms $\{\varphi_\alpha:A\to \calf(U_\alpha)\}_{\alpha\in I}$ makes the following diagram communtative (we omit the restriction map in the diagram),
\[
	\xymatrix{
		A\ar[r]^{\varphi_{\alpha}}\ar[d]_{\varphi_{\beta}}&\calf(U_\beta)\ar[d]\\
		\calf(U_\alpha)\ar[r]&\calf(U_{\alpha}\cap U_{\beta})
	}
\]
then there exists the unique morphism $\varphi:A\to \calf(U)$ makes the following diagram communtative.''}
\[
	\xymatrix{
		A\ar@/_/[ddr]_{\varphi_{\alpha}}\ar@/^/[drr]^{\varphi_{\beta}}\ar@{.>}[dr]|{\varphi}&& \\
		&\calf(U)\ar[r]\ar[d]&\calf(U_\beta)\ar[d]\\
		&\calf(U_\alpha)\ar[r]&\calf(U_{\alpha}\cap U_{\beta})
	}
\]
is hold, then $K$-persheaf $\calf$ is called a $K$-sheaf. We call this condition the sheaf condition. If the category $K$ is not important, we will omit it.

\para All presheafs on $X$ construct a category, whose morphisms are the natural transformation: Suppose $\calf$ and $\calg$ are two presheafs on $X$, and $\{\varphi(U)\}$ is a family of morphisms make the following diagram communtative,
\[
	\xymatrix{
		\calf(U)\ar[rr]^{\varphi(U)} \ar[d]_{\rho^U_V}&&\calg(U) \ar[d]^{\pi^U_V}\\
		\calf(V)\ar[rr]^{\varphi(V)}&&\calg(V)
	}
\]
then we call it a morphism from $\calf$ to $\calg$ and denote it by $\varphi:\calf\to\calg$. 