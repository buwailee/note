\chapter{向量丛}

\section{向量丛和局部自由模}

在流形语境中,向量丛是局部向量空间的黏合,这章中,我们将讨论概形中的类似物。
这里,很自然的,需要将局部向量空间换成相应的代数对象的谱。我们先复习一下
张量代数和对称代数。

\para[对称代数]
从一个$R$-模$M$出发,不难直接构成出张量代数
\[
    T(M):=\bigoplus_{n\geq 0}M^{\otimes n},
\]
其中乘法取成张量积,这是一个分次$R$-代数。而对称代数即是模去由$m\otimes m'-
m'\otimes m$张成的理想构成的代数,记作$\operatorname{Sym}(M)$,此时乘法都是
交换的。$\operatorname{Sym}(M)$依然是分次的,其中$M$是其中的一次分量。
$\operatorname{Sym}(M)$是$R$-代数范畴的自由对象,他伴随于忘掉$R$-代数结构
只记得$R$-模结构的遗忘函子,即存在自然同构
\[
    \Hom_{R\text{-}\mathsf{Alg}}(\operatorname{Sym}(M),A)\cong
    \Hom_{R\text{-}\mathsf{Mod}}(M,A).
\]
换言之,只要给出所有的$M\to A$的$R$-模同态,就能唯一确定一个$R$-代数同态
$\operatorname{Sym}(M)\to A$. 若$R\to S$是一个环同态,那么自然有同构
\[
    \operatorname{Sym}_R(M)\otimes_R S\cong \operatorname{Sym}_S(M\otimes_R S).
\]
此外,对$M\oplus M'$,我们有自然同构
\[
    \operatorname{Sym}(M\oplus M')\cong \operatorname{Sym}(M)\otimes_R
    \operatorname{Sym}(M').
\]
于是,多项式代数是自由模的对称代数。

现在整体化上述构造,考虑一个环层空间$X$以及一个$\oo_X$-模层$\mathscr E$,
可以定义预层
\[
    U\mapsto
    \operatorname{Sym}_{\mathscr{O}_X(U)}(\mathscr E(U))
\]
伴随的层为$\operatorname{Sym}(\mathscr E)$,称为$\mathscr E$的对称代数。他依然
左伴随于忘掉$\oo_X$-代数结构只记得$\oo_X$-模结构的遗忘函子,即存在自然同构
\[
    \Hom_{\oo_X\text{-}\mathsf{Alg}}(\operatorname{Sym}(\mathscr E),\mathscr A)\cong
    \Hom_{\oo_X\text{-}\mathsf{Mod}}(\mathscr E,\mathscr A).
\]
局部地,在仿射概形$\spec R$上,我们令$\mathscr E$为一个$R$-模$M$对应的模层
$\tilde M$,则不难看到
\[
    \operatorname{Sym}(\tilde M)=\widetilde{\operatorname{Sym}(M)}.
\]
于是,在概形上,任意的拟凝聚层$\mathscr E$的对称代数层$\operatorname{Sym}(\mathscr E)$
都是拟凝聚的。且对于概形间态射$f$,我们有自然同构
\[
    f^*\operatorname{Sym}(\mathscr E)\cong \operatorname{Sym}(f^*\mathscr E).
\]

\para[整体$\spec$] 令$A$是一个$R$-代数,而任意的$R$-概形$X$都有自然同构
\[
    \Hom_{\spec R}(X,\spec A)\cong \Hom_{R\text{-}\mathsf{Alg}}(A,\Gamma(X,\oo_X)),
\]
换言之,$R$-概形$X$是函子$\Hom_{R\text{-}\mathsf{Alg}}(-,\Gamma(X,\oo_X))$的表示对象。

现在令$X$是一个概形,$\mathscr A$是一个拟凝聚代数层,对任意的$X$-概形态射
$f:Y\to X$,函子
\[
    F(T):=\Hom_{\oo_X\text{-}\mathsf{Mod}}(\mathscr A,f_*\oo_Y)
\]
可表,即存在一个$X$-概形$\spec \mathscr A$使得
\[
    \Hom_{X\text{-}\mathsf{Sch}}(Y,\spec \mathscr A)\cong
    \Hom_{\oo_X\text{-}\mathsf{Mod}}(\mathscr A,f_*\oo_Y).
\]
实际上,由于$f^*$和$\Hom$都是左正和的,且$F$满足层公理,所以只要局部检查就好了,
而局部就是上面的情况。这个表示对象就被叫做拟凝聚代数层$\spec \mathscr A$的谱。
局部地,如果$U$是$X$的仿射开集,$U$关于结构态射$h:\spec\mathscr A\to X$的原像即
仿射概形$\spec(\mathscr A(U))$. 

这个构造当然是函子性的,取拟凝聚代数层态射$\mathscr A\to \mathscr B$,则自然给出了
态射$\spec \mathscr B\to \spec \mathscr A$,即存在自然同构
\[
    \Hom_{\oo_X\text{-}\mathsf{Mod}}(\mathscr A,\mathscr B)\cong
    \Hom_{X\text{-}\mathsf{Sch}}(\spec \mathscr B,\spec \mathscr A).
\]
类似于将代数看成相应的模,$h$的直像将$\mathcal O_{\spec \mathscr A}$变成了$\mathscr A$,
即$h_*\mathcal O_{\spec \mathscr A}=\mathscr A$.

\para 对给定的拟凝聚模层$\mathscr E$,定义相应的
\[
    \mathbb V(\mathscr E):=\spec \operatorname{Sym}(\mathscr E).
\]
给定态射$\mathscr E\to \mathscr F$,则自然有态射$\mathbb V(\mathscr F)\to \mathbb V(\mathscr E)$,
这个对应是单的,但不是满的,我们将$\Hom(\mathbb V(\mathscr F),\mathbb V(\mathscr E))$
中这个给出的态射叫做\textit{线性}的。任何$X$-概形态射
$f:\mathbb V(\mathscr F)\to \mathbb V(\mathscr E)$诱导了分次代数同态
$\varphi:\operatorname{Sym}(\mathscr F)\to \operatorname{Sym}(\mathscr E)$,则
$f$是线性的当且仅当$\varphi$作为分次代数同态保持分次。那么,
自然$\varphi=\operatorname{Sym}(\varphi^1)$.

\para[向量丛] 秩$n$的向量丛的定义和流形上的一样,即是一个概形$X$上,存在一族
开覆盖,每个开集$U_\alpha$上有同构$c_\alpha:U_\alpha\to \mathbb A_{U_\alpha}^n$,
并且存在相容性条件,在$U_\alpha\cap U_\beta$,$c_\alpha c_\beta^{-1}$是线性同构。
类似地,不同的坐标册可以给出一样的向量丛构造,所以必要的话可以在定义中模去等价类。
向量丛的态射也是类似的,即是概形态射使得局部上是线性映射。


\begin{pro}
函子$\mathscr E\to \mathbb V(\mathscr E)$是一个反变函子,他给出了秩为$n$的
局部自由层$\mathscr E$和秩为$n$的向量丛$\mathbb V(\mathscr E)$之间的范畴等价。
\end{pro}

证明是直接的,从一个向量丛出发,考虑其截面层,他有一个自然的局部自由
$\mathscr O_X$-模层结构。


\begin{para}[Serre扭层]
    对一个分次$R$-模$M$,我们可以定义一个新的分次$R$-模$M(n)$为$M(n)_d=M_{n+d}$.
    设$R$是一个分次环,并设$X=\proj R$. 我们定义
    $\mathcal O_X(n):=\widetilde{R(n)}$,称为Serre扭层。

    令$f\in R_d$,则对任意的$d$的倍数$n=kd$,在仿射开集$D_+(f)$上,则乘以$f^k$
    给出了同构
    \[
        R_{(f)}=(R_f)_0\to R(n)_{(f)}=(R_f)_n.
    \] 
    特别地,对于我们最关心的情况,$R$是由其一次分量生成的,
    即$R$作为$R_0$-代数是由$\{f_i\in R_1\}$生成的,则$X$是由
    $\{D_+(f_i)\}$所覆盖,而在$D_+(f_i)$上,乘以$f_i^{-n}$给出了同构
    \[
        \mathcal O_X(n)|_{D_+(f_i)}\cong \mathcal O_X|_{D_+(f_i)},
    \]
    所以$\mathcal O_X(n)$是秩为$1$的局部自由模层,或者我们称为一个可逆模层。
    当然,局部自由层就是向量丛,所以使用向量丛的视角,$\mathcal O_X(n)$
    在$D_+(f_i)$上的,是由$f_i^n$生成的$\mathcal O_X|_{D_+(f_i)}$-自由模,
    这是因为任取$n$-次元$a\in R_n$,$a/f_i^n$都是零次的,于是在
    $\mathcal O_X|_{D_+(f_i)}$中。任取$i,j$,在$D_+(f_i)\cap D_+(f_j)$上,
    我们有转移函数$(f_j/f_i)^n$将两个截面联系起来。反过来,如果一族局部截面
    可以通过转移函数粘起来,则他构成了一个整体截面。

    借用向量丛视角,不难看到有同构
    \[
        \mathcal O_X(m)\otimes_{\mathcal O_X}\mathcal O_X(n)
        \to \mathcal O_X(m+n).
    \]
    所以$\mathcal O_X(-n)$就是$\mathcal O_X(n)$的逆。
\end{para}

\section{除子}

\para[正规截面] 设$M$是一个$R$-模,如果对$m\in M$,非零的$a\in R$都有$am\neq 0$,
或者说$\operatorname{ann}_R(m)=\{0\}$,则称$m$是一个正规元。用同态来表示的话,
这就是说$a\mapsto am$是单的。任取概形$X$和上面的拟凝聚层$\mathcal F$,我们可以
类似谈论一个截面$s\in \mathcal F(U)$是否正规。容易看到,截面$s$是正规的,当且
仅当$s_x$对每个$x$都是正规的,也当且仅当,任取仿射开集$U$,$s|_U$是正规的,还
当且仅当由乘以$s$给出的同态$\mathcal O_X\to \mathcal F$是单的。
\endpara

给定一个环$R$,作为$R$-模的正规元等价于$R$的非零因子,此时我们定义
$\operatorname{Frac}(R)=S_R^{-1}R$,其中$S_R$是$R$的非零因子构成的集合。

\para[亚纯函数层]
在概形$X$上,我们定义$S_U$为$U$上所有正规截面的集合,这当然是一个乘性子集,
进而我们定义亚纯函数层$\mathcal K_X$为预层
\[
    U\mapsto S_U^{-1}\Gamma(U,\mathcal O_X)
\]
所伴随的层。一般来说,上面的这个预层不是层。如果$X$是整概形,则$X$有一个
唯一的一般点$\eta$,我们定义域$K(X)=\mathcal O_{X,\eta}$. 不难检查,此时
$\mathcal K_X$就是常值层$U\mapsto K(X)$,这是因为每个$\Gamma(U,\mathcal O_X)$
都是整环,放到仿射覆盖上,$\eta$对应到整环的零点,在零点处的局部化给出这个
整环的分式域,而整环的正规元就是所有的非零元,故而
$S_U^{-1}\Gamma(U,\mathcal O_X)$也就是该整环的分式域。
\endpara

由于$S_U$都是$\Gamma(U,\mathcal O_X)$中的非零因子,所以$\mathcal O_{X,x}$
到$\mathcal K_{X,x}$显然是个单射,进而$\mathcal O_{X}\to \mathcal K_{X}$
也是单的,所以我们可以将$\mathcal O_{X}$看成$\mathcal K_{X}$的子层。令
$f=\Gamma(X,\mathcal K_{X})$,我们定义
\[
    \operatorname{dom}(f):=\{x\,:\,f_x\in \mathcal O_{X,x}
    \subset \mathcal K_{X,x}\}.
\]

\begin{lem}
    $\operatorname{dom}(f)$是概形式稠密开子集
\end{lem}

\begin{proof}
首先如果
$f_x\in \mathcal O_{X,x}$,则$f_x$可以被某个$(U,f)$表示,于是在$y\in U$上都有
$f_y\in \mathcal O_{X,x}$,所以$\operatorname{dom}(f)$是开集。现在任取
$x\in \operatorname{dom}(f)$以及某个$f_x=(U,f)\in K_{X,x}$,这里我们将代表元
中的$U$选成一个仿射开子集,于是存在一个非零因子$g$(即正规元)使得$gf \in 
\Gamma(U,\mathcal O_X)$. 由于$g$是非零因子,则$D(g)$在$U$中概形式稠密,且
$D(g)\subset \operatorname{dom}(f)\cap U$,进而$\operatorname{dom}(f)\cap U$
在$U$中概形式稠密,进而$\operatorname{dom}(f)$是概形式稠密的。
\end{proof}

\para[有理函数层]
    在概形$X$上,我们定义层$U\mapsto R(U)$为有理函数层,这里$R(U)$是$U$上
    有理函数构成的环。

    为看到这是一个层,取$U$的一个开覆盖$\{U_i\}$,如果一族截面$s_i\in R(U_i)$
    有$s_i|_{U_i\cap U_j}=s_j|_{U_i\cap U_j}$,我们需要证明存在一个$s\in R(U)$
    使得$s_i=s|_{U_i}$. 事实上,选每个$s_i\in R(U_i)$的代表元$(V_i,t_i)$,
    其中$V_i$是$U_i$的概形式稠密开子集,而$t_i$是$\mathcal O_X$在$V_i$
    上的截面。我们只需证明$t_i|_{V_i\cap V_j}=t_j|_{V_i\cap V_j}$,进而
    可以在$V=\bigcup_i V_i$上粘出来一个截面$t$,则$(V,t)\in R(U)$就是我们想要的
    代表元。由于
    $s_i|_{U_i\cap U_j}=s_j|_{U_i\cap U_j}$,所以存在一个概形式稠密开子集
    $W\subset U_i\cap U_j$使得代表元在上面相同,与$V_i\cap V_j$再交一交,
    我们可知$t_i$和$t_j$在$V_i\cap V_j$的某个概形式稠密开子集上相同,继而
    在$V_i\cap V_j$上相同。
\endpara

给定一个亚纯函数$f\in \Gamma(X,\mathcal K_X)$,我们可以定义一个有理函数,
其代表元为$(\operatorname{dom}(f),f|_{\operatorname{dom}(f)})$. 类似地,
这定义了同态$\alpha_U:\Gamma(U,\mathcal K_X)\to R(U)$,而且是单同态,因为
概形式稠密,于是这给出了层单态射
\[
    \alpha:\mathcal K_X\to \mathcal R_X.
\]
当$X$是整概形的时候,这个当然是个同构。

\begin{pro}
    如果$X$是局部Noether概形,则
    $\mathcal K_X(U)=S_U^{-1}\Gamma(U,\mathcal O_X)$,且
    $\alpha:\mathcal K_X\to \mathcal R_X$是一个同构,
    即亚纯函数层和有理函数层相同。
\end{pro}

\begin{proof}
    记预层$U\mapsto S_U^{-1}\Gamma(U,\mathcal O_X)$为$\mathcal K'_X$,复合上
    层化和$\alpha$,我们有环同态$\mathcal K'_X(U)\to R(U)$. 现在找一个仿射
    开覆盖,每个仿射开集当然都是Noether环的谱,此时
    \[
        \mathcal K'_X(U)=\varinjlim_{t} \Gamma(D(t),\mathcal O_X),\quad 
        R(U)=\varinjlim_{V} \Gamma(V,\mathcal O_X),
    \]
    其中$t$跑遍$\Gamma(U,\mathcal O_X)$中的正规元(非零因子),
    $V$跑遍$U$中的概形式稠密开子集。
    由于$D(t)$是概形式稠密的,且对于Noether环的谱,$U$中某个概形式稠密开子集
    都包含一个非零因子$t$的$D(t)$. 所以两个余极限相同,$\mathcal K'_X(U)=R(U)$.
    进而$\mathcal R_X=\mathcal K'_X=\mathcal K_X$.
\end{proof}

\begin{para}[Cartier除子]
    设$X$是一个概形,$\mathcal K_X^\times /\mathcal O_X^\times$的一个整体截面
    被称为一个Cartier除子,其中$\mathcal K_X^\times$是$\mathcal K_X$的可逆截面
    构成的层,$\mathcal O_X^\times$也类似,我们记
    \[
        \operatorname{Div}(X)=
        \Gamma(X,\mathcal K_X^\times /\mathcal O_X^\times).
    \]
    这是一个交换群,来自于可逆元的乘法,但是我们后面记作加法。
    显然存在一个典范同态
    \[
        \operatorname{div}:\Gamma(X,\mathcal K_X^\times)\to 
        \operatorname{Div}(X).
    \]
    这个同态的像中的除子都被称为主除子。如果两个除子之差为\textit{主}除子,即
    $D_1-D_2=\operatorname{div}(f)$,则称$D_1$和$D_2$\textit{线性等价}。
    最后,我们定义
    \[
        \operatorname{Div}_+(X)=
        \Gamma(X,(\mathcal K_X^\times \cap \mathcal O_X) /\mathcal O_X^\times).
    \]
    注意到$\mathcal K_X^\times \cap \mathcal O_X\neq \mathcal O_X^\times$是因为
    前者有正规元(非零因子),后者只有可逆元。$\operatorname{Div}_+(X)$中的除子
    被称为\textit{有效}除子。如果$D_1-D_2\in \operatorname{Div}_+(X)$,则记
    $D_1\geq D_2$(或者$D_2\leq D_1$).
\end{para}

\begin{para}[Cartier除子与线丛]
    给一个除子$D\in \operatorname{Div}(X)$,等价于说个开覆盖$\{U_i\}$,并
    在每个$U_i$上有一个可逆亚纯函数$f_i\in \Gamma(U_i,\mathcal K_X^\times)$,
    使得在$U_i\cap U_j$上,
    转移函数$f_if_j^{-1}\in \Gamma(U_i\cap U_j,\mathcal O_X^\times)$.
    于是,这定义了一个$\mathcal K_X$的秩为$1$的局部自由
    $\mathcal O_X$-子模层$\mathcal L(D)$,他在$U_i$上的
    $\Gamma(U_i,\mathcal O_X)$-模是由$f_i$生成的模。

    反过来,给定一个$\mathcal K_X$的秩为$1$的局部自由$\mathcal O_X$-子模层
    $\mathcal L$,在点$x\in X$,$\mathcal L_x$的生成元记作$f$. 由于
    $f\in \mathcal K_{X,x}$,所以存在一个非零因子$g\in \mathcal O_{X,x}$
    使得$fg\in \mathcal O_{X,x}$. 由于这是自由模,
    $f$乘以$\mathcal O_{X,x}$中的任意非零元都不为零,$fg$也是,故$fg$是一个
    非零因子,他在$\mathcal K_{X,x}$中可逆,故$f\in \mathcal K_{X,x}^\times$.
    对于任意的开集上的截面,我们也不难粘合出一个逆,所以找一个开覆盖$\{U_i\}$,
    以及$\mathcal L$在每个开集$U_i$上的生成元$f_i$,
    我们有$f_i\in \Gamma(U_i,\mathcal K_{X,x}^\times)$,并且在$U_i\cap U_j$上
    转移函数$f_if_j^{-1}\in \Gamma(U_i\cap U_j,\mathcal O_X^\times)$. 这样,
    我们就给出了一个除子。

    现在,给定一个除子$D=\{(U_i,f_i)\}$,我们定义一个与之相关的线丛
    \[
        \mathcal O_X(D)|_{U_i}=f_i^{-1}\mathcal O_X|_{U_i},
    \]
    注意我们这里有个逆。容易看到
    \[
        D_1\leq D_2\quad \Leftrightarrow \quad \mathcal O_X(D_1)\subset \mathcal O_X(D_2),
    \]
    以及
    \begin{equation}
        \mathcal O_X(D_1+D_2)=\mathcal O_X(D_1)\mathcal O_X(D_2)\cong 
        \mathcal O_X(D_1)\otimes_{\mathcal O_X} \mathcal O_X(D_2),
    \end{equation}
    这里第一个乘法是在$\mathcal K_X$上的。除子的加法单位元是$0$,它对应于
    一个$\mathcal O_X^\times$的截面,比如常值截面$1$,此时容易看到
    \[
        \mathcal O_X(0)=\mathcal O_X.
    \]
    所以,一个有效除子$D\geq 0$就是说$\mathcal O_X(-D)\subset \mathcal O_X$
    是一个理想层,他定义了一个闭子概形$Z$,这将在一些情况中回到所谓的Weyl除子。
\end{para}

式(\theequation)告诉我们$D\mapsto \mathcal O_X(D)$构成了一个交换群同态。

\begin{lem}
    除子$D$是主除子当且仅当$\mathcal O_X(D)\cong \mathcal O_X=\mathcal O_X(0)$.
    于是,$D_1$和$D_2$线性等价当且仅当$\mathcal O_X(D_1)\cong \mathcal O_X(D_2)$.
\end{lem}

\begin{proof}
    如果$D$是主除子,即$D=\operatorname{div}(f)$,则
    $\mathcal O_X(D)=f^{-1}\mathcal O_X\cong \mathcal O_X$.
    反过来,如果$\mathcal O_X\cong \mathcal O_X(D)$,则左侧常值截面$1$
    在右侧的像给出一个截面$f\in \Gamma(X,\mathcal K_X^\times)$,
    于是$D=\operatorname{div}(f^{-1})$. 第二个命题直接来自于
    $\mathcal O_X(D+\operatorname{div}(f))\cong 
    \mathcal O_X(D)\otimes_{\mathcal O_X} \mathcal O_X\cong \mathcal O_X(D)$.
\end{proof}

考虑层的短正和列
\[
    1\to \mathcal O_X^\times \to \mathcal K_X^\times \to
    \mathcal K_X^\times/\mathcal O_X^\times \to 1,
\]
作用上左正和函子$\Gamma(X,*)$,我们有正和列
\[
    1\to \Gamma(X,\mathcal O_X^\times) \to \Gamma(X,\mathcal K_X^\times) \to
    \operatorname{Div}(X).
\]
如果我们考虑更高阶的上同调($\Gamma$看成$H^0$),则我们可以进一步推广到长正和列
\[
    1\to \Gamma(X,\mathcal O_X^\times) \to \Gamma(X,\mathcal K_X^\times) \to
    \operatorname{Div}(X)\xrightarrow{\delta} H^1(X,\mathcal O_X^\times)
    \to \cdots,
\]
这里的$\delta$实际上就是$D\mapsto \mathcal O_X(D)$,容易看到$\ker \delta$
就是所有的主除子。这里的$H^1(X,\mathcal O_X^\times)$就是所谓的Picard群,记作
$\operatorname{Pic}(X)$,利用\v{C}ech上同调,$\operatorname{Pic}(X)$也可以是
所有$X$上可逆$\mathcal O_X$-模层模掉同构类给出的交换群,乘法是张量积。
同时,由于$\delta$一般不是满的,则$\operatorname{Pic}(X)$一般不等于
$\operatorname{Div}(X)/\Gamma(X,\mathcal K_X^\times)$,后者是除子模掉主除子
构成的交换群。

\begin{para}[支集]
    给定一个除子$D$,我们定义他的支集为
    \[
        \supp(D)=\{x\in X\,:\,D_x \neq 1\}.
    \]
    这里的$1$对应的是乘法的单位元,在除子上我们是看成加法群的“零”,
    所以支集是“非零”的$D_x$的$x$的集合,和其他支集的定义完全类似。
    $\supp(D)$显然是个闭集,因为如果$D_x=1$,则在$x$附近,$D=1$,这就说明
    $\supp(D)$的补集是开的。
\end{para}

\begin{para}[Weyl除子]
令$X$为一个Noether概形,令$C$为他的一个整闭子概形,一般点为$\eta$,我们记
局部环$\mathcal O_{X,C}:=\mathcal O_{X,\eta}$,则此时余维数
$\codim_X C=\dim \mathcal O_{X,C}$. 更一般地,任取$X$的闭子集$Z$,
他的余维数满足$\codim_X Z=\inf_{C\subset Z} \codim_X C$,
这里的$C$跑遍$Z$的不可约分支。
我们记$Z^k(X)$为$X$余维数为$k$的整闭子概形生成的自由群,特别地将$Z^1(X)$中的
元素称为Weyl除子,单个的余一维的整闭子概形叫做素除子,而非负系数的线性组合
叫做有效Weyl除子,我们记$Z^1_+(X)$为所有有效Weyl除子构成的子集。
\end{para}

\begin{lem}
    设$R$是一个一维Noether局部环,而$f$, $g$都不是零因子,则
    $R/\langle f\rangle$是有限长的,且我们有长度的等式
    \[
        \operatorname{length}(R/\langle fg\rangle) =
        \operatorname{length}(R/\langle f\rangle) +
        \operatorname{length}(R/\langle g\rangle).
    \]
\end{lem}
    
通过这个引理,我们可以延拓$f\mapsto \operatorname{length}(R/\langle f\rangle)$
到群同态 $\operatorname{Frac}(R)^\times /R^\times \to \mathbb Z$,记作
$\operatorname{mult}_R$. 注意到,如果$f$和$g$都不是零因子,则
$\operatorname{mult}_R(f/g)=\operatorname{mult}_R(f)-\operatorname{mult}_R(g)$.

\begin{proof}
    因为$f$不是零因子,所以不存在$R$的极小素理想$\mathfrak p$包含他。
    这是因为,$R$的极小素理想都在$\operatorname{Ass}(R)$中,而他们形如
    $\operatorname{ann}(r)$. 于是,$R/\langle f\rangle$是零维的,
    是一个Artin环,所以是有限长的。
    
    现在,考虑短正和列
    \[
        0\to \langle g\rangle /\langle fg\rangle \to 
        R/\langle fg\rangle\to R/\langle g\rangle \to 0,
    \]
    并注意到因为$g$不是零因子,由$r\mapsto gr$给出的$R$-模同态
    \[
        R/\langle f\rangle \to\langle g\rangle/\langle fg\rangle
    \]
    是一个同构,于是我们就得到了结论。
\end{proof}

令$X$是一个局部Noether概形,而$D\in \operatorname{Div}(X)$是一个Cartier除子。
对于任意的余一维的整闭子概形$C$,则$\mathcal O_{X,C}$是一个一维Noether局部环,
而$D$在$C$的一般点$\eta$处属于
$(\mathcal K^\times_X/\mathcal O_X^\times)_\eta=
\operatorname{Frac} (\mathcal O_{X,C})^\times/\mathcal O_{X,C}^\times$,
于是我们定义
\[
    \operatorname{mult}_C(D):= \operatorname{mult}_{\mathcal O_{X,C}}(D_x).
\]

\begin{lem}
若$X$是一个局部Noether概形,而$D\in \operatorname{Div}(X)$是一个Cartier除子,
则$\supp(D)$的余维数大于等于一,即$\codim_X(\supp(D))\geq 1$,或者说,
任取$x\in \supp(D)$,$\dim(\mathcal O_{X,x})\geq 1$.
\end{lem}

\begin{proof}
令$x\in X$有$\dim(\mathcal O_{X,x})=0$. 此时$\mathcal O_{X,x}$是Artin环,
于是所有$\mathcal O_{X,x}$中的非零因子都是可逆的,所以
$\mathcal O_{X,x}=\mathcal K_{X,x}$,那么$D_x=1$,推出$x\not\in \supp(D)$.
\end{proof}

\begin{lem}
给定Noether概形$X$上的Cartier除子$D$,则使得$\operatorname{mult}_C(D)\neq 0$的
余一维的整闭子概形$C$的数目是有限的。
\end{lem}

\begin{proof}
由于$\supp(D)$是闭集,所以$C$不包含于$\supp(D)$,则$C$的一般点$\eta$不属于
$\supp(D)$,此时$D_\eta=1$,进而$\operatorname{mult}_C(D)=0$. 现在,如果
$C$包含于$\supp(D)$,由于$X$是Noether空间,则$\supp(D)$也是,只有有限个
不可约分支。由于$C$是不可约的,所以$C$必然包含于$\supp(D)$的某个不可约分支$Z$中,
如果$C\neq Z$,且$Z$的余维数大于等于一,所以存在$X$的不可约闭子集$Z'$,使得
$C\subsetneq Z\subsetneq Z'$,这就告诉我们$\codim_X C\geq 2$,这是不可能的,
所以$C$就是$\supp(D)$的某个不可约分支,进而有限。
\end{proof}

\begin{para}[Cartier除子与Weyl除子]
    给定一个Cartier除子$D$,我们定义一个Weyl除子$\operatorname{cyc}(D)$为
    \[
        \operatorname{cyc}(D)=\sum_C \operatorname{mult}_C(D) [C],
    \]
    这个求和根据上面的引理是有限的。
\end{para}