\chapter{向量丛}

\section{向量丛和局部自由模}

在流形语境中,向量丛是局部向量空间的黏合,这章中,我们将讨论概形中的类似物。
这里,很自然的,需要将局部向量空间换成相应的代数对象的谱。我们先复习一下
张量代数和对称代数。

\para[对称代数]
从一个$R$-模$M$出发,不难直接构成出张量代数
\[
    T(M):=\bigoplus_{n\geq 0}M^{\otimes n},
\]
其中乘法取成张量积,这是一个分次$R$-代数。而对称代数即是模去由$m\otimes m'-
m'\otimes m$张成的理想构成的代数,记作$\operatorname{Sym}(M)$,此时乘法都是
交换的。$\operatorname{Sym}(M)$依然是分次的,其中$M$是其中的一次分量。
$\operatorname{Sym}(M)$是$R$-代数范畴的自由对象,他伴随于忘掉$R$-代数结构
只记得$R$-模结构的遗忘函子,即存在自然同构
\[
    \Hom_{R\text{-}\mathsf{Alg}}(\operatorname{Sym}(M),A)\cong
    \Hom_{R\text{-}\mathsf{Mod}}(M,A).
\]
换言之,只要给出所有的$M\to A$的$R$-模同态,就能唯一确定一个$R$-代数同态
$\operatorname{Sym}(M)\to A$. 若$R\to S$是一个环同态,那么自然有同构
\[
    \operatorname{Sym}_R(M)\otimes_R S\cong \operatorname{Sym}_S(M\otimes_R S).
\]
此外,对$M\oplus M'$,我们有自然同构
\[
    \operatorname{Sym}(M\oplus M')\cong \operatorname{Sym}(M)\otimes_R
    \operatorname{Sym}(M').
\]
于是,多项式代数是自由模的对称代数。

现在整体化上述构造,考虑一个环层空间$X$以及一个$\oo_X$-模层$\mathscr E$,
可以定义预层
\[
    U\mapsto
    \operatorname{Sym}_{\mathscr{O}_X(U)}(\mathscr E(U))
\]
伴随的层为$\operatorname{Sym}(\mathscr E)$,称为$\mathscr E$的对称代数。他依然
左伴随于忘掉$\oo_X$-代数结构只记得$\oo_X$-模结构的遗忘函子,即存在自然同构
\[
    \Hom_{\oo_X\text{-}\mathsf{Alg}}(\operatorname{Sym}(\mathscr E),\mathscr A)\cong
    \Hom_{\oo_X\text{-}\mathsf{Mod}}(\mathscr E,\mathscr A).
\]
局部地,在仿射概形$\spec R$上,我们令$\mathscr E$为一个$R$-模$M$对应的模层
$\tilde M$,则不难看到
\[
    \operatorname{Sym}(\tilde M)=\widetilde{\operatorname{Sym}(M)}.
\]
于是,在概形上,任意的拟凝聚层$\mathscr E$的对称代数层$\operatorname{Sym}(\mathscr E)$
都是拟凝聚的。且对于概形间态射$f$,我们有自然同构
\[
    f^*\operatorname{Sym}(\mathscr E)\cong \operatorname{Sym}(f^*\mathscr E).
\]

\para[整体$\spec$] 令$A$是一个$R$-代数,而任意的$R$-概形$X$都有自然同构
\[
    \Hom_{\spec R}(X,\spec A)\cong \Hom_{R\text{-}\mathsf{Alg}}(A,\Gamma(X,\oo_X)),
\]
换言之,$R$-概形$X$是函子$\Hom_{R\text{-}\mathsf{Alg}}(-,\Gamma(X,\oo_X))$的表示对象。

现在令$X$是一个概形,$\mathscr A$是一个拟凝聚代数层,对任意的$X$-概形态射
$f:Y\to X$,函子
\[
    F(T):=\Hom_{\oo_X\text{-}\mathsf{Mod}}(\mathscr A,f_*\oo_Y)
\]
可表,即存在一个$X$-概形$\spec \mathscr A$使得
\[
    \Hom_{X\text{-}\mathsf{Sch}}(Y,\spec \mathscr A)\cong
    \Hom_{\oo_X\text{-}\mathsf{Mod}}(\mathscr A,f_*\oo_Y).
\]
实际上,由于$f^*$和$\Hom$都是左正和的,且$F$满足层公理,所以只要局部检查就好了,
而局部就是上面的情况。这个表示对象就被叫做拟凝聚代数层$\spec \mathscr A$的谱。
局部地,如果$U$是$X$的仿射开集,$U$关于结构态射$h:\spec\mathscr A\to X$的原像即
仿射概形$\spec(\mathscr A(U))$. 

这个构造当然是函子性的,取拟凝聚代数层态射$\mathscr A\to \mathscr B$,则自然给出了
态射$\spec \mathscr B\to \spec \mathscr A$,即存在自然同构
\[
    \Hom_{\oo_X\text{-}\mathsf{Mod}}(\mathscr A,\mathscr B)\cong
    \Hom_{X\text{-}\mathsf{Sch}}(\spec \mathscr B,\spec \mathscr A).
\]
类似于将代数看成相应的模,$h$的直像将$\mathcal O_{\spec \mathscr A}$变成了$\mathscr A$,
即$h_*\mathcal O_{\spec \mathscr A}=\mathscr A$.

\para 对给定的拟凝聚模层$\mathscr E$,定义相应的
\[
    \mathbb V(\mathscr E):=\spec \operatorname{Sym}(\mathscr E).
\]
给定态射$\mathscr E\to \mathscr F$,则自然有态射$\mathbb V(\mathscr F)\to \mathbb V(\mathscr E)$,
这个对应是单的,但不是满的,我们将$\Hom(\mathbb V(\mathscr F),\mathbb V(\mathscr E))$
中这个给出的态射叫做\textit{线性}的。任何$X$-概形态射
$f:\mathbb V(\mathscr F)\to \mathbb V(\mathscr E)$诱导了分次代数同态
$\varphi:\operatorname{Sym}(\mathscr F)\to \operatorname{Sym}(\mathscr E)$,则
$f$是线性的当且仅当$\varphi$作为分次代数同态保持分次。那么,
自然$\varphi=\operatorname{Sym}(\varphi^1)$.

\para[向量丛] 秩$n$的向量丛的定义和流形上的一样,即是一个概形$X$上,存在一族
开覆盖,每个开集$U_\alpha$上有同构$c_\alpha:U_\alpha\to \mathbb A_{U_\alpha}^n$,
并且存在相容性条件,在$U_\alpha\cap U_\beta$,$c_\alpha c_\beta^{-1}$是线性同构。
类似地,不同的坐标册可以给出一样的向量丛构造,所以必要的话可以在定义中模去等价类。
向量丛的态射也是类似的,即是概形态射使得局部上是线性映射。


\begin{pro}
函子$\mathscr E\to \mathbb V(\mathscr E)$是一个反变函子,他给出了秩为$n$的
局部自由层$\mathscr E$和秩为$n$的向量丛$\mathbb V(\mathscr E)$之间的范畴等价。
\end{pro}

证明是直接的,从一个向量丛出发,考虑其截面层,他有一个自然的局部自由
$\mathscr O_X$-模层结构。