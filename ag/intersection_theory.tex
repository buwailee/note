\chapter{相交理论}

\section{背景知识}

从这章开始,我们默认概形是一个特征为零的代数闭域上的有限型可分概形。
一个概形是整的,如果他是不可约且既约的,而一个簇是指一个整概形。
子概形默认是闭子概形,层则默认是凝聚层。

有限型假设可以保证任意子概形都存在一个准素分解,即他可以写成有限个准素子概形的并。
局部地,这就是说定义子概形的理想可以写成有限个准素理想的交。如同环中一样,
准素分解也并不唯一,但是极小的准素理想的集合是唯一的。反过来到概形的情况,
子概形可以写成一族极大准素子概形的并,这些极大准素子概形被叫做该子概形的
不可约分支,而其他的准素子概形被叫做嵌入分支。

若$X$是一个概形,其任意不可约闭子集$Y$(特别地,整子概形,即子簇)都有唯一的一般点$y$.
对仿射概形这是平凡的,对一般的概形,对任意的仿射开集$U$,则$Y\cap U$如果非空
他则是不可约的,于是在仿射开集$U$中,$Y\cap U$有一个一般点$y$,继而他在$X$中的闭包
包含了$Y\cap U$,实际上只能是$Y$. 否则$Y$可以写成非空闭集$Y-(Y\cap U)$和
$\overline{\{y\}}$的并。同时,概形的分离性保证了这个一般点是唯一的。
于是,我们可以定义局部环$\mathcal O_{X,Y}:=\mathcal O_{X,y}$,称为$X$沿着$Y$的
局部环。此外,如果我们有$\oo_X$模层$\mathcal F$,则我们可以定义$\mathcal F_Y$
为$\oo_{X,Y}$-模$\mathcal F_y$.

概形$X$上的\textit{线性系统},是一个线丛$\mathcal L$(秩为1的局部自由层)和
一个截面的向量空间$V\subset H^0(\mathcal L)$的偶对$(\mathcal L,V)$. 对每个截面
$\sigma$,我们定义其零点集为$V(\sigma)$. 等价地,我们也可以认为其为一族由
$\mathbb PV$所参数化的除子$\{V(\sigma)\,:\,\sigma\in V\}$(射影化是因为零点允许
差一个相乘非零常数),继而我们定义这个线性系统的维数为$\dim V-1$. 一维的线性
系统被称为\textit{束}(pencil),二维的叫做net,三维的叫做web.

\begin{pro}[主理想定理]
    如果$f:Y\to X$是一个簇间的态射,且$X$是光滑的,则对任何$X$的子簇$Z$,都有
    \[
        \operatorname{codim} f^{-1}(Z)\leq \operatorname{codim} Z.
    \]
    特别地,如果$A$, $B$是$X$的子簇,$C$为$A\cap B$的不可约分支,则
    \[
        \operatorname{codim} C\leq \operatorname{codim} A+\operatorname{codim} B.
    \]
\end{pro}

\section{Chow环}