% !TEX root = main.tex

\chapter{一些态射的性质}

\section{分离性}

回忆,在拓扑空间范畴中,一个拓扑空间$X$是Hausdorff的,当且仅当对角线
$\Delta_X:=\{(x,x)|x\in X\}\subset X\times X$在乘积空间$X\times X$的
一个闭子集。对于概形来说,他的拓扑空间几乎不会是Hausdorff的,但是
我们可以将上面的刻画直接挪到概形上,它具有相似的效果,被称为分离性。

\begin{para}[对角线、图像、等值子]
	设$X$是一个$S$-概形,则态射$\id_X\times_S\id_X:X\to X\times_S X$
	被称为对角线态射,记作$\Delta_{X|S}$,如果没有误解,则简单记作
	$\Delta_X$或$\Delta$. 任取$f:X\to Y$为一个态射,则
	$\Gamma_f:=\id_X\times_S f:X\to X\times_S Y$被称为$f$的图像。
	令$f$, $g:X\to Y$为两个$S$-概形态射,则态射$i: K\to X$被称为
	等值子,如果其对任意的$S$-概形$T$诱导了双射
	\[
		K_S(T)\xrightarrow{\sim} \{x\in X_S(T)\,:\, f(T)(x)=g(T)(x)\}.
	\]
	我们记$K$为$\operatorname{Eq}(f,g)_S$,或者如果没有误解,写作
	$\operatorname{Eq}(f,g)$.

	所有构造都是纯范畴的,即只依赖于纤维积的泛性质。利用Yoneda引理,
	不难证明以下命题,
	$i:\operatorname{Eq}(f,g)_S\to X$是$\Gamma_f:X\to X\times_S Y$
	和$\Gamma_g:X\to X\times_S Y$的纤维积。
\end{para}

仿射情况是值得标记的:若$X=\spec B$,$S=\spec A$,则
$X\times_S X=\spec (B\otimes_A B)$,此时$\Delta$是同态
$\varphi:a\otimes b\mapsto ab$对应的态射$\widetilde{\varphi}$.

\begin{lem}\label{lem:4.1.2}
	如果$X$, $S$是仿射概形,则$\Delta_{X|S}$是一个闭浸入。
\end{lem}

\begin{proof}
由于$\varphi:a\otimes b\mapsto ab$是满的,利用
Proposition \ref{pro:3.4.11},我们只要说明$\Delta=\widetilde{\varphi}$
是一个到$\spec B$到$\spec (B\otimes_A B)$中闭集
$\im \widetilde{\varphi}$的同胚即可,同胚直接来自于$\varphi$是满的
以及Lemma \ref{lem:3.9}. 闭集部分也不难,同构基本定理告诉我们,
$\varphi$可以分解为一个同构$\psi:(B\otimes_A B)/\ker \varphi\to B$
和商映射$\pi:B\otimes_A B\to (B\otimes_A B)/\ker \varphi$的复合,
所以,$\im \widetilde{\pi}=V(\ker \varphi)$是一个闭集。此外,由于
\[
\widetilde{\varphi}=\widetilde{\psi\pi}=\widetilde{\pi}\widetilde{\psi}
\]
以及$\widetilde{\psi}$是同胚,所以
$\im \widetilde{\varphi}=\im \widetilde{\pi}=V(\ker \varphi)$是一个闭集。
\end{proof}

\begin{pro}\label{pro:4.1.3}
以下命题成立:
\begin{compactenum}
\item 设$X$, $Y$是两个$S$-概形,如果我们典范等同
$(X\times Y)\times (X\times Y)$和$(X\times X)\times (Y\times Y)$,
则态射$\Delta_{X\times Y}$等同于$\Delta_X\times \Delta_Y$.	
\item 对于任意的基概形扩张$S\to S'$,我们有
$\Delta_{X_{(S')}}=(\Delta_{X})_{(S')}$.
\item 态射$f:X\to Y$是单态射,
当且仅当$\Delta_{X|Y}:X\to X\times_Y X$是一个同构。
\item 设有$S$-概形$f:X\to S$, $g:Y\to S$,以及态射$\varphi:S\to T$,
它使得$X$, $Y$都变成了$T$-概形。记$p$, $q$为$X\times_S Y$的投影,
此时,我们有交换图
\[
	\begin{xy}
	\xymatrix{
	X\times_S Y\ar[rr]^{p\times_T q}\ar[d]_{\pi}&&X\times_T Y\ar[d]^{f\times_T g}\\
	S\ar[rr]^{\Delta_{S|T}}&&S\times_TS
	}
	\end{xy}
\]
其中$\pi=fp=gq$. 同时,$X\times_S Y$就是$S\times_T S$-概形
$X\times_T Y$和$S$的乘积,而$\pi$和$p\times_T q$是两个典范投影。
\end{compactenum}	
\end{pro}

\begin{proof}
这些都是在任意(纤维积存在的)范畴成立的东西,这里就不那么细致检验了。
这里的第二点直接来自于第一点,来自于典范等同
$(X\times_S X)_{(S')}=X_{(S')}\times_{S'}X_{(S')}$.

第四点的交换图这里不检验,至于$X\times_S Y$就是$S\times_T S$-概形
$X\times_T Y$和$S$的乘积这点,我们可以直接检查泛性质。
设$\alpha:A\to X \times_T Y$, $\beta:A\to S$满足交换图,
即$\Delta\beta=(f\times_T g)\alpha$. 设$p_T$, $q_T$是对应
$X\times_T Y$的投影,则我们有$p_T\alpha:A\to X$, $q_T\alpha:A\to Y$. 
在$\Delta\beta=(f\times_T g)\alpha$两边作用$S\times_T S$的投影后立刻得到
\[
	fp_T\alpha=\beta=gq_T\alpha:A\to S.
\]
于是$X\times_S Y$告诉我们存在唯一的$h:A\to X\times_S Y$使得
$p_T\alpha=ph$以及$q_T\alpha=qh$成立,此时
\[
	\pi h=fph=fp_T\alpha=\beta, \quad 
	(p\times_T q)h=(p_T\alpha)\times_T (q_T\alpha)=\alpha,
\]
此即乘积所需的泛性质。
\end{proof}

利用Proposition \ref{pro:3.2.2},第四点告诉我们,此时$p\times_T q$
可以典范等同于$\id_{X\times_T Y}\times_{P}\Delta_{S|T}$,
其中$P=S\times_T S$. 

\begin{lem}\label{lem:4.1.4}
	对角线、图像、等值子态射在概形范畴都是浸入。
\end{lem}

对角线态射对应的子概形也被称$X\times_S X$的对角线。

\begin{proof}
由于后两者都可以通过对对角线态射基变换得到,而浸入的基变换还是浸入,
所以只需证明对角线态射是浸入即可。
设$f:X\to S$是一个$S$-概形,$\pi_1$, $\pi_2$是$X\times_S X$到$X$的
两个典范投影。给一个$S$的仿射开覆盖$\{S_\alpha\}_\alpha$,
对每一个开集$f^{-1}(S_\alpha)$,我们都可以找相应的仿射开覆盖
$\{X_{\alpha\beta}\}_\beta$. 由于$f(X_{\alpha\beta})\subset S_{\alpha}$,
所以将其限制在$X_{\alpha\beta}$,我们就给出了$X_{\alpha\beta}$的
$S_\alpha$概形结构,同时Propostition \ref{pro:3.2.2}告诉我们,
\[
	X_{\alpha\beta}\times_{S_\alpha}X_{\alpha\gamma}=
	X_{\alpha\beta}\times_{S}X_{\alpha\gamma},
\]
且Corollary \ref{coro:3.2.8}告诉我们右侧是$X\times_S X$的一个开集
$\pi_1^{-1}(X_{\alpha\beta})\cap \pi_2^{-1}(X_{\alpha\gamma})$
,而且左侧告诉我们这是仿射的。
由于任取$p\in \im \Delta$,我们有$\pi_1(p)=\pi_2(p)$,
所以只要找一个包含$\pi_1(p)$的开集$X_{\alpha\beta}$,我们有
\[
p\in \pi_1^{-1}(X_{\alpha\beta})\cap \pi_2^{-1}(X_{\alpha\beta})=
X_{\alpha\beta}\times_{S_\alpha}X_{\alpha\beta}.
\]
所以,形如$X_{\alpha\beta}\times_{S_\alpha}X_{\alpha\beta}$
的仿射开集覆盖了$\im \Delta$. 

最后,考虑$\Delta$在$\Delta^{-1}(X_{\alpha\beta}\times_{S_\alpha}
X_{\alpha\beta})=X_{\alpha\beta}$上的限制
\[
	\Delta_{\alpha\beta}:X_{\alpha\beta}\to X_{\alpha\beta}
	\times_{S_\alpha}X_{\alpha\beta},
\]
从仿射的情况,即Lemma \ref{lem:4.1.2},这是一个闭浸入。
从而,Corollary \ref{coro:3.4.12}告诉我们$\Delta$是$X$到
\[
\bigcup_{\alpha,\beta}X_{\alpha\beta}\times_{S_\alpha}X_{\alpha\beta}
=\bigcup_{\alpha,\beta}\left(\pi_1^{-1}(X_{\alpha\beta})\cap 
\pi_2^{-1}(X_{\alpha\beta})\right)\subset X\times_S X
\]
中的闭浸入,进而是到$X\times_S X$中的浸入。
\end{proof}

作为推论,回顾Proposition \ref{pro:4.1.3}的第四点,如果$S$是一个$T$-概形,
则典范态射$X\times_S Y\to X\times_T Y$是两个浸入的复合,
所以也是一个浸入。特别地,如果$\Delta_{S|T}$是闭浸入,则它也是闭浸入。

\begin{para}[分离性]
设$f:X\to Y$是一个概形间态射,如果对角线$\Delta:X\to X\times_Y X$
是一个闭浸入,则称$f$是一个分离态射,也称$X$是一个分离的$Y$-概形。
如果$X$是分离的$\spec \zz$-概形,则简称$X$是分离的。

利用闭浸入在基变换下不变,还可以将分离性定义为,任取两个$f,g:Z\to X$,
等值子$\operatorname{Eq}(f,g)_Y\to Z$是一个闭浸入;或者,任取态射
$f:Z\to X$,图像$\Gamma_f$是一个闭浸入。
\end{para}

\begin{pro}
	设$X$为$S$-概形而$Y$-是分离$S$-概形,令$U\subset X$为一个稠密开子概形,
	若$f,g:X\to Y$为两个态射,如果$f|_U=g|_U$,则
	$f|_{X_{\operatorname{red}}}=g|_{X_{\operatorname{red}}}$.
\end{pro}

\begin{proof}
	由于$Y$是分离的,所以$\operatorname{Eq}(f,g)$是$X$的闭子概形,而$U$
	包含于$\operatorname{Eq}(f,g)$中,$U$又是稠密的,所以从拓扑空间上,
	$\operatorname{Eq}(f,g)$只能是整个$X$,进而
	$X_{\operatorname{red}}\subset\operatorname{Eq}(f,g)$.
\end{proof}

但是,这不能说明$f=g$,这是因为包含$U$的闭子概形可能不止一个。
比如,考虑域$k$和两个$k$-代数
$k[t]$和$k[x,y]/(xy,y^2)$,由于仿射概形间态射都是分离的,所以他们都是
分离$k$-概形。现在我们考虑$t\mapsto 0$和$t\mapsto y$给出的两个
$k$-代数同态$k[t]\to k[x,y]/(xy,y^2)$,以及他们对应的
\[
	f,g:\spec k[x,y]/(xy,y^2)\to \spec k[t].
\]
如果我们将$\spec k[x,y]/(xy,y^2)$看出$\spec k[x,y]$中的闭子集$V(xy,y^2)$. 
注意到准素分解$(y^2,xy)=(y)\cap (x,y)^2$,所以我们大致可以说这是直线$y=0$
(或者说$x$-轴)与原点处的二重点的并。

进一步考虑$\spec k[x,y]/(xy,y^2)$的基本开集$D(x)$,在这上面$x$是可逆的,
此时$xy=0$给出$y=0$蕴含了$y^2=0$,故$D(x)$就是仿射概形
$\spec k[x]_x=\spec k[x,x^{-1}]$. 几何上,这就是仿射平面扣掉原点的$x$-轴,
这当然是$V(xy,y^2)$的稠密开集。在这个稠密开集上,我们考虑上面给出的两个
$k$-代数同态,换言之,我们复合上局部化同态
\[
	k[t]\to k[x,y]/(xy,y^2)\to k[x,x^{-1}],
\]
最后的局部化同态将$y$映射到$0$,所以两个$k$-代数同态复合上局部化同态后相等。
所以我们有在稠密开集$D(x)$上$f|_{D(x)}=g|_{D(x)}$,但是$f\neq g$.

最后,考虑$(\spec k[x,y]/(xy,y^2))_{\text{red}}$,由于此时$y$是
幂零元,所以我们需要模掉$y$,这也使得两个态射$f$, $g$在
$(\spec k[x,y]/(xy,y^2))_{\text{red}}$是相同的。

\begin{pro}
	设概形态射的性质$P$在复合和基变换下稳定,且任何闭浸入都满足$P$. 此时,若
	$f:X\to Y$是态射,$g:Y\to Z$是分离态射,如果$gf:X\to Z$是具有性质$P$,
	则$f$也具有性质$P$.
\end{pro}

\begin{proof}
	考虑$f$的分解$f=(gf)_{(Y)}\Gamma_f$,其中$\Gamma_f:X\to X\times_Z Y$为$f$
	的图像,由于$g$是分离的,所以$\Gamma_f$是闭浸入进而满足$P$,而
	$(gf)_{(Y)}:X\times_Z Y\to Y$也满足$P$,所以复合$f$也满足$P$.
\end{proof}

从上一个引理,我们只需$\im \Delta$是$X\times_Y X$中的闭集,
就可以知道$f$是分离的。
从Lemma \ref{lem:4.1.2},所有仿射概形之间的态射都是分离的。
所以仿射概形是分离概形。
此外,由于同构是闭浸入,所以单态射是分离的。

\begin{pro}
两个分离态射的复合还是分离的。
\end{pro}

\begin{proof}
设$f:X\to Y$, $g:Y\to Z$是分离态射,此时不难看到交换图成立
\[
\begin{xy}
\xymatrix{
	X \ar[rr]^{\Delta_{X|Z}}\ar[dr]_{\Delta_{X|Y}}&&X\times_Z X\\
	&X\times_Y X \ar[ur]_j&
	}
\end{xy}
\]
其中$j$来自于$X\times_Z X$的泛性质,使得$\pi_i=p_ij$,
这里$\pi_i$, $p_i$分别是$X\times_Y X$和$X\times_Z X$的投影,
$i$可以取$1$和$2$. 由于$\Delta_{X|Y}$是闭浸入,我们只需说明$j$是闭浸入,
见Lemma \ref{lem:4.1.4}~下面的推论。
\end{proof}

所以比如分离概形的子概形是分离的。

\begin{pro}
	设$\{V_\alpha\}$是$Y$的一个开覆盖,则态射$f:X\to Y$分离当且仅当
	每个限制态射$f_\alpha:f^{-1}(V_\alpha)\to V_\alpha$都分离。
\end{pro}

\begin{proof}
	$f$分离当且仅当$\im \Delta$是闭的。现在,令
	$U_\alpha=f^{-1}(V_\alpha)$,以及对应的$\Delta$的限制
	$\Delta_\alpha:U_\alpha\to U_\alpha\times_Y U_\alpha$的像
	$\im \Delta_\alpha$是闭的。

	如果$f$是分离的,则$\im \Delta$是闭的,而
	$\im \Delta_{\alpha}=\im \Delta \cap (U_\alpha\times_Y U_\alpha)$
	在$(U_\alpha\times_Y U_\alpha)$是闭的,此时$f_\alpha$分离。反过来,
	如果所有$f_\alpha$分离,则
	$
	\im \Delta=\bigcup_\alpha \im \Delta_{\alpha}
	$
	在$\bigcup_\alpha (U_\alpha\times_Y U_\alpha)$中是闭的,
	我们只需说明$\bigcup_\alpha (U_\alpha\times_Y U_\alpha)=X\times_Y X$,
	就得到了$f$的分离性。

	令$V_{\alpha\beta}=V_\alpha\cap V_\beta$以及
	$U_{\alpha\beta}=U_\alpha\cap U_\beta$,于是
	$U_{\alpha\beta}=f^{-1}(V_{\alpha\beta})$. 我们考虑纤维积
	$U_{\alpha\beta}\times_{V_{\alpha\beta}}U_{\alpha\beta}$,
	在$V_\beta$-概形范畴应用Lemma \ref{lem:3.2.11},
	\[
	U_{\alpha\beta}\times_{V_{\alpha\beta}}U_{\alpha\beta}=
	U_{\alpha\beta}\times_{V_\beta} U_\beta=U_{\alpha}\times_{Y} U_\beta,
	\]
	其中最后一个等式我们在$Y$-概形范畴再一次应用了
	Lemma \ref{lem:3.2.11}. 所以
	\[
	U_\alpha\times_Y U_\beta=U_{\alpha\beta}\times_{V_{\alpha\beta}}
	U_{\alpha\beta}=U_{\alpha\beta}\times_{Y}U_{\alpha\beta}
	\]
	是一个$U_\alpha\times_Y U_\alpha$的开子集,此即所证。
\end{proof}

所以,分离性对$Y$是一个局部条件,特别地,我们可以将问题归结到
$Y$为仿射概形的情况。为描述这点,我们首先需要一个引理。

\begin{lem}
概形态射$f:X\to \spec R$是一个闭浸入,当且仅当$X$是仿射概形且
$f$对应的坐标环同态是满的。
\end{lem}

\begin{proof}
如果$f$是闭浸入,则$X$同构于某个闭子概形
$V(\mathfrak{a})\cong \spec (R/\mathfrak a)$,
当然是仿射概形,且坐标环同态满。
反过来,设$X=\spec S$,而$f$对应的同态为$\varphi:R\to S$,
利用同构基本定理,我们可以将其分解为$\varphi=g\psi$,
其中$g:R/\ker \varphi\to S$是同构,而$\psi:R\to R/\ker\varphi$
是商同态,将其换到概形范畴,这就给出了闭浸入需要的分解。
\end{proof}

\begin{pro}
设$Y=\spec R$是一个仿射概形,而$X$是一个概形,
$\{U_\alpha\}$是$X$的一个仿射开覆盖,则态射$f:X\to Y$分离当且仅当,
对任意一组指标$(\alpha,\beta)$,$U_{\alpha\beta}:=U_\alpha\cap U_\beta$
是仿射开集且$\oo_X(U_{\alpha})|_{U_{\alpha\beta}}$和
$\oo_X(U_{\beta})|_{U_{\alpha\beta}}$可以生成$R$-代数
$\oo_X(U_{\alpha\beta})$.
\end{pro}

\begin{proof}
记$\pi_1$和$\pi_2$是$X\times_Y X$到$X$的两个典范投影,
则$U_\alpha\times_Y U_\beta=\pi_1^{-1}(U_\alpha)\cap\pi_2^{-1}(U_\beta)$
构成了$X\times_Y X$的一个开覆盖,且
\[
	\Delta^{-1}\left(\pi_1^{-1}(U_\alpha)\cap\pi_2^{-1}(U_\alpha)\right)=
	\Delta^{-1}\left(\pi_1^{-1}(U_\alpha)\right)\cap\Delta^{-1}
	\left(\pi_2^{-1}(U_\alpha)\right)=U_\alpha\cap U_\beta=U_{\alpha\beta},
\]
这里我们应用了$\pi_1\Delta=\pi_2\Delta=\id_X$. 
所以$f$分离就等价于$\Delta$在每个$U_{\alpha\beta}$上的限制
$\Delta_{\alpha\beta}$是到$U_\alpha\times_Y U_\beta$上的一个闭浸入。

记$j:U_\alpha\times_Y U_\beta\hookrightarrow X \times_Y X$, 
$k:U_{\alpha\beta}\hookrightarrow X$和$i_\alpha:U_\alpha\hookrightarrow X$
是典范含入,$\pi_\alpha:U_\alpha\times_Y U_\beta\to U_\alpha$是典范投影,
则$\Delta k=j\Delta_{\alpha\beta}$且$i_\alpha\pi_\alpha=\pi_1j$,所以
\[
	k=\pi_1\Delta k=\pi_1\Delta k=\pi_1j\Delta_{\alpha\beta}=
	i_\alpha\pi_\alpha\Delta_{\alpha\beta},
\]
所以$\pi_\alpha\Delta_{\alpha\beta}$就是典范含入
$j_\alpha:U_{\alpha\beta}\hookrightarrow U_\alpha$. 实际上,
$k=i_\alpha j_\alpha$,而$i_\alpha$是单态射。同理,
$\pi_\alpha\Delta_{\alpha\beta}$就是典范含入
$j_\beta:U_{\alpha\beta}\hookrightarrow U_\beta$,因此
\[
	\Delta_{\alpha\beta}=j_\alpha\times_Y j_\beta:
	U_{\alpha\beta}\to U_\alpha\times_Y U_\beta.
\]

由于$U_\alpha\times_Y U_\beta$是仿射概形,其坐标环为
$A(U_\alpha\times_Y U_\beta)=\oo_X(U_\alpha)\otimes_{R}\oo_X(U_\beta)$.
而$\Delta_{\alpha\beta}$对应的环同态即从$A(U_\alpha\times_Y U_\beta)$
到$\oo_X(U_{\alpha\beta})$的映射$\varphi:a\otimes b\mapsto ab$. 
所以,上个引理告诉我们$\Delta_{\alpha\beta}$是闭浸入等价于
$U_{\alpha\beta}$是仿射的且$\varphi$是满的。

最后,$\varphi$是满的,当且仅当,任取$c\in \oo_X(U_{\alpha\beta})$
都可以写成一个有限和$c=\sum_{i}r_ia_ib_i$,
其中$a_i\in \oo_X(U_\alpha)|_{U_{\alpha\beta}}$, 
$b_i\in \oo_X(U_\beta)|_{U_{\alpha\beta}}$而$r_i\in R$,
也当且仅当命题中的条件成立。
\end{proof}

\begin{coro}
	如果概形$X$是分离的,则任意$X$的两个仿射开集的交是仿射的。
\end{coro}

\begin{proof}
	在上个命题中,取$Y=\spec \zz$,并将$X$取做这两个仿射开集的并。
\end{proof}

\begin{coro}
如果$Y$是仿射概形,则$f:X\to Y$分离当且仅当$X$是一个分离概形。
\end{coro}

\begin{proof}
首先假设$f$是分离的,由于$Y$是仿射概形,
则典范态射$Y\to \spec \zz$是分离的,
进而$X\to\spec \zz$作为两个分离态射的复合,依然是分离的。反过来,回忆,
如果$A$, $B$可以生成$\zz$-代数$C$,则也可以生成作为$R$-代数的$C$,
所以上一个命题告诉我们,如果$X$是一个分离概形,则$f$是分离的。
\end{proof}


\section{有限性条件}

\begin{para}[拟紧态射]
设$f:X\to Y$是一个概形间态射,如果存在一个$Y$的仿射开覆盖$\{V_\alpha\}$使得
$f^{-1}(V_\alpha)$都是拟紧的,则称$f$是拟紧态射。
\end{para}

\begin{pro}
	设$f:X\to Y$是一个概形间态射,则$f$是拟紧的当且仅当,
	对任意的拟紧开集$V\subset Y$,$f^{-1}(V)$都是拟紧的。
\end{pro}

\begin{proof}
	一个方向是显然的。显然设$f$是拟紧的,所以存在一个$Y$的仿射开覆盖
	$\{V_\alpha\}$使得$f^{-1}(V_\alpha)$都是拟紧的。现在,对每个
	$f^{-1}(V_\alpha)$找仿射开覆盖,从拟紧性,可以找到有限覆盖。于是,
	$f^{-1}(V_i)=\sum_{j=1}^{n_i}U_{ij}$. 现在,取$V_i$截面中的元素$s$,
	他在$U_{ij}$中的像为$t_j$. 于是,
	$f^{-1}(D_{V_i}(s))=\bigcup_{j=1}^{n_i}D_{U_{ij}}(t_j)$是拟紧开集的有限并,
	故而是拟紧的。而任意$Y$的拟紧开子集$V$都可以由这样的$D_{V_i}(s)$所有限覆盖,
	故而$f^{-1}(V)$是拟紧的。
\end{proof}

容易看到,若$Y$是仿射概形,则$f:X\to Y$是拟紧当且仅当$X$拟紧,特别是$Y=\spec Z$
的情况。此外,拟紧态射的复合是拟紧的,且闭浸入也是拟紧的,因为拟紧空间的闭子集
是拟紧的。

\begin{pro}
态射的拟紧性在基变换下稳定,即对拟紧态射$f:X\to S$,任何的基变换$S'\to S$给出的
态射$f_{(S')}:X_{S'}\to S'$也是拟紧的。
\end{pro}

\begin{proof}
由于拟紧性在目标概形上是局部的,我们可以取$S$为一个仿射概形,此时$X$是拟紧的。
现在任取$S'$的仿射开集$V'$,他在$X\times_S S'$中的原像为$X\times_S U'$,我们用
有限的仿射开集$\{U_i\}$覆盖$X$,则$\{U_i\times_S U'\}$是一个有限开覆盖,故而
$X\times_S U'$是拟紧的。这样,我们就验证了$f_{(S')}$是拟紧的。
\end{proof}

\begin{para}[拟可分态射]
	设$f:X\to Y$是一个概形间态射,如果对角线态射$X\to X\times_YX$是拟紧的,
	则称$f$是拟可分态射。等价地,$f$是拟可分的,当且仅当任取$Y$的仿射开集$V$,
	则以及仿射开集$U_1,U_2\in f^{-1}(V)$,$U_1\cap U_2$是拟紧的。
\end{para}

因为闭浸入是拟紧的,所以可分态射当然是拟可分的。等价性来自于,
$U_1\cap U_2=U_1\times_V U_2$可以覆盖$X\times_Y X$,所以只需简单翻译对角线态射
的拟紧性即可。

拟可分性在复合和基变换下稳定,这是显然的。

\begin{pro}
若$f:X\to Y$是拟紧且分离的,且$\mathcal F$是一个拟凝聚$\oo_X$-模层,则
$f_*\mathcal F$是一个拟凝聚$\oo_Y$-模层。
\end{pro}

\begin{para}[局部Noether概形]
	如果概形$X$可以写成仿射开集的并,且每个仿射开集的坐标环都是Noether环,
	则称$X$是一个局部Noether概形。特别地,如果并还是有限的,
	则称$X$是一个Noether概形。
\end{para}

一个局部Noether概形是Noether的,当且仅当底空间拟紧。于是对仿射概形而言,
局部Noether与Noether等价。

局部Noether概形的开子概形也显然是局部Noether概形。由于Noether概形的开子概形
还必须是拟紧的,加上其是局部Noether概形所以也是Noether概形。

\begin{pro}
	$X$是局部Noether概形当且仅当其任何仿射开子集都是局部Noether概形。
\end{pro}

\begin{proof}
一个方向是显然的。现在任取$X$的仿射开子集$U$,由于仿射概形是拟紧的,
所以从$X$的仿射Noether开覆盖上可以找到覆盖$U$的有限子覆盖,
而有限个Noether环的谱的并还是Noether的。
\end{proof}

\begin{para}[Noether拓扑空间]
一个拓扑空间被称为是Noether的,如果其闭子集的严格降链是稳定的。
这又当且仅当其任何开子集(实际上是任何子空间)都拟紧。并且可以看到Noether
空间具有的不可约分支的个数有限。

任何Noether仿射概形的底空间都是Noether的,因为其任何的闭集都可以翻译到理想,
故而闭子集的严格降链条件就变成了理想的升链条件。如果一个空间
可以写成Noether开子集的有限并,其显然也必须是Noether的。
所以Noether概形显然是一个Noether拓扑空间。
\end{para}

\begin{pro}
	若$X$是局部Noether概形,则$f:X\to Y$是拟可分的。若$X$是Noether概形,
	则$f:X\to Y$是拟紧的。
\end{pro}

\begin{proof}
	第一点是因为局部Noether概形的仿射开集都是拟紧的,第二点是因为
	Noether空间的任何开子集都拟紧。
\end{proof}

\begin{pro}
	仿射概形$\spec R$是Noether的当且仅当$R$是Noether的。
\end{pro}

\begin{proof}
一个方向是显然的。我们假设$\spec R$是Noether的,令$I$是$R$的一个理想,
我们证明其是有限生成的。考虑$\spec R$的一个开覆盖$\{D(f_i)\}_i$,由于
$\spec R$是Noether的,我们还可以将其取成有限的。进而,$I_{f_i}$
是有限生成$R_{f_i}$-模。现在,$I_{f_i}$可以由$\{m_{ij}/f_i^{n_{ij}}\}_j$
生成,我们将这些有限的$\{m_{ij}\}_{i,j}$生成的$R$-模记作$N$. 于是我们有
$N_{f_i}=I_{f_i}$对任意的$i$都成立。再取局部化到每个素理想,我们有
$N_{x}=I_x$,于是$I=N$是有限生成的。
\end{proof}



\begin{para}[有限型态射]
设$f:X\to Y$是一个概形间态射,如果存在$Y$的一个仿射开覆盖
$\{Y_\alpha\}_\alpha$使得每一个$f^{-1}(Y_\alpha)$可以写成有限个仿射开集
$\{U_{\alpha\beta}\}_\beta$的并,且环$A(U_{\alpha\beta})$是有限生成
$A(Y_\alpha)$-代数,其中$A(U_{\alpha\beta})$和$A(Y_\alpha)$分别是仿射开集
$U_{\alpha\beta}$和$Y_\alpha$对应的坐标环。有时候,如果略去态射的记号,
会称$X$是一个有限型$Y$-概形。
\end{para}

有限型态射是拟紧的,因为仿射开集都是拟紧的,进而有限个仿射开集的并是拟紧。

\begin{pro}
	如果$f:X\to Y$是一个有限型态射,则任取$Y$的仿射开集$V$,
	$f^{-1}(V)$可以写成有限个仿射开集$\{U_{\alpha}\}$的并,
	且环$A(U_{\alpha})$是有限生成$A(V)$-代数。
\end{pro}

换而言之,如果$f$是有限型的,则定义中的“存在$Y$的一个仿射开覆盖”
可以改为“任取$Y$的一个仿射开覆盖”。

\begin{proof}
	方便起见,如果$Y$的仿射开集$V$成立“$f^{-1}(V)$可以写成有限个仿射开集
	$\{U_{\alpha}\}$的并,且环$A(U_{\alpha})$是有限生成$A(V)$-代数”,
	则我们称作$V$具有性质$P$. 此外,回忆一下,如果$f\in R$,
	则$R_{f}$是一个有限生成$R$-代数,因为它可以看成
	$R[x]/\langle fx-1\rangle$.

	首先设$U\subset Y$是一个仿射开集,具有性质$P$,则任取开集
	$D(g)\subset U$都具有性质$P$. 实际上,如果$f^{-1}(U)$可以写作$X$
	的有限仿射开集族$\{Z_i\}$的并,其中每个$A(Z_i)$就是有限生成
	$A(U)$-代数。考虑$f$在$Z_i$上的限制$f_i$,其对应了环同态
	$\varphi_i:A(U)\to A(Z_i)$. 
	所以,$f^{-1}(D(g))\cap Z_i=D(g_i)$,其坐标环$A(D(g_i))=A(Z_i)_{g_i}$
	是一个有限生成$A(Z_i)$-代数,继而是一个有限生成$A(U)$-代数,当然也是有
	限生成$A(D(g))=A(U)_g$-代数。
	综上,$f^{-1}(D(g))$可以写作$D(g_i)$的并,
	其中$A(D(g_i))$是有限生成$A(D(g))$-代数,因此$D(g)$满足性质$P$.

	现在,回到一般情况。假设$f:X\to Y$是一个有限型态射,则存在$Y$
	的一个仿射开覆盖$\{Y_\alpha\}_\alpha$使得每一个$Y_\alpha$都具有性质$P$. 
	任取$Y$的仿射开集$V$,由于他是拟紧的,所以可以找其有限仿射开覆盖
	$\{D(g_i)\}$,其中每个开集$D(g_i)$包含于某个$Y_{\alpha(i)}$. 所以,
	每个$D(g_i)$具有性质$P$. 下面我们说明$V$具有性质$P$.
	
	由于$D(g_i)\subset V$是拟紧的,所以可以写成有限个$D(h_{ij})$的并,
	其中$h_{ij}\in A(V)$. 取$\varphi$为$D(g_i)\hookrightarrow V$对应的环同态,
	则$D(h_{ij})=D(\varphi(h_{ij}))$,其中$\varphi(h_{ij})\in A(D(g_i))$.
	由$D(g_i)$的性质$P$,$D(h_{ij})$也具有性质$P$,即每个$f^{-1}(D(h_{ij}))$
	都具有有限仿射开覆盖$\{U_{ijk}\}_k$使得每一个$A(U_{ijk})$都是有限生成
	$A(D(h_{ij}))=A(V)_{h_{ij}}$-代数,自然也是有限生成$A(V)$-代数。
	此时,所有仿射开集$\{U_{ijk}\}_{i,j,k}$就使得$V$满足性质$P$.
\end{proof}

\begin{lem}
	$\spec R$是一个有限型$\spec S$-概形,当且仅当$S$是一个有限生成
	$R$-代数。
\end{lem}

\begin{pro}
设$f:X\to Y$是有限型的,如果$Y$是(局部)Noether的,
则$X$也是(局部)Noether的。
\end{pro}

\begin{proof}
由于有限型对$Y$是局部条件,所以我们只需检验$Y$是Noether的情况即可。现在,
由于$Y$是Noether的,所以$Y$具有有限仿射开覆盖$\{V_i\}$,且其中每个仿射开集
$V_i$的坐标环$A(V_i)$都是Noether环。从$f$是有限型的,则$f^{-1}(V_i)$
被有限个仿射开集$\{W_{ij}\}$所覆盖,且坐标环$A(W_{ij})$是有限生成
$A(V_i)$-代数,所以也是Noether环。
\end{proof}

\begin{para}[有限(整)态射]
	我们称一个$R$-代数是有限的,是指这个$R$-代数作为$R$-模也是有限生成的。
	一个$R$-代数$S$在$R$上整,是指$S$中的元素都是以$R[x]$中的某个首一
	多项式的根。注意到,一个$S$是有限-$R$代数当且仅当$S$在$R$上整且$S$是有限生成
	$R$-代数。

	一个概形间态射$f:X\to Y$是有限/整的,是指存在$Y$的一个仿射开覆盖$\{V_i\}$,
	使得对每个$i$,$f^{-1}(V_i)$是仿射开集,且环同态
	$\Gamma(V_i,\mathcal O_Y)\to\Gamma(f^{-1}(V_i),\mathcal O_X)$
	是有限/整的。类似地,一个概形间态射是有限的等价于说他是有限型且整的。
\end{para}

类似于有限型,我们可以将存在一个开覆盖,直接改成任意的开覆盖,即仿射开集的
原像也是仿射的,且相应的环同态是有限/整的。为此,只需注意到,若
$\spec R=\bigcup_i D(f_i)$,则$M$是有限生成$R$-模当且仅当$M_{f_i}$是
有限生成$R_{f_i}$-模。

容易检查这些有限性条件都在复合和基变换下稳定。

\begin{pro}
	有限型/整/有限性在复合和基变换下稳定。
\end{pro}

\begin{pro}
	整态射是闭态射,其将闭集映射为闭集,且保持维数。即对整态射$f:X\to Y$,
	若$Z$是$X$的闭子集,则$f(Z)$是闭的,且$\dim Z=\dim f(Z)$.
\end{pro}

这个命题是局部的,所以只要回到仿射的情况。对于仿射的情况,我们知道若$R\subset S$
是一个整扩张,则任取$R$的素理想$\mathfrak p$,都存在$S$的素理想$\mathfrak q$
使得$\mathfrak p=\mathfrak q\cap R$. 这就保证了整态射保持了维数。

\begin{para}[全景闭性]
	一个概形间态射$f:X\to Y$被称为是全景闭的,即对任何态射
	$g:Z\to Y$,典范投影$X\times_Y Z\to Z$作为拓扑空间间
	的连续映射都是闭的。类似其他性质,如果$X\to \spec \zz$是全景闭的,
	则称$X$是全景闭的。
\end{para}

整态射是可分且全景闭的,这是因为整态射是闭态射,且在基变换下稳定。

全景闭性在拓扑空间范畴是紧性的体现,具体来说,一个拓扑空间$X$是紧的,
当且仅当对任何的拓扑空间$Y$,投影$X\times Y\to Y$都是闭映射。
回忆,一个连续映射$f:X\to Y$是逆紧的,如果任何紧集的原像都是紧的。如果
$X$是Hausdorff的,且$Y$是局部紧Hausdorff,逆紧等价于说$f$是全景闭的。
如果如果适当加上一定的可分条件,在概形范畴即可分性,则我们可以定义出逆紧
态射的类似物。

\begin{para}[紧合性]
	一个概形间态射$f:X\to Y$被称为是紧合的,如果他是有限型、可分和全景闭的。
\end{para}

我们不再称逆紧,是因为这里不再可以解释为“紧集的原像是紧的”。
容易看到,有限态射是紧合的,而紧合的另一大类例子是闭浸入。

\begin{thm}[赋值判别法]
	设$f:X\to Y$是一个概形间态射,则$f$是可分(全景闭、紧合)的当且仅当,
	$f$是拟可分(拟紧、拟可分且有限型)的,且对任何的赋值环$R$,其分式域记作$K$,
	任意一对与$f$相容的$K$-值点$u:\spec K\to X$和$R$-值点$v:\spec R\to Y$,
	即满足交换图
	\[
		\xymatrix{
			\spec K\ar[r]^-u\ar[d] & X\ar[d]^f\\
			\spec R\ar[r]^-{v} & Y
		}
	\]
	存在至多一个(至少一个、唯一)的$w:\spec R\to X$使得下图交换
	\[
		\xymatrix{
			\spec K\ar[r]^-u\ar[d] & X\ar[d]^f\\
			\spec R\ar[ru]^w\ar[r]^-{v} & Y
		}
	\]
\end{thm}

注意,至多一个也可以不存在。若我们进一步将$Y$取成$\spec \mathbb Z$,
且假设$X$是拟可分(拟紧、拟可分且有限型)的,则赋值判别法是在说,
任取$X$的任意$K$-值点,$X$是是可分(全景闭、紧合)的,当且仅当,
对$K$的任意赋值环$R$,都存在至多一个(至少一个、唯一)分解
$\spec K\to \spec R\to X$. 注意到,我们之前证明过,如果$K$-值点位于$x$,
则$\overline{\{x\}}$中的每一个点都对应存在$K$的一个赋值环$R$使得分解
$\spec K\to \spec R\to X$成立,但是这样的分解的存在性和唯一性不是对$K$的
每个赋值环都有的,赋值判别法告诉我们,这需要一定的条件。

如果$X$是局部Noether的,则$f$自动是拟可分的,而Noether赋值环都是离散赋值环,
同时为了避免出现非有限扩张,所以还需要加入有限型条件,这样我们有以下较弱的命题:

\begin{thm}[赋值判别法]
	设$X$是局部Noether概形,$f:X\to Y$是一个概形间的有限型态射,
	则$f$是可分(紧合)的当且仅当,对任何的离散赋值环$R$,其分式域记作$K$,
	任意一对与$f$相容的$K$-值点$u:\spec K\to X$和$R$-值点$v:\spec R\to Y$,
	即满足交换图
	\[
		\xymatrix{
			\spec K\ar[r]^-u\ar[d] & X\ar[d]^f\\
			\spec R\ar[r]^-{v} & Y
		}
	\]
	存在至多一个(唯一)的$w:\spec R\to X$使得下图交换
	\[
		\xymatrix{
			\spec K\ar[r]^-u\ar[d] & X\ar[d]^f\\
			\spec R\ar[ru]^w\ar[r]^-{v} & Y
		}
	\]
\end{thm}

% \section{仿射态射}

% 仿射态射简而言之就是仿射概形的整体推广,即,态射$X\to \spec \zz$是仿射的,当且仅当$X$是仿射概形。

% \begin{para}[仿射态射]
% 设$f:X\to S$是一个概形态射,如果存在一个$S$的仿射开覆盖$\{S_\alpha\}$
% 使得每个$f^{-1}(S_\alpha)$都是仿射开集,则称$f$是一个仿射态射。此时,
% 也称$S$-概形$X$是$S$-仿射的。
% \end{para}

% 当然,$X$是$S$-仿射的不一定能推出$X$是仿射概形。比方说,
% 取$S$是一个非仿射概形,再取$X=S$.

% \begin{lem}
% 如果$f:X\to S$是一个仿射态射,则$f$是一个分离态射。
% \end{lem}

% \begin{proof}
% 这直接来自于仿射概形间的态射是分离的,以及分离性的局部判别法。
% \end{proof}

% 正如我们希望从一个环构造一个仿射概形一样,下面我们希望从一个$\oo_S$-模层,
% 这里$S$是一个概形,构造一个$S$-仿射的概形。不难看到,
% 每一个概形$X$伴随着一个典范的$\oo_\zz:=\oo_{\spec \zz}$-代数层,
% 因为我们总有唯一的态射$(\varphi,\theta):X\to \spec \zz$,其中层态射写作
% \[
% 	\theta:\oo_\zz \to \varphi_*X,
% \]
% 此时$\varphi_* X$就是一个$\oo_\zz$-代数层。
% 下面我们将这个概念写得更细一些。

% \begin{para}[$S$-概形伴随的$\oo_S$-代数层]
% 	设$S$是一个概形,而$X$是一个$S$-概形,结构态射为$f:X\to S$. 
% 	由于结构态射诱导了$S$上的层之间的态射$\oo_S\to f_*\oo_X$,
% 	此时$f_*\oo_X$就是一个$\oo_S$-代数层,我们将其记作
% 	$\mathcal A(X)$. 于是,$\Gamma(U,\mathcal A(X))=\Gamma(
% 	f^{-1}(U),X)$. 类似地,考虑一个$\oo_X$-模层$\mathcal F$,
% 	我们用$\mathcal A(\mathcal F)$来记作为$\mathcal A(X)$-模层
% 	的$f_*\mathcal F$,进而还是一个$\oo_S$-模层。

% 	现在,考虑$S$-概形态射$h:X\to Y$,其中结构态射为$f:X\to S$, 
% 	$g:Y\to S$,满足$f=gh$. 概形态射$h$给出了层映射$\oo_Y\to
% 	h_*\oo_X$,然后取$g$的顺像后,给出了层映射
% 	\[
% 		g_*\oo_Y\to g_*h_*\oo_X=f_*\oo_X,
% 	\]
% 	这是一个$\oo_S$-代数层同态,即$\mathcal A(Y)\to \mathcal A(X)$,
% 	这个映射我们记作$\mathcal A(h)$. 不难看到,$\mathcal A$此时是
% 	一个反变函子。

% 	类似地,如果$\mathcal F$是一个$\oo_X$-模层,而$\mathcal G$是一个
% 	$\oo_Y$-模层,$u:\mathcal G\to h_*\mathcal F$是一个$\oo_Y$-模层
% 	同态,那么
% 	\[
% 		g_*u:g_*\mathcal G\to g_*h_*\mathcal F=f_*\mathcal F
% 	\]
% 	是一个$\oo_S$-模层同态$\mathcal A(\mathcal G)\to \mathcal A
% 	(\mathcal F)$,我们记作$\mathcal A(u)$. 
% \end{para}

% \begin{pro}
% 设$X$, $Y$都是$S$-概形,则$h\mapsto \mathcal A(h)$是一个从$\Hom_S(Y,X)$
% 到$\Hom_{\oo_S}(\mathcal A(X),\mathcal A(Y))$的一个典范双射。
% \end{pro}

% 这个命题大概是双射$\Hom(\spec R,\spec S)\cong \Hom(S,R)$的整体对应。

% \begin{proof}
% 设$f:X\to S$和$g:Y\to S$是结构态射。\notprove
% \end{proof}

% \begin{thm}\label{thm:4.3.5}
% 设$S$是一个概形,而$\mathcal B$是一个拟凝聚$\oo_S$-代数层,则存在
% 一个$S$-仿射概形$X$使得$\mathcal A(X)=\mathcal B$,而满足这个性质的
% 概形$X$至多只差一个$S$-概形同构。
% \end{thm}

% \begin{proof}
% 唯一性用上面的命题即可,只需说明存在性。对任意仿射开集$U\subset S$,
% 设$X_U=\spec(\Gamma(U,\mathcal B))$,将这些$X_U$黏起来,我们就得到了
% 想要的概形。\notprove
% \end{proof}

% \begin{para}[整体谱]
% 设$S$是一个概形,而$\mathcal B$是一个拟凝聚$\oo_S$-代数层,则上面定理
% 确定的概形我们记作$\spec (\mathcal B)$,且对仿射开集$U\subset S$有
% $\Gamma(U,\spec \mathcal B)=\spec (\Gamma(U,\mathcal B))$. 
% \end{para}

% 从Theorem \ref{thm:4.3.5},我们立刻得到了
% \[
% 	\Hom_{\oo_S}(\mathcal B,\mathcal C)\cong 
% 	\Hom_S(\spec\mathcal B,\spec\mathcal C),
% \]
% 这里左边的$\mathcal B$, $\mathcal C$是拟凝聚$\oo_S$-代数层。

% \begin{lem}
% 设$f:X\to S$仿射,$\mathcal F$是一个拟凝聚$\oo_X$-模层,则$f_*\mathcal F$
% 是一个拟凝聚$\oo_S$-模层。
% \end{lem}

% \begin{proof}
% \notprove
% \end{proof}

% \begin{pro}
% 设$f:X\to S$仿射,则任取仿射开集$U$,$f^{-1}(U)$是一个仿射概形。
% 特别地,如果$S$是仿射概形,则$f:X\to S$仿射当且仅当$X$是仿射概形。
% \end{pro}

% \begin{proof}
% 从引理,可以知道此时$\mathcal A(X)$是一个拟凝聚$\oo_S$-模层,
% 而且$X$满足Theorem \ref{thm:4.3.5}中对$\mathcal A(X)$的整体谱的
% 要求,从唯一性以及Theorem \ref{thm:4.3.5}证明中的构造,立即可得结论。
% \end{proof}

% 作为推论,$f:X\to S$仿射当且仅当,任取仿射开集$U\subset S$,$f^{-1}(U)$也是仿射开集。将这个等价定义应用到两个仿射态射的复合上,立刻可得,仿射态射复合依然是仿射的。

% \begin{pro}
% 设$X$是一个$S$-仿射概形,$g:S'\to S$是一个概形态射,则基扩张
% $X'=X_{(S')}$是一个$S'$-仿射概形。
% \end{pro}

% \begin{proof}
% \notprove
% \end{proof}

\section{有理映射和函数域}

\para[概形式稠密]
令$X$是一个概形,开子概形$U$被称为概形式稠密的,即指对任意的开子概形$V$,
则$V$中包含$U\cap V$的最小的闭子概形为$V$. 
\endpara

\begin{pro}
设$X$是一个$S$-概形,$U$是他的一个开集,而$i:U\hookrightarrow X$是典范含入,则
以下命题等价:
\begin{compactenum}[(1)]
	\item $U$是概形式稠密的;
	\item $\mathcal O_X\to i_*\mathcal O_U$是单的;
	\item 设$V$是$X$的一个开子概形,
	对任意的分离$S$-概形$Y$和两个$S$-概形态射$f,g:V\to Y$,
	如果他们的限制$U\cap V\to Y$是相同的,则$f=g$. 
\end{compactenum}
\end{pro}

\begin{proof}
(1)到(2):令$V$是一个开子概形,而$Z$是$V$的包含$U\cap V$的闭子概形,
其由拟凝聚理想层$\mathcal I$所定义。单的$\mathcal O_X\to i_*\mathcal O_U$
限制到$V$上$\mathcal O_V\to i_*\mathcal O_{U\cap V}$也是单的,
因为$Z$包含$U\cap V$,所以可以经由
$\mathcal O_V\to \mathcal O_V/\mathcal I$分解,进而他也是单的,则$\mathcal I=0$. 


(1)到(3):如果$U$是概形式稠密的,则闭子概形$\operatorname{Eq}(f,g)$
包含$U\cap V$,所以只能是$V$. 

(3)到(2):我们可以将截面看成$\mathbb A_S^1$-值函数的集合,即
$\Hom_S(W,\mathbb A_S^1)=\Gamma(W,\mathcal O_X)$对任何$X$的
开子概形$W$都成立。取$Y=\mathbb A_S^1$,则(3)告诉我们此时
$\Hom_S(U,\mathbb A_S^1)=\Hom_S(X,\mathbb A_S^1)$,进而
$\mathcal O_X\to i_*\mathcal O_U$是单的。
\end{proof}

既约概形的稠密开子概形是概形式稠密的,因为此时$X=X_{\text{red}}$,然后利用
分离性的判别法。两个概形式稠密开集的交也依然是概形式稠密的,复合当然也是,
即概形式稠密开子集的概形式稠密的开子集是概形式稠密的。此外,概形式稠密是局部的,
即考虑一个开覆盖,则概形式稠密当且仅当在每个覆盖的元素中是概形式稠密的。

\begin{para}[associate prime]
对概形$X$,我们定义$\operatorname{Ass}(X)$是$x\in X$使得$\mathfrak m_x$
是$\mathcal O_{X,x}$的associate prime的集合,
即他一定是某个元素$m\in \mathcal O_{X,x}$的
$\mathfrak m_x=\operatorname{ann}(m)$. 如果$X=\spec R$是仿射的,则 
$\operatorname{Ass}(X)=\operatorname{Ass}(R)$是我们在交换代数中熟知的。
同时,任取一个开覆盖$\{U_i\}$,容易看到
\[
	\operatorname{Ass}(X)=\bigcup_i \operatorname{Ass}(U_i).
\]
\end{para}

\begin{pro}
	若$X$是一个局部Noether概形,而$U\subset X$是一个开子集,则$U$是概形式稠密的,
	当且仅当$U$包含了$\operatorname{Ass}(X)$. 
\end{pro}

这是完全局部的,所以我们可以只考虑Noether仿射概形的情况。

\begin{lem}
令$X=\spec R$是一个仿射概形,而$\mathfrak a\subset A$是一个理想,
$U=X-V(\mathfrak a)$是一个开集,则以下命题从强到弱排列:
\begin{compactenum}[\quad (1)]
	\item 存在一个非零因子$t$使得$D(t)\subset U$,或者说$\mathfrak a$包含一个
	非零因子;
	\item $U$是概形式稠密的;
	\item $\operatorname{ann}(\mathfrak a)=0$;
	\item $U$包含$\operatorname{Ass}(R)$.
\end{compactenum}
如果$R$进一步是Noether环,则(4)蕴含(1),故以上命题等价。
\end{lem}

\begin{proof}
(1)到(2)只需说明$D(t)$是概形式稠密的。令$V$是$X$的一个开子集,而$s$是上面的一个
截面使得$s|_{V\cap D(t)}=0$,则对$V$的任何仿射开子集$W$都存在一个正整数$n$
使得$t^n s|_W=0$故$s|_W=0$,进而$s=0$.

(2)到(3),令$s=\operatorname{ann}(\mathfrak a)$且$x\in U$. 由于$\mathfrak p_x 
\not \in V(\mathfrak a)$,所以$\mathfrak p_x$不包含$\mathfrak a$,则存在一个
$a\in \mathfrak a$但$a\not\in \mathfrak p_x$. 进而$a$在$\mathcal O_{X,x}$中
可逆,此时$sa=0$给出$s_x=0$. 由于$x$跑遍$U$,故$s|_U=0$,但是$U$是概形式稠密的,
所以$s=0$.

(2)到(4),如果$\operatorname{Ass}(R)\cap V(\mathfrak a)\neq \varnothing$,则
存在素理想$\ann(r)\in \operatorname{Ass}(R)$且包含$\mathfrak a$,此时
$r\in \operatorname{ann}(\mathfrak a)$.

假设$R$是Noether环,我们证明(4)到(1)的逆否:对给定的环$R$,
所有形如$\operatorname{ann}(r)$的理想按照包含构成偏序集,其中的极大元是素理想。
由于$R$是Noether环,所以这样的极大元一定存在,他们都在$\operatorname{Ass}(R)$
中。由于每个零因子一定处于某个极大的$\operatorname{ann}(r)$中,所以他也在某个
$\operatorname{Ass}(R)$的元素中。此外,由于$R$是Noether环,所以
$\operatorname{Ass}(R)$是有限集合。此时,假设(1)不对,$\mathfrak a$的元素全是
零因子,则$\mathfrak a$包含于$\operatorname{Ass}(R)$中所有素理想的有限并中,
根据素理想互相排斥,所以$\mathfrak a$一定包含于某个
$\operatorname{ann}(r)\in \operatorname{Ass}(R)$,故
$V(\mathfrak a)$与$\operatorname{Ass}(R)$交非空,即(4)不对。
\end{proof}

\begin{para}[有理映射]
设$X$, $Y$是$S$-概形,所谓的从$X$到$Y$的有理映射集$R_S(X,Y)$是指一个偶对的集合
$\{(U,f)\}$模掉下面说的等价关系,其中$U$是一个概形式稠密开子集,
而$f:U\to Y$是一个$S$-概形态射。两个偶对$(U,f)$和$(V,g)$是等价的,
是指存在一个概形式稠密子集$W\subset U\cap V$使得$f|_W=g|_W$. 而$R_S(X,Y)$中的
一个等价类叫做从$X$到$Y$的有理映射。如果$S$无需强调,我们则简单将$R_S(X,Y)$
记作$R(X,Y)$.

对一个有理$S$-映射$f\in R_S(X,Y)$,我们令$\operatorname{dom}_S(f)$为所有如下$x$
的集合,我们需要这个$x$包含于某个$f$的代表元$(U,f)$的$U$中,换言之也可以选成
所有代表元的开集部分之并,他被称为$f$的定义域。显然这也是一个概形式稠密的开子集。
\end{para}

\begin{pro}
	设$X$, $Y$是$S$-概形,$Y$进一步是分离的,则对有理映射$f\in R_S(X,Y)$,存在
	$S$-概形态射$f_0:\operatorname{dom}_S(f)\to Y$使得
	$f=(\operatorname{dom}_S(f),f_0)$,且这是最大的表示元,即不能再扩张到更大的
	概形式稠密开子集上了。
\end{pro}

\begin{proof}
	任取代表元$(U,f)$和$(V,g)$,我们知道存在一个概形式稠密开子集
	$W\subset U\cap V$使得$f|_W=g|_W$,由于$Y$是分离的,所以
	$f|_{U\cap V}=g|_{U\cap V}$. 于是我们可以将不同代表元上的态射拼起来得到
	$\operatorname{dom}_S(f)$上的一个态射。
\end{proof}

\begin{para}[有理函数]
	设$X$是$S$-概形,$X$上的有理函数是$R_S(X,\mathbb A_{S}^1)$中的有理映射,
	有理函数的集合记作$R_S(X)$.
\end{para}

注意到,选一个代表元$(U,f)$,由于
$\Hom(U,\mathbb A_{S}^1)=\Gamma(U,\mathcal O_X)$,所以有理函数也可以看到
某个概形式稠密开子集上的一个截面,而 
\[
	R(X)=\varinjlim_{U\subset X} \Gamma(U,\mathcal O_X),
\]
其中$U$跑遍所有$X$的概形式稠密开子集。