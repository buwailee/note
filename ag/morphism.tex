% !TEX root = main.tex

\chapter{一些态射的性质}

\section{分离性}

\begin{para}[对角线]
	设$X$是一个$S$-概形,则态射$\id_X\times_S\id_X:X\to X\times_S X$被称为对角线态射,记作$\Delta_{X|S}$,如果没有误解,则简单记作$\Delta_X$或$\Delta$.
\end{para}

仿射情况是值得标记的:若$X=\spec B$,$S=\spec A$,则$X\times_S X=\spec (B\otimes_A B)$,此时$\Delta$是同态$\varphi:a\otimes b\mapsto ab$对应的态射$\widetilde{\varphi}$.

\begin{lem}\label{lem:4.1.2}
	如果$X$, $S$是仿射概形,则$\Delta_{X|S}$是一个闭浸入。
\end{lem}

\begin{proof}
由于$\varphi:a\otimes b\mapsto ab$是满的,利用Proposition \ref{pro:3.4.11},我们只要说明$\Delta=\widetilde{\varphi}$是一个到$\spec B$到$\spec (B\otimes_A B)$中闭集$\im \widetilde{\varphi}$的同胚即可,同胚直接来自于$\varphi$是满的以及Lemma \ref{lem:3.9}. 闭集部分也不难,同构基本定理告诉我们,$\varphi$可以分解为一个同构$\psi:(B\otimes_A B)/\ker \varphi\to B$和商映射$\pi:B\otimes_A B\to (B\otimes_A B)/\ker \varphi$的复合,所以,$\im \widetilde{\pi}=V(\ker \varphi)$是一个闭集。此外,由于
\[
	\widetilde{\varphi}=\widetilde{\psi\pi}=\widetilde{\pi}\widetilde{\psi}
\]
以及$\widetilde{\psi}$是同胚,所以$\im \widetilde{\varphi}=\im \widetilde{\pi}=V(\ker \varphi)$是一个闭集。
\end{proof}

\begin{pro}\label{pro:4.1.3}
以下命题成立:
\begin{compactenum}
\item 设$X$, $Y$是两个$S$-概形,如果我们典范等同$(X\times Y)\times (X\times Y)$和$(X\times X)\times (Y\times Y)$,则态射$\Delta_{X\times Y}$等同于$\Delta_X\times \Delta_Y$.	
\item 对于任意的基概形扩张$S\to S'$,我们有$\Delta_{X_{(S')}}=(\Delta_{X})_{(S')}$.
\item 态射$f:X\to Y$是单态射,当且仅当$\Delta_{X|Y}:X\to X\times_Y X$是一个同构。
\item 设有$S$-概形$f:X\to S$, $g:Y\to S$,以及态射$\varphi:S\to T$,它使得$X$, $Y$都变成了$T$-概形。记$p$, $q$为$X\times_S Y$的投影,此时,我们有交换图
\[
	\begin{xy}
	\xymatrix{
	X\times_S Y\ar[rr]^{p\times_T q}\ar[d]_{\pi}&&X\times_T Y\ar[d]^{f\times_T g}\\
	S\ar[rr]^{\Delta_{S|T}}&&S\times_TS
	}
	\end{xy}
\]
其中$\pi=fp=gq$. 同时,$X\times_S Y$就是$S\times_T S$-概形$X\times_T Y$和$S$的乘积,而$\pi$和$p\times_T q$是两个典范投影。
\end{compactenum}	
\end{pro}

\begin{proof}
这些都是在任意(纤维积存在的)范畴成立的东西,这里就不那么细致检验了。这里的第二点直接来自于第一点,来自于典范等同$(X\times_S X)_{(S')}=X_{(S')}\times_{S'}X_{(S')}$.

第四点的交换图这里不检验,至于$X\times_S Y$就是$S\times_T S$-概形$X\times_T Y$和$S$的乘积这点,我们可以直接检查泛性质。设$\alpha:A\to X \times_T Y$, $\beta:A\to S$满足交换图,即$\Delta\beta=(f\times_T g)\alpha$. 设$p_T$, $q_T$是对应$X\times_T Y$的投影,则我们有$p_T\alpha:A\to X$, $q_T\alpha:A\to Y$. 在$\Delta\beta=(f\times_T g)\alpha$两边作用$S\times_T S$的投影后立刻得到
\[
	fp_T\alpha=\beta=gq_T\alpha:A\to S.
\]
于是$X\times_S Y$告诉我们存在唯一的$h:A\to X\times_S Y$使得$p_T\alpha=ph$以及$q_T\alpha=qh$成立,此时
\[
	\pi h=fph=fp_T\alpha=\beta, \quad (p\times_T q)h=(p_T\alpha)\times_T (q_T\alpha)=\alpha,
\]
此即乘积所需的泛性质。
\end{proof}

利用Proposition \ref{pro:3.2.2},第四点告诉我们,此时$p\times_T q$可以典范等同于$\id_{X\times_T Y}\times_{P}\Delta_{S|T}$,其中$P=S\times_T S$. 

\begin{lem}\label{lem:4.1.4}
	对角线态射是一个浸入。
\end{lem}

对角线态射对应的子概形也被称$X\times_S X$的对角线。

\begin{proof}
设$f:X\to S$是一个$S$-概形,$\pi_1$, $\pi_2$是$X\times_S X$到$X$的两个典范投影。给一个$S$的仿射开覆盖$\{S_\alpha\}_\alpha$,对每一个开集$f^{-1}(S_\alpha)$,我们都可以找相应的仿射开覆盖$\{X_{\alpha\beta}\}_\beta$. 由于$f(X_{\alpha\beta})\subset S_{\alpha}$,所以将其限制在$X_{\alpha\beta}$,我们就给出了$X_{\alpha\beta}$的$S_\alpha$概形结构,同时Propostition \ref{pro:3.2.2}告诉我们,
\[
	X_{\alpha\beta}\times_{S_\alpha}X_{\alpha\gamma}=X_{\alpha\beta}\times_{S}X_{\alpha\gamma},
\]
且Corollary \ref{coro:3.2.8}告诉我们右侧是$X\times_S X$的一个开集
$\pi_1^{-1}(X_{\alpha\beta})\cap \pi_2^{-1}(X_{\alpha\gamma})$,而且左侧告诉我们这是仿射的。由于任取$p\in \im \Delta$,我们有$\pi_1(p)=\pi_2(p)$,所以只要找一个包含$\pi_1(p)$的开集$X_{\alpha\beta}$,我们有
\[
p\in \pi_1^{-1}(X_{\alpha\beta})\cap \pi_2^{-1}(X_{\alpha\beta})=X_{\alpha\beta}\times_{S_\alpha}X_{\alpha\beta}.
\]
所以,形如$X_{\alpha\beta}\times_{S_\alpha}X_{\alpha\beta}$的仿射开集覆盖了$\im \Delta$. 

最后,考虑$\Delta$在$\Delta^{-1}(X_{\alpha\beta}\times_{S_\alpha}X_{\alpha\beta})=X_{\alpha\beta}$上的限制
\[
	\Delta_{\alpha\beta}:X_{\alpha\beta}\to X_{\alpha\beta}\times_{S_\alpha}X_{\alpha\beta},
\]
从仿射的情况,即Lemma \ref{lem:4.1.2},这是一个闭浸入。从而,Corollary \ref{coro:3.4.12}告诉我们$\Delta$是$X$到
\[
\bigcup_{\alpha,\beta}X_{\alpha\beta}\times_{S_\alpha}X_{\alpha\beta}=\bigcup_{\alpha,\beta}\left(\pi_1^{-1}(X_{\alpha\beta})\cap \pi_2^{-1}(X_{\alpha\beta})\right)\subset X\times_S X
\]
中的闭浸入,进而是到$X\times_S X$中的浸入。
\end{proof}

作为推论,回顾Proposition \ref{pro:4.1.3}的第四点,如果$S$是一个$T$-概形,则典范态射$X\times_S Y\to X\times_T Y$是两个浸入的复合,所以也是一个浸入。特别地,如果$\Delta_{S|T}$是闭浸入,则它也是闭浸入。

\begin{para}[分离性]
	设$f:X\to Y$是一个概形间态射,如果对角线$\Delta:X\to X\times_Y X$是一个闭浸入,则称$f$是一个分离态射,也称$X$是一个分离的$Y$-概形。如果$X$是分离的$\spec \zz$-概形,则简称$X$是分离的。
\end{para}

从上一个引理,我们只需$\im \Delta$是$X\times_Y X$中的闭集,就可以知道$f$是分离的。

从Lemma \ref{lem:4.1.2},所有仿射概形之间的态射都是分离的。所以仿射概形是分离概形。

此外,由于同构是闭浸入,所以单态射是分离的。

\begin{pro}
两个分离态射的复合还是分离的。
\end{pro}

\begin{proof}
设$f:X\to Y$, $g:Y\to Z$是分离态射,此时不难看到交换图成立
\[
\begin{xy}
\xymatrix{
	X \ar[rr]^{\Delta_{X|Z}}\ar[dr]_{\Delta_{X|Y}}&&X\times_Z X\\
	&X\times_Y X \ar[ur]_j&
	}
\end{xy}
\]
其中$j$来自于$X\times_Z X$的泛性质,使得$\pi_i=p_ij$,这里$\pi_i$, $p_i$分别是$X\times_Y X$和$X\times_Z X$的投影,$i$可以取$1$和$2$. 由于$\Delta_{X|Y}$是闭浸入,我们只需说明$j$是闭浸入,见Lemma \ref{lem:4.1.4}下面的推论。
\end{proof}

\begin{pro}
	设$\{V_\alpha\}$是$Y$的一个开覆盖,则态射$f:X\to Y$分离当且仅当每个限制态射$f_\alpha:f^{-1}(V_\alpha)\to V_\alpha$都分离。
\end{pro}

\begin{proof}
	$f$分离当且仅当$\im \Delta$是闭的。现在,令$U_\alpha=f^{-1}(V_\alpha)$,以及对应的$\Delta$的限制$\Delta_\alpha:U_\alpha\to U_\alpha\times_Y U_\alpha$的像$\im \Delta_\alpha$是闭的。

	如果$f$是分离的,则$\im \Delta$是闭的,而$\im \Delta_{\alpha}=\im \Delta \cap (U_\alpha\times_Y U_\alpha)$在$(U_\alpha\times_Y U_\alpha)$是闭的,此时$f_\alpha$分离。反过来,如果所有$f_\alpha$分离,则
	$
	\im \Delta=\bigcup_\alpha \im \Delta_{\alpha}
	$
	在$\bigcup_\alpha (U_\alpha\times_Y U_\alpha)$中是闭的,我们只需说明$\bigcup_\alpha (U_\alpha\times_Y U_\alpha)=X\times_Y X$,就得到了$f$的分离性。

	令$V_{\alpha\beta}=V_\alpha\cap V_\beta$以及$U_{\alpha\beta}=U_\alpha\cap U_\beta$,于是$U_{\alpha\beta}=f^{-1}(V_{\alpha\beta})$. 我们考虑纤维积$U_{\alpha\beta}\times_{V_{\alpha\beta}}U_{\alpha\beta}$,在$V_\beta$-概形范畴应用Lemma \ref{lem:3.2.11},
	\[
	U_{\alpha\beta}\times_{V_{\alpha\beta}}U_{\alpha\beta}=U_{\alpha\beta}\times_{V_\beta} U_\beta=U_{\alpha}\times_{Y} U_\beta,
	\]
	其中最后一个等式我们在$Y$-概形范畴再一次应用了Lemma \ref{lem:3.2.11}. 所以
	\[
	U_\alpha\times_Y U_\beta=U_{\alpha\beta}\times_{V_{\alpha\beta}}U_{\alpha\beta}=U_{\alpha\beta}\times_{Y}U_{\alpha\beta}
	\]
	是一个$U_\alpha\times_Y U_\alpha$的开子集,此即所证。
\end{proof}

所以,分离性对$Y$是一个局部条件,特别地,我们可以将问题归结到$Y$为仿射概形的情况。为描述这点,我们首先需要一个引理。

\begin{lem}
概形态射$f:X\to \spec R$是一个闭浸入,当且仅当$X$是仿射概形且$f$对应的坐标环同态是满的。
\end{lem}

\begin{proof}
如果$f$是闭浸入,则$X$同构于某个闭子概形$V(\mathfrak{a})\cong \spec (R/\mathfrak a)$,当然是仿射概形,且坐标环同态满。反过来,设$X=\spec S$,而$f$对应的同态为$\varphi:R\to S$,利用同构基本定理,我们可以将其分解为$\varphi=g\psi$,其中$g:R/\ker \varphi\to S$是同构,而$\psi:R\to R/\ker\varphi$是商同态,将其换到概形范畴,这就给出了闭浸入需要的分解。
\end{proof}

\begin{pro}
	设$Y=\spec R$是一个仿射概形,而$X$是一个概形,$\{U_\alpha\}$是$X$的一个仿射开覆盖,则态射$f:X\to Y$分离当且仅当,对任意一组指标$(\alpha,\beta)$,$U_{\alpha\beta}:=U_\alpha\cap U_\beta$是仿射开集且$\oo_X(U_{\alpha})|_{U_{\alpha\beta}}$和$\oo_X(U_{\beta})|_{U_{\alpha\beta}}$可以生成$R$-代数$\oo_X(U_{\alpha\beta})$.
\end{pro}

\begin{proof}
记$\pi_1$和$\pi_2$是$X\times_Y X$到$X$的两个典范投影,则$U_\alpha\times_Y U_\beta=\pi_1^{-1}(U_\alpha)\cap\pi_2^{-1}(U_\beta)$构成了$X\times_Y X$的一个开覆盖,且
\[
	\Delta^{-1}\left(\pi_1^{-1}(U_\alpha)\cap\pi_2^{-1}(U_\alpha)\right)=\Delta^{-1}\left(\pi_1^{-1}(U_\alpha)\right)\cap\Delta^{-1}\left(\pi_2^{-1}(U_\alpha)\right)=U_\alpha\cap U_\beta=U_{\alpha\beta},
\]
这里我们应用了$\pi_1\Delta=\pi_2\Delta=\id_X$. 
所以$f$分离就等价于$\Delta$在每个$U_{\alpha\beta}$上的限制$\Delta_{\alpha\beta}$是到$U_\alpha\times_Y U_\beta$上的一个闭浸入。

记$j:U_\alpha\times_Y U_\beta\hookrightarrow X \times_Y X$, $k:U_{\alpha\beta}\hookrightarrow X$和$i_\alpha:U_\alpha\hookrightarrow X$是典范含入,$\pi_\alpha:U_\alpha\times_Y U_\beta\to U_\alpha$是典范投影,则$\Delta k=j\Delta_{\alpha\beta}$且$i_\alpha\pi_\alpha=\pi_1j$,所以
\[
	k=\pi_1\Delta k=\pi_1\Delta k=\pi_1j\Delta_{\alpha\beta}=i_\alpha\pi_\alpha\Delta_{\alpha\beta},
\]
所以$\pi_\alpha\Delta_{\alpha\beta}$就是典范含入$j_\alpha:U_{\alpha\beta}\hookrightarrow U_\alpha$. 实际上,$k=i_\alpha j_\alpha$,而$i_\alpha$是单态射。同理,$\pi_\alpha\Delta_{\alpha\beta}$就是典范含入$j_\beta:U_{\alpha\beta}\hookrightarrow U_\beta$,因此
\[
	\Delta_{\alpha\beta}=j_\alpha\times_Y j_\beta:U_{\alpha\beta}\to U_\alpha\times_Y U_\beta.
\]

由于$U_\alpha\times_Y U_\beta$是仿射概形,其坐标环为$A(U_\alpha\times_Y U_\beta)=\oo_X(U_\alpha)\otimes_{R}\oo_X(U_\beta)$.
而$\Delta_{\alpha\beta}$对应的环同态即从$A(U_\alpha\times_Y U_\beta)$到$\oo_X(U_{\alpha\beta})$的映射$\varphi:a\otimes b\mapsto ab$. 
所以,上个引理告诉我们$\Delta_{\alpha\beta}$是闭浸入等价于$U_{\alpha\beta}$是仿射的且$\varphi$是满的。

最后,$\varphi$是满的,当且仅当,任取$c\in \oo_X(U_{\alpha\beta})$都可以写成一个有限和$c=\sum_{i}r_ia_ib_i$,
其中$a_i\in \oo_X(U_\alpha)|_{U_{\alpha\beta}}$, $b_i\in \oo_X(U_\beta)|_{U_{\alpha\beta}}$而$r_i\in R$,也当且仅当命题中的条件成立。
\end{proof}

\begin{coro}
如果$Y$是仿射概形,则$f:X\to Y$分离当且仅当$X$是一个分离概形。
\end{coro}

\begin{proof}
首先假设$f$是分离的,由于$Y$是仿射概形,则典范态射$Y\to \spec \zz$是分离的,进而$X\to\spec \zz$作为两个分离态射的复合,依然是分离的。反过来,回忆,如果$A$, $B$可以生成$\zz$-代数$C$,则也可以生成作为$R$-代数的$C$,所以上一个命题告诉我们,如果$X$是一个分离概形,则$f$是分离的。
\end{proof}

\section{有限性条件}

\begin{para}[(局部)Noether概形]
	如果概形$X$可以写成仿射开集的并,且每个放射开集的坐标环都是Noether环,则称$X$是一个局部Noehter概形。特别地,如果并还是有限的,则称$X$是一个Noether概形。
\end{para}

从定义,一个局部Noether概形是Noether的,当且仅当底空间拟紧。于是,对仿射概想而言,局部Noether与Noether等价。

\begin{pro}
	仿射概形$\spec R$是Noether的,当且仅当$R$是Noether的。
\end{pro}

\begin{para}[有限型态射]
	设$f:X\to Y$是一个概形间态射,如果存在$Y$的一个仿射开覆盖$\{Y_\alpha\}_\alpha$使得每一个$f^{-1}(Y_\alpha)$可以写成有限个仿射开集$\{U_{\alpha\beta}\}_\beta$的并,且环$A(U_{\alpha\beta})$是有限生成$A(Y_\alpha)$-代数,其中$A(U_{\alpha\beta})$和$A(Y_\alpha)$分别是仿射开集$U_{\alpha\beta}$和$Y_\alpha$对应的坐标环。有时候,如果略去态射的记号,会称$X$是一个有限型$Y$-概形。
\end{para}

\begin{pro}
	如果$f:X\to Y$是一个有限型态射,则取$Y$的仿射开集$V$,$f^{-1}(V)$可以写成有限个仿射开集$\{U_{\alpha}\}$的并,且环$A(U_{\alpha})$是有限生成$A(V)$-代数。
\end{pro}

换而言之,如果$f$是有限型的,则定义中的“存在$Y$的一个仿射开覆盖”可以改为“任取$Y$的一个仿射开覆盖”。

\begin{proof}
	方便起见,如果$Y$的仿射开集$V$成立“$f^{-1}(V)$可以写成有限个仿射开集$\{U_{\alpha}\}$的并,且环$A(U_{\alpha})$是有限生成$A(V)$-代数”,则我们称作$V$具有性质$P$. 此外,回忆一下,如果$f\in R$,则$R_{f}$是一个有限生成$R$-代数,因为它可以看成$R[x]/\langle fx-1\rangle$.

	首先设$U\subset Y$是一个仿射开集,具有性质$P$,则任取开集$D(g)\subset U$都具有性质$P$. 实际上,如果$f^{-1}(U)$可以写作$X$的有限仿射开集族$\{Z_i\}$的并,其中每个$A(Z_i)$就是有限生成$A(U)$-代数。考虑$f$在$Z_i$上的限制$f_i$,其对应了环同态$\varphi_i:A(U)\to A(Z_i)$. 所以,$f^{-1}(D(g))\cap Z_i=D(g_i)$,其坐标环$A(D(g_i))=A(Z_i)_{g_i}$是一个有限生成$A(Z_i)$-代数,继而是一个有限生成$A(U)$-代数,当然也是有限生成$A(D(g))=A(U)_g$-代数。综上,$f^{-1}(D(g))$可以写作$D(g_i)$的并,其中$A(D(g_i))$是有限生成$A(D(g))$-代数,因此$D(g)$满足性质$P$.

	现在,回到一般情况。假设$f:X\to Y$是一个有限型态射,则存在$Y$的一个仿射开覆盖$\{Y_\alpha\}_\alpha$使得每一个$Y_\alpha$都具有性质$P$. 任取$Y$的仿射开集$V$,由于他是拟紧的,所以可以找其有限仿射开覆盖$\{D(g_i)\}$,其中每个开集$D(g_i)$包含于某个$Y_{\alpha(i)}$. 所以,每个$D(g_i)$具有性质$P$. 下面我们说明$V$具有性质$P$.
	
	由于$D(g_i)\subset V$是拟紧的,所以可以写成有限个$D(h_{ij})$的并,其中$h_{ij}\in A(V)$. 取$\varphi$为$D(g_i)\hookrightarrow V$对应的环同态,则$D(h_{ij})=D(\varphi(h_{ij}))$,其中$\varphi(h_{ij})\in A(D(g_i))$. 由$D(g_i)$的性质$P$,$D(h_{ij})$也具有性质$P$,即每个$f^{-1}(D(h_{ij}))$都具有有限仿射开覆盖$\{U_{ijk}\}_k$使得每一个$A(U_{ijk})$都是有限生成$A(D(h_{ij}))=A(V)_{h_{ij}}$-代数,自然也是有限生成$A(V)$-代数。此时,所有仿射开集$\{U_{ijk}\}_{i,j,k}$就使得$V$满足性质$P$.
\end{proof}

\begin{lem}
	$\spec R$是一个有限型$\spec S$-概形,当且仅当$S$是一个有限生成$R$-代数。
\end{lem}