% !TEX root = main.tex
\appendix
\renewcommand{\thepara}{\Alph{chapter}.\arabic{para}}
\renewcommand\chapterimg{Pictures/5.png}
\chapter{一点点范畴论}

这里就不重复范畴、函子的定义了,并且不会补充对太过抽象的一些东西,原因是,要实际证明一点东西,还是要深入到对象里面去。对于大范围的一些结论,范畴论则提供了足够合适的一套语言。

\no{1} 设$\mathcal{C}$是一个范畴,$\mathcal{C}^\circ$是它的对偶范畴,所以任何反变函子$\varphi:\mathcal{C}\to D$都是一个(协变)函子$\varphi:\mathcal{C}^\circ\to \mathcal{D}$.

\no{2} 设$\mathcal{C}$和$\mathcal{D}$都是范畴,$\mathcal{C}$到$\mathcal{D}$之间所有的(协变)函子能如下构成一个范畴$\text{Funct}(\mathcal{C},\mathcal{D})$:首先其中的对象为$\mathcal{C}$到$\mathcal{D}$的函子。其次,设$F$和$G$是$\mathcal{C}$到$\mathcal{D}$的函子,那么函子态射(或称为自然变换)$f:F\to G$是一族态射$f(X):F(X)\to G(X)$,其中$X\in \mathcal{C}$,使得对所有的态射$\varphi:X\to Y$如下交换图成立:
\[
\xymatrix{
		F(X)\ar[rr]^{f(X)} \ar[d]_{F(\varphi)}&&G(X) \ar[d]^{G(\varphi)}\\
		F(Y)\ar[rr]^{f(Y)}&&G(Y)
	}
\]

如果$F$同构于$G$ (记作$F\cong G$),那么应该有两个自然变换$f$和$g$使得$f(X)\circ g(X)=\id_{G(X)}$以及$g(X)\circ f(X)=\id_{F(X)}$对所有$X\in \mathcal{C}$都成立。

\para 设$\mathcal{C}$是一个范畴,$A\in \mathcal{C}$是一个对象。现在我们定义一个函子$\Hom_\mathcal{C}(A,\star):\mathcal{C}\to \text{Set}$如下:对于对象,$B\mapsto \Hom_\mathcal{C}(A,B)$,对于态射$f:B\to C$,它得到了态射
\[
	\Hom_\mathcal{C}(A,f)=f_*: g\mapsto f\circ g.
\]
此时$\Hom_\mathcal{C}(A,\star)$被称为关于$A$的\idx{态射函子}。

与态射函子对偶地,可以定义所谓的\idx{反变态射函子}$\Hom_\mathcal{C}(\star,A):\mathcal{C}^{\circ}\to \text{Set}$,它记作$\hat{A}$。显然$\hat{A}\in \text{Funct}(\mathcal{C}^\circ,\text{Set})$,我们记$\hat{\mathcal{C}}=\text{Funct}(\mathcal{C}^\circ,\text{Set})$.

\lem Yoneda引理:设$\mathcal{C}$是一个范畴,$A\in \mathcal{C}$是一个对象,$F:\mathcal{C}\to \text{Set}$是一个函子,则从$\Hom_\mathcal{C}(A,\star)$到函子$F$上的自然变换一一对应着$F(A)$里面的元素。

\proof 对每个自然变换$\alpha:\Hom_\mathcal{C}(A,\star)\to F$定义$\theta(\alpha)=\alpha(A)(\id_A)\in F(A)$,现在定义一个逆映射即可。对$x\in F(A)$,定义自然变换$\psi(x):\Hom_\mathcal{C}(A,\star)\to F$通过$\psi(x)(B)(f)=F(f)(x)$,不难验证这是一个自然变换,而且$\psi$和$\theta$是互逆的。\qed

\para 设$X\in\text{Set}$是一个集合,那么它能够被看作集合$\hat{X}(e)=\Hom_{\text{Set}}(e,X)$,其中$e$是任意的一个单点集。但事实上,在任意范畴$\mathcal{C}$,类似于集合范畴中的单点集的东西是不一定存在的,所以我们转而同时考虑所有的$\Hom_{\mathcal{C}}(Y,X)$,其中$Y\in \mathcal{C}$,此时我们可以得到对象$X\in \mathcal{C}$足够多乃至于全部的信息,这就是表示论的思想。

一个函子$F\in \hat{\mathcal{C}}$被称为可表示的,如果$F\cong \hat{X}$对某个$X\in \mathcal{C}$成立。

\lem 与Yoneda引理对偶地,我们有:设$F\in \hat{\mathcal{C}}$是一个$\mathcal{C}$到集合范畴的反变函子,则$F(X)\cong \Hom_{\hat{\mathcal{C}}} (\hat{X},F)$对任意的$X\in \mathcal{C}$都成立。

特别地,如果$F=\hat{Y}$,此时$\hat{Y}(X)=\Hom_{\mathcal{C}}(X,Y)\cong \Hom_{\hat{\mathcal{C}}}(\hat{X},\hat{Y})$. 所以如果$X$表示了函子$F$,则$\Hom_{\hat{\mathcal{C}}}(\hat{Y},F)\cong \Hom_{\hat{\mathcal{C}}}(\hat{Y},\hat{X})\cong\Hom_\mathcal{C}(Y,X)$. 如果$Y$也表示函子$F$,那么就存在一个同构$\varphi:\hat{Y}\cong F$,同时诱导出了$X$和$Y$之间的同构。所以可表示函子的表示物件确定到一个同构。 

\para 设$\mathcal{J}$是一个小范畴(即范畴对象的全体能够构成一个集合),我们称函子$D:\mathcal{J}\to \mathcal{C}$为$\mathcal{C}$中的一个$\mathcal{J}$-图,或者简略叫做图。对于$j\in \mathcal{J}$,$D(j)$称为该图的一个顶点,而对任意的态射$\alpha:j_1\to j_2$,态射$D(\alpha)$称为该图的一条边。

\para 称$(A,\lambda)$为$J$-图的一个\idx{锥形},如果$A$是$\mathcal{C}$的一个对象,而$\lambda$为一族态射$\lambda_{j}:A\to D(j)$使得如下交换图对所有的顶点和边都成立
\[
	\xymatrix{
		&A \ar[dl]_{\lambda_{j}}\ar[dr]^{\lambda_{j'}}&\\
		D(j)\ar[rr]^{D(\alpha)}&&D(j')
	}
\]

称呼一个锥形$(A,\lambda)$是$J$-图的一个\idx{极限},如果对于任意的锥形$(B,\mu)$,都有唯一的态射$f:B\to A$使得如下分解$\mu_j:B\xrightarrow{f}A\xrightarrow{\lambda_j}D(j)$成立。

在实际过程中,如果极限$(A,\lambda)$中的态射族是明确的(乃至于是由对象$A$确定的),那么我们会用$A$来表示极限。并且一般将$J$-图$D$的极限写作$\varprojlim_{j\in J} D(j)$.

\para 锥形和极限都有对偶的概念,在交换图中,无外乎就是将箭头完全反过来。称$(A,\lambda)$为$J$-图的一个\idx{余锥形},如果$A$是$\mathcal{C}$的一个对象,而$\lambda$为一族态射$\lambda_{j}:D(j)\to A$使得如下交换图对所有的顶点和边都成立
\[
	\xymatrix{
		&A&\\
		D(j)\ar[rr]^{D(\alpha)} \ar[ur]^{\lambda_{j }}&&D(j')\ar[ul]_{\lambda_{j'}}
	}
\]

称呼一个余锥形$(A,\lambda)$是$J$-图的一个\idx{余极限},如果对于任意的余锥形$(B,\mu)$,都有唯一的态射$f:A\to B$使得如下分解$\mu_j:D(j)\xrightarrow{\lambda_j}A\xrightarrow{f}B$成立。一般将$J$-图$D$的余极限写作$\varinjlim_{j\in J} D(j)$.

\para 设$I$是一个偏序集,偏序关系为$\leq$,那么族$\{\{x\}\,:\, x\in I\}$构成一个(小)范畴,其中的态射定义为
\[
	\Hom_{I}\left(\{x\},\{y\}\right)=\begin{cases}
	\bigl\{x\mapsto y\bigr\}&\text{, if }x\leq y\text{;}\\
	\varnothing&\text{, otherwise}.
	\end{cases}
\]
当然,我们一般直接将$I$和$\{\{x\}\,:\, x\in I\}$等同,此时会说偏序集具有一个(小)范畴结构,其中的态射记作$x\leq y$.

如果我们定义$x\geq y$当且仅当$y\leq x$,那么新的偏序集$(I,\geq)$就是偏序集$(I,\leq)$的对偶范畴,略去偏序符号,有时候会记作$I^\circ$.

$D:I\to \mathcal{C}$是一个协变函子,就是对$i\leq j\leq k$成立$D(j\leq k)\circ D(i\leq j)=D(i\leq k)$,就是说从小到大有映射。反之,反变函子$D:I\to \mathcal{C}$对$i\leq j\leq k$成立$D(i\leq j)\circ D(j\leq k)=D(i\leq k)$,就是说从大到小有映射。

对于一个反变函子$D:I\to \mathcal{C}$,我们可以定义一个协变函子$D^\circ :I^\circ\to \mathcal{C}$通过$D^\circ(j\geq i)=D(i\leq j)$,此时对$k\geq j\geq i$成立$D^\circ(j\geq i)\circ D^\circ(k\geq j)=D^\circ(k\geq i)$,可见这确实是一个协变函子。

\para 称$I$是一个滤相的偏序集,即对于任意的$i$, $j\in I$都存在$k\in I$使得$k\leq i$和$k\leq j$同时成立。称$I$是一个定向的偏序集,即对于任意的$i$, $j\in I$都存在$k\in I$使得$i\leq k$和$j\leq k$同时成立。

很容易看到,如果偏序集$I$是滤相的(定向的),那么$I^\circ$就是定向的(滤相的),反之亦然。

作为例子,考虑拓扑空间中所有非空开集按照包含构成的偏序集,即$U\leq V$当且仅当$U\subset V$,那么这个偏序集是定向的,但不是滤相的,因为对于任意的$U$和$V$总有$U\leq U\cup V$和$V\leq U\cup V$成立,但对于不交的$U$和$V$,并不存在非空开集同时包含于他们其中。

同样是拓扑的例子,考虑所有包含$p$的非空开集按照包含构成的偏序集,那么这个偏序集既是滤相的又是定向的。

\para 将滤相的(定向的)偏序集$I$看作一个范畴,对任意的范畴$\mathcal{C}$,如果一个$I$-图$D$,$D:I\to \mathcal{C}$是协变函子,则这称为一个$C$上的一个\idx{逆系统}(\idx{定向系统})。逆系统一般谈论极限$\varprojlim_{i\in I} D(i)$,而定向系统一般谈论$\varinjlim_{i\in I} D(i)$,为了思考这个原因,我们考虑如下交换图
\[
	\xymatrix{
		&A \ar[dl]_{\lambda_{i}}\ar[dr]^{\lambda_{j}}\ar[dd]^{\lambda_{k}}&\\
		D(i)&&D(j)\\
		&D(k)\ar[ul]^{D(k\leq i)}\ar[ur]_{D(k\leq j)}&
	}
	\quad
% \xymatrix{
% 	&A \ar[dl]_{\lambda_{i}}\ar[dr]^{\lambda_{j}}\ar[dd]^{\lambda_{k}}&\\
% 	D(i)\ar[dr]_{D(i\leq k)}&&D(j)\ar[dl]^{D(j\leq k)}\\
% 	&D(k)&
% }
	\xymatrix{
		&A &\\
		\ar[ur]^{\mu_{i}}D(i)\ar[dr]_{D(i\leq k)}&&D(j)\ar[ul]_{\mu_{j}}\ar[dl]^{D(j\leq k)}\\
		&D(k)\ar[uu]_{\mu_{k}}&
	}
\]
左边是对逆系统考虑锥形,右边是对定向系统考虑余锥形。从左边来看,如果$i\leq j\leq k$成立,则$D(k)$构成了一个锥形的顶点,当我们考虑极限的时候,这时候极限就应该表现得像那些“极小”的元素$D(k)$一样。而且如果$A$是极限,那么箭头$A\to D(k)$也构成了唯一分解。类似地考虑右边的图,那些“极大”的元素$D(k)$构成了余锥形的顶点,如果$A$是余极限,那么箭头$D(k)\to A$也构成了唯一分解。

\para 考虑$I$是滤相的(定向的),而$D$是协变函子,那么$I$-图$D$是逆系统(定向系统)。考虑$I$是滤相的(定向的),而$D$是反变函子,那么$I^\circ$-图$D^\circ$是定向系统(逆系统)。

\para 设$I$是一个偏序集,如果$i\in I$有,$i>j$对所有$j\in I$不成立,或者说,与可以比较的元素$j\in I$都有$i\leq j$,则称$i$是$I$的一个极小元。如果$i\leq j$对所有$j\in I$都成立,则称$i\in I$是$I$的最小元。

在滤相的偏序集中,极小元和最小元等价。最小元是极小的,这是显然的。反之,因为如果存在极小元$i$,那么任取一个$j\in I$都存在一个$k\in I$使得$k\leq i$和$k\leq j$都成立,然而$i$是极小元,所以$i=k$,所以$i\leq j$.

同理我们可以定义极大元与最大元,在定向的偏序集中,此二者等价。

更广义地,对于一个偏序集$I$的子集$J$,如果对于任意的$i\in I$,都存在$j\in J$使得$j\leq i$,则称$J$和$I$是共尾的。显然,滤相的偏序集中的极小元构成的单点集就和原来的偏序集共尾。

\pro 设$I$-图$D$是一个逆系统,如果$I$存在一个与其共尾的子集$J$,则$J$-图$D$是一个逆系统,且$\varprojlim_{i\in I} D(i)=\varprojlim_{i\in J} D(i)$. 对偶地,设$I$-图$D$是一个定向系统,如果$I^\circ $存在一个与其共尾的子集$J^\circ $,则$J$-图$D$是一个定向系统,且$\varinjlim_{i\in I} D(i)=\varinjlim_{i\in J} D(i)$. \rule{2mm}{2mm}

所以对于逆系统,如果存在极小元,那么极小元就是它的极限,对偶地,对于定向系统,如果存在极大元,那么极大元就是它的对偶极限。这符合我们上面的直观。

\para 一个拓扑空间$X$能被看成一个范畴,对象取作他的所有开集,而态射取作
\[
	\Hom_{X}(U,V)=\begin{cases}
	\bigl\{i^U_V:U\hookrightarrow V\bigr\}&\text{, if }U\subset V\text{;}\\
	\varnothing&\text{, otherwise}.
	\end{cases}
\]

考虑拓扑空间$X$中一个非空开集族$\mathfrak{B}$,对于任意的两个$U$, $V\in \mathfrak{B}$,都存在一个$W\in \mathfrak{B}$使得$U\cup V\subset W$成立,这样的$\mathfrak{B}$按照包含构成了定向的偏序集。现在定义函子$i:\mathfrak{B}\to X$通过$i(U)=U$以及$i(U\leq V)=i^U_V:U\hookrightarrow V$,它显然成立复合$i(U\leq W)=i(V\leq W)\circ i(U\leq V)$,所以这个$\mathfrak{B}$-图$i$是一个定向系统。可以看到它的余极限$\varinjlim_{U\in \mathfrak{B}} U$实际上就是$\bigcup_{U\in \mathfrak{B}} U$,而态射族就是$i_U:U\hookrightarrow \bigcup_{U\in I}$.

同样考虑拓扑空间$X$中一个包含一个点$p\in X$的所有非空开集构成的族$\mathfrak{B}$,它按照包含构成一个滤相的偏序集。同样可以定义函子$i:\mathfrak{B}\to X$通过$i(U)=U$以及$i(U\leq V)=i^U_V:U\hookrightarrow V$,同样显然成立复合$i(U\leq W)=i(V\leq W)\circ i(U\leq V)$,所以这个$\mathfrak{B}$-图$i$是一个逆系统。

但它的极限$\varprojlim_{U\in \mathfrak{B}} U$就不一定存在,因为类比于上一个例子,它的极限应该类似于所有$\mathfrak{B}$中元素的交,但这不一定是一个开集。

\para 设我们有一个$X$上的$\mathcal{K}$-预层$\calf$,他是$X\to \mathcal{K}$的一个反变函子。赋予$X$一个偏序,此时函数$U\leq V$即$U\hookrightarrow V$. 那么$X$的那些包含$p\in X$的非空开集构成的子集族$\mathfrak{B}$继承了偏序结构,也是一个范畴,而预层$\calf$限制在$\mathfrak{B}$给出了反变函子$\mathfrak{B}\to \mathcal{K}$.

由于$\mathfrak{B}$是滤相的偏序集,而$\calf$是反变函子,因此$\mathfrak{B}^\circ$-图$\calf^\circ$是定向系统,进而我们会考虑余极限$\varinjlim_{U\in \mathfrak{B}^\circ} \calf^\circ(U)=\varinjlim_{U\in \mathfrak{B}^\circ} \calf(U)$,如果这个余极限存在,那么就称为预层$\calf$在点$p$处的纤维,记作$\calf_p$.

% \para 考虑$X$的一个拓扑基,它按照包含也构成一个偏序集,考虑它的一个子族$\mathfrak{B}$,使得每一个元素$V\in \mathfrak{B}$都有$V\subset U$,他继承了来自于拓扑基的偏序结构。

% 设我们有一个$X$上的$\mathcal{K}$-预层$\calf$,他是$X\to \mathcal{K}$的一个反变函子。赋予$X$一个偏序,此时函数$U\leq V$即$U\hookrightarrow V$. 那么$X$的那些包含$p\in X$的非空开集构成的子集族$\mathfrak{B}$继承了偏序结构,也是一个范畴,而预层$\calf$限制在$\mathfrak{B}$给出了反变函子$\mathfrak{B}\to \mathcal{K}$.

\theo 在左$R$-模范畴,任意的极限与余极限存在。 \rule{2mm}{2mm}