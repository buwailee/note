% !TEX root = main.tex

\chapter{一些定义}
\ThisULCornerWallPaper{1}{../../Pictures/7.png}

我们假设本文所出现的环都是含幺交换环,即有一个乘法单位元的交换环,并且,我们所谓的环同态,他将单位元映到单位元。特别地,我们约定,零环即$R=\{0\}$,其中乘法单位元和加法单位元相等,即$0=1$。在环(即我们这里所谓的环)范畴中,零环的地位大致相当于集合范畴中的空集。所以下面我们在提到环的时候,经常是默认该环不是零环。此外,我们下面谈及的模也都是含幺交换环上的模,所以不分左右双边。

\section{层与赋环空间}

\begin{para}
一个拓扑空间$X$能被看成一个范畴,对象取作他的所有开集,而态射取作
\[
	X(U,V)=\begin{cases}
	\bigl\{i_{VU}:U\hookrightarrow V\bigr\}&\text{, if }U\subset V\text{;}\\
	\varnothing&\text{, otherwise}.
	\end{cases}
\]

现在,任取$X$的一个开覆盖$\mathscr{U}=\{U_i\,:\, i\in I\}$,为后续的方便起见,我们记$\overline{\mathscr{U}}$为在$\mathscr{U}$中添入所有形如$U_i\cap U_j$的交集后得到的新的开覆盖。显然,通过
\[
	\overline{\mathscr{U}}(U_i,U_j)=\begin{cases}
	\bigl\{i_{U_jU_i}:U_i\hookrightarrow U_j\bigr\}&\text{, if }U_i\subset U_j\text{;}\\
	\varnothing&\text{, otherwise}.
	\end{cases}
\]
也可以认为$\overline{\mathscr{U}}$是一个范畴。
\end{para}

\begin{para}
对于任意的范畴$K$,一个拓扑空间$X$,我们称呼一个取值在$K$中的\idx{预层}$\calf$(简称为$K$-预层)是一个$X\to K$的反变函子。特别地,设$V\subset U$是两个开集,记态射$\calf(i_{UV}):\calf(U)\to \calf(V)$为$\rho_{UV}$,称为限制态射。如果$K$是集合范畴的子范畴,则$\calf(U)$中的元素$s$被称为$U$上的一个\idx{截面},而$s|_V:=\rho^U_V(s)$被称为截面$s$在$V$上的限制。

设$\calf$是一个$X$上的$K$-预层,任取一个开集$U$以及一个$U$的开覆盖$\mathscr{U}$. 显然,将$\calf$限制在$\overline{\mathscr{U}}$上也将得到一个反变函子$\calf_\mathscr{U}:\overline{\mathscr{U}}\to K$. 如果$\calf_\mathscr{U}$-图的极限是$\calf(U)$,即
\[
	{\varprojlim}_{U_\alpha\in \overline{\mathscr{U}}}\calf(U_\alpha)=\calf(U),
\]
则称预层$\calf$为一个\idx{层}。上面这个条件被称为层公理。如果需要明确取值的范畴是$K$,则称为$K$-层。
\end{para}

层公理可以用泛性质写出来,这无外乎就是极限的泛性质。设$A\in K$是范畴$K$里面的一个对象,而$\calf$是$X$上的一个预层,$U$是$X$的任意开子集, 而$\{U_\alpha\}_{\alpha\in I}$是$U$的任意开覆盖,如果存在态射族$\{\varphi_\alpha:A\to \calf(U_\alpha)\}_{\alpha\in I}$使得下图交换(画图的时候省去了限制映射)
\[
	\xymatrix{
		A\ar[r]^{\varphi_{\alpha}}\ar[d]_{\varphi_{\beta}}&\calf(U_\beta)\ar[d]\\
		\calf(U_\alpha)\ar[r]&\calf(U_{\alpha}\cap U_{\beta})
	}
\]
则存在唯一的态射$\varphi:A\to \calf(U)$使得下图交换的时候,
\[
	\xymatrix{
		A\ar@/_/[ddr]_{\varphi_{\alpha}}\ar@/^/[drr]^{\varphi_{\beta}}\ar@{.>}[dr]|{\varphi}&& \\
		&\calf(U)\ar[r]\ar[d]&\calf(U_\beta)\ar[d]\\
		&\calf(U_\alpha)\ar[r]&\calf(U_{\alpha}\cap U_{\beta})
	}
\]

如果$K$是集合范畴或者交换群范畴,则它可以容纳任意的积,很容易发现上面关于层的定义和我们熟知的层的定义是等价的,即局部相容的截面可以唯一地拼成一个整体的截面\footnote{具体来说,就是给定$U$的一个开覆盖$\{U_\alpha\}$,如果存在一族$\{s_\alpha\in U_\alpha\}$使得$s_\alpha|_{U_\alpha\cap U_\beta}=s_\beta|_{U_\alpha\cap U_\beta}$,则存在唯一的$s\in \calf(U)$使得$s|_{U_\alpha}=s_\alpha$.}. 这点只要将$A$取作$\prod_{\alpha\in I}\calf(U_\alpha)$即可。

\begin{para}
所有$X$上的预层构成一个范畴,态射就是所谓的函子间态射,或者叫做\idx{自然变换}:设$\calf$和$\calg$是$X$上的两个预层,设$\varphi(U)$是使得如下交换图交换的一族态射,
\[
	\xymatrix{
		\calf(U)\ar[rr]^{\varphi(U)} \ar[d]_{\rho^U_V}&&\calg(U) \ar[d]^{\pi^U_V}\\
		\calf(V)\ar[rr]^{\varphi(V)}&&\calg(V)
	}
\]
则称呼$\varphi$是$\calf$到$\calg$的一个态射,记做$\varphi:\calf\to\calg$. 层之间的态射就是作为预层的态射。很自然地,如果已知一个预层态射$\varphi:\calf\to\calg$,即使$\calf$是一个层,也不能断言$\calg$是一个层。
\end{para}

\begin{para}
两个层的例子,流形$M$上,每一个开集$U$上的光滑函数的全体构成一个环,记作$\calf(U)$,则这是一个预层,$U$上的截面就是$U$上的光滑函数。这显然是一个层,因为光滑性是局部概念,如果两个光滑函数限制在同一个开集上是相同的,那么我们自然可以拼成一个唯一的更大的光滑函数,一族也是类似的。

另一个例子来自代数簇,设$X$是一个代数簇,他的每一个开集$U$都是一个子代数簇,自然有相应的正则函数环$\calf(U)$,则这是一个预层,$U$上的截面就是$U$上的正则函数。这依然是一个层,因为正则性还是局部概念,可以拼接的理由同上。
\end{para}

\begin{para}
设$\calf$是$X$上的一个预层,如果范畴$K$能容纳余极限,那么定义$\calf$在$p\in X$处的\idx{茎}$\calf_p$为$\varinjlim \calf(U)$,其中$U$跑遍所有$p$的邻域,这些领域通过包含构成一个显然的direct system. 

如果$K$是集合范畴的子范畴,并设$s\in \calf(U)$是一个截面,而$p\in U$,则记$s$在典范态射$\calf(U)\to \calf_p$下的像为$s_p$. 它被称为截面$s$在$p$处的\idx{芽}。

对于交换群范畴(我们知道这是余完备的),直接写出芽的具体构造还是比较方便的,$\calf_p$中的元素(芽)是这样的等价类$\langle U,f\rangle$的全体,其中$f$是$U$上的截面,而$U$是$p$的邻域。等价关系定义为$\langle U,f\rangle\sim \langle V,g\rangle$,当且仅当存在一个$p$的邻域$W\subset U\cap V$使得$f|_W=g|_W$. 
\end{para}

利用余极限的泛性质,我们发现,预层间的态射$\varphi:\calf\to\calg$如下图在局部诱导了$\varphi_p:\calf_p\to\calg_p$.
\[
	\xymatrix{
		&&\calf(U)\ar[dl]^{\rho^U_{p}}\ar[dd]^{\rho^U_{V}}\ar@/_/[lld]_{{\rho'}^U_{p}\circ\varphi(U)} \\
		\calg_p&\calf_p\ar@{-->}[l]_(0.4){\varphi_p}&\\
		&&\calf(V)\ar[ul]_{\rho^V_{p}}\ar@/^/[llu]^{{\rho'}^V_{p}\circ\varphi(V)}
	}
\]

\begin{lem}\label{lem:1}
假设$\calf$是一个$X$上的交换群层,且$s$, $t\in \calf(U)$是两个截面。如果任取$p\in U$,都有$s_p=t_p$,则$s=t$. 此外,设$\calg$是$X$上的另一个交换群层,$\psi$, $\varphi:\calf\to\calg$都是层之间的态射,则$\psi|_U=\varphi|_U$当且仅当$\psi_p=\varphi_p$对任意的$p\in U$都成立。
\end{lem}

\begin{proof}
不失一般性,可以假设$U=X$. 由交换群层上芽的构造,我们知道,如果$s_p=t_p$,则存在$p$的一个邻域$U_p$使得$s|_{U_p}=t|_{U_p}$. 遍历$p$,$\{U_p\}$是$U$的一个开覆盖,由层上截面的黏合,我们就得到了$s=t$.

此外,如果$\psi_p=\varphi_p$对任意的$p\in X$都成立,我们要证明,任取开集$U$,有$\psi(U)=\varphi(U)$. 任取$s\in \calf(U)$,由于$(\psi(U)(s))_p=\psi_p(s_p)=\varphi_p(s_p)=(\varphi(U)(s))_p$对所有$p\in U$都成立,所以$\psi(U)(s)=\varphi(U)(s)$. 由于$s$是任意的,所以$\psi(U)=\varphi(U)$. 反过来是显然的。
\end{proof}

\begin{para}
考虑$\mathfrak{B}$是$X$的拓扑基,则同拓扑空间一样,我们取对象为基中的元素,态射为内含映射,则自然他也构成一个范畴。类似地,我们可以定义$\mathfrak{B}$上取值在$K$的预层为$\mathfrak{B}$到$K$的一个反变函子,层公理也可以一模一样搬过来。

如果范畴$K$可以容纳极限,而$\calf$是$\mathfrak{B}$上的一个预层,则我们可以通过$\calf'(U)=\varprojlim \calf(V)$定义出$X$上的一个预层$\calf'$,其中$V$跑遍集合$\{V\in \mathfrak{B}\,:\, V\subset U\}$,而这个集合通过包含构成一个显然的投影系。
\end{para}

\begin{pro}
如果范畴$K$可以容纳极限,且$\calf$是$\mathfrak{B}$上的一个层,$\mathfrak{B}$上的层诱导的预层$\calf'$也是一个层。
\end{pro}

证明完全是泛性质的练习,见[EGA, Chap 0, 3.2].

\begin{para}
对于$X$上的$K$-层$\calf$,很自然可以定义层$\calf$在$U$上的限制$\calf|_U$,这是一个$U$上的预层,对于$U$中的开集,有$\calf|_U(V)=\calf(V)$,限制态射直接从$\calf$那里继承过来。显然这也是一个层。
\end{para}

现在我们可以讨论层的黏合。

\begin{pro}
设$\{U_\alpha\}_{\alpha \in I}$是$X$的一个开覆盖,如果分别在每一个$U_\alpha$上有一个层$\calf_\alpha$,而且对于任意的指标组$(\alpha,\beta)$,都有一个同构$\theta_{\alpha\beta}:\calf_\alpha|_{U_\alpha\cap U_\beta}\to \calf_\beta|_{U_\alpha\cap U_\beta}$. 以及对于任意的指标组$(\alpha,\beta,\gamma)$成立$\theta'_{\alpha\gamma}=\theta'_{\alpha\beta}\circ \theta'_{\beta\gamma}$,其中$\theta'_{\star\star}$是态射$\theta_{\star\star}$在$U_\alpha\cap U_\beta \cap U_\gamma$上的限制。则存在一个层$\calf$和一族同构$\eta_\alpha:\calf|_{U_\alpha}\to \calf_\alpha$.
\end{pro}

\begin{proof}
选取包含在某个开覆盖的开集的全体,这是全空间的一个拓扑基,命题中的黏合条件自然地给出层的条件。详细见[EGA, Chap 0, 3.3].
\end{proof}

\begin{para}[顺像]
设$f:X\to Y$是拓扑空间之间的连续映射,而$\calf$是$X$上的一个预层,则通过$f_*\calf(U)=\calf(f^{-1}(U))$我们可以定义出$Y$上的一个预层$f_*\calf$,他被称为$\calf$关于$f$的\idx{顺像}。 由于原象是保持包含关系的,即如果$V\subset U$,则$f^{-1}(V)\subset f^{-1}(U)$,所以$f_*\calf$上的限制映射就很自然地写作$(f_*\rho)^U_V=\rho^{f^{-1}(U)}_{f^{-1}(V)}$. 不难检验,当$\calf$是$X$上的一个层的时候,$f_*\calf$是$Y$上的一个层。

设$\varphi:\calf\to\calg$是$X$上两个预层之间的态射,那么,很自然,$f$诱导了一个新的态射$f_*\varphi:f_*\calf\to f_*\calg$,具体写出来就是$(f_*\varphi)(U)=\varphi(f^{-1}(U))$,通过自然变换$\varphi$可以验证$f_*\varphi$是一个自然变换。

考虑两个连续映射的复合,很自然地可以验证$(f\circ g)_*=f_*\circ g_*$,这来自于关于原象的等式$(f\circ g)^{-1}(U)=g^{-1}(f^{-1}(U))$. 

设$u:\mathcal{F}\to\mathcal{G}$是一个$X$上预层间的态射,则$f_*u(U)=u(f^{-1}(U))$就是一个$f_*\mathcal{F}\to f_*\mathcal{G}$之间的态射。并且,设$v:\mathcal{G}\to\mathcal{H}$是令一个$X$上预层间的态射,成立$f_*(v\circ u)=f_*v\circ f_* u$. 于是$f_*$就构成了一个$X$上预层范畴到$Y$上预层范畴之间的协变函子。并且,由于$f_*$将层变成层,所以$f_*$也诱导了一个$X$上层范畴到$Y$上层范畴之间的协变函子。
\end{para}

现在,我们考虑$(f_*\calf)_{f(p)}$与$\calf_p$之间的联系。设$\rho$是$\calf$上的限制映射族,从交换图
\[
	\xymatrix{
		&&&f_*\calf(U)=\calf(f^{-1}(U))\ar[dl]^{(f_*\rho)_{p}^U}\ar[dd]^{(f_*\rho)^U_{V}}\ar@/_/[llld]_-{\rho_{p}^{f^{-1}(U)}} \\
		\calf_p&&(f_*\calf)_{f(p)}\ar@{.>}[ll]_-{f_p}&\\
		&&&f_*\calf(V)=\calf(f^{-1}(V))\ar[ul]_{(f_*\rho)_{p}^V}\ar@/^/[lllu]^-{\rho_{p}^{f^{-1}(V)}}
	}
\]
我们可以得到一个唯一的典范态射
\[
f_p:(f_*\calf)_{f(p)}\to \calf_p,
\]
这个态射的性质暂时不能确定,即使在交换群范畴,它可能既不是单的也不是满的。

\begin{para}
所谓的\idx{赋环空间}(ringed space),就是一个拓扑空间$X$,和拓扑空间上的交换环层$\mathcal{O}_X$构成的资料$(X,\mathcal{O}_X)$. 流形和上面的光滑函数构成一个赋环空间,同样,代数簇和上面的正则函数构成一个赋环空间。所谓的\idx{局部赋环空间},就是说任意一点$p\in X$处的茎$\mathcal{O}_p=\varinjlim \mathcal{O}_X(U)$是局部环。
\end{para}

流形的例子是一个局部赋环空间,因为在$p$附近的光滑函数,如果在$p$处不为零,则存在一个邻域是可逆的,故而,在$p$处为零的光滑函数构成的等价类是唯一的极大理想。代数簇的例子还是一个局部赋环空间,因为代数簇局部是仿射的,所以我们可以假设这是一个仿射簇,对于仿射簇而言,$X$中的点$p$(如果是非代数闭的情况,那就是一条Galois轨道)一一对应着坐标环$A(X)$中的一个极大理想$\mm_p$(这来自于Hilbert零点定理),而$\mathcal{O}_p$正好同构于$A(X)$关于$\mm_p$的局部化,所以是一个局部环。

\begin{para}
赋环空间之间的态射如下定义:设有赋环空间$(X,\mathcal{O}_X)$和$(Y,\mathcal{O}_Y)$,以及连续映射$\psi:X\to Y$和$Y$上的层之间的态射$\theta:\mathcal{O}_Y\to \psi_*\mathcal{O}_X$,这样的一个二元组$\Psi=(\psi,\theta)$构成了赋环空间范畴的态射,记做$\Psi:(X,\mathcal{O}_X)\to (Y,\mathcal{O}_Y)$.
\end{para}

要确切地理解是一个态射,则需要验证复合,假设$\Phi=(\psi',\theta'):(Y,\mathcal{O}_Y)\to (Z,\mathcal{O}_Z)$是另一个态射,在$\Phi\circ \Psi$中,连续映射的复合不用多说,对于层的复合,则如图所示
\[
	\mathcal{O}_Z\xrightarrow{\theta'} \psi'_*\mathcal{O}_Y \xrightarrow{\psi'_*\theta} \psi'_*\psi_*\mathcal{O}_X,
\]
写成$(\psi'_*\theta)\circ \theta'$的形式。

\begin{para}
有了顺像,我们现在考虑\idx{逆像}。设$f:X\to Y$是一个连续映射,而$\calf$是$X$上的一个层,$\calg$是$Y$上的一个预层。$f_*$是$X$上层范畴到$Y$上预层范畴的一个函子,如果它存在左伴随函子$f^*$,则$f^*\calg$被称为预层$\calg$关于$f$的逆像。换而言之,$\Hom_X(f^*\calg,\calf)$和$\Hom_Y(\calg,f_*\calf)$对层$\calf$与预层$\calg$是双函子同构。
\end{para}

% 明确来说,预层$\calg$关于$f$的逆像首先是这样一个二元组$(f^*\calg,\rho_\calg)$,其中$f^*\calg$是$X$上的层,而$\rho_\calg$是$\calg$到$f_*f^*\calg$的典范态射。其次对于任意的态射$u:\calg \to f_*\calf$,都可以找到唯一的态射$v:f^*\calg \to \calf$使得分解$u:\calg\xrightarrow{\rho_\calg} f_*f^*\calg \xrightarrow{f_*v} f_*\calf$成立。$(f^*\calg,\rho_\calg)$满足的是一个泛性质,所以逆像确定到一个同构上。当然对于逆像的存在性我们现在还不清楚。

设$u:\calg\to f_*\calf$,则在双函子同构下,在$\Hom_X(f^*\calg,\calf)$中对应的态射我们记作$u^\#$. 对应地,设$v:f^*\calg\to \calf$,在双函子同构下,在$\Hom_Y(\calg,f_*\calf)$中对应的态射我们记作$v^\flat$.

特别地,考虑$\id_{f_*\calf}:f_*\calf\to f_*\calf$,则记
\[
	f^\#_\calf=\left(\id_{f_*\calf}\right)^\#: f^*f_*\calf \to \calf.
\]
类似地,考虑$\id_{f^*\calg}$,记
\[
	f^\flat_\calg =\left(\id_{f^*\calg}\right)^\flat : \calg \to f_*f^*\calg.
\]

并且,我们可以有如下唯一分解
\[
	u^\#: f^*\calg \xrightarrow{f^*(u)} f^*f_*\calf \xrightarrow{f^\#_\calf}\calf
\]
以及
\[
	v^\flat : \calg \xrightarrow{f^\flat_\calg} f_*f^*\calg\xrightarrow{f_*(v)} f_*\calf.
\]
这些唯一分解体现的其实就是双函子同构。

\begin{pro}
\label{pro:1}对于一些特殊情况,逆像层总是存在的。考虑一个内含映射的例子,设$U$是$X$的开子集,相应的内含映射为$i:U\hookrightarrow X$. 对于$X$上的$K$-层$\calf$,$\calf|_U$即其在$U$上的逆像层$i^*\calf$.
\end{pro}

\begin{proof} 为此,我们先假设这是对的,然后检查泛性质即可。现在还缺一个典范态射$\calf\to i_*i^*\calf$,由于我们假设了$i^*\calf=\calf|_U$,所以$i_*i^*\calf(V)=i^*\calf(i^{-1}(V))=\calf|_U(U\cap V)=\calf(U\cap V)$. 那么限制映射族$\rho^V_{U\cap V}:\calf(V)\to \calf(U\cap V)$可以构成层之间的态射$\calf\to i_*i^*\calf$,我们将其记做$i^\flat$,我们希望这就是典范态射。

剩下的不过是检查$(\calf|_U,i^\flat)$所需要满足的泛性质。设$\calg$是$U$上的一个层,对于任意的态射$u:\calf\to i_*\calg$,他具有形式$u(V):\calf(V)\to i_*\calg(V)=\calg(U\cap V)$. 遍历$U$中的开集$V$,我们定义$v(V)=u(V):\calf(V)\to \calg(U\cap V)=\calg(V)$,此时由于$i^*\calf(V)=\calf|_U(V)=\calf(V)$,所以我们定义出了一族态射$v(V):i^*\calf(V)=\calf(V)\to \calg(V)$,自然也得到了态射$v:i^*\calf\to \calg$. 这样定出来的$v$显然是唯一的。最后,需要检查分解$u:\calf\xrightarrow{i^\flat} i_*i^*\calf \xrightarrow{i_*v} i_*\calg$成立,这个分解其实说白了就是,已知如下交换图(图中的限制映射都略去了)成立,
\[
	\xymatrix{
		\calf(V)\ar[r]^-{u(V)} \ar[d]&\calg(U\cap V) \ar[d]\\
		\calf(W)\ar[r]^-{u(W)}&\calg(U\cap W)
	}
\]
其中$W\subset V$是$X$中的开集,要验证定出来的$v$使得如下交换图成立。
\[
	\xymatrix{
		\calf(V)\ar[r] \ar[d]&\ar[r]^{v(U\cap V)}\calf(U\cap V) \ar[d]&\calg(U\cap V) \ar[d]\\
		\calf(W)\ar[r]&\ar[r]^{v(U\cap W)}\calf(U\cap W)&\calg(U\cap W)
	}
\]
这个并不困难,左边一个矩形的交换性来自于限制映射的复合,这是自然的,右边一个矩形的交换性来自于$v$的定义和上面一个交换图,所以最后要检验的不过是横向的那个分解,即对于任意的开集$V$,分解$u(V):\calf(V)\to \calf(U\cap V) \xrightarrow{v(U\cap V)} \calg(U\cap V)$成立。

现考虑如下交换图,
\[
	\xymatrix{
		\calf(V)\ar[r]^-{u(V)} \ar[d]&\calg(U\cap V) \ar@{=}[d]\\
		\calf(U\cap V)\ar[r]^{u(U\cap V)}&\calg(U\cap V)
	}
\]
这就得出了$u(V)=u(U\cap V)\circ \rho^V_{U\cap V}$,而$v(U\cap V)=u(U\cap V)$,所以$u(V)=v(U\cap V)\circ \rho^V_{U\cap V}$,即交换图横向的分解成立。逆像层的泛性质检验完毕。
\end{proof}

\begin{para}
设$1_X:X\to X$是恒同映射,对于$X$上的预层$\calf$,如果层$1_X^*\calf$存在,则他称为$\calf$的\idx{伴随层},有时候也被记做$\calf^+$. 对于任意的层$\calf'$和态射$u:\calf\to\calf'$,总有唯一分解$u:\calf\xrightarrow{(1_X)^\flat_\calf} \calf^+ \xrightarrow{v} \calf'$成立。
\end{para}

注意到双函子同构$\Hom_X (1_X^* \mathcal{F},\mathcal{G})\cong \Hom_X(\mathcal{F},\mathcal{G})$,前者是$X$上层范畴之间的态射集,后者是$X$上预层范畴的态射集。所以$1_X^*$是层范畴的自由对象,即层范畴到预层范畴的遗忘函子的左伴随函子。若$\calf$是一个层,则$\Hom_X(\calf,\calg)$同时是预层与层的态射集,由Yoneda引理,$\calf\cong 1_X^* \mathcal{F}$. 因此,层的伴随层总存在,且同构于他本身。

\begin{lem}
如果范畴$K$可以容纳极限,且对$Y$上的预层$\mathcal{F}$存在伴随层,则对任意连续映射$f :X\to Y$,逆像层$f^*\mathcal{F}$存在,且同构于层$U\mapsto \varprojlim_{V\supset f(U)} \mathcal{F}^+(V)$,其中$V$跑遍所有包含$f(U)$的开集。
\end{lem}

\begin{proof}
	我们先验证$U\mapsto \varprojlim_{V\supset f(U)} \mathcal{F}^+(V)$是一个层。设$U$是一个开集,而$U_\alpha\subset U$也是一个开集。由于使得$f(U_\alpha)\subset W$的$W$,我们都可以找到一个$W'$使得$f(U)\subset W'$且$W\subset W'$. 此时典范投影$\varprojlim_{V\supset f(U)} \mathcal{F}^+(V)\to \mathcal{F}^+(W')$和限制态射$\mathcal{F}^+(W')\to \mathcal{F}^+(W)$复合给出了态射$\varprojlim_{V\supset f(U)} \mathcal{F}^+(V)\to \mathcal{F}^+(W)$. 由限制态射的相容性与极限的泛性质,我们得到了新的限制态射
	\[
	{\varprojlim}_{V\supset f(U)} \mathcal{F}^+(V)\to {\varprojlim}_{V\supset f(U_\alpha)} \mathcal{F}^+(V).
	\]
	所以这确实是一个预层,暂时记作$\mathcal{H}$.

	现在设$\mathscr{U}$是$U$的一个开覆盖,若存在一个对象$A$使得下列交换图成立
	\[
	\xymatrix{
		A\ar[r]^{\varphi_{\alpha}}\ar[d]_{\varphi_{\beta}}&\mathcal{H}(U_\beta)\ar[d]\\
		\mathcal{H}(U_\alpha)\ar[r]&\mathcal{H}(U_{\alpha}\cap U_{\beta})
	}
	\]
	则我们只需证明存在唯一的态射$\varphi:A\to \mathcal{H}(U)$使得下图交换的时候,
	\[
		\xymatrix{
			A\ar@/_/[ddr]_{\varphi_{\alpha}}\ar@/^/[drr]^{\varphi_{\beta}}\ar@{.>}[dr]|{\varphi}&& \\
			&\mathcal{H}(U)\ar[r]\ar[d]&\mathcal{H}(U_\beta)\ar[d]\\
			&\mathcal{H}(U_\alpha)\ar[r]&\mathcal{H}(U_{\alpha}\cap U_{\beta})
		}
	\]
	由极限$\mathcal{H}(U)=\varprojlim_{V\supset f(U)} \mathcal{F}^+(V)$的泛性质,实际上只需要知道一族态射$A\to \mathcal{F}^+(V)$即可,其中$V\supset f(U)$. 而这直接来自于$\mathcal{F}^+$是一个层以及上面的交换图。

	任取$u:f^*\mathcal{F}\to\calg$,任取$Y$中的开集$V$,考虑开集$U=f^{-1}(V)$. 由于$f(U)=f(f^{-1}(V))=V$就是一个开集,所有包含$f(U)$的开集拥有极小元$V$,因此
	\[
	f^*\mathcal{F}(f^{-1}(V))=\mathcal{F}^+(V).
	\]
	那么
	\[
	u(f^{-1}(V)):\mathcal{F}^+(V)\to \calg(f^{-1}(V)),
	\]
	即我们给出了态射$u':\mathcal{F}^+\to f_*\calg$. 最后,再复合上典范态射$(1_X)^\flat_\calf:\calf \to \mathcal{F}^+$. 我们定义$u^\flat=u'\circ (1_X)^\flat_\calf: \mathcal{F}\to f_*\calg$.

	反过来。由于$\calg$是一个层,所以$f_*\calg$也是一个层,根据证明最开始的推理,
	\[
		f^*f_*\calg(U)=\varprojlim_{V\supset f(U)} (f_*\calg)^+(V)=\varprojlim_{V\supset f(U)} f_*\calg(V)=\varprojlim_{V\supset f(U)} \calg(f^{-1}(V))
	\]
	也是一个层。由于
	\[
	f^{-1}(V)\supset f^{-1}(f(U))\supset U,
	\]
	所以我们有一族限制态射$\mathcal{G}(f^{-1}(V))\to \mathcal{G}(U)$. 于是有映射
	\[
	\varprojlim_{V\supset f(U)} \calg(f^{-1}(V))\to \calg(f^{-1}(W))\to \mathcal{G}(U),
	\]
	其中$W$是任意的开集$W\supset f(U)$. 由限制映射的相容性,我们可以得知,这个映射的定义与$W$的选取无关。因此,我们就得到了态射$f^\#_\calg:f^*f_*\calg\to \mathcal{G}$.

	任取$v:\mathcal{F}\to f_*\mathcal{G}$,我们有$f^*(v):f^*\mathcal{F}\to f^*f_*\mathcal{G}$(这直接来自极限的泛性质),然后复合上面的态射$f^\#_\calg$,就得到了$v^\#=f^\#_\calg\circ f^*(v):f^*\mathcal{F}\to \mathcal{G}$.

	最后,无外乎检查一些泛性质。由构造,这是直接的。
\end{proof}

从这个引理,Proposition \ref{pro:1}就是显然的了。事实上,由于包含$i(V)=U\cap V$的任意开集中有极小元$U\cap V$,所以极限$\varprojlim_W \mathcal{F}^+(W)$总存在且就等于$\mathcal{F}^+(U\cap V)$. 并且,由于$\mathcal{F}$是一个层,所以$\mathcal{F}^+=\mathcal{F}$也存在。

\begin{para}
将层的限制的知识应用到赋环空间,我们就有了一个赋环空间的态射$(i,i^\flat):(U,\mathcal{O}_U)\to (X,\mathcal{O}_X)$,其中$\mathcal{O}_U=\mathcal{O}_X|_U$,他被称为赋环空间的典范内含态射。

设$\Phi:(X,\mathcal{O}_X)\to (Y,\mathcal{O}_Y)$是一个态射,他在$U\subset X$上的限制记做$\Phi|_U:=\Phi\circ (i,i^\flat)$,如果$\Phi=(\psi,\theta)$,则$\Phi|_U=\bigl(\psi|_U, \psi_*(i^\flat) \circ \theta\bigr):(U,\mathcal{O}_U)\to (Y,\mathcal{O}_Y)$,层之间的态射具体写出来即
\[
	\mathcal{O}_Y\xrightarrow{\theta} \psi_*\mathcal{O}_X \xrightarrow{\psi_*(i^\flat)} \psi_*i_*\mathcal{O}_U=\bigl(\psi|_U\bigr)_*\mathcal{O}_U,
\]
其中$\psi_*(i^\flat)(V)=i^\flat\bigl(\psi^{-1}(V)\bigr)=\rho^{\psi^{-1}(V)}_{U\cap \psi^{-1}(V)}$,设$s\in \mathcal{O}_Y(V)$是一个截面,那么表现为
\[
	\psi_*(i^\flat)(V)\circ \theta(V)(s)=\theta(V)(s)|_{U\cap \psi^{-1}(V)}.
\]
所以$\Phi|_U=(\psi|_U,\psi_*\theta|_U)$.

很容易验证,如果$V$又是$U$的一个开子集,则$(\Phi|_U)|_V=\Phi|_V$.
\end{para}

有了限制的知识,就可以讨论赋环空间间态射的黏合。

\begin{pro}
设$(X,\mathcal{O}_X)$是一个赋环空间,并且$\{U_\alpha\}_{\alpha\in I}$是一个$X$的开覆盖。如果我们有一族态射$\Phi_\alpha:(U_{\alpha},\mathcal{O}_{U_{\alpha}})\to (Y,\mathcal{O}_Y)$,对于任意的两个指标$(\alpha,\beta)$,成立$\Phi_\alpha|_{U_{\alpha}\cap U_{\beta}}=\Phi_\beta|_{U_{\alpha}\cap U_{\beta}}$,则存在一个态射$\Phi:(X,\mathcal{O}_{X})\to (Y,\mathcal{O}_Y)$使得$\Phi|_{U_{\alpha}}=\Phi_\alpha$.
\end{pro}

\begin{proof} 连续函数在开覆盖上的黏合是显然的,对于层的态射那部分,用的技术无外乎层上截面的黏合,这是很琐碎的检验,这里略去了。\end{proof}

\begin{pro}\label{proinverse}
设$\psi:X\to Y$是一个连续映射,$\calf$是$X$上的一个$K$-层,而$\calg$是$Y$上的一个$K$-预层。如果范畴$K$是交换群范畴,则$\calg$关于$\psi$的逆像层$\psi^*\calg$存在。
\end{pro}

由于交换群作为$\zz$-模,对任意族都一定存在极限. 所以从上面的引理,逆像的存在性就归结到了伴随层的存在性上面。不过下面我们还是直接证明这个命题。

\begin{proof} 
	证明是直接的构造。交换群范畴,我们知道预层的茎总是存在的。所以我们这样定义$\psi^*\calg$:对$X$中的开集$U$, 令$\psi^*\calg(U)$是函数$s:U\to \bigoplus_{x\in U}\calg_{\psi(x)}$的集合(若只是集合范畴,考虑不交并即可),这些函数需要满足
	\begin{itemize}
		\item 对任意的$x\in U$, $s(x)\in \calg_{\psi(x)}$,以及

		\item 对任意的$x\in U$,总存在$\psi(x)$的一个开邻域$V$,$x$的一个邻域$W\subset \psi^{-1}(V)\cap U$和一个元素$t\in \calg(V)$,使得当$p\in W$的时候总有$s(p)=t_{\psi(p)}$.
	\end{itemize}

	$\psi^*\calg(U)$上的加法直接定义为$(s+t)(x)=s(x)+t(x)$,因此$\psi^*\calg(U)$是一个交换群。如果$V$是$U$的开子集,那么函数限制$i^U_V:\psi^*\calg(U)\to \psi^*\calg(V)$就是所需要的预层的限制映射,记$i^U_V(s)=s|_V$,因此$\psi^*\calg$是一个预层。令$\{U_i\}$是$U$的一个开覆盖,且$s_i\in \psi^*\calg(U_i)$是局部截面。如果$s_i|_{U_i\cap U_j}=s_j|_{U_i\cap U_j}$,那么我们能定义一个函数$s:U\to \bigoplus_{x\in U}\calf_p$通过$s|_{U_i}=s_i$,函数完全由其值确定,所以唯一性也得证。因此$\psi^*\calg$是一个层。

	此外,成立典范同构$(\psi^*\calg)_x\cong \calg_{\psi(x)}$. 实际上,$s\mapsto s(x)$给出了态射$\psi^*\calg(U)\to \calg_{\psi(x)}$,他与限制映射相容,所以由余极限的泛性质,存在唯一的态射$\theta:(\psi^*\calg)_x\to \calg_{\psi(x)}$满足$\theta(s_x)=s(x)$. 它显然是单射,因为如果$s_x=t_x$,那么存在一个领域$U$使得$s|_U=t|_U$,所以$s(x)=s|_U(x)=t|_U(x)=t(x)$. 同时,它也是满射,任取$t_{\psi(x)}=(V,t)\in \calg_{\psi(x)}$,其中$V$是$\psi(x)$的一个邻域,则取$x$附近的一个邻域$U$使得$\psi(U)\subset V$,在$\psi^*\calg(U)$上,我们定义$s:p\mapsto t_{\psi(p)}$. 于是$t_{\psi(x)}=s(x)$. 

	因此,我们一般会通过$s_x=s(x)$来等同$(\psi^*\calg)_x$和$\calg_{\psi(x)}$. 下面我们默认这种等同。

	现在设$U$是$Y$的一个开集,对于$s\in \calg(U)$, 通过$s^+(x)=s_{\psi(x)}$可以定义一个截面$s^+\in \psi^*\calg(\psi^{-1}(U))=\psi_*\psi^*\calg(U)$,其中$x\in \psi^{-1}(U)$. 于是,$\psi^\flat_{\calg}(U):s\mapsto s^+$定义了态射$\psi^\flat_{\calg}(U):\calg(U)\to \psi_*\psi^*\calg(U)$. 当$U$改变的时候,还要检验$\psi^\flat_{\calg}$是一个预层之间的态射,即一个自然变换。

	由于$i$在$\psi_*\psi^*\calg$诱导的限制映射的形式为$(\psi_* i)^U_V=i^{\psi^{-1}(U)}_{\psi^{-1}(V)}$,所以对任意的$s\in \calg(U)$,因为
	\[
		(\psi_* i)^U_V\bigl(\psi^\flat_{\calg}(U)(s)\bigr)=i^{\psi^{-1}(U)}_{\psi^{-1}(V)}\left(\{x\mapsto s_{\psi(x)}\,:\, x\in \psi^{-1}(U)\}\right)=\{x\mapsto s_{\psi(x)}\,:\, x\in \psi^{-1}(V)\}=(s|_V)^+,
	\]
	以及
	\[
		\psi^\flat_{\calg}(V)\bigl(\rho^U_V(s)\bigr)=\psi^\flat_{\calg}(V)(s|_V)=(s|_V)^+.
	\]
	所以$\psi^\flat_{\calg}$是一个预层之间的态射。

	最后就是要检验泛性质,即对于任意的态射$u:\calg \to \psi_*\calf$,都可以找到唯一的态射$u^\#:\psi^*\calg \to \calf$使得分解
	\[
		u:\calg\xrightarrow{\psi^\flat_{\calg}} \psi_*\psi^*\calg \xrightarrow{\psi_*u^\#} \psi_*\calf
	\]
	成立。

	固定$U$是$X$的开子集以及$s\in \psi^*\calg(U)$,在每一点$x\in U$附近,我们可以找到一个开子集$W_x$, 一个$\psi(x)$附近的开子集$V_x$使得$W_x\subset \psi^{-1}(V_x)\cap U$以及一个$t(x)\in \calg(V_x)$使得$p\in W_x$的时候$s(p)=t(x)_{\psi(p)}$恒成立。

	% 记
	% \[
	% t(x)^+:=\left\{p\mapsto t(x)_{\psi(p)}\,:\, p\in W_x\right\}\in \psi^*\calg(W_x),
	% \]
	% 则最后一点即意味着$s|_{W_x}=t(x)^+$. 

	% 现在我们定义
	% \[
	% 	u^\#(W_p)(s|_{W_p}):=\bigl(u(V_p)(t(x))\bigr)|_{W_p},
	% \]
	% 其中$u(V_p)(t(x))\in \psi_*\calf(V_p)=\calf(\psi^{-1}(V_p))$. 

	遍历$x\in U$,$\{W_x\}_{x\in U}$是$U$的一个开覆盖,由于$\calf$是一个层,可以把$\bigl(u(V_x)(t(x))\bigr)|_{W_x}$拼成$U$上的一个截面$s'$. 实际上,利用Lemma \ref{lem:1},我们只要证明$\bigl(u(V_x)(t(x))\bigr)|_{W_x}$的芽不依赖选取的$x$即可。任取$p\in W_x$,我们有
	\[
	\bigl(u(V_x)(t(x))\bigr)_p=\psi_p\left(u(V_x)(t(x))_{\psi(p)}\right)=\psi_p\circ u_{\psi(p)}(t(x)_{\psi(p)})=\psi_p\circ u_{\psi(p)}(s(p)).
	\]
	于是,定义$u^\#(U):s\mapsto s'$就得到了层的态射$u^\#:\psi^*\calg \to \calf$的完整定义,显然这由$u$唯一确定。具体到茎间,即
	\[
	u^\#_x(s_x)=\psi_x u_{\psi(x)}(s(x))=\psi_x u_{\psi(x)}(s_x),
	\]
	这里我们应用了等同$s(x)=s_x$. 所以$u^\#_x=\psi_x u_{\psi(x)}$.

	证明的最后一步就是检验分解是否成立。任取$r\in \mathcal{G}(V)$,我们有
	\[
	u(V)(r)\in \psi_*\calf(V)=\mathcal{F}(\psi^{-1}(V)),
	\]
	任取$x\in \psi^{-1}(V)$,我们有
	\[
	(u(V)(r))_x=\psi_x (u(V)(r))_{\psi(x)}=\psi_x u_{\psi(x)}(r_{\psi(x)})=u^\#_x(r_{\psi(x)}).
	\]
	另一条路上,首先
	\[
	\psi^\flat_\calg(V)(r)=r^+\in \psi_*\psi^*\calg(V)= \psi^*\calg(\psi^{-1}(V)),
	\]
	此外,$\psi_*u^\#(V)=u^\#(\psi^{-1}(V))$,所以我们就得到了$\psi_*u^\#(V)\circ \psi^\flat_\calg(V)(r)=u^\#(\psi^{-1}(V))(r^+)\in \calf(\psi^{-1}(V))$以及
	\[
	\bigl(u^\#(\psi^{-1}(V))(r^+)\bigr)_x=u^\#_x(r^+_x)=u^\#_x(r_{\psi(x)})=(u(V)(r))_x.
	\]
	利用Lemma \ref{lem:1},在$\mathcal{F}(\psi^{-1}(V))$上成立
	\[
	\psi_*u^\#(V)\circ \psi^\flat_\calg(V)(r)=u^\#(\psi^{-1}(V))(r^+)=u(V)(r),
	\]
	由于$r$是任意的,所以$u(V)=\psi_*u^\#(V)\circ \psi^\flat_\calg(V)$,此即分解。
\end{proof}

证明之中的一些映射是值得注意的。由于$\psi^\flat_{\calg}$是预层之间的态射,所以任取$x\in X$,它诱导了
\[
	(\psi^\flat_{\calg})_{\psi(x)}:\calg_{\psi(x)}\to (\psi_*\psi^*\calg)_{\psi(x)}.
\]
结合典范态射$\psi_x:(\psi_*\psi^*\calg)_{\psi(x)}\to (\psi^*\calg)_{x}$,我们可以得到典范态射
\[
	\psi_x(\psi^\flat_{\calg})_{\psi(x)}:\calg_{\psi(x)}\to (\psi^*\calg)_{x},
\]
按照我们之前的论断,$\calg_{\psi(x)}$和$(\psi^*\calg)_{x}$之间存在典范同构,实际上,它就是$\psi_x(\psi^\flat_{\calg})_{\psi(x)}$. 为说明这点,只需说明在前面默认的等同下这个态射是恒等即可。任取$r_{\psi(x)}=(V,r)\in \calg_{\psi(x)}$,其中$V$是$\psi(x)$的一个邻域,我们有
\[
	(\psi^\flat_{\calg})_{\psi(x)}r_{\psi(x)}=(\psi^\flat_{\calg}(V)(r))_{\psi(x)}=(r^+)_{\psi(x)},
\]
其中$r^+\in \psi_*\psi^*\calg(U)$. 此时
\[
	\psi_x(\psi^\flat_{\calg})_{\psi(x)}(r_{\psi(x)})=\psi_x((r^+)_{\psi(x)})=r^+_x=r^+(x)=r_{\psi(x)},
\]
此即所证。

\begin{para}
设$f:\calf\to\calg$是两个$Y$上的交换群层之间的态射,而$\psi:X\to Y$是一个连续映射。我们考察$\psi^*f:\psi^*\calf\to \psi^*\calg$在茎上面的表现。

由如下双函子同构,
\[
	\xymatrix{
		\Hom_X(f^*\calg,\mathcal{H})\ar[rr] \ar[d]_{(\psi^*f)^*}&&\Hom_Y(\calg,f_*\mathcal{H}) \ar[d]^{f^*}\\
		\Hom_X(f^*\calf,\mathcal{H})\ar[rr]&&\Hom_Y(\calf,f_*\mathcal{H})
	}
\]
任取态射$u:\calg\to \psi_*\mathcal{H}$,应当成立
\[
	u^\#\circ \psi^* f=(uf)^\#.
\]
特别地,取$u=\psi_\calg^\flat$,则应有
\[
	\psi^* f=(\psi_\calg^\flat f)^\#.
\]
所以
\[
	(\psi^* f)_x=(\psi_\calg^\flat f)^\#_x =\psi_x (\psi_\calg^\flat f)_{\psi(x)}=\psi_x (\psi_\calg^\flat)_{\psi(x)} f_{\psi(x)}=f_{\psi(x)}.
\]
这里应用了上面我们约定的等同。
\end{para}

\begin{para}
$X$上的交换群层构成一个范畴,态射就是层之间的态射。并且,这是一个准加性范畴,常值层$0$(即任取开集$U$,$0(U)=0$)是它的零对象。

由于极限与极限是可以交换的,所以一族交换群层的极限依然是一个层。特别地,我们可以断言,任意有限极限是存在的。所以$X$上的交换群层构成一个加性范畴。

特别地,由于核是一个极限,所以对
态射$u:\calf\to\calg$,预层$U\mapsto \ker(u(U))$是一个层,我们记作$\ker u$. 结合典范含入
\[
	i(U):\ker(u(U))\hookrightarrow \calf(U),
\]
它们构成了$u$的核。因此,$X$上的交换群层范畴存在核。

在茎上,我们有$(\ker u)_x=\ker(u_x)$. 实际上,任取$(U,s)\in (\ker u)_x$,由于此时$s\in \ker u(U)$即$u(U)(s)=0$,所以$u_x(U,s)=(U,u(U)(s))=0$. 反之,任取$(U,s)\in \ker(u_x)$,我们应有$(U,u(U)(s))=0$,所以存在一个$V\subset U$使得$u(V)(s|_V)=0$,此即说明$(U,s)=(V,s|_V)\in (\ker u)_x$.

对余核,预层$U\mapsto \coker(u(U))$不一定是一个层,暂时将这个预层记作$\mathcal{K}$,所以我们转而考虑其伴随层$\mathcal{K}^+$. 可以检验,$\mathcal{K}^+$就是$u$的余核。实际上,对每一个$X$的开集$U$,我们考虑如下交换图
\[
	\xymatrix{
		\calf(U)\ar@<0.3ex>[r]^-{u(U)} \ar@<-0.3ex>[r]_-{0}&\calg(U)\ar[r]\ar[dr]_-{v(U)}&\coker(u(U))=\mathcal{K}(U)\ar[d]\ar[r]^-{i(U)}&\mathcal{K}^+(U)\ar@{.>}[ld]\\
		&&\mathcal{H}(U)&
	}
\]
图中略去了显然的典范态射的符号,而$i:\mathcal{K}\to \mathcal{K}^+$是预层到伴随层的典范态射,虚线直接来自于伴随层的泛性质。所以$\mathcal{K}^+$正是$u$的余核,我们记作$\coker u$. 于是,$X$上的交换群层范畴存在余核。
\end{para}

\begin{lem}
在茎上,我们有$(\coker u)_x\cong \coker(u_x)$. 
\end{lem}

一般而言,我们会直接等同$(\coker u)_x$和$\coker(u_x)$,这个并不会有什么问题,因为在茎是一个余极限,本身就容许一个同构。此时$(\coker u)_x=\coker(u_x)$.

\begin{proof}
预层与伴随层的茎是相同的,所以我们只需考虑预层$U\mapsto \coker(u(U))$. 首先记$\pi(U):\calg(U)\to \coker (u(U))$为典范同态,$\pi:\calg\to \coker u$是一个预层同态,以及$p_x:\calg_x\to \coker(u_x)$是$u_x:\calf_x\to\calg_x$对应的典范同态。

现在考虑如下交换图
\[
	\xymatrix{
	\calf(U)\ar[r]^{u(U)}\ar[d]^{\mu^U_x}&\calg(U)\ar[r]^-{\pi(U)}\ar[d]^{\nu^U_x}&\coker (u(U))\ar@{.>}[d]^{\alpha(U)}\\
	\calf_x\ar[r]^{u_x}&\calg_x\ar[r]^-{p_x}&\coker (u_x)
	}
\]
其中$\mu$, $\nu$是对应的限制映射。由于$0=0\mu^U_x=p_xu_x\mu^U_x=p_x\nu^U_xu(U)$,由$\coker(u(U))$的泛性质,我们有唯一分解$p_x\nu^U_x=\alpha(U)\pi(U)$,此即图中$\alpha(U)$的由来。并且,它与限制映射相容,这是分解的唯一性保证的。再由余极限的泛性质,我们得到了$\alpha_x:(\coker u)_x\to \coker (u_x)$使得$\alpha(U)=\alpha_x\pi^U_x$. 我们下面证明$\alpha_x$是一个同构。

首先,证明$\alpha_x$是一个单同态。任取$s_x\in (\coker u)_x$使得$\alpha_x(s_x)=0$. 设$s_x=(U,s)$,其中$s\in \coker (u(U))$,于是$0=\alpha_x(s_x)=\alpha_x \pi_x^U(s)=\alpha(U)(s)$. 由于$\pi(U)$是满同态,故存在$t\in \calg(U)$使得$s=\pi(U)(t)$. 由上面的交换图,我们有
\[
	0=\alpha(U)\circ \pi(U)(t)=p_x\nu_x^U(t)=p_x(t_x).
\]
从交换群中熟知的构造$\coker(u_x)=\calg_x/\im u_x$,所以$t_x\in \im u_x$,即存在一个$r_x\in \calf_x$使得$t_x=u_x(r_x)$. 因此存在$U$的开子集$V$以及$r_x=(V,r)$使得$t_x=\nu_x^V\circ u(V)(r)$,所以存在一个$V$的开子集$W$使得$t|_W=u(V)(r)|_W=u(W)(r|_W)$. 综上
\[
	s|_W=\pi(W)(t|_W)=\pi(W)\circ u(W)(r|_W)=0,
\]
这就给出了$s_x=0$.

然后证明$\alpha_x$是满同态。任取$a\in \coker(u_x)$,由于$p_x$是满同态,所以存在$b_x\in\calg_x$使得$a=p_x(b_x)$. 设$b_x=(U,b)$,所以$\alpha(U)\circ \pi(U)(b)=a$,又因为$\alpha(U)=\alpha_x\pi^U_x$,所以$a=\alpha_x\left(\pi^U_x\circ \pi(U)(b)\right)$. 因此$\alpha_x$是一个满同态。
\end{proof}

稍微提一句,在等同下$(\coker u)_x=\coker(u_x)$,上述证明中就会有等同$\pi_x=p_x$. 因此,如果将余核的典范态射依然记作$\coker$的话,我们还是有$\coker(u_x)=(\coker u)_x$,它们都是$u_x:\calf_x\to\calg_x$的余核,而$\coker u$是$u:\calf \to \calg$的余核。对$\ker$这点也是类似的。

\begin{para}
设$\calf$, $\calg$是$X$上的交换群层,如果对每一个$U$,$\calg(U)$是$\calf(U)$的子群,且典范含入$i(U):\calg(U)\to\calf(U)$是一个层的态射,则称$\calg$是$\calf$的子层。在模范畴中,商定义为典范含入的余核。现在我们依然如此定义,定义$\calf/\calg$为层$\coker(i)$,具体来说,$\calf/\calg$是预层$U\mapsto \calf(U)/\calg(U)$的伴随层。它被称为商层。在茎上,我们有$(\calf/\calg)_x=\calf_x/\calg_x$.

就像在任何加性范畴中那样,我们依然会用$\ker u$和$\coker u$来记典范态射。此时,定义$\im u=\ker\coker u$,从核与余核的知识,这当然是一个层。实际上,它就是$U\mapsto \im(u(U))$的伴随层,利用Lemma \ref{lem:1},只要说明它们拥有相同的茎即可。注意到预层和其伴随层的茎相同,所以一个是$x\mapsto \im u_x$,另一个是$x\mapsto \ker\coker u_x=\im u_x$,它们相同。
\end{para}

最后,我们考虑单态和满态。所谓单态,就是说态射是左可消的,所谓满态,就是说态射是右可消的。在准加性范畴中,$u$是单态当且仅当$\ker u=0$,$u$是满态当且仅当$\coker u=0$. 

\begin{lem}
设$\calf$, $\calg$是两个$X$上的交换群层,则$u:\calf\to\calg$是一个单(满)态当且仅当$u_x:\calf_x\to\calg_x$是单(满)同态对每一个$x\in X$都成立。
\end{lem}

所以,如果$u_x$处处是同构,则$u$也是同构。因为在Abel范畴中,既单又满即同构。

\begin{proof}
单的部分直接来自于Lemma \ref{lem:1}. 所以我们假设$u:\calf\to\calg$是一个满态,或者说$\coker(u)=0$. 首先,由于伴随层的茎不变,所以$\coker u_x=(\coker u)_x=0$,因此$u_x$是满同态。

反过来,给定任意态射$v:\calg\to\mathcal{H}$,如果$vu=0$,则其等价于$(vu)_x=v_xu_x=0$. 由于$u_x$处处是满的,所以$v_x=0$. 利用Lemma \ref{lem:1},我们得到了$v=0$. 所以$u$是一个满态。
\end{proof}

\begin{para}
下面我们说明,单态是一个核,而满态是一个余核。因此$X$上的交换群层构成一个Abel范畴。

不难检验,如果$u:\calf\to\calg$是一个单态,则$u=\ker(\coker(u))$. 检验这点并不困难,利用Lemma \ref{lem:1},只要它们在每一点上的茎相同即可,而
\[
	\ker(\coker(u))_x=\ker(\coker(u)_x)=\ker(\coker(u_x))=u_x,
\]
因为$u_x$是单的。

类似地,如果$u:\calf\to \calg$是一个满射,则$u=\coker (\ker(u))$. 实际上
\[
	\coker (\ker(u))_x=\coker(\ker(u)_x)=\coker(\ker(u_x))=u_x,
\]
因为$u_x$是满的。
\end{para}

\begin{pro}
在$X$上的交换群层的Abel范畴和$Y$上的交换群层的Abel范畴间,函子$\psi^*$是一个正合函子。
\end{pro}

\begin{proof}
首先它当然是右正和的,因为左伴随。所以我们只要证明他将单态变成单态即可。考虑$f:\calf \to \calg$是一个单态,所以$f_{\psi(x)}:\calf_{\psi(x)} \to \calg_{\psi(x)}$都是单的。在等同$(\psi^* f)_x=f_{\psi(x)}$下,他给出了
\[
	(\psi^* f)_x=f_{\psi(x)}:(\psi^*\calf)_{x} \to (\psi^*\calg)_{x}
\]
都是单的,而这又等价于$\psi^* f$是单的。
\end{proof}

% \para 设$w:\calg_1\to \calg_2$是$Y$上的一个$K$-预层态射,如果$\calg_2$和$\calg_2$关于$\psi:X\to Y$的逆像是存在的,则复合$\rho_{\calg_2}\circ w:\calg_1\to \psi_*\psi^*\calg_2$,由逆像的泛性质,我们可以找到一个态射$v:\psi^*\calg_1\to \psi^*\calg_2$来分解上面的复合态射,我们将其记做$\psi^*(w)$,他将$Y$上的$K$-预层态射,变成了$X$上的$K$-层态射。这里不加验证,$\psi^*(w_1)\circ \psi^*(w_2)=\psi^*(w_1\circ w_2)$,所以这是一个协变函子。

\section{模层}

\begin{para}[模层]
设$(X,\mathcal{O}_X)$是一个赋环空间,我们称$\calf$是一个$\mathcal{O}_X$-\idx{模层},如果对每一个开集$U$,$\calf(U)$都是一个$\mathcal{O}_X(U)$-模,并且$\calf$和$\mathcal{O}_X$的限制映射是相容的,即:设$V\subset U$是$X$的开子集,记$\calf$的限制映射为$\pi^U_V$,相应地$\mathcal{O}_X$的限制映射为$\rho^U_V$,那么对于标量乘法而成的截面$a\cdot s\in\calf(U)$,则$\pi^U_V(a\cdot s)=\rho^U_V(a)\cdot \pi^U_V(s)$. 类似地,可以定义$\mathcal{O}_X$-\idx{代数层}:环层$\calf$被称为$\mathcal{O}_X$-代数层,如果他是一个$\mathcal{O}_X$-模层。
\end{para}

我们知道$\mathcal{O}_X(U)$自己作为$\mathcal{O}_X(U)$-模的子模就是$\mathcal{O}_X(U)$的理想,所以一个$\mathcal{O}_X$的$\mathcal{O}_X$-子模层,被称为$\mathcal{O}_X$-\idx{理想层}。

\para 考虑任意的小范畴$I$,以及一系列$K$-层$\{\calf_i\}_{i\in I}$,由于极限的极限还是一个极限,所以可以看到预层$U\mapsto \varprojlim_{i\in I}\calf_i(U)$依然是一个$K$-层。他被称作\idx{极限层},记做$\varprojlim_{i\in I}\calf_i$. 放到模范畴中,考虑一系列$\mathcal{O}_X$-模层$\{\calf_i\}_{i\in I}$,则$\prod_{i\in I}\calf_i$依然是一个$\mathcal{O}_X$-模层,他被称为\idx{直积层}。

但对于直和,箭头反过来了导致预层$U\mapsto \bigoplus_{i\in I}\calf_i(U)$就不一定是一个层。所以我们定义\idx{直和层}为上述预层的伴随层,记做$\bigoplus_{i\in I}\calf_i$. 但对于有限指标集$I$,直和和直积等价,所以$U\mapsto \bigoplus_{i\in I}\calf_i(U)$此时就是一个层。

\begin{para}
考虑一个$\mathcal{O}_X$-模层同态$u:\mathcal{O}_X\to \calf$,这个同态完全由截面$s=u(X)(1)\in \calf(X)$决定,其他的任意截面$t\in \mathcal{O}_X(U)$,则$u(U)(t)=t\cdot (s|_U)$.

类似地,考虑直和层$\bigoplus_{i\in I}\mathcal{O}_X=\mathcal{O}_X^{(I)}$,对每一个$i\in I$,都有典范含入$h_i:\mathcal{O}_X\to \mathcal{O}_X^{(I)}$. 设$\calf$是一个$\mathcal{O}_X$-模层,而$u:\mathcal{O}_X^{(I)}\to \calf$是一个$\mathcal{O}_X$-模层同态,现在考虑复合$u\circ h_i:\mathcal{O}_X\to \calf$,他被截面$s_i=(u\circ h_i)(X)(1)$唯一确定,故$u$被$\{s_i\}_{i\in I}$唯一确定,对应着$u(U)\bigl(\bigoplus_i a_i\bigr)=\sum_i a_i\cdot (s_i|_U)$.
\end{para}

现在,我们来看所谓的有限生成模等概念如何推广到模层上面来。回忆一下,$R$-模$M$称被某个集合$E\subset M$生成,就是说,$M$是所有$s\in E$生成的自由模$R^{(E)}$的商模,即如下正和列成立$R^{(E)}\to M \to 0$. 所谓的有限生成,即是指存在一个有限集$E$生成$M$。

\begin{para}[有限生成模层]
一个$\mathcal{O}_X$-模层$\calf$被称作被截面$\{s_i\}_{i\in I}$生成,就是说由截面$\{s_i\}_{i\in I}$定义的同态$\mathcal{O}_X^{(I)}\to \calf$是满射。类似地,$\calf$被称为\idx{有限生成}$\mathcal{O}_X$-模层,如果存在一族有限截面$\{s_i\}_{i\in I}$生成他。
\end{para}

\begin{para}
设$u:\calf\to \calg$是$\mathcal{O}_X$-模层同态,他要使得$u(U)$是$\mathcal{O}_X(U)$-模$\calf(U)$和$\calg(U)$之间的同态。完全类似交换群层范畴,我们可以定义$U\to \ker(u(U))$, $U\to \im(u(U))$, $U\to \coker(u(U))$的伴随层为相应的$\mathcal{O}_X$-模层$\ker(u)$, $\im(u)$以及$\coker(u)$. $u$被称为满的,就是说$\im(u)=\calg$或$\coker u=0$,$u$被称为单的$\ker(u)=0$. 
\end{para}

和交换群层上类似,$\ker(u_x)=(\ker u)_x$, $\coker(u_x)=(\coker u)_x$, $\im(u_x)=(\im u)_x$. 同样,$\mathcal{O}_X$-模层构成一个Abel范畴,态射集记作$\Hom_{\oo_X}$,我们将在其中讨论正和列。下面是一个实用而简单的引理。

\begin{lem}
设$\calf\to \calg \to \mathcal{H}$是$\mathcal{O}_X$-模层的一个列,它是正和列当且仅当在每条茎上
\[
	\calf_x\to\calg_x\to \mathcal{H}_x
\]
是正和列。
\end{lem}

\begin{para}[拟凝聚层]
一个$\mathcal{O}_X$-模层$\calf$被称作\idx{拟凝聚}的,就是说,对任意的$x\in X$,存在他的一个开邻域$U$使得$\calf|_U$同构于某个形如$\mathcal{O}_X^{(I)}|_U\to \mathcal{O}_X^{(J)}|_U$的同态的余核,此处$I$, $J$是任意的指标集。如果一个$\mathcal{O}_X$-代数层作为$\mathcal{O}_X$-模层是拟凝聚的,则他被称为拟凝聚$\mathcal{O}_X$-代数层。

换句话说,对拟凝聚$\mathcal{O}_X$-模层$\calf$,在每一点附近都有一个开邻域$U$使得如下正和列\footnote{顺带一提,对于任何一个$R$-模$M$而言,正合列
\[
	R^{(I)}\to R^{(J)}\to M\to 0
\]
总是存在的。实际上,取$J=M$,同态取为$i_M:1_m\mapsto m$. 由同构基本定理,我们有短正合列
\[
	0\to \ker i_M\hookrightarrow R^{(M)} \xrightarrow{i_M} M \to 0.
\]
现在取$I=\ker i_M$,将$i_{\ker i_M}$复合上$\ker i_M\hookrightarrow R^{(M)}$,我们就得到了
\[
	R^{(\ker i)}\to R^{(M)} \xrightarrow{i_M} M \to 0.
\]
他显然是正和的。}成立
\[
	\mathcal{O}_X^{(I)}|_U\to \mathcal{O}_X^{(J)}|_U\to \calf|_U\to 0.
\]
考虑显然的正和列$0\to \mathcal{O}_X^{(I)}\to \mathcal{O}_X^{(I)} \to 0$,所以$\mathcal{O}_X^{(I)}$本身就是拟凝聚$\mathcal{O}_X$-模层。
\end{para}



% 后面我们会看到,每一个环都有一个自然的拓扑空间,在这个拓扑空间上有一个环层,这俩构成一个局部赋环空间。对于一个$R$-模$M$,因为$R$已经诱导了一个局部赋环空间,我们在上面还将搞出一个模层,称为$M$的伴随层。所谓的拟凝聚模层就是局部是某个模的伴随层。这点可以类比流形,局部是欧几里得的。

\begin{para}
设$\mathcal{F}$和$\mathcal{G}$都是$\mathcal{O}_X$-模层,我们定义$\mathcal{F}\otimes_{\mathcal{O}_X}\mathcal{G}$是预层$U\mapsto \mathcal{F}(U)\otimes_{\mathcal{O}_X(U)}\mathcal{G}(U)$的伴随层。如果张量积的下标是清楚的,我们一般会略去它。

类似模范畴中那样,我们有典范同构$\oo_X\otimes \calf\cong \calf$, $\calf\otimes \calg\cong \calg\otimes \calf$, $(\calf\otimes \calg) \otimes \mathcal{H}\cong \calf\otimes (\calg \otimes \mathcal{H})$,且这些同构都是函子性的。为了证明它,我们可以在每个开集上找到与限制映射相容的模同态,这族模同态构成了一个模层之间的态射,为证明这是同构,只需它在每一点处的茎上诱导的是同构就行。为此,我们需要下面这个命题,有了它,一切都是显然的了。
\end{para}

\begin{pro}
$(\mathcal{F}\otimes\mathcal{G})_x=\mathcal{F}_x\otimes\mathcal{G}_x$.
\end{pro}

\begin{proof}
我们直接验证$\mathcal{F}_x\otimes\mathcal{G}_x=\varinjlim_{U\ni x} \mathcal{F}(U)\otimes\mathcal{G}(U)$,即$\mathcal{F}_x\otimes\mathcal{G}_x$满足余积的泛性质。记$\rho^U_V$为预层$U\mapsto \mathcal{F}(U)\otimes \mathcal{G}(U)$的限制映射,它将$\sum_{i=1}^n s_i\otimes t_i$变成$\sum_{i=1}^n s_i|_V\otimes t_i|_V$. 记态射$\rho^U_x:\mathcal{F}(U)\otimes \mathcal{G}(U)\to \mathcal{F}_x\otimes\mathcal{G}_x$,它将$\sum_{i=1}^n s_i\otimes t_i$变成$\sum_{i=1}^n (s_i)_x\otimes (t_i)_x$.

如果存在对象$A$以及态射族$\varphi(U):\mathcal{F}(U)\otimes \mathcal{G}(U)\to A$使得,如果$V$是$U$的开子集,则$\varphi(U)=\varphi(V)\rho^U_V$. 我们需要定义一个$\varphi_x:\mathcal{F}_x\otimes\mathcal{G}_x\to A$使得$\varphi(U)=\varphi_x \rho^U_x$对每一个开集$U$都成立,并验证这是唯一满足要求的态射。此即泛性质。

任取$\sum_{i=1}^n (s_i)_x\otimes (t_i)_x$,其中$(s_i)_x=(s_i,U_i)$, $(t_i)_x=(t_i,V_i)$,任取开集$U\subset \bigcap_{i=1}^n U_i\cap V_i$,定义
\[
	\varphi_x\left(\sum_{i=1}^n (s_i)_x\otimes (t_i)_x\right)=\varphi(U)\left(\sum_{i=1}^n s_i|_U\otimes t_i|_U\right),
\]
由于$\varphi(U)=\varphi(V)\rho^U_V$,所以这个定义是良定的,不依赖于选取的$U$.

现在来检验分解,给定开集$U$,任取$\sum_{i=1}^n s_i\otimes t_i\in \mathcal{F}(U)\otimes \mathcal{G}(U)$,从构造有
\[
	\varphi_x\circ \rho^U_x\left(\sum_{i=1}^n s_i\otimes t_i\right)=\varphi_x\left(\sum_{i=1}^n (s_i)_x\otimes (t_i)_x\right)=\varphi(U)\left(\sum_{i=1}^n s_i\otimes t_i\right),
\]
所以$\varphi(U)=\varphi_x \rho^U_x$成立。

最后,$\varphi_x$的唯一性来自于所有$\mathcal{F}_x\otimes\mathcal{G}_x$的元素都可以写成某个$\rho_x^U$的像($U$并不固定),所以分解$\varphi(U)=\varphi_x \rho^U_x$告诉我们$\varphi_x$由所有的$\varphi(U)$确定,此即唯一性。再具体一些,给定$a\in \mathcal{F}_x\otimes\mathcal{G}_x$,一定存在一个$b$和$V$使得$a=\rho^V_x(b)$,此时$\varphi_x(a)=\varphi(V)(b)$. 遍历$a$,我们就得到了$\varphi_x$的唯一性。
\end{proof}

\begin{para}[顺像]
设$(X,\oo_X)$和$(Y,\oo_Y)$是两个赋环空间,而$\Psi=(\psi,\theta):(X,\oo_X)\to (Y,\oo_Y)$是赋环空间之间的态射。现在给一个$X$上的$\oo_X$-模层$\calf$,将$\calf$只看成一个交换群层,我们有顺像$\psi_*\calf$. 由于$\psi_*\calf(U)=\calf(\psi^{-1}(U))$与$\psi_*\oo_X(U)=\oo_X(\psi^{-1}(U))$相容,所以$\psi_*\calf$具有$\psi_*\oo_X$-模层结构。此外,由于存在态射$\theta:\oo_Y\to \psi_*\oo_X$,所以$\psi_*\calf$具有$\oo_Y$-模层结构,我们将这个$\oo_Y$-模层记作$\Psi_*\calf$,称为$\calf$在$\Psi$下的顺像。

设$u:\calf\to\calg$是一个$\oo_X$-模层态射,于是$\psi_*u:\psi_*\calf\to\psi_*\calf$是一个$\psi_*\oo_X$-模层态射,同时也就是一个$\oo_Y$-模层态射,我们将这个态射记作$\Psi_*u$,称为态射$u$的顺像。

结合上述两点,我们可以知道,顺像$\Psi_*$构成一个函子。
\end{para}

\begin{para}[张量积的顺像]
设$(X,\oo_X)$和$(Y,\oo_Y)$是两个赋环空间,而$\Psi=(\psi,\theta):(X,\oo_X)\to (Y,\oo_Y)$是赋环空间之间的态射。再设$\calf$, $\calg$都是$\oo_X$-模层。任取$Y$的开集$U$,我们有预层到其伴随层的典范态射
\[
	\calf(\psi^{-1}(U))\otimes_{\oo_X(\psi^{-1}(U))} \calg(\psi^{-1}(U))\to \calf\otimes \calg(\psi^{-1}(U)),
\]
他对$\oo_X(\psi^{-1}(U))=\psi_*\oo_X(U)$双线性的,所以也对$\oo_Y$是双线性的,张量积的泛性质将给出态射
\[
	\Psi_*\calf (U)\otimes_{\oo_Y(U)}\Psi_*\calg (U)\to \Psi_*(\calf\otimes \calg) (U).
\]
他与限制映射相容,由伴随层的泛性质,所以存在典范态射
\[
	\Psi_*(\calf)\otimes_{\oo_Y}\Psi_*(\calg) \to \Psi_*(\calf\otimes_{\oo_X} \calg).
\]
\end{para}

\begin{para}[逆像]
设$(X,\oo_X)$和$(Y,\oo_Y)$是两个赋环空间,而$\Psi=(\psi,\theta):(X,\oo_X)\to (Y,\oo_Y)$是赋环空间之间的态射。逆像我们这里依然定义为顺像的左伴随函子,即
\[
	\Hom_{\oo_X}(\Psi^*\calf,\calg)\cong \Hom_{\oo_Y}(\calf,\Psi_*\calg)
\]
对$\calf$, $\calg$都是函子性同构。

逆像的存在可以如下构造:首先对$\theta:\oo_Y\to \psi_*\oo_X$,由交换环层的只是,我们有对应的$\theta^\#:\psi^*\oo_Y\to \oo_X$,因此$\oo_X$具有一个$\psi^*\oo_Y$-模层结构,我们将其记作$(\oo_X)_{[\theta]}$.

此外,对$\oo_Y$-模层$\calf$,作为交换群层,存在交换群层$\psi^*\calf$. 它实际上具有一个$\psi^*\oo_Y$-模层结构,任取$s\in \psi^*\calf(U)$以及$r\in \psi^*\oo_Y(U)$,从构造(见Proposition \ref{proinverse}),我们可以如下定义$rs\in \psi^*\calf(U)$
\[
	rs(x)=r(x)s(x)\in \calf_{\psi(x)}.
\]
它确实处于$\psi^*\calf(U)$中:对$s$和$r$,存在一个$\psi(x)$的邻域$V$,$x$的一个邻域$W\subset \psi^{-1}(V)\cap U$以及$a\in \oo_Y(V)$, $b\in \calf(V)$使得任取$x\in W$都有$r(x)=a_{\psi(x)}$, $s(x)=b_{\psi(x)}$. 此时,$rs(x)=r(x)s(x)=a_{\psi(x)}b_{\psi(x)}=(ab)_{\psi(x)}$.

综上,$\psi^*\calf$和$(\oo_X)_{[\theta]}$都是$\psi^*\oo_Y$-模层。最后,我们定义
\[
	\Psi^*\calf=\psi^*(\calf)\otimes_{\psi^*\oo_Y} (\oo_X)_{[\theta]},
\]
它当然是$(\oo_X)_{[\theta]}$-模层,也自然是一个$\oo_X$-模层。

为定义一个函子,我们还需要构造其在态射上的表现。设$u:\calf_1\to\calf_2$是一个$\oo_Y$-模层态射,很容易看到,$\psi^*u :\psi^*\calf_1\to\psi^*\calf_2$是一个$\psi^*\oo_Y$-模层态射。最后,定义$\Psi^*u=\psi^*u\otimes 1$,他就给出了态射$\Psi^*\calf_1\to \Psi^*\calf_2$.
\end{para}

\begin{pro}
在茎上,$(\Psi^*\calf)_x=\calf_{\psi(x)}\otimes_{(\oo_Y)_{\psi(x)}} (\oo_X)_{x}$.
\end{pro}

\begin{proof}
它直接来自于
\[
	(\Psi^*\calf)_x=(\psi^*(\calf)\otimes_{\psi^*\oo_Y} (\oo_X)_{[\theta]})_x=(\psi^*\calf)_x\otimes_{(\oo_Y)_{\psi(x)}}((\oo_X)_{[\theta]})_x=\calf_{\psi(x)}\otimes_{(\oo_Y)_{\psi(x)}}(\oo_X)_x,
\]
注意,此时$(\oo_X)_x$具有来自于$(\theta^\#)_x:(\oo_Y)_{\psi(x)}\to (\oo_X)_x$的$(\oo_Y)_{\psi(x)}$-模结构。
\end{proof}

\begin{para}[张量积的逆像]
考虑两个$\oo_Y$-模层$\calf$和$\calg$,以及态射$\Psi=(\psi,\theta):(X,\oo_X)\to (Y,\oo_Y)$. 由于$\psi^*\calf$和$\psi^*\calg$都具有$\psi^*\oo_Y$-模层结构,所以存在$\psi^*\calf\otimes_{\psi^*\oo_Y}\psi^*\calg$.

固定$X$的子集$U$,我们可以给出一个映射
\[
	\varphi(U):\psi^*\calf(U)\times \psi^*\calg(U) \to \psi^*(\calf\otimes \calg)(U)
\]
通过
\[
	\varphi(U)(s,t)(p)=s(p)\otimes t(p).
\]
这当然是一个双线性映射,且与限制映射相容。于是,张量积泛性质给出了一个典范态射
\[
	\psi^*\calf\otimes_{\psi^*\oo_Y} \psi^*\calg \to \psi^*(\calf\otimes_{\oo_Y} \calg).
\]
这是一个同构,因为在每条茎上它都是同构。

与$(\oo_X)_{[\theta]}$取张量积,我们就得到了典范同构
\[
	\Psi^*\calf\otimes_{\oo_X} \Psi^*\calg \to \Psi^*(\calf\otimes_{\oo_Y} \calg)
\]
\end{para}

\section{仿射概形与伴随层}

\para 设$R$是一个交换环,所有$R$的素理想构成一个集合,记作$\spec(R)$,称为$R$的素谱。如果$R$是零环,则$R$没有素理想,所以$\spec(R)=\varnothing$,我们一般对此没什么兴趣。所以类似一开始说的一般原则,我们默认$R$不是零环。

我们一般以$\pp$来表示素理想,但后面会看到$\spec(R)$有一个自然的拓扑结构,所以里面的点按照拓扑空间的习惯一般是记作拉丁字母$x$或者$p$等来表示这是一个点。所以下面我们以$\pp_x$来表示$x\in \spec(R)$对应的素理想。类似的,以$R_x$来表示局部化$R_{\pp_x}$,一个环局部化后是一个局部环,只有一个极大理想,以$\mm_x$来标记$R_x$的极大理想,同时记域$k(x):=R_x/\mm_x$.

\para 现在设$f\in R$是环中的任意元素,则经过如下自然的映射
\[
	f\mapsto f(x):R\to R_x\to k(x)
\]
得到的$f(x)$成为环元$f$在$x$处的值,其中第一个映射将$f$映作$f/1$,而第二个映射为商映射。显然$f(x)=0$等价于$f\in \pp_x$,或者$(f)\subset \pp_x$.

可以看到,$f(x)$并不是真正意义上的一个函数,因为在每一点处的$k(x)$都是不同的。但从某种程度上,他很像一个函数。

我们可以在层上进行类似的操作。比方说$M$是一个光滑流形,他上面有一个光滑函数层$\mathcal{O}$,在一点$x$处,有局部环$\mathcal{O}_x$,他的极大理想$\mm_x$就是那些在$x$附近为零的光滑函数构成的等价类,此时任意$\mathcal{O}$的截面就是一个光滑函数$f$,而作为函数值的$f(x)$和通过$\mathcal{O}(M)\to \mathcal{O}_x\to \mathcal{O}_x/\mm_x\cong \rr$得到的$f(x)$是相同的。

所以,想要$\spec(R)$获得一个层结构,首先$R$中的元素作为截面要表现得像一个连续函数,这将给出$\spec(R)$的拓扑结构,然后局部环$R_x$应该表现得和$\mathcal{O}_x$类似,这将给出$\spec(R)$的一个层结构。

\begin{para}[素谱的拓扑结构]
定义$D(f):=\{x\in \spec(R)\,:\, f(x)\neq 0\}$,我们下面的目标是让$f(x)$尽可能看上去像一个连续函数,我们要证明$D(f)$满足开集公理。等价地,我们要验证$V(f):=\{x\in \spec(R)\,:\, f(x)=0\}=\{\pp\in \spec(R)\,:\, (f)\subset \pp\}$满足闭集公理,或者更一般地,我们可以验证$V(E):=\{\pp\in \spec(R)\,:\, E\subset \pp\}$满足闭集公理,其中$E$是$R$的任意子集。

\begin{pro}
$V(E)$满足闭集公理,所以$\spec(R)$以$V(E)$为闭集是一个拓扑空间。
\end{pro}

\begin{proof} 显然的是$V(1)=\varnothing$以及$V(0)=\spec(R)$,剩下的我们需要检验$\bigcap_{i\in I}V(E_i)=V(\bigcup_{i\in I}E_i)$. 设$\pp \in \bigcap_{i\in I}V(E_i)$,则$\pp\in V(E_i)$对$i\in I$都成立,这等价于$E_i\subset \pp$对$i\in I$都成立,而这又等价于$\bigcup_{i\in I}E_i \subset \pp$,因此得证。\end{proof}

作为推论,我们可以得出$D(f)$是开集,更加神奇的是,他构成了$\spec(R)$的一个拓扑基。设$U$是任意的开集,则其写作$U=\spec(R)-V(E)$的形式,而$\spec(R)-V(E)=\spec(R)-V\bigl(\bigcup_{f\in E} f\bigr)=\spec(R)-\bigcap_{f\in E}V(f)=\bigcup_{f\in E}D(f)$,所以$U=\bigcup_{f\in E}D(f)$.

此外,环的素谱满足$T_0$分离公理,实际上,设$\pp_x$和$\pp_y$分别是两个素理想,则或者有$\pp_x\not\subset \pp_y$或者有$\pp_y\not\subset \pp_x$,因此有$y\not\in \overline{\{x\}}$或者$x\not\in \overline{\{y\}}$. 用开集描述就是,$y\in \spec(A)-\overline{\{x\}}$或者$x\in \spec(A)-\overline{\{y\}}$,这就使得$T_0$分离公理成立。
\end{para}

\begin{pro}
关于$D(f)$,我们也有如下两个简单的性质:
\begin{compactenum}[~~~1.]
\item $D(fg)=D(f)\cap D(g)$.
\item 如果$a\in R$是一个可逆元,则$D(af)=D(f)$.
\end{compactenum}
\end{pro}

第二点在分式环的时候很常用,比如在$\spec R_f$中,$f/1$和$1/f$都是可逆元,故$D(g/f^n)=D(g/1)$.

\begin{proof}
第一点,任取$x\in D(fg)$,则$(fg)(x)=f(x)g(x)\neq 0\in k(x)$. 由于$k(x)$是一个域,所以$x\in D(fg)$当且仅当$f(x)\neq 0$且$g(x)\neq 0$当且仅当$x\in D(f)\cap D(g)$.

第二点,任取$x\in D(af)$即$(af)(x)=a(x)f(x)\neq 0$,由于$a\in R$可逆,所以$a(x)\neq 0$对任意的$x$都成立,故$(af)(x)\neq 0$等价于$f(x)\neq 0$等价于$x\in D(f)$.
\end{proof}

记$\pp(Y)=\{f\in A\,:\, f(Y)=0\}$,即在$Y$上消失的所有环元构成的理想,利用这个记号,显然的是$\pp(\{x\})=\pp_x$. 把$f(Y)=0$的意义详细写出来,则$\pp(Y)=\bigcap_{y\in Y}\pp_y$. 于是,$\pp(V(E))=\sqrt{(E)}$,这来自于$\sqrt{(E)}$是所有包含$(E)$的素理想的交。这是传统的Hilbert零点定理的对应。

\begin{pro}如下命题成立:
\begin{compactenum}[~~~1.]
\item $V(E)=V((E))=V(\sqrt{(E)})$;
\item 若$E\subset F$,则$V(F)\subset V(E)$;
\item $\bigcap_{i\in I}V(E_i)=V(\bigcup_{i\in I}E_i)$;
\item 若$X\subset Y$,则$\pp(Y) \subset \pp(X)$;
\item $\pp(V(E))=\sqrt{(E)}$;
\item $V(\pp(Y))=\overline{Y}$.
\end{compactenum}
\end{pro}

于是,由1, 2, 4, 5和6,可知$\pp$与$V$构成了一对Galois联络,于是它们的像构成一一对应。即$\spec R$中的闭集一一对应于$R$中的根式理想。

\begin{proof}
% 对$R$的任意子集$E$,显然有$V(E)=V((E))$,其中$(E)=ER$是$E$生成的理想,由于$\sqrt{(E)}$是所有包含$(E)$的素理想的交,所以有$V(E)=V((E))=V(\sqrt{(E)})$. 再比如,如果$E\subset F$,则$V(F)\subset V(E)$.

除了最后一点其他都是容易的。由于$\pp(Y)=\bigcap_{y\in Y}\pp_y$,所以$\pp(Y) \subset \pp_y$,其中$y\in Y$,所以$Y\subset V(\pp(Y))$. 剩下我们要证明,$V(\pp(Y))$就是包含$Y$的最小闭集。考虑$Y\subset V(E)$,对于任意的$f\in E$以及$y\in Y$,都有$f(y)=0$,从而$E\subset \pp(Y)$,进而$V(\pp(Y))\subset V(E)$. 
\end{proof}

设$x\in \spec(A)$,$\overline{\{x\}}=V(\pp(\{x\}))=V(\pp_x)$,则$\{x\}$是闭集的充分必要条件就是$V(\pp_x)=\{x\}=\{\pp_x\}$,或者说$\pp_x$是一个极大理想。所以极大理想对应着素谱中的闭点,类似于坐标环的极大理想对应着仿射簇中的点一样。

\begin{para}
设$\varphi:R_1\to R_2$是一个环同态,由于素理想在$\varphi$的原像也是素理想,所以$\varphi$通过$\spec(R_2)\ni x\mapsto \varphi^{-1}(\pp_x)\in \spec(R_1)$诱导了素谱之间的映射$^a\varphi:\spec(R_2)\to \spec(R_1)$. 这是一个连续映射,这个我们只要检验$^a\varphi^{-1}(V(f))$是一个闭集就可以了,事实上,更一般的命题是:$^a\varphi^{-1}(V(E))=V(\varphi(E))$,老实按照定义检验即可。有时也把$^a\varphi$记作$\spec \varphi$. 由于$^a(\varphi\circ \psi)={^a\psi}\circ{^a\varphi}$,所以$\spec$是环范畴到拓扑空间范畴的反变函子。

局部来看,我们用$\varphi^x$来表示$\varphi$在商环$R_1/\varphi^{-1}(\pp_x)\to R_2/\pp_x$诱导出的同态,以及用同样的符号来表示诱导的域同态$\varphi^x:k({^a}\varphi(x))\to k(x)$. 根据定义,对任意的$f\in R_1$,以及$x\in \spec(R_2)$,有等式
\[
	\varphi^x(f({^a}\varphi(x)))=\varphi^x\left(f \text{ mod } \varphi^{-1}(\pp_x)\right)=\varphi(f)\text{ mod } \pp_x=\varphi(f)(x).
\]
\end{para}

\begin{pro}\label{pro:3.8}
如下命题成立:
\begin{compactenum}[~~~1.]
\item $^a\varphi^{-1}(V(E))=V(\varphi(E))$;
\item $\varphi^x(f({^a}\varphi(x)))=\varphi(f)(x)$;
\item $\overline{{^a}\varphi(V(E))}=V(\varphi^{-1}(E))$.
\end{compactenum}
\end{pro}

\begin{proof}
除了最后一点,我们都是清楚的。由于根式理想的原象还是根式理想,所以可以干脆假设$E=\mathfrak{a}$是一个根式理想。设$Y=V(\mathfrak{a})$以及$\mathfrak{a}'=\pp({^a}\varphi(Y))$,则$\overline{{^a}\varphi(Y)}=V(\mathfrak{a}')$. 我们先证明$V(\varphi^{-1}(\mathfrak{a}))\subset V(\mathfrak{a}')=\overline{{^a}\varphi(Y)}$. 而这又等价于去证明$\mathfrak{a}'\subset \varphi^{-1}(\mathfrak{a})$. 由定义,$f'\in \mathfrak{a}'$等价于$f'({^a\varphi}(Y))=0$,因此对任意的$y\in Y$成立$\varphi(f')(y)=\varphi^y(f'(y))=0$,所以$\varphi(f')\in \pp(Y)=\pp(V(\mathfrak{a}))=\sqrt{\mathfrak{a}}=\mathfrak{a}$. 于是$f'\in \varphi^{-1}(\mathfrak{a})$.

再证明反过来的包含$\overline{{^a}\varphi(V(E))}\subset V(\varphi^{-1}(E))$. 设$\varphi^{-1}(\pp)\in {^a}\varphi(V(E))$,其中$\pp\in V(E)$或者$E\subset \pp$,则$\varphi^{-1}(E)\subset \varphi^{-1}(\pp)$,于是$\varphi^{-1}(\pp)\in V(\varphi^{-1}(E))$或者说${^a}\varphi(V(E))\subset V(\varphi^{-1}(E))$. 由于$V(\varphi^{-1}(E))$是一个闭集,所以$\overline{{^a}\varphi(V(E))}\subset V(\varphi^{-1}(E))$.
\end{proof}

\begin{lem}
设$\varphi:A'\to A$是一个环同态,对任意的$a\in A$,总存在$a'\in A'$和$A$中的可逆元$h$使得$\varphi(a')=ha$,那么$^a\varphi$是$\spec(A)$到$\im (^a\varphi)$的同胚。
\end{lem}

\begin{proof} 先对每一个$A$的子集$E$,证明总存在$A'$中的一个子集$E'$满足$V(E)=V(\varphi(E'))$即可。固定$E$,对每一个$a\in E$,取一个$a'\in A$和一个可逆元$h$使得$\varphi(a')=ha$,注意到此时$D(\varphi(a'))=D(a)$或者$V(\varphi(a'))=V(a)$,所以这些$a'$组成的集合$E'$就满足我们的需求。

首先证明$^a\varphi$是一个单射。假设$^a\varphi(x)={}^a\varphi(y)$但$x\neq y$. 由$T_0$分离性,可以假设$y\not\in \overline{\{x\}}$,选择$V(E)=\overline{\{x\}}$,于是存在$E'$使得$x\in V(E)=V(\varphi(E'))$,所以${}^a\varphi(y)={}^a\varphi(x)\in V(E')$. 因为$y\not\in \overline{\{x\}}=V(E)=V(\varphi(E'))={}^a\varphi^{-1}(V(E'))$,所以${}^a\varphi(y)\not\in V(E')$,矛盾。因此$^a\varphi$是一个单射。设$\psi$是它在$\im (^a\varphi)$上的函数逆,下面我们证明这是一个连续映射。

任取$E$,存在$E'$使得$V(E)=V(\varphi(E'))={}^a\varphi^{-1}(V(E'))$,于是
\[
	\psi^{-1}(V(E))={}^a\varphi(V(E))={}^a\varphi({}^a\varphi^{-1}(V(E')))=V(E'),
\]
任意闭集关于$\psi$的原像是闭集,故$\psi$连续。进而$^a\varphi$是$\spec(A)$到$\im (^a\varphi)$的同胚。\end{proof}

如果$\varphi$是满射,按照上一个命题,$^a\varphi$是$\spec(A)$到$\im (^a\varphi)$的同胚。设$\mathfrak{a}$是$R$的一个理想,则商映射$\varphi:R\to R/\mathfrak{a}$诱导的连续映射$^a\varphi:\spec(R/\mathfrak{a})\to \spec(R)$给出了$\spec(R/\mathfrak{a})$与$\spec(R)$的闭子集$V(\mathfrak{a})$之间的同胚。特别地,$\spec(R/\sqrt{0})$和$\spec(R)$同胚。

设$S$是$A$的一个乘性子集,我们由典范的同态$i:A\to S^{-1}A$,且,对任意的$a/s\in S^{-1}A$,我们都有$a'=a$使得$i(a')=a/1=s(a/s)$,所以引理条件满足。此时${}^ai:\spec(S^{-1}A)\to \im ({}^ai)$是一个同胚,由$S^{-1}A$的素理想的结构,它的素理想一一对应于$A$中与$S$不相交的素理想,所以$\im ({}^ai)$即$\spec A$中满足$\pp_x\cap S=\varnothing$的$x$构成的子空间。

特别地,任取$f\in R$,考虑$\spec(R_f)$. 它应与$\spec R$中所有不包含$f$的素理想构成的集合同胚,也即$D(f)$. 

\begin{pro}
$\spec R$不可约当且仅当$\sqrt{0}$是一个素理想。$\spec(R)$的不可约闭子集一一对应着$R$的素理想,特别地,$\spec(R)$的不可约分支对应着$R$的极小素理想。
\end{pro}

\begin{proof}
由于$\spec(R)\cong \spec(R/\sqrt{0})$,所以如果$\sqrt{0}$是素理想,可以假设$R$是一个整环。现在假设$\spec R$是可约的,则存在两个真闭子集$V_1$和$V_2$使得$\spec R=V_1\cup V_2$. 于是$\pp(V_1)\cap \pp(V_2)=\pp(V_1\cup V_2)=\pp (\spec R)=0$,但由于$V_1$, $V_2$是真闭子集,所以$\pp(V_1)$和$\pp(V_2)$都不为零,因此$R$不是整环。

反过来,由于$\spec(R/\sqrt{0})$和$\spec R$的同胚,可以假设$\sqrt{0}=0$. 如果$\spec R$不是整环,则存在非零的$f$, $g$使得$fg=0$,于是$\spec R=V(0)=V(fg)=V(f)\cup V(g)$. 但由于$f$, $g$非零,且$\sqrt{0}=0$,所以$V(f)$与$V(g)$是$\spec R$的真闭子集。进而$\spec R$可约。
\end{proof}

设$V(\mathfrak{a})$是$\spec R$的一个不可约闭子集,由于$V(\mathfrak{a})$与$\spec (R/\mathfrak{a})$同胚,所以$\spec (R/\mathfrak{a})$是不可约的。因此,$\sqrt{\mathfrak{a}}$是极小的素理想,故是$V(\mathfrak{a})$的一个一般点。由于$\spec R$是$T_0$的,所以这个一般点也是唯一的一般点。于是$\spec R$中的任意不可约闭子集都具有唯一的一般点。或者说,$\spec R$中所有的不可约闭子集都具有$\overline{\{x\}}$的形式,其一般点是$x$. 

反过来,给定$x\in\spec R$,$\overline{\{x\}}$是不可约闭子集,所以$x\mapsto \overline{\{x\}}$是$\spec R$上的点到$\spec R$中不可约闭子集之间的一一对应。

\begin{para}
由于在拓扑空间的一组基上给出层的公理,就可以得到一个层,所以我们通过$D(f)\mapsto R_f$来定义$\spec(R)$的一个预层结构$\mathcal{O}_R$,下面会证明这是一个环层,称为结构层或者说伴随层。类似地,如果有一个$R$-模$M$,我们可以通过$D(f)\mapsto M_f$来定义出一个$\mathcal{O}_R$-模层$\widetilde{M}$. 

特别地,$\mathcal{O}_R(\spec(R))=R_1=R$. 素谱连同其上的结构层构成一个赋环空间,我们依旧记做$\spec(R)$,如有必要,他的结构层写作$\mathcal{O}_R$,这样一个赋环空间我们称其为一个仿射概形。

% 有层同构$\widetilde{M_f}=\widetilde{M}|_{D(f)}$.

由于$x\in \spec(R)$附近的邻域$D(f)$关于包含于是共尾(cofinal)的,所以茎$(\mathcal{O}_{R})_x=R_x$,故而这正符合我们上面对于层结构的需求。故而仿射概形是局部赋环空间。
\end{para}

为了描述层结构,我们需要限制映射,这直接来自于下面的引理。

\begin{lem}
设$R$是一个环,而$M$是一个$R$-模。以下命题成立:
\begin{compactenum}[~~~1.]
\item 设$S$是$R$的一个乘性子集,而$S'$是$S$中元素的所有因子构成的新的乘性子集,则$(S')^{-1}M$和$S^{-1}M$典范同构。所以一般我们将其看成等同的,即$(S')^{-1}M=S^{-1}M$. 
\item 存在典范同构$(M_f)_g\cong M_{fg}$,我们也将其看成是等同的。
\item 如果$D(g)\subset D(f)$,则$M_{fg}=M_{g}$.
\item 设$n$是一个正整数,则$D(f^n)=D(f)$且$M_{f}=M_{f^n}$,其中$n$是一个正整数。
%\item 对一个有限环元族$\{f_\alpha\}$和一个给定的正整数$n$,$\{f_\alpha^n\}$生成单位理想当且仅当$\{f_\alpha\}$生成单位理想。
\end{compactenum}
\end{lem}

\begin{proof}
一条一条来证明:
\begin{compactenum}[~~~1.]
\item 注意到$S\subset S'$,所以存在一个典范同态$i:S^{-1}M\to (S')^{-1}M$,他将$m/s$变成$m/s$. 现在设$i(m/s)=0$,因此存在一个$s'\in S'$使得$s'm=0$. 由于$s'$是$S$中某个元$t$的因子,而因此存在$t\in S$使得$tm=0$,或者说$m/s=0$. 所以$i$是一个单同态。反之,任取$m/s'\in S^{-1}M$,因为$s'$是某个$t\in S$的因子,写作$t=s'r$. 所以$m/s'=rm/t=i(rm/t)$. 因此$i$是一个满射。

\item 由于$(M_f)_g=S^{-1}M_f$,其中$S=\{g/1$, $g^2/1$, $\dots\}$. 我们通过$(m/f^i)/(g^j/1)\mapsto f^j g^i m/(fg)^{i+j}$定义$i:(M_f)_g\to M_{fg}$. 它显然是单的,因为$(m/f^i)/(g^j/1)=0\in (M_f)_g$等价于存在$k$和$l$使得$g^kf^l m=0$,而$f^j g^i m/(fg)^{i+j}=0\in M_{fg}$告诉我们存在$n$使得$(fg)^nf^j g^i m=0$. 它也是一个满同态,因为任取$m/(fg)^i\in M_{fg}$,我们可以写成$(m/f^i)/(g^i/1)$的像。所以这是一个同构。

\item 记$S_f$为$\{1$, $f$, $f^{2}$, $\dots\}$全部因子构成的乘性子集。首先,由于$D(g)\subset D(f)$,等价地,$V(g)\supset V(f)$,因之,$\sqrt{(g)}=\pp(V(g))\subset \pp(V(f))=\sqrt{(f)}$. 所以,存在
一个正整数$n$和一个$r\in R$使得$g^n=rf$. 现在,注意$\{1$, $g$, $g^{2}$, $\dots\}$包含$g^{n+1}=rfg$,所以$S'_g$中包含$fg$,故$S_{fg}\subset S_g$. 反之,由于$g$是$fg$的一个因子,所以$S_g\subset S_{fg}$. 结合此二者,$S_g=S_{fg}$,故$M_g=M_{fg}$.

\item 注意到$D(f^n)=D(f)\cap D(f^{n-1})=\cdots=D(f)$,然后反复应用上面这个命题即可。\qedhere
\end{compactenum}
\end{proof}

所以,如果$D(g)\subset D(f)$,我们定义限制映射$\rho^{D(f)}_{D(g)}$为典范同态$M_f\to (M_f)_g=M_{fg}=M_g$. 具体到元素上,即$m/f^n\mapsto mg^n/(fg)^n$. 特别地,当$M=R$的时候,我们有
\[
	\rho^{D(f)}_{D(g)}(r_1/f^{n_1} r_2/f^{n_2})=\rho^{D(f)}_{D(g)}(r_1r_2/f^{n_1+n_2})=r_1r_2 g^{n_1+n_2}/(fg)^{n_1+n_2},
\]
另一方面
\[
	\rho^{D(f)}_{D(g)}(r_1/f^{n_1})\rho^{D(f)}_{D(g)}(r_2/f^{n_2})=(r_1g^{n_1}/(fg)^{n_1})(r_2g^{n_2}/(fg)^{n_2})=r_1r_2 g^{n_1+n_2}/(fg)^{n_1+n_2},
\]
所以$\rho^{D(f)}_{D(g)}$是一个环同态。

\begin{lem}\label{lem:3.12}
给定环$R$以及一个环元$f\in R$. 设$\{f_i\}_{i\in I}$是$R$的一族元素,则$\{D(f_i)\}_{i\in I}$是$D(f)$的一个开覆盖当且仅当$\{f_i/1\}$生成了$R_f$. 此外,$D(f)$是预紧拓扑空间。
\end{lem}

特别地,如果$f=1$,则$\{D(f_i)\}_{i\in I}$是$D(1)=\spec R$的一个开覆盖当且仅当$\{f_i\}_{i\in I}$生成了单位理想。此外,我们还得到$D(1)=\spec R$是预紧的。

\begin{proof}
任取$x\in D(f)$即$f(x)\neq 0$,由于$\{D(f_i)\}_{i\in I}$是$D(f)$的一个开覆盖,存在一个$i$使得$x\in D(f_i)$即$f_i(x)\neq 0$. 所以,如果存在一个素理想$\pp_x$使得$f_i(x)=0$对$i\in I$都成立,则$x\not\in D(f)$或等价地$x\in V(f)$. 或者说,对素理想$\pp$,$f_i\in \pp$对所有$i\in I$都成立可以推出$f\in \pp$.

现在考虑$\{f_i/1\}_{i \in I}$在$R_f$中生成的理想$\langle f_i/1\,:\,i \in I \rangle$. 如果$\langle f_i/1\,:\,i \in I \rangle$不是单位理想,因此存在一个素理想$\mm$使得$\langle f_i/1\,:\,i \in I \rangle\subset \mm$. 设$i:R\to R_f$为典范同态,则我们有$R$中理想的包含关系
\[
	\langle f_i\,:\,i \in I \rangle \subset i^{-1}\left(\langle f_i/1\,:\,i \in I \rangle\right)\subset i^{-1}(\mm),
\]
其中$i^{-1}(\mm)$是一个与乘性子集$\{1$, $f$, $f^2$, $\dots\}$不相交的素理想。但由于素理想$i^{-1}(\mm)$包含所有$f_i$,所以也包含$f$,矛盾。

反过来,如果$\{f_i/1\}_{i \in I}$生成了$R_f$,但存在一个$x\in D(f)$使得对所有$i\in I$都有$x\not\in D(f_i)$. 或者说,存在一个素理想$\pp_x$使得$f\not\in \pp_x$但是$\langle f_i\,:\,i \in I \rangle\subset \pp_x$. 现在考虑$\langle f_i\,:\,i \in I \rangle\subset \pp_x$在$i$下的像,由于$f\not\in \pp_x$,所以$i(\pp_x)$必然是$R_f$中的一个素理想,它包含了所有的$f_i/1$,这与$\{f_i/1\}_{i \in I}$生成了$R_f$矛盾。

最后,任取$D(f)$的开覆盖,我们都可以加细成$D(f)=\bigcup_i D(f_i)$的形式,所以只需考虑这种覆盖即可。由于$\{f_i/1\}$生成$R_f$,所以$1\in R_f$可以由有限个$f_i/1$线性组合而成,故这些有限个$D(f_i)$构成了$D(f)$的一个有限子覆盖。
\end{proof}

现在可以来证明$\widetilde{M}$是一个层了。

\begin{thm}
设$R$是一个环,则预层$D(f)\mapsto M_f$是$\spec R$上的一个交换群层,记作$\widetilde{M}$. 特别地,当$M=R$的时候,$\widetilde{R}$是一个环层,记作$\mathcal{O}_R$. 对于任意的$R$-模$M$,$\widetilde{M}$是一个$\mathcal{O}_R$-模层。
\end{thm}

\begin{proof}
设开集$D(f)$被$\{D(f_i)\subset D(f)\}$所覆盖,给定$f_i$,设$m_1$, $m_2\in M_f=\widetilde{M}(D(f))$限制在$D(f_i)$上都相同,我们需要证明$m_1=m_2\in M_f$. 此即截面的唯一性。

由于$D(f)$是预紧的,所以可以假设覆盖是有限的。于是存在整数$N_i$使得$(f_i^{N_i}/1)(m_1-m_2)=0$. 由于覆盖是有限的,所以可以找一个足够大的$N$使得$(f_i^N/1)(m_1-m_2)=0$对任意的$i$都成立。由于$\{D(f_i)=D(f_i^N)\}$是$D(f)$的一个开覆盖,所以$\{f_i^N/1\}$生成了$R_f$,即有$1=\sum_i r_i(f_i^N/1)$,其中$r_i\in R_f$. 因此
\[
	m_1-m_2=\sum_i r_i(f_i^N/1)(m_1-m_2)=0.
\]

此外,给定$D(f)$的一个开覆盖$\{D(f_i)\}$,其中$D(f_i)\subset D(f)$. 如果存在一族截面$\{m_i\in M_{f_i}\}$使得$m_i$和$m_j$限制在$D(f_i)\cap D(f_j)=D(f_if_j)$上时是相等的,我们需要说明存在$m\in M_f$使得$m_i=\rho^{D(f)}_{D(f_i)}(m)$. 此即局部相容的截面可以拼起来。

我们先选一个有限子覆盖$\{f_i\,:\,i\in J\}$. 在这些$M_{f_i}$中,因为$m_i$和$m_j$在$M_{f_if_j}$中相同,所以有
\[
	(f_if_j/1)^{n_{ij}} m_j = (f_if_j/1)^{n_{ij}} m_i.
\]
由于是有限覆盖,所以可以将$n_{ij}$都选成一个数$N_1$. 此外,存在$(f_i^{n_i}/1)m_i=h_i\in M_f$,由于是有限子覆盖,所以可以将所有$n_i$都选成相同的$N>N_1$. 此时,
\[
	(f_i/1)^N h_j=(f_if_j/1)^Nm_j=(f_if_j/1)^Nm_i=(f_j/1)^N h_i.
\]

因为是有限子覆盖,这些$f_i^N/1$生成了$M_f$,所以存在$r_i\in R_f$使得$1=\sum_i r_i (f_i^N/1)$. 定义$m=\sum_i r_ih_i$,我们有
\[
	(f_i/1)^N m =\sum_j r_j (f_i/1)^N h_j =\sum_j r_j (f_j/1)^N h_i= h_i = (f_i/1)^N m_i.
\]
所以$m$限制在$D(f_i)$上就是$m_i$.

最后回到原覆盖,设$f_\alpha$不在我们选取的有限子覆盖中。由于$\{D(f_\alpha f_i)\,:\, i\in J\}$构成了$D(\alpha)$的一个有限开覆盖。从上面的推导,$m$和$m_\alpha$限制到每个$D(f_\alpha f_i)$上都是相同的,都等于$m_i$限制在$D(f_\alpha f_i)=D(f_\alpha)\cap D(f_i)$上得到的截面,所以由一开始证明的唯一性,$m$限制在$D(f_\alpha)$将得到$m_\alpha$.

由于$\widetilde{R}$的限制映射是环同态,所以我们实际上已经得到了一个环层$\mathcal{O}_R$. 在任意的$D(f)$上,交换群层$\widetilde{M}(D(f))=M_f$是一个自然的$\oo_R(D(f))=R_f$-模,且这个模结构与限制映射相容。最后,对于任意的开集$U$,我们需要证明,$\widetilde{M}(U)$是一个$\oo_R(U)$-模。

我们知道
\[
	\oo_R(U)={\varprojlim}_{D(f)\subset U} \oo_R(D(f)),\quad \widetilde{M}(U)={\varprojlim}_{D(f)\subset U} \widetilde{M}(D(f)).
\]
或者具体写出构造
\[
	{\varprojlim}_{D(f)\subset U} \oo_R(D(f))=\left\{
	(r_f)\in \prod_{D(f)\subset U}\oo_R(D(f))\,:\,\text{如果$D(g)\subset D(f)\subset U$,则成立$r_g=\rho^{D(f)}_{D(g)}r_f$}
	\right\},
\]
\[
	{\varprojlim}_{D(f)\subset U} \widetilde{M}(D(f))=\left\{
	(m_f)\in \prod_{D(f)\subset U}\widetilde{M}(D(f))\,:\,\text{如果$D(g)\subset D(f)\subset U$,则成立$m_g=\rho^{D(f)}_{D(g)}m_f$}
	\right\},
\]
于是,我们可以在$\widetilde{M}(U)$上如下定义一个$\oo_R(U)$-模结构
\[
	(r_f)(m_f)=(r_fm_f).
\]
他显然与限制映射相容。因此$\widetilde{M}$是一个$\mathcal{O}_R$-模层。
\end{proof}

\begin{pro}
成立典范的层同构$\widetilde{M_f}\cong \widetilde{M}|_{D(f)}$. 
\end{pro}

\begin{proof}
首先,$D(f)$作为拓扑空间同胚于$\spec(R_f)$,他将$D(g)\subset D(f)$映射为$\spec(R_f)$中的$D(g/1)$,反过来,任取$D(h/f^n)\subset \spec(R_f)$,它在$D(f)$中对应的开集为$D(fh)=D(f)\cap D(h)$. 它们显然互逆。

其次,任取$D(g/f^n)\subset \spec(R_f)$,它上面的模是$(M_f)_g=M_{fg}$. 同样,考虑$D(g/f^n)$在$D(f)$中对应的开集,即$D(fg)\subset D(f)$,它上面的模是$M_{fg}$. 于是,模$M_{fg}$的恒等映射就定义了一族同构$\widetilde{M_f}(D(g/f^n))\to \widetilde{M}|_{D(f)}(D(fg))$,这族同构显然与限制映射相容,所以是一个层的同构。
\end{proof}

\begin{pro}所有$R$-模的伴随层实际上构成一个范畴,这里我们给出这个范畴的一些结论:
\begin{compactenum}[~~~1.]
\item $\Hom_R(M,N)\cong \Hom_{\oo_R}(\widetilde{M},\widetilde{N})$. 将$\psi\in\Hom_R(M,N)$在$\Hom_{\oo_R}(\widetilde{M},\widetilde{N})$中的像记作$\widetilde{\psi}$. 特别地,Yoneda引理告诉我们,$M=0$实际与$\widetilde{M}=0$等价。
\item $M\mapsto \widetilde{M}$, $\psi\mapsto \widetilde{\psi}$构成了一个正合函子。
\item 设$\{M_i\}$是一族$R$-模,则$M=\varinjlim_{i}M_i$的伴随层为$\varinjlim_{i} \widetilde{M_i}$.
\item 设$M$, $N$是两个$R$-模,则$M\otimes N$伴随层为$\widetilde{M}\otimes_{\oo_R}\widetilde{N}$.
\item 设$\psi:M\to N$是一个$R$-模同态,则$\ker \psi$, $\im \psi$和$\coker \psi$的伴随层分别是$\ker \widetilde{\psi}$, $\im \widetilde{\psi}$和$\coker \widetilde{\psi}$. 因此,$\psi$是单同态、满同态、同构当且仅当$\widetilde{\psi}$是单态、满态、同构。
\end{compactenum}
\end{pro}

\begin{proof}
第一点,任取$\varphi:\Hom_{\oo_R}(\widetilde{M},\widetilde{N})$,则$\varphi(\spec R):\widetilde{M}(\spec R)\to \widetilde{N}(\spec R)$就是从$M$到$N$的$R$-模同态。反过来,任取$R$-模同态$\psi:M\to N$,我们需要对每一个$D(f)$定义
\[
	\widetilde{\psi}(D(f)):\widetilde{M}(D(f))\to \widetilde{N}(D(f)),
\]
注意到
\[
	\widetilde{M}(D(f))=M_f=R_f\otimes_R M,\quad \widetilde{N}(D(f))=N_f=R_f\otimes_R N,
\]
所以我们定义$\widetilde{\psi}(D(f))=\id_{R_f}\otimes_R \psi=\psi_f$. 不难检验,这是一个$\oo_R$-模层同态,并且$\psi\mapsto \widetilde{\psi}$和$\varphi\mapsto \varphi(\spec R)$互逆。

对于第二点,考虑正合列$M_1\to M_2\to M_3$,对应地有列$\widetilde{M_1}\to \widetilde{M_2}\to \widetilde{M_3}$. 我们知道,一个模层的列是正和的,当且仅当在每条茎上是正和的,所以这里只要考察
\[
	\left(\widetilde{M_1}\right)_x\to \left(\widetilde{M_2}\right)_x\to \left(\widetilde{M_3}\right)_x
\]
或者等价的
\[
	(M_1)_{\pp_x}\to (M_2)_{\pp_x}\to (M_3)_{\pp_x}
\]
即可。由于局部化函子总是正和的,所以上述列都是正合列。

第三点,对每一个$D(f)$,我们有
\[
	\widetilde{M}(D(f))=\left({\varinjlim}_{i}M_i\right)_f,
\]
同时,由于局部化函子作为张量函子,是左伴随函子,与余极限可交换,所以
\[
	\widetilde{M}(D(f))={\varinjlim}_{i}(M_i)_f={\varinjlim}_{i}\widetilde{M_i}(D(f)),
\]
复合上预层$U\mapsto {\varinjlim}_{i}\widetilde{M_f}(U)$到其伴随层${\varinjlim}_{i}\widetilde{M_f}$的典范态射,我们就得到了预层同态$\widetilde{M}\to {\varinjlim}_{i}\widetilde{M_f}$. 现在回到每条茎上,这个态射诱导了茎之间的同构,因为局部化函子作为张量函子,是左伴随函子,与余极限可交换,所以上述预层同态是一个同构。

第四点,对每一个$D(f)$,我们有
\[
	\widetilde{M\otimes_R N}(D(f))=(M\otimes_R N)_f=M_f\otimes_{R_f}N_f=\widetilde{M}(D(f))\otimes_{\oo_R(D(f))}\widetilde{N}(D(f)),
\]
复合上预层$U\mapsto \widetilde{M}(U)\otimes_{\oo_R(U)}\widetilde{N}(U)$到其伴随层的典范态射,我们就得到了预层之间的态射$\widetilde{M\otimes_R N}\to \widetilde{M}\otimes_{\oo_R}\widetilde{N}$. 回到每条茎上,上述态射诱导了以下同构(等同)
\[
	(\widetilde{M}\otimes_{\oo_R}\widetilde{N})_x=\widetilde{M}_x\otimes_{(\oo_R)_x}\widetilde{N}_x=M_{x}\otimes_{R_x}N_x=(M\otimes_R N)_x,
\]
所以预层之间的态射是同构。

最后一点考察正合列
\[
	0\to \ker \psi \to M \to \im \psi\to 0
\]
和
\[
	0\to \im \psi \to N \to \coker \psi\to 0
\]
即可。
\end{proof}

\section{概形}

\para 设$(X,\mathcal{O}_X)$是一个赋环空间,如果对于任意的$x\in X$,他都一个开邻域$U$使得$(U,\mathcal{O}_X|_U)$同构于一个仿射概形$\spec{R}$,则称呼$(X,\mathcal{O}_X)$是一个概形。很容易看到任意的概形是局部赋环空间,因为仿射概形是局部赋环空间。

\subsection{概形的乘积}

\subsection{子概型}

\subsection{分离性}

\subsection{有限性条件}