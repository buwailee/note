%!TEX program = xelatex
\documentclass[8pt]{article}
\usepackage{amssymb,amsfonts,amsthm,amsmath,bm,mathrsfs}
\usepackage{ctex}
%\usepackage[adobefonts]{ctex}%ubuntu用
%\usepackage{extarrows}
\usepackage[left=21mm,text={148mm,200mm},paperwidth=185mm,paperheight=250mm,includehead,vmarginratio=1:1]{geometry}
% \usepackage[all]{xy}
\usepackage{tikz}
\usepackage{titletoc}%使用目录
	\theoremstyle{plain}%定理环境样式
	\newtheorem{pro}{Proposition}[section]% 定义命题环境
	\newtheorem{theo}{Theorem}[section]% 定义定理环境
	\newtheorem{defi}{Definition}[section]% 定义定义环境
	\newtheorem{lem}{Lemma}[section]% 定义例子环境

\definecolor{shadecolor}{rgb}{0.92,0.92,0.92}

\newcommand{\no}[1]{{$(#1)$}}
% \renewcommand{\not}[1]{#1\!\!\!/}
\newcommand{\rr}{\mathbb{R}}
\newcommand{\zz}{\mathbb{Z}}
\newcommand{\aaa}{\mathfrak{a}}
\newcommand{\pp}{\mathfrak{p}}
\newcommand{\mm}{\mathfrak{m}}
\newcommand{\dd}{\mathrm{d}}
\newcommand{\oo}{\mathcal{O}}
\newcommand{\calf}{\mathcal{F}}
\newcommand{\calg}{\mathcal{G}}
\newcommand{\bbp}{\mathbb{P}}
\newcommand{\bba}{\mathbb{A}}
\newcommand{\osub}{\underset{\mathrm{open}}{\subset}}
\newcommand{\csub}{\underset{\mathrm{closed}}{\subset}}

\DeclareMathOperator{\im}{Im}
\DeclareMathOperator{\Hom}{Hom}
\DeclareMathOperator{\id}{id}
\DeclareMathOperator{\rank}{rank}
\DeclareMathOperator{\tr}{tr}
\DeclareMathOperator{\supp}{supp}
\DeclareMathOperator{\coker}{coker}
\DeclareMathOperator{\codim}{codim}
\DeclareMathOperator{\height}{height}
\DeclareMathOperator{\sign}{sign}

\DeclareMathOperator{\ann}{ann}
\DeclareMathOperator{\Ann}{Ann}
\DeclareMathOperator{\ev}{ev}

\begin{document}
下面出现的环都是交换含幺环。
\section{Field extensions}
设有两个环$A$, $B$和一个环同态$f:A\to B$,再设$a\in A$, $b\in B$,我们可以通过$f$定义他们的乘法为$a\cdot b=f(a)b\in B$,这样环$B$就被赋予了一个$A$-模结构。
\begin{defi}
	环$B$若被赋予一个$A$-模结构,则称$B$是一个$A$-代数。
\end{defi}
设$B$是$A$-代数,有$f:A\to B$,设$C$也是$A$-代数,有$g:A\to C$,那么$A$-代数之间的同态$h:B\to C$,首先是$B$和$C$之间的环同态,还要和$A$-模结构相容,即$g=h\circ f$.
\begin{defi}
设$B$是$A$-代数,我们称呼$B$是有限生成$A$-代数,如果他同构于$A[x_1,\cdots ,x_n]/\aaa$,其中$\aaa$是$A[x_1,\cdots ,x_n]$的一个理想。一个环称为有限生成的就是指他作为$\zz$-代数是有限生成的。如果$B$作为$A$-模是有限生成的,则称$B$作为$A$-代数是有限的。
\end{defi}

如果$k$是一个域,则任意的$k$-代数都是$k$-矢量空间。与此同时,赋予$k$-代数的$k$-模结构的那个同态的核$\ker(f)$作为域$k$中的理想,他只能是$k$本身或者$\{0\}$,前者太平凡,我们舍去,后者得出了$f:k\to f(k)$是一个域同构。

\begin{defi}
	如果域$K$是一个$k$-代数,称$K$是$k$的一个域扩张,记作$K/k$。如果域$K$还是一个有限生成$k$-代数,称$K$是$k$的一个有限生成扩张。

	因为$K$是$k$-矢量空间,我们可以定义域扩张的大小为$[K:k]=\dim_k(K)$.若$[K:k]$有限,称这个扩张是有限扩张。
\end{defi}
前面说了,对于任意的$k$-代数$K$,$k$都同构于$K$中的一个子域,所以通常也将域扩张定义为包含$k$的更大的域。为了行文的简练,必要的时候,我们就假设域扩张为包含我们的域更大的域。
\begin{defi}
	设$B$是环,$A$是他的子环,如果对$a\in B$,存在$f\in A[x]$使得$f(a)=0$,称$a$在$A$上代数。如果$B$中任意的元素都在$A$上代数,则称$B$在$A$上代数。特别地,设$K/k$是一个扩张,若$K$在$k$上代数,则$K$被称为$k$的一个代数扩张。
\end{defi}
每一个$k$中元素当然在$k$上代数,因为他是线性多项式的根。如果$\alpha$有逆且在$k$上代数,那么他的逆$1/\alpha$也在$k$上代数。实际上,因为$\alpha$在$k$上代数,所以存在多项式$f=\sum_{i=0}^na_ix^i$使得$f(\alpha)=0$。很容易检验,多项式$g=\sum_{i=0}^na_{n-i}x^i$使得$g(1/\alpha)=0$成立,所以$1/\alpha$在$k$上代数。

作为域扩张的例子,考虑多项式环$k[x_1,\cdots,x_n]$是一个$k$-代数,他的商域$F(k[x_1,\cdots,x_n])$就是$k$的一个扩张,并且$\dim_k(F(k[x_1,\cdots,x_n]))=\infty$,实际上,比如$\{x_1,\cdots,x_1^n,\cdots\}$是线性无关的。或者,如果$\mm$是$k[x_1,\cdots,x_n]$的一个极大理想,则$k[x_1,\cdots,x_n]/\mm$也是$k$的一个扩张,后面我们会看到这个扩张是一个有限扩张。

假如有一个域$K$,而$k$是他的子域,那么必然存在一个$\alpha\in K$但$\alpha\notin k$,我们考虑$K$中包含$\alpha$的最小的子域$k(\alpha)$。首先$k[\alpha]\subset k(\alpha)$,如果不存在多项式$f\in k[x]$使得$f(\alpha)=0$,则$k[\alpha]\cong k[x]$,所以包含$k[\alpha]$的最小的域就是他的商域$F(k[\alpha])$,即$k(\alpha)=F(k[\alpha])$.

反之,如果存在多项式$f\in k[x]$使得$f(\alpha)=0$,取$g$是$k[x]$中以$\alpha$为零点的次数最低的首一多项式,称为$\alpha$的极小多项式。我们需要下面这个引理。
\begin{lem}
	极小多项式不可约。如果$f$也以$\alpha$为零点,则存在$h\in k[x]$使得$f=gh$.
\end{lem}
\begin{proof}
	假设可约,设$g=g_1g_2$,其中$g_1$和$g_2$都是次数比$g$低的多项式。那么在$\alpha$处,我们有$g_1(\alpha)g_2(\alpha)=0$,所以$g_1(\alpha)$和$g_2(\alpha)$中至少有一个为零,而他们都是次数比$g$低的在$\alpha$处为零的多项式,和极小多项式的选取矛盾。

	辗转相除,我们有分解$f=gh+r$,其中$r$是比$g$次数更低的多项式或者$r=0$。如果是前者,在$\alpha$处$r(\alpha)=f(\alpha)-g(\alpha)h(\alpha)=0$,所以$r$和极小多项式的选取矛盾。
\end{proof}

通过$x\mapsto \alpha$可以定义同态$k[x]\to k[\alpha]$,他当然是满的,他的核是那些在$\alpha$为零的多项式所构成的理想,从引理可以知道,他就是极小多项式生成的极大理想\footnote{若$k$是一个域,则多项式环$k[x]$是一个主理想整环,他的极大理想被不可约多项式生成。}$(g)$,所以$k[\alpha]\cong k[x]/(g)$是一个域。因而$k[\alpha]$就是我们想要的域$k(\alpha)$,他同构于$k[x]/(g)$,其中$g$是$\alpha$的极小多项式。

所以,我们在$K$中分两种情况找到了包含$\alpha$的最小的$k$的域扩张$k(\alpha)$,这样的扩张称为单扩张,前者被称为(单)超越扩张,扩张的元素被称为超越元,后者被称为(单)代数扩张\footnote{下节我们会看到这个命名是合理的。}。

\begin{theo}
	域的单扩张总是存在的。
\end{theo}
\begin{proof}
	上面我们预设了一个比$k$大的域$K$的存在,免去了担心单扩张存在性的烦恼。现在有了上面单扩张的知识,我们也就可以直接构造单扩张来证明存在性。

	如果$\alpha$关于$k$超越,那么我们取$k(\alpha)$为$k[\alpha]$的商域$F(k[\alpha])$,这显然是一个$k$-代数,因此是一个$k$的扩张。如果$\alpha$关于$k$代数,就取$k[\alpha]$,他显然是一个$k$-代数,并且同构于域$k[x]/\mm$,其中$\mm$是$\alpha$的极小多项式生成的极大理想。
\end{proof}
\begin{pro}
	两个单扩张同构,即$k(\alpha)\cong k(\beta)$,当且仅当他们或者同为代数扩张且极小多项式相同,或者同为超越扩张。
\end{pro}
\begin{proof}
	如果$k(\alpha)$是超越扩张,而$k(\beta)$是代数扩张。前面已经知道$\dim_k(k(\alpha))=\infty$。如果$\beta$的极小多项式是$n$次的,那么$k[x]/\mm$中$\{1,x,\cdots,x^{n-1},x^n\}$是线性相关的,即$\dim_k(k(\beta))\leq n<\infty$,所以$k(\alpha)\not\cong k(\beta)$.

	现在,如果两者都是超越扩张,则$k(\alpha)\cong F(k[x])\cong k(\beta)$.如果两者都是代数扩张,则$k[x]/\mm_\alpha\cong k[x]/\mm_\beta$,即可推出$\mm_\alpha=\mm_\beta$,继而拥有相同的极小多项式。反过来,如果有相同的极小多项式,则$k(\alpha)\cong k[x]/\mm\cong k(\beta)$.
\end{proof}

现在对$k$单扩张$\alpha_1$得到了$k(\alpha_1)$,再对其单扩张$\alpha_2$就得到了$k(\alpha_1)(\alpha_2)$,他也是$k$的一个扩张,不妨将其记作$k(\alpha_1, \alpha_2)$. 如是继续,就可以得到$k(\alpha_1,\cdots,\alpha_n)$。

\begin{pro}$k(\alpha_1)(\alpha_2)\cong k(\alpha_2)(\alpha_1)$.\end{pro}
\begin{proof}
这里可以不妨假设,我们的域扩张都做成了包含的形式。因为$k(\alpha_2)(\alpha_1)$是一个域,而$\alpha_1$是他的元素,所以$k(\alpha_1)$可以看成$k(\alpha_2)(\alpha_1)$的子域,然后再对$k(\alpha_1)$做$\alpha_2$的单扩张,因为单扩张$k(\alpha_1)(\alpha_2)$是$k(\alpha_2)(\alpha_1)$中包含$\alpha_2$的最小的域,所以$k(\alpha_1)(\alpha_2)\subset k(\alpha_2)(\alpha_1)$。反过来同理。
\end{proof}
上面这个简单的命题告诉我们,有限次单扩张的顺序无关紧要,于是,对于扩张$k(\alpha_1,\cdots,\alpha_n)/k$,如果我们乐意,可以先超越扩张,然后再代数扩张。

\section{Algebraic extensions}
\begin{defi}
	设$A$和$B$是环,且$A$是$B$的子环。称$x\in B$在$A$上整,如果他是某个$A[x]$中的首一多项式的根。如果$B$中任意的元素都在$A$上整,则称$K$在$k$上整。
\end{defi}
设$k$是一个域,如果$\alpha$在$k$上整,则他在$k$上代数。反之,如果$\alpha$在$k$上代数,则存在一个多项式$f=\sum_{i=0}^na_ix^i$使得$f(\alpha)=0$,此时首一多项式$g=f/a_n$也满足$g(\alpha)$,所以$\alpha$在$k$上整。通过上面的讨论,我们发现在域上代数和在域上整等价。下面我们先证明几个关于整的结论,因为在域上面的等价性,他们也可以自然应用到域扩张上。

\begin{pro}
	设$A$和$B$是环,且$A$是$B$的子环。以下命题等价:

	\no{1} $\,\alpha$在$A$上整。

	\no{2} $A[\alpha]$是一个有限生成$A$-模。

	\no{3} $A[\alpha]$包含在$B$的一个子环$C$中,$C$是一个有限生成$A$-模。

	\no{4} 存在忠实$A[\alpha]$-模$M$,他作为$A$-模时是有限生成的。
	\label{p2:1}
\end{pro}
\begin{proof}
	\no{1} $\Rightarrow$ \no{2} :由于$\alpha$在$A$上代数,他的满足方程$\alpha^n+a_1\alpha^{n-1}+\cdots+a_n=0$,那么通过$\alpha^{n+r}=-(a_1\alpha^{n+r-1}+\cdots+a_n\alpha^r)$即可知$A[\alpha]$是一个有限生成$A$-模。

	\no{2} $\Rightarrow$ \no{3} :取$C=A[\alpha]$.

	\no{3} $\Rightarrow$ \no{4} :取$M=C$,这是一个忠实$A[\alpha]$-模,因为如果$aC=0$,由$C$有单位元,所以$a\cdot 1=a=0$.

	\no{4} $\Rightarrow$ \no{1} :因为$M$是$A[\alpha]$-模,所以$\alpha M\subset M$。因为$M$是有限生成$A$-模,设$M$被$\{x_1,\cdots,x_m\}$生成,则$\alpha M\subset M$告诉我们对任意的$i$都成立$\alpha x_i=\sum_{j=1}^m a_{ij} x_j$,其中$a_{ij}\in A$。所以
	\[
		\sum_{j=1}^m (\alpha\delta_{ij} -a_{ij})x_j=0,
	\]
	左乘$(\alpha\delta_{ij} -a_{ij})$的伴随矩阵,则$\det(\alpha\delta_{ij} -a_{ij})x_j=0$对任意的$1\leq j \leq m$都成立,也即$\det(\alpha\delta_{ij} -a_{ij})M=0$。由$M$的忠实性,$\det(\alpha\delta_{ij} -a_{ij})=0$,将行列式展开就是我们需要的首一多项式。
\end{proof}
如果$\{\alpha_1,\cdots,\alpha_n\}\subset B$都在$A$上整,那么$k[\alpha_1,\cdots,\alpha_n]$也是一个有限生成$k$-模,这只要利用$k[\alpha_1,\cdots,\alpha_n]=k[\alpha_1,\cdots,\alpha_{n-1}][\alpha_n]$经过有限次归纳即可。
\begin{pro}
	设$A$和$B$是环,且$A$是$B$的子环。则所有在$A$上整的元素构成$B$的一个子环。
\end{pro}
\begin{proof}
	如果$\alpha$和$\beta$在$A$上面整,$A[\alpha,\beta]$有限生成。因为$A[\alpha\pm\beta]\subset A[\alpha,\beta]$和$A[\alpha\beta]\subset A[\alpha,\beta]$,由上一个命题的\no{3},$\alpha\pm\beta$和$\alpha\beta$在$A$上面整。
\end{proof}
设$K/k$是一个扩张,这个命题告诉我们$K$中在$k$上代数的元素构成$K$中的子环。并且,因为如果$\alpha$代数,那么$1/\alpha$也代数,所以$K$中在$k$上代数的元素构成$K$中的子域。特别地,现在我们终于可以说明,单代数扩张是代数扩张。
\begin{pro}
	设$A\subset B\subset C$是环,且$B$在$A$上整,$C$在$B$上整,则$C$在$A$上整。这就是整的传递性。
\end{pro}
\begin{proof}
	设$x\in C$,因为$x$在$B$上整,所以存在方程$x^n+\cdots+b_{n-1}x+b_n=0$,因为$b_i\in B$都在$A$上整,所以$B'=A[b_1,\cdots,b_n]$是有一个有限生成$A$-模。由同一个首一多项式,$x$也在$B'$上整,于是$B'[x]$是一个有限生成$B'$-模,由模的有限生成的传递性,则$B'[x]$是一个有限生成$A$-模,所以$x$在$A$上整。
\end{proof}
\begin{pro}
	设环$A\subset B\subset C$,且$A$是Northerian的,$C$是有限生成$A$-代数,以及$C$或者是一个有限生成$B$-模,或者$C$在$B$上整,那么,$B$是一个有限生成$A$-代数。
	\label{p:2.4}
\end{pro}
\begin{proof}
	在题目的条件下,由Proposition \ref{p2:1},$C$是一个有限生成$B$-模与$C$在$B$上整等价。所以只对$C$是一个有限生成$B$-模的情况证明。

	令$C=A[\bar{x}_1,\cdots,\bar{x}_m]\cong A[x_1,\cdots,x_m]/\aaa$,以及令$y_1$, $y_2$, $\cdots$, $y_n$是$C$作为有限生成$B$-模的生成元,那么存在
	\begin{equation}
		\bar{x}_i=\sum_i\alpha_{ij}y_j,\quad y_iy_j=\sum_{k}\beta_{ijk}y_k,
		\label{1}
	\end{equation}
	令$B_0$是由$\alpha_{ij}\in B$和$\beta_{ijk}\in B$生成的$A$-代数,由于$A$是Northerian的,所以$B_0$是Northerian的\footnote{这是因为有限生成$A$-代数同构于$A[x_1,\cdots,x_{N_B}]/\aaa_B$,而他是Northerian的。},以及$A\subset B_0 \subset B$.

	由于$C$中的元素都是关于$\{\bar{x}_i\}$的、系数处于$A$中的多项式,那么\eqref{1}告诉我们,这个元素可以写成$\sum_i b_i y_i$,其中$b_i\in B_0$,所以$C$是一个有限生成$B_0$-模。而$B_0$是Northerian的就保证了$C$是一个Northerian的$B_0$-模。因为$B$又是$C$的子模,所以$B$是一个有限生成$B_0$-模。又$B_0$是一个有限生成$A$-代数,所以$B$是一个有限生成$A$-代数。
\end{proof}

\begin{pro}
	如果$K/k$是$L/K$都是扩张,则$L/k$是一个扩张。 特别地,如果$K/k$和$L/K$都是代数扩张,则$L/k$是代数扩张。
\end{pro}
\begin{proof}
	不妨设$k\subset K\subset L$,第一点显然。$K/k$和$L/K$都是代数扩张等价于$K$在$k$上整且$L$在$K$上整。由整的传递性,$L$在$k$上整,所以$L/k$是一个代数扩张。
\end{proof}
\begin{pro}
	设$K/k$是$L/K$都是扩张,则$[L:k]\leq[K:k]$. 特别地,如果$K/k$和$L/K$都是有限扩张且$[K:k]=m$以及$[L:K]=n$,则$L/k$是有限扩张且$[L:k]=mn$。这就是说,有限扩张的有限扩张还是有限扩张。
\end{pro}
\begin{proof}
	设$\{a_1,\cdots,a_r\}$是$K$中的任意$k$-代数无关组,而$\{b_1,\cdots,b_s\}$是$L$中的任意$K$-代数无关组,我们来证明$\{a_ib_j\}$是$k$-线性无关组。设$\alpha=\sum_{i,j}c_{ij}a_ib_j$,其中$c_{ij}\in k$,因为$\sum_i c_{ij}a_i\in K$,所以如果$\alpha=0$,那么由$\{b_1,\cdots,b_s\}$的$K$-线性无关性,所以$\sum_i c_{ij}a_i=0$,然后再应用一次$\{a_1,\cdots,a_r\}$的$k$-线性无关性,就得到了对于任意的$i$, $j$都成立$c_{ij}=0$,于是$\{a_ib_j\}$是$k$-线性无关组。由此,维度的结论显然。
\end{proof}
\begin{pro}
	有限扩张等价于有限次单代数扩张。
\end{pro}
\begin{proof}
注意到单代数扩张是有限扩张,这是因为,如果他的极小多项式为$n$次的,那么$\{1,x,\cdots,x^n\}$线性相关,而有限次有限扩张是有限扩张。

反之,设$K/k$不是代数扩张,那么存在一个元素$\alpha\in K$是超越的。因为$k\subset k(\alpha)\subset K$,所以$[K:k]\geq [k(\alpha):k]=\infty$. 如果$K/k$是代数扩张,但不是有限次单代数扩张,则对于任何的$n\in \zz^+$,一定存在一组$n$个元素的线性无关组,这和$\dim_k K$有限矛盾。所以一个有限扩张由有限次单代数扩张而成。
\end{proof}
\begin{lem}
Zariski's lemma:有限生成扩张是代数扩张。这还可以表述为,设$\mm$是$A=k[x_1,\cdots,x_n]$的一个极大理想,则$A(\mm)=A/\mm$是$k$的一个代数扩张。
\end{lem}
\begin{proof}
	归纳证明这个命题。

	$n=1$的时候是简单的,$k[x]$中任意的极大理想$\mm$都是由一个不可约多项式$f$生成的,所以$\mm=(f)$,而单扩张的知识告诉我们,$k[\bar{x}]=k[x]/(f)$是一个代数扩张,他给$k$添加上了$f$的一个根。

	对$n$个变元的情况,假设对任意的$\mm$有$A(\mm)=k[\bar{x}_1,\cdots,\bar{x}_n]$中的$\{\bar{x}_i\}$都在$k$上代数。

	对$n+1$个变元的情况,$k[\bar{x}_0,\cdots,\bar{x}_{n}]$是$k$的一个有限生成扩张,那么他可以分解成有限个单扩张,$k[\bar{x}_0,\cdots,\bar{x}_{n}]=k(\bar{x}_0)[\bar{x}_1,\cdots,\bar{x}_{n}]$,其中$k(\bar{x}_0)$是一个$k$的单扩张,根据归纳假设$\bar{x}_1,\cdots,\bar{x}_{n}$都在$k(\bar{x}_0)$上是代数的。如果$k(\bar{x}_0)$在$k$上是代数的,那么所有的$\bar{x}_i$就都是在$k$上代数的,也就是$k[\bar{x}_0,\cdots,\bar{x}_{n}]$是$k$的代数扩张了。

	假设$k(\bar{x}_0)$是超越扩张,即$k(\bar{x}_0)=F(k[\bar{x}_0])$,$k(\bar{x}_0)$是$k[\bar{x}_0]$的商域。因为$\bar{x}_i$在$k(\bar{x}_0)$上是代数的,所以存在多项式
	\[
		a_{i0}\bar{x}_i^{N_i}+a_{i1}\bar{x}_i^{N_i-1}+\cdots +a_{i,N_i+1}=0,
	\]
	其中$a_{ij}\in k(\bar{x}_0)=F(k[\bar{x}_0])$。将其通分,可以得到一个新的等式,系数属于$k[\bar{x}_0]$,为了符号上的简单,不妨直接设$a_{ij}\in k[\bar{x}_0]$.

	将等式两边乘以$a_0^{N_i-1}$后可以看到$a_{i0}\bar{x}_i$在$k[\bar{x}_0]$上是整的,实际上,对所有的$i>0$和$\bar{x}_0$都可以找到这么一个$a_{i0}$。由于在$k[\bar{x}_0]$上整的元素构成一个环,而且$k[\bar{x}_0]$是他的一个子环,特别地,所有的$a_{i0}\in k[\bar{x}_0]$以及$\bar{x}_0\in k[\bar{x}_0]$都是整的,所以我们可以说存在一个因子$a=\prod_{i>0}a_{i0}$,对每一个$\bar{x}_i$都成立$a\bar{x}_i$在$k[\bar{x}_0]$上是整的。

	现在任取一个$y\in k[\bar{x}_0,\cdots,\bar{x}_n]$,写作$y=\sum y_{i_0 \cdots i_n}\bar{x}_0^{N_{i_0}}\cdots\bar{x}_{n}^{N_{i_n}}$.
	因为在$k[\bar{x}_0]$上整的元素构成一个环,两边乘以$a^N$后可以得到$a^Ny$在$k[\bar{x}_0]$上是整的,其中$N$足够大,因为所有的求和都是有限的,所以$N$总是可以选出来的。

	我们已经证明了,随便取一个$y\in k(\bar{x}_0)$,则存在$N\in \mathbb{Z}^+$使得$a^Ny\in k(\bar{x}_0)$在$k[\bar{x}_0]$上整。由于$k[\bar{x}_0]$作为域上的多项式环是唯一分解整环
	\footnote{假设$R$是唯一分解整环,$F(R)$是他的商域,假设$x\in F(R)$在$R$上整,对于唯一分解整环有分解$x=r/s$,其中$r$和$s$互素,那么就有方程
	\[
		r^n+a_1r^{n-1}s+\cdots+a_n s^n=0,
	\]
	其中$a_i\in R$,因此$s$需要整除$r^n$,而$r$和$s$互素,所以只能有$s=\pm 1$.这就说明了$x=\pm r\in R$.},
	所以$a^Ny=f\in k[\bar{x}_0]$,或$y=f/a^N\in k[\bar{x}_0]_{a}$,
	其中$k[\bar{x}_0]_{a}$是$k[\bar{x}_0]$关于$\{1,a,a^2,\cdots\}$的分式环,而$a$和$y\in  k(\bar{x}_0)$的选取没有关系,只有$N$和$f$的选取和$y$有关系,但是不管取哪个$N$,他们都在同一个分式环里面,而分式环又真包含于商环里面,所以
	\[
		k(\bar{x}_0)\subset k[\bar{x}_0]_{a}\subsetneq k(\bar{x}_0),
	\]
	这就完成了矛盾,故$k(\bar{x}_0)$不可能是超越扩张,$k(\bar{x}_0)$是代数扩张。所以$k[\bar{x}_0,\cdots,\bar{x}_{n}]$是$k$的代数扩张。
\end{proof}
\begin{pro}
有限次单代数扩张等价于有限生成扩张。
\end{pro}
\begin{proof}
对有限次单代数扩张而成的域$k(\alpha_1,\cdots,\alpha_m)$,由$:x_i\mapsto \alpha_i$可以构造一个满的$k$-代数同态$\varphi:k[x_1,\cdots,x_m]\to k(\alpha_1,\cdots,\alpha_m)$,根据同构基本定理,$k(\alpha_1,\cdots,\alpha_m)\cong k[x_1,\cdots,x_m]/\ker \varphi$,所以$k(\alpha_1,\cdots,\alpha_m)$是一个有限生成扩张。

反之,因为$A(\mm)=k[x_1,\cdots,x_n]/\mm=k(\alpha_1,\cdots,\alpha_m)$,其中$m\leq n$,由Zariski's lemma,$A(\mm)$是一个代数扩张,即$\alpha_i$都在$k$上代数,所以$A(\mm)=k(\alpha_1)\cdots(\alpha_m)$,其中每一次单扩张都是代数的。
\end{proof}
在某些书上,有限生成扩张被定义为有限次单代数扩张,通过上面的命题,我们知道了这两个定义是等价的。最后我们再提供一个Zariski's lemma的证明,他比上面的证明要短一些,用到了Propostion \ref{p:2.4}。
\begin{proof}
设$A(\mm)=k[\alpha_1,\cdots,\alpha_n]$,如果$A(\mm)$关于$k$不是代数扩张,因为有限生成扩张一定是有限次单扩张而成的(这些单扩张是否是代数的我们还不知道),那么假设$\alpha_1,\cdots ,\alpha_r$关于$k$超越。我们可以先单扩张这些超越元,至于剩下的则关于域$B=k(\alpha_1,\cdots ,\alpha_r)$代数。

现在因为$A(\mm)$是$B$的有限扩张,根据包含关系$k\subset B\subset A(\mm)$和Propostion \ref{p:2.4},我们可以得知$B$是一个有限生成$k$-代数,设$B=k[\beta_1,\cdots ,\beta_s]$,其中每一个$\beta_i$都有着形式$f_i/g_i$,而$f_i,g_i\in k[\alpha_1,\cdots \alpha_r]$。但是,$k[\alpha_1,\cdots \alpha_r]$中有多项式$h=g_1g_2\cdots g_{s}+1$使得$h^{-1}$不能写成$\beta_1,\cdots ,\beta_s$的多项式,矛盾。
\end{proof}
\section{trans deg}
上面一节解决了有限扩张的分类问题,即有限扩张就是有限生成扩张。如果一个扩张不是有限扩张,则,要么这个扩张包含超越元,或者,他是代数扩张却不能由有限次单代数扩张而成。这节要更细致地对非有限扩张进行分类。

\begin{defi}
设$K/k$是一个扩张,一个$K$中的元素$t$被称为在$\{u_1,\cdots ,u_n\}$上关于域$k(u_1,\cdots ,u_n)$代数相关的,就是说存在一个非零多项式$f\in k[u_1,\cdots ,u_n][x]$,使得$f(t)=0$。
\end{defi}

他有如下性质:

\no{1} 因为存在$f(x)=u_i-x$,所以$u_i$是代数相关的。

\no{2} 如果$x$关于$\{u_1,\cdots ,u_n\}$相关,但是关于$\{u_1,\cdots ,u_{n-1}\}$无关,则$u_n$关于$\{u_1,\cdots ,u_{n-1},x\}$相关。

\no{3} 如果$\{v_i\}$相关于$\{w_j\}$,且$u$相关于$\{v_i\}$,则$u$相关于$\{w_j\}$.

然后可以类比线性代数中基的性质以及证明。类比线性无关,我们定义$\{u_i\}$代数无关如下:对任意的$i$,$u_i$不代数相关于其他$u_j$.

\begin{pro}
$\{u_i\}$是代数无关的当且仅当,如果多项式$f$使得$f(u_1,\cdots ,u_n)=0$,那么$f=0$.
\end{pro}

如果$\{u_i\}$代数无关,那么他们之间不存在代数方程相互联系,所以他们也被称为超越独立。

\begin{defi}
一个域$k$被称为代数闭域,就是说$k[x]$中的每个多项式都可以分解为线性因子的乘积。等价地,任何多项式都在$k$中有至少一个根。
\end{defi}

每一个域扩张都可以分解为先超越扩张,然后再代数扩张。分解不一定唯一,但是超越扩张的基数却是相同的,如果有限,那么就是次数相同。这个数就是所谓的超越次数。 从这里很容易看出$F(x_1,\cdots ,x_n)$作为单纯的超越扩张$n$次,那么他的超越次数为$n$。也可以这样定义,对于域扩张$E/F$, $E$中的极大代数无关集(超越基)的元素个数被称为超越次数。
\begin{lem}
设$E/F$和$E'/F'$是域扩张,且$\varphi:E\to E'$是域同态满足$\varphi(F)\subset F'$.现在设$f(x)\in F[x]$,若$\alpha\in E$是$f(x)$的根,则$\alpha'=\varphi(\alpha)$是$\varphi(f(x))$的根。
\end{lem}
\begin{proof}
设$f(x)=\sum_i a_i x^i$,那么$\varphi(f(x))=\sum_i \varphi(a_i) x^i$,因此
\[
	\varphi(f(\alpha'))=\sum_i \varphi(a_i) \varphi(\alpha)^i=\varphi\left(\sum_i a_i\alpha^i\right)=\varphi\left(f(\alpha)\right)=0.
\]
\end{proof}
特别地,如果$\varphi$是$E$的自同态,且在$F$上的限制为恒等映射,那么如果$\alpha$是$f(x)$的一个根,则$\varphi(\alpha)$也是$f(x)$的一个根。
\begin{defi}
设$E/F$是域扩张,称所有$E$在$F$上的限制为恒等映射的自同态构成的群为这个域扩张的Galois群,群运算为复合,记作$\mathrm{Gal}(E/F)$.
\end{defi}
一般来说,如果$E/F$是有限扩张,那么他Galois群元的个数不多于$[E:F]$,如果$|\mathrm{Gal}(E/F)|=[E:F]$,则称扩张$E/F$是Galois扩张。

\begin{theo}
Galois理论基本定理:设$E/F$是Galois扩张,

\no{1} 设$H$是$\mathrm{Gal}(E/F)$的子群,那么他和$E/F$的一个中间域$L=\{x\in E:h(x)=x,\forall h\in H\}$存在一一对应,他的逆为$L\mapsto \mathrm{Gal}(E/L)$.且
\[
[E:L]=|\mathrm{Gal}(E/L)|,\quad [L:F]=(G:\mathrm{Gal}(E/L)).
\]

\no{2} 上述对应诱导$G$的所有正规子群和$E/F$的Galois子扩张$L/F$之间的一一对应,此时
\[
	\mathrm{Gal}(L/F)\cong \mathrm{Gal}(E/F)/\mathrm{Gal}(E/L).
\]
\end{theo}

如果一个域经过任意的代数扩张之后还是其本身,那么我们就称呼这个域是代数闭域。
\begin{theo}
对任意的域$k$,在同构意义上唯一存在包含他代数闭域$\bar{k}$。$\bar{k}$也被称为$k$的代数闭包。
\end{theo}
\begin{pro}
任意多项式$f\in k[x]$都在$\bar{k}$中有根。
\end{pro}

\begin{defi}
一个域$k$被称为代数闭域,就是说$k[x]$中的每个多项式都可以分解为线性因子的乘积。等价地,任何多项式都在$k$中有至少一个根。
\end{defi}

\begin{defi}
假设$P$是$F$的一个扩张,一个$P$中的元素$v$被称为在$u_1,\cdots ,u_n$上关于域$F(u_1,\cdots ,u_n)$代数相关的,就是说存在一个非零多项式$f$,使得$f(v)=0$,这个多项式的系数是$F[u_1,\cdots ,u_n]$中的多项式。
\end{defi}

\section{Algebras and the proof of Hilbert's Nullstellensatz}

\begin{defi}
设$k$是代数闭域,如果$I$是$k[x_1,\cdots,x_n]$的一个理想,记$Z(I)$是这个理想的共同零点集,即$Z(I)$是使得理想$I$内所有多项式都为$0$的点的集合。反过来,对于一个集合$U\in k^n$,我们记$I(U)$为所有在$U$上为零的多项式所构成的理想。
\end{defi}

可以从Zariski's lemma推出Hilbert's Nullstellensatz.这就是Atiyah\&Maconald第七章的习题14:
\begin{theo}Hilbert's Nullstellensatz:设$k$是代数闭域,假如我们有一个多项式$f\in k[x_1,\cdots,x_n]$在$Z(I)$为零,那么存在一个正整数$n$使得$f^n\in I$,这就是说$f\in r(I)$,其中$r(I)$是$I$的半径,常常也记做$\sqrt{I}$.
\end{theo}
\begin{proof}
让$A=k[x_1,\cdots ,x_n]$,$I$是他的一个理想,假设$f$在$Z(I)$上为0,即$f$属于$I(Z(I))$,但$f$不属于$r(I)$。因为$r(I)$是所有包含a的素理想之交,所以$f$必然不属于某个包含$T$的素理想$p$,让$f'$是$f$在$B=A/p$中的象,再设$C=B[1/f']$,$C$是一个有限生成的$k$代数,由${1'/f',x_1'/1'\cdots ,x_n'/1'}$生成。取$m$是$C$中的一个极大理想,$C/m$是一个域,也是一个有限生成$k$代数,由Zariski's lemma,所以是一个$k$的有限扩张,但是$k$是代数闭域,所以也就是$k$。

让$t_i$是$x_i$在映射
\[
	\psi:A\xrightarrow{\pi_1}B\xrightarrow{\phi}C\xrightarrow{\pi_2}C/m\xrightarrow[\cong]{\pi_3} k
\]
的象$t_i=\psi(x_i)$,我们记$t=(t_1,\cdots ,t_n)$,由于$\psi(x_i)=t_i=x_i(t)$对任意的$x_i$都成立,所以对任意的$g$属于$A$,我们有$\psi(g)=g(t)$.
现在假设$g$是$I$的元素,那么$\pi_1(g)=0$,故$g(t)=\psi(g)=0$,这就是说$t\in Z(I)$,此外,$\phi(\pi_1(f))=f'/1'$是$C$里面的一个单位,因此$\phi(\pi_1(f))=f'/1'$不在$m$里面(否则$m=C$),那么$\psi(f)$不等于$0$,所以$f(t)=\psi(f)$不等于零,矛盾,证毕。
\end{proof}

\end{document}
