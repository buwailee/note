\chapter{Lie群}

这章处理一大类光滑流形,Lie群。Lie群是一个光滑群,即是他一方面有着群的结构,而另一方面还是一个光滑流形,其中群的运算乘法和逆是光滑的。最重要的Lie群的例子是很久以前提到过的一般线性群及其子群。

所谓的一般线性群,就是将$n\times n$实矩阵放入$\rr^{n^2}$内,那么行列式不为零的那些矩阵就构成一个光滑流形,记作$\mathrm{GL}(n,\rr)$. 更一般的,我们记$\mathrm{GL}(V)$为$V$上行列式非零的线性变换构成的群。

\section{基础}

\para 设$G$为群,单位元记做$e$,群运算记做$\mu:G\times G\to G$,如果$G$是一个光滑流形,且$\mu$是一个光滑映射,则称$G$是一个光滑Lie群。当然可以谈论不怎么光滑的Lie群,但是下面所指的Lie群都是光滑的。记$l_g$是左作用算符,即$l_g=\mu(g,\cdot)$,或者写作$l_gh=gh$,同样,右作用算符记做$r_g$,即$r_gh=hg$.显然这些都是光滑映射。

现在考虑方程$\mu(x,y)=e$,由于$(l_e)_{*e}$是一个恒同映射,所以在$e$附近,按隐函数定理,方程$\mu(x,y)=e$在$e$附近有光滑解,即$y=\nu(x)=x^{-1}$中的逆函数$\nu$在$e$附近光滑,由于$(\nu\circ l_g)(h)=h^{-1}g^{-1}=(r_{g^{-1}}\circ \nu)(h)$,所以逆函数$\nu$处处光滑。

因为Lie群有着光滑流形结构,那么我们就可以对其局部线性化,特别地,单位元附近的局部线性化就构成了Lie代数的内容。

\para 一个Lie群$G$的Lie代数$\lag$就是其单位元处的切空间。

由于$l_g$和$r_g$都是$G\to G$的光滑同胚,所以$(l_g)_*$或者$(r_g)_*$就是将光滑切矢量场映射到光滑切矢量场的双射。如果矢量场$X_x$满足$(l_a)_*X_x=X_{ax}$,则称$X$是一个左不变矢量场。对于左不变矢量场$X$而言,由于$(l_a)_*X:g\mapsto (l_a)_{*a^{-1}g}X_{a^{-1}g}=X_g$,所以$(l_a)_*X=X$.

和任意的矢量场一样,左不变矢量场$X$在$e$处诱导了$\lag$中的元素$X_e$. 反过来,设$X_e\in\lag$,我们可以构造一个左不变矢量场$X_x=(l_{x})_*X_e=(l_x)^{-1}_*X_x$,因此我们就建立了Lie代数和左不变矢量场之间的一一对应。

\para 左不变矢量场都是光滑的且完备的。

实际上,任取$f\in\calf(G)$以及$g\in G$,我们来看$(Xf)(g)=X_gf=(l_g)_{*}X_ef=X_e(f\circ l_g)$. 取一个在$e$处切矢量为$X_e$的曲线$\sigma$,则
\[
	Xf(g)=\frac{\dd }{\dd t}\bigg|_{t=0}f(g\sigma(t))=\frac{\dd }{\dd t}\bigg|_{t=0}f\circ \mu(g,\sigma(t)),
\]
是一个光滑函数,所以左不变矢量场都是光滑矢量场。至于完备性,设$\sigma$是左不变矢量场$X$的积分曲线,则$l_g\circ \sigma$也是$X$的积分曲线,这就使得我们可以把一条局部积分曲线拼到无穷远,这就是说$X$是完备的。

\para 设$X$是左不变矢量场,因为$X$是完备的,所以他诱导的单参数变换群$\{\sigma^X_t\}$在整个Lie群上是有定义的,特别地,在单位元上,我们定义$\exp(t,X):=\sigma^X_t(e)$,其中$X$是左不变矢量场,因为左不变矢量场一一对应着Lie代数,所以也可以说$X\in \lag$.同样,对固定的$X$,映射$\exp(t,X)$对改变的$t$就构成了$G$的一个子群,这被称为单参子群。由积分曲线的存在唯一性,我们也得到了单参子群与Lie代数的一一对应关系。

\pro 设$f:G\to G$是一个微分同胚,我们有$f_*[X,Y]=[f_*X,f_*Y]$,若$f=l_a$,那么我们立刻就得到了左不变矢量场的对易子也是左不变的。因此对于Lie代数来说,他继承了切矢量场的Lie括号$[\star,\star]:\lag\times \lag\to \lag$,这是一个二元线性运算,所以Lie代数确实是一个代数。

通过直接的计算,Lie代数上满足:

\no{1} $[X,Y]=-[Y,X]$,

\no{2} $[X,[Y,Z]]+[Y,[Z,X]]+[Z,[X,Y]]=0$.

第一条反对称性从矢量场的$[X,Y]=XY-YX$来看是显然的。而第二条称为Jacobi恒等式,直接计算即可验证。可以如下记忆Jacobi恒等式,$X$, $Y$和$Z$的三种右手方向构成的置换和为$0$,或者说,$[X_i,[X_j,X_k]]$中$ijk$是$123$的偶置换。

\para 前面我们定义了$\exp:\lag\times \rr\to G$,特别地,我们记$\exp(1,X)$为$\exp(X)$,这样定义的映射$\exp:\lag\to G$他被称为指数映射。

可以看到$\exp(tX)=\sigma^{tX}_1(e)=\sigma^{X}_t(e)=\exp(t,X)$,所以实际上,我们指数映射已经能够完全包含$\exp:\lag\times \rr\to G$的内容了。特别地,
\[
	\exp(tX)\exp(sX)=\exp(t,X)\exp(s,X)=\exp(t+s,X)=\exp((t+s)X).
\]
就和一般的指数表现得那样。但如果$[X,Y]\neq 0$,一般来说$\exp(X)\exp(Y)\neq \exp(X+Y)$.我们有时候也会通过$e^{X}$来记$\exp(X)$.

\lem \label{exp}我们找一个光滑函数$f:G\to \rr^n$,那么$g(t)=f(xe^{tX})$就是一个$\rr$上的光滑函数,我们来归纳证明他的$n$阶导数为
\[
	\frac{\dd^n}{\dd t^n}g(t)=(X^nf)(x e^{tX}).
\]

\proof $n=0$是显然的,$n=1$需要直接计算验证
\[
	(Xf)(x)=\left\{\frac{\dd}{\dd t}f(x e^{tX})\right\}_{t=0},
\]
这个的计算只要使用链式法则
\[
	\left\{\frac{\dd}{\dd t}f(x e^{tX})\right\}_{t=0}=f_{*x}(e^{tX})_{*0}=f_{*x}X=(Xf)(x).
\]
注意最后一个等式要依赖于$f$是矢量值的,某种程度来说这就是$T\rr^n=\rr^n$的结果。由于矩阵也可以看成在欧氏空间$\rr^{n\times n}$里,所以$f$也可以取值为矩阵。

假设$n=k$是成立的,那么因为$X^{k+1}=X\circ X^k$,
\[
	(X^{k+1}f)(x e^{tX})=(X(X^{k}f))(x e^{tX})=\left\{\frac{\dd}{\dd s}(X^kf)(x e^{(s+t)X})\right\}_{s=0}=\frac{\dd}{\dd t}(X^kf)(x e^{tX})=\frac{\dd^{k+1}}{\dd t^{k+1}}g(t).
\]\qed

\para 为了下面的讨论,我们先将实数值的1-形式拓展到矢量值的1-形式。设$V$是一个矢量空间,对于切矢量场的$V$-值函数$\omega:\Gamma(TM)\to V$被称为一个$V$-值1-形式。如果对任意的光滑切矢量场$X$,我们都有$\omega(X)$是$M$上的$V$-值光滑函数,则$\omega$被称为光滑的。

\para Lie群$G$的切丛$TG$相当简单,因为我们可以定义$(l_{a^{-1}})_*$把$T_aG$始终映射到$T_eG=\lag$来考虑,所以切丛就被平凡化了,它同构于$G\times \lag$. 与这相关的概念即Maurer-Cartan形式。设$G$是一个Lie群,他的切丛记做$TG$,映射$\omega_G:(g,v)\mapsto (l_{g^{-1}})_*v$被称为Maurer-Cartan形式。可以看到$\omega_G:\Gamma(TG)\to \lag$,因此Maurer-Cartan形式可以看做一个$\lag$值1-形式。且对于任意的$l_h^*$,我们都有
\[
	(l_h^*\omega_G)v=\omega_G((l_h)_*v)=(l_{(hg)^{-1}})_*(l_h)_*v=(l_{(g)^{-1}})_*v=\omega_G(v).
\]
所以Maurer-Cartan形式是左不变的。

\para 现在来看具体的例子,设所有$n\times n$的实(复)矩阵构成的集合为$\mathrm{M}(n,\rr)$($\mathrm{M}(n,\cc)$),其中$\det A\neq 0$的矩阵按矩阵乘法构成一个群$\mathrm{GL}(n,\rr)$($\mathrm{GL}(n,\cc)$),我们称为一般线性群,单位元是$I$。由于他可以开嵌入$\rr^{n^2}$内,所以他有自然的光滑流形结构。因此一般线性群是一个Lie群,矩阵群上的微分定义使得我们可以直接计算一般线性群的Lie代数,他的Lie代数为$\mathfrak{gl}(n,\rr)$。在一般线性群$G$上
\[
	(l_g)_{*a}v=\frac{1}{t}(l_g(a+tv)-l_g(a))
	=\frac{1}{t}(l_g(tv))=l_g(v)=gv.
\]
其中$v\in T_aG$.

所以一般线性群上面的Maurer-Cartan形式即为$\omega_G(v)=l_{g^{-1}}(v)=g^{-1}v$,其中$g$和$v$都是矩阵,矩阵乘矩阵还是矩阵,所以Lie代数$\mathfrak{gl}(n,\rr)$也是矩阵的形式。设$\dd g=(\dd x_{ij})$,那么$v$就可以写成$\dd g(v)$,因为$\dd x_{ij}(v)=v_{ij}$,则$\omega_G=g^{-1}\dd g$.

由于$\mathrm{GL}(n,\rr)$的微分结构是熟知的,我们可以直接计算其Lie代数$\mathfrak{gl}(n,\rr)$上的交换子形式。设$A\in\mathfrak{gl}(n,\rr)$而$g\in\mathrm{GL}(n,\rr)$,容易验证$A_g=gA$是左不变矢量场,因为$(l_h)_{*}A_g=(l_h)_{*}gA=hgA=A_{hg}$.

记$g=(x_{ij})$,考虑与$A=(a_{ij})$和$B=(b_{ij})$相关的左不变矢量场为
\[
A_g=\sum_{i,j,k}x_{ij}a_{jk}\partial_{ik},\quad B_g=\sum_{i,j,k}x_{ij}b_{jk}\partial_{ik},
\]
于是
\[
[A_g,B_g]=\left[\sum_{i,j,k}x_{ij}a_{jk}\partial_{ik},\sum_{i,j,k}x_{ij}b_{jk}\partial_{ik}\right]=\sum_{i,k}\left(\sum_{j}x_{ij}\sum_{r}(a_{jr}b_{rk}-b_{jr}a_{rk})\right)\partial_{ik},
\]
或者$[A_g,B_g]=(AB-BA)_g$,所以$\mathfrak{gl}(n,\rr)$上的对易子为$[A,B]=AB-BA$,其中的乘法就是矩阵乘法。

\para 对于$A\in\mathfrak{gl}(n,\rr)$,指数映射有如下级数展开
\[
	e^A=1+\sum_{n=1}^\infty \frac{A^n}{n!}=\sum_{n=0}^\infty \frac{A^n}{n!},
\]
对于任意的矩阵$A$都是收敛的。可以看到其完全类似于实数值指数函数的展开$e^x=\sum_{n=0}^\infty x^n/n!$

\proof 由Lemma \ref{exp},对一般的Lie群$G$和光滑函数$f:G\to \rr^n$,使用Taylor公式
\[
	f(xe^{tX})=\sum_{k=0}^n\frac{(tX)^{k}}{k!}f(x)+O(t^{n+1}),
\]
如果可以展开无数项,那么
\[
	f(xe^{tX})=\sum_{k=0}^\infty\frac{(tX)^{k}}{k!}f(x).
\]

现在取$f(A)=A$, $x=I$和$t=0$就可以了。至于收敛性,因为对于任意一个矩阵,$A$的范数都是有界的,那么$e^A$就被$A$的范数的级数控制,因此收敛。\qed

当然可以用其他的方式猜出这个关系,我们考虑$e^{tA}$,将其在$t=0$附近展开,有$e^{tA}=I+tA+O(t^2)$,然后对于任意的正整数$n$和固定的$t$我们有
\[
	e^{tA}=\left(e^{tA/n}\right)^n=\left(I+\frac{t}{n}A+O\left(\frac{1}{n^2}\right)\right)^n,
\]
然后令$n\to\infty$,就有$e^{tA}=\lim_{n\to\infty}\left(I+\frac{t}{n}A\right)^n$.使用二项式展开,就可以得到其级数展开
\[
	e^{tA}=1+\sum_{n=1}^\infty \frac{(tA)^n}{n!}=\sum_{n=0}^\infty \frac{(tA)^n}{n!},
\]
最后$t=1$即可。

上面的过程可能不怎么严谨,在矩阵的情况下,直接用级数定义指数映射反而可能更加简单。

\para 设$\mathrm{M}(n,k)$是域$k$上的$n\times n$矩阵全体构成的集合,则以下矩阵构成一般线性群的子群:

\no{1} 特殊线性群:$\mathrm{SL}(n,\rr)=\{A\in \mathrm{M}(n,\rr)|\det A=1\};$

\no{2} 正交群:$\mathrm{O}(n) = \{ Q \in \mathrm{M}(n,\rr) \mid Q^T Q = Q Q^T = I \};$

\no{3} 酉群:$\mathrm{U}(n) = \{ Q \in \mathrm{M}(n,\cc) \mid Q^\dag Q = Q Q^\dag = I \};$

\no{4} 特殊正交群:$\mathrm{SO}(n) =\{ Q \in \mathrm{O}(n) \mid \det Q=1 \};$

\no{5} 特殊酉群:$\mathrm{SU}(n) =\{ Q \in \mathrm{U}(n) \mid \det Q=1 \};$

我们来考虑最简单的一个群$\mathrm{SO}(2)$,他的群元素由矩阵
\[
	\begin{pmatrix}
	\cos \theta&-\sin \theta\\
	\sin \theta&\cos \theta\\
	\end{pmatrix}
\]
构成。这是一个Abel群,而且可以注意到,他同构于群$\mathrm{S}^1=\{e^{i\theta}:\theta\in\rr\}$,这是一个圆周。

\pro 对于矩阵群构成的空间,我们可以使用Heine-Borel定理断言有界闭子群是紧的。所以正交群是紧的,但是一般线性群不是紧的。

但是对于$\mathrm{GL}(n,\rr)$的子群来说,正交群就是极大的紧子群了。

\pro 一般线性群$\mathrm{GL}(n,\rr)$的子群$G$如果是紧的,那么存在可逆矩阵$P$使得$PGP^{-1}\subset \mathrm{O}(n)$.

\proof 设$A\in G$,则序列$A,A^2,\cdots,A^n,\cdots$都在$G$里面。如果$|\det A|>1$,那么$|\det A^n|=|\det A|^n$就可以任意大,和紧性对应的有界性相悖。如果$|\det A|<1$,那么由$G$的紧性,其$A^n$这个序列收敛到$G$内,但$|\det A^n|=|\det A|^n$却又小于任意的正数,所以$A$不可逆但在$G$内,则和$G$作为一般线性群的子群相悖。所以$|\det A|=1$.

现在只考虑$\det A=1$,这就是特殊线性群的情况。然后还设$\mathrm{SO}(3)$在我们的子群里。由于我们总可以将特殊线性群的元素唯一分解为$A=RP$,其中$R\in \mathrm{SO}(3)$,而$P$是正定对称矩阵,且$\det P=1$。由于正定对称一定可以对角化,适当选择基,我们使得$P$就是对角的,所以$\det P=\lambda_1\cdots\lambda_n=1$.

那么$P=R^{-1}A$,由于$R$在我们的群内,那么$R^{-1}$也是,因此$P$也是。如果存在一个$|\lambda_i|>1$,不妨假设就是$\lambda_1$,此时$P^{k}=\mathrm{diag}(\lambda_1^k,\cdots,\lambda_n^k)$中的$|\lambda_1|^k$会比任意正数大,这和紧性相悖。所以所有的$|\lambda_i|=1$,但是由于是正定的,所以$P=I$.因此$A=R\in\mathrm{SO}(3)$.

对于$\det A=-1$的情况类似,对于共轭,也是显然的。 \qed

现在我们来求这个极大的紧子群$\mathrm{O}(n)$的Lie代数,因为Maurer-Cartan形式取$\lag$值,所以我们只要整理出Maurer-Cartan形式就可以了。对恒等式$AA^T=I$求导有$\dd A A^T+A(\dd A)^T=0$,或者
\[
	A^{-1}\dd A+(A^{-1}\dd A)^T=0,
\]
$A^{-1}\dd A$就是Lie代数。$\mathrm{O}(n)$的Lie代数就是满足方程$B+B^T=0$的矩阵,换而言之,反对称矩阵。

此外,由矩阵恒等式,$AA^{-1}=I$,对其求导我们有$0=(\dd A) A^{-1}+A\dd(A^{-1})$,所以
\[
	\dd(A^{-1})=-A^{-1}(\dd A) A^{-1}.
\]
如果$A(t):\rr \to \mathrm{GL}(n,\rr)$是一个单参子群,那么有类似的
\[
	(A^{-1}(t))'=-A^{-1}(t)A'(t) A^{-1}(t).
\]
以上就是微积分里面$(1/x)'=-1/x^2$的矩阵对应。

\section{伴随表示}

这节讨论表示,我们这里感兴趣的是表示是一种矢量空间为$\lag$的表示。这需要从伴随作用开始。先说几个记号,设$l_g:h\mapsto gh$以及$r_g:h\mapsto hg$分别是左平移与右平移,他们都是群的自同构。

\para 设$\lag$是一个Lie代数,那么不难看出$\lag$的自同构群
\[\mathrm{Aut}_{\mathrm{Lie}}(\lag)=\{T\in \mathrm{GL}(\lag)\,:\,T[u,v]=[Tu,Tv],\,\forall u,\,v\in\lag\}\]
构成一个Lie群,他的Lie代数是
\[\mathfrak{gl}_{\mathrm{Lie}}(\lag)=\{T\in \mathfrak{gl}(\lag)\,:\,T[u,v]=[Tu,v]+[u,Tv],\,\forall u,\,v\in\lag\}.\]
由于$[X,*]:Y\mapsto [X,Y]\in \lag$,且根据Jacobi恒等式,我们可以得知$[X,*]\in \mathfrak{gl}_{\mathrm{Lie}}(\lag)$.

\para 设$G$是一个Lie群,他的Lie代数是$\lag$。Lie代数的伴随来自于Lie群的伴随$\mathbf{Ad}(g):h\mapsto ghg^{-1}$或者$\mathbf{Ad}(g)=r_{g^{-1}}\circ l_g=l_g\circ r_{g^{-1}}$在单位元上的导数$\mathrm{Ad}_g=\mathbf{Ad}(g)_{*e}:T_eG\to T_eG$,但注意到Lie代数$\lag$就是Lie群在单位元的切空间$T_eG$,所以$\mathrm{Ad}_g:\lag\to \lag$。因为$\mathbf{Ad}(g)$是Lie群的一个自同构,所以$\mathrm{Ad}_g:\lag\to\lag$是线性空间的同构,即$\mathrm{Ad}_g\in \mathrm{GL}(\lag)$. 利用指数函数,可以把$\mathrm{Ad}_g$和$\mathbf{Ad}(g)$之间的微分关系联系起来,即$\mathbf{Ad}(g)\exp(X)=\exp(\mathrm{Ad}_gX)$成立。

\para 一般而言,我们也可以通过左不变矢量场来描述Lie代数,这时候最好把$\mathrm{Ad}_g$理解成$(l_g)_*\circ (r_{g^{-1}})_*$,此时我们有断言:如果$X$是$G$上的左不变矢量场,那么$\mathrm{Ad}_gX$对于任意$g\in G$也是左不变矢量场。实际上,注意到左作用和右作用是可交换的,因此他们的导数也是可以交换的,那么
\[
	(l_h)_*(\mathrm{Ad}_gX)=(l_h)_*\circ (l_g)_*\circ (r_{g^{-1}})_*(X)=(r_{g^{-1}})_*(X)=(r_{g^{-1}})_*\circ (l_g)_*(X)=\mathrm{Ad}_gX.
\]

此外,$\mathrm{Ad}_g$还是一个Lie代数同态,为此只要检验$\mathrm{Ad}_g([X,Y])=[\mathrm{Ad}_gX,\mathrm{Ad}_gY]$,其中$X$和$Y$都是左不变矢量场。由于$\mathrm{Ad}_g=(l_g)_*\circ (r_{g^{-1}})_*$,且$l_g$与$r_{g^{-1}}$作为Lie群的自同构,有$(l_g)_*[X,Y]=[(l_g)_*X,(l_g)_*Y]$和$r_{g^{-1}}[X,Y]=[r_{g^{-1}}X,r_{g^{-1}}Y]$成立,于是$\mathrm{Ad}_g([X,Y])=[\mathrm{Ad}_gX,\mathrm{Ad}_gY]$自然得证。

综上,$\mathrm{Ad}_g$是矢量空间同构,同时也是Lie代数同态,故$\mathrm{Ad}_g$是Lie代数$\lag$的一个自同构,或者说$\mathrm{Ad}_g\in \mathrm{Aut}_{\mathrm{Lie}}(\lag)$. 于是构造映射$\mathrm{Ad}:g\mapsto \mathrm{Ad}_g$. 不难检验$\mathrm{Ad}:G\to \mathrm{Aut}(\lag)$是一个Lie群同态,称为Lie群的伴随表示。其中,同态来自于$\mathrm{Ad}_g\circ \mathrm{Ad}_h=(l_g)_*\circ (r_{g^{-1}})_*\circ (l_h)_*\circ (r_{h^{-1}})_*=(l_g)_*\circ (l_h)_*\circ (r_{g^{-1}})_*\circ (r_{h^{-1}})_*=(l_{gh})_*\circ (r_{(hg)^{-1}})_*=\mathrm{Ad}_{gh}$.

% 第三个设$v\in T_hG$,因此$(r_g)_*v\in T_{hg}G$,于是
% \[
% 	(r_g)^*\omega_G(v)=\omega_G((r_g)_*v)=(l_{{hg}^{-1}})_*(r_g)_*
% 	=(l_{{g}^{-1}})_*(r_g)_*(l_{{h}^{-1}})_*v=\mathrm{Ad}(g^{-1})\omega_G.
% \]

\pro 令$G$的Lie代数为$\lag$,则$\mathrm{Ad}:G\to \mathrm{Aut}_{\mathrm{Lie}}(\lag)$在$e$的导数$\ad=\mathrm{Ad}_{*e}:\lag\to \mathfrak{gl}_{\mathrm{Lie}}(\lag)$满足$\ad(X)Y=[X,Y]$.

\proof 
	因为$\mathrm{Ad}$是$G$上矢量值的函数,证明就是很直接地要去计算
	\[
		\ad(X)Y=\left.\frac{\dd}{\dd t}\right|_{t=0}\mathrm{Ad}_{\exp(tX)}Y,
	\]
	找一个$G$上的光滑函数$f$,我们计算其单位元处的导数,注意到
	\[
		Yf=\left.\frac{\dd}{\dd u}\right|_{u=0}f(\exp(uY)),
	\]
	以及$(\mathrm{Ad}_{g}Y)f=Y(f\circ l_g\circ r_{g^{-1}})$,所以
	\[
		\ad(X)Yf=\left.\frac{\dd}{\dd t}\right|_{t=0}(\mathrm{Ad}_{\exp(tX)}Y)f=\left.\frac{\dd}{\dd t}\frac{\dd}{\dd u}\right|_{u=t=0}f\bigl(\exp(tX)\exp(uY)\exp(-tX)\bigr),
	\]
	注意到对$t$求导的时候,可以利用多元实函数$F(t_1,t_2)$的求导等式
	\[
		\left.\frac{\dd}{\dd t}\right|_{t=0}F(t,t)=\frac{\partial F}{\partial t_1}(0,0)+\frac{\partial F}{\partial t_2}(0,0),
	\]
	所以我们得到了
	\[
	\begin{split}
		&\left.\frac{\dd}{\dd t}\frac{\dd}{\dd u}\right|_{u=t=0}f\bigl(\exp(tX)\exp(uY)\exp(-tX)\bigr)\\
		=&\left.\frac{\dd}{\dd t_1}\frac{\dd}{\dd u}\right|_{u=t_1=0}f\bigl(\exp(t_1X)\exp(uY)\bigr)-\left.\frac{\dd}{\dd t_2}\frac{\dd}{\dd u}\right|_{u=t_2=0}f\bigl(\exp(uY)\exp(t_2X)\bigr)\\
	=&(XY-YX)f(e),
	\end{split}
	\]
	所以$\ad(X)Y=[X,Y]$.
\qed

% Lie代数上的双线性映射$f$如果满足$f([X,Y])=[f(X),Y]+[X,f(Y)]$,则$f$被称为一个导子。适当改写Jacobi恒等式,我们可以得到$[X,[Y,Z]]=[[X,Y],Z]+[Y,[X,Z]]$,所以
% \[
% 	\ad(X)([Y,Z])=[\ad(X)Y,Z]+[Y,\ad(X)Z]
% \]
% 告诉我们$\ad(X)$是Lie代数上一个自然的导子。而又从上面的命题,多少可以知晓导子的名字实际上是来自于导数的,因为$\ad(X)$是伴随表示的导数。

\para 对于一般线性群,前面已经计算过了$(l_g)_*=l_g$,那么同样$(r_g)_*=r_g$,所以
\[
	\mathrm{Ad}_g=(l_g)_*(r_{g^{-1}})_*=l_gr_{g^{-1}}.
\]
那么
\[
	\mathrm{Ad}_g(v)=(l_g)_*(r_{g^{-1}})_*v=l_gr_{g^{-1}}v=gvg^{-1}.
\]
我们现在求他的Lie代数,考虑$u,v\in \lag$,我们令$u(t)$是一个以$u$为初速度的单参子群,那么我们有
\[
	\frac{\dd}{\dd t}\bigl(\mathrm{Ad}_{u(t)}(v)\bigr)=u'(t)vu^{-1}(t)+u(t)v(u^{-1}(t))'=u'(t)vu^{-1}(t)-u(t)vu^{-1}(t)u'(t)u^{-1}(t).
\]
然后令$t=0$,那么$u(0)=u^{-1}(0)=I$,而$u'(0)=u$,那么就得到了单位元处的切矢量,也就是Lie代数$\ad(u)v=uv-vu=[u,v]$.

类似的手段譬如对$T(t)[u,v]=[T(t)u,T(t)v]$求个导,然后在$t=0$处的值为
\[
	T'(0)[u,v]=[T'(0)u,T(0)v]+[T(0)u,T'(0)v],
\]
注意到$T(0)$是恒等变换,而$T'(0)$就是我们需要的Lie代数$B$,他需要满足的关系就是
\[
	B[u,v]=[Bu,v]+[u,Bv],
\]
其显然是Lie代数上面的一个导子。

\section{Lie群同态}

Lie群很容易构成一个范畴,两个Lie群的态射只要定义成光滑的群同态就好了。比如对一般线性群而言,行列式就是一个显然的Lie群同态。这样子做之后,Lie群既是光滑流形范畴的子范畴,也是群范畴的子范畴。

Lie群因为本身就是一个群,所以他的很多整体性质都来自于单位元附近。对于拓扑性质,由于左乘是一个同胚,所以如果$U$是一个开集,那么$gU=\{gh:h\in U\}$也是一个开集。因为Lie群也是拓扑群,所以附录中关于拓扑群的结论都可以用到这里。

\para 设$\varphi:G\to H$是一个Lie群同态,因为$\varphi\circ l_g=l_{\varphi(g)}\circ \varphi$,所以在点$e$的切映射满足$\varphi_{*g}\circ (l_g)_{*e}=(l_{\varphi(g)})_{*e}\circ \varphi_{*e}$,因为左移是同胚,所以$\rank_e \varphi=\rank_g \varphi$.

上面意味着Lie群同态的秩都是常数,此时我们可以对$\ker \varphi=\varphi^{-1}(e)$利用Theorem \ref{ranktheo},因为他一定非空,所以$\ker \varphi$是$G$的一个正则子流形,维度等于$\dim G-\rank \varphi$.

同样因为Lie群同态的秩都是常数,所以单的Lie群同态必定为浸入,Lie群同构必为同胚,其逆也光滑。

\para 设$\varphi:H\to G$是Lie群同态,如果$\varphi$是单的,则$H$被称为$G$的Lie子群,如果$\varphi$是嵌入,则$H$被称为$G$的闭Lie子群。显然,上面谈论的Lie群同态的核是一个闭Lie子群。

因为单的Lie群同态必定为浸入,所以如果$\varphi$是单的,则$H$是$G$的一个子流形。如果$H$是$G$的一个正则子流形,我们称$H$是一个闭Lie子群,是因为在拓扑上,$\varphi(H)$真的是$G$的一个闭子集。

\pro 设$H\subset G$的一个闭Lie子群,则$H$是$G$的闭子集。

\para 如果$\varphi:G\to H$是Lie群同态,且$\varphi_{*e}$是满射,由于Lie群同态常秩,所以$\varphi_{*g}$在每一个$g$上都是满射。因此$\varphi$是一个淹没,是一个开映射。

特别地,如果$H$还是连通的,那么因为$\varphi$是一个开映射,所以存在一个$e_H$的邻域$V\subset H$使得$V\subset \varphi(H)$,那么$H=\bigcup_{n\geq 1}V^n\subset \varphi(G)$,因此$\varphi$就是一个满射。所以,对Lie群而言,局部满就可以说明整体满。

反过来,如果$\varphi:G\to H$是Lie群同态是一个满Lie群同态,由于$\varphi$常秩,由Proposition \eqref{pro:surrank},$\varphi$是一个淹没。

因此,对Lie群同态$\varphi$来说,满等价于淹没等价于在原点$\varphi_{*e}$是满射。

\para \label{covering_space}设$M$是底空间为$B$的一个纤维丛,他的纤维被赋予离散拓扑,此时,$M$被称为$B$的一个覆叠空间,投影$\pi$现在被称为覆叠映射。一般来说,我们要求覆叠空间是连通的,如果覆叠空间还是单连通的,则被称为万有覆叠空间。关于覆叠空间的更多内容,可以参看附录。

很容易看到,Lie群的万有覆叠空间有很自然的Lie群结构。附录中已经对拓扑群证明了这一点,有必要的话,利用一些附录中的技术性引理,我们就得到了光滑性,当然也就得到了Lie群的结论。

\para 一个连通Lie群$G$,记$G'$为其万有覆叠空间,覆叠映射为$\pi:G'\to G$.对于任意的选择$e'\in \pi^{-1}(e)$,总有唯一的$G'$上的Lie群结构使得$e'$是单位元且$\pi$是群同态。

\section{Lie群和Lie代数的联系}

下面陈述Lie群和Lie代数的联系。

\para 如果$H$是Lie群$G$的子群,且$H$本身是一个拓扑群,则称呼$H$是$G$的Lie子群。

\para 设$H$是$G$的Lie子群,那么$\mathfrak{h}$是$\lag$的Lie子代数。

线性子空间部分是显然的,而Lie括号的封闭性也基本是显然的,故而整个结论是显然的。考虑了子代数,现在来看看两个Lie群和Lie代数的联系。

\pro 反过来,设$\lag$是$G$的Lie代数,而$\mathfrak{h}$是$\lag$的Lie子代数,则在$G$中存在一个Lie子群$H$使得$H$的Lie代数就是$\mathfrak{h}$.

\proof 用左不变矢量场来表示$\mathfrak{h}$,那么它是一个对合分布,所以存在一个极大积分子流形经过点$e$,我们记为$H$. 下面只要证明这是一个连通子群即可,即任取$h\in H$都有$h^{-1}H=H$,这样$H^{-1}H=H$就告诉我们他是一个连通子群。由于$\mathfrak{h}$是左不变的,所以$h^{-1}H$也是它的一个积分子流形,由定义,$h^{-1}H$经过点$e$,由极大性,$h^{-1}H=H$.\qed


% 下面这个定理来自Ado,就像流形中的Whitney嵌入定理一样,告诉我们,对于有限维Lie代数,其实我们只需要考虑$\mathfrak{gl}(n,\rr)$的子代数就可以了。(记得Arnold用Whitney嵌入定理来说“根本没有抽象的流形”。)

% \theo 任何$\rr$上的有限维Lie代数总可以同构于$\mathfrak{gl}(n,\rr)$的一个子代数,$n$为一个足够大的整数。

\para 设$\varphi:G_1\to G_2$是Lie群间的群同态(同构),则其诱导了Lie代数间的同态(同构)$\varphi_{*e}:\lag_1\to \lag_2$.

上面对于微分同胚我们已经证明了$\varphi_{*e}[u,v]=[\varphi_{*e}u,\varphi_{*e}v]$,但Lie群之间的群同态使得我们可以做得更好。

令$u$, $v\in\lag_1$,那么我们拓展到$G_1$上的对应的左不变矢量场$X$, $Y$上。因为$\varphi$是一个群同态,故$\varphi\circ l_g=l_{\varphi(g)}\circ \varphi$.使用这个关系,
\[
	\varphi_{*}(X_g)=\varphi_{*}((l_g)_*u)=(l_{\varphi(g)})_*(\varphi_{*}(u))=(\varphi_{*}X)_{\varphi(g)}.
\]
设$f$是$G_2$上任意的可微实函数,那么
\[
\varphi_{*}(X_g)f=X_g(f\circ \varphi),
\]
或者使用$\varphi_{*g}(X_g)=(\varphi_{*}X)_{\varphi(g)}$写作
\[
(\varphi_{*}(X)f)\circ \varphi=X(f\circ \varphi),
\]
同理有$(\varphi_{*}(Y)f)\circ \varphi=Y(f\circ \varphi)$.所以
\[
	\begin{split}
		([\varphi_*(X),\varphi_*(Y)]f)\circ \varphi&=((\varphi_*(X)\circ\varphi_*(Y)-\varphi_*(Y)\circ\varphi_*(X))f)\circ \varphi\\
		&=X((\varphi_*(Y)f)\circ \varphi)-Y((\varphi_*(X)f)\circ \varphi)\\
		&=X(Y(f\circ \varphi))-Y(X(f\circ \varphi))\\
		&=[X,Y](f\circ \varphi).
	\end{split}
\]
这就是说对任意的$g$都有
\[
	\varphi_{*g}([X,Y]_g)=[\varphi_{*g}(X_g),\varphi_{*g}(Y_g)]
\]
特别地$g=e$,则
\[
	\varphi_{*e}[u,v]=[\varphi_{*e}u,\varphi_{*e}v].
\]

\para 由于Lie群的表示即一个Lie群同态$G\to \mathrm{GL}(V)$,那么自然我们就有了Lie代数的表示的概念。令$\lag$是一个Lie代数,那么他的表示$(\phi,V)$就是一个映射$\phi:\lag\to\mathfrak{gl}(V)$.

一个Lie群的表示$(\pi,V)$引出他的Lie上面的一个表示如下:
\[
	\pi_*(X)=\left.\frac{\dd}{\dd t}\pi\bigl(\exp(tX)\bigr)\right|_{t=0},
\]
或者写作$\pi(\exp(tX))=\exp(\pi_*(X))$.

对两个Lie群的表示的直积求导,就得到了Lie代数表示的直积形式应该满足:
\[
	\left.\frac{\dd}{\dd t}(\pi(g)\otimes \eta(g))\right|_{t=0}=\pi_*(g)\otimes I+I\otimes \eta_*(g)=(\pi_*\otimes I+I\otimes \eta_*)(g),
\]
我们就将其作为Lie代数表示直积的定义。类似的还有直和$(\pi\oplus \eta)_*=\pi_*\oplus \eta_*$,这可以看成Lie代数表示的直和的定义。

\para 现有Lie代数$\lag$的表示$(\pi,V)$,如果存在$V$的一个子空间$W$使得$\pi(X)(W)\subset W$对任意$X\in \lag$成立,则$(\pi,W)$构成了$\lag$的一个新的表示,被称为$(\pi,V)$的子表示。如果表示$(\pi,V)$没有非平凡的子表示,则称呼该表示为不可约表示。如果一个表示,能分解成诸不可约表示的直和,则称这个表示是完全分解的。

\pro 因为Lie群表示可以诱导一个Lie代数表示,在$\exp:\lag\to G$是满射的假设下,我们断言:$\pi$是不可约的当且仅当$\pi_*$是不可约的;$\pi$是完全可约的当且仅当$\pi_*$是完全可约的;两个$\pi$和$\eta$是等价的当且仅当$\pi_*$和$\eta_*$是等价的。

上面基本都是从Lie群去得到Lie代数,而且是唯一确定的。那么反过来,我们是否可以从Lie代数得到Lie群,如果可以得到,又能确定到哪种程度?这些内容是下面两个定理的内容。

\para 设$G$是Lie群,而$G'$是他的万有覆叠空间,则$G$和$G'$的Lie代数同构。

从上一节,我们知道$G'$有着自然的Lie群结构,而且从万有覆叠空间的定义来看,$G'$的单位元$e'$的附近同胚于$G$的单位元$e$的附近,那么因为Lie代数是单位元处的切空间,则他们俩的Lie代数自然是同构的。

\pro 我们有一个任意的有限维Lie代数$\lag$,那么在同构意义下有唯一的单连通Lie群$G$,他的Lie代数就是$\lag$.

如果两个Lie群有着相同的有限维Lie代数,我们虽然不能判断他们是同构的,但是我们可以找到他们同构的万有覆叠空间,一个单连通Lie群。

\para 本节的最后,讲一下Lie群的结构方程,我们将Maurer-Cartan形式求一下外微分,可以得到
\[
\dd \omega_G(X,Y)=X(\omega_G(Y))-Y(\omega_G(X))-\omega_G([X,Y]).
\]
假设,$X$和$Y$是左不变矢量场,则$\omega_G(Y)$和$\omega_G(X)$都是常数,那么得到
\[
\dd \omega_G(X,Y)+\omega_G([X,Y])=0.
\]
因为$[X,Y]$在点$e$的值是在$\lag$里面的,所以这一项也等于$\omega_G([X,Y])=[\omega_G(X),\omega_G(Y)]$,最后就得到了Lie群的结构方程
\[
\dd \omega_G(X,Y)+[\omega_G(X),\omega_G(Y)]=0.
\]

由于这个方程是微分过的结果,可以看成是微分方程,他确定了Lie群的局部结构。虽然我们的证明是选取了两个左不变矢量场,但是每一个矢量场在在局部都可以变成左不变矢量场的限制,所以我们的方程总是成立的。如果群是Abel群,那么方程的第二项为0,即$\dd \omega_G=0$.

\section{Lie群的作用}

% \section{Tori}

% \para 交换Lie群的Lie代数是交换的,即对于任意的$X$, $Y\in \lag$都成立$[X,Y]=0$.

% \para 如果$G$是Lie群,而且对于$X$, $Y\in \lag$,我们有$\exp(Y)\exp(X)=\exp(X)\exp(Y)$. 这来自于$[X,Y]=0$当且仅当他们的流$\sigma_t^X$和$\sigma_s^Y$可交换。更强的结论来自于计算$\exp(tX)\exp(tY)$在$t=0$的时候的导数,计算出来为$X+Y$,所以$\exp(X+Y)=\exp(X)\exp(Y)$.

% 所以,对于交换Lie群的Lie代数,$\exp$是一个交换群同态。

% \para 所谓的(紧)环群,就是一个紧致连通交换Lie群。一个显然的环群,就是$\mathrm{SO}(2)$,他同构于$\rr/\zz$,当然,这些圆周的直积$(\rr/\zz)^r$也是环群。

% \para 对于环群,维数直接刻画了他的全部结构,因为设$r=\dim T$,则$T\cong (\rr/\zz)^r$. 证明思路为:$\exp$是他的Lie代数(作为加法群就是$\rr^r$)到他的交换群同态,利用同态基本定理,我们考察$\Lambda=\ker(\exp)$,由于$\exp$是局部同胚,所以$\Lambda$是$\rr^r$的一个离散加法子群,所以$T\cong (\rr/\zz)^r$.

% \para 进而,有Surch引理和经典Fourier分析,我们可以知道$(\rr/\zz)^r$的不可约(非平凡)复表示都是一维的,而且都只具有形式$(x_1,\cdots,x_r)\mapsto \exp(2\pi i \sum_j k_jx_j)$,其中$(k_1,\cdots,k_r)\in \zz^r$.

% 对于实表示,利用$\mathrm{U}(1)$到$\mathrm{SO}(2)$的同构,我们就可以知道不可约(非平凡)实表示都是二维的且具有旋转矩阵的形式,旋转角是$2\pi \sum_i k_ix_i$.

% \para 设$G$是一个紧Lie群,那么他的交换闭子群的含单位元的连通分支是一个环群。Lie群的闭子群是Lie子群。所以这个环群被称为子环群。

% \section{Matrix Group}

% 这一节限制在矩阵群上面,这节将给出很多可计算的例子,先来求一些矩阵群的Lie代数。

% \theo 任意的矩阵$M\in\mathrm{GL}(n,\cc)$都可以写作$e^X$的形式,其中$X\in\mathrm{M}(n,\cc)$。

% 这个直接从矩阵幂那一套证明并不那么方便,但是直接从Lie代数作为单位元附近的Lie群与单参子群的对应关系,这个结论就是比清晰的了。这个定理也可以用来说明,一般线性群的Lie代数就是整个方阵构成的集合。

% \para 利用$\mathrm{Ad}_D(A)=DAD^{-1}$,我们有$\exp(DAD^{-1})=D\exp(A)D^{-1}$.这当然也可以直接用矩阵指数映射的展开来证明。

% \pro 指数映射联系了矩阵的迹与行列式,$\det(\exp(A))=\exp(\tr(A))$.

% \proof 将矩阵分解为幂零的和可对角化的两个矩阵的乘积$A=SN$,因为$S$和$N$可交换,所以
% \[
% 	\det \left(\exp(A)\right)=\det\left(\exp(S)\right)\det\left(\exp(N)\right).
% \]
% 如果可以对角化,那么直接对角化为
% \[
% 	\exp(S)=\exp(D\Lambda D^{-1})=D\exp(\Lambda) D^{-1},
% \]
% 而$\exp(\Lambda)$是可以直接计算的,即$\mathrm{diag}\left(\exp(\lambda_1),\dots,\exp(\lambda_n)\right)$.那么
% \[
% 	\det \left(\exp(S)\right)=\det \left(D\exp(\Lambda) D^{-1}\right)=\det \left(\exp(\Lambda)\right)=\exp\left(\sum_i \lambda_i\right)=\exp{\tr(\Lambda)}=\exp(\tr(S))
% \]
% 对于幂零矩阵来说,容易验证$\det\circ\exp(N)=1$,最后就可以得到任意矩阵都有
% \[
% 	\det(\exp(A))=\exp(\tr(A)).
% \]
% \qed

% 有了上面的一些结论,我们来求特殊线性群$\mathrm{SL}(n,\cc)$的Lie代数$\mathfrak{sl}(n,\cc)$。如果$\det e^A=1$,那么$e^A\in \mathfrak{sl}(n,\cc)$,而$A \in \mathfrak{sl}(n,\cc)$,因为$e^{\tr(A)}=\det e^A=1$,
% 所以$\tr(A)=0$.这就是说$\mathfrak{sl}(n,\cc)$就是那么迹为0的矩阵的集合。

% 作为$\mathrm{SL}(n,\cc)$的子群,$\mathrm{SU}(n)$的元素满足$AA^\dag=I$,那么
% \[
% 	\dd A A^\dag+A\dd A^\dag=0,
% \]
% 或者整理成Maurer-Cartan形式的样子
% \[
% 	A^{-1}\dd A +(A^{-1}\dd A)^\dag=0,
% \]
% 这就是说,$\mathfrak{su}(n)$是由满足$B+B^\dag=0$的矩阵零迹矩阵$B$构成的集合。当然,这也可以用矩阵幂来做,我们略去了。

% 从矩阵幂的形式
% \[
% 	e^A=\sum_{i=0}^n\frac{A^n}{n!},
% \]
% 我们当然会去遐想,是否其他函数也有这样的幂函数展开?其中最有趣的展开莫过于$\log$了,因为他是$\exp$的反函数。就这样,我们定义
% \[
% 	\log A=\sum_{m=1}^\infty (-1)^{m+1}\frac{(A-I)^m}{m}.
% \]
% 当$\|A-I\|<1$的时候,这个幂级数显然是收敛且连续的。我们也确实可以证明在收敛的时候他和$\exp$是反函数,适当对角化(不能的话就用可对角的矩阵序列趋近)之后就可以直接计算验证。

% 前面说过$e^{X+Y}$在$X$, $Y$不对易的时候是一般不等于$e^Xe^Y$,这里我们举一个例子:
% \[
% X=\begin{pmatrix}
% 1&\\
% &2\\
% \end{pmatrix},
% \quad
% Y=\begin{pmatrix}
% &1\\
% 2&\\
% \end{pmatrix},
% \]
% 直接计算就知道$[X,Y]\neq 0$.然后$e^X$是容易计算的,因为他是对角的
% \[
% e^X=\begin{pmatrix}
% e&\\
% &e^2\\
% \end{pmatrix},
% \]
% 后面的$Y$对角化之后也是容易的,
% \[
% Y=
% \begin{pmatrix}
%  -1/\sqrt{2} & 1/\sqrt{2} \\
%  1 & 1 \\
% \end{pmatrix}
% \begin{pmatrix}
%  -\sqrt{2} &  \\
%   & \sqrt{2} \\
% \end{pmatrix}
% \begin{pmatrix}
%  -1/\sqrt{2} & 1/\sqrt{2} \\
%  1 & 1 \\
% \end{pmatrix}^{-1}
% \]
% 所以
% \[
% e^Y=
% \begin{pmatrix}
%  -1/\sqrt{2} & 1/\sqrt{2} \\
%  1 & 1 \\
% \end{pmatrix}
% \begin{pmatrix}
%  \exp(-\sqrt{2}) &  \\
%   & \exp(\sqrt{2}) \\
% \end{pmatrix}
% \begin{pmatrix}
%  -1/\sqrt{2} & 1/\sqrt{2} \\
%  1 & 1 \\
% \end{pmatrix}^{-1}
% \]
% 那么
% \[
% e^Xe^Y=
% \begin{pmatrix}
% e&\\
% &e^2\\
% \end{pmatrix}
% \begin{pmatrix}
% \cosh \left(\sqrt{2}\right)&\sinh \left(\sqrt{2}\right)/\sqrt{2}\\
% \sqrt{2} \sinh \left(\sqrt{2}\right)&\cosh \left(\sqrt{2}\right)\\
% \end{pmatrix}.
% \]
% $e^{X+Y}$的计算不再重复了
% \[
% e^{X+Y}=\begin{pmatrix}
%  2/3+e^3/3 & e^3/3-1/3 \\
%  2 e^3/3-2/3 & 2 e^3/3-2/3 \\
% \end{pmatrix}.
% \]
% 所以$e^{X+Y}\neq e^Xe^Y$.

% \pro
% 令$X$, $Y$都是$n\times n$的矩阵,则
% \[
% 	e^{X+Y}=\lim_{m\to\infty}\left(e^{X/m}e^{Y/m}\right)^m.
% \]

% 这证明挺简单的,所以略去了。下面这个公式更加复杂,因此证明也很复杂,以至于都需要一个名字来标记这个公式了。证明也略去了。

% \theo Campbell-Baker-Hausdorff公式:
% \[
% \log\left(e^Xe^Y\right)=X+\int_0^1 \varphi\left(e^{\ad(X)}e^{t\ad(Y)}\right)(Y)\dd t,
% \]
% 其中
% \[
% \varphi(z)=\frac{z\log z}{z-1}.
% \]

% 幂级数展开$\varphi(z)$我们有
% \[
% 	\varphi(z)=1+\sum_{n=2}^\infty (-1)^n\left(\frac{1}{n-1}-\frac{1}{n}\right)(z-1)^{n-1}.
% \]
% 带入上面的公式,就可以得到漂亮的展开
% \[
% \log\left(e^Xe^Y\right)=X+Y+\frac{1}{2}[A,B]+\frac{1}{12}[A,[A,B]]-\frac{1}{12}[B,[A,B]]+\cdots.
% \]

% 级数展开不是上面公式的重点,重点是展开后全是对易子$[\star,\star]$的形式,直接来自于积分里面的$\ad$.所以使用这个公式可以证明矩阵Lie群和Lie代数关系中的映射关系,因为Lie代数同态是可以分配进对易子的但是对于一般的$X$和$Y$的线性组合却是不可以的。

% 我们现在来对Lie代数进行展开,设Lie代数的基为$\{a_i\}$,那么
% \[
% 	[x,y]=\sum_{i,j}x_iy_j[a_i,a_j].
% \]
% 但是因为交换子是封闭的,所以
% \[
% 	[a_i,a_j]=\sum_k c_{kij}a_k.
% \]
% 然后考虑伴随表示的展开
% \[
% 	\ad(a_i)(a_j)=\sum_k \ad(a_i)_{kj}a_k.
% \]
% 对于一般线性群的Lie代数来说$\ad(a_i)(a_j)=[a_i,a_j]$,这就是说
% \[
% 	[a_i,a_j]=\sum_k \ad(a_i)_{kj}a_k.
% \]
% 使用线性性,我们最后得到
% \[
% 	[x,a_i]=\sum_k \ad(x)_{ki}a_k.
% \]
% 这样,我们就可以求得$\ad(x)$的矩阵。

% 现在来考虑$\mathfrak{so}(3)$的Lie代数,因为$\mathfrak{so}(3)$是那些$3\times 3$的反对称矩阵构成的集合,我们取下面三个矩阵作为基
% \[
% 	\eta_1=
% 		\begin{pmatrix}
% 			0&0&0\\
% 			0&0&-1\\
% 			0&1&0\\
% 		\end{pmatrix},\quad
% 	\eta_2=
% 		\begin{pmatrix}
% 			0&0&1\\
% 			0&0&0\\
% 			-1&0&0\\
% 		\end{pmatrix},\quad
% 	\eta_3=
% 		\begin{pmatrix}
% 			0&-1&0\\
% 			1&0&0\\
% 			0&0&0\\
% 		\end{pmatrix},
% \]
% 然后可以计算对易子如下
% \[
% 	[\eta_1,\eta_2]=\eta_3,\quad [\eta_1,\eta_3]=-\eta_2,\quad [\eta_2,\eta_3]=\eta_1.
% \]

% 如果建立映射$\eta_i\mapsto e_i$,其中$e_i$就是$\rr^3$的标准基,而映射将$[\star,\star]$映射为叉乘,那么这就是一个Lie代数同构。

% 那么我们也可以计算得伴随表示为
% \[
% 	\ad(\eta_1)=
% 		\begin{pmatrix}
% 			0&0&0\\
% 			0&0&-1\\
% 			0&1&0\\
% 		\end{pmatrix},\quad
% 	\ad(\eta_2)=
% 		\begin{pmatrix}
% 			0&0&1\\
% 			0&0&0\\
% 			-1&0&0\\
% 		\end{pmatrix},\quad
% 	\ad(\eta_3)=
% 		\begin{pmatrix}
% 			0&-1&0\\
% 			1&0&0\\
% 			0&0&0\\
% 		\end{pmatrix}.
% \]
% 有趣的是,在这组基的选取下$\ad(\eta_i)=\eta_i$.

% 下面来看看$\mathfrak{sl}(2,\cc)$,他是所有二阶零迹矩阵构成的群,我们设他的基为
% \[
% h=\begin{pmatrix}
% 	-1&0\\
% 	0&1\\
% \end{pmatrix},\quad
% e=\begin{pmatrix}
% 	0&1\\
% 	0&0\\
% \end{pmatrix},\quad
% f=\begin{pmatrix}
% 	0&0\\
% 	1&0\\
% \end{pmatrix},
% \]
% 那么
% \[
% [h,e]=2e,\quad[h,f]=-2f,\quad[e,f]=h.
% \]
% 所以
% \[
% 	\ad(h)=
% 		\begin{pmatrix}
% 			0&2&0\\
% 			0&0&0\\
% 			0&0&-2\\
% 		\end{pmatrix},\quad
% 	\ad(e)=
% 		\begin{pmatrix}
% 			0&0&1\\
% 			-2&0&0\\
% 			0&0&0\\
% 		\end{pmatrix},\quad
% 	\ad(f)=
% 		\begin{pmatrix}
% 			0&-1&0\\
% 			0&0&0\\
% 			2&0&0\\
% 		\end{pmatrix}.
% \]

% \para 定义Killing形式为$(x,y)_K$如下
% \[
% 	(x,y)_K=\tr(\ad(x)\ad(y)),
% \]
% 其中$x,y$是任意的Lie代数元素。

% 在选取基的形式下,我们可以得到矩阵$K_{ij}=(a_i,a_j)_K$.

% \begin{pro}
% Killing形式是双线性的,且

% \no{1} 对于任意的Lie代数自同构$\varphi$我们对任意的$x,y$都有$(\varphi x,\varphi y)_K=(x,y)_K$,

% \no{2} 对于任意的$x,y,z$有$([x,y],z)_K=(x,[y,z])_K$,

% \no{3} 如果$\mathfrak{h}$是$\lag$的理想,那么在$\mathfrak{h}$上的Killing形式$(\star,\star)_{K_\mathfrak{h}}$和原本的Killing形式$(\star,\star)_{K}$对于所有$x,y\in \mathfrak{h}$满足$(x,y)_{K_\mathfrak{h}}=(x,y)_K$.
% \end{pro}

% 对于第二点,注意到$\ad([x,y])=\ad(x)\ad(y)-\ad(y)\ad(x)$,
% \[
% 	\begin{split}
% 		([x,y],z)_K-(x,[y,z])_K&=\tr\{\ad([x,y])\ad(z)-\ad(x)\ad([y,z])\}\\
% 		&=\tr\{\ad(x)\ad(y)\ad(z)-\ad(y)\ad(z)\ad(x)\}\\
% 		&=0.
% 	\end{split}
% \]

% \begin{theo}
% 一个Lie代数是半单的当且仅当他没有一个非平凡的交换理想\footnote{交换理想首先是一个理想,然后也是一个交换子代数。}。
% \end{theo}
% 这个结论经常也被当成定义,可以看到这种半单的定义自动抛弃了一维理想的存在,因为一维理想一定是交换的。

% \begin{theo}
% 一个Lie代数是半单的,当且仅当他的Killing形式是非退化的,即矩阵$K_{ij}=(a_i,a_j)_K$是非退化的,或者对任意的$a$都成立$(a,b)_K=0$的话能推出$b=0$。
% \end{theo}

% 靠这个定理,我们来看看$\mathfrak{so}(3)$是否是半单的。容易计算得他的Killing形式为
% \[
% 	K_{ij}=(\eta_i,\eta_j)_K=\tr(\eta_i\eta_j)=-2\delta_{ij}.
% \]
% 当然是非退化的,所以$\mathfrak{so}(3)$是半单Lie代数。同理$\mathfrak{sl}(2)$也是半单Lie代数。

% \section{$\mathfrak{sl}(2,\cc)$, $\mathfrak{so}(3)$ and $\mathfrak{su}(2)$}
% 在分析一般的半单Lie代数之前,我们先来看看几个比较简单的Lie代数$\mathfrak{sl}(2,\cc)$, $\mathfrak{so}(3)$和$\mathfrak{su}(2)$,他们之间存在着紧密的联系。

% 前面已经证明过$\mathfrak{sl}(2,\cc)$, $\mathfrak{so}(3)$的基和相互的对易关系有
% \[
% [h,e]=2e,\quad[h,f]=-2f,\quad[e,f]=h,
% \]
% \[
% 	[\eta_1,\eta_2]=\eta_3,\quad [\eta_1,\eta_3]=-\eta_2,\quad [\eta_2,\eta_3]=\eta_1.
% \]

% 现在我们来看$\mathfrak{su}(2)$的表现,他是所有满足$B+B^\dag=0$的复二阶零迹矩阵$B$的集合。我们选如下三个矩阵作为基:
% \[
% \mu_1=\frac{1}{2}\begin{pmatrix}
% 	0&i\\
% 	i&0\\
% \end{pmatrix},\quad
% \mu_2=\frac{1}{2}\begin{pmatrix}
% 	0&-1\\
% 	1&0\\
% \end{pmatrix},\quad
% \mu_3=\frac{1}{2}\begin{pmatrix}
% 	i&0\\
% 	0&-i\\
% \end{pmatrix}.
% \]
% 容易验证
% \[
% 	[\mu_1,\mu_2]=\mu_3,\quad [\mu_1,\mu_3]=-\mu_2,\quad [\mu_2,\mu_3]=\mu_1.
% \]
% 这和$\mathfrak{so}(3)$的对易关系一模一样,于是我们可以引入$\rr$-线性映射建立两者作为实Lie代数的同构,也就是$\mathfrak{so}(3)\cong \mathfrak{su}(2)$.

% 很容易证明$\mathfrak{sl}(2,\rr)$的复化即$\mathfrak{sl}(2,\cc)$,这就是说$\mathfrak{sl}(2,\rr)_\cc=\mathfrak{sl}(2,\cc)$.为了分析$\mathfrak{sl}(2,\cc)$的结构,我们来看
% $\mathrm{SL}(2,\mathbb{C})$的结构。

% 任何一个复可逆$2\times 2$矩阵都可以唯一分解(极分解)为$
% \lambda=ue^{h}$,其中$u$幺正而$h$是Hermite矩阵。现在假如$\det \lambda=1$,则
% \[
% \det(u)e^{\tr(h)}=1
% \]
% 于是$\det(u)=1$而$\tr(h)=0$.前者的一般形式为
% \[
% u=
% \begin{pmatrix}
% a+ib&c+id\\
% -c+id&a-ib
% \end{pmatrix},
% \]
% 且满足$a^2+b^2+c^2+d^2=1$,因此其拓扑上等价为3-球面$\mathbb{S}^3$.而前者的一般形式为
% \[
% h=\begin{pmatrix}
% e&f-ig\\
% f+ig&-e
% \end{pmatrix},
% \]
% 拓扑上等价于$\rr^4$,因此$\mathrm{SL}(2,\cc)$在拓扑上等价于$\rr^4\times \mathbb{S}^3$.当然拓扑上的结论在我们这里暂时没什么用。

% 这样来看$\mathrm{SL}(2,\cc)$的Lie代数$\mathfrak{sl}(2,\mathbb{C})$,在$\mathrm{SL}(2,\cc)$的极分解中,令$b=-ih$,则
% \[
% \tr(b)=0,\quad b^\dag+b=0,
% \]
% 以及$u=e^a$有
% \[
% \tr(a)=0,\quad a^\dag+a=0,
% \]
% 所以$a,b\in\mathfrak{su}(2)$,且$\lambda=e^{a+ib}$.

% 注意到任取一个实数$t$和$a\in\mathfrak{su}(2)$,还有$ta \in\mathfrak{su}(2)$,所以任意的一个$\lambda \in \mathrm{SL}(2,\cc)$都可以写成
% \[
% \lambda=e^{ta+itb}
% \]
% 他在$t=0$的导数$a+ib$就构成$\mathfrak{sl}(2,\cc)$,那么任意的$c\in \mathfrak{sl}(2,\cc)$都可以写成
% \[
% c=a+ib,
% \]
% 其中$a,b\in\mathfrak{su}(2)$,这就是说$\mathfrak{sl}(2,\cc)$是$\mathfrak{su}(2)$的复化。

% 单纯从Lie代数来看,我们在$\mathfrak{su}(2)_\cc$中引入$L_n=i\mu_n$,则
% \[
% 	[L_1,L_2]=iL_3,\quad [L_1,L_3]=-iL_2,\quad [L_2,L_3]=iL_1.
% \]
% 再引入$L_\pm=L_1\pm iL_2$,则
% \[
% 	[L_+,L_-]=2L_3,\quad [L_3,L_+]=L_+,\quad [L_3,L_-]=-L_-.
% \]
% 我们令$h'=2h,e'=2e,f'=2f$,则$\mathfrak{sl}(2,\cc)$的三个基的对易关系变成
% \[
% [e',f']=2h',\quad[h',e']=e',\quad[h',f']=-f'.
% \]
% 可见一模一样。

% 这样,三个Lie代数之间的关系就清楚了
% \[
% 	\mathfrak{su}(2)\cong\mathfrak{so}(3),\quad \mathfrak{su}(2)_\cc\cong \mathfrak{sl}(2,\mathbb{C}) \cong\mathfrak{sl}(2,\rr)_\cc.
% \]

% 现在来看$\mathfrak{sl}(2,\mathbb{C})$的有限维不可约表示,每一个Lie代数的元素$a$都变成了有限维矢量空间$V$上面的线性映射$\pi(a)$,复数域的代数完备性可以推知$\pi(h)$有一个特征值,即
% \[
% 	\pi(h)v=\lambda v.
% \]
% 那么
% \[
% 	\pi(h)\pi(e)v=[\pi(h),\pi(e)]v+\pi(e)\pi(h)v=(\lambda+2)\pi(e)v.
% \]
% 所以$\pi(e)v$也是$\pi(h)$的本征矢量,本征值是$\lambda+2$,同理$\pi(f)v$也是$\pi(h)$的本征矢量,本征值是$\lambda-2$.

% 我们反复作用$\pi(e)$和$\pi(f)$到$v$上就可以得到
% \[
% 	\pi(h)\pi(e)^nv=(\lambda+2n)\pi(e)^nv,
% \]
% 所以,要么$\pi(e)^nv$也是一个本征矢,本征值为$\lambda+2n$,或者$\pi(e)^nv=0$。由于$V$是有限维的,我们不可能有着无穷多个不同的本征值,因此存在一个$N\geq 0$使得
% \[
% 	\pi(e)^Nv\neq 0,\quad \pi(e)^{N+1}v=0
% \]
% 这就是说存在一个$u_0$使得
% \[
% 	\pi(h)u_0=\lambda u_0,\quad \pi(e)u_0=0.
% \]
% $\lambda$是$\pi(h)$的最大的本征值。

% 我们再定义$u_k=\pi(f)^ku_0$,那么
% \[
% 	\pi(h)u_k=(\lambda-2k) u_k,
% \]
% 也不可能无限地进行下去。就是说存在一个$m\in \mathbb{N}$使得$k\leq m$满足$u_k\neq 0$但$u_{m+1}=0$.

% 使用归纳法和对易关系$[\pi(e),\pi(f)]=\pi(h)$可以算得
% \[
% \pi(e)u_k=(k\lambda -k(k-1)) u_{k-1}\quad (k>0).
% \]
% 假如$u_{m+1}=0$,那么
% \[
% 0=\pi(e)u_{m+1}=((m+1)\lambda -m(m+1)) u_{m}=(m+1)(\lambda-m)u_{m}.
% \]
% 这就是说$\lambda=m$.那么一个本征值为正整数,其他的本征值可以通过$\lambda-2n$得到,所以也是整数,这就推出了:

% \pro
% $\mathfrak{sl}(2,\mathbb{C})$的$m+1$维不可约表示中$\pi(h)$的本征值都是整数,且可以对角化为$\mathrm{diag}(m,m-2,\cdots,-m+2,m)$。

% 到上面为止,唯一留下的就是要证明对任意的$m$上面的表示都是不可约的,而且确实是$\mathfrak{sl}(2,\mathbb{C})$的表示。为此,我们可以证明$\{u_0,u_1,\cdots,u_m\}$构成一组基,这是因为每一个$u_k$都是$\pi(h)$对应不同本征值的本征矢量,于是$\pi(f)u_k=u_{k+1}$保证了不可约性。而他是$\mathfrak{sl}(2,\mathbb{C})$的表示则是可以直接计算验证的,略去之。

% 将表示的矩阵写出来可能更加清晰,$\pi_m(h)=\mathrm{diag}(m,m-2,\cdots,-m+2,m)$和
% \[
% \pi_m(e)=\begin{pmatrix}
% 	0&m&&&\\
% 	&0&m-1&&\\
% 	&&\ddots&\ddots&\\
% 	&&&0&1\\
% 	&&&&0\\
% \end{pmatrix},\quad
% \pi_m(f)=\begin{pmatrix}
% 	0&&&&\\
% 	1&0&&&\\
% 	&2&\ddots&&\\
% 	&&\ddots&\ddots&\\
% 	&&&m&0\\
% \end{pmatrix}.
% \]

% 在物理上更习惯用整数或者半整数$j=m/2$来表示维度,并且会适当调整$\pi(h)$的系数,这就造成了角动量问题中的半整数的出现。

% \pro $\mathfrak{sl}(2,\mathbb{C})$的$2j+1$维不可约表示中$\pi(h)$的本征值或者是整数或者是半整数,且可以对角化为$\mathrm{diag}(-j,-j+1,\cdots,j-1,j)$。

% \section{$\mathrm{SU}(2)$ and $\mathrm{SO}(3)$}
% 上面看到了$\mathfrak{sl}(2,\mathbb{C})$的Lie代数,或者说$\mathfrak{su}(2)_\cc\cong\mathfrak{so}(3)_\cc$的Lie代数,现在我们回到Lie群$\mathrm{SU}(2)$和$\mathrm{SO}(3)$。

% 首先复习一下$\mathrm{SO}(3)$的一些更细的结构,$\mathrm{SO}(3)$是三维旋转群,不包括反射。

% \pro
% 绕着固定单位矢量$\mathbf{a}$右手螺旋的逆时针方向旋转$\theta$后的$\mathbf{x}$表示为
% \[
% \mathrm{Rot}(\mathbf{a},\theta)\mathbf{x}=\mathbf{x}+(1-\cos\theta)\mathbf{a}\times(\mathbf{a}\times\mathbf{x})+
% \sin\theta\,\mathbf{a}\times\mathbf{x}.
% \]

% 为了证明他,首先注意到当$\mathbf{a}$和$\mathbf{x}$垂直的时候显然有
% \[
% \mathrm{Rot}(\mathbf{a},\theta)\mathbf{x}=\cos\theta\,\mathbf{x}+
% \sin\theta\,\mathbf{a}\times\mathbf{x}.
% \]
% 对于一般的$\mathbf{x}$,分解为平行于$\mathbf{a}$和垂直于$\mathbf{a}$(设其单位矢量为$\mathbf{b}$)的两个部分$\mathbf{x}=\mu\mathbf{a}+\nu\mathbf{b}$,而平行于$\mathbf{a}$的部分在旋转下不变$\mathrm{Rot}(\mathbf{a},\theta)\mu\mathbf{a}=\mu\mathbf{a}$。由于$\mathrm{Rot}(\mathbf{a},\theta)$的线性性,我们有
% \[
% \begin{split}
% \mathrm{Rot}(\mathbf{a},\theta)\mathbf{x}&=\mu\mathbf{a}+\cos\theta\,(\nu\mathbf{b})+
% \sin\theta\,\mathbf{a}\times(\nu\mathbf{b})\\
% &=\mathbf{x}\cos\theta+(1-\cos\theta)\mu\mathbf{a}+
% \sin\theta\,\mathbf{a}\times\mathbf{x}\\
% &=\mathbf{x}\cos\theta+(1-\cos\theta)(\mathbf{a}\cdot\mathbf{x})\mathbf{a}+
% \sin\theta\,\mathbf{a}\times \mathbf{x},
% \end{split}
% \]
% 但是$\mathbf{a}\times(\mathbf{a}\times\mathbf{x})=(\mathbf{a}\cdot\mathbf{x})\mathbf{a}-\mathbf{x}$.所以得证。

% 现在对$\mathrm{Rot}(\mathbf{a},\theta)$中的$\theta$求$\theta=0$处的导数即可得到其Lie代数
% \[
% \left.\frac{\dd}{\dd \theta}\right|_{\theta=0}\mathrm{Rot}(\mathbf{a},\theta)\mathbf{x}=\mathbf{a}\times\mathbf{x}
% \]
% 选$\mathbf{a}=\mathbf{e}_i$,得到Lie代数的三个基为
% \[
% 	\eta_1=\begin{pmatrix}
% 		0&0&0\\
% 		0&0&-1\\
% 		0&1&0\\
% 	\end{pmatrix},\quad
% 	\eta_2=\begin{pmatrix}
% 		0&0&1\\
% 		0&0&0\\
% 		-1&0&0\\
% 	\end{pmatrix},\quad
% 	\eta_3=\begin{pmatrix}
% 		0&-1&0\\
% 		1&0&0\\
% 		0&0&0\\
% 	\end{pmatrix}.
% \]