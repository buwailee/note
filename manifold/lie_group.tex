\chapter{Lie Group}

这章处理一大类光滑流形,Lie群。Lie群是一个光滑群,即是他一方面有着群的结构,而另一方面还是一个光滑流形,其中群的运算乘法和逆是光滑的。最重要的Lie群的例子是很久以前提到过的一般线性群及其子群。

所谓的一般线性群,就是将$n\times n$实矩阵放入$\rr^{n^2}$内,那么行列式不为零的那些矩阵就构成一个光滑流形,记作$\mathrm{GL}(n,\rr)$. 更一般的,我们记$\mathrm{GL}(V)$为$V$上行列式非零的线性变换构成的群。

\section{Foundation}

\para 设$G$为群,单位元记做$e$,群运算记做$\mu:G\times G\to G$,如果$G$是一个光滑流形,且$\mu$是一个光滑映射,则称$G$是一个光滑Lie群。当然可以谈论不怎么光滑的Lie群,但是下面所指的Lie群都是光滑的。记$l_g$是左作用算符,即$l_g=\mu(g,\cdot)$,或者写作$l_gh=gh$,同样,右作用算符记做$r_g$,即$r_gh=hg$.显然这些都是光滑映射。

现在考虑方程$\mu(x,y)=e$,由于$(l_e)_{*e}$是一个恒同映射,所以在$e$附近,按隐函数定理,方程$\mu(x,y)=e$在$e$附近有光滑解,即$y=\nu(x)=x^{-1}$中的逆函数$\nu$在$e$附近光滑,由于$(\nu\circ l_g)(h)=h^{-1}g^{-1}=(r_{g^{-1}}\circ \nu)(h)$,所以逆函数$\nu$处处光滑。

因为Lie群有着光滑流形结构,那么我们就可以对其局部线性化,特别地,单位元附近的局部线性化就构成了Lie代数的内容。

\para 一个Lie群$G$的Lie代数$\lag$就是其单位元处的切空间。

由于$l_g$和$r_g$都是$G\to G$的光滑同胚,所以$(l_g)_*$或者$(r_g)_*$就是将光滑切矢量场映射到光滑切矢量场的双射。如果矢量场$X_x$满足$(l_a)_*X_x=X_{ax}$,则称$X_a$是一个左不变矢量场。和任意的矢量场一样,左不变矢量场$X$在$e$处诱导了$\lag$中的元素$X_e$.

反过来,设$X_e\in\lag$,我们可以构造一个左不变矢量场$X_x=(l_{x})_*X_e=(l_x)^{-1}_*X_x$,因此我们就建立了Lie代数和左不变矢量场之间的一一对应。

\para 左不变矢量场都是光滑的且完备的。

实际上,任取$f\in\calf(G)$以及$g\in G$,我们来看$(Xf)(g)=X_gf=(l_g)_{*}X_ef=X_e(f\circ l_g)$. 取一个在$e$处切矢量为$X_e$的曲线$\sigma$,则
\[
	Xf(g)=\frac{\dd }{\dd t}\bigg|_{t=0}f(g\sigma(t))=\frac{\dd }{\dd t}\bigg|_{t=0}f\circ \mu(g,\sigma(t)),
\]
是一个光滑函数,所以左不变矢量场都是光滑矢量场。至于完备性,设$\sigma$是左不变矢量场$X$的积分曲线,则$l_g\circ \sigma$也是$X$的积分曲线,这就使得我们可以把一条局部积分曲线拼到无穷远,这就是说$X$是完备的。

\para 设$X$是左不变矢量场,因为$X$是完备的,所以他诱导的单参数变换群$\{\sigma^X_t\}$在整个Lie群上是有定义的,特别地,在单位元上,我们定义$\exp(t,X):=\sigma^X_t(e)$,其中$X$是左不变矢量场,因为左不变矢量场一一对应着Lie代数,所以也可以说$X\in \lag$.同样,对固定的$X$,映射$\exp(t,X)$对改变的$t$就构成了$G$的一个子群,这被称为单参子群。由积分曲线的存在唯一性,我们也得到了单参子群与Lie代数的一一对应关系。

\pro 设$f:G\to G$是一个微分同胚,我们有$f_*[X,Y]=[f_*X,f_*Y]$,若$f=l_a$,那么我们立刻就得到了左不变矢量场的对易子也是左不变的。因此对于Lie代数来说,他继承了切矢量场的Lie括号$[\bullet,\bullet]:\lag\times \lag\to \lag$,这是一个二元线性运算,所以Lie代数确实是一个代数。

通过直接的计算,Lie代数上满足:

\no{1} $[X,Y]=-[Y,X]$,

\no{2} $[X,[Y,Z]]+[Y,[Z,X]]+[Z,[X,Y]]=0$.

第一条反对称性从矢量场的$[X,Y]=XY-YX$来看是显然的。而第二条称为Jacobi恒等式,直接计算即可验证。可以如下记忆Jacobi恒等式,$X$, $Y$和$Z$的三种右手方向构成的置换和为$0$,或者说,$[X_i,[X_j,X_k]]$中$ijk$是$123$的偶置换。

\para Lie代数上的双线性映射$f$如果满足$f([X,Y])=[f(X),Y]+[X,f(Y)]$,则$f$被称为一个导子。

在Lie代数上我们可以找到一个自然的导子。适当改写Jacobi恒等式,我们可以得到$[X,[Y,Z]]=[[X,Y],Z]+[Y,[X,Z]]$,如果记$\ad(X):Y\mapsto [X,Y]$,于是
\[
	\ad(X)([Y,Z])=[\ad(X)Y,Z]+[Y,\ad(X)Z],
\]
因此$\ad(X)$就是一个Lie代数上面的导子。

\para 前面我们定义了$\exp:\lag\times \rr\to G$,特别地,我们记$\exp(1,X)$为$\exp(X)$,这样定义的映射$\exp:\lag\to G$他被称为指数映射。

可以看到$\exp(tX)=\sigma^{tX}_1(e)=\sigma^{X}_t(e)=\exp(t,X)$,所以实际上,我们指数映射已经能够完全包含$\exp:\lag\times \rr\to G$的内容了。特别地,
\[
	\exp(tX)\exp(sX)=\exp(t,X)\exp(s,X)=\exp(t+s,X)=\exp((t+s)X).
\]
就和一般的指数表现得那样。但如果$[X,Y]\neq 0$,一般来说$\exp(X)\exp(Y)\neq \exp(X+Y)$.我们有时候也会通过$e^{X}$来记$\exp(X)$.

\lem \label{exp}我们找一个光滑函数$f:G\to \rr^n$,那么$g(t)=f(xe^{tX})$就是一个$\rr$上的光滑函数,我们来归纳证明他的$n$阶导数为
\[
	\frac{\dd^n}{\dd t^n}g(t)=(X^nf)(x e^{tX}).
\]

\proof $n=0$是显然的,$n=1$需要直接计算验证
\[
	(Xf)(x)=\left\{\frac{\dd}{\dd t}f(x e^{tX})\right\}_{t=0},
\]
这个的计算只要使用链式法则
\[
	\left\{\frac{\dd}{\dd t}f(x e^{tX})\right\}_{t=0}=f_{*x}(e^{tX})_{*0}=f_{*x}X=(Xf)(x).
\]
注意最后一个等式要依赖于$f$是矢量值的,某种程度来说这就是$T\rr^n=\rr^n$的结果。由于矩阵也可以看成在欧氏空间$\rr^{n\times n}$里,所以$f$也可以取值为矩阵。

假设$n=k$是成立的,那么因为$X^{k+1}=X\circ X^k$,
\[
	(X^{k+1}f)(x e^{tX})=(X(X^{k}f))(x e^{tX})=\left\{\frac{\dd}{\dd s}(X^kf)(x e^{(s+t)X})\right\}_{s=0}=\frac{\dd}{\dd t}(X^kf)(x e^{tX})=\frac{\dd^{k+1}}{\dd t^{k+1}}g(t).
\]\qed

\para Lie群$G$的切丛$TG$倒是相当简单,因为我们可以定义$(l_{a^{-1}})_*$把$T_aG$始终映射到$T_eG=\lag$来考虑,所以切丛就被平凡化了。与这相关的概念即Maurer-Cartan形式。设$G$是一个Lie群,他的切丛记做$TG$,映射$\omega_G:(g,v)\mapsto (l_{g^{-1}})_*v$被称为Maurer-Cartan形式。可以看到$\omega_G:TG\to \lag$,因此Maurer-Cartan形式可以看做一个$\lag$值函数。且对于任意的$l_h^*$,我们都有
\[
(l_h^*\omega_G)v=\omega_G((l_h)_*v)=(l_{(hg)^{-1}})_*(l_h)_*v=(l_{(g)^{-1}})_*v=\omega_G(v).
\]
所以Maurer-Cartan形式是左不变的。

\para 现在来看具体的例子,设所有$n\times n$的实(复)矩阵构成的集合为$\mathrm{M}(n,\rr)$($\mathrm{M}(n,\cc)$),其中$\det A\neq 0$的矩阵按矩阵乘法构成一个群$\mathrm{GL}(n,\rr)$($\mathrm{GL}(n,\cc)$),我们称为一般线性群,单位元是$I$。由于他可以开嵌入$\rr^{n^2}$内,所以他有自然的光滑流形结构。因此一般线性群是一个Lie群,矩阵群上的微分定义使得我们可以直接计算一般线性群的Lie代数,他的Lie代数为$\mathfrak{gl}(n,\rr)$。在一般线性群$G$上
\[
	(l_g)_{*a}v=\frac{1}{t}(l_g(a+tv)-l_g(a))
	=\frac{1}{t}(l_g(tv))=l_g(v)=gv.
\]
其中$v\in T_aG$.

所以一般线性群上面的Maurer-Cartan形式即为$\omega_G(v)=l_{g^{-1}}(v)=g^{-1}v$,其中$g$和$v$都是矩阵,矩阵乘矩阵还是矩阵,所以Lie代数$\mathfrak{gl}(n,\rr)$也是矩阵的形式。设$\dd g=(\dd x_{ij})$,那么$v$就可以写成$\dd g(v)$,因为$\dd x_{ij}(v)=v_{ij}$,则$\omega_G=g^{-1}\dd g$.

由于$\mathrm{GL}(n,\rr)$的微分结构是熟知的,我们可以直接计算其Lie代数$\mathfrak{gl}(n,\rr)$上的交换子形式。设$A\in\mathfrak{gl}(n,\rr)$而$g\in\mathrm{GL}(n,\rr)$,容易验证$A_g=gA$是左不变矢量场,因为$(l_h)_{*}A_g=(l_h)_{*}gA=hgA=A_{hg}$.

记$g=(x_{ij})$,考虑与$A=(a_{ij})$和$B=(b_{ij})$相关的左不变矢量场为
\[
A_g=\sum_{i,j,k}x_{ij}a_{jk}\partial_{ik},\quad B_g=\sum_{i,j,k}x_{ij}b_{jk}\partial_{ik},
\]
于是
\[
[A_g,B_g]=\left[\sum_{i,j,k}x_{ij}a_{jk}\partial_{ik},\sum_{i,j,k}x_{ij}b_{jk}\partial_{ik}\right]=\sum_{i,k}\left(\sum_{j}x_{ij}\sum_{r}(a_{jr}b_{rk}-b_{jr}a_{rk})\right)\partial_{ik},
\]
或者$[A_g,B_g]=(AB-BA)_g$,所以$\mathfrak{gl}(n,\rr)$上的对易子为$[A,B]=AB-BA$,其中的乘法就是矩阵乘法。

\para 对于$A\in\mathfrak{gl}(n,\rr)$,指数映射有如下级数展开
\[
	e^A=1+\sum_{n=1}^\infty \frac{A^n}{n!}=\sum_{n=0}^\infty \frac{A^n}{n!},
\]
对于任意的矩阵$A$都是收敛的。可以看到其完全类似于实数值指数函数的展开$e^x=\sum_{n=0}^\infty x^n/n!$

\proof 由\lemref{exp},对一般的Lie群$G$和光滑函数$f:G\to \rr^n$,使用Taylor公式
\[
	f(xe^{tX})=\sum_{k=0}^n\frac{(tX)^{k}}{k!}f(x)+O(t^{n+1}),
\]
如果可以展开无数项,那么
\[
	f(xe^{tX})=\sum_{k=0}^\infty\frac{(tX)^{k}}{k!}f(x).
\]

现在取$f(A)=A$, $x=I$和$t=0$就可以了。至于收敛性,因为对于任意一个矩阵,$A$的范数都是有界的,那么$e^A$就被$A$的范数的级数控制,因此收敛。\qed

当然可以用其他的方式猜出这个关系,我们考虑$e^{tA}$,将其在$t=0$附近展开,有$e^{tA}=I+tA+O(t^2)$,然后对于任意的正整数$n$和固定的$t$我们有
\[
	e^{tA}=\left(e^{tA/n}\right)^n=\left(I+\frac{t}{n}A+O\left(\frac{1}{n^2}\right)\right)^n,
\]
然后令$n\to\infty$,就有$e^{tA}=\lim_{n\to\infty}\left(I+\frac{t}{n}A\right)^n$.使用二项式展开,就可以得到其级数展开
\[
	e^{tA}=1+\sum_{n=1}^\infty \frac{(tA)^n}{n!}=\sum_{n=0}^\infty \frac{(tA)^n}{n!},
\]
最后$t=1$即可。

上面的过程可能不怎么严谨,在矩阵的情况下,直接用级数定义指数映射反而可能更加简单。

\para 以下矩阵构成一般线性群的子群:

\no{1} 特殊线性群:$\mathrm{SL}(n,\rr)=\{A\in \mathrm{M}(n,\rr)|\det A=1\};$

\no{2} 正交群:$\mathrm{O}(n) = \{ Q \in \mathrm{M}(n,\rr) \mid Q^T Q = Q Q^T = I \};$

\no{3} 酉群:$\mathrm{U}(n) = \{ Q \in \mathrm{M}(n,\cc) \mid Q^\dag Q = Q Q^\dag = I \};$

\no{4} 特殊正交群:$\mathrm{SO}(n) =\{ Q \in \mathrm{O}(n) \mid \det Q=1 \};$

\no{5} 特殊酉群:$\mathrm{SU}(n) =\{ Q \in \mathrm{U}(n) \mid \det Q=1 \};$

我们来考虑最简单的一个群$\mathrm{SO}(2)$,他的群元素由矩阵
\[
	\begin{pmatrix}
	\cos \theta&-\sin \theta\\
	\sin \theta&\cos \theta\\
	\end{pmatrix}
\]
构成。这是一个Abel群,而且可以注意到,他同构于群$\mathrm{S}^1=\{e^{i\theta}:\theta\in\rr\}$,这是一个圆周。

\pro 对于矩阵群构成的空间,我们可以使用Heine-Borel定理断言有界闭子群是紧的。所以正交群是紧的,但是一般线性群不是紧的。

但是对于$\mathrm{GL}(n,\rr)$的子群来说,正交群就是极大的紧子群了。

\pro 一般线性群$\mathrm{GL}(n,\rr)$的子群$G$如果是紧的,那么存在可逆矩阵$A$使得$AGA^{-1}\subset \mathrm{O}(n)$.

\proof 设$A\in G$,则序列$A,A^2,\cdots,A^n,\cdots$都在$G$里面。如果$|\det A|>1$,那么$|\det A^n|=|\det A|^n$就可以任意大,和紧性对应的有界性相悖。如果$|\det A|<1$,那么由$G$的紧性,其$A^n$这个序列收敛到$G$内,但$|\det A^n|=|\det A|^n$却又小于任意的正数,所以$A$不可逆但在$G$内,则和$G$作为一般线性群的子群相悖。所以$|\det A|=1$.

现在只考虑$\det A=1$,这就是特殊线性群的情况。然后还设$\mathrm{SO}(3)$在我们的子群里。由于我们总可以将特殊线性群的元素唯一分解为$A=RP$,其中$R\in \mathrm{SO}(3)$,而$P$是正定对称矩阵,且$\det P=1$。由于正定对称一定可以对角化,适当选择基,我们使得$P$就是对角的,所以$\det P=\lambda_1\cdots\lambda_n=1$.

那么$P=R^{-1}A$,由于$R$在我们的群内,那么$R^{-1}$也是,因此$P$也是。如果存在一个$|\lambda_i|>1$,不妨假设就是$\lambda_1$,此时$P^{k}=\mathrm{diag}(\lambda_1^k,\cdots,\lambda_n^k)$中的$|\lambda_1|^k$会比任意正数大,这和紧性相悖。所以所有的$|\lambda_i|=1$,但是由于是正定的,所以$P=I$.因此$A=R\in\mathrm{SO}(3)$.

对于$\det A=-1$的情况类似,对于共轭,也是显然的。 \qed

现在我们来求这个极大的紧子群$\mathrm{O}(n)$的Lie代数,因为Maurer-Cartan形式取$\lag$值,所以我们只要整理出Maurer-Cartan形式就可以了。对恒等式$AA^T=I$求导有$\dd A A^T+A(\dd A)^T=0$,或者
\[
	A^{-1}\dd A+(A^{-1}\dd A)^T=0,
\]
$A^{-1}\dd A$就是Lie代数。$\mathrm{O}(n)$的Lie代数就是满足方程$B+B^T=0$的矩阵,换而言之,反对称矩阵。

此外,由矩阵恒等式,$AA^{-1}=I$,对其求导我们有$0=\dd A A^{-1}+A\dd(A^{-1})$,所以
\[
	\dd(A^{-1})=-A^{-1}(\dd A) A^{-1}.
\]
如果$A(t):\rr \to \mathrm{GL}(n,\rr)$是一个单参子群,那么有类似的
\[
	(A^{-1}(t))'=-A^{-1}(t)A'(t) A^{-1}(t).
\]
以上就是微积分里面$(1/x)'=-1/x^2$的矩阵对应。

\section{Basic Representation Theory}

上节谈了Lie群的“内部线性化”,即Lie代数的内容。这节我们来谈论群表示,他使得我们可以把群进行“外部线性化”,粗略地来说就是我们把群元素看做了一个线性变换。

\para 令$V$是一个域$k$(后面我们只会考虑$\rr$和$\cc$的情况)上的有限维矢量空间,群$G$的一个表示$U$指存在这样的一个群同态$\pi:G\rightarrow \mathrm{Gl}(V)$,使得
\[
	\pi(g)\pi(g')=\pi(gg'),\quad \pi(g^{-1})=\pi(g)^{-1},\quad \pi(e_G)=e_{\mathrm{Gl}(V)}
\]
成立。表示$\pi$的维度被定义为$V$的维度。

如果没什么会混淆的话,就直接略去$\pi$,写$gx$来表达$\pi(g)x$,当然还有用$Gx$来表达$\pi{G}x=\{\pi(g)x:g\in G\}$。如果要用不会混淆的说法来表达群表示的话,就应该用$(\pi,V)$二元组,$\pi$是群同态,$V$是被作用的空间。

如果$\pi$是一个单同态,那么我们称这个表示为忠实表示。

\para 同一个群$G$的两个表示$(\pi_1,V_1)$和$(\pi_2,V_2)$可以构造出新的表示:

\no{1} 直和:$(\pi_1\oplus \pi_2,V_1\oplus V_2)$即满足
\[
	(\pi_1\oplus \pi_2)(g)(x,y)=(\pi_1(g)x,\pi_2(g)y).
\]
或者简单地写作$g(x,y)=(gx,gy)$.

\no{2} 直积:$(\pi_1\otimes \pi_2,V_1\otimes V_2)$即满足
\[
	(\pi_1\otimes \pi_2)(g)(x\otimes y)=(\pi_1(g)x)\otimes (\pi_2(g)y).
\]
或者简单地写作$g(x\otimes y)=(gx)\otimes(gy)$.

\para 设$G$是一个群,$(\pi,V)$是他的一个表示,$W\subset V$是一个子空间,如果$\pi(G)W\subset W$则$W$称为不变子空间。显然,$\{0\}$和$V$是两个不变子空间,略去这两个平凡不变子空间,如果没有其他不变子空间了,则$V$被称为是不可约的,此时表示被称为不可约表示。

\para 如果一个表示能被分解成几个不可约表示的直和,则称该表示为完全可约的。

下面的一系列定义涉及表示的等价,当然还有很重要的Schur引理。

\para 设$G$有两个表示$(\pi_1,V_1)$和$(\pi_2,V_2)$,如果存在线性映射$T:V_1\to V_2$,对任意的$g\in G$都满足$\pi_2(g)\circ T=T\circ \pi_1(g)$. 这样的$T$被称为缠结映射。当缠结映射是同构的时候,两个表示被称为是等价的。显然,这是一个等价关系。自反对称显然,而传递性自然来自两个同构复合还是同构。

\lem 如果$\pi_1$和$\pi_2$之间存在缠结映射$T$,那么$\ker T$是$\pi_1$的不变子空间,而$\im T$是$\pi_2$的不变子空间。

\proof 因为$T(\pi_1(g)x)=\pi_2(g)Tx$对于任何$x$使得$Tx=0$的,都有$\pi_1(g)x$使得$T(\pi_1(g)x)=0$,所以前半句话证明完了。对于$\im T$中的元素$y$可以找到原象$x$,由于$\pi_2(g)y=\pi_2(g)Tx=T(\pi_1(g)x)$,则$\pi_2(g)y$也在$\im T$中,后半句话证完。\qed

\lem 相当有名的Schur引理:如果$(\pi_1,V_1)$和$(\pi_2,V_2)$是不等价的不可约表示,若存在缠结映射$T$,则$T=0$.换句话说,不等价的表示没有非平凡的缠结映射。

\proof 如果两个表示不等价,则$T$不是双射,所以$\ker T\neq \{0\}$,而他是不变子空间,由不可约性,则$\ker T= V_2$,同理$\im  T= \{0\}$,这就是说$T=0$。\qed

\para 反之,在复表示情况下,我们考虑不可约表示$(\pi_1,V_1)=(\pi_2,V_2)$的情况下。此时如果存在一个非平凡的双的缠结映射$T\neq I$使得$T\circ \pi(g)=\pi(g)\circ T$.

在复数域上,$T$一定存在一个本征值$\lambda$,令$E_\lambda$是$\lambda$的本征空间,任取$v\in E_\lambda$,我们有$T(\pi(g)(v))=\pi(g)\circ T(v)=\lambda\pi(g)(v)$,所以$\pi(G)E_\lambda\subset E_\lambda$,由$\pi$的不可约性,所以他要么是零空间,要么是全空间。而本征值的存在性说明了零空间不可能,所以本征空间就是全空间,这也就是说$T=\lambda I$.

\para 如果我们遇到的群是Abel群,那么他的不可约群表示也是可交换的,同时其本身就构成了一个缠结映射,即$T=\pi(g)$对任意的$h$成立$T\circ \pi(h)=\pi(h)\circ T$,所以通过上面的定理我们就可以知道,$\pi(g)=T=\lambda I$。

如果$V$是大于$1$维的,那么任意的$V$的子空间都是$\pi$的不变子空间,但不可约性否决了这点,所以我们得到了:凡Abel群的不可约复表示都是一维的。

\para 如果$(\bullet,\bullet)$是$V$上的一个内积,如果对任意的$g\in G$,$u,v\in V$有$(u,v)=(gu,gv)$,则我们称呼这个表示为幺正表示,幺正表示也可以写作$\pi(g)^{-1}=\pi(g)^\dag$。对于有限群我们总可以找到幺正表示,因为我们可以重新构造内积
\[
	(u,v)'=\frac{1}{|G|}\sum_{g\in G}(gu,gv),
\]
那么
\[
	(hu,hv)'=\frac{1}{|G|}\sum_{g\in G}(hgu,hgv)=\frac{1}{|G|}\sum_{hg\in G}(hgu,hgv)=\frac{1}{|G|}\sum_{k\in G}(ku,kv)=(u,v)'.
\]

幺正表示的好处是,一个不变子空间的正交空间也是不变的。对于一个有限维的幺正表示,如果不是完全可约的,那么就分解出一个不变子空间,他是不可约的,然后对这个不变子空间的正交空间,我们又得到了一个可约或不可约的不变子空间,靠着有限归纳(因为有限维),我们就得到了如果群存在一个有限维幺正表示,则这个表示一定是完全可约的。

\theo 结合上面所说的,我们就得到了Maschke定理:有限群的有限维表示总是完全可约的。 

\theo 在局部紧的拓扑群上存在Haar测度$\mu$,他是一个正则的左不变的Borel测度。所谓的左不变就是指对于一个集合$S$,通过左作用$L_g$,我们有$\mu(S)=\mu(L_gS)$,当然还有右作用和右不变的概念。

对于$G$上的可测函数$f$,上述测度满足
\[
	\int_G f(l_h g)\dd \mu(g)=\int_G f(g)\dd \mu(g),
\]
可以从中直接看出左不变的意义。以后我们将$\dd \mu(g)$直接写作$\dd g$.

\para 上面的定理我们不证明,同时也不证明如下命题:如果$G$是紧Lie群,则我们存在双不变(既左不变也右不变)测度$\mu$,且$\mu(G)$有限。

\pro 紧Lie群上存在幺正表示,所以紧Lie群的有限维表示总是完全可约的。 

换而言之,我们可以找到内积$\langle \bullet,\bullet\rangle$使得$\langle gx,gy\rangle=\langle x,y\rangle$。假设原本存在内积$\langle \bullet,\bullet \rangle$,那么再设$\dd g$是$G$上的Haar测度(因为紧所以存在,而且已经归一化),那么
\[
	\langle x,y\rangle'=\int_G \langle gx,gy \rangle \dd g
\]
就满足了要求。下面我们谈论紧Lie群的表示的时候,总使用幺正表示。

\para 对于Lie群$G$上的复值光滑函数$f$和$g$,我们可以定义内积为
\[
	(f,g)=\int_G f(h)g^*(h)\dd h,
\]
其中$*$代表的是复共轭。

\para 对于矢量空间$V$和$W$,$\Hom(V,W)$也有自然的矢量空间结构,所以如果已知$V$上有群表示$\pi_1$,以及$W$上有群表示$\pi_2$,则对于$A\in \Hom(V,W)$,我们可以定义群表示$(\pi_1^*\otimes\pi_2,\Hom(V,W))$如下
\[
	\pi_1^*\otimes\pi_2(g)A=\pi_2(g)A\pi_1(g)^{-1}.
\]
很容易验证$\pi_1^*\otimes\pi_2(g)\pi_1^*\otimes\pi_2(h)=\pi_1^*\otimes\pi_2(gh)$,所以这是一个群表示。为了省略空间,我们经常也直接记做$gA$.

对每一个群元$g$,表示诱导了映射$g:A\mapsto gA$,现在如果$A$是该映射的不动点$gA=A$,则有$A\pi_1(g)=\pi_2(g)A$。如果对每一个$g$,$A$都是其不动点,那么我们就称$A$是表示$\pi_1^*\otimes\pi_2$的不动点。那么对于群表示$(\pi_1,V)$和$(\pi_2,W)$,$A$就是一个缠结映射。

\para 利用Schur引理,假若$\pi_1$, $\pi_2$都是不可约的且不等价的,则群表示$(\pi_1^*\otimes\pi_2,\Hom(V,W))$的不动点集平凡(即只有$A=0$这一个不动点)。

假如$\pi$, $\rho$都是不可约的且不等价的,现在我们考虑线性映射$A\in \Hom(V,W)$在紧Lie群上的平均\footnote{在有限群上也有平均,所以下面一套对有限群也成立。}$\bar{A}=\int_G \pi^*\otimes\rho(g)A \dd g$,显然$\bar{A}$是一个不动点,因此由Schur引理,
\[
	\bar{A}=\int_G \rho(g)A\pi(g)^{-1}\dd g=0
\]
对任意的线性映射$A\in \Hom(V,W)$都成立。

同样,如果$\pi$, $\rho$等价,此时不妨就直接记$\pi=\rho$.因为$\bar{A}$是一个缠结映射,那么由Schur引理,$\bar{A}=\lambda_{\bar{A}} I$,所以
\[
	\int_G \pi(g)A\pi(g)^{-1}\dd g=\lambda_{\bar{A}} I
\]
对任意的线性映射$A\in \Hom(V,V)$都成立,但是由于$\lambda_{\bar{A}}$并不已知,所以这公式用着不是很方便。为了求出$\lambda_{\bar{A}}$,两边求迹,就有$\tr(A)=\lambda_{\bar{A}} \dim V$,即$\lambda_{\bar{A}}=\tr(A)/\dim V$,所以我们也可以将上式写作更实用的形式
\[
	\bar{A}=\int_G \pi(g)A\pi(g)^{-1}\dd g=\frac{\tr(A)}{\dim V}I.
\]

\para 现在,为了做一点小小的计算,我们将固定矢量空间的基,此时任意的线性算子都可以看做矩阵。记$E_{ij}$为只在$(a,b)$位置为1,其他位置为0的矩阵。很容易可以计算得到$E_{ij}AE_{kl}=A_{jk}E_{il}$.现在对
\[
	\bar{A}=\int_G \rho(g)A\pi(g)^{-1}\dd g
\]
左乘$E_{ij}$,右乘$E_{kl}$,并令$A=E_{ab}$,可以得到
\[
	\int_G E_{ij}\rho(g)E_{ab}\pi(g)^{-1}E_{kl}\dd g=\int_G \rho(g)_{ja}E_{ib}\pi(g)^{-1}E_{kl}\dd g=\int_G (\rho(g))_{ja}\left(\pi(g)^{-1}\right)_{bk}E_{il}\dd g,
\]
由于$\pi$是幺正表示,所以$\left(\pi(g)^{-1}\right)_{bk}=\left(\pi(g)^{\dag}\right)_{bk}=\pi(g)_{kb}^*$,并记$\pi_{ia}:g\to \pi(g)_{ia}$为分量函数,则利用复函数内积的写法,上式可以写作
\[
	E_{ij}\overline{E_{ab}}E_{kl}=\left(\rho_{ja},\pi_{kb}\right)E_{il}.
\]

\para 设$\rho$和$\pi$都是不可约的,对于$\rho$和$\pi$不等价的情况,我们有$\overline{E_{ab}}=0$,所以$\left(\rho_{ja},\pi_{kb}\right)=0$.类似地,对于$\rho=\pi$的情况,我们有$\overline{E_{ab}}=\tr(E_{ab})I/\dim V=\delta_{ab}I/\dim V$,所以
\[
	\left(\pi_{ja},\pi_{kb}\right)E_{il}=E_{ij}\overline{E_{ab}}E_{kl}=\frac{\delta_{ab}}{\dim V}E_{ij}IE_{kl}=\frac{\delta_{ab}\delta_{jk}}{\dim V}E_{il},
\]
或者
\[
	\left(\pi_{ja},\pi_{kb}\right)=\frac{\delta_{ab}\delta_{jk}}{\dim V}.
\]

\para 令$\phi:G\to \mathrm{GL}(V)$是一个$G$的复表示,那么复值函数$\chi_\phi:G\xrightarrow{\phi}\mathrm{GL}(V)\xrightarrow{\tr}\cc:g\mapsto \tr(\phi(g))$被称为$\phi$的特征标。

对于紧Lie群的不可约幺正表示,我们有
\[
(\chi_\pi, \chi_\pi)=\sum_{i,j}(\pi_{ii},\pi_{jj})=\sum_{i,j}\frac{\delta_{ij}^2}{\dim V}=\sum_{i}\frac{1}{\dim V}=1,
\]
对于两个不可约的不等价表示,我们有$(\chi_\pi, \chi_\rho)=0$.

令$\hat{G}$是紧Lie群$G$的复不可约表示的等价类,那么由于是紧Lie群,有限维表示必然是完全可约的,就是说任何一个有限维表示能写作$\rho=\sum_{\pi\in\hat{G}}m(\rho,\pi)\pi$,其中$m(\rho,\pi)$是乘数,是一个自然数,就是说分解出来的等价的$\pi$各数。直接计算就可以得到$m(\rho,\pi)=(\chi_\rho,\chi_\pi)$,以及
\[
	(\chi_{\rho_1},\chi_{\rho_2})=\sum_{\pi\in\hat{G}}m(\rho_1,\pi)m(\rho_2,\pi).
\]

前面说过,紧Lie群的一个不可约的复有限维表示$\pi$满足$(\chi_\pi, \chi_\pi)=1$,反过来,如果$(\chi_\pi, \chi_\pi)=1$,则$(\chi_{\pi},\chi_{\pi})=\sum_{\rho\in\hat{G}}m(\rho,\pi)^2=1$,因此存在一个$\rho\in\hat{G}$使得$m(\rho,\pi)=1$,而其他的$\pi\in\hat{G}$都有$m(\rho,\psi)=0$.这样我们就得到了:

\para 一个复有限维表示$\pi$是不可约的当且仅当$(\chi_{\pi},\chi_{\pi})=1$.

\pro 如果$G$是一个紧Lie群,那么两个表示$\pi_1$, $\pi_2$等价当且仅当$\chi_{\pi_1}=\chi_{\pi_2}$.

实际上,假若$\pi_1$和$\pi_2$等价,那么存在$A$使得
$A\pi_1(g)=\pi_2(g)A$,于是$\tr(\pi_1(g))=\tr(A^{-1}\pi_2(g)A)=\tr(\pi_2(g))$.反过来,如果两个表示$\pi_1$, $\pi_2$特征标相等,则$	m(\pi_1,\pi)=(\chi_{\pi_1},\chi_\pi)=(\chi_{\pi_2},\chi_\pi)=m(\pi_2,\pi)$,对任意的$\pi\in\hat{G}$都成立,因此$\pi_1$, $\pi_2$等价。

\section{Morphisms of Lie Group}

Lie群很容易构成一个范畴,两个Lie群的态射只要定义成光滑的群同态就好了。比如对一般线性群而言,行列式就是一个显然的Lie群同态。这样子做之后,Lie群既是光滑流形范畴的子范畴,也是群范畴的子范畴。

Lie群因为本身就是一个群,所以他的很多整体性质都来自于单位元附近。对于拓扑性质,由于左乘是一个同胚,所以如果$U$是一个开集,那么$gU=\{gh:h\in U\}$也是一个开集。

\para 对于连通Lie群,设$U$是单位元的任意一个邻域,则$G=\bigcup_{n\geq 1}U^n$,其中$U^k=\{g_1\cdots g_k:g_i\in U,\, 1\leq i \leq k\}$是开集。

令$V=U\cap U^{-1}$,其中$U^{-1}=\{g^{-1}:g\in U\}$,显然$V=V^{-1}\subset U$以及$H=\bigcup_{n\geq 1}V^n\subset \bigcup_{n\geq 1}U^n$,而且$H$还是一个子群。下面我们只要证明$H=\bigcup_{n\geq 1}V^n$既是开的也是闭的,那么连通性自然给出了结论。

他是开的,如果$\sigma\in V^k$,那么$\sigma V\in V^{k+1}\subset H$就是他的一个开邻域。他是闭的,因为每一个$\sigma H$都是开的,于是$H=G-\bigcup_{\sigma\notin H}\sigma H$是一个闭集。

\para 设$\varphi:G\to H$是一个Lie群同态,因为$\varphi\circ l_g=l_{\varphi(g)}\circ \varphi$,所以在点$e$的切映射满足$\varphi_{*g}\circ (l_g)_{*e}=(l_{\varphi(g)})_{*e}\circ \varphi_{*e}$,因为左移是同胚,所以$\rank_e \varphi=\rank_g \varphi$.

上面意味着Lie群同态的秩都是常数,此时我们可以对$\ker \varphi=\varphi^{-1}(e)$利用\theoref{ranktheo},因为他一定非空,所以$\ker \varphi$是$G$的一个正则子流形,维度等于$\dim G-\rank \varphi$.

\para 因此单的Lie群同态必定为浸入,Lie群同构必为同胚,其逆也光滑。

\para 设$\varphi:H\to G$是Lie群同态,如果$\varphi$是单的,则$H$被称为$G$的Lie子群,如果$\varphi$是嵌入,则$H$被称为$G$的闭Lie子群。显然,上面谈论的Lie群同态的核是一个闭Lie子群。

闭Lie子群作为集合是闭集,这里就略去证明了。

\para 如果$\varphi:G\to H$是Lie群同态,且$\varphi_{*e}$是满射,由于Lie群同态常秩,所以$\varphi_{*g}$在每一个$g$上都是满射。由反函数定理,他在每一个局部都可以写作投影,投影将开集映射成开集,所以此时$\varphi$也将开集映射到开集,即$\varphi$是一个开映射,

特别地,如果$H$还是连通的,那么因为$\varphi$是一个开映射,所以存在一个$e_H$的邻域$V\subset H$使得$V\subset \varphi(H)$,那么$H=\bigcup_{n\geq 1}V^n\subset \varphi(G)$,因此$\varphi$就是一个满射。所以,对Lie群而言,局部满就可以说明整体满。

\para 设$M$是底空间为$B$的一个纤维丛,他的纤维被赋予离散拓扑,此时,$M$被称为$B$的一个覆叠空间,投影$\pi$现在被称为覆叠映射。一般来说,我们要求覆叠空间是连通的,如果覆叠空间还是单连通的,则被称为万有覆叠空间。

\para Lie群的覆叠空间有很自然的Lie群结构。

\section{Relation between Lie Group and Lie Algebra}

下面陈述Lie群和Lie代数的联系,有些证明是朴实的,有些证明是困难的,然而我都是略去了。首先给出一般的Lie代数定义。

\para 设有一个实数域上面的矢量空间$V$,他上面赋予了一个双线性映射$[\bullet,\bullet]:V\times V\to V$,且满足

\no{1} $[X,Y]=-[Y,X]$,

\no{2} $[X,[Y,Z]]+[Y,[Z,X]]+[Z,[X,Y]]=0$.

则此时$V$以及上面的双线性映射构成一个代数,即Lie代数。

一个古典的例子,$(\rr^3,\times)$构成一个Lie代数,其中$\times$是矢量的叉乘。

\para 如果对于$V$的一个子集$H$,成立$[H,H]\subset H$,则$H$就被称为Lie代数的子代数。

\theo 设$H$是$G$的Lie子群,那么$\mathfrak{h}$是$\lag$的Lie子代数。

下面这个定理来自Ado,就像流形中的Whitney嵌入定理一样,告诉我们,对于有限维Lie代数,其实我们只需要考虑$\mathfrak{gl}(n,\rr)$的子代数就可以了。(记得Arnold用Whitney嵌入定理来说“根本没有抽象的流形”。)

\theo 任何$\rr$上的有限维Lie代数总可以同构于$\mathfrak{gl}(n,\rr)$的一个子代数,$n$为一个足够大的整数。

考虑了子代数,现在来看看两个Lie群和Lie代数的联系。

\para 两个Lie代数$\lag_1$和$\lag_2$间的同态$f$首先是一个线性映射,其次满足
\[
f([a,b]_1)=[fa,fb]_2.
\]

\pro 设$\varphi:G_1\to G_2$是Lie群间的群同态(同构),则其诱导了Lie代数间的同态(同构)$\varphi_{*e}:\lag_1\to \lag_2$.

上面对于微分同胚我们已经证明了$\varphi_{*e}[u,v]=[\varphi_{*e}u,\varphi_{*e}v]$,但Lie群之间的群同态使得我们可以做得更好。

让$u,v\in\lag_1$,那么我们拓展到$G_1$上的对应的左不变矢量场$X,Y$上。因为$\varphi$是一个群同态,故$\varphi\circ l_g=l_{\varphi(g)}\circ \varphi$.使用这个关系,
\[
	\varphi_{*}(X_g)=\varphi_{*}((l_g)_*u)=(l_{\varphi(g)})_*(\varphi_{*}(u))=(\varphi_{*}X)_{\varphi(g)}.
\]
设$f$是$G_2$上任意的可微实函数,那么
\[
\varphi_{*}(X_g)f=X_g(f\circ \varphi),
\]
或者使用$\varphi_{*g}(X_g)=(\varphi_{*}X)_{\varphi(g)}$写作
\[
(\varphi_{*}(X)f)\circ \varphi=X(f\circ \varphi),
\]
同理有$(\varphi_{*}(Y)f)\circ \varphi=Y(f\circ \varphi)$.所以
\[
	\begin{split}
		([\varphi_*(X),\varphi_*(Y)]f)\circ \varphi&=((\varphi_*(X)\circ\varphi_*(Y)-\varphi_*(Y)\circ\varphi_*(X))f)\circ \varphi\\
		&=X((\varphi_*(Y)f)\circ \varphi)-Y((\varphi_*(X)f)\circ \varphi)\\
		&=X(Y(f\circ \varphi))-Y(X(f\circ \varphi))\\
		&=[X,Y](f\circ \varphi).
	\end{split}
\]
这就是说对任意的$g$都有
\[
	\varphi_{*g}([X,Y]_g)=[\varphi_{*g}(X_g),\varphi_{*g}(Y_g)]
\]
特别地$g=e$,则
\[
	\varphi_{*e}[u,v]=[\varphi_{*e}u,\varphi_{*e}v].
\]

上面基本都是从Lie群去得到Lie代数,而且是唯一确定的。那么反过来,我们是否可以从Lie代数得到Lie群,如果可以得到,又能确定到哪种程度?这些内容是下面两个定理的内容。

\theo 我们有一个任意的有限维Lie代数$\lag$,那么在同构意义下有唯一的单连通Lie群$G$,他的Lie代数就是$\lag$.

如果两个Lie群有着相同的有限维Lie代数,我们虽然不能判断他们是同构的,但是我们可以找到同构的单连通Lie群。而下面的定义告诉我们,这个单连通Lie群就是原Lie群的万有覆叠空间。

\theo 一个连通Lie群$G$,记$G'$为其万有覆叠空间,覆叠映射为$\pi:G'\to G$.对于任意的选择$e'\in \pi^{−1}(e)$,总有唯一的$G'$上的Lie群结构使得$e'$是单位元且$\pi$是群同态。

本节的最后,讲一下Lie群的结构方程,我们将Maurer-Cartan形式求一下外微分,可以得到
\[
\dd \omega_G(X,Y)=X(\omega_G(Y))-Y(\omega_G(X))-\omega_G([X,Y]).
\]
假设,$X$和$Y$是左不变矢量场,则$\omega_G(Y)$和$\omega_G(X)$都是常数,那么得到
\[
\dd \omega_G(X,Y)+\omega_G([X,Y])=0.
\]
因为$[X,Y]$在点$e$的值是在$\lag$里面的,所以这一项也等于$\omega_G([X,Y])=[\omega_G(X),\omega_G(Y)]$,最后就得到了Lie群的结构方程
\[
\dd \omega_G(X,Y)+[\omega_G(X),\omega_G(Y)]=0.
\]

由于这个方程是微分过的结果,可以看成是微分方程,他确定了Lie群的局部结构。虽然我们的证明是选取了两个左不变矢量场,但是每一个矢量场在在局部都可以变成左不变矢量场的限制,所以我们的方程总是成立的。如果群是Abel群,那么方程的第二项为0,即$\dd \omega_G=0$.

\section{More on Algebraic Structure}

这节稍稍谈谈Lie代数的代数构造,很重要的一点就是其为矢量空间,这样我们就可以谈直和直积复化什么的。

我们这里感兴趣的是群表示是一种矢量空间为$\lag$的表示。这需要从伴随作用开始。

Lie群$G$的伴随和他的Lie代数$\lag$的联系来自于伴随$\mathbf{Ad}(g):h\mapsto ghg^{-1}$在单位元上的导数$\mathrm{Ad}_g=\mathbf{Ad}(g)_*:T_eG\to T_eG$,但注意到Lie代数$\lag$就是Lie群在单位元的切空间$T_eG$,所以$\mathrm{Ad}_g\in \mathrm{GL}(\lag)$。我们将$\mathrm{Ad}:G\to \mathrm{GL}(\lag)$称为Lie群的伴随表示。

\para 对于伴随,我们有

\no{1} $\mathrm{Ad}_g:\lag\to \lag$是一个Lie代数间的同构,而$\mathrm{Ad}:G\to \mathrm{GL}(\lag)$是一个Lie群间的同态。

\no{2} 如果$X$是$G$上的左不变矢量场,那么$\mathrm{Ad}_gX$对于任意$g\in G$也是。

\no{3} 记右作用为$r$,那么$r_g^*\omega_G=\mathrm{Ad}(g^{-1})\omega_G$.


第一个是显然的。第二个首先注意到左作用和右作用是可交换的,因此他们的导数也是可以交换的,那么:
\[
	(l_h)_*(\mathrm{Ad}_gX)=(l_h)_*(l_g)_*(r_{g^{-1}})_*X=(r_{g^{-1}})_*X=(r_{g^{-1}})_*(l_g)_*X=\mathrm{Ad}_gX.
\]

第三个设$v\in T_hG$,因此$(r_g)_*v\in T_{hg}G$,于是
\[
	(r_g)^*\omega_G(v)=\omega_G((r_g)_*v)=(l_{{hg}^{-1}})_*(r_g)_*
	=(l_{{g}^{-1}})_*(r_g)_*(l_{{h}^{-1}})_*v=\mathrm{Ad}(g^{-1})\omega_G.
\]

\para 设$\lag$是一个Lie代数,那么

\no{1} $\mathrm{Aut}_{\mathrm{Lie}}(\lag)=\{T\in \mathrm{GL}(\lag)\,|\,T[u,v]=[Tu,Tv],\,\forall u,v\in\lag\}$

\no{2} $\mathfrak{gl}_{\mathrm{Lie}}(\lag)=\{T\in \mathfrak{gl}(\lag)\,|\,T[u,v]=[Tu,v]+[u,Tv],\,\forall u,v\in\lag\}$

\para 如下陈述成立:

\no{1} $\mathrm{Aut}_{\mathrm{Lie}}(\lag)$是一个Lie群。

\no{2} $\mathrm{Aut}_{\mathrm{Lie}}(\lag)$的Lie代数是$\mathfrak{gl}_{\mathrm{Lie}}(\lag)$.

\no{3} 令$G$的Lie代数为$\lag$,那么$\ad(u)v=[u,v]$是$\mathrm{Ad}:G\to \mathrm{GL}_{\mathrm{Lie}}(\lag)$在$e$的导数$\ad=\mathrm{Ad}_{*e}:\lag\to \mathfrak{gl}_{\mathrm{Lie}}$.

在第三点中,我们看到了曾经在Jacobi恒等式那边指出的导子,可以看到,这确确实实就是一个导数。

我们拿一般线性群举个例子,前面已经计算过了$(l_g)_*=l_g$,那么同样$(r_g)_*=r_g$,所以
\[
	\mathrm{Ad}_g=(l_g)_*(r_{g^{-1}})_*=l_gr_{g^{-1}}.
\]
那么
\[
	\mathrm{Ad}_g(v)=(l_g)_*(r_{g^{-1}})_*v=l_gr_{g^{-1}}v=gvg^{-1}.
\]
我们现在求他的Lie代数,考虑$u,v\in \lag$,我们令$u(t)$是一个以$u$为初速度的单参子群,那么我们有
\[
\frac{\dd}{\dd t}(\mathrm{Ad}_{u(t)}(v))=u'(t)vu^{-1}(t)+u(t)v(u^{-1}(t))'=u'(t)vu^{-1}(t)-u(t)vu^{-1}(t)u'(t)u^{-1}(t).
\]
然后令$t=0$,那么$u(0)=u^{-1}(0)=I$,而$u'(0)=u$,那么就得到了单位元处的切矢量,也就是Lie代数
\[
\ad(u)v=uv-vu=[u,v].
\]

类似的手段譬如
\[
T(t)[u,v]=[T(t)u,T(t)v],
\]
求个导,然后在$t=0$处的值为
\[
T'(0)[u,v]=[T'(0)u,T(0)v]+[T(0)u,T'(0)v],
\]
注意到$T(0)$是恒等变换,而$T'(0)$就是我们需要的Lie代数$B$,他需要满足的关系就是
\[
B[u,v]=[Bu,v]+[u,Bv],
\]
其显然是Lie代数上面的一个导子。

\para 令$\lag$是一个Lie代数,那么他的有限维表示$(\pi,V)$就是一个映射$\pi:\lag\to\mathfrak{gl}(V)$.所以说$\ad$就是一个Lie代数的表示。

一个Lie群的表示可以引出他的Lie上面的一个表示如下:
\[
	\pi_*(X)=\left.\frac{\dd}{\dd t}\pi(e^{tX})\right|_{t=0},
\]
或者
\[
	\pi(e^{tX})=e^{\pi_*(X)}.
\]

\pro $\pi$是不可约的当且仅当$\pi_*$是不可约的;$\pi$是完全可约的当且仅当$\pi_*$是完全可约的;两个$\pi$和$\eta$是等价的当且仅当$\pi_*$和$\eta_*$是等价的。

对两个Lie群的表示的直积求导,就得到了Lie代数表示的直积形式应该满足:
\[
	\left.\frac{\dd}{\dd t}(\pi(g)\otimes \eta(g))\right|_{t=0}=\pi_*(g)\otimes I+I\otimes \eta_*(g)=(\pi_*\otimes I+I\otimes \eta_*)(g),
\]
我们就将其作为Lie代数表示直积的定义。

迄今为止,我们都只谈论了实数域上面的Lie代数,当然,我们可以直接拓展到复数域上面去,但是复化的手段也是常用的。

\para 如果$V$是有限维实的矢量空间,那么所有$v_1+iv_2$所构成的空间被称为他的复化,记做$V_\cc$,其中$v_1,v_2\in V$.如果我们再定义
\[
i(v_1+iv_2)=-v_2+iv_1,
\]
那么$V_\cc$就是有限维的复矢量空间。

由于Lie代数也是矢量空间,我们可以对其复化,显然,我们希望复化后对易子满足
\[
[X_1+iX_2,Y_1+iY_2]=([X_1,Y_1]-[X_2,Y_2])+i([X_1,Y_2]+[X_2,Y_1]),
\]
也就是说对易子对$i$也是线性的。可以直接验证,我们这样定义的确实是对易子且是唯一的。

因为Lie代数是矢量空间,所以作为矢量空间,两个Lie代数可以直积,如果有两个Lie代数$\lag_1$和$\lag_2$,我们在$\lag_1\otimes\lag_2$如下定义交换子:
\[
[a_1\otimes b_1,a_2\otimes b_2]=[a_1,b_1]\otimes [a_2,b_2],
\]
那么很容易看到$\lag_1\otimes\lag_2$也变成了Lie代数。

同样,两个Lie代数也可以直和。如果有两个Lie代数$\lag_1$和$\lag_2$,那么直和的$\lag_1\oplus\lag_2$上的交换子写作
\[
	[(X_1,X_2),(Y_1,Y_2)]=([X_1,Y_1],[X_2,Y_2]).
\]

上面的直积直和的Lie代数结构的验证,尤其是Jacobi恒等式的验证就略去了。谈了直积和直和,我们就可以谈论Lie代数的分类和分解。

\para 令$\lag$是一个实(复)Lie代数,如果他的实(复)子代数$\mathfrak{h}$满足对所有的$X\in \lag,H \in \mathfrak{h}$都有$[X,H]\in \mathfrak{h}$,则称$\mathfrak{h}$是$\lag$的一个理想。\footnote{可以搜一搜ring和ideal的笑话。}

一个代数$\lag$显然有两个理想,一个是$0$一个是$\lag$,我们称这两个为平凡理想。

\para 一个实(复)Lie代数$\lag$如果没有非平凡理想,则称其为不可分解的。如果一个Lie代数的维度大于$1$且是不可分解的,则称其为单的。

从定义来看,不可分解但非单的Lie代数只可能是一维的。对于一维Lie代数来说,他没有非平凡的子代数,其中任意两个元素的交换子为0,从而可以看到他是不可分解的。

我们可以将有限群和有限维Lie代数类比。Lie代数的子代数就对应有限群的子群,Lie代数的理想就对应有限群的正规子群,一维Lie代数就对应素数阶的循环群(他是没有子群的,这个很简单就可以证明,使用Lagrange公式还可以得知素数阶的有限群一定是循环群)。

\para 一个实(复)Lie代数$\lag$如果同构于不可分解Lie代数的直和,那么称呼其为可约的。如果其同构于单Lie代数的直和,就称呼其为半单的。

下面的定理连接了紧矩阵Lie群以及半单Lie代数。

\theo 一个复Lie代数是半单的当且仅当他同构于一个单连通的紧矩阵Lie群的Lie代数的复化。

我们已经做过了从实Lie代数复化得到一个复Lie代数,那么自然地,我们可以问反问题,对一个复Lie代数是否可以寻找一个实Lie代数,那个实Lie代数的复化就是原本的复Lie代数。

\para 如果$\lag$是一个复的半单Lie代数,那么$\lag$的紧实形式(compact real form)是$\lag$的一个子代数$\mathfrak{l}$使得对任意的$X\in\lag$都可以找到两个$X_1,X_2\in\mathfrak{l}$满足$X=X_1+iX_2$.因此,存在一个单连通的紧矩阵Lie群$K_1$使得$K_1$的Lie代数同构于$\lag$的紧实形式。

上面一个定理告诉我们上面这个反问题在半单Lie代数的情况下始终是有解的。

\pro 如果$\lag$是一个实的Lie代数,那么他是半单的当且仅当他的复化$\lag_\cc$是半单的。

从这个可以推知,紧的单连通矩阵Lie群的实Lie代数是半单的。当然,反过来,不是任何半单实Lie代数都可以找到紧的单连通矩阵Lie群。

\para 如果$\lag$是一个复的半单Lie代数,那么$\lag$的一个子空间$\mathfrak{h}$被称为$\lag$的Cartan子代数,如果满足:

\no{1} 对于任意的两个元素$H_1,H_2\in\mathfrak{h}$,都有$[H_1,H_2]=0$;

\no{2} 对于任意的$X\in \lag$,如果有$[X,H]=0$对全部$H\in\mathfrak{h}$都成立,则$X\in \mathfrak{h}$.

\no{3} 对于全部$H\in \mathfrak{h}$,$\ad(H)$作为Lie代数的表示是完全可约的。

Cartan子代数的维度称为半单Lie代数的秩。

条件1说明了Cartan子代数是交换子代数,然后条件2就说明这个极大的交换子代数。

\pro 令$\lag$是一个复的半单Lie代数,令$\mathfrak{l}$是$\lag$的紧实形式,再令$\mathfrak{t}$是$\mathfrak{l}$任意的极大交换子代数,定义$\mathfrak{h}\in\lag$为$\mathfrak{h}=\mathfrak{t}+i\mathfrak{t}$.然后,$\mathfrak{h}$就是$\lag$的Cartan子代数。

我们以后下面要谈论Lie群和Lie代数的表示,但是却只限制在紧Lie群,特别是紧的矩阵群上面,从而根据上面的定理,我们也只要去研究半单Lie代数就可以了。

\section{Matrix Group}
这一节限制在矩阵群上面,这节将给出很多可计算的例子,先来求一些矩阵群的Lie代数。

\theo 任意的矩阵$M\in\mathrm{GL}(n,\cc)$都可以写作$e^X$的形式,其中$X\in\mathrm{M}(n,\cc)$。

这个直接从矩阵幂那一套证明并不那么方便,但是直接从Lie代数作为单位元附近的Lie群与单参子群的对应关系,这个结论就是比清晰的了。这个定理也可以用来说明,一般线性群的Lie代数就是整个方阵构成的集合。

\para 利用$\mathrm{Ad}_D(A)=DAD^{-1}$,我们有$e^{DAD^{-1}}=De^{A}D^{-1}$.这当然也可以直接用矩阵指数映射的展开来证明。

\pro $\det e^A=e^{\tr(A)}$.

这个的证明可以将矩阵分解为幂零的和可对角化的两个矩阵的乘积$A=SN$,因为$S$和$N$可交换,所以
\[
	\det \left(e^A\right)=\det\left(e^S\right)\det\left(e^N\right).
\]
如果可以对角化,那么直接对角化为
\[
	e^{S}=e^{D\Lambda D^{-1}}=De^{\Lambda} D^{-1},
\]
而$e^{\Lambda}$是可以直接计算的,即$\mathrm{diag}\left(e^{\lambda_1},\dots,e^{\lambda_n}\right)$.那么
\[
	\det \left(e^{S}\right)=\det \left(De^{\Lambda} D^{-1}\right)=\det \left(e^{\Lambda}\right)=e^{\sum_i \lambda_i}=e^{\tr(\Lambda)}=e^{\tr(S)}
\]
对于幂零矩阵来说,容易验证$\det e^N=1$,最后就可以得到任意矩阵都有
\[
	\det e^A=e^{\tr(A)}.
\]

有了上面的一些结论,我们来求特殊线性群$\mathrm{SL}(n,\cc)$的Lie代数$\mathfrak{sl}(n,\cc)$。如果$\det e^A=1$,那么$e^A\in \mathfrak{sl}(n,\cc)$,而$A \in \mathfrak{sl}(n,\cc)$,因为$e^{\tr(A)}=\det e^A=1$,
所以$\tr(A)=0$.这就是说$\mathfrak{sl}(n,\cc)$就是那么迹为0的矩阵的集合。

作为$\mathrm{SL}(n,\cc)$的子群,$\mathrm{SU}(n)$的元素满足$AA^\dag=I$,那么
\[
	\dd A A^\dag+A\dd A^\dag=0,
\]
或者整理成Maurer-Cartan形式的样子
\[
	A^{-1}\dd A +(A^{-1}\dd A)^\dag=0,
\]
这就是说,$\mathfrak{su}(n)$是由满足$B+B^\dag=0$的矩阵零迹矩阵$B$构成的集合。当然,这也可以用矩阵幂来做,我们略去了。

从矩阵幂的形式
\[
	e^A=\sum_{i=0}^n\frac{A^n}{n!},
\]
我们当然会去遐想,是否其他函数也有这样的幂函数展开?其中最有趣的展开莫过于$\log$了,因为他是$\exp$的反函数。就这样,我们定义
\[
	\log A=\sum_{m=1}^\infty (-1)^{m+1}\frac{(A-I)^m}{m}.
\]
当$\|A-I\|<1$的时候,这个幂级数显然是收敛且连续的。我们也确实可以证明在收敛的时候他和$\exp$是反函数,适当对角化(不能的话就用可对角的矩阵序列趋近)之后就可以直接计算验证。

前面说过$e^{X+Y}$在$X,Y$不对易的时候是一般不等于$e^Xe^Y$,这里我们举一个例子:
\[
X=\begin{pmatrix}
1&\\
&2\\
\end{pmatrix},
\quad
Y=\begin{pmatrix}
&1\\
2&\\
\end{pmatrix},
\]
直接计算就知道$[X,Y]\neq 0$.然后$e^X$是容易计算的,因为他是对角的
\[
e^X=\begin{pmatrix}
e&\\
&e^2\\
\end{pmatrix},
\]
后面的$Y$对角化之后也是容易的,
\[
Y=
\begin{pmatrix}
 -1/\sqrt{2} & 1/\sqrt{2} \\
 1 & 1 \\
\end{pmatrix}
\begin{pmatrix}
 -\sqrt{2} &  \\
  & \sqrt{2} \\
\end{pmatrix}
\begin{pmatrix}
 -1/\sqrt{2} & 1/\sqrt{2} \\
 1 & 1 \\
\end{pmatrix}^{-1}
\]
所以
\[
e^Y=
\begin{pmatrix}
 -1/\sqrt{2} & 1/\sqrt{2} \\
 1 & 1 \\
\end{pmatrix}
\begin{pmatrix}
 \exp(-\sqrt{2}) &  \\
  & \exp(\sqrt{2}) \\
\end{pmatrix}
\begin{pmatrix}
 -1/\sqrt{2} & 1/\sqrt{2} \\
 1 & 1 \\
\end{pmatrix}^{-1}
\]
那么
\[
e^Xe^Y=
\begin{pmatrix}
e&\\
&e^2\\
\end{pmatrix}
\begin{pmatrix}
\cosh \left(\sqrt{2}\right)&\sinh \left(\sqrt{2}\right)/\sqrt{2}\\
\sqrt{2} \sinh \left(\sqrt{2}\right)&\cosh \left(\sqrt{2}\right)\\
\end{pmatrix}.
\]
$e^{X+Y}$的计算不再重复了
\[
e^{X+Y}=\begin{pmatrix}
 2/3+e^3/3 & e^3/3-1/3 \\
 2 e^3/3-2/3 & 2 e^3/3-2/3 \\
\end{pmatrix}.
\]
所以$e^{X+Y}\neq e^Xe^Y$.

\pro
令$X,Y$都是$n\times n$的矩阵,则
\[
	e^{X+Y}=\lim_{m\to\infty}\left(e^{X/m}e^{Y/m}\right)^m.
\]

这证明挺简单的,所以略去了。下面这个公式更加复杂,因此证明也很复杂,以至于都需要一个名字来标记这个公式了。证明也略去了。

\theo Campbell-Baker-Hausdorff公式:
\[
\log\left(e^Xe^Y\right)=X+\int_0^1 \varphi\left(e^{\ad(X)}e^{t\ad(Y)}\right)(Y)\dd t,
\]
其中
\[
\varphi(z)=\frac{z\log z}{z-1}.
\]

幂级数展开$\varphi(z)$我们有
\[
	\varphi(z)=1+\sum_{n=2}^\infty (-1)^n\left(\frac{1}{n-1}-\frac{1}{n}\right)(z-1)^{n-1}.
\]
带入上面的公式,就可以得到漂亮的展开
\[
\log\left(e^Xe^Y\right)=X+Y+\frac{1}{2}[A,B]+\frac{1}{12}[A,[A,B]]-\frac{1}{12}[B,[A,B]]+\cdots.
\]

级数展开不是上面公式的重点,重点是展开后全是对易子$[\bullet,\bullet]$的形式,直接来自于积分里面的$\ad$.所以使用这个公式可以证明矩阵Lie群和Lie代数关系中的映射关系,因为Lie代数同态是可以分配进对易子的但是对于一般的$X$和$Y$的线性组合却是不可以的。

我们现在来对Lie代数进行展开,设Lie代数的基为$\{a_i\}$,那么
\[
[x,y]=\sum_{i,j}x_iy_j[a_i,a_j].
\]
但是因为交换子是封闭的,所以
\[
	[a_i,a_j]=\sum_k c_{kij}a_k.
\]
然后考虑伴随表示的展开
\[
	\ad(a_i)(a_j)=\sum_k \ad(a_i)_{kj}a_k.
\]
对于一般线性群的Lie代数来说$\ad(a_i)(a_j)=[a_i,a_j]$,这就是说
\[
	[a_i,a_j]=\sum_k \ad(a_i)_{kj}a_k.
\]
使用线性性,我们最后得到
\[
	[x,a_i]=\sum_k \ad(x)_{ki}a_k.
\]
这样,我们就可以求得$\ad(x)$的矩阵。

现在来考虑$\mathfrak{so}(3)$的Lie代数,因为$\mathfrak{so}(3)$是那些$3\times 3$的反对称矩阵构成的集合,我们取下面三个矩阵作为基
\[
	\eta_1=
		\begin{pmatrix}
			0&0&0\\
			0&0&-1\\
			0&1&0\\
		\end{pmatrix},\quad
	\eta_2=
		\begin{pmatrix}
			0&0&1\\
			0&0&0\\
			-1&0&0\\
		\end{pmatrix},\quad
	\eta_3=
		\begin{pmatrix}
			0&-1&0\\
			1&0&0\\
			0&0&0\\
		\end{pmatrix},
\]
然后可以计算对易子如下
\[
	[\eta_1,\eta_2]=\eta_3,\quad [\eta_1,\eta_3]=-\eta_2,\quad [\eta_2,\eta_3]=\eta_1.
\]

如果建立映射$\eta_i\mapsto e_i$,其中$e_i$就是$\rr^3$的标准基,而映射将$[\bullet,\bullet]$映射为叉乘,那么这就是一个Lie代数同构。

那么我们也可以计算得伴随表示为
\[
	\ad(\eta_1)=
		\begin{pmatrix}
			0&0&0\\
			0&0&-1\\
			0&1&0\\
		\end{pmatrix},\quad
	\ad(\eta_2)=
		\begin{pmatrix}
			0&0&1\\
			0&0&0\\
			-1&0&0\\
		\end{pmatrix},\quad
	\ad(\eta_3)=
		\begin{pmatrix}
			0&-1&0\\
			1&0&0\\
			0&0&0\\
		\end{pmatrix}.
\]
有趣的是,在这组基的选取下$\ad(\eta_i)=\eta_i$.

下面来看看$\mathfrak{sl}(2,\cc)$,他是所有二阶零迹矩阵构成的群,我们设他的基为
\[
h=\begin{pmatrix}
	-1&0\\
	0&1\\
\end{pmatrix},\quad
e=\begin{pmatrix}
	0&1\\
	0&0\\
\end{pmatrix},\quad
f=\begin{pmatrix}
	0&0\\
	1&0\\
\end{pmatrix},
\]
那么
\[
[h,e]=2e,\quad[h,f]=-2f,\quad[e,f]=h.
\]
所以
\[
	\ad(h)=
		\begin{pmatrix}
			0&2&0\\
			0&0&0\\
			0&0&-2\\
		\end{pmatrix},\quad
	\ad(e)=
		\begin{pmatrix}
			0&0&1\\
			-2&0&0\\
			0&0&0\\
		\end{pmatrix},\quad
	\ad(f)=
		\begin{pmatrix}
			0&-1&0\\
			0&0&0\\
			2&0&0\\
		\end{pmatrix}.
\]

\para
定义Killing形式为$(x,y)_K$如下
\[
	(x,y)_K=\tr(\ad(x)\ad(y)),
\]
其中$x,y$是任意的Lie代数元素。

在选取基的形式下,我们可以得到矩阵$K_{ij}=(a_i,a_j)_K$.

\begin{theo}
Killing形式是双线性的,且

\no{1} 
对于任意的Lie代数自同构$\varphi$我们对任意的$x,y$都有$(\varphi x,\varphi y)_K=(x,y)_K$,

\no{2}
对于任意的$x,y,z$有$([x,y],z)_K=(x,[y,z])_K$,

\no{3}
如果$\mathfrak{h}$是$\lag$的理想,那么在$\mathfrak{h}$上的Killing形式$(\bullet,\bullet)_{K_\mathfrak{h}}$和原本的Killing形式$(\bullet,\bullet)_{K}$对于所有$x,y\in \mathfrak{h}$满足$(x,y)_{K_\mathfrak{h}}=(x,y)_K$.
\end{theo}

对于第二点,注意到$\ad([x,y])=\ad(x)\ad(y)-\ad(y)\ad(x)$,
\[
	\begin{split}
		([x,y],z)_K-(x,[y,z])_K&=\tr\{\ad([x,y])\ad(z)-\ad(x)\ad([y,z])\}\\
		&=\tr\{\ad(x)\ad(y)\ad(z)-\ad(y)\ad(z)\ad(x)\}\\
		&=0.
	\end{split}
\]

\begin{theo}
一个Lie代数是半单的当且仅当他没有一个非平凡的交换理想\footnote{交换理想首先是一个理想,然后也是一个交换子代数。}。
\end{theo}
这个结论经常也被当成定义,可以看到这种半单的定义自动抛弃了一维理想的存在,因为一维理想一定是交换的。

\begin{theo}
一个Lie代数是半单的,当且仅当他的Killing形式是非退化的,即矩阵$K_{ij}=(a_i,a_j)_K$是非退化的,或者对任意的$a$都成立$(a,b)_K=0$的话能推出$b=0$。
\end{theo}

靠这个定理,我们来看看$\mathfrak{so}(3)$是否是半单的。容易计算得他的Killing形式为
\[
	K_{ij}=(\eta_i,\eta_j)_K=\tr(\eta_i\eta_j)=-2\delta_{ij}.
\]
当然是非退化的,所以$\mathfrak{so}(3)$是半单Lie代数。同理$\mathfrak{sl}(2)$也是半单Lie代数。

\section{$\mathfrak{sl}(2,\cc)$, $\mathfrak{so}(3)$ and $\mathfrak{su}(2)$}
在分析一般的半单Lie代数之前,我们先来看看几个比较简单的Lie代数$\mathfrak{sl}(2,\cc)$, $\mathfrak{so}(3)$和$\mathfrak{su}(2)$,他们之间存在着紧密的联系。

前面已经证明过$\mathfrak{sl}(2,\cc)$, $\mathfrak{so}(3)$的基和相互的对易关系有
\[
[h,e]=2e,\quad[h,f]=-2f,\quad[e,f]=h,
\]
\[
	[\eta_1,\eta_2]=\eta_3,\quad [\eta_1,\eta_3]=-\eta_2,\quad [\eta_2,\eta_3]=\eta_1.
\]

现在我们来看$\mathfrak{su}(2)$的表现,他是所有满足$B+B^\dag=0$的复二阶零迹矩阵$B$的集合。我们选如下三个矩阵作为基:
\[
\mu_1=\frac{1}{2}\begin{pmatrix}
	0&i\\
	i&0\\
\end{pmatrix},\quad
\mu_2=\frac{1}{2}\begin{pmatrix}
	0&-1\\
	1&0\\
\end{pmatrix},\quad
\mu_3=\frac{1}{2}\begin{pmatrix}
	i&0\\
	0&-i\\
\end{pmatrix}.
\]
容易验证
\[
[\mu_1,\mu_2]=\mu_3,\quad [\mu_1,\mu_3]=-\mu_2,\quad [\mu_2,\mu_3]=\mu_1.
\]
这和$\mathfrak{so}(3)$的对易关系一模一样,于是我们可以引入$\rr$-线性映射建立两者作为实Lie代数的同构,也就是$\mathfrak{so}(3)\cong \mathfrak{su}(2)$.

很容易证明$\mathfrak{sl}(2,\rr)$的复化即$\mathfrak{sl}(2,\cc)$,这就是说$\mathfrak{sl}(2,\rr)_\cc=\mathfrak{sl}(2,\cc)$.为了分析$\mathfrak{sl}(2,\cc)$的结构,我们来看
$\mathrm{SL}(2,\mathbb{C})$的结构。

任何一个复可逆$2\times 2$矩阵都可以唯一分解(极分解)为
\[
\lambda=ue^{h}
\]
其中$u$幺正而$h$是Hermite矩阵。现在假如$\det \lambda=1$,则
\[
\det(u)e^{\tr(h)}=1
\]
于是$\det(u)=1$而$\tr(h)=0$.前者的一般形式为
\[
u=
\begin{pmatrix}
a+ib&c+id\\
-c+id&a-ib
\end{pmatrix},
\]
且满足$a^2+b^2+c^2+d^2=1$,因此其拓扑上等价为3-球面$\mathbb{S}^3$.而前者的一般形式为
\[
h=\begin{pmatrix}
e&f-ig\\
f+ig&-e
\end{pmatrix},
\]
拓扑上等价于$\rr^4$,因此$\mathrm{SL}(2,\cc)$在拓扑上等价于$\rr^4\times \mathbb{S}^3$.当然拓扑上的结论在我们这里暂时没什么用。

这样来看$\mathrm{SL}(2,\cc)$的Lie代数$\mathfrak{sl}(2,\mathbb{C})$,在$\mathrm{SL}(2,\cc)$的极分解中,令$b=-ih$,则
\[
\tr(b)=0,\quad b^\dag+b=0,
\]
以及$u=e^a$有
\[
\tr(a)=0,\quad a^\dag+a=0,
\]
所以$a,b\in\mathfrak{su}(2)$,且$\lambda=e^{a+ib}$.

注意到任取一个实数$t$和$a\in\mathfrak{su}(2)$,还有$ta \in\mathfrak{su}(2)$,所以任意的一个$\lambda \in \mathrm{SL}(2,\cc)$都可以写成
\[
\lambda=e^{ta+itb}
\]
他在$t=0$的导数$a+ib$就构成$\mathfrak{sl}(2,\cc)$,那么任意的$c\in \mathfrak{sl}(2,\cc)$都可以写成
\[
c=a+ib,
\]
其中$a,b\in\mathfrak{su}(2)$,这就是说$\mathfrak{sl}(2,\cc)$是$\mathfrak{su}(2)$的复化。

单纯从Lie代数来看,我们在$\mathfrak{su}(2)_\cc$中引入$L_n=i\mu_n$,则
\[
	[L_1,L_2]=iL_3,\quad [L_1,L_3]=-iL_2,\quad [L_2,L_3]=iL_1.
\]
再引入$L_\pm=L_1\pm iL_2$,则
\[
	[L_+,L_-]=2L_3,\quad [L_3,L_+]=L_+,\quad [L_3,L_-]=-L_-.
\]
我们令$h'=2h,e'=2e,f'=2f$,则$\mathfrak{sl}(2,\cc)$的三个基的对易关系变成
\[
[e',f']=2h',\quad[h',e']=e',\quad[h',f']=-f'.
\]
可见一模一样。

这样,三个Lie代数之间的关系就清楚了
\[
	\mathfrak{su}(2)\cong\mathfrak{so}(3),\quad \mathfrak{su}(2)_\cc\cong \mathfrak{sl}(2,\mathbb{C}) \cong\mathfrak{sl}(2,\rr)_\cc.
\]

现在来看$\mathfrak{sl}(2,\mathbb{C})$的有限维不可约表示,每一个Lie代数的元素$a$都变成了有限维矢量空间$V$上面的线性映射$\pi(a)$,复数域的代数完备性可以推知$\pi(h)$有一个特征值,即
\[
	\pi(h)v=\lambda v.
\]
那么
\[
	\pi(h)\pi(e)v=[\pi(h),\pi(e)]v+\pi(e)\pi(h)v=(\lambda+2)\pi(e)v.
\]
所以$\pi(e)v$也是$\pi(h)$的本征矢量,本征值是$\lambda+2$,同理$\pi(f)v$也是$\pi(h)$的本征矢量,本征值是$\lambda-2$.

我们反复作用$\pi(e)$和$\pi(f)$到$v$上就可以得到
\[
	\pi(h)\pi(e)^nv=(\lambda+2n)\pi(e)^nv,
\]
所以,要么$\pi(e)^nv$也是一个本征矢,本征值为$\lambda+2n$,或者$\pi(e)^nv=0$。由于$V$是有限维的,我们不可能有着无穷多个不同的本征值,因此存在一个$N\geq 0$使得
\[
	\pi(e)^Nv\neq 0,\quad \pi(e)^{N+1}v=0
\]
这就是说存在一个$u_0$使得
\[
	\pi(h)u_0=\lambda u_0,\quad \pi(e)u_0=0.
\]
$\lambda$是$\pi(h)$的最大的本征值。

我们再定义$u_k=\pi(f)^ku_0$,那么
\[
	\pi(h)u_k=(\lambda-2k) u_k,
\]
也不可能无限地进行下去。就是说存在一个$m\in \mathbb{N}$使得$k\leq m$满足$u_k\neq 0$但$u_{m+1}=0$.

使用归纳法和对易关系$[\pi(e),\pi(f)]=\pi(h)$可以算得
\[
\pi(e)u_k=(k\lambda -k(k-1)) u_{k-1}\quad (k>0).
\]
假如$u_{m+1}=0$,那么
\[
0=\pi(e)u_{m+1}=((m+1)\lambda -m(m+1)) u_{m}=(m+1)(\lambda-m)u_{m}.
\]
这就是说$\lambda=m$.那么一个本征值为正整数,其他的本征值可以通过$\lambda-2n$得到,所以也是整数,这就推出了:

\pro
$\mathfrak{sl}(2,\mathbb{C})$的$m+1$维不可约表示中$\pi(h)$的本征值都是整数,且可以对角化为$\mathrm{diag}(m,m-2,\cdots,-m+2,m)$。

到上面为止,唯一留下的就是要证明对任意的$m$上面的表示都是不可约的,而且确实是$\mathfrak{sl}(2,\mathbb{C})$的表示。为此,我们可以证明$\{u_0,u_1,\cdots,u_m\}$构成一组基,这是因为每一个$u_k$都是$\pi(h)$对应不同本征值的本征矢量,于是$\pi(f)u_k=u_{k+1}$保证了不可约性。而他是$\mathfrak{sl}(2,\mathbb{C})$的表示则是可以直接计算验证的,略去之。

将表示的矩阵写出来可能更加清晰,$\pi_m(h)=\mathrm{diag}(m,m-2,\cdots,-m+2,m)$和
\[
\pi_m(e)=\begin{pmatrix}
	0&m&&&\\
	&0&m-1&&\\
	&&\ddots&\ddots&\\
	&&&0&1\\
	&&&&0\\
\end{pmatrix},\quad
\pi_m(f)=\begin{pmatrix}
	0&&&&\\
	1&0&&&\\
	&2&\ddots&&\\
	&&\ddots&\ddots&\\
	&&&m&0\\
\end{pmatrix}.
\]

在物理上更习惯用整数或者半整数$j=m/2$来表示维度,并且会适当调整$\pi(h)$的系数,这就造成了角动量问题中的半整数的出现。

\pro $\mathfrak{sl}(2,\mathbb{C})$的$2j+1$维不可约表示中$\pi(h)$的本征值或者是整数或者是半整数,且可以对角化为$\mathrm{diag}(-j,-j+1,\cdots,j-1,j)$。

\section{Semi-simple Lie Algebras}

对于有限维半单复Lie代数$\lag$,我们可以找到其也是有限维的紧实形式。因为是半单的,所以这个紧实形式也同构于一个紧Lie群$H$的Lie代数$\mathfrak{h}$(直接就看成是了),所以$\lag=\mathfrak{h}+i\mathfrak{h}$,那么$\mathfrak{h}$上面存在着内积在表示$\mathrm{Ad}:H\to \mathrm{GL}(\mathfrak{h})$作用下不变,即
\[
	(e^{t\ad(h)}x,e^{t\ad(h)}y)=(x,y).
\]
其中$x,y\in\mathfrak{h}$,可以把这个内积推广到$\lag$内取复值的内积,让我们还是使用符号$(\bullet,\bullet)$来标记他,那么求导就有
\[
	(\ad(h)x,y)+(x,\ad(h)y)=0.
\]
所以表示$\ad$是反Hermit的,即
\[
	\ad(h)+\ad(h)^\dag=0.
\]
此时$i\ad(h)$就是Hermit的,因此根据有限维的谱定理,我们一定可以对角化$i\ad(h)$,也就是说可以对角化$\ad(h)$当$h\in \mathfrak{h}$,而且是对角元是纯虚的。特别地,如果$h$在$\mathfrak{h}$的极大交换子代数$\mathfrak{l}$里面,$\ad(h)$也是可以对角化的。因此如果$h_1,h_2\in\mathfrak{l}$,$\ad(h_1),\ad(h_2)$是可交换的,那么他们的线性组合$h=h_1+ih_2$对应的$\ad(h)$也是可以对角化的\footnote{这是因为可交换线性变换可以同时对角化。}。此时$\mathfrak{l}+i\mathfrak{l}$就是$\lag$的Cartan子代数,所以如果$h$在$\lag$的Cartan子代数里面,那么$\ad(h)$是可以对角化的。

我们可以分解有限维半单复Lie代数$\lag$为这样的两个子代数,其中一个是Cartan子代数$\mathfrak{h}$
\[
	\lag=\mathfrak{h}\oplus \lag'.
\]

令有限维半单复Lie代数$\lag$的秩为$l$,那么设Cartan子代数$\mathfrak{l}$的基为$\{h_1,h_2,\cdots,h_l\}$,因为$\ad(h_j):\lag'\to \lag'$且$h_j$在$\lag$的Cartan子代数里面,所以$\ad(h_j)$是可以对角化的,因此存在$\lag'$的基$\{v_1,v_2,\cdots,v_{n-l}\}$使得
\[
	\ad(h_j)v_k=[h_j,v_k]=\alpha_{k}(h_j)v_k,
\]
其中$\alpha_{k}(h_j)$就是$\ad(h_j):\lag'\to \lag'$的本征值。而Cartan子代数作为交换子代数,我们有交换子$[h_j,h_k]=0$成立。但Cartan子代数又是极大的交换子代数,所以对任意的$k$,我们总可以找到一个$j$使得$\alpha_{k}(h_j)\neq 0$,否则$[h_j,v_k]=0$就可以推出$v_k$在Cartan子代数$\mathfrak{h}$里面了。

设$h=\sum_{j=1}^l\mu_jh_j$是任意的Cartan子代数里面的元素,那么
\[
	\ad(h)v_k=[h,v_k]=\sum_{j=1}^l\mu_j\alpha_{k}(h_j)v_k,
\]
我们定义Cartan子代数$\mathfrak{h}$上的线性函数$\alpha_{k}:\mathfrak{h}\to\cc$如下
\[
\alpha_{k}(h)=\sum_{j=1}^l\mu_j\alpha_{k}(h_j),
\]
那么
\[
	\ad(h)v_k=[h,v_k]=\alpha_{k}(h)v_k.
\]
这样定义的线性函数被称为$\lag$的根,所有根的集合记做$\Delta$.

\para 对有限维半单复Lie代数$\lag$的一个根$\alpha$,我们将满足方程的
\[
[h,v_\alpha]=\alpha(h)v_\alpha
\]
矢量$v_\alpha$构成的集合称为对应于根$\alpha$的根子空间。

我们于是把有限维半单复Lie代数的分解更加细化为
\[
	\lag=\mathfrak{h}\oplus \bigoplus_{\alpha\in\Delta} \lag_\alpha.
\]
其中$\lag_\alpha$是诸根子空间。Cartan子代数的根也可以认为是恒为0的,那么$\mathfrak{h}$也可以认为是一个根子空间$\lag_0$。

两个根子空间可以进行对易子运算,即$[\lag_\alpha,\lag_\beta]$,设$a_\alpha\in\lag_\alpha,a_\beta\in\lag_\beta$,则
$[a_\alpha,a_\beta]$也是在某个根子空间里面的,为了求他的根,我们用对任意的$h\in \mathfrak{h}$求$\ad(h)[a_\alpha,a_\beta]$,因为$\ad(h)$是导子,所以
\[
	\begin{split}
		\ad(h)[a_\alpha,a_\beta]&=[\ad(h)a_\alpha,a_\beta]+[a_\alpha,\ad(h)a_\beta]\\
		&=\alpha(h)[a_\alpha,a_\beta]+[a_\alpha,\beta(h)a_\beta]\\
		&=(\alpha(h)+\beta(h))[a_\alpha,a_\beta]
	\end{split}
\]
因此,如果$\alpha+\beta\in\Delta$,则$\ad(h)[a_i,a_j]\in \lag_{\alpha+\beta}$,于是$[\lag_\alpha,\lag_\beta]\subset \lag_{\alpha+\beta}$,否则$[\lag_\alpha,\lag_\beta]=\{0\}$.

\theo 设$a_\alpha\in\lag_\alpha,a_\beta\in\lag_\beta$,那么如果$\alpha+\beta\neq 0$,则$(a_\alpha,a_\beta)_K=0$.

这个证明没什么难度,考虑$\ad(a_\alpha) \ad(a_\beta) a_\gamma$就可以了。特别地,我们考虑$h\in \mathfrak{h}$且$a\in \lag_\alpha$,则
\[
	(h,a_\alpha)_K=0.
\]
以此和Killing形式在$\mathfrak{g}$上非退化可以推知,Killing形式在$\mathfrak{h}$上也是非退化的。

那么我们就有可能用非退化的Killing形式来表示根,即对根$\alpha$存在$h_\alpha\in\mathfrak{h}$使得\footnote{使用类似定义对偶空间的方式。}
\[
	(h_\alpha,h)_K=\alpha(h),
\]
因之
\[
	h_{\alpha+\beta}=h_\alpha+h_\beta,
\]
因为Killing形式是对称的,所以$(h_\alpha,h_\beta)_K$可以推知$\alpha(h_\beta)=\beta(h_\alpha)$.

\para 我们定义$\langle \bullet,\bullet \rangle$如下:
\[
	\langle \alpha,\beta \rangle=(h_\alpha,h_\beta)_K.
\]

那么$\alpha(h_\beta)=\beta(h_\alpha)=\langle \alpha,\beta \rangle=\langle \beta,\alpha \rangle$.那么$\langle -\alpha,\beta \rangle=-\alpha(h_\beta)=-\langle \alpha,\beta \rangle$以及
\[
	[h_\beta,a_\alpha]=\langle \beta,\alpha \rangle a_\alpha.
\]

因为$\ad(h)$是对角的,且对角元素为$\{\alpha(h)\}$,其中$\alpha\in\Delta$,那么直接根据Killing形式的定义,就有
\[
	(h,h')_K=\sum_{\gamma \in \Delta}(\dim \lag_\gamma)\gamma(h)\gamma(h'),
\]
特别地,我们有
\[
	\langle \alpha,\beta \rangle=(h_\alpha,h_\beta)_K=\sum_{\gamma \in \Delta}(\dim \lag_\gamma)\gamma(h_\alpha)\gamma(h_\beta)=\sum_{\gamma \in \Delta}(\dim \lag_\gamma)\langle \gamma,\alpha\rangle\langle \gamma,\beta\rangle.
\]


\pro 如果$\alpha\in\Delta$,则$-\alpha\in\Delta$.

假若$-\alpha\notin\Delta$,那么对于任何$\beta\in\Delta$都有$\alpha+\beta\neq 0$,那么就是说,对于任何的$a\in \lag$都有$(a_\alpha,a)_K=0$,由Killing形式的非退化可以知道此时$a_\alpha=0$,这不可能。那么,这样也顺便推出了,对于任意的$a_\alpha\in\lag_{\alpha}$,一定存在一个$a_{-\alpha}\in\lag_{-\alpha}$使得$(a_\alpha,a_{-\alpha})_K\neq 0$.因为Killing形式是双线性的,调整系数我们可以使得$(a_\alpha,a_{-\alpha})_K$成为任意的复常数。


\pro 设$a_\alpha \in \lag_\alpha$和$a_{-\alpha} \in \lag_{-\alpha}$,则$[a_{\alpha},a_{-\alpha}]=(a_{\alpha},a_{-\alpha})_Kh_\alpha$.这也说明了,$[\lag_{\alpha},\lag_{-\alpha}]$是一维的,且由$h_\alpha$张成。


这个直接计算就是了,如果我们选$E_\alpha\in\lag_\alpha,F_\alpha\in\lag_{-\alpha}$满足$(E_\alpha,F_{\alpha})_K=2/\langle \alpha,\alpha \rangle$,再选
\[
	H_\alpha=\frac{2}{\langle \alpha,\alpha \rangle}h_\alpha
\]
那么计算对易关系
\[
	\begin{split}
	&[H_\alpha,E_\alpha]=\frac{2}{\langle \alpha,\alpha \rangle}[h_\alpha,E_\alpha]=\frac{2}{\langle \alpha,\alpha \rangle}\langle \alpha,\alpha \rangle E_\alpha=2E_\alpha,\\
	&[H_\alpha,F_\alpha]=\frac{2}{\langle \alpha,\alpha \rangle}[h_\alpha,F_\alpha]=\frac{2}{\langle \alpha,\alpha \rangle}\langle \alpha,-\alpha \rangle F_\alpha=-2F_\alpha,\\
	&[E_\alpha,F_\alpha]=\frac{2}{\langle \alpha,\alpha \rangle}h_\alpha=H_\alpha.
	\end{split}
\]
总结一下就是
\[
	[H_\alpha,E_\alpha]=2E_\alpha,\quad[H_\alpha,F_\alpha]=-2F_\alpha,\quad[E_\alpha,F_\alpha]=H_\alpha.
\]
这其实就是$\mathfrak{sl}(2)$的对易关系,所以呢,这三个基$E_\alpha,F_\alpha,H_\alpha$生成了一个同构于$\mathfrak{sl}(2)$的子代数,我们记做$\mathfrak{sl}(2)_\alpha$.\

\theo 如下三个陈述成立:

\no{1}如果$\alpha\in\Delta$,那么$k\alpha\notin \Delta$,其中$k\in \cc$,除非$k=-1$.

\no{2}如果$\alpha\in\Delta$,那么$\lag_\alpha$是一维的。

\no{3}对每一个$\alpha$,我们都可以找到$E_\alpha\in\lag_\alpha,F_\alpha\in\lag_{-\alpha}$以及$H_\alpha\in\mathfrak{h}$使得对易关系
\[
[H_\alpha,E_\alpha]=2E_\alpha,\quad[H_\alpha,F_\alpha]=-2F_\alpha,\quad[E_\alpha,F_\alpha]=H_\alpha
\]
成立,且$H_\alpha$是唯一的,不依赖于$E_\alpha$和$F_\alpha$的选取。

第三点前面部分刚刚已经证明过了,而我们定义的$H_\alpha$正比于$h_\alpha$,所以他就是唯一的。
第一第二点基本都是来自于$\mathfrak{sl}(2)$的同构。我们下面用$H$代替$h$来看一些公式是怎么变化的。首先
\[
	[H_\beta,a_\alpha]=\frac{2}{\langle \alpha,\alpha \rangle}[h_\beta,a_\alpha]=2\frac{\langle \alpha,\beta \rangle}{\langle \alpha,\alpha \rangle} a_\alpha.
\]

\pro 对任意的$\alpha,\beta\in \Delta$,$2\langle \alpha,\beta \rangle/\langle \alpha,\alpha \rangle$是一个整数。

这个是因为$2\langle \alpha,\beta \rangle/\langle \alpha,\alpha \rangle$是$\ad(H_\beta)$的本征值,而$\mathfrak{sl}(2)$的有限维不可约表示的本征值只能是整数。如果把$\langle \bullet,\bullet \rangle$当成内积,那么$\langle \alpha,\beta \rangle/\langle \alpha,\alpha \rangle$就是$\beta$往$\alpha$方向的投影,可以看到投影只能是整数和半整数。

我们继续做一些计算
\[
(H_\alpha,H_\beta)_K=\frac{4\langle \alpha,\beta \rangle}{\langle \alpha,\alpha \rangle\langle \beta,\beta \rangle},
\]
特别地,
\[
(H_\alpha,H_\alpha)_K=\frac{4}{\langle \alpha,\alpha \rangle},
\]
% 因此
% \[
% 	2H_\alpha=\frac{4}{\langle \alpha,\alpha \rangle}h_\alpha=(H_\alpha,H_\alpha)_Kh_\alpha,
% \]
% 或者
% \[
% 	h_\alpha=\frac{2H_\alpha}{(H_\alpha,H_\alpha)_K}.
% \]
% 我们也可以计算得到
% \[
% 	(H_\alpha,h_\alpha)_K=2,
% \]
于是
\[
	2\frac{\langle \alpha,\beta \rangle}{\langle \alpha,\alpha \rangle}=2\frac{4\langle \alpha,\beta \rangle}{\langle \alpha,\alpha \rangle\langle \beta,\beta \rangle}\frac{\langle \beta,\beta \rangle}{4}=2\frac{(H_\alpha,H_\beta)_K}{(H_\beta,H_\beta)_K}
\]
都是整数。
% \begin{theo}
% 对任意的$\alpha,\beta\in \Delta$,$\langle \alpha,\beta \rangle$是实的有理数。此外,$\langle \alpha,\alpha \rangle>0$.
% \end{theo}
% \begin{theo}
% 设$\lag$是有限维半单复Lie代数,而Cartan子代数是$\mathfrak{h}$,那么$\mathfrak{h}$就是那个包含所有的
% \[
% \sum_{\alpha\in\Delta}\mu_\alpha h_\alpha,\quad\mu_\alpha\in\cc,
% \]
% 的$\lag$的子空间。
% \end{theo}

\section{$\mathrm{SU}(2)$ and $\mathrm{SO}(3)$}
上面看到了$\mathfrak{sl}(2,\mathbb{C})$的Lie代数,或者说$\mathfrak{su}(2)_\cc\cong\mathfrak{so}(3)_\cc$的Lie代数,现在我们回到Lie群$\mathrm{SU}(2)$和$\mathrm{SO}(3)$。

首先复习一下$\mathrm{SO}(3)$的一些更细的结构,$\mathrm{SO}(3)$是三维旋转群,不包括反射。


\pro
绕着固定单位矢量$\mathbf{a}$右手螺旋的逆时针方向旋转$\theta$后的$\mathbf{x}$表示为
\[
\mathrm{Rot}(\mathbf{a},\theta)\mathbf{x}=\mathbf{x}+(1-\cos\theta)\mathbf{a}\times(\mathbf{a}\times\mathbf{x})+
\sin\theta\,\mathbf{a}\times\mathbf{x}.
\]


为了证明他,首先注意到当$\mathbf{a}$和$\mathbf{x}$垂直的时候显然有
\[
\mathrm{Rot}(\mathbf{a},\theta)\mathbf{x}=\cos\theta\,\mathbf{x}+
\sin\theta\,\mathbf{a}\times\mathbf{x}.
\]
对于一般的$\mathbf{x}$,分解为平行于$\mathbf{a}$和垂直于$\mathbf{a}$(设其单位矢量为$\mathbf{b}$)的两个部分$\mathbf{x}=\mu\mathbf{a}+\nu\mathbf{b}$,而平行于$\mathbf{a}$的部分在旋转下不变$\mathrm{Rot}(\mathbf{a},\theta)\mu\mathbf{a}=\mu\mathbf{a}$。由于$\mathrm{Rot}(\mathbf{a},\theta)$的线性性,我们有
\[
\begin{split}
\mathrm{Rot}(\mathbf{a},\theta)\mathbf{x}&=\mu\mathbf{a}+\cos\theta\,(\nu\mathbf{b})+
\sin\theta\,\mathbf{a}\times(\nu\mathbf{b})\\
&=\mathbf{x}\cos\theta+(1-\cos\theta)\mu\mathbf{a}+
\sin\theta\,\mathbf{a}\times\mathbf{x}\\
&=\mathbf{x}\cos\theta+(1-\cos\theta)(\mathbf{a}\cdot\mathbf{x})\mathbf{a}+
\sin\theta\,\mathbf{a}\times \mathbf{x},
\end{split}
\]
但是$\mathbf{a}\times(\mathbf{a}\times\mathbf{x})=(\mathbf{a}\cdot\mathbf{x})\mathbf{a}-\mathbf{x}$.所以得证。

现在对$\mathrm{Rot}(\mathbf{a},\theta)$中的$\theta$求$\theta=0$处的导数即可得到其Lie代数
\[
\left.\frac{\dd}{\dd \theta}\right|_{\theta=0}\mathrm{Rot}(\mathbf{a},\theta)\mathbf{x}=\mathbf{a}\times\mathbf{x}
\]
选$\mathbf{a}=\mathbf{e}_i$,得到Lie代数的三个基为
\[
	\eta_1=
		\begin{pmatrix}
			0&0&0\\
			0&0&-1\\
			0&1&0\\
		\end{pmatrix},\quad
	\eta_2=
		\begin{pmatrix}
			0&0&1\\
			0&0&0\\
			-1&0&0\\
		\end{pmatrix},\quad
	\eta_3=
		\begin{pmatrix}
			0&-1&0\\
			1&0&0\\
			0&0&0\\
		\end{pmatrix}.
\]