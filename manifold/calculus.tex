% !TeX root = manifold.tex
% author: buwailee@nmhs
\chapter{微积分}

\section{分布与余切矢量场}

在\ref{1.3.22}中,我们已经谈到,
如果我们的研究主要对象是积分曲面(积分子流形),那么其实重要的不是切矢量场的选取,而似乎是切矢量场张成的空间。在这节中,
我们就研究这样的子空间,以及相应的 Frobenius 定理。

\begin{para}[分布]
	设$M$是一个流形,$M$上的一个(光滑)分布$D$是切丛$TM$的
	(光滑)子丛,即(光滑地)对每一点$p$给出
	$T_p M$的一个子空间$D_p$. 若矢量场$X$处处满足
	$X_p\in D_p$,则称$X$属于$D$,记作$X\in D$.
\end{para}

显然,对于连通流形,若分布是连续的,则$\dim D_p$处处相同,
我们将其称为分布的秩。下面,我们只研究$\dim D_p$处处相同
的分布(即使流形不连通)。利用切丛的局部平凡化,我们可以在局部如下刻画一个分布$D$.


\begin{lem}
	若$D$是$M$上一个秩为$k$的分布,任取$p\in M$,都存在一个
	开集$U$,以及$U$上的$k$个切矢量场$X_1$, $\dots$, $X_k$
	使得$(X_1)_p$, $\dots$, $(X_k)_p$是$D_p$的一组基。
	如果$D$是光滑地,则$X_1$, $\dots$, $X_k$也可以取成光滑的。
\end{lem}

由于这个引理,我们可以反过来通过它来定义什么叫光滑分布。
以后,我们只考虑光滑分布,所以下面将略去光滑二字。
下面我们用分布的语言改写一下 Frobenius 定理。首先引入两个
概念,第一点来自于 Frobenius 定理的条件,第二点是
积分曲面概念的推广。

\begin{para}[对合分布]
	若任取$X$, $Y\in D$均有$[X,Y]\in D$,则称$D$是一个对合分布。
\end{para}

\begin{para}[积分流形]
	设$i:N\to M$是一个单浸入,$D$是一个分布,如果任取$p\in N$,都有$i_{*p}(T_p N)=D_{i(p)}$,则称$(N,i)$是分布 
	$D$的积分流形。若积分流形$(N,i)$并不真包含于另一
	积分流形中,则称这是一个极大积分流形。
\end{para}

此时,Theorem \ref{frobenius}定理可以改写为:

\begin{thm}[局部Frobenius定理]
	设$M$是光滑流形,$D$是一个光滑对合分布,则任取$p\in M$,局部总存在一个子流形$(S,\tau)$使得$(S,\tau)$是
	$D$的积分流形。
\end{thm}

\begin{para}
设$f$是$U$上的光滑函数,显然$\dd f$是一个$U$上的光滑余切矢量场。记$U$上的光滑函数的集合为$\Omega^0(U)$,记$U$上的光滑余切矢量场的集合为$\Omega^1(U)$,则$\dd: \Omega^0(U)\to \Omega^1(U)$.
\end{para}

下面我们要把Frobenius定理改写成余切矢量场的形式,这就变成了经典的Pfaff方程,正是当年关于Pfaff方程的研究,Cartran第一次提出了(高阶)外微分和微分形式的概念(我们现在只谈了一阶的情况),在他那里,$1$-形式之间的乘法被定义成反对称的。从前一节谈论Frobenius定理来看,Pfaff方程是一个关于积分曲面的问题,所以从这个角度来看,反对称的来由归根结底是为了积分。

\begin{para}[完全可积] 称一个$1$-形式$\omega$是完全可积的,如果存在两个光滑函数$f$和$g$使得$\omega=f\dd g$,此时$f$被称为$\omega$的积分因子。
\end{para}

$1$-形式的完全可积性联系着所谓的首次积分问题。设$f$是一个光滑函数且$\dd_p f$处处不为零,则$f(p)=a$(如果解存在)决定了$M$中的一个正则子流形$N_a$(有时候叫做一个曲面)。再设$X$是$M$内的光滑矢量场,则$\dd f(X)=0$恒成立当且仅当处处成立$X_p\in T_pN_{f(p)}$.

实际上,任取一点$p\in M$,只要检验$X_p(f)=0$即可,选一条$N_{f(p)}$上的一条光滑曲线$c$,使得$c(0)=p$且$c'(0)=X_p\in T_pN_{f(p)}$,由于$f(c(t))=f(p)$恒成立,对其在$t=0$处求导就得到了$\dd_pf(X_p)=X_p(f)=0$.反过来,如果在一点处$X_p\notin T_pN_{f(p)}$,则$X_p(f)\neq 0$。

固定$f$,将所有$\dd f(X)=0$的$X$拿出来,他组成一个$n-1$维的分布,$\{N_a\}$就是这个分布的一族积分流形,因为$X$在每一点都完全位于经过那一点的某个$N_a$的切空间内。我们称$N_a=\{p\in M:f(p)=a\}$是Paffa方程$\dd f=0$的解。从上面来看,一个Paffa方程要有解,那么解应该是一个积分曲面才是,即,Paffa方程$\omega=0$的解是使得$\omega(X)=0$的所有的$X$的积分曲面。

现在假设一个1-形式$\omega$是完全可积的,即他可以写作$\omega=f\dd g$,那么Paffa方程$\omega=0$等价于$\dd g=0$,这就确定了一个积分曲面。

\para 设$\omega$是一个1-形式,记分布$\ker \omega$是由满足$\omega(X)=0$的所有$X$张成的一个分布。记$\ker(\omega_1$, $\cdots$, $\omega_r)=\bigcap_{i=1}^r\ker \omega_i$.

% 则这个$\omega$完全可积当且仅当$\ker \omega$存在积分曲面。

\para 在局部,对任意一个分布$L$,存在一族余切矢量场$\{\omega_i:1\leq i \leq r\}$使得$L=\ker(\omega_1$, $\cdots$, $\omega_r)$.

实际上,一个分布在局部,和他在流形一点处(一个矢量空间内)是很相似的。设$L$是一个分布,由$r$个光滑的切矢量场$\{X_i:1\leq i \leq r\}$张成,则在局部,我们可以找到$n-r$个光滑矢量场$\{X_i:r+1\leq i \leq n\}$,使得$\{X_i:1\leq i \leq n\}$处处线线性无关。依然在局部,我们可以找到与其对偶\footnote{即满足$\omega_i(X_j)=\delta_{ij}$.}的1-形式$\{\omega_i:1\leq i \leq n\}$,那么那些使得$\omega(X)$对$X\in L$成立的1-形式局部由$\{\omega_i:r+1\leq i \leq n\}$张成。因此,局部上,一个分布$L$可以写作$L=\ker(\omega_{r+1}$, $\cdots$, $\omega_n)$,等价地,可以写作一个Paffa方程组$\{\omega_i=0:r+1\leq i \leq n\}$.

因为局部存在积分曲面的充要条件是任取$X$, $Y\in D$满足$[X,Y]\in D$,所以如果$L=\ker(\omega_{r+1}$, $\cdots$, $\omega_n)$存在积分曲面,应该有$\omega_i([X,Y])=0$。

\para 设$\omega=f\dd g$,其中$f$和$g$是光滑函数,则对一般的光滑矢量场$X$, $Y$成立。
\begin{equation}
\begin{aligned}
	\omega([X,Y])=f\dd g([X,Y])&=f [X,Y](g)\\
	&=fX(Y(g))-fY(X(g))\\
	&=X(fY(g))-X(f)Y(g)-Y(fX(g))+Y(f)X(g)\\
	&=X(\omega(Y))-Y(\omega(X))-\bigl(X(f)Y(g)-Y(f)X(g)\bigr).
\end{aligned}
\end{equation}

对于$X(f)Y(g)-Y(f)X(g)$,我们可以将其改写为$\dd f(X)\dd g(Y)-\dd g(X)\dd f(Y)$,因为这是关于$X$和$Y$的双线性函数,我们可以引入一个张量$\dd f\otimes \dd g$使得$\dd f\otimes \dd g(X,Y)=\dd f(X)\dd g(Y)$,则
\[
	\dd f(X)\dd g(Y)-\dd f(Y)\dd g(X)=\dd f\otimes \dd g(X,Y)-\dd g\otimes \dd f(X,Y)=(\dd f\otimes \dd g-\dd g\otimes \dd f)(X,Y).
\]
记$\dd f\wedge \dd g=\dd f\otimes \dd g-\dd g\otimes \dd f$,其中的$\wedge$被称为\idx{楔积},记$D(f\dd g)=\dd f\wedge \dd g$,则式(\theequation)变成了
\[
	D(f\dd g)(X,Y)=X(\omega(Y))-Y(\omega(X))-\omega([X,Y]).
\]
从式子右端来看,$\dd f\wedge \dd g(X,Y)$并不依赖于$\omega$的具体形式$\omega=f\dd g$. 实际上,任意一个1-形式$\omega$可以写作$\omega=\sum_i f_i\dd g_i$,所以对于一般的情况,式(\theequation)应该写作
\[
	\sum_iD(f_i\dd g_i)(X,Y)=X(\omega(Y))-Y(\omega(X))-\omega([X,Y]).
\]
如果我们把$D$看做线性算子,则对于任意一个1-形式,我们都定义了一个线性算子,满足
\begin{equation}
	D(\omega)(X,Y)=X(\omega(Y))-Y(\omega(X))-\omega([X,Y]).
	\label{eq:1.3}
\end{equation}

\para 当然,我们也可以反过来通过式\eqref{eq:1.3}来定义$D(\omega)$,顺序在这里不是紧要的。紧要的是,$D(\omega)$决定了$\ker \omega$是否容许一个积分曲面。这是因为,如果$X$, $Y\in \ker \omega$,则式\eqref{eq:1.3}变成了
\[
	D(\omega)(X,Y)=-\omega([X,Y]),
\]
所以$D(\omega)(X,Y)=0$当且仅当$[X,Y]\in \ker \omega$.

这正是Cartan当年提出微分形式时候的处境,那时候,他从Frobenius和Darboux那里知道了,不同的Pfaff形式的等价条件就联系在一个bilinear covariant上面,而这个bilinear covariant就是我们这里的$D(\omega)$.

从这个角度来看,正因为有Frobenius定理,或者更本质一点,我们需要把积分曲线拼成积分曲面,我们需要考察两个矢量场$X$和$Y$的Lie括号$[X,Y]$,而这个Lie括号的反对称性来自于我们比较两条路径。现在,这种反对称性反应在了1-形式之间的楔积,使得他构成了一个(吃掉两个矢量场的)反对称函数。所以,从Cartan这里,反对称性的来源应该是为了处理积分曲面的存在性,而由$[X,Y]$自然诱导出来的。

\begin{pro}\label{dd=0}
	$D(\dd f)=0$.
\end{pro}
	
\begin{proof}
从\eqref{eq:1.3}
\[
	D(\dd f)(X,Y)=X(Y(f))-Y(X(f))-[X,Y](f)=[X,Y](f)-[X,Y](f)=0,
\]
对任意的矢量场$X$, $Y$都成立。
\end{proof}
	

\para 外代数的复习在附录,对于矢量空间$V$的$k$-次外代数记做$\Omega^k(V)$。在流形上的一点$p$处,记$\Omega_p^k=\Omega^k(T_pM)$,则$\Omega_p^1=T_p^*M$.类似切丛和余切丛,我们可以使用$\Omega_p^k$拼出一个$k$-形式丛$\Omega^k$.

在$U$上的一个光滑$k$-形式被定义为$\Omega^k$在$U$上的一个光滑截面。所有$U$上的光滑$k$-形式的集合记做$\Gamma(\Omega^k,U)$.显然,$U$上的光滑函数可以看成一个光滑$0$-形式,一个光滑余切矢量场是一个光滑1-形式。如果$\omega$是一个光滑$1$-形式,则$D(\omega)$是一个光滑$2$-形式。下面我们所称的形式都是光滑的,我们将省略光滑二字。

\para 设分布$L$由$\{X_i:1\leq i \leq r\}$张成,且$L=\ker(\omega_{r+1}$, $\cdots$, $\omega_n)$,如果$D(\omega)(X_i,X_j)=0$成立,则$D(\omega)$可以写作\[
	D(\omega)=\sum_{i=r+1}^n \psi_i\wedge \omega_i,
\]
其中$\psi_i$是一次微分式.

实际上,局部地$\{\omega_i:1\leq i\leq n\}$构成一组基,则
\[
	\dd \omega = \sum_{i=r+1}^n \psi_i\wedge w_i +\sum_{i,j=1}^r a_{ij}\omega_i\wedge \omega_j,
\]
其中$\psi_i$是一次微分式,而$a_{ij}$是光滑函数,且关于指标是反对称的。因为$\omega_i(X_j)=\delta_{ij}$,所以
\[
	0=\dd \omega(X_i,X_j)=\sum_{p,q=1}^r a_{pq}\omega_p\wedge \omega_q(X_i,X_j)=\sum_{p,q=1}^r a_{pq}(\delta_{ip}\delta_{jq}-\delta_{jp}\delta_{iq})=2a_{ij}.
\]

\para 设分布$L$由$\{X_i:1\leq i \leq r\}$张成,且$L=\ker(\omega_{r+1}$, $\cdots$, $\omega_n)$,则$L$存在积分曲面当且仅当,
\[
	D(\omega_i)=\sum_{j=r+1}^n \psi_{ij}\wedge \omega_i
\]
对每一个$i$都成立。

特别地,设分布$L=\ker(\omega)$,则局部积分曲面存在的充分必要条件是$D(\omega)=\psi\wedge \omega$,再或者$D(\omega)\wedge \omega=0$.则也就是$\omega$完全可积的充分必要条件。

\section{外微分}

函数$f$的支集$\supp(f)$被定义为$\{x\in M:f(x)\neq 0\}$的闭包。下面关于单位分解的引理的证明看附录,这里仅摘录定义和引理:

\vspace{0.5em}
\noindent\textbf{\ref{POUdef}.} 单位分解:设$\{U_\alpha\}_{\alpha\in I}$是$M$的一个开覆盖,如果存在可数个光滑函数$g_i\in \calf(M)$满足:

\begin{compactenum}
\item 对于任意的$x\in M$和$i\in I$,都有$0\leq g_i(x)\leq 1$.
\item 对每个$g_i$,都存在一个${\alpha_i}$使得$\supp(g_i)\subset U_{\alpha_i}$.
\item 集族$\{\supp(g_i)\}$局部有限,即任取$p\in M$,存在$p$的邻域$U$使得$U$只和集族$\{\supp(g_i)\}$中的有限个集合相交非空。
\item $\sum_i g_i=1$. 求和是有意义的,因为上一个性质,所以在一点累加$g_i$时,只有有限项非零。
\end{compactenum}
则称$\{g_i\}$是从属于开覆盖$\{U_\alpha\}_{\alpha\in I}$的一个单位分解。

\vspace{0.5em}

\noindent\textbf{Lemma \ref{POU}.} 流形$M$上的每一个开覆盖都存在从属于他的单位分解。

\para \label{unit} 设$V\subset U$,且$\overline{V}\subset U$,则$M-\overline{V}$和$U$构成$M$的一个开覆盖,我们可以找到$h$, $g\in \calf(M)$使得$h+g=1$处处成立,且$h|_{M-U}=0$以及$g|_{\overline{V}}=0$,因此$h|_{\overline{V}}=1$.

\para 在很久很久以前,对$U$上的一个光滑函数$f$,我们定义了外微分$\dd$,使得$\dd f$是一个1-形式,而在上一节,我们对$U$上的一个1-形式$\omega$,定义了$D$使得$D(\omega)$是一个2-形式。更一般地,我们希望可以定义如下一个算符
\[
	\dd_k:\Gamma(\Omega^k,U)\to \Gamma(\Omega^{k+1},U),
\]
使得$\dd_0=\dd$,$\dd_1=D$. 我们将$\{\dd_k\}$统称为外微分算符,统一记做$\dd$,他完成了一个$k$-形式到一个$(k+1)$-形式的转变。

所以,我们对于线性算符$\dd$,需要满足如下性质,

\begin{compactenum}[~~~(1)]
\item 对于光滑函数$f$,$\dd f$就是我们前面定义的微分。
\item 任取光滑函数$f$,$\dd^2 f=\dd (\dd f)=0$,这来自于Proposition \ref{dd=0},也是暗示$\dd$作用在1-形式上应该就是$D$.
\item 作为微分算符的Leibniz法则:设$\omega$是$k$-形式,则
\[
	\dd(\omega\wedge \eta)=\dd \omega \wedge \eta +(-1)^k \omega\wedge \dd\eta,
\]
其中$(-1)^k$的出现来自楔积的交换。这个Leibniz法则我们已经在计算$D(f\dd g)$和$\dd(fg)$的时候遇到过了。
\end{compactenum}

\para \label{localform}先假设$\dd$在$U$上是存在的。若$\omega$是$V\subset U$上的$k$-形式,设$W$是$U$的一个开子集且$V$真包含$\overline{W}$.那么正如\ref{unit}说的,可以找一个单位分解$h$,使得$h|_{\overline{W}}=1$以及$h|_{U-W}=0$,利用他就可以定义一个$U$上的光滑$k$-形式$h\omega$(他在$V$外为零,在$W$内等同于$\omega|_W$)。

\begin{lem}\label{regular}设$U$是流形$M$上的开集且$V\subset U$,则存在$V$的一个开覆盖$\{W_\alpha\}_{\alpha\in I}$使得对每个$\alpha$成立$\overline{W_\alpha}\subset V$.
\end{lem}

实际上这就是分离公理之一,正则性的直接推论。流形$M$是局部紧的Hausdorff空间,所以他是正则的。而对于正则空间内的每一个点$p$,已知邻域$V$,我们可以找到一个邻域$W_p\subset V$使得$\overline{W_p}\subset V$.然后遍历$p$,就找到了开覆盖$\{W_p\}_{p\in V}$.

\begin{lem}\label{localpro}
	设$V\subset U$,如果$\omega|_V=0$,则$(\dd \omega)|_V=0$,即$\dd$是一个局部算符。
\end{lem}

\begin{proof}
利用\ref{regular}找一个$V$的开覆盖,我们只要证明对开覆盖中的一个开集$W$,$(\dd \omega)|_{W}=0$即可。同\ref{localform}利用单位分解找一个光滑函数$h$,则$h\omega$在$U$上恒为零,所以
\[
	0=\dd (h\omega)=\dd h \wedge \omega +h\dd \omega
\]
限制在$W$上,$h|_{W}=1$且$(\dd h)|_{W}=0$(利用$\dd h$是一个微分这一个事实),所以$(\dd \omega)|_{W}=0$.
\end{proof}

\begin{pro}
\label{localdef}利用\ref{localpro},设$V\subset U$,如果$U$上存在外微分形式$\dd$,则他诱导出了$V$上的一个外微分形式$\dd_V$。并且,再设$W\subset W$,则$\dd$在$W$上诱导的$\dd_W$和$\dd_V$在$W$诱导的$\dd_{VW}$是相同的。此外$(\dd_V \omega)|_W=\dd_W (\omega|_W)$对任意的$V$上的$k$-形式$\omega$成立。
\end{pro}

\begin{proof} 和上面的想法差不多,利用\ref{regular},我们可以找一个开覆盖。然后在每个开覆盖中的开集$W$上,同\ref{localform}利用单位分解找一个光滑函数$h_W$,使得$h_W\omega$成为$U$上的光滑$k$-形式,定义$(\dd_V \omega)|_{W}=\dd (h_W\omega)|_{W}$.

设$W'$是开覆盖中的另外的开集,且$W\cap W'\neq \varnothing$。由于$(h_W\omega)|_{W\cap W'}=(h_{W'}\omega)|_{W\cap W'}$,因为\ref{localpro},则
\[
	\dd (h_W\omega)|_{W\cap W'}=\dd (h_{W'}\omega)|_{W\cap W'},
\]
所以
\[
	((\dd_V \omega)|_{W})|_{W\cap W'}=((\dd_V \omega)|_{W'})|_{W\cap W'}
\]
保证了$(\dd_V \omega)|_W$在相交的开集上是相同的,这就使得我们可以黏结他们定义出一个$V$上的$(k+1)$-形式$\dd_V \omega$. 容易检验$\dd_V$满足所有外微分的性质。至于$\dd_W=\dd_{VW}$,从构造来看,这是显然的。

最后我们来检验等式$(\dd_V \omega)|_W=\dd_W (\omega|_W)$,对$W$利用\ref{regular}找个开覆盖,在每一个开集$X$上,把上式左边限制到$X$上即$((\dd_V \omega)|_W)|_X=(\dd_V \omega)|_X=\dd(h_X\omega)$,同样,右边限制到$X$上即$(\dd_W (\omega|_W))|_X=\dd(h_X\omega|_W)=\dd(h_X\omega)$,所以等式成立。\end{proof}

\para 设$W\subset V$是$U$中的开集,以及$\rho^k_{VW}$和$\rho^{k+1}_{VW}$分别是$k$-形式和$(k+1)$-形式的限制映射。设$\omega$是$V$上的任意$k$-形式,由于$\rho^{k+1}_{VW}\circ \dd_V (\omega)=(\dd_V \omega)|_W=\dd_W (\omega|_W)=\dd_W \circ \rho^k_{VW}(\omega)$,所以我们有
\[
\dd_W\circ \rho^k_{VW}=\rho^{k+1}_{VW}\circ \dd_V.
\]
换句话说,$\dd$诱导了预层$U\mapsto \Gamma(\Omega^k,U)$和预层$U\mapsto \Gamma(\Omega^{k+1},U)$间的自然变换(或者叫做函子间的态射)。

\para 容易证明,以$p$的邻域$W$赋予包含而成的归纳系,
\[
	\Omega_p^k=\varinjlim_{W\ni p}\Gamma(\Omega^k,W),\quad \Omega_p^{k+1}=\varinjlim_{W\ni p}\Gamma(\Omega^{k+1},W),
\]
以及有到点上的限制映射$\rho^k_{Wp}$和$\rho^{k+1}_{Wp}$. 则colimt的泛性质,如下交换图告诉我们$\dd_p:\Omega_p^k\to \Omega_p^{k+1}$存在。
\[
	\xymatrix{
	&&\ar[dl]^{\rho^k_{Vp}}\Gamma(\Omega^k,V)\ar[dd]^{\rho^k_{VW}}\ar@/_/[lld]_{\rho^{k+1}_{Vp}\circ \dd_V} \\
	\Omega_p^{k+1}&\ar@{-->}[l]_(0.3){\dd_p}\Omega_p^{k}&\\
	&&\ar[ul]_{\rho^k_{Wp}}\Gamma(\Omega^k,W)\ar@/^/[llu]^{\rho^{k+1}_{Wp}\circ \dd_W}
	}
\]
以后,方便起见,即使$\omega$只是$U$的开集$V$上的$k$-形式的时候,我们依旧用$\dd \omega$来标记$\dd_V \omega$.这当然也就意味着,在记号上,$\dd(\omega|_W)=(\dd\omega)|_W$.

\begin{pro}
在流形$M$上,外微分算子$\dd$存在且唯一。
\end{pro}

\begin{proof} 有了上面这些铺垫,我们只要在局部证明其唯一存在即可,然后把他拼起来,就像Proposition \ref{localdef}做的那样。首先证明局部唯一性,为此当然假设$\dd$是存在的。在局部,我们找一族坐标,对于$k$-形式,他写作
\[
	\omega=\sum_{i_1,\dots ,i_k} a_{i_1\dots i_k}\dd x^{i_{1}}\wedge\cdots\wedge x^{i_{k}},
\]
由于$\dd$是线性算子,我们可以只考虑
\[
	\omega=a\dd x^{1}\wedge\cdots\wedge \dd x^{k}.
\]

由于Leibniz法则,
\[
	\dd \omega=\dd a \wedge \dd x^{1}\wedge\cdots\wedge \dd x^{k}+a\dd(\dd x^{1}\wedge\cdots\wedge \dd x^{k}),
\]
对于第二项,使用Leibniz法则,以及$x^1$是光滑函数,所以有的$\dd^2 x^1=0$,于是
\[
	\dd(\dd x^{1}\wedge\cdots\wedge \dd x^{k})=-\dd x^{1}\wedge\dd(\dd x^{2}\wedge\cdots\wedge \dd x^{k}),
\]
这样进行下去就知道他是零。于是
\[
	\dd \omega=\dd a \wedge \dd x^{1}\wedge\cdots\wedge \dd x^{k},
\]
这样$\dd$在局部的表现完全由他那些性质唯一决定,唯一性得证。

剩下的存在性,我们就把上面的过程反过来,在局部的$k$-形式写作
\[
	\omega|_U=\sum_{i_1,\dots ,i_k} a_{i_1\dots i_k}\dd x^{i_{1}}\wedge\cdots\wedge x^{i_{k}},
\]
则定义其外微分为
\[
	\dd(\omega|_U)=\sum_{i_1,\dots ,i_k} \dd a_{i_1\dots i_k}\wedge\dd x^{i_{1}}\wedge\cdots\wedge x^{i_{k}},
\]
不难检验外微分的性质都可以得到满足。这样我们就在局部证明了外微分的存在性。

最后,由于$(\dd(\omega|_U))|_{U\cap V}=\dd(\omega|_U|_{U\cap V})=\dd(\omega|_V|_{U\cap V})=(\dd(\omega|_V))|_{U\cap V}$,所以我们可以将局部定义的外微分算子粘起来得到流形$M$上的一个外微分算子。由于外微分算子的局部唯一性,所以也就得到了他的整体唯一性。\end{proof}

\begin{pro}
虽然是一个很简单的命题,但是很重要:$\dd^2=0$.
\end{pro}

\begin{proof} 局部对单项式证明即可,设$\omega=a\dd x^{1}\wedge\cdots\wedge \dd x^{k}$,则
\[
	\dd \omega=\dd a\wedge \dd x^{1}\wedge\cdots\wedge \dd x^{k}
\]
以及
\[
	\dd^2 \omega=\dd^2 a\wedge \dd x^{1}\wedge\cdots\wedge \dd x^{k}-\dd a\wedge \dd^2 x^{1}\wedge\cdots\wedge \dd x^{k}+\cdots=0.
\]
\end{proof}

\para 我们将$\dd \omega=0$的形式$\omega$称为闭形式,将$\omega=\dd \eta$的形式$\omega$称为恰当形式,则这一命题告诉我们,恰当形式一定是闭形式,反之不一定。我们定义$H^k(M)=\ker(d_k)/\im(d_{k-1})$为流形$M$的第$k$个上同调群,其中模的运算是看做加法群的商群运算,这个$H^k(M)$即表征了闭的$k$-形式在除去一个恰当形式后的等价类。此外约定当$k<0$的时候,$H^k(U)=0$.

当$k=0$的时候,$H^0(U)$就是$\mathrm{ker}\left(\dd:\calf(U)\to\Omega^1(U)\right)$.如果$U$是连通的,$\dd f=0$就昭示着$f$在恒为常数,$H^0(U)$就是这些函数构成的矢量空间。那么如果有着不同的连通分支,那么$H^0(U)$就是在在各个连通分支上为常数(整体不一定是一样的)的那些函数构成的矢量空间。而他的维度就是连通分支的个数。

下一节将会提到一点,上同调群是流形拓扑的表征,他是一个拓扑范畴到交换群范畴的反变函子,对于同伦等价的拓扑空间,有着同构的上同调群。所以这是一个拓扑不变量,如果我们可以把两个拓扑空间的上同调群计算出来得到他们不同构,则这两个拓扑空间必然不同伦。流形的上同调群也被称为de Rham上同调,他是上同调群的一个典型,这里$\dd$是自然的边缘算子的对偶,而楔积是自然的cup product(比起同调,上同调里的cup product是比“对偶”部分多出来的东西)。

\para 正如\ref{eq:1.3}我们看到的,对$1$-形式$\omega$,我们有
\[
	\dd(\omega)(X,Y)=X(\omega(Y))-Y(\omega(X))-\omega([X,Y]),
\]
类似地,对$k$-形式$\omega$,经过一些不算复杂的计算,我们可以证明
\[
\begin{aligned}
	\dd(\omega)(X_1,\dots,X_{k+1})=&\sum_{i=1}^{k+1}(-1)^{i+1}X_i\bigl(\omega(X_1,\dots,\hat{X_i},\dots,X_{k+1})\bigr)\\
	&+\sum_{1\leq i<j\leq k+1}(-1)^{i+j}\omega\bigl([X_i,X_j],X_1,\dots,\hat{X_i},\dots,\hat{X_j},\dots,X_{k+1}\bigr),
\end{aligned}
\]
其中
\[
	\omega\bigl(X_1,\dots,\hat{X_i},\dots,X_{k+1}\bigr)=\omega\bigl(X_1,\dots,X_{i-1},X_{i+1},\dots,X_{k+1}\bigr),
\]
以及
\[
\begin{aligned}
	\omega\bigl([X_i,X_j],X_1,\dots,\hat{X_i},\dots,&\hat{X_j},\dots,X_{k+1}\bigr)=\\
	&\omega\bigl([X_i,X_j],X_1,\dots,X_{i-1},X_{i+1},\dots,X_{j-1},X_{j+1},\dots,X_{k+1}\bigr),
\end{aligned}
\]
就是说,头上带帽等于把他去掉。这个式子可以反过来拿来定义$\dd$,而且确实是一个坐标无关的定义。我们并没有那么做,因为就实际计算而言,更多时候我们并不会把形式作用在所有的光滑切矢量场上。

\para 设$f:M\to N$,我们前面已经谈到了,这个$f$在0-形式,即光滑函数之间诱导了一个拉回$f^*$,即$f^*g=g\circ f$. 对于$1$-形式,由\ref{f*d=df*},我们知道了$f^*\dd g=\dd (f^* g)$,对比矢量场之间$f_*$的定义还需要$f$是单射,这里1-形式之间的$f^*$则没有这个限制。

通过$(f^*\omega)_p=\omega_{f(p)}\bigl(f_{*p}(X_1)_p,\dots,f_{*p}(X_k)_p\bigr)$,我们可以定义出$k$-形式之间的$f^*$.

\begin{para}
所以,很容易检验,$f^*(\omega\wedge \eta)=(f^*\omega)\wedge (f^*\eta)$.
\end{para}

实际上,这个式子也可以反过来定义$k$-形式之间的$f^*$.可是这样定义可能会遇到一些技术性问题,比如一个$k$-形式大范围来说是否一定能写成一些比较低阶的形式楔积并线性组合而成?如果不,怎么通过上式定义?这个问题的回答是,类似\ref{localpro},拉回是一个局部算符,即如果$\omega$在局部为零,则$f^*\omega$也在局部为零,所以我们可以不从大范围考虑这个问题。

\para 按照上面这种定义方式,假设$f^*$存在,则$f^*$是一个局部算符。

首先注意到$f^*(h\omega)=(f^*h)(f^*\omega)$。设$\omega$在$U$上为零,利用\ref{regular},我们可以找一个$U$开覆盖。然后在每个开覆盖中的开集$W$上,同\ref{localform}利用单位分解找一个光滑函数$h_W$,则按照上式$f^*(h_W\omega)$只可能在$\overline{W}$内不为零,因此,如果$\omega$在$U$上为零,则$h_W\omega$就恒为零,所以$(f^*h_W)(f^*\omega)=f^*(h_W\omega)=0$. 由于$f^*h_W=h_W\circ f$在$f^{-1}(W)$内等于$1$,所以$(f^*\omega)|_{f^{-1}(W)}=0$,遍历开覆盖,我们就得到了$(f^*\omega)|_{f^{-1}(U)}=0$.

特别地,如果$\omega_1$和$\omega_2$在$U$上相同,则$0=(f^*(\omega_1-\omega_2))|_{f^{-1}(U)}=(f^*\omega_1)|_{f^{-1}(U)}-(f^*\omega_2)|_{f^{-1}(U)}$,这就意味着$f^*\omega_1$和$f^*\omega_2$在局部相同,此即黏合条件。

我们这里不再重复类似$\dd_U$诱导出$\dd_V$的过程,以及这些$f^*_U$什么的和限制映射之间的关系,统统直接记做$f^*$,则$f^*(\omega_{U})=(f^* \omega)_{f^{-1}(U)}$。因为局部来说$k$-形式确实可以写成一些比较低阶的形式楔积并线性组合而成,以及我们已经对$0$-形式和$1$-形式定义了拉回,则局部的$k$-形式的拉回可以使用我们的定义递归地定义出来,那么剩下的只要拼起来就好,而这正需要黏合条件。

最后,这样定义的$f^*$在一点诱导出的$f^*_p$(类似$\dd$到$\dd_p$)可以验证和$1$-形式间已经存在的$f^*_p$用张量积诱导出来的映射是相同的。

\para 设$f:M\to N$,则$f^*(\dd \omega)=\dd (f^*\omega)$.实际上,我们只要局部对单项式证明即可,设$\omega=a\dd x^{1}\wedge\cdots\wedge \dd x^{k}$,则$\dd \omega=\dd a\wedge \dd x^{1}\wedge\cdots\wedge \dd x^{k}$,以及
\[
	f^*(\dd \omega)=(f^*\dd a)\wedge f^*(\dd x^{1}\wedge\cdots\wedge \dd x^{k})=\dd (f^*a)\wedge \dd f^*(x^{1})\wedge\cdots\wedge \dd f^*(x^{k})=\dd (f^*\omega).
\]

因为$f^*$将恰当形式映射成恰当形式,即$f^*(\im \dd_N)\subset \im \dd_M$,所以他诱导了两个流形上同调群之间的同态$f^*:H^k(N)\to H^k(M)$.

\section{de Rham上同调初步}

我们这节将目标放在$\rr^n$中的de Rham上同调。

\para 一系列矢量空间和上面的线性映射$A\xrightarrow{f}B\xrightarrow{g}C$称为正和列,就是说$\ker g=\mathrm{Im} f$.而$0\to A\xrightarrow{f}B$
就是说$f$是单射,而$B\xrightarrow{g}C\to 0$就是说$g$是满射。而正和列
\[
0\to A\xrightarrow{f}B\xrightarrow{g}C\to 0
\]
称为短正合列。

一个矢量空间和线性映射链$A^*=\{A_i,\dd_i\}$
\[
	\cdots\to A^{i-1}\xrightarrow{\dd^{i-1}}A^i\xrightarrow{\dd^i}A^{i+1}\to \cdots
\]
称为链复形,如果对于任意的$i$都有$\dd^i \circ \dd^{i+1}=0$.当对任意的$i$都有$\ker \dd^i=\mathrm{Im}\, \dd^{i-1}$,则这个链复形称为正和的。

很容易看到,$\Omega^*(M)$和外微分算子$\dd$构成一个链复形。那么同样,对于任意的链复形都可以定义上同调群$H^p(A^*)=\ker (\dd^p)/\mathrm{Im} (\dd^{p-1})$,其中的元素同样用等价类符号$[a]$记。

\para 如果在两条链的每一个对应链的对象之间,譬如说$A_i$和$B_i$之间,存在线性映射$f^i$,那么自然就在两条链之间引入了一个映射$f:A^*\to B^*$,需要交换图如下:
	\[
	\xymatrix{
		\cdots\ar[r]&A^{p-1}\ar[r]^{\dd_A^{p-1}}\ar[d]^{f^{p-1}}&A^p\ar[r]^{\dd_A^p}\ar[d]^{f^{p}}&A^{p+1}\ar[r]\ar[d]^{f^{p+1}}&\cdots\\
		\cdots\ar[r]&B^{p-1}\ar[r]^{\dd_B^{p-1}}&B^p\ar[r]^{\dd_B^p}&B^{p+1}\ar[r]&\cdots
	}
	\]
从交换图可以看到,应该满足$\dd^{p}_B\circ f^p=f^{p+1}\circ \dd^{p}_A$.既然在链复形之间引入了映射,则他诱导了上同调群之间的映射。通过$f^*([a])=[f^p(a)]$,我们诱导了$f^*=H^p(f):H^p(A^*)\to H^p(B^*)$.

\para 链复形也可以构成一个链,尤其重要的是短正合列$0\to A^*\xrightarrow{f}B^*\xrightarrow{g}C^*\to 0$.链的短正和列就是当对任意的$p$都有短正合列$0\to A^p\xrightarrow{f^p}B^p\xrightarrow{g^p}C^p\to 0$.

\begin{pro}
链复形的短正合列$0\to A^*\xrightarrow{f}B^*\xrightarrow{g}C^*\to 0$引入了上同调群的正合列$H^p(A^*)\xrightarrow{f^*}H^p(B^*)\xrightarrow{g^*}H^p(C^*)$.
\end{pro}

\begin{proof} 其实就是证明$\ker g^*=\mathrm{Im}\, f^*$.首先证明$\mathrm{Im}\, f^* \subset \ker g^*$,任取$[a]\in H^p(A^*)$,我们有
\[
g^*\circ f^*([a])=[g^p\circ f^p(a)]=[0]=0.
\]
这是从正合列$A^p\xrightarrow{f^p}B^p\xrightarrow{g^p}C^p$中得知的。

然后证明$\ker g^*\subset \mathrm{Im}\, f^*$.这就是说,任意的$g^*[b]=0$的$[b]$都可以找到$[a]$使得$f^*[a]=[b]$.

因为对任意的$p$有$0=g^*[b]=[g^p(b)]$,因此存在一个$c$使得$g^p(b)=\dd^{p-1}_C(c)$,而$g^{p-1}$又是满射,所以可以找到$b'$使得$g^{p-1}(b')=c$,因此用交换图变换
\[
g^p(\dd^{p-1}_B(b'))=d^{p-1}_C(g^{p-1}(b'))=g^p(b).
\]

所以$g^p(b-\dd^{p-1}_B(b'))=0$,所以存在$a$使得$f^p(a)=b-\dd^{p-1}_B(b')$. 现在只要证明这个$a$确实在$\ker \dd^p_A$里面就可以了。为此只要证明$\dd^p_A a=0$就可以,但是因为$f^{p+1}$是单射,所以也等价于证明$f^{p+1}\circ \dd^p_A (a)=0$. 用交换图变换
\[
f^{p+1}\circ \dd^p_A (a)=\dd^p_B\circ f^p (a)=\dd^p_B(b-\dd^{p-1}_B(b'))=\dd^p_B(b)=0.
\]
因为$a$确实在$\ker \dd^p_A$里面,所以他在$H^p(A^*)$里面对应了一个等价类$[a]$,成立$f^*[a]=[b]$.\end{proof}

\para 链复形的短正和列还引入了其他两个正合列。对于链复形的短正合列$
0\to A^*\xrightarrow{f}B^*\xrightarrow{g}C^*\to 0$,定义$\partial^*:H^p(C^*)\to H^{p+1}(A^*)$为线性映射
\[
	\partial^*([c])=\left[(f^{p+1})^{-1}\left(\dd^p_B\left((g^p)^{-1}(c)\right)\right)\right].
\]

$\partial^p$即交换图
	\[
		\xymatrix{
			&0\ar[d]&0\ar[d]&0\ar[d]&\\
			\cdots\ar[r]&A^{p-1}\ar[r]^{\dd_A^{p-1}}\ar[d]^{f^{p-1}}&A^p\ar[r]^{\dd_A^p}\ar[d]^{f^{p}}&A^{p+1}\ar[r]\ar[d]^{f^{p+1}}&\cdots\\
			\cdots\ar[r]&B^{p-1}\ar[r]^{\dd_B^{p-1}}\ar[d]^{g^{p-1}}&B^p\ar[r]^{\dd_B^p}\ar[d]^{g^{p}}&B^{p+1}\ar[r]\ar[d]^{g^{p+1}}&\cdots\\
			\cdots\ar[r]&C^{p-1}\ar[r]^{\dd_B^{p-1}}\ar[d]&C^p\ar[r]^{\dd_B^p}\ar[d]\ar[ruu]&C^{p+1}\ar[r]\ar[d]&\cdots\\
			&0&0&0&
		}
	\]
中的斜线。这里就不证明这是良定义的了。因此,链复形的短正合列$0\to A^*\xrightarrow{f}B^*\xrightarrow{g}C^*\to 0$诱导了上同调群的正合列
\[
\begin{aligned}
&H^p(B^*)\xrightarrow{g^*}H^p(C^*)\xrightarrow{\partial^*}H^{p+1}(A^*),\\
&H^p(C^*)\xrightarrow{\partial^*}H^{p+1}(A^*)\xrightarrow{f^*}H^{p+1}(B^*).
\end{aligned}
\]

\begin{thm}\label{longexact}
链复形的短正合列$0\to A^*\xrightarrow{f}B^*\xrightarrow{g}C^*\to 0$引入了上同调群的正合列
\[
\cdots\to H^p(A^*)\xrightarrow{f^*}H^p(B^*)\xrightarrow{g^*}H^p(C^*)\xrightarrow{\partial^*}H^{p+1}(A^*)\xrightarrow{f^*}H^{p+1}(B^*)\to\cdots.
\]
\end{thm}

\begin{para}\label{directsum} 链复形可以谈论直和,即是对链中每一个矢量空间进行直和。那么从$\ker$和$\mathrm{Im}$对于直和的显然性质,我们有$H^p(A^*\oplus B^*)=H^p(A^*)\oplus H^p(B^*)$.
\end{para}

$\Omega^*(U)$和外微分算子$\dd$构成一个链复形,下面的定理给出了有关于欧氏空间两个开集和他们的并与交的短正合列。

\begin{thm}
设$U_1$和$U_2$是$\rr^n$中的开集,记$i_\nu:U_\nu \to U_1 \cup U_2$和$j_\nu:U_1\cap U_2 \to U_\nu$是嵌入,则有如下的短正合列:
\[
0\to \Omega^p(U_1\cup U_2)\xrightarrow{I^p}\Omega^p(U_1)\oplus\Omega^p(U_2)\xrightarrow{J^p}\Omega^p(U_1\cap U_2)\to 0.
\]
其中$I^p(\omega)=(i_1^*(\omega),i_2^*(\omega))$, $J^p(\omega_1,\omega_2)=j_1^*(\omega_1)-j_2^*(\omega_2)$.
\end{thm}

\begin{proof} 先证明$I^*$是单射,这就是说除了$I^*(\omega)=0$只有解$\omega=0$. 设$\varphi$是$\rr^n$中的开集的嵌入,则对于任意的$p$-形式$\dd x_I=\dd x_{i_1}\wedge\cdots\wedge\dd x_{i_p}$都有$\varphi^* \dd x_I =\dd x_I$,因此
\[
\varphi^*\omega=\varphi^*\sum_If_I\dd x_I=\sum_If_I\circ\varphi \dd x_I.
\]
那么$I^*(\omega)=0$就是说$i_1^*(\omega)=i_2^*(\omega)=0$,这就是说$f_I\circ i_1=f_I\circ i_2=0$,但是由于$U_1$和$U_2$是$U_1\cup U_2$的一个开覆盖,所以这就等价于$f_I=0$,所以$\omega=0$.

现在证明$J^*$是一个满射。将单位分解应用到$U_1$和$U_2$上面取,存在$p_\nu$为定义在$U_1\cup U_2$上的光滑函数,而他的非零点集包含于$U_\nu$,且$p_1(x)+p_2(x)=1$.

设$f$定义在$U_1\cap U_2$上。定义$U_1$上的光滑函数$f_1$,他在$U_1\cap U_2$上的限制为$f(x)p_2(x)$和$U_2$上的光滑函数$f_2(x)$,他在$U_1\cap U_2$上的限制为$-f(x)p_1(x)$.那么在$U_1\cap U_2$上$f_1(x)-f_2(x)=f(x)$.

所以任选一个$U_1\cap U_2$上的$p$-形式$\omega$,系数$f_I$都可以定义出$f_{1,I}$和$f_{2,I}$,并且在$U_1\cap U_2$上满足$f_{1,I}-f_{2,I}=f$,因此也定义了两个$\omega_1$和$\omega_2$得到$J^p(\omega_1,\omega_2)=\omega$.

然后证明$\ker J^p=\mathrm{Im}\, I^p$. 分两个包含。

\begin{compactenum}[(1)]
\item $\mathrm{Im}\, I^p\subset \ker J^p$
\[
J^p\circ I^p(\omega)=j_2^*\circ i_2^*(\omega)-j_1^*\circ i_1^*(\omega)
\]
但其实$i_2\circ j_2=i_1\circ j_1$,所以$J^p\circ I^p(\omega)=0$.

\item $\ker J^p\subset \mathrm{Im}\, I^p$

设$\omega_1=\sum_I f_I \dd x_I\in \Omega^p(U_1)$和$\omega_2=\sum_I g_I \dd x_I\in \Omega^p(U_2)$,从$J^p(\omega_1,\omega_2)=0$我们有$j_1^*(\omega_1)=j_2^*(\omega_2)$,这就是说$f\circ j_1=g\circ j_2$,或者说$f$和$g$在$U_1\cap U_2$恒等。我们可以构成一个光滑函数$h_I$,他在$U_1$上的限制恒等于$f_I$,而$U_2$上恒等于$g_I$.那么
\[
I^p\left(\sum_Ih_I\dd x_I\right)=(\omega_1,\omega_2).
\]
\end{compactenum}
\end{proof}

将Theorem \ref{longexact} 和 Proposition \ref{directsum} 应用到上面这个定理。就得到下面这个定理。

\begin{thm}[Mayer-Vietoris列]
设$U_1$和$U_2$是$\rr^n$中的开集,则有如下的正合列:
\[
\cdots\to H^p(U_1\cup U_2)\xrightarrow{I^*}H^p(U_1)\oplus H^p(U_2)\xrightarrow{J^*}H^p(U_1\cap U_2)\xrightarrow{\partial^*}H^{p+1}(U_1\cup U_2)
\to \cdots
\]
其中$I^*([\omega])=(i_1^*([\omega]),i_2^*([\omega]))$,$J^*([\omega_1],[\omega_2])=j_1^*([\omega_1])-j_2^*([\omega_2])$.
\end{thm}

如果$U_1\cap U_2=\varnothing$,那么$H^p(U_1\cap U_2)=0$.所以
\[
0\xrightarrow{\partial^*} H^p(U_1\cup U_2)\xrightarrow{I^*}H^p(U_1)\oplus H^p(U_2)\xrightarrow{J^*}0,
\]
那么$I^*$既单又满,故而是个同构。

如果我们已知$U_1$和$U_2$的上同调群,那么通过Mayer-Vietoris列我们就有可能计算他们的并或者交的上同调群。

\para \label{homotopy}两个连续函数$f$, $g:X\to Y$被称为\idx{同伦}的,就是说存在一个连续函数$H:X\times [0,1]\to Y$使得$H(0,x)=f(x)$以及$H(1,x)=g(x)$.同伦是等价关系,就是说,如果还有$h$和$f$同伦,则$g$和$h$也同伦。一个空间$X$是可缩的,如果$\id_X$同伦于映到自身的常值映射。比如$\rr^n$或者与他同胚的开实心球$D^n$都是可缩的。

两个集合$X$和$Y$称为同伦等价的,如果存在$f:X\to Y$和$g:Y\to X$满足$f\circ g$和$\id_Y$同伦以及$g\circ f$和$\id_X$同伦。下面我们尝试说明,上同调群是同伦不变量。即在同伦意义下,上同调群是同构的。

\begin{lem}
光滑函数的同伦的一个技术性引理(见单位分解的附录):在欧氏空间背景下,任何一个连续映射都同伦于一个光滑映射。如果两个光滑函数$f_1$, $f_2:U\to V$是同伦的,则存在光滑函数$F:U\times \rr\to V$满足$F(x,0)=f_1(x)$和$F(x,1)=f_2(x)$.
\end{lem}

\begin{pro}
两条链复形和两个映射$f$, $g:A^*\to B^*$如果对每一个$p$都存在线性映射$s^p:A^p \to B^{p-1}$满足
\[
\dd_B^{p-1}s^p+s^{p+1}\dd_A^p=f^p-g^p:A^p\to B^p.
\]
则$f^*=g^*$.
\end{pro}

这样的两个映射被称为链同伦的。

\begin{proof} 对任意的$[a]\in H^p(A^*)$,我们有$\dd_A^p(a)=0$,所以
\[
(f^*-g^*)[a]=[(f^p-g^p)a]=[\dd^{p-1}_Bs^p(a)+s^{p+1}\dd_A^p(a)]=[\dd^{p-1}_Bs^p(a)],
\]
而$\dd^{p-1}_Bs(a)$显然被等价为0,所以$f^*=g^*$.\end{proof}

\begin{pro}
如果两个光滑函数$f$, $g:U\to V$是同伦的,则$
f^*$, $g^*:\Omega^*(V)\to \Omega^*(U)$是链同伦的,即$f^*=g^*$.
\end{pro}

\begin{proof} 利用我们的技术性引理,由于$f$, $g:U\to V$是同伦的,所以存在光滑函数$F:U\times \rr \to V$使得$F(x,0)=f(x)$和$F(x,1)=g(x)$.

注意到任意的$U\times \rr$上的$p$-形式可以写作
\[
\omega=\sum_If_I(x,t)\dd x_I+\sum_J g_J(x,t)\dd t\wedge \dd x_J.
\]
让$\varphi_0:x\mapsto (x,0)$和$\varphi_1:x\mapsto (x,1)$,则$F\circ \varphi_0=f$和$F\circ \varphi_1=g$,且将上面的形式分别拉回到
\[
\varphi_0^*\omega=\sum_I f_I(x,0)\dd x_I,\quad
\varphi_1^*\omega=\sum_I f_I(x,1)\dd x_I.
\]

现在我们需要构造一个$S^p:\Omega^p(U\times \rr)\to \Omega^{p-1}(U)$使得
\[
(\dd \circ S^p+S^{p+1}\circ \dd)(\omega)=(\varphi_1^*-\varphi_0^*)(\omega),
\]
如果这样,对于任意$U$中的形式我们有
\[
(\dd \circ S^p+S^{p+1}\circ \dd)(F^*\omega)=(\varphi_1^*-\varphi_0^*)(F^*\omega)=((F\circ \varphi_1)^*-(F\circ \varphi_0)^*)(\omega)=(g^*-f^*)(\omega),
\]
而最左边又有
\[
(\dd \circ S^p\circ F^*+S^{p+1}\circ F^* \circ \dd)(\omega)
\]
所以只要定义$s^p=S^p\circ F^*$,这就是链同伦。

为此定义
\[
S^p(\omega)=\sum_J\left(\int_0^1g_J(x,t)\dd t\right)\dd x_J.
\]\end{proof}

利用这个命题,我们可以知道,同伦等价对应到上同调群就有了上同调群的同构,因此上同调群只依赖于同伦型。

\para 由于对于可缩开子集$0^*=\id^*$,所以如果$U\subset \rr^n$可缩,那么$U$上的闭形式是恰当形式。这被称为Poincar\'{e}引理,通过他,我们知道可缩开集$U$的上同调群如下,$H^0(U)=\rr$,而对于$p>0$,则为$H^p(U)=0$.

\begin{para}[用Mayer-Vietoris列计算$H^{p}(\rr^2-\{0\})$]
设$U_1$为去掉正实轴(包括原点)的平面,$U_2$为去掉负实轴(包括原点)的平面,因此$U_1\cup U_2=\rr^2-\{0\}$.两者都是可缩的,所以由Poincar\'{e}引理可以知道,$H^0(U_1)=H^0(U_2)=\rr$以及如果$p>0$有$H^p(U_1)=H^p(U_2)=0$.

注意到$U_1\cap U_2$为去掉实轴的平面$I_1\cup I_2$,两个部分无交且各自可缩,所以
\[
H^p(U_1\cap U_2)=H^p(I_1\cup I_2) \cong H^p(I_1)\oplus H^p(I_2) =\begin{cases}
\rr\oplus\rr&,p=0;\\
0&,p>0.
\end{cases}
\]
而$\rr^2-\{0\}$是连通的,所以$H^0(\rr^2-\{0\})=\rr$.

当$p>0$的时候,代入Mayer-Vietoris列
\[
H^p(U_1\cap U_2)\xrightarrow{\partial^*}H^{p+1}(U_1\cup U_2)\xrightarrow{I^*}H^{p+1}(U_1)\oplus H^{p+1}(U_2)\xrightarrow{J^*}H^{p+1}(U_1\cap U_2),
\]
头尾通过计算都为$0$,所以
\[
H^{p+1}(\rr^2-\{0\})=H^{p+1}(U_1\cup U_2)\cong H^{p+1}(U_1)\oplus H^{p+1}(U_2)=0.
\]
这就是说,平面挖一个洞的$2$阶以上的上同调群为$0$.

现在考察一阶$\rr^2-\{0\}$的上同调群,由于负阶都为$0$,所以
 \[
0\to H^0(U_1\cup U_2)\xrightarrow{I^*}H^0(U_1)\oplus H^0(U_2)\xrightarrow{J^*}H^0(U_1\cap U_2)\xrightarrow{\partial^*}H^{1}(U_1\cup U_2)\to 0
\]
或者
 \[
0\to \rr\xrightarrow{f}\rr\oplus\rr\xrightarrow{g}\rr\oplus\rr\xrightarrow{h}H^{1}(\rr^2-\{0\})\to 0.
\]
由于$f$是单射而且正和性给出$\ker g = \mathrm{Im}\,f$,所以$\ker g =\rr$,而由线性代数基本定理,有$\mathrm{Im}\,g \cong \rr$,因此正合列给出$\ker h \cong \rr$,而因为$h$是满射,所以根据同构基本定理$H^{1}(\rr^2-\{0\})\cong(\rr\oplus\rr)/\ker h \cong \rr$.

综上,\[H^{p}(\rr^2-\{0\})\cong
\begin{cases}
\rr&,p=0,1;\\
0&,p>1.
\end{cases}\]
\end{para}

\begin{para}[用Mayer-Vietoris列计算$H^{p}(\rr^{n+1}-\{0\})$]
将$\rr^n$看做$\rr^{n+1}$的子空间,设$A$是$\rr^n$中的闭子集,则
\[
\begin{cases}
H^{p+1}(\rr^{n+1}-A)\cong H^p(\rr^n-A),&\text{when}\,p>1,\\
H^{1}(\rr^{n+1}-A)\cong H^0(\rr^n-A)/\rr,&\\
H^{0}(\rr^{n+1}-A)\cong \rr.&
\end{cases}
\]

通过归纳法可以得到
\[H^{p}(\rr^n-\{0\})\cong
\begin{cases}
\rr&,p=0,n-1;\\
0&,\text{otherwise}.
\end{cases}\]
这个结论可以用来证明$\rr^m$和$\rr^n$之间不存在同胚,如果存在,将同胚调整为$0$映射到$0$,而由于同胚将产生上同调群间的同构,所以$H^{p}(\rr^n-{0})$与$H^{p}(\rr^m-{0})$对于任意$p$都是同构的,但这不可能。

由于当$n>0$的时候,$S^{n-1}$和$\rr^n-\{0\}$同胚(当然也自然同伦等价),所以我们也计算出了球面的上同调群为
\[H^{p}(S^{n})\cong
\begin{cases}
\rr&,p=0,n;\\
0&,\text{otherwise}.
\end{cases}\]
\end{para}

上面一些计算出的上同调群可以产生许多有名经典的拓扑结论,比如Jordan-Brouwer分割定理,Brouwer不动点定理等。最简单的,比如$S^n$和$S^m$在$n\neq m$时候不同胚。

\section{积分}

在一般的微积分教材中,积分分两种,一种是曲线(曲面)积分,一种是按测度的积分。我们这节中分别讨论二者,前者关于链的积分,后者是关于密度的积分。链是积分路径的推广,而密度是测度在流形上的推广。

\begin{para}[单形]
对于$p\geq 0$,以及$\rr^{p+1}$中的$p+1$个矢量$\{v_i:0\leq i\leq p\}$满足$\{v_i-v_0:1\leq i\leq p\}$是一个线性无关组,我们定义$p$-单形为
\[
	[v_0,\dots,v_p]=\left\{\sum_{i=0}^p a_iv_i\in \rr^{p+1}:\sum_{i=1}^p a_i=1,\text{and each } a_i\leq 0\right\}.
\]
一个标准$p$-单形为$[e_0$, $\dots$, $e_p]$,其中$\{e_i\}_{0\leq i \leq p}$是$\rr^{p+1}$的标准基,可以看到,每一个$p$单形都和标准$p$-单形同胚。但是,流形$M$上的一个(光滑)$p$-单形是指一个(光滑)映射$\sigma:[v_0$, $\dots$, $v_p]\to M$,这是光滑曲线的自然推广,因为流形上的光滑曲线就是一个1-单形。我们常说沿着某条曲线积分,这里我们就将推广到沿着某个单形积分。
\end{para}

一个0-单形是一个点$1$,一个1-单形$[v_0,v_1]$是一个线段,端点为$v_0$和$v_1$,一个2-单形是一个三角形,他的三个顶点位于$v_0$, $v_1$和$v_2$,一个3-单形位于四维空间,不太好想,但是从上面的类比,他应该是一个四面体,顶点位于$v_0$, $v_1$, $v_2$和$v_3$.记$M$上全部$p$-单形生成的自由Abel群(或者称$\zz$-模)为$C_p(M)$,当然,这里两个单形之间的加法的定义意义似乎还不太明朗。我们将$C_p(M)$中的元素称为$p$-链,显然,一个$p$-单形就是一个$p$-链。

\begin{para}
一个$p$-单形内的每一个点都可以写作$\sum_{i=0}^p a_iv_i$的形式,我们称某个$(p-1)$-边界是指某个$a_i=0$的情况,一个$(p-1)$边界是自然的$(p-1)$-单形,记做$[v_0$, $\dots$, $\hat{v}_i$, $\dots$, $v_p]$。显然,一个$p$-单形有$p+1$个边界,比如一个三角形,是一个2-单形,有三条边。

对于流形$M$上的一个$p$-单形$\sigma:[v_0,\dots,v_p]\to M$,定义他的第$(i+1)$个边缘为一个$(p-1)$-链$\sigma|_{[v_0,\dots,\hat{v}_i,\dots,v_p]}$。
\end{para}

\para 在流形$M$上,定义边缘算子$\partial:C_p(M)\to C_{p+1}(M)$:对单项式,
\[
	\partial \sigma=\sum_{i=0}^p (-1)^i\sigma|_{[v_0,\dots,\hat{v}_i,\dots,v_p]}.
\]
对一般的$\sigma=\sum_i n_i \sigma_i$,定义$\partial \sigma =\sum_i n_i \partial \sigma_i$. 特别地,我们有$\partial^2=0$.

\begin{proof} 直接对单项式计算
\[
\begin{aligned}
	\partial^2 \sigma&=\sum_{i=0}^p (-1)^i\partial\sigma|_{[v_0,\dots,\hat{v}_i,\dots,v_p]}\\
	&=\sum_{0\leq i<j\leq n}^p (-1)^{i+j-1}\bigl(\sigma|_{[v_0,\dots,\hat{v}_i,\dots,\hat{v}_j,\dots,v_p]}-\sigma|_{[v_0,\dots,\hat{v}_i,\dots,\hat{v}_j,\dots,v_p]}\bigr)=0.\qedhere
\end{aligned}
\]
\end{proof}

$\partial^2=0$和$\dd^2=0$是如此的相似!实际上,我们可以定义$H_k(M)=\ker\partial/\im\partial$,这就是流形$M$的第$k$个奇异同调群。这里不展开,我们暂时的目标还是积分。

\para 由于每一个$[v_0$, $\dots$, $v_p]$都同胚与标准$p$-单形,所以,在同胚意义下,以标准$p$-单形为定义域的流形上的$p$-单形与定义域为花式的$[v_0$, $\dots$, $v_p]$的流形上的$p$-单形应该是一样多的。所以我们可以限定,流形上的$p$-单形的定义域为标准$p$-单形。

\para 由于$[v_0$, $\dots$, $v_p]$里面的点写作$\sum_i a_i v_i$,注意到$\sum_i a_i=1$,我们可以将其改写为
\[
	\sum_{i=0}^pa_iv_i=\left(1-\sum_{i=1}^pa_i\right)v_0+\sum_{i=1}^pa_iv_i=v_0+\sum_{i=1}^pa_i(v_i-v_0),
\]
所以,除去一个平移$v_0$,我们完全可以由$\sum_{i=1}^pa_i(v_i-v_0)$确定一个$p$-单形。显然,他们是光滑同胚的。所以对于$M$上的$p$-单形,我们在光滑同胚意义下,还可以假设定义域为
\[
	\Delta^p=\left\{(a_1,\dots,a_p):a_i\geq 0\text{ and }\sum_i a_i\leq 1\right\},
\]
这时候,$\sigma:\Delta^p\to M$常常会写作$\sigma(a_1,\dots,a_p)$.

此时,对于边缘算子,这种约定下,符号要变得复杂一些,记
\[
	k^p_0(a_1,\dots,a_{p})=\left(1-\sum_{i=1}^{p-1}a_i,\dots,a_{p}\right),
\]
当$i>0$时,
\[
	k^p_i(a_1,\dots,a_{p})=(a_1,\dots,a_{i-1},0,a_{i+1},\dots,a_{p}),
\]
以及$\sigma^i=\sigma\circ k^p_i$,则
\[
	\partial\sigma=\sum_{i=0}^p(-1)^i \sigma^i,
\]
不难检验,$\partial^2=0$还是成立的,下面我们会暂时采用这种符号约定。

\para 设$\omega$是$\rr^n$中的$n$-形式,写作$f\dd x^1\wedge\cdots\wedge \dd x^n$,则我们定义他在区域$A$上的积分算符为线性函数
\[
	\int_A \omega=\int_A f \dd x^1\cdots\dd x^n,
\]
有时候为了省空间,我们以$\Int_A$来记$\int_A$. 并且,反过来,我们也将欧式空间里的$n$-重积分写作对微分形式积分的样子
\[
	\int_A f \dd x^1\cdots\dd x^n=\int_A f\dd x^1\wedge\cdots\wedge \dd x^n.
\]

有了这个定义,我们可以重写积分变量替换公式为
\[
	\int_{\varphi(A)}\omega=\operatorname{sign}\left(\det(\varphi_*)\right)\int_A \varphi^*\omega.
\]

\begin{para}[链的积分]
设$\sigma$是一个$M$上的0-单形,即一个点$\sigma(0)\in M$,定义在$\sigma$上的积分为$\Int_\sigma \omega=\omega(\sigma(0))$.

对于$p\geq 1$的情况,设$\sigma:\Delta^p\to M$是一个$M$上的一个$p$-单形,这里$p>0$,那么$\sigma$诱导了一个拉回映射$\sigma^*$,我们定义一个$p$-形式$\omega$在$\sigma$上的积分如下:
\[
	\int_\sigma \omega=\int_{\Delta^p} \sigma^*\omega,
\]
注意到等式右边就是一个$\rr^{p}$内对$p$-形式的积分,这个积分上面我们已经定义了。

当在一条链上积分的时候,比如$c=\sum_{i} n_i \sigma_i$,我们定义
\[
	\int_c \omega=\sum_i n_i \int_{\sigma_i} \omega.
\]
此时,单形之间的加法在积分的意义下清楚了不少。
\end{para}

\para 设$f:M\to N$是一个光滑映射,设$\sigma$是$M$上的一个单形,则$f\circ \sigma$是$N$上的一个单形,那么对$N$上的$p$-形式
\[
	\int_{f\circ\sigma} \omega = \int_{\Delta^p}(f\circ \sigma)^*\omega= \int_{\Delta^p}\sigma^*\circ f^*(\omega)=\int_{\sigma}f^*\omega.
\]
这推广了重积分变量替换公式。

\begin{thm}[微积分基本定理]
设$c$是$M$上的一个1-链,而$f$是$M$上的一个光滑函数,则
\[
	\int_{\partial c} f=\int_c \dd f.
\]
对于$c$是一个1-单形的时候,这个定理翻译成我们熟知的语言即$\int_a^b \dd f=f(b)-f(a)$,而两边对于$c$都是线性的,所以上述定理对链依然成立。
\end{thm}

下面一个定理时常也被叫做微积分基本定理。

\begin{thm}[Stocks定理]
设$c$是$M$上的一个$p$-链,而$\omega$是$M$上的一个$(p-1)$-单形,则
\[
	\int_c \dd \omega=\int_{\partial c} \omega.
\]
\end{thm}

这是两个(一个)极伟大的定理,尽管证明他并不是特别困难。他的重要性无需强调,每一个做过定积分(包括曲面积分等)计算的人都知道他多么有用。

\begin{proof} 对于$p=1$的情况,这就是微积分基本定理,所以我们下面假设$p\leq 2$.由于积分对链是线性的,所以我们只要对一个单形去证明就可以了,即
\[
	\int_{\sigma} \dd \omega=\int_{\partial \sigma} \omega,
\]
利用积分的定义,我们把单形拉回到欧式空间里面,证明几乎就可以在欧式空间里面进行,即
\[
	\int_{\Delta^p} \sigma^*(\dd \omega)=\sum_{i=0}^p(-1)^i\int_{\Delta^{p-1}} (\sigma^i)^*\omega=\sum_{i=0}^p(-1)^i\int_{\Delta^{p-1}} (k^p_i)^*\circ \sigma^*(\omega).
\]

我们假设
\[
	\sigma^*\omega=\sum_{i=1}^pa_i\dd x^1\wedge\cdots\wedge\widehat{\dd x^i}\wedge\cdots \dd x^p,
\]
所以等式左边写作
\[
\begin{aligned}
	\int_{\Delta^p} \sigma^*(\dd \omega)=\int_{\Delta^p} \dd(\sigma^* \omega)&=\sum_{i=1}^p\int_{\Delta^{p}} \dd a_i\wedge \dd x^1\wedge\cdots\wedge\widehat{\dd x^i}\wedge\cdots\wedge\dd x^p\\
	&=\sum_{i=1}^p\int_{\Delta^{p}} (-1)^{i-1}\partial_i a_i\dd x^1\wedge\cdots \wedge\dd x^p.
\end{aligned}
\]

现在来看右边,由于
\[(k^p_0)^*(x^j)=
\begin{cases}
	1-\sum_{i=1}^{p-1}x^i,& j =1;\\
	x^{j-1},& j>1,
\end{cases}\quad
(k^p_i)^*(x^j)=
\begin{cases}
	x^j,& 1\leq j \leq i-1;\\
	0,& j=i;\\
	x^{j-1},& i+1\leq j \leq p.
\end{cases}
\]
所以
\[
\begin{aligned}
	\sum_{i=0}^p(-1)^i\int_{\Delta^{p-1}}& (k^p_i)^*(a_j\dd x^1\wedge\cdots\wedge\widehat{\dd x^j}\wedge\cdots \dd x^p)\\
	&=(-1)^{j-1}\int_{\Delta^{p-1}} (a_j\circ k^p_0)\dd x^1\wedge\cdots \wedge\dd x^{p-1}+(-1)^j\int_{\Delta^{p-1}} (a_j\circ k_j^p)\dd x^1\wedge\cdots \wedge\dd x^{p-1},
\end{aligned}
\]
对右侧第一项做一个适当的变量替换$\varphi^i$如下
\[
	\varphi^i(x)=\begin{cases}
		x,& i=1;\\
		\bigl(1-\sum_{i=1}^{p-1}x^i,x^2,\cdots,x^{p-1}\bigr),& i=2;\\
		\bigl(x^2,\cdots,x^{i-1},1-\sum_{i=1}^{p-1}x^i,x^{i+1},\cdots,x^{p-1}\bigr),& 2<i\leq p.
	\end{cases}
\]
容易证明$\varphi^i(\Delta^{p-1})=\Delta^{p-1}$以及$\det ((\varphi^i)_*)=1$,所以,由重积分变量替换公式,我们得到
\[
\begin{aligned}
	\sum_{i=0}^p&(-1)^i\int_{\Delta^{p-1}} (k^p_i)^*(a_j\dd x^1\wedge\cdots\wedge\widehat{\dd x^j}\wedge\cdots \dd x^p)\\
	=&(-1)^{j-1}\int_{\Delta^{p-1}} a_j\left(x^1,\cdots,x^{j-1},1-\sum_{i=1}^{p-1}x^i,x^{j},\cdots,x^p\right)\dd x^1\wedge\cdots \wedge\dd x^{p-1}+\\
	&(-1)^{j}\int_{\Delta^{p-1}}a_j(x^1,\cdots,x^{j-1},0,x^{j},\cdots,x^p)\dd x^1\wedge\cdots \wedge\dd x^{p-1},
\end{aligned}
\]
所以最后我们只需要证明
\[
\begin{aligned}
	\int_{\Delta^{p}}\partial_i a_i\dd x^1\wedge\cdots \wedge\dd x^p=
	&\int_{\Delta^{p-1}} a_j\left(x^1,\cdots,x^{j-1},1-\sum_{i=1}^{p-1}x^i,x^{j},\cdots,x^p\right)\dd x^1\wedge\cdots \wedge\dd x^{p-1}\\
	&-\int_{\Delta^{p-1}}a_j(x^1,\cdots,x^{j-1},0,x^{j},\cdots,x^p)\dd x^1\wedge\cdots \wedge\dd x^{p-1},
\end{aligned}
\]
注意这其实是重积分,利用一元的微积分基本定理即得。\end{proof}

% \para 回到积分曲面的问题,我们谈过,积分曲面的存在性就在于曲边四边形的可拼合与否。这里,利用沿着曲线(1-单形)的积分,我们再来演示一次$\dd\omega$决定了他是否可积。

% 我们和以前一样考虑两条路径,他们分别沿着$X$方向走过$\epsilon$时间,再沿着$Y$方向走过$\epsilon$时间,或者次序反过来,他们的终点分布记做$P$和$Q$,那么取一条光滑曲线连接$PQ$,则我们就做成了一个回路。此时,取一个$\omega$使得$X$, $Y\in \ker(\omega)$,那么利用微积分基本定理,就可以把$\omega$沿着回路的积分化到曲面上关于$\dd \omega$的积分。

% 现在开始估计积分,从起点开始,每走$\epsilon$,高度就增加阶为$\epsilon^2$,那么如果有错开,他错开的垂直部分面积就应该至多是$\epsilon\cdot \epsilon^2=\epsilon^3$的阶,否则积分曲面就不存在。而$\dd \omega$关于回路围成的曲面的积分就是垂直错开部分的面积,即一个三阶小量。

% 利用微积分基本定理,我们就得到了,$\omega$沿着一条路径(先$X$和后$Y$或者反过来),与沿着另一条路径的积分,就应该至多相差一个三阶小量。因此,积分曲面存在当且仅当$\dd \omega|_{\omega=0}=0$.

\begin{para}[密度]

\end{para}