% author: buwailee@nmhs
\documentclass[11pt]{article}
\usepackage{amssymb,amsfonts,amsmath,amsthm,bm,mathrsfs,color}
\usepackage[a4paper, top=16mm, text={170mm, 248mm}, includehead, includefoot, hmarginratio=1:1, heightrounded]{geometry}
\usepackage[dvipdfmx]{hyperref}

\theoremstyle{definition}
\newtheorem{para}{}[part]
\renewcommand{\thepara}{\arabic{para}}

\theoremstyle{plain}
\newtheorem{thm}[para]{Theorem}
\newtheorem{pro}[para]{Proposition}
\newtheorem{lem}[para]{Lemma}

\usepackage{paralist}

\definecolor{shadecolor}{rgb}{0.92,0.92,0.92}

\newcommand{\no}[1]{{$(#1)$}}
% \renewcommand{\not}[1]{#1\!\!\!/}
\newcommand{\rr}{\mathbb{R}}
\newcommand{\zz}{\mathbb{Z}}
\newcommand{\aaa}{\mathfrak{a}}
\newcommand{\pp}{\mathfrak{p}}
\newcommand{\mm}{\mathfrak{m}}
\newcommand{\dd}{\mathrm{d}}
\newcommand{\oo}{\mathcal{O}}
\newcommand{\calf}{\mathcal{F}}
\newcommand{\calg}{\mathcal{G}}
\newcommand{\bbp}{\mathbb{P}}
\newcommand{\bba}{\mathbb{A}}
\newcommand{\osub}{\underset{\mathrm{open}}{\subset}}
\newcommand{\csub}{\underset{\mathrm{closed}}{\subset}}

\DeclareMathOperator{\im}{Im}
\DeclareMathOperator{\Hom}{Hom}
\DeclareMathOperator{\id}{id}
\DeclareMathOperator{\rank}{rank}
\DeclareMathOperator{\tr}{tr}
\DeclareMathOperator{\supp}{supp}
\DeclareMathOperator{\coker}{coker}
\DeclareMathOperator{\codim}{codim}
\DeclareMathOperator{\height}{height}
\DeclareMathOperator{\sign}{sign}

\DeclareMathOperator{\ann}{ann}
\DeclareMathOperator{\Ann}{Ann}
\DeclareMathOperator{\ev}{ev}
\newcommand{\cc}{\mathbb{C}}

\begin{document}
\title{Connection on Bundle}
\author{Buwai Lee@NJU}
\date{\today}
\maketitle

The manifolds in this article are all smooth.

\section{$G$-bundle and Principal Bundle}

\begin{para}[Fiber bundle]
A fiber bundle $(E,M,\pi,F)$ consists of three manifold, total space $E$, base space $M$ and fiber space $F$, and a smooth map $\pi:E\to M$, such that for all $p\in M$, there exists an open neighbourhood $U$ of $p$ and a homeomorphism $\varphi:\pi^{-1}(U)\to U\times F$. 
Suppose on $U_{\alpha}$, there's a homeomorphism $\varphi_\alpha=(\pi,\Phi_\alpha):\pi^{-1}(U_\alpha)\to U\times F$, and on $U_\beta$, a homeomorphism $\varphi_\beta=(\pi,\Phi_\beta):\pi^{-1}(U_\beta)\to U\times F$. Then for all $p\in U_\alpha\cap U_\beta$,
\[
	\Phi_{\alpha}|_{\pi^{-1}(p)}:\pi^{-1}(p)\to F\quad \text{and}\quad \Phi_{\beta}|_{\pi^{-1}(p)}:\pi^{-1}(p)\to F
\]
are homeomorphism, so $\varphi_{\alpha\beta}(p):=\Phi_{\alpha}|_{\pi^{-1}(p)}(\Phi_{\beta}|_{\pi^{-1}(p)})^{-1}:F\to F$ is a homeomorphism such that
\begin{compactenum}
	\item $\varphi_{\alpha\alpha}(p)=\id_F$;
	\item $\varphi_{\alpha\beta}(p)\varphi_{\beta\alpha}(p)=\id_F$;
	\item $\varphi_{\alpha\beta}(p)\varphi_{\beta\gamma}(p)\varphi_{\gamma\alpha}(p)=\id_F$.
\end{compactenum}
\end{para}

\begin{para}[$G$-bundle]
Suppose $G$ is a Lie group, $(E,M,\pi,F)$ is a fiber bundle and $\lambda:G\times M\to M$ is a group action. Let $\{U_\alpha\}_{\alpha\in \Gamma}$ be a cover of $M$ and $\varphi_\alpha$ be the local trivialization for $U_\alpha$, if there exists a $g_{\alpha\beta}(p)\in G$ such that $\varphi_{\alpha}\circ \varphi_{\beta}^{-1}(p,v)=(p,\lambda(g_{\alpha\beta}(p),v))$ is valid for all $p\in U_\alpha\cap U_\beta$ and all $v\in F$. 

Note, it is not necessary to say that 
\begin{compactenum}
	\item $g_{\alpha\alpha}(p)=\id_F$;
	\item $g_{\alpha\beta}(p)g_{\beta\alpha}(p)=\id_F$;
	\item $g_{\alpha\beta}(p)g_{\beta\gamma}(p)g_{\gamma\alpha}(p)=\id_F$.
\end{compactenum}
since the group action is not necessary to be {effective}. If these conditions is valid, then $(E,M,\pi,F)$ is a $G$-bundle. Set $\{g_{\alpha\beta}\}$ is usually called a \textit{$G$-cocycle} for cover $\{U_\alpha\}_{\alpha\in \Gamma}$.
\end{para}

\begin{thm}[$G$-bundle construction theorem]
	Let $M$ and $F$ be smooth manifolds and let $G$ be a Lie group. Let $\{U_\alpha\}_{\alpha\in \Gamma}$ be a cover of $M$ and $\{g_{\alpha\beta}\}$ be a $G$-cocycle. For {every} action $\lambda:G\times F\to F$, there exists a fiber bundle with bundle-atlas $\{U_\alpha,\varphi_\alpha\}$ satisfying $\varphi_\alpha\circ \varphi_\beta^{-1}(p,y)=(p,\lambda(g_{\alpha\beta}(p),y))$ on noempty overlaps $U_\alpha\cap U_\beta$. Thus the resulting bundle has a $(G,\lambda)$-bundle structure, or simply $G$-bundle.
\end{thm}

\begin{proof}
	On the disjoint union $\Sigma:=\bigsqcup_{\alpha\in \Gamma}U_\alpha\times F$ define an equivalence relation such that $(p,y)\in U_\alpha\times F$ is equivalent to $(p',y')\in U_\beta\times F$ iff $p=p'$ and $y=g_{\alpha\beta}(p)\cdot y'=\lambda(g_{\alpha\beta}(p),y')$. This equivalent relation is well-defined because of the cocycle conditions. 
	
	The total space of our bundle is then $\Sigma/\!\!\sim$ whose elements are the equivalence class $\epsilon$. Define $\pi:(p,y)\mapsto p$ and $\varphi_\alpha(\epsilon)$ be the unique member $(p,y)$ of $U_\alpha\times F$ such that $[p,y]:=(p,y)/\!\!\sim\,\,\in \epsilon$, then $(\Sigma/\!\!\sim,M,\pi,F)$ is what we need.
\end{proof}

\begin{para}[Vector bundle]
	Suppose $\lambda$ is a representation of Lie group $G$ on $V$, then $(G,\lambda)$-bundle $(E,M,\pi,V)$ is called a vector bundle with with typical $V$.

	On a vector bundle $(E,M,\pi,V)$, there's a natural vector space structure on each fiber $E_p:=\pi^{-1}(p)\cong V$. In fact, suppose $(U_\alpha,\varphi_\alpha)$ is a local trivialization near $p$, $a$ is a scale and $s$, $t\in \pi^{-1}(p)$, we can define
	\[
	s+t:=\Phi_{\alpha}|_{E_p}^{-1}\left(\Phi_{\alpha}|_{E_p}(s)+\Phi_{\alpha}|_{E_p}(t)\right),\quad a\cdot s:=\Phi_{\alpha}|_{E_p}^{-1}\left(a\Phi_{\alpha}|_{E_p}(s)\right).
	\]
	It is well-defined. If $(U_\beta,\varphi_\beta)$ is another trivialization near $p$, then
	\[
	\begin{aligned}
	\Phi_{\alpha}|_{E_p}^{-1}\left(\Phi_{\alpha}|_{E_p}(s)+\Phi_{\alpha}|_{E_p}(t)\right)&=\varphi_\beta|_{E_p}^{-1}\varphi_{\beta\alpha}(p)\left(\Phi_{\alpha}|_{E_p}(s)+\Phi_{\alpha}|_{E_p}(t)\right)\\
	&=\Phi_\beta|_{E_p}^{-1}\lambda\left(g_{\beta\alpha}(p),\Phi_{\alpha}|_{E_p}(s)+\Phi_{\alpha}|_{E_p}(t)\right)\\
	&=\Phi_\beta|_{E_p}^{-1}\left(\lambda\left(g_{\beta\alpha}(p),\Phi_{\alpha}|_{E_p}(s)\right)+\lambda\left(g_{\beta\alpha}(p),\Phi_{\alpha}|_{E_p}(t)\right)\right)\\
	&=\Phi_\beta|_{E_p}^{-1}\left(\varphi_{\beta\alpha}(p)\Phi_{\alpha}|_{E_p}(s)+\varphi_{\beta\alpha}(p)\Phi_{\alpha}|_{E_p}(t)\right)\\
	&=\Phi_\beta|_{E_p}^{-1}\left(\Phi_{\beta}|_{E_p}(s)+\Phi_{\beta}|_{E_p}(t)\right).
	\end{aligned}
	\]
	The check of $a\cdot s$ is similar.

	Conversely, suppose $(E,M,\pi,V)$ is a fiber bundle and $V$ is a vector space, if the transition maps are all linear isomorphisms of $V$. By $G$-bundle construction theorem, $(E,M,\pi,V)$ is a $G$-bundle, where $G$ is a subgroup of $\mathrm{GL}(V)$ and the representation is the standard representation.
\end{para}

\begin{para}[Principal $G$-bundle]
A principal $G$-bundle $(P,M,\pi,G)$ is a fiber bundle where $G$ is a Lie group, and there's a right free $G$-action on $P$, that is $\mu:P\times G\to P$ such that
\begin{compactenum}
	\item $\pi(u)=\pi(ug)$ for all $u\in P$ and $g\in G$;
	\item for all $p\in M$, there exists an open neighbourhood $U\subset M$ and a homeomorphism $\varphi=(\pi,\Phi):\pi^{-1}(U)\to U\times G$ such that
	\[
	\Phi(u)g=\Phi(ug)
	\]
	for all $u\in \pi^{-1}(U)$ and $g\in G$.
\end{compactenum}
\end{para}

\begin{pro}
The fiber of a principal bundle is exactly the orbit of the right action.
\end{pro}

\begin{proof}
	Fix $p\in M$. 

	$(\Leftarrow)$ part: For all $u\in\pi^{-1}(p)$ and $g\in G$,
	\[
		\pi(ug)=\pi(u)=p
	\]
	tells us that $ug\in \pi^{-1}(p)$. 

	$(\Rightarrow)$ part: For all $u$, $v\in\pi^{-1}(p)$, there's an open neighbourhood $U\subset M$ of $p$ such that
	\[
	\varphi=(\pi,\Phi):\pi^{-1}(U)\to U\times G,
	\]
	so $\Phi(u)$, $\Phi(v)\in G$. Let $g=\Phi(v)^{-1}\Phi(u)$, then
	\[
	\varphi(u)=(p,\Phi(u))=(p,\Phi(v)g)=(\pi(v),\Phi(vg))=(\pi(vg),\Phi(vg))=\varphi(vg).
	\]
	Since $\varphi$ is bijective, $u=vg$.
\end{proof}

As a manifold, $M\cong P/G$.

\begin{para}
What's more,
\[
	\Phi_{\alpha}(ug)\Phi_{\beta}(ug)^{-1}=\Phi_{\alpha}(u)gg^{-1}\Phi_{\beta}(u)^{-1}=\Phi_{\alpha}(u)\Phi_{\beta}(u)^{-1}.
\]
So $g_{\alpha\beta}=\Phi_{\alpha}(u)\Phi_{\beta}(u)^{-1}$ for all $u\in \pi^{-1}(p)$.
\end{para}

\begin{lem}\label{lem:8}
$\Phi_{\alpha}|_{\pi^{-1}(p)}\circ \left(\Phi_{\beta}|_{\pi^{-1}(p)}\right)^{-1}(g)=g_{\alpha\beta}(p)g$.
\end{lem}

\begin{proof}
	Let $\left(\Phi_{\beta}|_{\pi^{-1}(p)}\right)^{-1}(g)=u$, then $g=\Phi_\beta(u)$ and $\Phi_{\alpha}|_{\pi^{-1}(p)}\circ \left(\Phi_{\beta}|_{\pi^{-1}(p)}\right)^{-1}(g)=\Phi_{\alpha}(u)$. On the other hand, $u\in \pi^{-1}(p)$ and so
		\[
		g_{\alpha\beta}(p)g=\Phi_{\alpha}(u)\Phi_{\beta}(u)^{-1}\Phi_\beta(u)=\Phi_{\alpha}(u)=\Phi_{\alpha}|_{\pi^{-1}(p)}\circ \left(\Phi_{\beta}|_{\pi^{-1}(p)}\right)^{-1}(g).\qedhere
		\]
\end{proof}

So the transition function is just the left multiplication of $g_{\alpha\beta}(p)=\Phi_{\alpha}(u)\Phi_{\beta}(u)^{-1}$! Thus $P$ is indeed a $G$-bundle, the left action is free.

\begin{para}
Conversely, if $(P,M,\pi,G)$ is a $G$-bundle that $\lambda$ is just the left multiplication, then $(P,M,\pi,G)$ is a principal $G$-bundle.

In fact, define $ug=\varphi_{\alpha}^{-1}(p,\Phi_{\alpha}(u)g)$, where $p=\pi(u)$. It is well-defined. Suppose $\varphi_\beta$ is another trivialization, then $\varphi_\beta(ug)=(p,\Phi_{\beta\alpha}(u)\Phi_\alpha(u)g)=(p,\Phi_{\beta}(u)g)$. Therefore $\pi(ug)=\pi(u)$ and $\Phi_{\alpha}(ug)=\Phi_{\alpha}(u)g$, $(P,M,\pi,G)$ is really a principal $G$-bundle.
\end{para}

% As a corollary, in $(G,\lambda)$-vector bundle $(E,M,\pi,V)$, if $G$ is a subgroup of general linear group of $V$ and $\lambda$ is the standard representation, then $(E,M,\pi,V)$ is a principal $G$-bundle.

\begin{thm}
If $\pi:P\to M$ is a surjective submersion and a Lie group G acts freely on $P$ so that for each $p\in M$ and orbit of $p$ is exactly $\pi^{-1}(p)$, then $(P,M,\pi,G)$ is a principal $G$-bundle.
\end{thm}

\begin{proof}
	Without loss of generality, we can assume that the action is right action, since if it is a left action, we can define an equivalent right action by $p\cdot g:=g^{-1}p$.
	
	Since $\pi$ is a surjective submersion, for each $p\in M$, there exists a open neighbourhood $U\subset M$ and a local section $\sigma:U\to P$. Consider the map $f_\sigma:U\times G\to \pi^{-1}(U)$ given by $f_\sigma(p,g)=\sigma(p)g$. It is injective. If $\sigma(p)g=\sigma(p')g'$ is on the same orbit, then $\pi(\sigma(p)g)=p$ tell us $p=p'$, and $\sigma(p)g=\sigma(p)g'$ gives that $g=g'$ since the action is free. It is surjective, too. For each $u\in \pi^{-1}(U)$, $\sigma(\pi(u))\in \pi^{-1}(\pi(u))$, then there exists a $h\in G$ such that $u = \sigma(\pi(u))h$ since the orbit of $\pi(u)$ is exactly $\pi^{-1}(\pi(u))$, thus $u= f_\sigma(\sigma(\pi(u)),h)$.
	
	Now, suppose $\gamma(t)$ is a smooth curve on $\pi^{-1}(U)$ and $\gamma(0)=\sigma(p)g$, since the action is free, we can find a curve $g(t)$ on $G$ such that $\gamma(t)=\sigma(\pi(\gamma(t)))g(t)$. Let us define a linear map $\varphi_\sigma: T_{\sigma(p)g}P\to T_{p}M\oplus T_{g}G$ by $\gamma'(0) \mapsto (\pi\circ \gamma)'(0)\oplus g'(0)$, it is not difficule to vertify that it is the inverse of $(f_\sigma)_{*(p,g)}$. By inverse function theorem, bijection $f_\sigma:U\times G\to \pi^{-1}(U)$ is a local diffeomorphism, and then it is a diffeomorphism.
	
	Let $\varphi:=f_\sigma^{-1}$, then we have $\varphi=(\pi,\Phi)$ for a uniquely determined smooth map $\Phi:U\to G$. If $p=\pi(u)$, we have $\varphi(ug)=(p,\Phi(ug))$ and so
	\[
		ug=\varphi^{-1}(p,\Phi(ug)),
	\]
	while
	\[
	\begin{aligned}
		\varphi^{-1}(p,\Phi(u)g)&=f_\sigma(p,\Phi(u)g)=\sigma(p)(\Phi(u)g)=(\sigma(p)\Phi(u))g\\
		&=f_\sigma(p,\Phi(u))g=\varphi^{-1}(p,\Phi(u))g=ug=\varphi^{-1}(p,\Phi(ug)).
	\end{aligned}
	\]
	Since $\varphi^{-1}$ is a bijection, we have $\Phi(ug)=\Phi(u)g$. Thus the 	section $\sigma$ give rise to a principal bundle chart $(U,\varphi)$, where $\varphi=(\pi,\Phi)$.
\end{proof}

\begin{para}[Associated bundle]\label{11}
	Let $\pi:P\to M$ is a principal $G$-bundle, and suppose we are given a smooth left action $\lambda:G\times F\to F$ on a smooth manifold $F$. Then we can construct a $G$-bundle with base space $M$ and fiber space $F$ as follow.

	Define a right action of $G$ on $P\times F$ according to
	\[
	(u,y)g:=(ug,g^{-1}y)=(ug,\lambda(g^{-1},y)).
	\]
	The total space of our new bundle is the orbit space of this action $P\times_G F$. Denote the equivalence class of $(u,y)$ by $[u,y]$, and define $\bar{\pi}([u,y]):=\pi(u)$. $\bar{\pi}$ is a well-defined map because $\bar{\pi}([ug,g^{-1}y])=\pi(ug)=\pi(u)$. By the next lemma, we call bundle $(P\times_G F,M,\bar{\pi},F)$ the associated $G$-bundle of principal bundle $P$. 
\end{para}

If $\lambda$ is more important in the context, we usually use $P\times_\lambda F$ to denote the total space of associated bundle.

\begin{lem}
	$(P\times_G F,M,\bar{\pi},F)$ is a $G$-bundle with transition maps $\{\lambda(g_{\alpha\beta},*)\}$, where $g_{\alpha\beta}$ is the transition map of principal $G$-bundle $\pi:P\to M$.
\end{lem}

\begin{proof}
	Let $\{(U_\alpha,\varphi_\alpha)\}$ be a principal bundle atlas for $\pi:P\to M$. Note that $[\pi^{-1}(U_\alpha)\times F]=\pi^{-1}(U_\alpha)$. For each $\alpha$, define $\bar{\Phi}_\alpha:\pi^{-1}(U_\alpha)\to F$ by requiring that $\bar{\Phi}_\alpha([u,y])=\Phi_\alpha(u)\cdot y$ for all $[u,y]\in \pi^{-1}(U_\alpha)$ and then let $\bar{\varphi}_\alpha:=(\pi,\bar{\Phi}_\alpha)$ on $\pi^{-1}(U_\alpha)$. We want to show that $\bar{\varphi}_\alpha$ is bijective by defining an inverse for $\bar{\varphi}_\alpha$. For every $p\in U_\alpha$, let $\sigma_\alpha(p):=\varphi_\alpha^{-1}(p,e)$, where $e$ is the identity element in $G$. Then we have
	\[
	\sigma_\alpha(p) \cdot \Phi_\alpha(u)=\varphi_\alpha^{-1}(p,e)\cdot \Phi_\alpha(u)=\varphi_\alpha^{-1}(p,\Phi_\alpha(u))=u.
	\]
	Define $\eta_\alpha:U_\alpha\times F\to \pi^{-1}(U_\alpha)$ by $\eta_{\alpha}(p,y):=[\sigma_\alpha(p),y]$. We have
	\[
	\eta_\alpha\bar{\varphi}_\alpha([u,y])=\eta_\alpha(p,\Phi_\alpha(u)\cdot y)=[\sigma_\alpha(p),\Phi_\alpha(u)\cdot y]=[\sigma_\alpha(p)\cdot \Phi_\alpha(u),y]=[u,y].
	\]
	Thus $\eta_\alpha$ is a left inverse for $\bar{\varphi}_{\alpha}$. It is easily checked that $\eta_\alpha$ is also a right inverse for $\bar{\varphi}_{\alpha}$. Thus $\bar{\varphi}_\alpha$ is a bijection. Next we check the overlaps. We use Lemma \ref{lem:8};
	\[
	\begin{aligned}
	\bar{\varphi}_{\alpha}\bar{\varphi}_{\beta}^{-1}(p,y)&=\bar{\varphi}_{\alpha}\eta_\beta(p,y)=\bar{\varphi}_{\alpha}([\sigma_\beta(p),y])\\
	&=(p,\Phi_\alpha(\sigma_\beta(p))\cdot y)=(p,\Phi_\alpha(\varphi_\beta^{-1}(p,e))\cdot y)\\
	&=(p,\Phi_\alpha|_p\circ \Phi_\beta|_p^{-1}(e)\cdot y)\\
	&=(p,g_{\alpha\beta}(p)\cdot e\cdot y)\\
	&=(p,g_{\alpha\beta}(p)y).
	\end{aligned}
	\]
	This shows that the transition mappings  have the stated form and that the overlap maps $\bar{\varphi}_{\alpha}\bar{\varphi}_{\beta}^{-1}$ are smooth. The family $\{(U_\alpha,\bar{\varphi}_\alpha)\}$ provides the induced smooth structure and is also a bundle atlas.
\end{proof}

According to the above proof, the local trivializations of this bundle is $\bar{\varphi}_\alpha([u,y])=(\pi(u),\Phi_\alpha(u)y)$.

\begin{para}\label{13}
	Suppose that $(E,M,\pi,F)$ is a $G$-bundle, then it has a $G$-atlas $\{(U_\alpha,\varphi_\alpha)\}$ with associated $G$-valued cocycle of transition functions $\{g_{\alpha\beta}\}$. Using $G$-bundle construction theorem, one may construction a bundle with typical fiber $G$ by using left translation as the action. The resulting bundle is then a principal bundle $(P,M,\pi',G)$, and it turns out that $P\times_G F$ is equivalent to the original bundle $E$.
\end{para}

\begin{para}
	Suppose $F=V$ is a real (complex) vector space, and $\lambda:G\to \mathrm{GL}(V)$ is a representation of $G$ on $V$. Then $P\times_\lambda V$ has a natural vector bundle structure.
\end{para}

\begin{para}[The dual bundle of a vector bundle]
	Suppose that $(E,M,\pi,V)$ is a $(G,\lambda)$-vector bundle, $V^*$ is the dual space of $V$, and $\lambda^*$ is the dual representation of $\lambda$ of $G$ on $V^*$ such that $\lambda^*(g)=\lambda(g^{-1})^*$. By \ref{13}, we could construct a principal bundle $(P,M,\pi',G)$ from $(E,M,\pi,V)$, and $P\times_{\lambda^*} V^*$ is a $(G,\lambda^*)$-vector bundle by \ref{11}. The vector bundle $P\times_{\lambda^*} V^*$ is often called the dual bundle of the vector bundle $E$, or dual bundle for short, written as $E^*$.
\end{para}

% \begin{para}[Tensor product of vector bundles]
% Suppose that $(E,M,\pi,V)$ is a $(G,\lambda)$-vector bundle, and $(E',M,\pi',V')$ is a $(G,\lambda')$-vector bundle. Then representation $\lambda\otimes \lambda' : G\to \mathrm{GL}(V\otimes V')$ defines a $(G,\lambda\otimes \lambda')$-vector bundle $(E\times E',M,\pi\times \pi',V\otimes V')$. 
% \end{para}

Similarily, suppose that $V^{m,n}$ is the tensor product $V^*\otimes \cdots \otimes V^*\otimes V \otimes \cdots \otimes V$. There's a natural representation $\lambda^{m,n}=\lambda^*\otimes \cdots \otimes \lambda^*\otimes \lambda \otimes \cdots \otimes \lambda$ of $G$ on $V^{m,n}$ induced from $\lambda$ on $V$. Thus, $P\times_{\lambda^{m,n}} V^{m,n}$ is a $(G,\lambda^{m,n})$-vector bundle, and it's often written as $E^*\otimes \cdots \otimes E^*\otimes E \otimes \cdots \otimes E$ or $E^{m,n}$ for short.

\section{Koszul Connection}

Suppose $\pi:E\to M$ is a fiber bundle, denote the set of all sections on open subset $U\subset M$ by $\Gamma(U,E)$, and denote $\Gamma(M,E)$ by $\Gamma(E)$ for simplicity.

\begin{para}[Koszul connection]
	A \textit{covariant derivative} or \textit{Koszul connection} on a smooth vector bundle $(E,M,\pi,V)$ is a map $D:\Gamma(TM)\times \Gamma(E)\to \Gamma(E)$ (where $D(X,s)$ is written as $D_Xs$) satisfying the following four properties:
	\begin{compactenum}
	\item $D_{fX}(s)=fD_Xs$ for all $f\in C^\infty(M)$, $X\in \Gamma(TM)$ and $s\in \Gamma(E)$;
	\item $D_{X_1+X_2}s=D_{X_1}s+D_{X_2}s$ for all $X_1$, $X_2\in \Gamma(TM)$ and $s\in\Gamma(E)$;
	\item $D_X(s_1+s_2)=D_Xs_1+D_Xs_2$ for all $X\in \Gamma(TM)$ and $s_1$, $s_2\in \Gamma(E)$;
	\item $D_{X}(fs)=(Xf)s+fD_X s$ for all $f\in C^\infty (M)$, $X\in \Gamma(TM)$ and $s\in \Gamma(E)$.
	\end{compactenum}
	For a fixed $X\in \Gamma(TM)$, the map $D_X:\Gamma(E)\to \Gamma(E)$ is called the \textit{covariant derivative with respect to} $X$.

	The value of $D_Xs$ only depends on $X$ and $s$ locally, i.e. $(D_Xs)|_U=(D_Ys)|_U$ if $X|_U=Y|_U$, $(D_Xs)|_U=(D_Xt)|_U$ if $s|_U=t|_U$, where $U$ is an open set. Since $(D_Xs)|_U$ is linear for the vector field $X$, we could assume $Y=0$ and then $D_Ys=0$.

	Suppose $X$ vanishes on a open set $U$, i.e. $X|_{U}=0$. Near a point $p\in U$, by the partition of unity, we can find a open set $V$ and a smooth function $g$ on $M$ such that $p\in V\subset \overline{V}\subset U$, $g|_V=1$ and $g|_{M-U}=0$. Then $gX$ vanishes on $M$, so that
	\[
		(D_Xs)(p)=g(p)(D_Xs)(p)=(D_{gX}s)(p)=(D_{0}s)(p)=0.
	\]
	Since $p$ is an arbitrarily point in $U$, we have that $(D_Xs)|_U=0$. 

	Similarily, if $s|_U=0$, we have
	\[
		0=(D_X0)(p)=(D_X(gs))(p)=(Xg)(p)s(p)+g(p)(D_Xs)(p)=(Xg)(p)s(p)+(D_Xs)(p).
	\]
	Since $g$ is a locally constant function near $p$, $Xg$ vanishes on $p$, and then $(D_Xs)(p)=0$.
\end{para}

Because of these reasons, we usually abuse $D_Xs$ to denote its restriction on some open set.

\begin{pro}
The vector $(D_Xs)(p)$ only depends on the vector $X(p)$, that is, if there's another tangent vector field $Y$ on $M$ such that $X(p)=Y(p)$, then $(D_Xs)(p)=(D_Ys)(p)$.
\end{pro}

\begin{proof}
	It's naturally equivalent to show that $(D_Xs)(p)=0$ when $X(p)=0$. Take a chart $(U,\varphi)$ containing $p$, then $X|_U=X^i\partial_i$ and
	\[
		(D_Xs)(p)=X^i(p)(D_{\partial_i}s)(p).
	\]
	If $X$ vanishes on $p$, $X^i(p)=0$ says that $(D_Xs)(p)=0$.
\end{proof}

\begin{para}
Near a point $p\in M$, we could find local trivialization $(U,\varphi)$ on $E$ and atlas $(U,\psi)$ on $M$ such that
\[
	\varphi:\pi^{-1}(U)\to U\times V
\]
is a homeomorphism and $X|_U=X^i\partial_i$ by $(\partial_i)_p=(\psi_{*p})^{-1}(e_i)$, where $\{e_i\}$ is the standard basis of $\rr^n$. We can also find a basis $\{v_i\}$ on $V$, and the corresponding vector fields $\{\mu_i(q)=\varphi^{-1}(q,v_i)\}$ form a basis of $\Gamma(U,E)$, so
\[
	D_Xs=D_{X}(s^j\mu_j)=X(s^j)\mu_j+s^jD_{X}\mu_j=X(s^j)\mu_j+s^jX^iD_{\partial_i}\mu_j.
\]
Since $D_{\partial_i}\mu_j$ is still a vector field in $\Gamma(U,E)$, we could decompose it into $D_{\partial_i}\mu_j=\Gamma_{ij}^k \mu_k$, then
\[
	D_Xs=\left(Xs^k+\Gamma_{ij}^kX^is^j\right)\mu_k.
\]
The coefficients $\Gamma_{ij}^k$ are the famous Christoffel symbols.
\end{para}

\begin{para}[Dual connection]
Suppose $D$ is a connection on $E$, then there's a natrual connection $D^*$ on $E^*$ as follow. It is called the dual connection of $D$.

For $s^*\in \Gamma(E^*)$ and $s\in \Gamma(E)$, the canonical action $s^*$ on $s$ is defined point-by-point by $\langle s^*,s\rangle(p)=\langle s^*(p),s(p)\rangle_p$, where $\langle *,* \rangle_p$ is the action between $E^*_p$ and $E_p$ induced from the canonical action between $V^*$ and $V$. Thus, we define $D^*_Xs^*$ by 
\[
	\langle D^*_Xs^*,s\rangle=X\langle s^*,s\rangle-\langle s^*,D_Xs\rangle
\]
for all $s\in \Gamma(E)$. It's not difficult to show $D^*:\Gamma(TM)\times \Gamma(E^*)$ is a connection on $E^*$. 

Locally, 
\[
	D^*_X t=\left(Xt_k+(\Gamma^*)_{ki}^{j}X^it_j\right)\mu^k,
\]
where $\{\mu^k\}$ is the dual basis of $\{\mu_k\}$ such that $\langle \mu^i,\mu_j\rangle=\delta^i_j$, and 
\[
	(\Gamma^*)_{ki}^{j}=\langle D^*_{\partial_i}\mu^j,\mu_k \rangle = \partial_i \langle \mu^j,\mu_k \rangle -\langle \mu^j, D_{\partial_i}\mu_k \rangle = -\langle \mu^j, \Gamma_{ik}^l\mu_l \rangle =-\Gamma_{ik}^j.
\]
\end{para}

It is also natrual to define $D^{m,n}$ on $E^{m,n}$ by
\begin{align*}
	(D^{m,n}_X\omega)(s_i;s^*_j)=&X(\omega(s_i;s^*_j))\\
	&-\sum_i \omega\left(s_{1},\dots,s_{i-1},D_X s_i,s_{i+1},\dots,s_p;s^*_j\right)\\
	&-\sum_j \omega\left(s_i;s^*_{1},\dots,s^*_{j-1},D^*_X s^*_j,s^*_{j+1},\dots,s^*_q\right).
\end{align*}
This can be regarded as a generalization of Leibniz's rule.

\begin{para}
As an example, we consider the case of $m=n=1$, and suppose that locally $\omega=\Lambda^i_j\mu^j\otimes \mu_i$, then
\begin{equation}\label{eq:1}
\begin{split}
	(D^{1,1}_X\omega)(s,t)&=X\left(\Lambda^i_j\mu^j\otimes \mu_i(s,t)\right)-\Lambda^i_j\mu^j\otimes \mu_i(D_Xs,t)-\Lambda^i_j\mu^j\otimes \mu_i(s,D^*_Xt)\\
	&=X\left(\Lambda^i_js^jt_i\right)-\Lambda^i_jt_i\langle\mu^j,D_Xs\rangle-\Lambda^i_js^j \langle D^*_Xt,\mu_i\rangle \\
	&=X\left(\Lambda^i_js^jt_i\right)-\Lambda^i_jt_i\langle\mu^j,D_Xs\rangle-\Lambda^i_js^j \left(X(t_i)-\langle t,D_X\mu_i\rangle\right) \\
	&=X\left(\Lambda^i_js^j\right)t_i-\Lambda^i_jt_i\langle\mu^j,D_Xs\rangle+\Lambda^i_js^j\langle t,D_X\mu_i\rangle \\
	&=X\left(\Lambda^i_js^j\right)t_i-\Lambda^i_jt_iX^k\left(\partial_ks^j+s^l\Gamma_{kl}^j\right)+\Lambda^i_js^jX^k\Gamma_{ki}^lt_l \\
	&=X\left(\Lambda^i_js^j\right)t_i-\Lambda^i_jt_iX(s^j)-\Lambda^i_jt_iX^ks^l\Gamma_{kl}^j+\Lambda^i_js^jX^k\Gamma_{ki}^lt_l \\
	&=X\left(\Lambda^i_j\right)s^jt_i+X^k\left(\Gamma_{kl}^j\Lambda^l_i-\Lambda^j_l\Gamma_{ki}^l\right)s^it_j \\
	&=X\left(\Lambda^i_j\right)s^jt_i+X^k [\Gamma_{k},\Lambda]^j_i s^it_j \\
	&=X\left(\Lambda^i_j\right)\mu^j\otimes \mu_i(s,t)+X^k [\Gamma_{k},\Lambda]^i_j\mu^j\otimes \mu_i(s,t),
\end{split}
\end{equation}
where $\Gamma_k$ is a matrix with components $(\Gamma_k)^i_j=\Gamma_{kj}^i$, so
\begin{equation}
D^{1,1}_X(\Lambda^i_j\mu^j\otimes \mu_i)=X\left(\Lambda^i_j\right)\mu^j\otimes \mu_i+X^k [\Gamma_{k},\Lambda]^i_j\mu^j\otimes \mu_i.
\end{equation}
\end{para}

\begin{pro}
Suppose that $\omega_1\in \Gamma(E^{p_1,q_1})$ and $\omega_2\in \Gamma(E^{p_2,q_2})$, then 
\[
	D^{p_1+p_2,q_1+q_2}_X(\omega_1\otimes \omega_2)=D^{p_1,q_1}_X(\omega_1)\otimes \omega_2+\omega_1\otimes D^{p_2,q_2}_X(\omega_2).
\]
\end{pro}

\begin{proof}
Obviously.
\end{proof}

\begin{para}
A connection $D$ is a map from $\Gamma(TM)\times \Gamma(E)$ to $\Gamma(E)$, however, if can also be seen as a map from $\Gamma(E)$ to $\Gamma(E)\otimes \Gamma(TM)$ by
\[
	\langle Ds,X\rangle=D_Xs.
\]
Thus the following properties of connection can be vertified directly,
\begin{compactenum}
	\item $D(s_1+s_2)=Ds_1+Ds_2$ for all $s_1$, $s_2\in \Gamma(E)$;
	\item $D(fs)=s\otimes \dd f+fD(s)$ for all $f\in C^\infty (M)$ and $s\in \Gamma(E)$.
\end{compactenum}
\end{para}

Locally, 
\[
	Ds=\mu_i\otimes \dd s^i+s^j\mu_i\otimes \Gamma^i_j,
\]
where 
\[
	\Gamma^i_j(X)=(X^k\Gamma_{k})^i_j=\Gamma^i_{kj}\dd x^k(X).
\]
The $1$-form matrix $\Gamma$ is called the connection form of $D$.

Dually, 
\[
	D_Y^*\omega = \dd \omega_j(Y)\mu^j-\Gamma^k_j(Y)\omega_k \mu^j,
\]
thus
\[
	D^*\omega= \mu^i \otimes \dd \omega_i-\omega_j \mu^i\otimes \Gamma^j_i.
\]

\begin{para}[$\Gamma(E)$-valued form]
Let $\Omega^p(M)$ be the set of $p$-form on $M$, then the elements of $\Gamma(E)\otimes_{C^\infty(M)} \Omega^p(M)$ are the $V$-valued $p$-form on $M$. Let's denote this space by $\Omega^p(E)$, and naturally identify $\Omega^0(E)$ with $\Gamma(E)$. 

Exterior derivative will be directly extended to $\dd:\Omega^p(E)\to \Omega^{p+1}(E)$ by
\[
	\dd\left(\sum_i s_i\otimes \omega_i\right)=\sum_i s_i\otimes \dd \omega_i.
\]

Suppose $\sigma$ a form on $M$, and $\omega$ is a $\Gamma(E)$-valued form on $M$, then $\omega$ can be written as
\[
	\omega=\sum_i s_i\otimes \omega_i,
\]
then we could define the wedge product between $\Gamma(E)$-valued forms and forms by
\[
	\sigma\wedge \omega =\sum_i s_i\otimes (\sigma\wedge\omega_i),\quad \omega\wedge \sigma =\sum_i s_i\otimes (\omega_i\wedge \sigma).
\]
\end{para}

\begin{para}
Suppose $E$ is a vector bundle and $D$ is a connection on $E$. If $\omega$ is a $\Gamma(E)$-valued form on $M$, $\omega$ can be written as the sum $\sum_i v_i\otimes \omega_i$, and we define the connection of $\Gamma(E)$-valued form by
\[
	\dd_D(\omega)=\sum_i \dd_D(v_i\otimes \omega_i)=\sum_i\bigl(D(v_i)\wedge \omega_i+v_i\otimes \dd\omega_i\bigr).
\]
The connection of $\Gamma(E)$-valued form is still a $\Gamma(E)$-valued form, or perciously, 
\[
	\dd_D:\Omega^p(E)\to \Omega^{p+1}(E).
\]
\end{para}

\begin{para}[Torsion]
Suppose $M$ is a smooth manifold, $TM$ is its tangent bundle, $\nabla$ is a connection on $TM$. Then $\id_* : p\mapsto \id_{*p}$ is a $\Gamma(TM)$-valued $1$-form which can be written as $\partial_i\otimes \dd x^i$ locally. Then we define the torsion $T_\nabla$ to be $\dd_\nabla\id_*$, a $\Gamma(TM)$-valued $2$-form. Locally, we can calculate it as
\[
	T_\nabla=\dd_\nabla\id_*=(\nabla\partial_i)\otimes \dd x^i=\partial_j\otimes\Gamma_i^j\wedge \dd x^i=\Gamma_{ki}^j\partial_j\otimes\dd x^k\wedge \dd x^i=\frac 1 2 (\Gamma_{ki}^j-\Gamma_{ik}^j)\partial_j\otimes\dd x^k\wedge \dd x^i.
\]
If $T_\nabla=0$, we call $\nabla$ a non-tensorial connection. Locally, it is equal to say that $\Gamma_{ij}^k=\Gamma_{ji}^k$ valids for all $i$, $j$ and $k$.
\end{para}

\begin{pro}
$T_\nabla(X,Y)=\nabla_X Y-\nabla_Y X-[X,Y]$.
\end{pro}

\begin{proof}
It is a straightforward calculation.
\[
	T_\nabla(X,Y)=X^k Y^i(\Gamma_{ki}^j-\Gamma_{ik}^j)\partial_j,
\]
and
\begin{align*}
	\nabla_X Y-\nabla_Y X&=X^k\nabla_{\partial_k}(Y^i\partial_i)-\{X\leftrightarrow Y\}\\
	&=(XY^i)\partial_i+X^kY^i\Gamma^j_{ki}\partial_j-\{X\leftrightarrow Y\}\\
	&=[X,Y]+X^kY^i\Gamma^j_{ki}\partial_j-Y^kX^i\Gamma^j_{ki}\partial_j\\
	&=[X,Y]+T_\nabla(X,Y).
\end{align*}
That's all.
\end{proof}

\begin{para}[Curvature]
Let $E$ be vector bundle on $M$ and $s$ be a vector field. Let's calculate $d_DDs$ locally. Firstly, let $f$ be a function on $M$,
\begin{align*}
	d_DD(fs)&=d_D(s\otimes \dd f+fD(s))\\
	&=D(s)\wedge \dd f+s \otimes \dd^2f+fd_DD(s)+\dd f\wedge D(s)\\
	&=D(s)\wedge \dd f+fd_DD(s)-\dd f\wedge D(s)\\
	&=fd_DD(s).
\end{align*}
Thus, $d_DD(s)|_U$ dependents only on $s|_U$ by partition of unity. Secondly, let $s=s^i\mu_i$ locally, 
\begin{align*}
	d_DD(s)&=d_D(\mu_i\otimes \dd s^i+s^j\mu_i\otimes \Gamma^i_j)\\
	&=D(\mu_i)\wedge \dd s^i+\mu_i \otimes \dd^2s^i+D(s^j\mu_i)\wedge \Gamma^i_j+s^j\mu_i\otimes \dd \Gamma^i_j\\
	&=D(\mu_i)\wedge \dd s^i+D(s^j\mu_i)\wedge \Gamma^i_j+s^j\mu_i\otimes \dd \Gamma^i_j,
\end{align*}
since $\dd^2=0$. Then, use
\begin{align*}
	d_DD(s)&=D(\mu_i)\wedge \dd s^i+D(s^j\mu_i)\wedge \Gamma^i_j+s^j\mu_i\otimes \dd \Gamma^i_j\\
	&=(\mu_j\otimes \Gamma^j_i)\wedge \dd s^i+
	\mu_i\otimes(\dd s^j\wedge \Gamma^i_j)+s^j(\mu_k\otimes \Gamma^k_i)\wedge \Gamma^i_j
	+s^j\mu_i\otimes \dd \Gamma^i_j\\
	&=\mu_i\otimes (\Gamma^i_j\wedge \dd s^j)+
	\mu_i\otimes(\dd s^j\wedge \Gamma^i_j)+s^j\mu_k\otimes (\Gamma^k_i\wedge \Gamma^i_j)
	+s^j\mu_i\otimes \dd \Gamma^i_j\\
	&=\mu_i\otimes \left(\Gamma^i_j\wedge \dd s^j+\dd s^j\wedge \Gamma^i_j\right)+s^j\mu_k\otimes (\Gamma^k_i\wedge \Gamma^i_j)
	+s^j\mu_i\otimes \dd \Gamma^i_j.
\end{align*}
Because $\Gamma^i_j$ and $\dd s^j$ are 1-forms, $\Gamma^i_j\wedge \dd s^j+\dd s^j\wedge \Gamma^i_j=0$. Finally,
\[
	d_DD(s)=s^j\mu_k\otimes (\Gamma^k_i\wedge \Gamma^i_j)
	+s^j\mu_i\otimes \dd \Gamma^i_j.
\]

We usually call the operator $d_DD:\Gamma(E)\to \Omega^2(E)$ curvature operter and use another symbol $R_D$ to represent it. If $D$ is clear in context, we will omit $D$ in $R_D$. If we write $R_D=\frac{1}{2}R_{ijk}^l\mu^k\otimes\mu_l\otimes\dd x^i\wedge \dd x^j$, then
\begin{align*}
R_{ijk}^l &= \langle \mu^l,R_D(\mu_k)(\partial_i,\partial_j)\rangle\\
	&=\Gamma^l_a\wedge \Gamma^a_k(\partial_i,\partial_j)
	+\dd \Gamma^l_k(\partial_i,\partial_j)\\
	&=\Gamma^l_{ia}\Gamma^a_{jk}-\Gamma^l_{ja}\Gamma^a_{ik}+\partial_i\Gamma^l_{jk}-\partial_j\Gamma^l_{ik}.
\end{align*}
\end{para}

\begin{pro}
\[
	R(s)(X,Y)=D_XD_Y(s)-D_YD_X(s)-D_{[X,Y]}(s).
\]
\end{pro}

\begin{proof}
It is enough to prove it locally, so we have known that
\[
	R(s)(X,Y)=s^j\mu_k (\Gamma^k_i(X)\Gamma^i_j(Y)-\Gamma^k_i(Y)\Gamma^i_j(X))
	+s^j\mu_i \dd \Gamma^i_j(X,Y),
\]
where
\[
	\dd \Gamma^i_j(X,Y)=X(\Gamma^i_j(Y))-Y(\Gamma^i_j(X))-\Gamma^i_j([X,Y]).
\]
Since $R(fs)=fR(s)$, we can reduce it to
\[
	R(\mu_i)(X,Y)=D_XD_Y(\mu_i)-D_YD_X(\mu_i)-D_{[X,Y]}(\mu_i).
\]
What's more,
\[
	R(s)(fX,gY)=fgs^j\mu_k (\Gamma^k_i(X)\Gamma^i_j(Y)-\Gamma^k_i(Y)\Gamma^i_j(X))
	+fgs^j\mu_i \dd \Gamma^i_j(X,Y)=fgR(s)(X,Y),
\]
so we can reduce to the case that $X=\partial_j$ and $Y=\partial_k$. Write $D_i=D_{\partial_i}$ for short, we have
\begin{align*}
	R(\mu_i)(\partial_j,\partial_k)&=\mu_a (\Gamma^a_b(\partial_j)\Gamma^b_i(\partial_k)-\Gamma^a_b(\partial_k)\Gamma^b_i(\partial_j))
	+\mu_a \dd \Gamma^a_i(\partial_j,\partial_k)\\
	&=\Gamma^a_i(\partial_k)D_{j}\mu_b-\Gamma^a_i(\partial_j)D_{k}\mu_a+\partial_j\left(\Gamma^a_i(\partial_k)\right)\mu_a-\partial_k\left(\Gamma^a_i(\partial_j)\right)\mu_a\\
	&=D_j(\Gamma^a_i(\partial_k)\mu_a)-D_k(\Gamma^a_i(\partial_j)\mu_a)\\
	&=D_jD_k\mu_i-D_kD_j\mu_i.
\end{align*}
That's all.
\end{proof}

\begin{pro}[The first Bianchi identity]
Suppose $\nabla$ is a non-tensorial connection on $TM$, where $TM$ is the tangent bundle of a manifold $M$, then
\[
	R_\nabla(X,Y)Z+R_\nabla(Y,Z)X+R_\nabla(Z,X)Y=0.
\]
Locally, the first Bianchi identity becomes that $R_{ijk}^l+R_{jki}^l+R_{kij}^l=0$.
\end{pro}

\begin{proof}
The first Bianchi identity is an equation of tensor, so we can assume $X=\partial_i$, $Y=\partial_j$, $Z=\partial_k$, then the left side of the first Bianchi identity is that
\[
	\frac 12(R_{ijk}^l+R_{jki}^l+R_{kij}^l)\partial_l,
\]
so we only need to calculate $R_{ijk}^l+R_{jki}^l+R_{kij}^l$. Since that
\[
	R_{ijk}^l=\Gamma^l_{ia}\Gamma^a_{jk}-\Gamma^l_{ja}\Gamma^a_{ik}+\partial_i\Gamma^l_{jk}-\partial_j\Gamma^l_{ik}
\]
and $\nabla$ is a non-tensorial connection, then $\Gamma^l_{ij}=\Gamma^l_{ji}$ valids for all $i$, $j$ and $l$. Thus
\begin{align*}
R_{ijk}^l+R_{jki}^l+R_{kij}^l=&\,\,\Gamma^l_{ia}\Gamma^a_{jk}-\Gamma^l_{ja}\Gamma^a_{ik}+\partial_i\Gamma^l_{jk}-\partial_j\Gamma^l_{ik}+\Gamma^l_{ja}\Gamma^a_{ki}-\Gamma^l_{ka}\Gamma^a_{ji}+\partial_j\Gamma^l_{ki}-\partial_k\Gamma^l_{ji}\\
&+\Gamma^l_{ka}\Gamma^a_{ij}-\Gamma^l_{ia}\Gamma^a_{kj}+\partial_k\Gamma^l_{ij}-\partial_i\Gamma^l_{kj}\\
=&\,\,0.\qedhere
\end{align*}
\end{proof}

\begin{pro}[The second Bianchi identity]
Suppose $E$ is a vector bundle of $M$, then for every connection $D$ on $E$, $d_{D^{1,1}}(R_D)=0$, where $D^{1,1}$ is the induced connection on $E^*\otimes E$.
\end{pro}

Note that
\[
	R_D:\Gamma(E)\to \Gamma(E)\otimes\Omega^2(M),
\]
then
\[
	R_D\in \Gamma(E^*)\otimes \Gamma(E)\otimes\Omega^2(M)=\Gamma(E^*\otimes E)\otimes\Omega^2(M).
\]
So we will see $R_D$ as a $\Gamma(E^*\otimes E)$-valued $2$-form.

\begin{proof}
We will prove it locally. Since
\[
	R_D(s)=s^j\mu_k\otimes (\Gamma^k_i\wedge \Gamma^i_j)+s^j\mu_i\otimes \dd \Gamma^i_j,
\]
we can right $R_D$ as
\[
	R_D=(\mu^j\otimes \mu_k)\otimes (\Gamma^k_i\wedge \Gamma^i_j)+(\mu^j\otimes\mu_i)\otimes \dd \Gamma^i_j,
\]
where $\{\mu^i\}$ is the dual basis of $\{\mu_i\}$. Therefore,
\[
	\dd_{D^{1,1}}(R_D)=D^{1,1}(\mu^j\otimes \mu_k)\wedge (\Gamma^k_i\wedge \Gamma^i_j)+D^{1,1}(\mu^j\otimes\mu_i)\wedge \dd \Gamma^i_j+(\mu^j\otimes \mu_k)\otimes \dd(\Gamma^k_i\wedge \Gamma^i_j)
\]
The equation (\theequation) tells us that
\[
	D^{1,1}(\Lambda^i_j\mu^j\otimes \mu_i)=\mu^j\otimes \mu_i\otimes\dd\Lambda^i_j+\mu^j\otimes \mu_i\otimes [\Gamma,\Lambda]^i_j,
\]
and in particular,
\[
	D^{1,1}(\mu^j\otimes \mu_k)=\mu^j\otimes \mu_a\otimes \Gamma^a_k-\mu^a\otimes \mu_k\otimes\Gamma^j_a
\]
Thus
\begin{align*}
	D^{1,1}(\mu^j\otimes \mu_k)\wedge (\Gamma^k_i\wedge \Gamma^i_j)&=\mu^j\otimes \mu_a\otimes \Gamma^a_k\wedge \Gamma^k_i\wedge \Gamma^i_j-\mu^a\otimes \mu_k\otimes\Gamma^j_a\wedge\Gamma^k_i\wedge \Gamma^i_j\\
	&=\mu^j\otimes \mu_a\otimes \Gamma^a_k\wedge \Gamma^k_i\wedge \Gamma^i_j-\mu^j\otimes \mu_a\otimes\Gamma^i_j\wedge\Gamma^a_k\wedge \Gamma^k_i\\
	&=0,
\end{align*}
\[
	D^{1,1}(\mu^j\otimes \mu_i)\wedge (\dd \Gamma^i_j)=\mu^j\otimes \mu_a\otimes \Gamma^a_i\wedge \dd \Gamma^i_j-\mu^a\otimes \mu_i\otimes\Gamma^j_a\wedge\dd \Gamma^i_j,
\]
and so
\begin{align*}
\dd_{D^{1,1}}(R_D)&=D^{1,1}(\mu^j\otimes \mu_k)\wedge (\Gamma^k_i\wedge \Gamma^i_j)+D^{1,1}(\mu^j\otimes\mu_i)\wedge \dd \Gamma^i_j+(\mu^j\otimes \mu_k)\otimes \dd(\Gamma^k_i\wedge \Gamma^i_j)\\
&=\mu^j\otimes \mu_a\otimes \Gamma^a_i\wedge \dd \Gamma^i_j-\mu^a\otimes \mu_i\otimes\Gamma^j_a\wedge\dd \Gamma^i_j+(\mu^j\otimes \mu_k)\otimes \dd(\Gamma^k_i\wedge \Gamma^i_j)\\
&=\mu^j\otimes \mu_k\otimes \Gamma^k_i\wedge \dd \Gamma^i_j-
\mu^j\otimes \mu_k\otimes\Gamma^i_j\wedge\dd \Gamma^k_i+
(\mu^j\otimes \mu_k)\otimes \dd(\Gamma^k_i\wedge \Gamma^i_j)\\
&=\mu^j\otimes \mu_k\otimes \left(\Gamma^k_i\wedge \dd \Gamma^i_j-\Gamma^i_j\wedge\dd \Gamma^k_i+\dd(\Gamma^k_i\wedge \Gamma^i_j)\right)\\
&=\mu^j\otimes \mu_k\otimes \left(\Gamma^k_i\wedge \dd \Gamma^i_j-\dd \Gamma^k_i\wedge\Gamma^i_j+\dd(\Gamma^k_i\wedge \Gamma^i_j)\right)\\
&=0.\qedhere
\end{align*}
\end{proof}

% If we use $R_{ijk}^l$ to express Banachi identity, this is
% \begin{align*}
% 	d_{D^{1,1}}(R_{ijk}^l\mu^k\otimes\mu_l\otimes\dd x^i\wedge \dd x^j)&= 
% 	D^{1,1}(R_{ijk}^l \mu^k\otimes\mu_l)\wedge \dd x^i\wedge \dd x^j\\
% 	&=\mu^k\otimes\mu_l\otimes \dd R_{ijk}^l \wedge \dd x^i\wedge \dd x^j+R_{ijk}^lD^{1,1}(\mu^k\otimes\mu_l)\wedge \dd x^i\wedge \dd x^j\\
% \end{align*}
% where
% \begin{align*}
% 	R_{ijk}^lD^{1,1}(\mu^k\otimes\mu_l)\wedge \dd x^i\wedge \dd x^j&=R_{ijk}^l(\mu^k\otimes \mu_a\otimes \Gamma^a_l-\mu^a\otimes \mu_l\otimes \Gamma^k_a)\wedge \dd x^i\wedge \dd x^j\\
% 	&=\mu^k\otimes \mu_l\otimes (R^a_{ijk} \Gamma^l_a-R^l_{ija}\Gamma^a_k)\wedge \dd x^i\wedge \dd x^j\\
% 	&=\mu^k\otimes \mu_l\otimes (R^a_{ijk} \Gamma^l_{ba}-R^l_{ija}\Gamma^a_{bk})\dd x^b\wedge \dd x^i\wedge \dd x^j,
% \end{align*}
% thus $\dd_{D^{1,1}}R_D=0$ is equivlant to
% \[
% 	\sum_{\text{p of }ijb}\left(R^a_{ijk} \Gamma^l_{ba}-R^l_{ija}\Gamma^a_{bk}+\partial_bR^l_{ijk}\right)=0.
% \]
% If we induce a new symbol $s^i_{;k}$ such that
% \[
% 	D_{\partial_k}s=\left(\partial_k s^i+s^j\Gamma_{kj}^i\right)\mu_i=s^i_{;k}\mu_i,
% \]
% the second Bianchi identity now reads that
% \[
% 	\sum_{\text{p of }ijb}\left(R^l_{ijk;b}-R^l_{ija}\Gamma^a_{bk}\right)=0.
% \]

% \begin{pro}[The Bianchi identity on tangent bundle]
% Suppose $M$ is a manifold and $\nabla$ is a non-tensorial connection of $TM$, then
% \[
% 	(\nabla_{X}R)(Y,Z,W) + (\nabla_{Z}R)(X,Y,W) + (\nabla_{Y}R)(Z,X,W) = 0.
% \]
% \end{pro}

% \begin{proof}
% We will give a ugly proof. Note that
% \begin{align*}
% 	\nabla_{X} (R(Y,Z)W)&=X^i\nabla_i(Y^jZ^kW^lR_{jkl}^a\partial_a)\\
% 	&=X^i\partial_i(Y^jZ^kW^l)R_{jkl}^a+X^iY^jZ^kW^lR_{jkl;i}^a\\
% 	&=X^i\partial_i(Y^jZ^kW^l)R_{jkl}^a+X^iY^jZ^kW^l
% 	R^a_{jkb}\Gamma^b_{il},
% \end{align*}
% then 
% \[
% 	\nabla_{X} (R(Y,Z)W) + \nabla_{Z}(R(X,Y)W) + \nabla_{Y}(R(Z,X)W)=X^iY^jZ^kW^l
% 	R^a_{jkb}\Gamma^b_{il}
% \]
% \end{proof}
\end{document}