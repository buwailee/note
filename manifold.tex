% author: buwailee@nmhs
\documentclass[9pt]{extbook}
\usepackage[book,zh]{noteheader}
\usepackage[winfonts]{ctex}

\newtheorem{para}{}[section]
\newcommand{\paracount}[1]{\refstepcounter{para}\textbf{#1 (\thepara)}}
	\renewcommand{\para}{\paracount{}}
	\newcommand{\theo}{\paracount{Theorem}}
	\newcommand{\pro}{\paracount{Proposition}}
	\newcommand{\lem}{\paracount{Lemma}}
	\renewcommand{\proof}{\textbf{Proof:}\hspace{0.5em}}

\newcommand{\pararef}[1]{(\ref{#1})}

\usepackage[xetex]{hyperref}%使用xetex引擎
	\hypersetup %一些选项
	{
		bookmarksnumbered=true,%书签中章节编号
		colorlinks=true,  % 彩色链接 false:边框链接 ; true: 彩色链接
	}

\definecolor{shadecolor}{rgb}{0.92,0.92,0.92}

\newcommand{\no}[1]{{$(#1)$}}
% \renewcommand{\not}[1]{#1\!\!\!/}
\newcommand{\rr}{\mathbb{R}}
\newcommand{\zz}{\mathbb{Z}}
\newcommand{\aaa}{\mathfrak{a}}
\newcommand{\pp}{\mathfrak{p}}
\newcommand{\mm}{\mathfrak{m}}
\newcommand{\dd}{\mathrm{d}}
\newcommand{\oo}{\mathcal{O}}
\newcommand{\calf}{\mathcal{F}}
\newcommand{\calg}{\mathcal{G}}
\newcommand{\bbp}{\mathbb{P}}
\newcommand{\bba}{\mathbb{A}}
\newcommand{\osub}{\underset{\mathrm{open}}{\subset}}
\newcommand{\csub}{\underset{\mathrm{closed}}{\subset}}

\DeclareMathOperator{\im}{Im}
\DeclareMathOperator{\Hom}{Hom}
\DeclareMathOperator{\id}{id}
\DeclareMathOperator{\rank}{rank}
\DeclareMathOperator{\tr}{tr}
\DeclareMathOperator{\supp}{supp}
\DeclareMathOperator{\coker}{coker}
\DeclareMathOperator{\codim}{codim}
\DeclareMathOperator{\height}{height}
\DeclareMathOperator{\sign}{sign}

\DeclareMathOperator{\ann}{ann}
\DeclareMathOperator{\Ann}{Ann}
\DeclareMathOperator{\ev}{ev}
\DeclareMathOperator{\Int}{int}

\usepackage{amsmath}
\begin{document}
\chapter{Calculus}
\section{Foundation}

\para 一个局部$n$维欧几里得空间是一个Hausdorff空间$M$满足,对每一个点$p\in M$,存在一个$p$的邻域$U\osub M$和一个同胚$\varphi:U\to V$,其中$V$是一个$\rr^n$中的开集。这个同胚有时候被称为一个坐标、坐标映射等,而资料$(U,\varphi)$被称为一个{\kaishu 坐标卡}。坐标$\varphi$经常写成分量形式,$\varphi=(x^1,\cdots,x^n)$,其中$x^i:U\to \rr$.

一个局部$n$维欧几里得空间是局部紧的,这是因为他局部同胚于欧式空间, 而欧式空间是局部紧的。

\para 局部欧几里得空间$M$上的一个光滑微分结构$\mathscr{F}$是这样一族坐标卡$(U_\alpha,\varphi_\alpha)$,满足:$\{U_\alpha\}$构成$M$的开覆盖,$\varphi_\alpha\circ\varphi^{-1}_\beta|_{U_\alpha\cap U_\beta}$是光滑映射,后者被称为坐标卡的相容性条件。此外,如果有一个坐标卡$(U,\varphi)$和每一个坐标卡都相容,那么可以推断出他在$\mathscr{F}$中,这样的微分结构被称为极大微分结构。极大微分结构当然不一定是唯一的,不过我们不担心这个,因为我们往往是固定一个微分结构来研究流形的,下面假设出现的微分结构总是极大的。

\para 一个$n$维光滑流形(smooth manifold)$(M,\mathscr{F})$是一个赋予了光滑微分结构$\mathscr{F}$的第二可数的局部欧几里得空间$M$.

我们想要做一个范畴,现在已经有了对象,那么自然需要态射,态射被如下定义:

\para 设$(M,\mathscr{F})$和$(N,\mathscr{G})$是两个光滑流形,连续函数$f:M\to N$被称为一个{\kaishu 光滑映射},如果$\psi\circ f\circ \varphi^{-1}$是一个光滑函数对任意的$\mathscr{F}$中的坐标卡$(U,\varphi)$和$\mathscr{G}$中的坐标卡$(V,\psi)$成立。

这样,光滑流形就构成了一个范畴,其中态射是流形间的光滑映射。他是拓扑空间范畴的子范畴。

从此以后,我们对一个固定的流形$(M,\mathscr{F})$,常常会略去他的微分结构,只写作$M$。对于一个光滑流形$M$的非空开子集$U$,显然,他有继承自$M$的一个拓扑结构和微分结构,所以$U$也是一个光滑流形。很容易看到,$\rr^n$是一个光滑流形,按照上面的结论,我们可以得到一类光滑流形,$\rr^n$的开子集。比如把$n\times n$矩阵放入$\rr^B{n^2}$内,那么行列式不为零的那些矩阵就构成一个光滑流形,记作$\mathrm{GL}(n,\rr)$,称作一般线性群。

特别地,$\rr$也是一个光滑流形。我们称光滑映射$f:M\to \rr$是一个$M$上的光滑函数。光滑函数$f$限制在$U\osub M$上也是一个光滑函数$f|_U$.

\para 设$M$是一个光滑流形,$U\osub M$上的光滑函数的集合记作$\calf(U)$,$\calf$被称为$M$上的光滑函数{\kaishu 层}。由于可以逐点定义加法和乘法,所以$\calf(U)$拥有$\rr$-代数结构。设$p\in M$,我们定义如下等价关系:设$U$和$V$都是$p$的邻域,以及$f\in \calf(U)$和$g\in \calf(V)$,如果在一个$W\osub U\cap V$上,$f|_W=g|_W$,则$f\sim g$.所有这样的等价类记作$\calf_p$,称为$p$处的光滑函数{\kaishu 茎},他的代表元素可以写成$f_p=\langle U,f\rangle$,称为{\kaishu 芽}. 显然$\calf_p$有继承自$\calf(U)$的自然的$\rr$-代数结构。上述过程也可以用colimit表示为$\calf_p=\varinjlim_{U\ni p}\calf(U)$,其中$p$的邻域以包含形成归纳系。

设$p\in M$,茎$\calf_p$是一个局部环。实际上,$\langle U,f\rangle \in \calf_p$且$f(p)= 0$的元素构成了$\calf_p$的一个理想。不在这个理想内的$\langle U,f\rangle$,由于$f(p)\neq 0$,那么适当缩小$U$到$V$,由$f$的连续性,总可以找到$V$使得$f|_V$处处不为零,这样$\langle V,1/f|_V\rangle$便是$\langle U,f\rangle$的一个逆。因此上面这个理想即$\calf_p$唯一的极大理想,我们其记作$\mm_p$。容易看到$\calf_p/\mm_p\cong \rr$,实际上,对每一个芽$f_p\in\calf_p$,都成立$f_p=f_p-f(p)+f(p)$,在$\calf_p/\mm_p$中看,他和$f(p)\in\rr$也就没区别了。

\lem 设$f:\rr^n\to \rr$光滑,则
\[
	f(x)=f(0)+\partial_if(0)x^i+\frac{1}{2}g_{ij}(x)x^ix^j,
\]
其中$g_{ij}$光滑。

\proof 利用微积分基本定理
\[
	f(x)-f(0)=\int_0^1f'(tx)\dd t=\int_0^1\partial_i f(tx)x^i\dd t=h_i(x)x^i,
\]
可以得到$h_i(0)=\partial_i f(0)$,然后再对$h_i$使用上面的步骤即可得到我们想要的表达式。\qed

\para 使用一个局部坐标$\varphi=(x^1,\cdots,x^n)$且$\varphi(p)=0$,可以将上面的引理翻译到流形上。设设$f:U\to \rr$光滑,则在$p$的一个邻域$V$上对任意的$q\in V$成立
\[
	f(q)=f(p)+\frac{\partial f\circ \varphi^{-1}}{\partial x^i}(p)x^i(q)+\frac{1}{2}g_{ij}(q)x^i(q)x^j(q),
\]
其中$g_{ij}$在$V$上光滑,以后我们就将那个偏微分记作$\partial_i f(p)$.


\para 设$p\in M$,$p$处的茎为$\calf_p$,他的极大理想为$\mm_p$,此时$p$处的 余切空间被定义为自然的矢量空间$T_p^*M:=\mm_p/\mm_p^2$。余切空间的元素被称为余切矢量。

$\mm_p/\mm_p^2$确实是一个矢量空间。首先它显然是一个$\calf_p$-模,然后任取$a\in \mm_p$,由于$a\mm_p/\mm_p^2=0$,所以$\mm_p/\mm_p^2$是一个$\calf_p/\mm_p$-模,即$\rr$-矢量空间。这样定义的余切空间,可以看到,是所有的那些一阶小量构成的集合,即其中的元素为“微分”。


\para 设$p\in M$,$M$是一个$n$维流形,则$T_p^*M:=\mm_p/\mm_p^2$是$n$维的。

我们可以选取一组局部坐标来算维数,由于选取不同的局部坐标都是通过同胚联系的,所以不同的选取对维数没什么影响。由上面的引理,设$f_p\in \mm_p$,则他可以写作
\[
	f_p=\partial_i f(p)x^i_p+\frac{1}{2}g_{ij}(q)x^i_px^j_p
\]
考虑一个局部坐标$\varphi=(x^1,\cdots,x^n)$,设自然同态$\dd_p:\mm_p\to \mm_p/\mm_p^2$,很简单就可以看到$\dd_p(x^i_p)\neq 0$.实际上,如果$x^i_p\in \mm_p^2$,那么$x^i_p=rs$,其中$r$, $s\in \mm_p$,然后根据上面的引理$r=a_ix^i_p+\cdots$以及$s=b_ix^i_p+\cdots$,于是$x^i=rs=a_jb_kx^j_px^k_p+\cdots$,但显然这是不可能的。

所以,如果$f_p\in \mm_p$,则
\begin{equation}
\label{c1:e1}
	\dd_p(f_p)=\partial_i f(p)\dd_p(x^i_p).
\end{equation}
这样所有的$T_p^*M=\mm_p/\mm_p^2$中的元素都可以由$\dd_p(x^i_p)$展开,他们都是非零的,而且容易证明是线性无关的,所以这是$T_p^*M$的一组基,余切空间的维数计算完毕。

以后我们用$\dd_p(f)$乃至$\dd_pf$来记$\dd_p(f_p)$。实际上,我们可以将$\dd_p$定义在$\calf_p$上,设$a$是一个常值芽,补充定义$\dd_pa=0$,可以看到,此时式\eqref{c1:e1}依旧满足。以后我们就这样来看$\dd_p:\calf_p\to T_p^*M$,他被称为{\kaishu 外微分}算子。

\para 此时\[\dd_p (fg)=\dd_p\bigl(\bigl(f-f(p)\bigr)\bigl(g-g(p)\bigr)+f(p)\bigl(g-g(p)\bigr)+\bigl(f-f(p)\bigr)g(p)\bigr)=f(p)\dd_pg+\dd_pf g(p).\]

\para 设$f:M\to N$是一个光滑映射,上面的光滑函数层分别为$\calf$和$\calg$。任取$\varphi\in \calg(V)$,可以通过$f^*\varphi=\varphi\circ f$定义$f^*\varphi\in \calf(f^{-1}(V))$.

下面我们考虑两个流形余切空间之间的映射。设$\langle V,\varphi\rangle\in \calg_{f(p)}$,于是$\langle f^{-1}(V),\varphi\circ f\rangle\in \calf_p$,所以$f^*$诱导了一个$\rr$-代数同态$f^*_p:\calg_{f(p)}\to \calf_p$,特别地,可以看到$f^*_p:\mm_{f(p)}\to \mm_{p}$,于是$f^*_p:\mm^2_{f(p)}\to \mm^2_{p}$.

\para 设$f:M\to N$是一个光滑映射,他诱导了一个线性映射\footnote{这里我们滥用一下记号。}$f^*_p:T_{f(p)}^*N\to T_p^*M$.

\para 对于复合,显然$(f\circ g)^*=f^*\circ g^*$,以及在一点处,$(f\circ g)^*_p=g^*_p\circ f^*_{g(p)}$.很容易看到$\id^*_p=\id_{T_p^*M}$,所以如果$f:M\to N$是同胚,则$f_p^*:T_{f(p)}^*N\to T_p^*M$是同构。

\para \label{f*d=df*}利用复合公式,设$f:M\to N$是光滑映射,则$f^*_p\bigl(\dd_{f(p)}g\bigr)=\dd_{p}(f^*g)=\dd_p(g\circ f)$.

\para 设$p\in M$,$M$是$n$维光滑流形,则{\kaishu 切空间}$T_pM$被定义为余切空间$T_p^*M$的对偶空间。切空间的元素被称为{\kaishu 切矢量}。由于余切空间是有限维的,他的对偶空间也和他有着相同的维度,即$n$维。

\para 由于切空间是余切空间的对偶空间,所以他是余切空间上的线性函数构成的空间,反过来,由于是有限维的,所以可以认为对偶空间的对偶空间就是原本的空间,这就是说可以将余切空间的矢量看成切空间矢量的线性函数:设$\dd_p f\in T_p^*M$和$v\in T_pM$,定义$\dd_p f(v):=v(\dd_p f)$.

虽然上面这些个定义都很短也很清楚,不过操作上却没有那么简单。下面,我们将一个切矢量扩张到$\calf_p^*$上面去。

\para 设$f$是在$p$附近的光滑函数,而$v\in T_pM$,可以通过$D_v(f_p):=v(f_p-f(p))$定义线性映射$i_p:v\mapsto i_p(v)=D_v\in \calf_p^*$,他是一个单射。

注意到$(fg)_p=f_pg_p$,所以
\begin{align*}
	D_v(f_pg_p)&=v(f_pg_p-f(p)g(p))\\
	&=v\bigl((f_p-f(p))(g_p-g(p))+f(p)(g_p-g(p))+(f_p-f(p))g(p)\bigr)\\
	&=f(p)D_v(g_p)+D_v(f_p)g(p),
\end{align*}
我们将满足这条性质的线性映射$D_v\in \calf_p^*$称为$p$处的导子,所有$p$处的导子构成的空间暂时记作$V_p$,而他其实和$T_pM$是同构的。

为了证明这点,任取导子$D\in V_p$,由于$D(1)=D(1\times 1)=2D(1)$,所以$D(1)=0$,继而靠着$D$的线性性,对于常值函数的芽$a$来说,$D(a)=aD(1)=0$。因为每一个$\calf_p$中的元素$f_p$都可以写成$f_p-f(p)+f(p)$的形式,所以$D(f_p)=D(f_p-f(p))$,这就是说,一个导子的性质完全由他在$\mm_p$上的值决定,这种关系是一对一的。即$\pi_p:D\mapsto D|_{\mm_p}$是一个线性同构。

同时,设$f_p$, $g_p\in \mm_p$,则$\pi_p(D)(f_pg_p)=f(p)\pi_p(D)(g_p)+g(p)\pi_p(D)(f_p)=0$,于是$\pi_p(D)(\mm_p^2)=0$,所以,$\pi_p(D)\in T_pM$,即$D|_{\mm_p}$是一个切矢量,因此导子$D$完全由一个切矢量$D|_{\mm_p}=\pi_p(D)$决定。这样,$i_p:T_pM\to V_p$也是一个满射,所以他是一个同构。当然我们也可以直接计算验证$\pi_p\circ i_p=\id_{T_pM}$以及$i_p\circ \pi_p=\id_{V_p}$。

因为有这个同构,所以以后我们用$T_pM$来标记导子构成的矢量空间,一个导子才是一个切矢量。这样的好处是,我们在具体计算的时候,可以直接在$\calf_p$上进行而非$\mm_p$上,特别地,现在对于一个切向量$v$来说,成立$\dd_pf(v)=v(f_p)$,这是因为对一个导子$v$来说$v(f_p)=v(\dd_pf)$.

\para 设$f:M\to N$是一个光滑映射,定义它在$p\in M$处的导数为$T_pf=f_{*p}:T_pM\to T_{f(p)}N$使得对任意的$v\in T_p M$和任意的$g_{f(p)}\in \mm_{f(p)}$成立$(f_{*p}v)(g_{f(p)})=v(f_p^*g_{f(p)})$.

为以后的处理方便,不妨通过等同$\partial_i$和标准基$e_i$来等同$T_p\rr^n$和$\rr^n$。此外,通过坐标卡上的同胚$\varphi$,我们用$\partial_i$来标记$\varphi^{-1}_{*p}(e_i)$,这显然是$T_pM$处的一组基。


\para 设$f$是在$p$附近的光滑函数,任取$v\in T_pM$.因为$f_{*p}:T_pM\to T_{f(p)}\rr=\rr$,所以$f_{*p}(v)$是一个数,故而
	\[
		f_{*p}(v)=f_{*p}(v)(\id_{\rr})=v\bigl((\id_{\rr}\circ f)_p\bigr)=v(f_p)=\dd_p f(v).
	\]
	因为对所有的切矢量$v$都成立上式,所以$f_{*p}=\dd_p f$.

选定一个局部坐标,因为$\dd_p x^i(\partial_j)=\partial_jx^i(p)=\delta^i_j$,所以$\dd_p x^i$就是$\partial_i$的对偶基。下面我们来计算一个特别的例子,设$f:M\to \rr^n$是一个流形$M$上的矢量值光滑函数,则$f^i:M\to \rr$是一个光滑函数,那么$f_{*p}=\dd_pf^i e_i$,其中$e_i$是$\rr^n$的标准基。再设$f:\rr^m\to \rr^n$,则$\dd_pf^i=\partial_j f^i(p) \dd x^j=\partial_j f^i(p) e^j$.写成矩阵即
\[
	(f_{*p})^{i}_{\phantom{i}j}=\partial_j f^i(p),
\]
此即$f$的Jacobian.

\para 复合函数求导法则:$(f\circ g)_{*p}=f_{*g(p)}\circ g_{*p}$.抽象表现出来是线性映射复合,表现在矩阵(即Jacobian)上就是两个矩阵相乘。

\para 设$U$上光滑曲线$\sigma:(-\epsilon,\epsilon)\to U$,在时间为零的时候经过点$p$,即$\sigma(0)=p$,于是$\sigma_{*0}=\dot\sigma(0)\in T_pM$. 局部来说,他可以写作
\[
	\dot{\sigma}(0)=\frac{\dd x^i\circ \sigma}{\dd t}(0)\partial_i=\dot \sigma^i(0)\partial_i,
\]
当他作用在一个光滑函数上时,写作
\[
	\dot{\sigma}(0)(f)=\dot \sigma^i(0)\partial_if(p).
\]
对于固定的$f$,$\dot{\sigma}(0)(f)$可以看做$f$沿着$\sigma$在点$p$切矢量的方向导数,实际上,在$\rr^n$中,我们通常将上式写作$\dot{\sigma}(0)(f)=v\cdot \nabla f$,其中$v=\dot \sigma^i(0)e_i$.

\para 反过来,给定一个点$p$处的切矢量$v$,我们可以找到一个光滑曲线$\sigma$使得在他点$p$的切矢量就是$v$。这是局部结论,在欧式空间里去证明就可以了。在欧式空间中,$\sigma(t)=p+vt$就是我们需要的光滑曲线。

\para 由于$v(f_p)$可以看做$f$沿着$v$方向在$p$点的方向导数,以及等式$\dd_pf(v)=v(f_p)$,所以$\dd_pf(v)$也理解为$f$沿着$v$方向在$p$点的方向导数。

\para 正如我们前面提到的,切矢量的几何直观可以靠曲线的切矢量来想象,那么余切矢量呢?为了做出适当的想象,不妨回到欧式空间里面,固定一个切矢量,方向导数$\dd_pf(v)$可以写作$v\cdot \nabla f(p)$,所以(在欧式空间的内积结构下)我们可以认为$\dd_pf$就是$\nabla f(p)$。现在改变$p$,我们得到了一个矢量场$\nabla f$,这是$\{f=\text{const}\}$确定的等能面的法矢量场,法矢量场和等能面一一联系。所以为了避免引入内积结构,我们可以认为$\dd_pf$就是局部的一族等能面。

\section{Submanifold}

\para 设$\varphi:M\to N$是一个光滑映射,\no{a}. 称$\varphi$是一个浸入,如果$\varphi_{*p}$处处非退化。\no{b}. 称$(M,\varphi)$是一个子流形,如果$\varphi$是单的。不是所有浸入都是子流形,比如圆周的参数表示$(\cos t,\sin t)$是一个浸入,但不是单的。

显然,对于光滑流形的一个开子集,他可以继承大流形的流形结构而形成一个新的流形,他是一个子流形,被称为开子流形。

后面我们经常会说“设流形$M$上有某某”这样的话,但一般来说,某某在流形上的整体存在性是很难保证的,往往他只是局部存在,即可以在流形$M$的某个开集上存在。但是注意到$M$的开集现在也有流形结构,即开子流形结构,于是我们的命题就可以在这个新的流形上正常工作了。所以经常为了方便,对于不少命题的陈述,我们会把对象直接定义到整个流形上。

\para 设$\varphi:M\to N$,如果$M$微分同胚于$N$的开子流形$\varphi(M)$,则称子流形$(M,\varphi)$是一个嵌入。

浸入子流形不一定是嵌入子流形,比如秩为$1$的单的光滑曲线$f(t)=((t^3+t)/(t^4+1),(t^3-t)/(t^4+1))$,在$\rr^2$中他的图像看起来是可以有自交点的。

\para 设$U$是$M$的一个子集,但$U$本身有一个流形的结构,如果此时$i:U\hookrightarrow M$是一个嵌入,则称$U$是$M$的一个正则子流形。

所谓的正则子流形就是说,它本身的流形结构和从大的流形那里继承来的流形结构是相同的的。

\para 设$M$和$N$是光滑流形,$f:M\to N$是一个单浸入。我们可以赋予$f(M)$一个微分结构通过把$f:M\to f(M)$做成一个微分同胚。此时,$f(M)$是$N$的正则子流形当且仅当$f$是一个嵌入。

\para 反函数定理:设$U\subset \rr^n$是一个开集,映射$f:U\to \rr^n$光滑,如果Jacobian在$p$处非奇异,即$f_{*p}$可逆,则存在$p$的一个邻域$V\osub U$,使得$f|_V:V\to f(V)$是一个(光滑)同胚。

证明见微积分教材,常见的证明有比如压缩映像定理。该定理说明,如果函数局部线性化后性质不错,那么在那点附近性质也不错。由于是局部性质,所以可以直接翻译到流形上没什么改变。

\theo 流形上的反函数定理:设$M$和$N$的维度相同,映射$f:M\to N$光滑,如果$f_{*p}$可逆,则存在$p$的一个邻域$U$,使得$f|_U:U\to f(U)\subset N$是一个(光滑)同胚。换句话说,浸入局部是嵌入。

\para 称一族$M$上的光滑函数$\{f_i\}_{1\leq i\leq n}$在点$p$相互无关,即指$\{\dd_p (f_i)=(f_i)_{*p}\in T_p^*M:1\leq i\leq n\}$们线性无关。

如果$\{f_i\}_{1\leq i\leq n}$相互无关,则函数$f=(f_1,\cdots,f_n):M\to \rr^n$在点$p$上的导数$f_{*p}$可逆,所以按照反函数定理,可以在$p$附近找一个领域,使得$f|_V$是一个$V$到$\rr^n$中开集的同胚,这样$(V,f|_V)$就是一张坐标卡。如果$\{f_i\}$个数不到$n$,那么补几个进去,照样可以找到一张坐标卡,其中前几个分量是$\{f_i\}$.

\lem 设$f_*:V\to W$是一个有限维矢量空间间的线性映射以及他的对偶映射是$f^*:W^*\to V^*$,则$\rank(f_*)=\rank(f^*)$. 特别地,当$f_*$是单(满)的时候,$f^*$是满(单)的。

\para 设$\varphi:M\to N$光滑,且$\varphi_{*p}$是单射。令$(x_1,\cdots,x_n)$是$\varphi(p)$附近的一个坐标,那么$x_i\circ\varphi$是$p$附近的一个坐标。特别地,$\varphi$在$p$附近是一个单射。如果$\varphi_{*p}$是满射,则$x_i\circ\varphi$是$p$附近的一个坐标中的一部分。

若$\varphi_{*p}$是单射,他的对偶映射$\varphi^*_p$就是满射,于是$\varphi^*_p(x_i)_{*\varphi(p)}=(x_i\circ\varphi)_{*p}=\dd_p(x_i\circ \varphi)$张成了$T_p^*M$,在其中选出一组极大线性无关组(不妨设为前$m$个),这就构成了$p$附近的一组坐标。而$(x_1,\cdots,x_m)\circ \varphi$局部是同胚,所以$\varphi$局部是单射。

若$\varphi_{*p}$是满射,他的对偶映射$\varphi^*_p$就是单射,于是$\varphi^*_p\dd_{\varphi(p)}x_i=\dd_p(x_i\circ \varphi)$相互独立,一般来说,他数量不够构成坐标,但是却可以构成坐标中的一部分。

\para 设$f:M\to N$是一个光滑映射,则$\rank_p f$被定义为$\rank_p f_{*p}$.取$p$和$f(p)$附近的坐标$\varphi$和$\psi$且使得$\varphi(p)=0$,则$f$在点$p$的秩就是Jacobian矩阵$(\psi\circ f \circ \varphi^{-1})_{*0}$的秩。

选取$f(p)$附近的坐标$\psi$,则$\psi\circ f:M\to \rr^n$,不妨将其写作$(f_1,\cdots,f_n)$,则$\rank_p f$就是${\dd_pf_1,\cdots,\dd_pf_n}$张成的线性空间的维度。实际上,因为这是局部结果,所以可以直接假设$N=\rr^n$,而此时$f_{*p}=(\dd_pf_1,\cdots,\dd_pf_n)$。

\theo 设$M$是一个$m$维流形且$f:M\to N$是一个光滑映射,如果存在常数$l$使得$\rank_p f$处处等于$l$,那么对于$q\in N$,$f^{-1}(q)$要么是空集,要么是$M$的一个正则子流形,维度为$m-l$。

这个定理我们就不证明了。特别当$N=\rr$的时候,$f$如果是一个秩处处为$1$的光滑函数(即$\dd_p f$处处不为零),则$f^{-1}(a)$或者是一个空集,或者是一个$m-1$维正则子流形。这就是所谓的等能面,或等势面。

\section{Vector Field}
和我们以前的直观一样,所谓的矢量场(vector field)就是每一点赋予一个矢量。

\para 设$U\osub M$,$U$上的映射$X:p\mapsto X(p)\in T_pM$被称为$U$上的(切)矢量场。因为在$U$的每一个局部$V$(至少一个坐标卡内),矢量场$X$都可以写作$X=X^i\partial_i$,其中$X^i$是$V$上的实值函数,而$\partial_i$在不同的点分属不同的切空间。如果$\{X^i\}$在点$p$是光滑函数,则称$X$在$p$处光滑。如果$X$在$U$处处光滑,则称$X$是$U$上的一个光滑矢量场。

对矢量场而言,他可以作用在光滑函数上得到一个函数,在局部的作用效果即$Xf=X^i\partial_if$. 显然,如果$X$是要给光滑矢量场,则$Xf$是一个光滑函数。反过来,如果$X^i\partial_if$对任意的光滑函数都光滑,则$X^i$自然也是光滑的,所以有下面一个结论。

\para 设$X$是一个$U$上的矢量场,如果$Xf$对任意的光滑函数$f$也是光滑的,那么$X$是一个光滑矢量场。这个命题可以看作矢量场光滑性的一个坐标无关的定义。

\para 设$f:M\to N$是一个光滑单射,而$X$是$M$上的一个光滑矢量场,则$f_*X:p\mapsto f_{*f^{-1}(p)}X_{f^{-1}(p)}$是$N$上的一个矢量场。因为$(f_*X)g=X(g\circ f)$成立,所以这也是一个光滑矢量场。

下面我们用(光滑)纤维丛的语言来抽象地定义场。

\para 设$E$, $B$, $F$是三个光滑流形,$\pi:E\to B$是一个光滑映射,若在每一点$p\in B$,都存在一个邻域$U$和光滑同胚$\varphi$使得$\pi^{-1}(U)$同胚于$U \times F$,且如下交换图成立。则称$(E, B, \pi, F)$是一个以$B$为底,以$F$为纤维的纤维丛(fiber bundle)。
	\[
		\xymatrix{
			\pi^{-1}\left(U\right)\ar[rr]^\varphi \ar[dr]_\pi&&U\times F \ar[dl]^{\mathrm{proj}_1}\\
			&U&
			}
	\]

一般称呼$B\times F$为平凡丛。如果$F$是一个矢量空间,则$(E, B, \pi, F)$被称为一个矢量丛。显然,$\calf(M)$是以$M$为底$\rr$为纤维的纤维丛。

\para 设$(E, B, \pi, F)$是一个纤维丛,设$U\subset B$是一个开集,则$U$上的光滑截面(section)定义为一个光滑映射$s:B\to E$满足$\pi\circ s=\id_U$.

\begin{figure}[htp]
\centering
	\begin{tikzpicture}[scale=1]
		\draw (-2,-0.3)--(2,-0.3)--(2,1)--(-2,1)--cycle;
		\node [label=left:$F$] (F) at (-1.8,0.35) {};
		\node [label=below:$M$] (M1) at (1.2,-0.2) {};
		\node [label=below:$M$] (M2) at (1.2,-1.2) {};
		\node [label=below:$E$] (E) at (0.7,0.7) {};
		\draw (-2,-1.3)--(2,-1.3);
		\node [fill=black, inner sep=1pt, label=below:$p$] (p) at (-0.3,-1.3) {};
		\draw [color=black, domain=-1.6:1.6] plot (\x,{0.3*sin(2*\x r)+0.5});
		%0.5-0.3*sin(0.6)=0.330607...
		\node [fill=black, inner sep=1pt, label=right:\tiny$s(p)$] (s) at (-0.3,0.3306) {};
		\draw (-0.3,-0.5)node[below]{\small$\pi^{-1}(p)$}--(-0.3,1.2);
	\end{tikzpicture}
	\caption{Trivial Bundle and its Section}
\end{figure}

一个纤维丛不一定有整体截面,但是一定有局部截面,因为纤维丛在局部都是平凡丛,而平反丛一定有截面,比如常值截面$s(x)=a\in F$。对于一个纤维丛,直观来看,就是在底流形$B$每一点$p$,都放一个$\pi^{-1}(p)\cong F$,而所谓的截面,就是在每一点$p$,都选定$\pi^{-1}(p)\cong F$中的一个元素,这其实也就是矢量场的基本想法。

\para 反过来,如果给定了每一点的纤维,则我们有可能拼出一个纤维丛。切丛和余切丛正是如此定义的。

\para 流形$M$的切丛(tangent bundle) $TM$被在集合上被定义为
\[
	TM=\coprod_{x\in M}T_xM=\bigcup_{x\in M} \left\{x\right\}\times T_xM=\bigcup_{x\in M} \left\{(x, y)\vert\; y\in T_xM\right\}.
\]
余切丛(cotangent bundle)$T^*M$在集合上被被定义为
\[
	T^*M=\coprod_{x\in M}T^*_xM=\bigcup_{x\in M} \left\{x\right\}\times T^*_xM=\bigcup_{x\in M} \left\{(x, y)\vert\; y\in T_x^*M\right\}.
\]

\para 来看切丛,$M$显然是底流形,而$\pi$也可以显然地通过把$(p,v)\in TM$映射到$p\in M$来定义。剩下的,我们要赋予$TM$一个光滑流形结构,然后检验是否满足纤维丛的定义。

为此,对于$p\in M$,找一个坐标卡$(U,\varphi)$,在这张卡内,$T_qM$通过$\varphi_{*q}$同构于$\rr^n$,我们这样选取$\pi^{-1}(U)$上的微分结构,使得他通过$\id_U\times \varphi_*$光滑同胚于$U \times \rr^n$,这样$TM$就有了一个坐标卡$(\pi^{-1}(U),\varphi\times \varphi_*)$,于是他是一个光滑流形,也是一个以$M$为底,$\rr^n$为纤维的纤维丛。同样地,$T^*M$也是一个纤维丛。

\para 所以$U$上的光滑切矢量场就是$TM$的$U$上的一个光滑截面。

\para 设$X$是一个$U$上的光滑矢量场,如果一条光滑曲线$\sigma:(-\epsilon,\epsilon)\to U$且$\sigma(0)=p$,满足$X(\sigma(t))=\dot{\sigma}(t)$,则称$\sigma$是$X$在$p$附近的一条积分曲线。

将矢量场局部写出来,$X(\sigma(t))=X^i(\sigma(t))\partial_t$,所以问题归结到了求解微分方程
\[
	\frac{\dd x^i\circ \sigma}{\dd t}(t)=X^i(\sigma(t)),
\]
他的初值为$\sigma(0)=p$。微分方程的(光滑)解在局部存在且唯一,所以我们得到了:

\para 在$p$附近,对$X$存在唯一的积分曲线$\sigma:(-\epsilon,\epsilon)\to U$.

\lem 设$X$是$U$上的光滑矢量场,如果$X_p\neq 0$,则存在一个$p$的邻域$V$,在$V$上存在一组坐标使得$X$可以写作$\partial_1$.

\proof
	完全是局部的结果,我们就在欧式空间里面证明,即找一组新的坐标来把$X=X^i\partial_i$变成$\partial'_1$. 此外,如果我们证明了可以写作$a \partial_1$,那么再令$x'$是$\partial x^1/\partial x'^1=a$的解(这个解积个分就出来了),那么自然就有$\partial'_1=a\partial_1$.

	不妨假设$X^1$在$p$的某个邻域不为零,现在我们来解常微分方程组
	\[
		\frac{\dd x^i}{\dd x^1}=\frac{X^n(x^1,\cdots,x^n)}{X^1(x^1,\cdots,x^n)},
	\]
	给定初值为$\{\varphi^i(0;v^2,\cdots,v^n)=v^i:2\leq i \leq n\}$,我们知道解$\{x^i=\varphi^i(x^1;v^2,\cdots,v^n)\}_{2\leq i \leq n}$局部存在且光滑依赖于初值$\{v^2,\cdots,v^n\}$以及$x^1$,所以我们选取新坐标$\{v^1,v^2,\cdots,v^n\}$使得
	\[
		\{x^1,x^2,\cdots,x^n\}=\{v^1,\varphi^2(v^1,\cdots,v^n),\cdots,\varphi^n(v^1,\cdots,v^n)\},
	\]
	容易计算他在$v^1=0$处的Jacobian行列式$\det(\partial x/\partial v)=1$,所以这是一个合理的坐标选取。

	最后,注意到$X^i=X^1 \dd x^i/\dd x^1=X^1 \dd x^i/\dd v^1$,所以
	\[
		X=X^i\partial_i=X^1 \frac{\dd x^i}{\dd v^1}\frac{\partial}{\partial x^i}=X^1\partial'_1.
	\]\qed

设$X$是$U$上的光滑矢量场,对$U$上的每一点$p$,都可以在$p$附近找到他的一条光滑积分曲线$\sigma_p$,上面的点$\sigma_p(t)$我们也记作$\sigma_t(p)$,这样我们就得到了一个新的一族映射$\{\sigma_t:U\to U\}$,当$t=0$的时候,$\sigma_0=\id$.这样的一族映射$\{\sigma_t\}$被称为矢量场$X$的流。如果需要明确是那个矢量场的时候写作$\{\sigma^X_t\}$. 由微分方程解的唯一性可以发现$\sigma_t\circ \sigma_s=\sigma_{s+t}$.

对于一个矢量场,整体流的存在性是不能保证的,比如对$t=1$的时候,是否对每一个$p$变换$\sigma_1$都有意义?但是,至少在局部,我们可以保证在一定范围内的参数都是有意义的,对于局部的问题,这个存在性已经基本够使了。

\para 光滑流形$M$上的光滑矢量场$X$的支集被定义那些使$X$不为$0$的点集的闭包。如果$X$有紧支集,则$X$的流的参数可以全局定义到$\rr$上面去。特别地,如果流形是紧的,则对每一个光滑矢量场都成立。

设支集为$K$,找一个他的开覆盖,使得每一个开覆盖内$\sigma_t$都对某一个小区间$t\in (-\epsilon,\epsilon)$上有定义,由于紧,所以可以找到有限的子覆盖,所以在这些子覆盖里面,把最小的$\epsilon_{\text{min}}$挑出来,则在$K$上,对$t\in(-\epsilon_{\text{min}},\epsilon_{\text{min}})$,$\sigma_t$都有定义。现在对$p\notin K$定义$\sigma_t(p)=p$,容易检验$\epsilon_t$定义良好且还是光滑映射。最后,对$t>\epsilon_{\text{min}}$,我们可以从某个$t_0\in (-\epsilon_{\text{min}},\epsilon_{\text{min}})$反复复合$\sigma_{\epsilon_{\text{min}}/2}$,反之对$t<-\epsilon$亦然。

\para 现在我们将积分曲线的问题稍稍拓展一下,比如我们现在有两个矢量场$X$和$Y$,他们在每一点张成一个平面,类比于积分曲线,我们要问是否存在一个光滑曲面,使得这个曲面在每一点的切空间都是这俩矢量场张成的平面?

可以直接想象一下怎么处理这样的问题,直观来看,积分曲面可以由积分曲线拼成,即在$p$附近,$X$的一条一条积分曲线和$Y$的一条积分曲线编成一张网,这张网其实应该就在积分曲面上。因此积分曲面存在与否当且仅当这张网足够光滑,不能突然错开。此时的错开,就是说,从网格的一个端点处,沿着两条路径走到的终点处于积分曲面的两侧。

为简单见,我们看看这样一个曲边四边形。从$p$出发,沿着$X$的积分曲线走$s$,走到了$\sigma^X_s(p)$,再从这点出发,沿着$Y$的积分曲线走$t$,走到$\sigma^Y_t\circ\sigma^X_s(p)$.同样,从$p$出发,沿着$Y$的积分曲线走$t$,走到了$\sigma^Y_t(p)$,再从这点出发,沿着$X$的积分曲线走$s$,走到$\sigma^X_s\circ\sigma^Y_t(p)$。

这样我们就得到了弯曲的四边形,$p$是其中一个端点,但是一般来说,沿着两条路径并不会有相同的终点,即$p$的对角线方向的另一端点不存在,此时两条路径并不能闭合成一个弯曲的四边形。甚至,即使积分曲面存在,我们也不能得到一个闭合的四边形(网格),但是,如果沿着两条路径的终点是错开的,则积分曲面依然不存在。

实际上,因为$X$和$Y$的光滑,我们可以适当延长(或缩短)相比$s$和$t$小量的在一条路径上走的时间使得弯曲的四边形变成一个网格(此时局部积分曲面存在),或者,永远是错开的(此时局部积分曲面不存在)。

我们现在需要比较两条路径两个终点的差,在欧式空间里面,我们可以比较这两个点的距离。但是在流形上这样做是不方便的,我们可能连一个很直接的计算两个点距离的手段都没有。为了克服这个困难,我们可以采用“表示”的手段,取一个$U$上的光滑函数$f$,比较$f(\sigma^Y_t\circ\sigma^X_s(p))$和$f(\sigma^X_s\circ\sigma^Y_t(p))$。但光取一个$f$肯定是不够的,取而代之,我们可以取遍$U$上所有的光滑函数,如果我们关于所有的光滑函数都计算出了$f(\sigma^Y_t\circ\sigma^X_s(p))-f(\sigma^X_s\circ\sigma^Y_t(p))$(在$s$, $t$都很小时),那么就可以确认这两个坐标相差很小的程度。所以现在就是要计算在$s$, $t$都很小时的
\[
	g(s,t)=f(\sigma^Y_t\circ\sigma^X_s(p))-f(\sigma^X_s\circ\sigma^Y_t(p)).
\]

因为$g$是光滑函数,我们在$(0,0)$局部展开他,求导就可以得到系数。显然,他到二阶为止的导数都为$0$,并且$\partial_s^2g(0,0)=\partial_t^2g(0,0)=0$,所以他最低阶不为零的只可能是$\partial_s\partial_t g(0,0)$,这就是说,我们要求$\lim_{s,t\to 0}g(s,t)/st$.

\para 在$t$很小的时候,$f(\sigma_t^X p)=f(p)+tXf(p)+o(t^2)$.为了证明他,只要求$p$处的导数就行了,设$\sigma^X(t)$是$\sigma^X(0)=p$的$X$的积分曲线,则
	\[
		\frac{\dd}{\dd t}\biggr|_{t=0}f(\sigma_t^X p)=\frac{\dd}{\dd t}\biggr|_{t=0}f\circ \sigma^X (t)=f_{* p}X=Xf(p).
	\]

所以(暂时在记号上略去高阶项)
\[
	f(\sigma^Y_t\circ\sigma^X_s(p))=f(\sigma^X_s(p))+tYf(\sigma^X_s(p))=f(p)+sXf(p)+tY(\sigma^X_s\circ f(p)),
\]
以及
\[
	g(s,t)=sXf(p)+tY(\sigma^X_s\circ f(p))-tYf(p)-sX(\sigma^Y_t\circ f(p)),
\]
其中
\[
	Y(\sigma^X_s\circ f(p))-Yf(p)=sX(Y(f))(p),
\]
所以
\[
	g(s,t)=st\bigl(X(Y(f))(p)-Y(X(f))(p)\bigr),
\]
这就是说$\partial_s\partial_t g(0,0)=X(Y(f))(p)-Y(X(f))(p)$.

\para 定义两个矢量场$X$, $Y$的Lie括号为$[X,Y]$,他满足$[X,Y](f)=X(Y(f))-Y(X(f))$。

如果采用局部表示$X=X^i\partial_i$和$X=Y^i\partial_j$,则我们可以计算出
	\begin{align*}
	[X,Y](f)&=X^i\partial_i(Y^j\partial_j f)-Y^i\partial_i(X^j\partial_j f)\\
	&=X^i\partial_iY^j\partial_j f-Y^i\partial_i X^j\partial_j f\\
	&=(X^i\partial_iY^j-Y^i\partial_i X^j)\partial_jf.
	\end{align*}
	因此,尽管形式上是二阶的,但$[X,Y]$还是一个切矢量场,局部写作$[X,Y]=(X^i\partial_iY^j-Y^i\partial_i X^j)\partial_j=(X(Y^j)-Y(X^j))\partial_j$.


\para 设$X_1$, $\cdots$, $X_k$是一族切矢量场,记
	\[
		D=\{f_1X_1+\cdots+f_kX_k:\forall 1\leq i \leq k,\,\, f_i\in \calf(M)\},
	\]
	称他为流形$M$上的一个被$\{X_i\}$张成的分布。

\para 设$X$和$Y$张成一个分布$D$,则局部积分曲面存在当且仅当在$[X,Y]\in D$.

如果局部积分曲面存在,那么曲边四边形完全处于积分曲面上面,尽管沿着两条路径得到的终点可能不同,但是这两个点的连线(或者他的切矢量)应该在$s\to 0$, $t\to 0$的时候确定了一个切矢量(由于$X$和$Y$的光滑性,所以这个切矢量并不依赖于连线的选取),他处于积分曲面的切子空间(由$X_p$和$Y_p$张成)里面,且正比于$[X,Y]_p$,所以$[X,Y]_p=a(p)X_p+b(p)Y_p$.而$a$和$b$的光滑性是显然的。

如果不存在,就是说两个端点分处$X_p$和$Y_p$张成的切子空间两侧,所以两个点的连线确定的那个切矢量应该不在和$X_p$和$Y_p$张成的切子空间里,即$[X,Y]_p\neq aX_p+bY_p$.


\para 问题可以问得更广一点,设$X_1$, $\cdots$, $X_k$是一族切矢量场,他们是否(至少在局部)有积分“曲面”存在?回答是Frobenius定理:设$X_1$, $\cdots$, $X_k$在点$p$张成的分布为$D$,则局部存在积分“曲面”当且仅当对任意的$i$和$j$成立$[X_i,X_j]\in D$。

当然,这个结论可以更加形式地证明他,如果不信任上面的直观想法,则可以参看任何一本微分流形的教材,我这里就略去了。

\para 令$\varphi$是流形$M$上的光滑可逆变换,设矢量场$X$的流为$\sigma_t$,则$\varphi_*X$的流为$\varphi\circ \sigma_t\circ\varphi$.

设在$p$处的切矢量为$X_p$,经过$p$的$X$的积分曲线为$\sigma(t)$,使用变换$\varphi$,变成了$q=\varphi(p)$,$q$处的切矢量$\varphi_{*p}X_p=(\varphi_*X)_q$以及积分曲线$\varphi(\sigma(t))=\varphi(\sigma_t q)=\varphi\circ\sigma_t\circ \varphi^{-1}(p)$.此即结论。

\para 所以,$\varphi_*X=X$当且仅当$\varphi\circ \sigma^X_t=\sigma^X_t\circ \varphi$成立。

\para 直接的计算,我们有:
	\[
		[X,Y]=\lim_{t\to 0}\frac{1}{t}\bigl(Y-(\sigma_t^X)_*Y\bigr).
	\]

所以,如果$\sigma_t^X$和$\sigma_s^Y$可交换,即$\sigma_t^X\circ \sigma_s^Y=\sigma_s^Y\circ \sigma_t^X$,则$(\sigma_t^X)_*Y=Y$以及$[X,Y]=0$.

设$\varphi:M\to N$是一个光滑映射,$X$和$Y$分布是$M$和$N$上的光滑函数,称他们是$\varphi$相关的,如果$\varphi_*X(f)=Y(f)\circ \varphi$对任意光滑函数$f$成立。局部来看,$X_p(f\circ \varphi)=Y_{\varphi(p)}(f)$.

若$X_1$与$Y_1$是$\varphi$相关的,$X_2$与$Y_2$是$\varphi$相关的,则$[X_1,X_2]$和$[Y_1,Y_2]$是$\varphi$相关的,因为
\begin{align*}
	\varphi_*[X_1,X_2](f)&=X_1(X_2(f\circ \varphi))-[1\leftrightarrow 2]\\
	&=X_1(Y_2(f)\circ\varphi)-[1\leftrightarrow 2]\\
	&=Y_1(Y_2(f))\circ \varphi-[1\leftrightarrow 2]\\
	&=[Y_1,Y_2](f)\circ \varphi.
\end{align*}
因为对于一个同胚而言,$\varphi_*(X)$被定义为$p\mapsto \varphi_{*\varphi^{-1}(p)}X_{\varphi^{-1}(p)}$,或者$\varphi(p)\mapsto \varphi_{*p}X_p$,这就是说
\[
	X_p(f)=\varphi_*(X)_{\varphi(p)}(f),
\]
因此$X$与$\varphi_*X$是$\varphi$相关的。


\para 令$\varphi$是$U$上的光滑同胚$\varphi:U\to \varphi(U)$,$X$和$Y$是$U$上的矢量场,则在$U$上$\varphi_{*}[X,Y]=[\varphi_{*}X,\varphi_{*}Y]$.

因此,作为局部的同胚,$\sigma_s^X$可以适用
\[
	(\sigma_s^X)_*[X,Y]=[X,(\sigma_s^X)_*Y]=\lim_{t\to 0}\frac{1}{t}\bigl((\sigma_s^X)_*Y-(\sigma_{s+t}^X)_*Y\bigr)=\frac{\dd}{\dd t}\biggr|_{t=s}(\sigma_t^X)_*Y.
\]

如果$[X,Y]=0$,则$(\sigma_t^X)_*Y$是一个常矢量(局部来看,系数为常数),因此$Y$在$(\sigma_t^X)_*$作用下不变,这就是说$Y=(\sigma_t^X)_*Y$,于是$\sigma_t^X$和$\sigma_s^Y$可交换.

\para 综上,$[X,Y]=0$当且仅当$\sigma_t^X$和$\sigma_s^Y$可交换。

既然流是可交换的,那么以前我们谈的那个曲边四边形总是可以闭合的,所以这种情况下积分流形局部肯定存在,因为在局部我们可以一块一块曲边四边形拼起来。由于显然的$[\partial_i,\partial_j]=0$,如果我们能够选取局部坐标使得一族矢量场$\{X_i:1\leq i \leq k\}$变成$\{\partial_i:1\leq i \leq k\}$,则积分曲面存在。那么什么时候$\{X_i\}$是可以变成$\{\partial_i\}$呢?答案前面已经有了,$[X_i,X_j]\in D$,分布$D$由$\{X_i\}$生成。当然,这可以直接证明,所以这也是证明积分曲面存在性的一种思路。

最后给个例子,$n$维欧式空间,$\{\partial_i:1\leq i \leq k\}$的可能的积分曲面$\{x_i=c_i:k+1\leq i \leq n\}$,其中$c_i$是常数。如果这个积分曲面还是连通的,设$\pi$是往最后$n-k$个坐标的投影,则$\pi_*\partial_i=0$,其中$1\leq i \leq k$,因此$(\pi\circ i)_*=0$,其中$i$是积分流形往欧式空间的嵌入,此时由积分流形的连通性,$\pi\circ i$是常值映射。此时的积分流形就是上面的$\{x_i=c_i:k+1\leq i \leq n\}$.

\section{Cotangent Vector Field}

\para 设$U\osub M$,$U$上的光滑余切矢量场(或者叫做一个1-形式)是余切丛在$U$上的一个光滑截面。余切矢量场$\omega$都可以写作$\omega=a_i\dd x^i$,其中$a_i$是$V$上的实值函数,而$\dd x^i$在不同的点分属不同的余切空间。因为是光滑余切矢量场,所以$\{a_i\}$在点$p$是光滑函数。

一个余切矢量场和一个切矢量场之间存在作用$\omega(X)=X(\omega)$可以得到一个光滑函数,具体来说就是在每一点$p$,$\omega_p(X_p)\in \rr$.

% \para 正如切矢量场,对余切矢量场$\omega$,光滑性也有如下判据:$\omega$是光滑的,当且仅当$\omega(X)$是光滑的对$U$上的任意光滑切矢量场$X$成立。

\para 设$f$是$U$上的光滑函数,显然$\dd f$是一个$U$上的光滑余切矢量场。记$U$上的光滑函数的集合为$\Omega^0(U)$,记$U$上的光滑余切矢量场的集合为$\Omega^1(U)$,则$\dd: \Omega^0(U)\to \Omega^1(U)$.

下面我们要把Frobenius定理改写成余切矢量场的形式,这就变成了经典的Pfaff方程,正是当年关于Pfaff方程的研究,Cartran第一次提出了(高阶)外微分和微分形式的概念(我们现在只谈了一阶的情况),在他那里,1-形式之间的乘法被定义成反对称的。从前一节谈论Frobenius定理来看,Pfaff方程是一个关于积分曲面的问题,所以从这个角度来看,反对称的来由归根结底是为了积分。

\para 称一个1-形式$\omega$是完全可积的,如果存在两个光滑函数$f$和$g$使得$\omega=f\dd g$,此时$f$被称为$\omega$的积分因子。

1-形式的完全可积性联系着所谓的首次积分问题。设$f$是一个光滑函数且$\dd_p f$处处不为零,则$f(p)=a$(如果解存在)决定了$M$中的一个正则子流形$N_a$(有时候叫做一个曲面)。再设$X$是$M$内的光滑矢量场,则$\dd f(X)=0$恒成立当且仅当处处成立$X_p\in T_pN_{f(p)}$。

实际上,任取一点$p\in M$,只要检验$X_p(f)=0$即可,选一条$N_{f(p)}$上的一条光滑曲线$c$,使得$c(0)=p$且$c'(0)=X_p\in T_pN_{f(p)}$,由于$f(c(t))=f(p)$恒成立,对其在$t=0$处求导就得到了$\dd_pf(X_p)=X_p(f)=0$.反过来,如果在一点处$X_p\notin T_pN_{f(p)}$,则$X_p(f)\neq 0$。

固定$f$,将所有$\dd f(X)=0$的$X$拿出来,他组成一个$n-1$维的分布,$\{N_a\}$就是这个分布的一族积分流形,因为$X$在每一点都完全位于经过那一点的某个$N_a$的切空间内。我们称$N_a=\{p\in M:f(p)=a\}$是Paffa方程$\dd f=0$的解。从上面来看,一个Paffa方程要有解,那么解应该是一个积分曲面才是,即,Paffa方程$\omega=0$的解是使得$\omega(X)=0$的所有的$X$的积分曲面。

现在假设一个1-形式$\omega$是完全可积的,即他可以写作$\omega=f\dd g$,那么Paffa方程$\omega=0$等价于$\dd g=0$,这就确定了一个积分曲面。

\para 设$\omega$是一个1-形式,记分布$\ker \omega$是由满足$\omega(X)=0$的所有$X$张成的一个分布。记$\ker(\omega_1,\cdots,\omega_r)=\cap_{i=1}^r\ker \omega_i$.

% 则这个$\omega$完全可积当且仅当$\ker \omega$存在积分曲面。

\para 在局部,对任意一个分布$L$,存在一族余切矢量场$\{\omega_i:1\leq i \leq r\}$使得$L=\ker(\omega_1,\cdots,\omega_r)$.

实际上,一个分布在局部,和他在流形一点处(一个矢量空间内)是很相似的。设$L$是一个分布,由$r$个光滑的切矢量场$\{X_i:1\leq i \leq r\}$张成,则在局部,我们可以找到$n-r$个光滑矢量场$\{X_i:r+1\leq i \leq n\}$,使得$\{X_i:1\leq i \leq n\}$处处线线性无关。依然在局部,我们可以找到与其对偶\footnote{即满足$\omega_i(X_j)=\delta_{ij}$.}的1-形式$\{\omega_i:1\leq i \leq n\}$,那么那些使得$\omega(X)$对$X\in L$成立的1-形式局部由$\{\omega_i:r+1\leq i \leq n\}$张成。因此,局部上,一个分布$L$可以写作$L=\ker(\omega_{r+1},\cdots,\omega_n)$,等价地,可以写作一个Paffa方程组$\{\omega_i=0:r+1\leq i \leq n\}$.

因为局部存在积分曲面的充要条件是任取$X$, $Y\in D$满足$[X,Y]\in D$,所以如果$L=\ker(\omega_{r+1},\cdots,\omega_n)$存在积分曲面,应该有$\omega_i([X,Y])=0$。

\para 设$\omega=f\dd g$,其中$f$和$g$是光滑函数,则对一般的光滑矢量场$X$, $Y$成立。
\begin{equation}
\begin{split}
	\omega([X,Y])=f\dd g([X,Y])&=f [X,Y](g)\\
	&=fX(Y(g))-fY(X(g))\\
	&=X(fY(g))-X(f)Y(g)-Y(fX(g))+Y(f)X(g)\\
	&=X(\omega(Y))-Y(\omega(X))-\bigl(X(f)Y(g)-Y(f)X(g)\bigr).
\end{split}
\end{equation}

对于$X(f)Y(g)-Y(f)X(g)$,我们可以将其改写为$\dd f(X)\dd g(Y)-\dd g(X)\dd f(Y)$,因为这是关于$X$和$Y$的双线性函数,我们可以引入一个张量$\dd f\otimes \dd g$使得$\dd f\otimes \dd g(X,Y)=\dd f(X)\dd g(Y)$,则
\[
	\dd f(X)\dd g(Y)-\dd f(Y)\dd g(X)=\dd f\otimes \dd g(X,Y)-\dd g\otimes \dd f(X,Y)=(\dd f\otimes \dd g-\dd g\otimes \dd f)(X,Y).
\]
记$\dd f\wedge \dd g=\dd f\otimes \dd g-\dd g\otimes \dd f$,其中的$\wedge$被称为楔积,记$D(f\dd g)=\dd f\wedge \dd g$,则式(\theequation)变成了
\[
	D(f\dd g)(X,Y)=X(\omega(Y))-Y(\omega(X))-\omega([X,Y]).
\]
从式子右端来看,$\dd f\wedge \dd g(X,Y)$并不依赖于$\omega$的具体形式$\omega=f\dd g$。实际上,任意一个1-形式$\omega$可以写作$\omega=\sum_i f_i\dd g_i$,所以对于一般的情况,式(\theequation)应该写作
\[
	\sum_iD(f_i\dd g_i)(X,Y)=X(\omega(Y))-Y(\omega(X))-\omega([X,Y]).
\]
如果我们把$D$看做线性算子,则对于任意一个1-形式,我们都定义了一个线性算子,满足
\begin{equation}
	D(\omega)(X,Y)=X(\omega(Y))-Y(\omega(X))-\omega([X,Y]).
	\label{eq:1.3}
\end{equation}

\para 当然,我们也可以反过来通过式\eqref{eq:1.3}来定义$D(\omega)$,顺序在这里不是紧要的。紧要的是,$D(\omega)$决定了$\ker \omega$是否容许一个积分曲面。这是因为,如果$X$, $Y\in \ker \omega$,则式\eqref{eq:1.3}变成了
\[
	D(\omega)(X,Y)=-\omega([X,Y]),
\]
所以$D(\omega)(X,Y)=0$当且仅当$[X,Y]\in \ker \omega$.

这正是Cartan当年提出微分形式时候的处境,那时候,他从Frobenius和Darboux那里知道了,不同的Pfaff形式的等价条件就联系在一个bilinear covariant上面,而这个bilinear covariant就是我们这里的$D(\omega)$.

从这个角度来看,正因为有Frobenius定理,或者更本质一点,我们需要把积分曲线拼成积分曲面,我们需要考察两个矢量场$X$和$Y$的Lie括号$[X,Y]$,而这个Lie括号的反对称性来自于我们比较两条路径。现在,这种反对称性反应在了1-形式之间的楔积,使得他构成了一个(吃掉两个矢量场的)反对称函数。所以,从Cartan这里,反对称性的来源应该是为了处理积分曲面的存在性,而由$[X,Y]$自然诱导出来的。

\para 外代数的复习在附录,对于矢量空间$V$的$k$-次外代数记做$\Omega^k(V)$。在流形上的一点$p$处,记$\Omega_p^k=\Omega^k(T_pM)$,则$\Omega_p^1=T_p^*M$.类似切丛和余切丛,我们可以使用$\Omega_p^k$拼出一个$k$-形式丛$\Omega^k$。

\para 在$U$上的一个光滑$k$-形式被定义为$\Omega^k$在$U$上的一个光滑截面。所有$U$上的光滑$k$-形式的集合记做$\Gamma(\Omega^k,U)$.显然,$U$上的光滑函数可以看成一个光滑$0$-形式,一个光滑余切矢量场是一个光滑1-形式。如果$\omega$是一个光滑$1$-形式,则$D(\omega)$是一个光滑$2$-形式。下面我们所称的形式都是光滑的,我们将省略光滑二字。

\para 设分布$L$由$\{X_i:1\leq i \leq r\}$张成,且$L=\ker(\omega_{r+1},\cdots,\omega_n)$,如果$D(\omega)(X_i,X_j)=0$成立,则$D(\omega)$可以写作\[
	D(\omega)=\sum_{i=r+1}^n \psi_i\wedge \omega_i,
\]
其中$\psi_i$是一次微分式.

实际上,局部地$\{\omega_i:1\leq i\leq n\}$构成一组基,则
\[
	\dd \omega = \sum_{i=r+1}^n \psi_i\wedge w_i +\sum_{i,j=1}^r a_{ij}\omega_i\wedge \omega_j,
\]
其中$\psi_i$是一次微分式,而$a_{ij}$是光滑函数,且关于指标是反对称的。因为$\omega_i(X_j)=\delta_{ij}$,所以
\[
	0=\dd \omega(X_i,X_j)=\sum_{p,q=1}^r a_{pq}\omega_p\wedge \omega_q(X_i,X_j)=\sum_{p,q=1}^r a_{pq}(\delta_{ip}\delta_{jq}-\delta_{jp}\delta_{iq})=2a_{ij}.
\]

\para 设分布$L$由$\{X_i:1\leq i \leq r\}$张成,且$L=\ker(\omega_{r+1},\cdots,\omega_n)$,则$L$存在积分曲面当且仅当,
\[
	D(\omega_i)=\sum_{j=r+1}^n \psi_{ij}\wedge \omega_i
\]
对每一个$i$都成立。

\para 特别地,设分布$L=\ker(\omega)$,则局部积分曲面存在的充分必要条件是$D(\omega)=\psi\wedge \omega$,再或者$D(\omega)\wedge \omega=0$.则也就是$\omega$完全可积的充分必要条件。

\para \label{dd=0}可以计算得到$D(\dd f)=0$。实际上,从\eqref{eq:1.3}
\[
	D(\dd f)(X,Y)=X(Y(f))-Y(X(f))-[X,Y](f)=[X,Y](f)-[X,Y](f)=0,
\]
对任意的矢量场$X$, $Y$都成立。

\section{Exterior Derivative}

\para 函数$f$的支集$\supp(f)$被定义为$\{x\in M:f(x)\neq 0\}$的闭包。

\para 单位分解:设$\{U_\alpha\}_{\alpha\in I}$是$M$的一个开覆盖,如果存在可数个光滑函数$g_i\in \cal(M)$满足:

\no{1} 对于任意的$x\in M$和$i\in I$,都有$0\leq g_i(x)\leq 1$.

\no{2} 对每个$g_i$,都存在一个${\alpha_i}$使得$\supp(g_i)\subset U_{\alpha_i}$.

\no{3} 集族$\{\supp(g_i)\}$局部有限,即任取$p\in M$,存在$p$的邻域$U$使得$U$只和集族$\{\supp(g_i)\}$中的有限个集合相交非空。

\no{4} 因为上一个性质,所以在一点累加$g_i$时,只有有限项非零。我们最后的要求就是$\sum_i g_i=1$.

则称$\{g_i\}$是从属于开覆盖$\{U_\alpha\}_{\alpha\in I}$的一个单位分解。

\lem 流形$M$上的每一个开覆盖都存在从属于他的单位分解。

这个引理的证明看附录。

\para \label{unit} 设$V\subset U$,且$\bar{V}\subset U$,则$M-\bar{V}$和$U$构成$M$的一个开覆盖,我们可以找到$h$, $g\in \calf(M)$使得$h+g=1$处处成立,且$h|_{M-U}=0$以及$g|_{\bar{V}}=0$,因此$h|_{\bar{V}}=1$.

\para 在很久很久以前,对$U$上的一个光滑函数$f$,我们定义了外微分$\dd$,使得$\dd f$是一个1-形式,而在上一节,我们对$U$上的一个1-形式$\omega$,定义了$D$使得$D(\omega)$是一个2-形式。更一般地,我们希望可以定义如下一个算符
\[
	\dd_k:\Gamma(\Omega^k,U)\to \Gamma(\Omega^{k+1},U),
\]
使得$\dd_0=\dd$,$\dd_1=D$。我们将$\{\dd_k\}$统称为外微分算符,统一记做$\dd$,他完成了一个$k$-形式到一个$(k+1)$-形式的转变。

所以,我们对于线性算符$\dd$,需要满足如下性质,

\no{1} 对于光滑函数$f$,$\dd f$就是我们前面定义的微分。

\no{2} 任取光滑函数$f$,$\dd^2 f=\dd (\dd f)=0$,这来自于\pararef{dd=0},也是暗示$\dd$作用在1-形式上应该和$D$差不多。

\no{3} 作为微分算符的Leibniz法则:设$\omega$是$k$-形式,则
\[
	\dd(\omega\wedge \eta)=\dd \omega \wedge \eta +(-1)^k \omega\wedge \dd\eta,
\]
其中$(-1)^k$的出现来自楔积的交换。这个Leibniz法则我们已经在计算$D(f\dd g)$和$\dd(fg)$的时候遇到过了。

\para \label{localform}先假设$\dd$在$U$上是存在的。若$\omega$是$V\osub U$上的$k$-形式,设$W$是$U$的一个开子集且$V$真包含$\overline{W}$.那么正如\pararef{unit}说的,可以找一个单位分解$h$,使得$h|_{\overline{W}}=1$以及$h|_{U-W}=0$,利用他就可以定义一个$U$上的光滑$k$-形式$h\omega$(他在$V$外为零,在$W$内等同于$\omega|_W$)。

\para \label{regular}设$U$是流形$M$上的开集且$V\osub U$,则存在$V$的一个开覆盖$\{W_\alpha\}_{\alpha\in I}$使得对每个$\alpha$成立$\overline{W_\alpha}\subset V$.

实际上这就是分离公理之一,正则性的直接推论。流形$M$是局部紧的Hausdorff空间,所以他是正则的。而对于正则空间内的每一个点$p$,已知邻域$V$,我们可以找到一个邻域$W_p\subset V$使得$\overline{W_p}\subset V$.然后遍历$p$,就找到了开覆盖$\{W_p\}_{p\in V}$.

\para \label{localpro}设$V\osub U$,如果$\omega|_V=0$,则$(\dd \omega)|_V=0$,即$\dd$是一个局部算符。

利用\pararef{regular}找一个$V$的开覆盖,我们只要证明对开覆盖中的一个开集$W$,$(\dd \omega)|_{W}=0$即可。同\pararef{localform}利用单位分解找一个光滑函数$h$,则$h\omega$在$U$上恒为零,所以
\[
	0=\dd (h\omega)=\dd h \wedge \omega +h\dd \omega
\]
限制在$W$上,$h|_{W}=1$且$(\dd h)|_{W}=0$(利用$\dd h$是一个微分这一个事实),所以$(\dd \omega)|_{W}=0$.

\pro \label{localdef}利用\pararef{localpro},设$V\osub U$,如果$U$上存在外微分形式$\dd$,则他诱导出了$V$上的一个外微分形式$\dd_V$。并且,再设$W\osub W$,则$\dd$在$W$上诱导的$\dd_W$和$\dd_V$在$W$诱导的$\dd_{VW}$是相同的。此外$(\dd_V \omega)|_W=\dd_W (\omega|_W)$对任意的$V$上的$k$-形式$\omega$成立。

\proof 和上面的想法差不多,利用\pararef{regular},我们可以找一个开覆盖。然后在每个开覆盖中的开集$W$上,同\pararef{localform}利用单位分解找一个光滑函数$h_W$,使得$h_W\omega$成为$U$上的光滑$k$-形式,定义$(\dd_V \omega)|_{W}=\dd (h_W\omega)|_{W}$.

设$W'$是开覆盖中的另外的开集,且$W\cap W'\neq \varnothing$。由于$(h_W\omega)|_{W\cap W'}=(h_{W'}\omega)|_{W\cap W'}$,因为\pararef{localpro},则
\[
	\dd (h_W\omega)|_{W\cap W'}=\dd (h_{W'}\omega)|_{W\cap W'},
\]
所以
\[
	((\dd_V \omega)|_{W})|_{W\cap W'}=((\dd_V \omega)|_{W'})|_{W\cap W'}
\]
保证了$(\dd_V \omega)|_W$在相交的开集上是相同的,这就使得我们可以黏结他们定义出一个$V$上的$(k+1)$-形式$\dd_V \omega$. 容易检验$\dd_V$满足所有外微分的性质。至于$\dd_W=\dd_{VW}$,从构造来看,这是显然的。

最后我们来检验等式$(\dd_V \omega)|_W=\dd_W (\omega|_W)$,对$W$利用\pararef{regular}找个开覆盖,在每一个开集$X$上,把上式左边限制到$X$上即$((\dd_V \omega)|_W)|_X=(\dd_V \omega)|_X=\dd(h_X\omega)$,同样,右边限制到$X$上即$(\dd_W (\omega|_W))|_X=\dd(h_X\omega|_W)=\dd(h_X\omega)$,所以等式成立。\qed

\para 设$W\subset V$是$U$中的开集,以及$\rho^k_{VW}$和$\rho^{k+1}_{VW}$分别是$k$-形式和$(k+1)$-形式的限制映射。设$\omega$是$V$上的任意$k$-形式,由于$\rho^{k+1}_{VW}\circ \dd_V (\omega)=(\dd_V \omega)|_W=\dd_W (\omega|_W)=\dd_W \circ \rho^k_{VW}(\omega)$,所以我们有
\[
\dd_W\circ \rho^k_{VW}=\rho^{k+1}_{VW}\circ \dd_V.
\]
换句话说,$\dd$诱导了预层$U\mapsto \Gamma(\Omega^k,U)$和预层$U\mapsto \Gamma(\Omega^{k+1},U)$间的自然变换(或者叫做函子间的态射)。

\para 容易证明,以$p$的邻域$W$赋予包含而成的归纳系,
\[
	\Omega_p^k=\varinjlim_{W\ni p}\Gamma(\Omega^k,W),\quad \Omega_p^{k+1}=\varinjlim_{W\ni p}\Gamma(\Omega^{k+1},W),
\]
以及有到点上的限制映射$\rho^k_{Wp}$和$\rho^{k+1}_{Wp}$。则colimt的泛性质,如下交换图告诉我们$\dd_p:\Omega_p^k\to \Omega_p^{k+1}$存在。
\[
	\xymatrix{
	&&\ar[dl]^{\rho^k_{Vp}}\Gamma(\Omega^k,V)\ar[dd]^{\rho^k_{VW}}\ar@/_/[lld]_{\rho^{k+1}_{Vp}\circ \dd_V} \\
	\Omega_p^{k+1}&\ar@{-->}[l]_(0.3){\dd_p}\Omega_p^{k}&\\
	&&\ar[ul]_{\rho^k_{Wp}}\Gamma(\Omega^k,W)\ar@/^/[llu]^{\rho^{k+1}_{Wp}\circ \dd_W}
	}
\]
以后,方便起见,即使$\omega$只是$U$的开集$V$上的$k$-形式的时候,我们依旧用$\dd \omega$来标记$\dd_V \omega$.这当然也就意味着,在记号上,$\dd(\omega|_W)=(\dd\omega)|_W$.

\pro 在流形$M$上,外微分算子$\dd$存在且唯一。

\proof 有了上面这些铺垫,我们只要在局部证明其唯一存在即可,然后把他拼起来,就像Proposition \pararef{localdef}做的那样。首先证明局部唯一性,为此当然假设$\dd$是存在的。在局部,我们找一族坐标,对于$k$-形式,他写作
\[
	\omega=\sum_{i_1,\cdots ,i_k} a_{i_1\cdots i_k}\dd x^{i_{1}}\wedge\cdots\wedge x^{i_{k}},
\]
由于$\dd$是线性算子,我们可以只考虑
\[
	\omega=a\dd x^{1}\wedge\cdots\wedge \dd x^{k}.
\]

由于Leibniz法则,
\[
	\dd \omega=\dd a \wedge \dd x^{1}\wedge\cdots\wedge \dd x^{k}+a\dd(\dd x^{1}\wedge\cdots\wedge \dd x^{k}),
\]
对于第二项,使用Leibniz法则,以及$x^1$是光滑函数,所以有的$\dd^2 x^1=0$,于是
\[
	\dd(\dd x^{1}\wedge\cdots\wedge \dd x^{k})=-\dd x^{1}\wedge\dd(\dd x^{2}\wedge\cdots\wedge \dd x^{k}),
\]
这样进行下去就知道他是零。于是
\[
	\dd \omega=\dd a \wedge \dd x^{1}\wedge\cdots\wedge \dd x^{k},
\]
这样$\dd$在局部的表现完全由他那些性质唯一决定,唯一性得证。

剩下的存在性,我们就把上面的过程反过来,在局部的$k$-形式写作
\[
	\omega|_U=\sum_{i_1,\cdots ,i_k} a_{i_1\cdots i_k}\dd x^{i_{1}}\wedge\cdots\wedge x^{i_{k}},
\]
则定义其外微分为
\[
	\dd(\omega|_U)=\sum_{i_1,\cdots ,i_k} \dd a_{i_1\cdots i_k}\wedge\dd x^{i_{1}}\wedge\cdots\wedge x^{i_{k}},
\]
不难检验外微分的性质都可以得到满足。这样我们就在局部证明了外微分的存在性。

最后,由于$(\dd(\omega|_U))|_{U\cap V}=\dd(\omega|_U|_{U\cap V})=\dd(\omega|_V|_{U\cap V})=(\dd(\omega|_V))|_{U\cap V}$,所以我们可以将局部定义的外微分算子粘起来得到流形$M$上的一个外微分算子。由于外微分算子的局部唯一性,所以也就得到了他的整体唯一性。\qed

\pro 虽然是一个很简单的命题,但是很重要:$\dd^2=0$.

\proof 局部对单项式证明即可,设$\omega=a\dd x^{1}\wedge\cdots\wedge \dd x^{k}$,则
\[
	\dd \omega=\dd a\wedge \dd x^{1}\wedge\cdots\wedge \dd x^{k}
\]
以及
\[
	\dd^2 \omega=\dd^2 a\wedge \dd x^{1}\wedge\cdots\wedge \dd x^{k}-\dd a\wedge \dd^2 x^{1}\wedge\cdots\wedge \dd x^{k}+\cdots=0.
\]
\qed

\para 我们将$\dd \omega=0$的形式$\omega$称为闭形式,将$\omega=\dd \eta$的形式$\omega$称为恰当形式,则这一命题告诉我们,恰当形式一定是闭形式,反之不一定。我们定义$H^k(M)=\ker(d_k)/\im(d_{k-1})$为流形$M$的第$k$个上同调群,其中模的运算是看做加法群的商群运算,这个$H^k(M)$即表征了闭的$k$-形式在除去一个恰当形式后的等价类。此外约定当$k<0$的时候,$H^k(U)=0$.

当$k=0$的时候,$H^0(U)$就是$\mathrm{ker}\left(\dd:\calf(U)\to\Omega^1(U)\right)$.如果$U$是连通的,$\dd f=0$就昭示着$f$在恒为常数,$H^0(U)$就是这些函数构成的矢量空间。那么如果有着不同的连通分支,那么$H^0(U)$就是在在各个连通分支上为常数(整体不一定是一样的)的那些函数构成的矢量空间。而他的维度就是连通分支的个数。

下一节将会提到一点,上同调群是流形拓扑的表征,他是一个拓扑范畴到交换群范畴的反变函子,对于同伦等价的拓扑空间,有着同构的上同调群。所以这是一个拓扑不变量,如果我们可以把两个拓扑空间的上同调群计算出来得到他们不同构,则这两个拓扑空间必然不同伦。流形的上同调群也被称为de Rham上同调,他是上同调群的一个典型,这里$\dd$是自然的边缘算子的对偶,而楔积是自然的cup product(比起同调,上同调里的cup product是比“对偶”部分多出来的东西)。

\para 正如\eqref{eq:1.3}我们看到的,对$1$-形式$\omega$,我们有
\[
	\dd(\omega)(X,Y)=X(\omega(Y))-Y(\omega(X))-\omega([X,Y]),
\]
类似地,对$k$-形式$\omega$,经过一些不算复杂的计算,我们可以证明
\[
\begin{split}
	\dd(\omega)(X_1,\cdots,X_{k+1})=&\sum_{i=1}^{k+1}(-1)^{i+1}X_i\bigl(\omega(X_1,\cdots,\hat{X_i},\cdots,X_{k+1})\bigr)\\
	&+\sum_{1\leq i<j\leq k+1}(-1)^{i+j}\omega\bigl([X_i,X_j],X_1,\cdots,\hat{X_i},\cdots,\hat{X_j},\cdots,X_{k+1}\bigr),
\end{split}
\]
其中
\[
	\omega\bigl(X_1,\cdots,\hat{X_i},\cdots,X_{k+1}\bigr)=\omega\bigl(X_1,\cdots,X_{i-1},X_{i+1},\cdots,X_{k+1}\bigr),
\]
以及
\[
\begin{split}
	\omega\bigl([X_i,X_j],X_1,\cdots,\hat{X_i},\cdots,&\hat{X_j},\cdots,X_{k+1}\bigr)=\\
	&\omega\bigl([X_i,X_j],X_1,\cdots,X_{i-1},X_{i+1},\cdots,X_{j-1},X_{j+1},\cdots,X_{k+1}\bigr),
\end{split}
\]
就是说,头上带帽等于把他去掉。这个式子可以反过来拿来定义$\dd$,而且确实是一个坐标无关的定义。我们并没有那么做,因为就实际计算而言,更多时候我们并不会把形式作用在所有的光滑切矢量场上。

\para 设$f:M\to N$,我们前面已经谈到了,这个$f$在0-形式,即光滑函数之间诱导了一个$f^*$(被称为一个拉回),即$f^*g=g\circ f$. 对于$1$-形式,由\pararef{f*d=df*},我们知道了$f^*\dd g=\dd (f^* g)$,对比矢量场之间$f_*$的定义还需要$f$是单射,这里1-形式之间的$f^*$则没有这个限制。

\para 通过$(f^*\omega)_p=\omega_{f(p)}\bigl(f_{*p}(X_1)_p,\cdots,f_{*p}(X_k)_p\bigr)$,我们可以定义出$k$-形式之间的$f^*$.

\para 所以,很容易检验,$f^*(\omega\wedge \eta)=(f^*\omega)\wedge (f^*\eta)$.

实际上,这个式子也可以反过来定义$k$-形式之间的$f^*$.可是这样定义可能会遇到一些技术性问题,比如一个$k$-形式大范围来说是否一定能写成一些比较低阶的形式楔积并线性组合而成?如果不,怎么通过上式定义?这个问题的回答是,类似\pararef{localpro},拉回是一个局部算符,即如果$\omega$在局部为零,则$f^*\omega$也在局部为零,所以我们可以不从大范围考虑这个问题。

\para 按照上面这种定义方式,假设$f^*$存在,则$f^*$是一个局部算符。

首先注意到$f^*(h\omega)=(f^*h)(f^*\omega)$。设$\omega$在$U$上为零,利用\pararef{regular},我们可以找一个$U$开覆盖。然后在每个开覆盖中的开集$W$上,同\pararef{localform}利用单位分解找一个光滑函数$h_W$,则按照上式$f^*(h_W\omega)$只可能在$\overline{W}$内不为零,因此,如果$\omega$在$U$上为零,则$h_W\omega$就恒为零,所以$(f^*h_W)(f^*\omega)=f^*(h_W\omega)=0$. 由于$f^*h_W=h_W\circ f$在$f^{-1}(W)$内等于$1$,所以$(f^*\omega)|_{f^{-1}(W)}=0$,遍历开覆盖,我们就得到了$(f^*\omega)|_{f^{-1}(U)}=0$.

特别地,如果$\omega_1$和$\omega_2$在$U$上相同,则$0=(f^*(\omega_1-\omega_2))|_{f^{-1}(U)}=(f^*\omega_1)|_{f^{-1}(U)}-(f^*\omega_2)|_{f^{-1}(U)}$,这就意味着$f^*\omega_1$和$f^*\omega_2$在局部相同,此即黏合条件。

我们这里不再重复类似$\dd_U$诱导出$\dd_V$的过程,以及这些$f^*_U$什么的和限制映射之间的关系,统统直接记做$f^*$,则$f^*(\omega_{U})=(f^* \omega)_{f^{-1}(U)}$。因为局部来说$k$-形式确实可以写成一些比较低阶的形式楔积并线性组合而成,以及我们已经对$0$-形式和$1$-形式定义了拉回,则局部的$k$-形式的拉回可以使用我们的定义递归地定义出来,那么剩下的只要拼起来就好,而这正需要黏合条件。

最后,这样定义的$f^*$在一点诱导出的$f^*_p$(类似$\dd$到$\dd_p$)可以验证和$1$-形式间已经存在的$f^*_p$用张量积诱导出来的映射是相同的。

\para 设$f:M\to N$,则$f^*(\dd \omega)=\dd (f^*\omega)$.实际上,我们只要局部对单项式证明即可,设$\omega=a\dd x^{1}\wedge\cdots\wedge \dd x^{k}$,则$\dd \omega=\dd a\wedge \dd x^{1}\wedge\cdots\wedge \dd x^{k}$,以及
\[
	f^*(\dd \omega)=(f^*\dd a)\wedge f^*(\dd x^{1}\wedge\cdots\wedge \dd x^{k})=\dd (f^*a)\wedge \dd f^*(x^{1})\wedge\cdots\wedge \dd f^*(x^{k})=\dd (f^*\omega).
\]

因为$f^*$将恰当形式映射成恰当形式,即$f^*(\im d_N)\subset \im d_M$,所以他诱导了两个流形上同调群之间的同态$f^*:H^k(N)\to H^k(M)$.

\section{Introduction to de Rham Cohomology}

我们这节将目标放在$\rr^n$中的de Rham上同调。

\para 一系列矢量空间和上面的线性映射$A\xrightarrow{f}B\xrightarrow{g}C$称为正和列,就是说$\ker g=\mathrm{Im} f$.而$0\to A\xrightarrow{f}B$
就是说$f$是单射,而$B\xrightarrow{g}C\to 0$就是说$g$是满射。而正和列
\[
0\to A\xrightarrow{f}B\xrightarrow{g}C\to 0
\]
称为短正合列。

\para 一个矢量空间和线性映射链$A^*=\{A_i,\dd_i\}$
\[
\cdots\to A^{i-1}\xrightarrow{\dd^{i-1}}A^i\xrightarrow{\dd^i}A^{i+1}\to \cdots
\]
称为链复形,如果对于任意的$i$都有$\dd^i \circ \dd^{i+1}=0$.当对任意的$i$都有$\ker \dd^i=\mathrm{Im}\, \dd^{i-1}$,则这个链复形称为正和的。

\para 很容易看到,$\Omega^*(M)$和外微分算子$\dd$构成一个链复形。那么同样,对于任意的链复形都可以定义上同调群$H^p(A^*)=\ker (\dd^p)/\mathrm{Im} (\dd^{p-1})$,其中的元素同样用等价类符号$[a]$记。

\para 如果在两条链的每一个对应链的对象之间,譬如说$A_i$和$B_i$之间,存在线性映射$f^i$,那么自然就在两条链之间引入了一个映射$f:A^*\to B^*$,需要交换图如下:
	\[
	\xymatrix{
		\cdots\ar[r]&A^{p-1}\ar[r]^{\dd_A^{p-1}}\ar[d]^{f^{p-1}}&A^p\ar[r]^{\dd_A^p}\ar[d]^{f^{p}}&A^{p+1}\ar[r]\ar[d]^{f^{p+1}}&\cdots\\
		\cdots\ar[r]&B^{p-1}\ar[r]^{\dd_B^{p-1}}&B^p\ar[r]^{\dd_B^p}&B^{p+1}\ar[r]&\cdots
	}
	\]
从交换图可以看到,应该满足$\dd^{p}_B\circ f^p=f^{p+1}\circ \dd^{p}_A$.既然在链复形之间引入了映射,则他诱导了上同调群之间的映射。通过$f^*([a])=[f^p(a)]$,我们诱导了$f^*=H^p(f):H^p(A^*)\to H^p(B^*)$.

\para 链复形也可以构成一个链,尤其重要的是短正合列$0\to A^*\xrightarrow{f}B^*\xrightarrow{g}C^*\to 0$.链的短正和列就是当对任意的$p$都有短正合列$0\to A^p\xrightarrow{f^p}B^p\xrightarrow{g^p}C^p\to 0$.

\pro 链复形的短正合列$0\to A^*\xrightarrow{f}B^*\xrightarrow{g}C^*\to 0$引入了上同调群的正合列$H^p(A^*)\xrightarrow{f^*}H^p(B^*)\xrightarrow{g^*}H^p(C^*)$.

\proof 其实就是证明$\ker g^*=\mathrm{Im}\, f^*$.首先证明$\mathrm{Im}\, f^* \subset \ker g^*$,任取$[a]\in H^p(A^*)$,我们有
\[
g^*\circ f^*([a])=[g^p\circ f^p(a)]=[0]=0.
\]
这是从正合列$A^p\xrightarrow{f^p}B^p\xrightarrow{g^p}C^p$中得知的。

然后证明$\ker g^*\subset \mathrm{Im}\, f^*$.这就是说,任意的$g^*[b]=0$的$[b]$都可以找到$[a]$使得$f^*[a]=[b]$.

因为对任意的$p$有$0=g^*[b]=[g^p(b)]$,因此存在一个$c$使得$g^p(b)=\dd^{p-1}_C(c)$,而$g^{p-1}$又是满射,所以可以找到$b'$使得$g^{p-1}(b')=c$,因此用交换图变换
\[
g^p(\dd^{p-1}_B(b'))=d^{p-1}_C(g^{p-1}(b'))=g^p(b).
\]

所以$g^p(b-\dd^{p-1}_B(b'))=0$,所以存在$a$使得$f^p(a)=b-\dd^{p-1}_B(b')$。现在只要证明这个$a$确实在$\ker \dd^p_A$里面就可以了。为此只要证明$\dd^p_A a=0$就可以,但是因为$f^{p+1}$是单射,所以也等价于证明$f^{p+1}\circ \dd^p_A (a)=0$.用交换图变换
\[
f^{p+1}\circ \dd^p_A (a)=\dd^p_B\circ f^p (a)=\dd^p_B(b-\dd^{p-1}_B(b'))=\dd^p_B(b)=0.
\]
因为$a$确实在$\ker \dd^p_A$里面,所以他在$H^p(A^*)$里面对应了一个等价类$[a]$,成立$f^*[a]=[b]$.\qed

\para 链复形的短正和列还引入了其他两个正合列。对于链复形的短正合列$
0\to A^*\xrightarrow{f}B^*\xrightarrow{g}C^*\to 0$,定义$\partial^*:H^p(C^*)\to H^{p+1}(A^*)$为线性映射
\[
	\partial^*([c])=\left[(f^{p+1})^{-1}\left(\dd^p_B\left((g^p)^{-1}(c)\right)\right)\right].
\]

$\partial^p$即交换图
	\[
		\xymatrix{
			&0\ar[d]&0\ar[d]&0\ar[d]&\\
			\cdots\ar[r]&A^{p-1}\ar[r]^{\dd_A^{p-1}}\ar[d]^{f^{p-1}}&A^p\ar[r]^{\dd_A^p}\ar[d]^{f^{p}}&A^{p+1}\ar[r]\ar[d]^{f^{p+1}}&\cdots\\
			\cdots\ar[r]&B^{p-1}\ar[r]^{\dd_B^{p-1}}\ar[d]^{g^{p-1}}&B^p\ar[r]^{\dd_B^p}\ar[d]^{g^{p}}&B^{p+1}\ar[r]\ar[d]^{g^{p+1}}&\cdots\\
			\cdots\ar[r]&C^{p-1}\ar[r]^{\dd_B^{p-1}}\ar[d]&C^p\ar[r]^{\dd_B^p}\ar[d]\ar[ruu]&C^{p+1}\ar[r]\ar[d]&\cdots\\
			&0&0&0&
		}
	\]
中的斜线。这里就不证明这是良定义的了。因此,链复形的短正合列$0\to A^*\xrightarrow{f}B^*\xrightarrow{g}C^*\to 0$诱导了上同调群的正合列
\[
\begin{split}
&H^p(B^*)\xrightarrow{g^*}H^p(C^*)\xrightarrow{\partial^*}H^{p+1}(A^*),\\
&H^p(C^*)\xrightarrow{\partial^*}H^{p+1}(A^*)\xrightarrow{f^*}H^{p+1}(B^*).
\end{split}
\]

\theo \label{longexact}
链复形的短正合列$0\to A^*\xrightarrow{f}B^*\xrightarrow{g}C^*\to 0$引入了上同调群的正合列
\[
\cdots\to H^p(A^*)\xrightarrow{f^*}H^p(B^*)\xrightarrow{g^*}H^p(C^*)\xrightarrow{\partial^*}H^{p+1}(A^*)\xrightarrow{f^*}H^{p+1}(B^*)\to\cdots.
\]

\pro \label{directsum} 链复形可以谈论直和,即是对链中每一个矢量空间进行直和。那么从$\ker$和$\mathrm{Im}$对于直和的显然性质,我们有$H^p(A^*\oplus B^*)=H^p(A^*)\oplus H^p(B^*)$.

$\Omega^*(U)$和外微分算子$\dd$构成一个链复形,下面的定理给出了有关于欧氏空间两个开集和他们的并与交的短正合列。

\theo 设$U_1$和$U_2$是$\rr^n$中的开集,记$i_\nu:U_\nu \to U_1 \cup U_2$和$j_\nu:U_1\cap U_2 \to U_\nu$是嵌入,则有如下的短正合列:
\[
0\to \Omega^p(U_1\cup U_2)\xrightarrow{I^p}\Omega^p(U_1)\oplus\Omega^p(U_2)\xrightarrow{J^p}\Omega^p(U_1\cap U_2)\to 0.
\]
其中$I^p(\omega)=(i_1^*(\omega),i_2^*(\omega))$, $J^p(\omega_1,\omega_2)=j_1^*(\omega_1)-j_2^*(\omega_2)$.

\proof 先证明$I^*$是单射,这就是说除了$I^*(\omega)=0$只有解$\omega=0$.

设$\varphi$是$\rr^n$中的开集的嵌入,则对于任意的$p$-形式$\dd x_I=\dd x_{i_1}\wedge\dots\wedge\dd x_{i_p}$都有$\varphi^* \dd x_I =\dd x_I$,因此
\[
\varphi^*\omega=\varphi^*\sum_If_I\dd x_I=\sum_If_I\circ\varphi \dd x_I.
\]

现在证明$J^*$是一个满射。将单位分解应用到$U_1$和$U_2$上面取,存在$p_\nu$为定义在$U_1\cup U_2$上的光滑函数,而他的非零点集包含于$U_\nu$,且$p_1(x)+p_2(x)=1$.

设$f$定义在$U_1\cap U_2$上。定义$U_1$上的光滑函数$f_1$,他在$U_1\cap U_2$上的限制为$f(x)p_2(x)$和$U_2$上的光滑函数$f_2(x)$,他在$U_1\cap U_2$上的限制为$-f(x)p_1(x)$.那么在$U_1\cap U_2$上$f_1(x)-f_2(x)=f(x)$.

所以任选一个$U_1\cap U_2$上的$p$-形式$\omega$,系数$f_I$都可以定义出$f_{1,I}$和$f_{2,I}$,并且在$U_1\cap U_2$上满足$f_{1,I}-f_{2,I}=f$,因此也定义了两个$\omega_1$和$\omega_2$得到$J^p(\omega_1,\omega_2)=\omega$.

那么$I^*(\omega)=0$就是说$i_1^*(\omega)=i_2^*(\omega)=0$,这就是说$f_I\circ i_1=f_I\circ i_2=0$,但是由于$U_1$和$U_2$是$U_1\cup U_2$的一个开覆盖,所以这就等价于$f_I=0$,所以$\omega=0$.

然后证明$\ker J^p=\mathrm{Im}\, I^p$.分两个包含。

\no{1} $\mathrm{Im}\, I^p\subset \ker J^p$
\[
J^p\circ I^p(\omega)=j_2^*\circ i_2^*(\omega)-j_1^*\circ i_1^*(\omega)
\]
但其实$i_2\circ j_2=i_1\circ j_1$,所以$J^p\circ I^p(\omega)=0$.

\no{2} $\ker J^p\subset \mathrm{Im}\, I^p$

设$\omega_1=\sum_I f_I \dd x_I\in \Omega^p(U_1)$和$\omega_2=\sum_I g_I \dd x_I\in \Omega^p(U_2)$,从$J^p(\omega_1,\omega_2)=0$我们有$j_1^*(\omega_1)=j_2^*(\omega_2)$,这就是说$f\circ j_1=g\circ j_2$,或者说$f$和$g$在$U_1\cap U_2$恒等。我们可以构成一个光滑函数$h_I$,他在$U_1$上的限制恒等于$f_I$,而$U_2$上恒等于$g_I$.那么
\[
I^p\left(\sum_Ih_I\dd x_I\right)=(\omega_1,\omega_2).
\]\qed

将Theorem \pararef{longexact} 和 Proposition \pararef{directsum} 应用到上面这个定理。就得到下面这个定理。

\theo Mayer-Vietoris列:
设$U_1$和$U_2$是$\rr^n$中的开集,则有如下的正合列:
\[
\cdots\to H^p(U_1\cup U_2)\xrightarrow{I^*}H^p(U_1)\oplus H^p(U_2)\xrightarrow{J^*}H^p(U_1\cap U_2)\xrightarrow{\partial^*}H^{p+1}(U_1\cup U_2)
\to \cdots
\]
其中$I^*([\omega])=(i_1^*([\omega]),i_2^*([\omega]))$,$J^*([\omega_1],[\omega_2])=j_1^*([\omega_1])-j_2^*([\omega_2])$.

如果$U_1\cap U_2=\varnothing$,那么$H^p(U_1\cap U_2)=0$.所以
\[
0\xrightarrow{\partial^*} H^p(U_1\cup U_2)\xrightarrow{I^*}H^p(U_1)\oplus H^p(U_2)\xrightarrow{J^*}0,
\]
那么$I^*$既单又满,故而是个同构。

如果我们已知$U_1$和$U_2$的上同调群,那么通过Mayer-Vietoris列我们就有可能计算他们的并或者交的上同调群。

\para 两个连续函数$f$, $g:X\to Y$被称为同伦的,就是说存在一个连续函数$H:X\times [0,1]\to Y$使得$H(0,x)=f(x)$以及$H(1,x)=g(x)$.同伦是等价关系,就是说,如果还有$h$和$f$同伦,则$g$和$h$也同伦。一个空间$X$是可缩的,如果$\id_X$同伦于映到自身的常值映射。比如$\rr^n$或者与他同胚的开实心球$D^n$都是可缩的。

两个集合$X$和$Y$称为同伦等价的,如果存在$f:X\to Y$和$g:Y\to X$满足$f\circ g$和$\id_Y$同伦以及$g\circ f$和$\id_X$同伦。下面我们尝试说明,上同调群是同伦不变量。即在同伦意义下,上同调群是同构的。

\lem 光滑函数的同伦的一个技术性引理(见单位分解的附录):在欧氏空间背景下,任何一个连续映射都同伦于一个光滑映射。如果两个光滑函数$f_1$, $f_2:U\to V$是同伦的,则存在光滑函数$F:U\times \rr\to V$满足$F(x,0)=f_1(x)$和$F(x,1)=f_2(x)$.

\pro 两条链复形和两个映射$f$, $g:A^*\to B^*$如果对每一个$p$都存在线性映射$s^p:A^p \to B^{p-1}$满足
\[
\dd_B^{p-1}s^p+s^{p+1}\dd_A^p=f^p-g^p:A^p\to B^p.
\]
则$f^*=g^*$.这样的两个映射被称为链同伦的。

\proof 对任意的$[a]\in H^p(A^*)$,我们有$\dd_A^p(a)=0$,所以
\[
(f^*-g^*)[a]=[(f^p-g^p)a]=[\dd^{p-1}_Bs^p(a)+s^{p+1}\dd_A^p(a)]=[\dd^{p-1}_Bs^p(a)],
\]
而$\dd^{p-1}_Bs(a)$显然被等价为0,所以$f^*=g^*$.\qed

\pro 如果两个光滑函数$f$, $g:U\to V$是同伦的,则$
f^*$, $g^*:\Omega^*(V)\to \Omega^*(U)$是链同伦的,即$f^*=g^*$。

\proof 利用我们的技术性引理,由于$f$, $g:U\to V$是同伦的,所以存在光滑函数$F:U\times \rr \to V$使得$F(x,0)=f(x)$和$F(x,1)=g(x)$.

注意到任意的$U\times \rr$上的$p$-形式可以写作
\[
\omega=\sum_If_I(x,t)\dd x_I+\sum_J g_J(x,t)\dd t\wedge \dd x_J.
\]
让$\varphi_0:x\mapsto (x,0)$和$\varphi_1:x\mapsto (x,1)$,则$F\circ \varphi_0=f$和$F\circ \varphi_1=g$,且将上面的形式分别拉回到
\[
\varphi_0^*\omega=\sum_I f_I(x,0)\dd x_I,\quad
\varphi_1^*\omega=\sum_I f_I(x,1)\dd x_I.
\]

现在我们需要构造一个$S^p:\Omega^p(U\times \rr)\to \Omega^{p-1}(U)$使得
\[
(\dd \circ S^p+S^{p+1}\circ \dd)(\omega)=(\varphi_1^*-\varphi_0^*)(\omega),
\]
如果这样,对于任意$U$中的形式我们有
\[
(\dd \circ S^p+S^{p+1}\circ \dd)(F^*\omega)=(\varphi_1^*-\varphi_0^*)(F^*\omega)=((F\circ \varphi_1)^*-(F\circ \varphi_0)^*)(\omega)=(g^*-f^*)(\omega),
\]
而最左边又有
\[
(\dd \circ S^p\circ F^*+S^{p+1}\circ F^* \circ \dd)(\omega)
\]
所以只要定义$s^p=S^p\circ F^*$,这就是链同伦。

为此定义
\[
S^p(\omega)=\sum_J\left(\int_0^1g_J(x,t)\dd t\right)\dd x_J.
\]\qed

利用这个命题,我们可以知道,同伦等价对应到上同调群就有了上同调群的同构,因此上同调群只依赖于同伦型。

\para 由于对于可缩开子集$0^*=\id^*$,所以如果$U\subset \rr^n$可缩,那么$U$上的闭形式是恰当形式。这被称为Poincaré引理,通过他,我们知道可缩开集$U$的上同调群如下,$H^0(U)=\rr$,而对于$p>0$,则为$H^p(U)=0$.

\para 用Mayer-Vietoris列计算$H^{p}(\rr^2-\{0\})$.

设$U_1$为去掉正实轴(包括原点)的平面,$U_2$为去掉负实轴(包括原点)的平面,因此$U_1\cup U_2=\rr^2-\{0\}$.两者都是可缩的,所以由Poincaré引理可以知道,$H^0(U_1)=H^0(U_2)=\rr$以及如果$p>0$有$H^p(U_1)=H^p(U_2)=0$.

注意到$U_1\cap U_2$为去掉实轴的平面$I_1\cup I_2$,两个部分无交且各自可缩,所以
\[
H^p(U_1\cap U_2)=H^p(I_1\cup I_2) \cong H^p(I_1)\oplus H^p(I_2) =\begin{cases}
\rr\oplus\rr&,p=0;\\
0&,p>0.
\end{cases}
\]
而$\rr^2-\{0\}$是连通的,所以$H^0(\rr^2-\{0\})=\rr$.

当$p>0$的时候,代入Mayer-Vietoris列
\[
H^p(U_1\cap U_2)\xrightarrow{\partial^*}H^{p+1}(U_1\cup U_2)\xrightarrow{I^*}H^{p+1}(U_1)\oplus H^{p+1}(U_2)\xrightarrow{J^*}H^{p+1}(U_1\cap U_2),
\]
头尾通过计算都为$0$,所以
\[
H^{p+1}(\rr^2-\{0\})=H^{p+1}(U_1\cup U_2)\cong H^{p+1}(U_1)\oplus H^{p+1}(U_2)=0.
\]
这就是说,平面挖一个洞的$2$阶以上的上同调群为$0$.

现在考察一阶$\rr^2-\{0\}$的上同调群,由于负阶都为$0$,所以
 \[
0\to H^0(U_1\cup U_2)\xrightarrow{I^*}H^0(U_1)\oplus H^0(U_2)\xrightarrow{J^*}H^0(U_1\cap U_2)\xrightarrow{\partial^*}H^{1}(U_1\cup U_2)\to 0
\]
或者
 \[
0\to \rr\xrightarrow{f}\rr\oplus\rr\xrightarrow{g}\rr\oplus\rr\xrightarrow{h}H^{1}(\rr^2-\{0\})\to 0.
\]
由于$f$是单射而且正和性给出$\ker g = \mathrm{Im}\,f$,所以$\ker g =\rr$,而由线性代数基本定理,有$\mathrm{Im}\,g \cong \rr$,因此正合列给出$\ker h \cong \rr$,而因为$h$是满射,所以根据同构基本定理$H^{1}(\rr^2-\{0\})\cong(\rr\oplus\rr)/\ker h \cong \rr$.

综上,\[H^{p}(\rr^2-\{0\})\cong
\begin{cases}
\rr&,p=0,1;\\
0&,p>1.
\end{cases}\]

\para  同样是Mayer-Vietoris列的算例。将$\rr^n$看做$\rr^{n+1}$的子空间,设$A$是$\rr^n$中的闭子集,则
\[
\begin{cases}
H^{p+1}(\rr^{n+1}-A)\cong H^p(\rr^n-A),&\text{when}\,p>1,\\
H^{1}(\rr^{n+1}-A)\cong H^0(\rr^n-A)/\rr,&\\
H^{0}(\rr^{n+1}-A)\cong \rr.&
\end{cases}
\]

通过归纳法可以得到
\[H^{p}(\rr^n-\{0\})\cong
\begin{cases}
\rr&,p=0,n-1;\\
0&,\text{otherwise}.
\end{cases}\]
这个结论可以用来证明$\rr^m$和$\rr^n$之间不存在同胚,如果存在,将同胚调整为$0$映射到$0$,而由于同胚将产生上同调群间的同构,所以$H^{p}(\rr^n-{0})$与$H^{p}(\rr^m-{0})$对于任意$p$都是同构的,但这不可能。

由于当$n>0$的时候,$S^{n-1}$和$\rr^n-\{0\}$同胚(当然也自然同伦等价),所以我们也计算出了球面的上同调群为
\[H^{p}(S^{n})\cong
\begin{cases}
\rr&,p=0,n;\\
0&,\text{otherwise}.
\end{cases}\]

上面一些计算出的上同调群可以产生许多有名经典的拓扑结论,比如Jordan-Brouwer分割定理,Brouwer不动点定理等。最简单的,比如$S^n$和$S^m$在$n\neq m$时候不同胚。

\section{Integration}

\para 对于$p\geq 0$,以及$\rr^{p+1}$中的$p+1$个矢量$\{v_i:0\leq i\leq p\}$满足$\{v_i-v_0:1\leq i\leq p\}$是一个线性无关组,我们定义$p$-单形为
\[
	[v_0,\cdots,v_p]=\left\{\sum_{i=0}^p a_iv_i\in \rr^{p+1}:\sum_{i=1}^p a_i=1,\text{and each } a_i\leq 0\right\}.
\]
一个标准$p$-单形为$[e_0,\cdots,e_p]$,其中$\{e_i\}_{0\leq i \leq p}$是$\rr^{p+1}$的标准基,可以看到,每一个$p$单形都和标准$p$-单形同胚。但是,流形$M$上的一个(光滑)$p$-单形是指一个(光滑)映射$\sigma:[v_0,\cdots,v_p]\to M$,这是光滑曲线的自然推广,因为流形上的光滑曲线就是一个1-单形。我们常说沿着某条曲线积分,这里我们就将推广到沿着某个单形积分。

一个0-单形是一个点$1$,一个1-单形$[v_0,v_1]$是一个线段,端点为$v_0$和$v_1$,一个2-单形是一个三角形,他的三个顶点位于$v_0$, $v_1$和$v_2$,一个3-单形位于四维空间,不太好想,但是从上面的类比,他应该是一个四面体,顶点位于$v_0$, $v_1$, $v_2$和$v_3$.记$M$上全部$p$-单形生成的自由Abel群(或者称$\zz$-模)为$C_p(M)$,当然,这里两个单形之间的加法的定义意义似乎还不太明朗。我们将$C_p(M)$中的元素称为$p$-链,显然,一个$p$-单形就是一个$p$-链。

\para 一个$p$-单形内的每一个点都可以写作$\sum_{i=0}^p a_iv_i$的形式,我们称某个$(p-1)$-边界是指某个$a_i=0$的情况,一个$(p-1)$边界是自然的$(p-1)$-单形,记做$[v_0,\cdots,\hat{v}_i,\cdots,v_p]$。显然,一个$p$-单形有$p+1$个边界,比如一个三角形,是一个2-单形,有三条边。

对于流形$M$上的一个$p$-单形$\sigma:[v_0,\cdots,v_p]\to M$,定义他的第$(i+1)$个边缘为一个$(p-1)$-链$\sigma|_{[v_0,\cdots,\hat{v}_i,\cdots,v_p]}$。

\para 在流形$M$上,定义边缘算子$\partial:C_p(M)\to C_{p+1}(M)$:对单项式,
\[
	\partial \sigma=\sum_{i=0}^p (-1)^i\sigma|_{[v_0,\cdots,\hat{v}_i,\cdots,v_p]}.
\]
对一般的$\sigma=\sum_i n_i \sigma_i$,定义$\partial \sigma =\sum_i n_i \partial \sigma_i$.

\para 我们有$\partial^2=0$.

\proof 直接对单项式计算
\[
\begin{split}
	\partial^2 \sigma&=\sum_{i=0}^p (-1)^i\partial\sigma|_{[v_0,\cdots,\hat{v}_i,\cdots,v_p]}\\
	&=\sum_{0\leq i<j\leq n}^p (-1)^{i+j-1}\bigl(\sigma|_{[v_0,\cdots,\hat{v}_i,\cdots,\hat{v}_j,\cdots,v_p]}-\sigma|_{[v_0,\cdots,\hat{v}_i,\cdots,\hat{v}_j,\cdots,v_p]}\bigr)=0.
\end{split}
\]\qed

$\partial^2=0$和$\dd^2=0$是如此的相似!实际上,我们可以定义$H_k(M)=\ker\partial/\im\partial$,这就是流形$M$的第$k$个奇异同调群。这里不展开,我们暂时的目标还是积分。

\para 由于每一个$[v_0,\cdots,v_p]$都同胚与标准$p$-单形,所以,在同胚意义下,以标准$p$-单形为定义域的流形上的$p$-单形与定义域为花式的$[v_0,\cdots,v_p]$的流形上的$p$-单形应该是一样多的。所以我们可以限定,流形上的$p$-单形的定义域为标准$p$-单形。

\para 除去一个平移,$[v_0,\cdots,v_p]$完全处于$\rr^p$里面。

由于$[v_0,\cdots,v_p]$里面的点写作$\sum_i a_i v_i$,注意到$\sum_i a_i=1$,我们可以将其改写为
\[
	\sum_{i=0}^pa_iv_i=\left(1-\sum_{i=1}^pa_i\right)v_0+\sum_{i=1}^pa_iv_i=v_0+\sum_{i=1}^pa_i(v_i-v_0),
\]
所以,除去一个平移$v_0$,我们完全可以由$\sum_{i=1}^pa_i(v_i-v_0)$确定一个$p$-单形。显然,他们是光滑同胚的。所以对于$M$上的$p$-单形,我们在微分同胚意义下,还可以假设定义域为
\[
	\Delta^p=\left\{(a_1,\cdots,a_p):a_i\geq 0\text{ and }\sum_i a_i\leq 1\right\},
\]
这时候,$\sigma:\Delta^p\to M$常常会写作$\sigma(a_1,\cdots,a_p)$.

此时,对于边缘算子,这种约定下,符号要变得复杂一些,记
\[
	k^p_0(a_1,\cdots,a_{p})=\left(1-\sum_{i=1}^{p-1}a_i,\cdots,a_{p}\right),
\]
当$i>0$时,
\[
	k^p_i(a_1,\cdots,a_{p})=(a_1,\cdots,a_{i-1},0,a_{i+1},\cdots,a_{p}),
\]
以及$\sigma^i=\sigma\circ k^p_i$,则
\[
	\partial\sigma=\sum_{i=0}^p(-1)^i \sigma^i,
\]
不难检验,$\partial^2=0$还是成立的,以后我们就采用这种符号约定。

\para 设$\omega$是$\rr^n$中的$n$-形式,写作$f\dd x^1\wedge\cdots\wedge \dd x^n$,则我们定义他在区域$A$上的积分算符为线性函数
\[
	\int_A \omega=\int_A f \dd x^1\cdots\dd x^n,
\]
有时候为了省空间,我们以$\Int_A$来记$\int_A$. 并且,反过来,我们也将欧式空间里的$n$-重积分写作对微分形式积分的样子
\[
	\int_A f \dd x^1\cdots\dd x^n=\int_A f\dd x^1\wedge\cdots\wedge \dd x^n.
\]

有了这个定义,我们可以重写积分变量替换公式为
\[
	\int_{\varphi(A)}\omega=\pm\int_A \varphi^*\omega,
\]
其中当$\det(\varphi_*)>0$的时候取正,反之取负。

\para 设$\sigma$是一个$M$上的0-单形,即一个点$\sigma(0)\in M$,定义在$\sigma$上的积分为$\Int_\sigma \omega=\omega(\sigma(0))$.

\para 对于$p\leq 1$的情况,设$\sigma:\Delta^p\to M$是一个$M$上的一个$p$-单形,这里$p>0$,那么$\sigma$诱导了一个拉回映射$\sigma^*$,我们定义一个$p$-形式$\omega$在$\sigma$上的积分如下:
\[
	\int_\sigma \omega=\int_{\Delta^p} \sigma^*\omega,
\]
注意到等式右边就是一个$\rr^{p}$内对$p$-形式的积分,这个积分上面我们已经定义了。

当我们在一条链上积分的时候,比如$c=\sum_{i} n_i \sigma_i$,我们定义
\[
	\int_c \omega=\sum_i n_i \int_{\sigma_i} \omega.
\]
此时,单形之间加起来成链在积分的意义下清楚了不少。

\theo 微积分基本定理:设$c$是$M$上的一个1-链,而$f$是$M$上的一个光滑函数,则
\[
	\int_{\partial c} f=\int_c \dd f.
\]
对于$c$是一个1-单形的时候,这个定理翻译成我们熟知的语言即$\int_a^b \dd f=f(b)-f(a)$,而两边对于$c$都是线性的,所以上述定理对链依然成立。

\theo 著名的Stocks定理(还能叫微积分基本定理):设$c$是$M$上的一个$p$-链,而$\omega$是$M$上的一个$(p-1)$-单形,则
\[
	\int_c \dd \omega=\int_{\partial c} \omega.
\]

这是两个(一个)极伟大的定理,尽管证明他并不是特别困难。他的重要性无需强调,每一个做过定积分(包括曲面积分等)计算的人都知道他多么有用。

\proof 对于$p=1$的情况,这就是微积分基本定理,所以我们下面假设$p\leq 2$.由于积分对链是线性的,所以我们只要对一个单形去证明就可以了,即
\[
	\int_{\sigma} \dd \omega=\int_{\partial \sigma} \omega,
\]
利用积分的定义,我们把单形拉回到欧式空间里面,证明几乎就可以在欧式空间里面进行,即
\[
	\int_{\Delta^p} \sigma^*(\dd \omega)=\sum_{i=0}^p(-1)^i\int_{\Delta^{p-1}} (\sigma^i)^*\omega=\sum_{i=0}^p(-1)^i\int_{\Delta^{p-1}} (k^p_i)^*\circ \sigma^*(\omega).
\]

我们假设
\[
	\sigma^*\omega=\sum_{i=1}^pa_i\dd x^1\wedge\cdots\wedge\widehat{\dd x^i}\wedge\cdots \dd x^p,
\]
所以等式左边写作
\[
\begin{split}
	\int_{\Delta^p} \sigma^*(\dd \omega)=\int_{\Delta^p} \dd(\sigma^* \omega)&=\sum_{i=1}^p\int_{\Delta^{p}} \dd a_i\wedge \dd x^1\wedge\cdots\wedge\widehat{\dd x^i}\wedge\cdots\wedge\dd x^p\\
	&=\sum_{i=1}^p\int_{\Delta^{p}} (-1)^{i-1}\partial_i a_i\dd x^1\wedge\cdots \wedge\dd x^p.
\end{split}
\]

现在来看右边,由于
\[(k^p_0)^*(x^j)=
\begin{cases}
	1-\sum_{i=1}^{p-1}x^i,& j =1;\\
	x^{j-1},& j>1,
\end{cases}\quad
(k^p_i)^*(x^j)=
\begin{cases}
	x^j,& 1\leq j \leq i-1;\\
	0,& j=i;\\
	x^{j-1},& i+1\leq j \leq p.
\end{cases}
\]
所以
\[
\begin{split}
	\sum_{i=0}^p(-1)^i\int_{\Delta^{p-1}}& (k^p_i)^*(a_j\dd x^1\wedge\cdots\wedge\widehat{\dd x^j}\wedge\cdots \dd x^p)\\
	&=(-1)^{j-1}\int_{\Delta^{p-1}} (a_j\circ k^p_0)\dd x^1\wedge\cdots \wedge\dd x^{p-1}+(-1)^j\int_{\Delta^{p-1}} (a_j\circ k_j^p)\dd x^1\wedge\cdots \wedge\dd x^{p-1},
\end{split}
\]
对右侧第一项做一个适当的变量替换$\varphi^i$如下
\[
	\varphi^i(x)=
	\begin{cases}
	x,& i=1;\\
	(1-\sum_{i=1}^{p-1}x^i,x^2,\cdots,x^{p-1}),& i=2;\\
	(x^2,\cdots,x^{i-1},1-\sum_{i=1}^{p-1}x^i,x^{i+1},\cdots,x^{p-1}),& 2<i\leq p.
\end{cases}
\]
容易证明$\varphi^i(\Delta^{p-1})=\Delta^{p-1}$以及$\det ((\varphi^i)_*)=1$,所以,由重积分变量替换公式,我们得到
\[
\begin{split}
	\sum_{i=0}^p&(-1)^i\int_{\Delta^{p-1}} (k^p_i)^*(a_j\dd x^1\wedge\cdots\wedge\widehat{\dd x^j}\wedge\cdots \dd x^p)\\
	=&(-1)^{j-1}\int_{\Delta^{p-1}} a_j\left(x^1,\cdots,x^{j-1},1-\sum_{i=1}^{p-1}x^i,x^{j},\cdots,x^p\right)\dd x^1\wedge\cdots \wedge\dd x^{p-1}+\\
	&(-1)^{j}\int_{\Delta^{p-1}}a_j(x^1,\cdots,x^{j-1},0,x^{j},\cdots,x^p)\dd x^1\wedge\cdots \wedge\dd x^{p-1},
\end{split}
\]
所以最后我们只需要证明
\[
\begin{split}
	\int_{\Delta^{p}}\partial_i a_i\dd x^1\wedge\cdots \wedge\dd x^p=
	&\int_{\Delta^{p-1}} a_j\left(x^1,\cdots,x^{j-1},1-\sum_{i=1}^{p-1}x^i,x^{j},\cdots,x^p\right)\dd x^1\wedge\cdots \wedge\dd x^{p-1}\\
	&-\int_{\Delta^{p-1}}a_j(x^1,\cdots,x^{j-1},0,x^{j},\cdots,x^p)\dd x^1\wedge\cdots \wedge\dd x^{p-1},
\end{split}
\]
注意这其实是重积分,利用一元的微积分基本定理即得。\qed

\para 回到积分曲面的问题,我们谈过,积分曲面的存在性就在于曲边四边形的可拼合与否。这里,利用沿着曲线(1-单形)的积分,我们再来演示一次$\dd\omega$决定了他是否可积。

我们和以前一样考虑两条路径,他们分别沿着$X$方向走过$\epsilon$时间,再沿着$Y$方向走过$\epsilon$时间,或者次序反过来,他们的终点分布记做$P$和$Q$,那么取一条光滑曲线连接$PQ$,则我们就做成了一个回路。此时,取一个$\omega$使得$X$, $Y\in \ker(\omega)$,那么利用微积分基本定理,就可以把$\omega$沿着回路的积分化到曲面上关于$\dd \omega$的积分。

现在开始估计积分,从起点开始,每走$\epsilon$,高度就增加阶为$\epsilon^2$,那么如果有错开,他错开的垂直部分面积就应该至多是$\epsilon\cdot \epsilon^2=\epsilon^3$的阶,否则积分曲面就不存在。而$\dd \omega$关于回路围成的曲面的积分就是垂直错开部分的面积,即一个三阶小量。

利用微积分基本定理,我们就得到了,$\omega$沿着一条路径(先$X$和后$Y$或者反过来),与沿着另一条路径的积分,就应该至多相差一个三阶小量。因此,积分曲面存在当且仅当$\dd \omega|_{\omega=0}=0$.

%\chapter{Connection}
\newcommand{\gl}{\mathfrak{gl}}

先复习一个物理,作为伪Riemann几何的引入。并且,下面我们假设使用Einstein指标规则,即存在一上一下存在重复指标的时候就对这个指标求和,比如$x^iy_i=\sum_{i}x^iy_i$等。

在惯性参考系中,间隔写作$\dd s^2=c^2 \dd t^2-\dd x^2-\dd y^2-\dd z^2$,如果采用保持惯性参考系的变换,即Lorentz变换,那么间隔不变。可是,如果我们变到了非惯性系,那么间隔一般来说不再能够写作$\dd s^2=c^2 \dd t^2-\dd x^2-\dd y^2-\dd z^2$,比如有一个变换$A$,他不是仿射的,那么$x_2=A(x_1)$后(指标规则依然如同狭义相对论)
\[
	\dd x_2^\mu=\frac{\partial A^\mu}{\partial x_1^\nu}(x_1)\dd x_1^\nu
\]
所以
\[
	(\dd s_2)^2=\eta_{\mu\nu}\dd x_2^\mu\dd x_2^\nu=\eta_{\mu\nu}\frac{\partial A^\mu}{\partial x_1^\rho}(x_1)\frac{\partial A^\nu}{\partial x_1^\sigma}(x_1)\dd x_1^\rho\dd x_1^\sigma,
\]
右边并不能写成$\eta_{\mu\nu}\dd x_1^\mu\dd x_1^\nu$,而是$g_{\mu\nu}\dd x_1^\mu\dd x_1^\nu$,其中
\[
	g_{\rho\sigma}(x_1)=\eta_{\mu\nu}\frac{\partial A^\mu}{\partial x_1^\rho}(x_1)\frac{\partial A^\nu}{\partial x_1^\sigma}(x_1)
\]
不再是对角的,而且也不再是一个常矩阵。

Einstein认为,非惯性参考系中产生的动力学效应可以等价为一个力场的作用(当然,这个力场和真的力场还是有些许不同的),从上面可以看到,这个力场应当完全由时空的几何结构(度规)决定。此外,Einstein认为,真实存在的重力场也应该是时空的几何结构的改变,同样被度规决定。所以,时空的几何结构不再是时空的固有性质,不再是一切物理现象的背景,而本身参与到物理过程中,本身是一种物理对象。

这就是史上第一个非Abel场论,即广义相对论的一些基本想法。狭义相对论的基本背景是平直空间,比如$\rr^4$或者$\rr^{3+1}$,但现在我们必须扩展到流形上,为此伪Riemann几何是必须的。

\para 假设$M$是一个可微流形,且在他每一点的切空间$T_pM$都存在度规,即一个非退化\footnote{即如果$\langle a,a\rangle=0$,则$a=0$.}的对称二重线性映射$\langle\star,\star\rangle_p:T_pM\times T_pM\to \rr$.

由于二重线性映射可以看作一个张量$g_p:T_pM\times T_pM\to \rr$,因此度规可以看出一个张量场$g$。局部来看$g_p(u,v)=u^iv^jg(e_i,e_j)=u^iv^jg_{ij}$,最后的要求,$(g_{ij})$的负本征值的个数不变\footnote{根据惯性定理,负本征值的个数和选取的基无关。},即如果在一点为$I$个,其他点都是$I$个。这样的$(M,g)$就被称为一个伪Riemann流形。当$I=0$时候,就变成Riemann流形。

因为$(g_{ij})$对实对称矩阵,所以可以对角化成实对角矩阵,所以非退化条件等价于没有零本征值或者
$\det (g_{ij})\neq 0$。

前面我们已经在$\rr^4$上选取了度规$\eta$使得$\rr^4$变成了一个伪Riemann流形$\rr^{3+1}$,但是这样的说法还是有些模糊。之所以我们能说是在$\rr^4$上选取了度规,是因为$\rr^4$在每一点的切空间$T_p\rr^4$都同构于$\rr^4$,并且,在每一点定义的内积,都可以看作全局定义的在$\rr^4$上定义的度规。但是到了可微流形上,就必须强调这些区别。

记$g^{il}$是$g_{il}$的逆,更准确地说就是$g^{il}g_{lm}=\delta^i_m$.抬升和下降指标依然由度规$g$完成,即$x_i=g_{ij}x^j$和$x^i=g^{ij}x_j$,需要注意的是,现在求导和提升或者下降指标不再对易,因为$g$是可以改变的。

\section{Geodesic}
在伪Riemann流形上可以使用度规来定义可微曲线$\gamma:[a,b]\to M$的长度
\[
	L(\gamma)=\int_a^b \sqrt{|\langle\gamma'(\tau),\gamma'(\tau)\rangle|}\dd \tau=\int_a^b \sqrt{|g(\gamma'(\tau),\gamma'(\tau))|}\dd \tau.
\]
如果在物理上理解一条$\gamma$为一个粒子的世界线,则还要加上$\langle\gamma'(\tau),\gamma'(\tau)\rangle>0$的假设,此时$L(\gamma)$就是世界线的线长,其中参数$\tau$不一定有什么物理意义。类比在$\rr^{3+1}$里面解耦合的作用量的形式,可以定义作用量积分为
\[
	S(\gamma)=-\frac{mc}{2}\int_a^b \langle\gamma'(\tau),\gamma'(\tau)\rangle\dd \tau,
\]
而使得$S(\gamma)$取极值的路径被称为测地线。

为了求出测地线,将作用量积分在局部坐标下面写出来,用$x$来表示从$M$到$\rr^n$的局部平凡化,就是说$x(\tau)=x(\gamma(\tau))$是$\rr^n$中的坐标,此外用$\dot{f}$来表示$\dd f/\dd \tau$,则
\[
	S(\gamma)=-\frac{mc}{2}\int_a^b g_{ij}(x(\tau))\dot{x}^i(\tau)\dot{x}^j(\tau)\dd \tau,
\]
他的Lagrange方程直接计算后,发现可以写作
\[
	\ddot{x}^i(\tau)+\Gamma^i_{\phantom{i}jk}(x(\tau))\dot{x}^j(\tau)\dot{x}^k(\tau)=0,
\]
其中
\[
	\Gamma^i_{\phantom{i}jk}=\frac{1}{2}g^{il}(\partial_k g_{jl}+\partial_j g_{kl}-\partial_l g_{jk})
\]
被称为Christoffel记号,下面我们讨论联络的时候还会再遇到。如果有了上述方程的一个解$x(\tau)$,下面将要指出的是$\langle\dot{x}(\tau),\dot{x}(\tau)\rangle$是一个常数,就像我们在狭义相对论里面说的$u^\mu u_\mu=1$一样。这点直接计算
\[
	\frac{\dd}{\dd \tau}\langle\dot{x}(\tau),\dot{x}(\tau)\rangle=\frac{\dd}{\dd \tau}\left(g_{ij}(x(\tau))\dot{x}^i(\tau)\dot{x}^j(\tau)\right)=0
\]
就可以了,其中要用到上面的Lagrange方程。这个结论可以理解成测地线的切矢量的模长$\langle\gamma',\gamma'\rangle$是一个常数。

测地线局部的存在唯一性靠着二阶常微分方程的理论就可以得到,从局部到整个空间,我们可以一个一个坐标卡延拓过去,但是否可行这里不做讨论。但类似积分曲线,紧的流形(紧是有限性条件)上这总是可以做到的。

在参数化过程中,选取一个有意义的参数有利于我们观察问题,这里,如同狭义相对论里面一样,可以采用线元来参数化我们的曲线,这个时候
\[
	L(\gamma)=\int \sqrt{|\langle\gamma'(s),\gamma'(s)\rangle|}\dd s=\int_a^b \sqrt{|g(\gamma'(s),\gamma'(s))|}\dd s,
\]
将$u=\gamma'(s)$看成速度矢量,可以类比定义动量$p=mcu$.又注意到$\langle u,u\rangle$在选取$s$作为参数的时候满足(不怎么严格,但是用其他方法强行算一波还是有下面这个结论)
\[
	(\dd s)^2\langle u, u\rangle=g_{ij}\frac{\dd x^i}{\dd s}\frac{\dd x^j}{\dd s}(\dd s)^2=g_{ij}\dd x^i\dd x^j=(\dd s)^2,
\]
所以$\langle u, u\rangle=1$,因此
\[
	p^ip_i=g_{ij}p^i p^j=m^2c^2.
\]

反过来,假如粒子的运动轨迹是测地线,此时在线元参数化下$\langle u,u\rangle=u^iu_i=1$构成一个约束,我们可以检查作用量实际上可以写作
\[
	S=-mc \int \dd s
\]
的形式。

因为约束的存在,我们需要在$L_s=-mc$加上如下乘子,构造新的Lagrangian
\[
	L^*_s=L_s+\lambda(x,u,s) (u^i u_i-1),
\]
为使下面的讨论对一般的Lagrangian都成立,设$L_s=L_{0,s}+L_{1,s}=-mc+L_{1,s}$,此时的Lagrange方程写作
\[
\frac{\partial L^*_s}{\partial x^i}-\frac{\dd}{\dd s}\frac{\partial L^*_s}{\partial u^i}=\frac{\partial L_s}{\partial x^i}-\frac{\dd}{\dd s}\frac{\partial L_s}{\partial u^i}+\frac{\partial}{\partial x^i}[\lambda (u^i u_i-1)]-\frac{\dd}{\dd s}\frac{\partial}{\partial u^i}[\lambda(u^i u_i-1)]=0.
\]
代入约束和$L$的具体形式
\[
	\frac{\partial L_{1,s}}{\partial x^i}-\frac{\dd}{\dd s}\frac{\partial L_{1,s}}{\partial u^i}=(2(u_i)'-\partial_ig_{mn}u^mu^n)\lambda+2u_i \lambda',
\]
两边对左边不妨记作$l_i$,同时记$2k_i=2(u_i)'-\partial_ig_{mn}u^mu^n$,只要存在一个和$k_i$正交但不和$u_i$正交的矢量$v^i$,这样
\[
	l_iv^i=2v^iu_i \lambda',
\]
只要满足$l_iv^i=0$就自然有$\lambda'=0$.这是我们最感兴趣的情况,这时选取$\lambda = -mc/2$,则
\[
	L=-mc\dd s-\frac{mc}{2}(u^\mu u_\mu-1)\dd s+L_1=-\frac{mc}{2}\dd s-\frac{mc}{2}u^\mu u_\mu\dd s+L_1,
\]
或者等价地去掉一个恰当形式项,我们可以把
\begin{equation}
	L=-\frac{mc}{2}u^i u_i\dd s+L_1
	\label{freeparticle}
\end{equation}
看作$u^i$自由时候的Lagrangian形式。广义动量即
\[
	-P_i=\frac{\partial L_s}{\partial u^i}=-\frac{mc}{2}\frac{\partial (u^\nu u_\nu)}{\partial u^i}+\frac{\partial L_{1,s}}{\partial u^i}=-p_i+\frac{\partial L_{1,s}}{\partial u^i},
\]
在自由粒子的情况下,广义动量即$-p_i$,符合上面的定义,尤其是空间部分,可以看到就是$\bm{p}$,因此$\lambda = -mc/2$的选取是合理的。

所以在这种情况下,作用量的自由粒子部分可以看作
\[
	S=-\frac{mc}{2}\int_a^b g_{ij}u^i u^j \dd s.
\]
而整个运动方程(测地线方程)就写成
\[
	l_i=-mck_i=-\frac{mc}{2}\left(2(u_i)'-\partial_ig_{mn}u^mu^n\right).
\]

\section{Connection}

联络的出现,代数上实现了对矢量场的方向导数,几何上实现了所谓的平行移动。

\para 矢量丛上的联络的首先目的是为了对矢量场进行求导的。按照一般的思路,将矢量场$s$参数化,假设他在曲线$\gamma(t)$上,$\gamma(0)=x_0$,而$\gamma'(0)=Y$,则似乎他沿$Y$方向的导数应该定义成
\[
	\lim_{t\to 0}\frac{s(\gamma(t))-s(x_0)}{t},
\]
可是$s(\gamma(t))$和$s(x_0)$属于不同点的切空间,不能相减。在$\rr^n$很容易克服这个困难,只要将两个切空间的矢量的端点平移到一起,这样就可以相减了。但是到了流形上,就会发现没那么简单,最简单的问题,什么是平移?下面我们先抽象地定义联络,然后再回来说这个问题。

\para 令$E$是一个矢量丛,一个(线性)联络或者说一个协变导数是指一个映射
\[
	D:\Gamma (E)\to\Gamma(E)\otimes \Gamma(T^*M)
\]
满足下面几条规则:记$Ds(V)=D_Vs$,则$D_V:\Gamma (E)\to\Gamma(E)$,对于任意的$V,W\in T_p M$, $r,s\in \Gamma (E)$和$f,g\in C^\infty(M,\rr)$满足
\[
\begin{array}{lcl} 
	D_{fV+gW}s &=& fD_Vs+gD_W s,\\
	D_V(r+s)   &=& D_Vr+D_Vs,\\
	D_V(fs)    &=& (Vf)s+fD_Vs.
\end{array}
\]

$T\rr^n$的截面可以写作$s(x)=x^i(x)e_i$,其中$e_i$是$\rr^n$的标准基,此时
\[
	D_Vs=\dd x^i(V) e_i=(Vx^i)e_i,
\]
是一个联络。

任何一个截面局部都可以写成$s=s^i\mu_i$,其中矢量场$\mu_i$对应着矢量空间的一组基,而对于切矢量来说,局部可以写作$X=X^i\partial_i$,所以
\[
	D_Xs=X^iD_{\partial_i}(s^j\mu_j)=X^i\partial_i s^j\mu_j+X^is^jD_{\partial_i}\mu_j=X(s^j)\mu_j+X^is^jD_{\partial_i}\mu_j,
\]
由于$D_{\partial_i}\mu_j$依然是一个截面,所以他是$\mu_i$的线性组合
\[
	D_{\partial_i}\mu_j=\Gamma^k_{\phantom{k}ij}\mu_k,
\]
其中$\Gamma^k_{\phantom{k}ij}$被称为Christoffel符号,以后会看到,前面出现过的Christoffel符号只是这里的特例。此外,以后在局部用$D_i$来记$D_{\partial_i}$.综上,联络在局部可以写作
\[
	D_Xs=X(s^j)\mu_j+X^is^j\Gamma^{k}_{ij}\mu_k.
\]

有了联络的概念,我们可以谈什么叫做平行移动,假设现在我们有一条可微曲线$c(t)$,那么他的切矢量是$\dot{c}(c(t))$,或者在局部写作$\dot{c}^k(t)\partial_k$,对于任意的$M$上的截面$s$可以计算沿着这个曲线切线的导数
\begin{align*}
	D_{\dot{c}}s(c(t))&=\dot{c}^k(t)\partial_ks^j\mu_j(c(t))+(\dot{c}^i s^j\Gamma^k_{\phantom{k}ij}\mu_k)(c(t))\\
	&=\left(\dot{s}^j\mu_j+\dot{c}^i s^j\Gamma^k_{\phantom{k}ij}\mu_k\right)(c(t)).
\end{align*}

\para 如果$c(t)$是$M$上一条可微曲线,则称满足$D_{\dot{c}(t)}s(t)=0$的解$s(t)$是$s(0)$在$E$中沿着曲线$c$的平行移动。

由于$\dot{s}^j\mu_j+\dot{c}^i s^j\Gamma^{k}_{ij}\mu_k=0$是系数$\dot{s}^j$的一阶常微分方程组,所以解出系数后,就解出了$s$,即给定一个初值$s(0)$,有唯一解$s(t)$满足上述方程。

现在来检查联络是如何和平行移动联系起来的。对于$M$上的两个点,找一条曲线$c(t)$连接两点,且在一点处的切矢量为$Y$,假设有一组联络的基$\mu_i(t)$他们沿着$c(t)$平行,即$D_{\dot{c}(t)}\mu_i(t)=0$,定义$P_{c,t}(s(t))$将$s(t)$平行移动回$s(0)$,局部来看就是
\[
	P_{c,t}(s^i(t)\mu_i(t))=s^i(t)\mu_i(0).
\]
此时按照联络的定义,可以计算得到$D_{\dot{c}(t)}s(t)=\dot{s}^i(t)\mu_i(t)$,
令$t\to 0$,则
\[
	D_{Y}s(0)=\lim_{t\to 0}\frac{s^i(t)\mu_i(0)-s^i(0)\mu_i(0)}{t}=\lim_{t\to 0}\frac{P_{c,t}(s(t))-s(0)}{t},
\]
这正是我们前面希望得到的矢量场沿一个方向的导数的定义。

现在解释“联络”的意思,假如有一个矢量丛$E$,他的切丛为$TE$,在每一点$p=(\pi(p),v)$,$T_pE$都有一个确定的子空间,即每一点纤维的切空间$T_vE_p$,所谓的联络就是在每一点选定了$T_pE$中$T_vE_p$的补空间\footnote{补空间不唯一,所以需要选定,比如在$\rr^3$中,一条直线的补空间可以是任意不和他平行的平面。}$H_p$,即选定了$H_p$使得$T_pE=T_vE_p\oplus H_p$。$T_vE_p$被称为垂直子空间,或记做$V_p$,而$H_p$则被称为水平子空间,这个命名的直观可以考察平凡丛$E=M\times F$,$T_vE_p=T_v F$是切于$F$的,而$F$可以看做和$M$垂直。垂直子空间的不交并构成一个矢量丛,我们称为垂直丛,他是原本矢量丛的子丛,同样我们有水平丛。

一旦有了一个联络,任意$M$中的一条曲线$c(t)$都可以将$\dot{c}(0)=X$通过平行移动得到$E$中的一条曲线$\psi(t)$。对于同一个$X$,可以考察不同曲线$c(t)$平行移动后得到的切矢量们$\dot{\psi}(0)$,他们构成$T_{(c(0),X)}E=T_{s(0)}E$的一个子空间,这就是$H_{\psi(0)}$.因为$\psi(t)$是使用$D$平行移动而成,他在纤维上的投影是$P_{c,t}^{-1}s(0)$,因此他的切矢量在垂直子空间上的投影$\mathrm{p_v}(\dot{\psi}(0))$是
\[
	\mathrm{p_v}(\dot{\psi}(0))=\lim_{t\to 0}\frac{P_{c,t}^{-1}s(0)-s(0)}{t}
	=-\lim_{t\to 0}P_{c,t}^{-1}D_{X}s(0)=0,
\]
因此$\dot{\psi}(0)\in H_{\psi(0)}$.

上面这种看法可以看做联络的一种定义,因为他是Ehresmann首先形式化定义的,所以也被称为Ehresmann联络。使用Ehresmann联络,可以比较方便对联络的存在性等问题进行考察,也方便搞清楚平行移动的直观等等,详细而严谨的讨论可以参看Jeffrey M. Lee的\emph{Manifolds and Differential Geometry}一书中的第12章。

重新观察
\[
	D_Xs=X(s^j)\mu_j+s^jX^i\Gamma^{k}_{ij}\mu_k=\dd s^j(X)\mu_j+s^jX^i\Gamma^k_{\phantom{k}ij}\mu_k,
\]
将$X^i\Gamma^{k}_{ij}\mu_k$写作$A(X)\mu_j$,那么就可以写出$D=\dd +A$或者更细致一些
\[
	D(s)=D(s^i\mu_i)=\dd s^i \mu_i+s^i A\mu_i.
\]
特别地,对于基$\mu_i$,$D\mu_i=A\mu_i$.现在来看看$A$到底是什么东西,由于$A(X):\mu_j \mapsto X^i\Gamma^k_{\phantom{k}ij}\mu_k=\dd x^i(X)\Gamma^k_{\phantom{k}ij}\mu_k$,或者将$A(X)$写成矩阵
\[
	A_{\phantom{j}i}^{j}(X)=\Gamma^j_{\phantom{j}ki}\dd x^k(X)
\]
因此$A$不是别的,而是一个1-形式值的矩阵,写形式一点$A\in \Gamma(\mathfrak{gl}(n,\rr)\otimes T^*M|_U)$,其中下标$U$指我们在局部考虑这个问题。以后将$A$称为联络$D$的联络形式。

局部来看矢量从$E$,设$U_\alpha$是$M$的一个坐标图册,那么丛的局部平凡化给出了$\varphi_\alpha:\pi^{-1}(U_\alpha):U_\alpha\to U_\alpha\times V$。这样,在非空的$U_\alpha\cap U_\beta$上,我们对于$E$的同一点$p$就有了按$\varphi_\alpha$和$\varphi_\beta$不同的平凡化,他们之间靠一个转移函数$\varphi_{\beta\alpha}:U_\alpha\cap U_\beta\to \gl(V)$相互联系,即
\[
	\varphi_\beta\circ \varphi_\alpha (x,v)=(x,\varphi_{\beta\alpha}(v)),
\]
这个式子也可以直接看做转移函数的定义。

对于同一处不同局部化的同一个截面$s$,我们使用转移函数将其联系起来$s_\beta=\varphi_{\beta\alpha}s_\alpha$,这样,对于局部来看的联络$D_\alpha=\dd+A_\alpha$和$D_\beta=\dd+A_\beta$就有
\[
	\varphi_{\beta\alpha}D_\alpha s_\alpha=D_\beta s_\beta=D_\beta (\varphi_{\beta\alpha}s_\alpha),
\]
具体写出来
\[
\begin{array}{llcl}
	&\varphi_{\beta\alpha}\dd s_\alpha+\varphi_{\beta\alpha}A_\alpha s_\alpha&=&\dd(\varphi_{\beta\alpha}s_\alpha)+A_\beta(\varphi_{\beta\alpha}s_\alpha)\\
	\Rightarrow &\varphi_{\beta\alpha}A_\alpha s_\alpha&=&\dd(\varphi_{\beta\alpha})s_\alpha+A_\beta(\varphi_{\beta\alpha}s_\alpha)\\
	\Rightarrow &A_\alpha &=&\varphi_{\beta\alpha}^{-1}\dd\varphi_{\beta\alpha}+\varphi_{\beta\alpha}^{-1}A_\beta \varphi_{\beta\alpha}.
\end{array}
\]
所以联络形式并不像一个张量一样变化,但是两个不同联络形式的差却是,所以两个联络的差是一个张量场。

一个矢量丛的对偶丛就是指纤维相互为对偶空间的丛,对偶丛上当然也会有联络,因为他也是一个矢量丛。对偶丛上的光滑截面取值为对偶矢量,因此,对偶丛上的光滑截面和矢量丛上的光滑截面在每点通过内积给出了一个值,这就是说$(\mu,\nu^*)$是$M$上的一个光滑实函数,因此我们可以对其求外微分,而两个丛的联络可以通过类似于Leibniz法则通过内积相互联系,即
\[
	\dd (\mu,\nu^*)=(D\mu,\nu^*)+(\mu,D^*\nu^*),
\]
左边右边都是1-形式,所以定义是合理的,这样,通过丛上的联络$D$就定义了对偶丛上的对偶联络$D^*$.令$\mu_j$是局部的一组基,则$(\mu_i,\mu^j)=\delta_{i}^j$且
\[
	0=\dd(\mu_i,\mu^j)=(A^k_{\phantom{k}i}\mu_k,\mu^j)+(\mu_i, A_{k}^{*j}\mu^k)=A^j_{\phantom{j}i}+A_{i}^{*j},
\]
即$A^*=-A^T$,其中上标$T$表示转置。如果用Christoffel符号表示对偶联络,则$D^*_{i}\mu^j=\mu^k\Gamma_{ki}^{\phantom{ki}j}$,两个系数的关系是
\[
	\Gamma_{\phantom{k}ij}^{k}=-\Gamma_{ji}^{\phantom{ji}k}
\]

\para 矢量丛$E_1$和$E_2$的张量积$E_1\otimes E_2$指底空间$M$不变,而纤维变成原本纤维们的张量积。设在$E_1$和$E_2$上有联络$D_1$和$D_2$,则在$E$上按如下方法定义了一个联络$D$
\[
	D(s_1\otimes s_2)=D_1s_1\otimes  s_2+s_1\otimes D_2s_2.
\]

一个有趣的空间是$\mathrm{End}(E)=E\otimes E^*$,令$\sigma=\sigma^{i}_{\phantom{i}j}\mu_i\otimes \mu^j$是$\mathrm{End}(E)$的一个截面,则
\begin{align*}
	D\sigma&=\dd \sigma^{i}_{\phantom{i}j} \mu_i\otimes \mu^j+\sigma^{i}_{\phantom{i}j}D\mu_i\otimes \mu^j+\sigma^{i}_{\phantom{i}j}\mu_i\otimes D^*\mu^j\\
	&=\dd \sigma+A_{\phantom{k}i}^{k}\sigma^{i}_{\phantom{i}j}\mu_k\otimes \mu^j+\sigma^{i}_{\phantom{i}j}\mu_i\otimes A_{k}^{*j}\mu^k\\
	&=\dd \sigma+A_{\phantom{k}i}^{k}\sigma^{i}_{\phantom{i}j}\mu_k\otimes \mu^j-\sigma^{i}_{\phantom{i}j}A_{\phantom{j}k}^{j}\mu_i\otimes\mu^k\\
	&=\dd \sigma+[A,\sigma].
\end{align*}

类似的计算可以给出任意张量场$\omega$的联络满足:

\pro 假如$\omega$是一个$(r,s)$型张量场,则
\begin{align*}
	X(\omega(\eta^i;X_j))=&(D_X\omega)(\eta^i;X_j)\\
	&+\sum_i \omega\left(\eta^{1},\cdots,\eta^{i-1},D^*_X \eta^i,\eta^{i+1},\cdots,\eta^r;X_j\right)\\
	&+\sum_j \omega\left(\eta^i;X_{1},\cdots,X_{j-1},D_X X_j,X_{j+1},\cdots,X_s\right).
\end{align*}

这个等式可以理解成广义的Leibniz法则。
\section{Curvature}
前面已经知道
\[
	D:\Gamma (E)\to\Gamma(E)\otimes \Gamma(T^*M)=\Gamma(E)\otimes \Omega^1(M),
\]
其中$\Omega^1(M)$是1-形式场,$\Gamma(E)\otimes \Omega^1(M)$中的元素可以看做矢量值的1-形式,那么自然,可以将$\Gamma(E)\otimes \Omega^p(M)$看做矢量值的$p$-形式,不妨将其记做$\Omega^p(E)$,以及$\Gamma (E)=\Omega^0 (E)$,则联络实际上是完成了矢量值的$0$-形式到矢量值的$1$-形式的转变:
\[
	D:\Omega^0 (E)\to \Omega^1(E).
\]
类似于$\dd$是从$p$-形式到$(p+1)$-形式的转变,我们希望拓展联络完成$\Omega^p (E)$到$\Omega^{p+1}(E)$的转变。

首先定义一个$s$-形式$\omega_1$和一个矢量值的$r$-形式$\sigma=\mu\otimes\omega_2\in \Omega^r (E)$的楔积,自然
\[
	\sigma\wedge \omega_1=\mu\otimes(\omega_2\wedge \omega_1)
\]
就构造出了一个矢量值的$(r+s)$-形式。然后我们扩展联络如下:对一个矢量值的$r$-形式$\sigma=\mu \otimes \omega$,其中$\omega$是一个$r$-形式,定义
\[
	D\sigma=D\mu \wedge\omega+\mu\otimes \dd \omega.
\]

但是,正如我们熟知的恒等式$\dd^2=0$,他其实在局部是等价于等式$\partial_i \partial_j=\partial_j \partial_i$,或者$[\partial_i,\partial_j]=0$,直观上就是说,沿着正交方向先后求导,他们的结果是一样的。前面已经看到了,$\rr^n$上存在联络$D=\dd$,因此在$\rr^n$上的这个联络依然是满足$D^2=0$的。可以指出,$[\partial_i,\partial_j]=0$是因为我们是在同一个切空间内求导的结果,而这就忽略掉了流形本身的具体结构,而联络并不会。

为了更清晰地看到这一点,设$s=s^i\mu_i$,计算
\begin{align*}
	D^2 s&=D(\dd s^i \mu_i+s^iA\mu_i)\\
	&=D\mu_i\wedge \dd s^i+D(s^i A^{j}_{\phantom{j}i}\mu_j)\\
	&=A\mu_i\wedge \dd s^i+D(s^i\mu_j)\wedge A^{j}_{\phantom{j}i}+s^i\mu_j\dd A^{j}_{\phantom{j}i}\\
	&=-\dd s^i\wedge A\mu_i+D(s^i\mu_j)\wedge A^{j}_{\phantom{j}i}+(\dd A)s\\
	&=s^i A\mu_j\wedge A^{j}_{\phantom{j}i}+(\dd A)s\\
	&=s^i \mu_k A^{k}_{\phantom{k}j}\wedge A^{j}_{\phantom{j}i}+(\dd A)s\\
	&=(A\wedge A)s+(\dd A)s.
\end{align*}
所以作用在矢量值0-形式上,$D^2=A\wedge A+\dd A$,这个量,我们另外给个名字。

\para 一个联络$D$的曲率为$F:=D^2:\Omega^0(E)\to \Omega^2(E)$.

根据上面求的,局部曲率算符有$F=A\wedge A+\dd A$,他一般来说不为0,如果曲率算符为0,则称这个联络是平坦的。

曲率算符可以看做$\mathrm{End}(E)$值的2-形式,因为$F:\Omega^0(E)\to \Omega^2(E)$,于是
\[
	F\in \Omega^2(E)\otimes (\Omega^0(E))^*=\Gamma(E)\otimes \Omega^2(M)\otimes \Gamma(E^*)=\Gamma(\mathrm{End}(E))\otimes \Omega^2(M),
\]
这就是说$F\in \Omega^2(\mathrm{End}(E))$.

设$A=A_i\dd x^i$,其中$A_i$是$n\times n$矩阵,直接计算给出了
\[
	F=\frac{1}{2}\left(\partial_{[i}A_{j]}+[A_i,A_j]\right)\dd x^i\wedge \dd x^j.
\]
观察这个式子是很有趣的,前面在电磁学里面定义了电磁场张量$F_{ij}=\partial_{[i}A_{j]}$,其中$A_i$是四维势。如果四维势对应联络形式,那么电磁场张量对应着曲率。曲率中多出来的$[A_i,A_j]$在电磁学中是自然消失的。

\para 直接的计算可以给出Bianchi等式:$DF=0$.

使用$A_\alpha=\varphi_{\beta\alpha}^{-1}\dd\varphi_{\beta\alpha}+\varphi_{\beta\alpha}^{-1}A_\beta \varphi_{\beta\alpha}$,我们考察$F$在坐标变换下的改变
\begin{align*}
	F_\alpha&=A_\alpha\wedge A_\alpha+\dd A_\alpha\\
	&=\varphi_{\beta\alpha}^{-1}A_\beta \wedge A_\beta \varphi_{\beta\alpha}+ \varphi_{\beta\alpha}^{-1}A_\beta \varphi_{\beta\alpha}\wedge \varphi_{\beta\alpha}^{-1}\dd\varphi_{\beta\alpha}+\varphi_{\beta\alpha}^{-1}\dd\varphi_{\beta\alpha}\wedge \varphi_{\beta\alpha}^{-1}A_\beta \varphi_{\beta\alpha}+\dd A_\alpha\\
	&=\varphi_{\beta\alpha}^{-1}A_\beta \wedge A_\beta \varphi_{\beta\alpha}+ \varphi_{\beta\alpha}^{-1}A_\beta \wedge \dd\varphi_{\beta\alpha}-\dd\varphi_{\beta\alpha}^{-1}\varphi_{\beta\alpha}\wedge \varphi_{\beta\alpha}^{-1}A_\beta \varphi_{\beta\alpha}+\dd A_\alpha\\
	&=\varphi_{\beta\alpha}^{-1}A_\beta \wedge A_\beta \varphi_{\beta\alpha}+ \varphi_{\beta\alpha}^{-1}A_\beta \wedge \dd\varphi_{\beta\alpha}-\dd\varphi_{\beta\alpha}^{-1}\wedge A_\beta \varphi_{\beta\alpha}+\dd A_\alpha\\
	&=\varphi_{\beta\alpha}^{-1}A_\beta \wedge A_\beta \varphi_{\beta\alpha}+ \varphi_{\beta\alpha}^{-1}A_\beta \wedge \dd\varphi_{\beta\alpha}+\varphi_{\beta\alpha}^{-1}\dd \left(A_\beta \varphi_{\beta\alpha}\right)\\
	&=\varphi_{\beta\alpha}^{-1}A_\beta \wedge A_\beta \varphi_{\beta\alpha}+\varphi_{\beta\alpha}^{-1}(\dd A_\beta)\varphi_{\beta\alpha}\\
	&=\varphi_{\beta\alpha}^{-1}F_\beta \varphi_{\beta\alpha}.
\end{align*}
在计算过程中,使用了
\[
	0=\dd I=\dd (\varphi_{\beta\alpha}^{-1}\varphi_{\beta\alpha})=\dd \varphi_{\beta\alpha}^{-1}\varphi_{\beta\alpha}+\varphi_{\beta\alpha}^{-1}\dd\varphi_{\beta\alpha},
\]
所以$F$变换得符合张量的变换方式,即$F$是一个张量。可以将$Fs$写作$R(\star,\star)s$,其中$R$被称为曲率张量,下面的定理实现了用联络表示曲率张量$R$,证明依然是直接的计算。

\para 一个联络$D$的曲率张量$R$满足
	\[
		R(X,Y)s=D_XD_Ys-D_YD_Xs-D_{[X,Y]}s.
	\]

因为他是张量,所以
\[
	R(X,Y)(fs)=fR(X,Y)s,
\]
此外对$X,Y$都函数线性从上面的定理来看是显然的,还有$R(X,Y)+R(Y,X)=0$也是显然的。

前面提到了,曲率涉及了流形本身(包含联络)的具体结构,从曲率张量来看,给定一个光滑曲线族$f(s,t):\rr\times \rr\to M$,按照曲线$f(s,0)$将切矢量$v$从$f(0,0)$平行移动到$f(1,0)$,然后按照曲线$f(1,t)$平行移动到$f(1,1)$,之后平行移动到$f(0,1)$,最后平行移动回$f(0,0)$得到了新的切矢量$v'$,曲率就是度量了$v'$和$v$之间的差异。

\section{Levi-Civita Connection}
将上面两节的理论应用到伪Riemann流形,应用到伪Riemann流形的切丛上。切丛上的联络不用$D$而写作$\nabla$.

由于伪Riemann流形的切丛上赋予了度规,这就比前面单纯的矢量场上的联络理论要有更多内容,这一点体现在联络和度规相容条件上:
\[
	\dd \langle \mu,\nu\rangle=\langle \nabla\mu,\nu\rangle+\langle\mu,\nabla\nu\rangle.
\]
或者
\[
	X \langle \mu,\nu\rangle=\langle \nabla_X\mu,\nu\rangle+\langle\mu,\nabla_X\nu\rangle.
\]

当然Levi-Civita联络要比这还要特殊一点。切丛上除了曲率,还能定义一个叫做挠率的张量
\[
	T(X,Y)=\nabla_X Y-\nabla_Y X-[X,Y].
\]
Levi-Civita联络的挠率为0,即Levi-Civita联络是无挠的。联络无挠在局部等价于$\Gamma^{k}_{\phantom{k}ij}=\Gamma^{k}_{\phantom{k}ji}$. 挠率的几何意义其实并不很清楚。

\theo 伪Riemann流形存在唯一的和度规相容的、无挠的联络,我们称之为Levi-Civita联络。

\proof 唯一性的证明,和度规相容的、无挠的联络$\nabla$一定满足:
\[
	\langle \nabla_XY,Z\rangle=\frac{1}{2}\left(X\langle Y,Z\rangle+Y\langle X,Z\rangle-Z\langle X,Y\rangle+\langle[X,Y],Z\rangle-\langle[X,Z],Y\rangle-\langle[Y,Z],X\rangle\right).
\]

利用和度规相容
\begin{align*}
	X\langle Y,Z\rangle&=\langle \nabla_XY,Z\rangle+\langle Y,\nabla_XZ\rangle,\\
	Y\langle X,Z\rangle&=\langle \nabla_YX,Z\rangle+\langle X,\nabla_YZ\rangle,\\
	Z\langle X,Y\rangle&=\langle \nabla_ZX,Y\rangle+\langle X,\nabla_ZY\rangle.
\end{align*}
利用无挠性$\nabla_X Y-\nabla_Y X=[X,Y]$,则
\[
	X\langle Y,Z\rangle+Y\langle X,Z\rangle-Z\langle X,Y\rangle=2\langle \nabla_XY,Z\rangle-\langle[X,Y],Z\rangle+\langle[X,Z],Y\rangle+\langle[Y,Z],X\rangle.
\]
唯一性证明完毕。

存在性的证明,固定$X,Y$,令右边的为$\omega(Z)$,则可以直接计算得$\omega(fZ)=f\omega(Z)$,此外$\omega(Z_1+Z_2)=\omega(Z_1)+\omega(Z_2)$。那么从度规的非退化可以得知,存在唯一的$A$使得$\langle A,Z\rangle=\omega(Z)$,这样子,令$\nabla_XY=A$,最后检验这是一个无挠、和度规相容的联络即可。\qed

伪Riemann流形的Levi-Civita联络的Christoffel记号为
\[
	\Gamma^i_{\phantom{i}jk}=\frac{1}{2}g^{il}(\partial_k g_{jl}+\partial_j g_{kl}-\partial_l g_{jk}),
\]
注意到,这就是前面我们里面作用量求测地线方程时候出现的Christoffel记号。

\para 有切丛上的联络$\nabla$,则一条可微曲线$c(t)$称为自平行的或者称为测地线,如果其满足$\nabla_{\dot{c}}\dot{c}=0$.

利用
\[
D_{\dot{c}}s(c(t))=\left(\dot{s}^j\mu_j+\dot{c}^i s^j\Gamma^k_{\phantom{k}ij}\mu_k\right)(c(t)),
\]
直接给出测地线方程为
\[
\nabla_{\dot{c}}\dot{c}=\ddot{c}^j\mu_j+\dot{c}^i \dot{c}^j\Gamma^k_{\phantom{k}ij}\mu_k=0,
\]
再简单一些
\[
\ddot{c}^k+\dot{c}^i \dot{c}^j\Gamma^k_{\phantom{k}ij}=0.
\]

很久以前,在谈论作用量原理的时候,讲到要寻找和$k_i$正交的矢量$v^i$,其中
\[
	2k_i=2(u_i)'-\partial_ig_{mn}u^mu^n,
\]
现在我们指出$k_i=(\nabla_{u}u)_i$,所以寻找和$k_i$正交的矢量$v^i$就是指寻找一个矢量场$v$使得$\langle \nabla_{u}u,v\rangle=0$. 利用Levi-Civita联络的性质
\[
	u\langle u,u\rangle=2\langle \nabla_{u}u,u\rangle.
\]
因为$\langle u,u\rangle=1$,所以$v=u$就是一个自然的解,这和狭义相对论里面讨论的一样。如果使用联络的记号,那么在$l_iu^i=0$情况下粒子的运动方程写作
\[
	l_i=-mc(\nabla_{u}u)_i=-(\nabla_{u}p)_i.
\]
这就是鼎鼎大名的牛顿第二定律。

本节的最后来谈谈活动标架法,活动标架法利用的是局部基和对偶基来计算联络和曲率。从直接的计算开始,可以得到:

\pro 设$D$是无挠联络,则
\[
	\dd \omega(X_1,\cdots,X_{r+1})=\sum_{i=1}^{r+1}(-1)^{i-1}D_{X_i}\omega(X_1,\cdots,\hat{X_i},\cdots,X_{r+1}),
\]
其中$\hat{X_i}$指在括号中没有这项。

对$r=1$和Levi-Civita联络,
\[
	\dd \omega(X,Y)=\nabla^*_{X}\omega(Y)-\nabla^*_{Y}\omega(X),
\]
对$\mu^i$使用上式,注意到$\nabla^*\mu^i=A^*\mu^i=\mu^jA^{*i}_j=-\mu^jA^i_{\phantom{i}j}$,得到
\begin{align*}
	\dd \mu^i(X,Y)=\nabla^*_{X}\mu^i(Y)-\nabla^*_{Y}\mu^i(X)&=-\mu^j(Y)A^i_{\phantom{i}j}(X)+\mu^j(X)A^i_{\phantom{i}j}(Y)\\
	&=(\mu^j\otimes A^i_{\phantom{i}j}-A^i_{\phantom{i}j}\otimes \mu^j)(X,Y)\\
	&=(\mu^j\wedge A^i_{\phantom{i}j})(X,Y),
\end{align*}
所以
\begin{equation}
	\dd \mu^i=\mu^j\wedge A^i_{\phantom{i}j}=-A^i_{\phantom{i}j}\wedge \mu^j,
	\label{cartan1}
\end{equation}
交换后有负号是因为这是两个1-形式。

前面已经对$s=s^i\mu_i$算过
\[
	F s=s^i \mu_k A^{k}_{\phantom{k}j}\wedge A^{j}_{\phantom{j}i}+s^i\mu_j\dd A^{j}_{\phantom{j}i},
\]
对$s=\mu_i$有
\[
	F \mu_i=\mu_k A^{k}_{\phantom{k}j}\wedge A^{j}_{\phantom{j}i}+\mu_j\dd A^{j}_{\phantom{j}i},
\]
所以
\[
	R(X,Y) \mu_i=F \mu_i(X,Y)=\mu_k A^{k}_{\phantom{k}j}\wedge A^{j}_{\phantom{j}i}(X,Y)+\mu_j\dd A^{j}_{\phantom{j}i}(X,Y),
\]
另一方面将$R(X,Y)$写成分量的形式
\[
	R(X,Y) \mu_k=X^iY^jR_{ijk}^{\phantom{ijk}l}\mu_l=R_{ijk}^{\phantom{ijk}l} \mu^i\otimes\mu^j(X,Y)\mu_l=\frac{1}{2}R_{ijk}^{\phantom{ijk}l} \mu^i\wedge\mu^j(X,Y)\mu_l
\]
其中$R_{ijk}^{\phantom{ijk}l}=\mu^l(R(\mu_i,\mu_j)\mu_k)$,最后一个等号使用了$R(X,Y)$关于$X$, $Y$是反对称的,所以$R_{ijk}^{\phantom{ijk}l}$中的$i$, $j$也是反对称的,于是
\begin{equation}
	\frac{1}{2}R_{mni}^{\phantom{mni}k} \mu^m\wedge\mu^n=A^{k}_{\phantom{k}j}\wedge A^{j}_{\phantom{j}i}+\dd A^{k}_{\phantom{k}i}.
	\label{cartan2}
\end{equation}

式\eqref{cartan1}和\eqref{cartan2}统称Cartan结构方程。有时候会记
\[
	\Omega^{k}_{\phantom{k}i}=-\frac{1}{2}R_{mni}^{\phantom{mni}k} \mu^m\wedge\mu^n,
\]
称之为曲率形式,那么结构方程写作
\[
	\dd A^{k}_{\phantom{k}i}=-\Omega^{k}_{\phantom{k}i}-A^{k}_{\phantom{k}j}\wedge A^{j}_{\phantom{j}i}.
\]

从一般的曲率张量本身可以构造几个特殊的曲率,比如:

\para 设$\Pi$为切空间$T_pM$的二维子空间,取他的一组基$X$, $Y$,定义$\Pi$的截面曲率为
\[
	K(\Pi)=\frac{\langle R(X,Y)X,Y\rangle}{\langle X,X\rangle\langle Y,Y \rangle-\langle X,Y \rangle^2}.
\]
容易直接计算验证$K(\Pi)$的定义和基的选取无关。

\para 设$X,Y,Z\in T_p M$,那么映射$X\to R(Y,X)Z$就是一个$T_p M$上的自同态,定义这个自同态的迹
\[
	\mathrm{Ric}(Y,Z)=\tr (X\to R(Y,X)Z)
\]
为Ricci曲率张量。他直接出现在Einstein场方程中。

\section{Hodge Star Operator}

假设流形是可定向的无边流形\footnote{Hodge用一套整体分析的方法来研究了de Rham上同调群,而他的星算子就是这套方法中很重要的一个部分。稍稍具体一点,Hodge通过在Riemann流形上引入度量,利用度量在每一个同调类中选出代表元,而每个代表元都是椭圆算子的核,然后就可以使用椭圆算子核的性质。具体的展开这里不可能表。},此外下面谈到的形式都是有紧支集的。所以根据Stokes定理,$(n-1)$-形式的外微分在整个流形上的积分为0.

从体积形式开始,在流形的局部,比如考虑$(U,\phi)$,上面的度规有$g_{ij}$,同样地,$(U',\phi')$和$g'_{ij}$,现在假设两个局部是相交的,则体积形式作为几何量,应该是不变的。设在$U$上的坐标为$x$,$U'$上的为$y$,所以
\[
	\mathrm{d}y^1\wedge \cdots \wedge\mathrm{d}y^n=\det\left(\frac{\partial y}{\partial x}\right)\mathrm{d}x^1\wedge \cdots \wedge \mathrm{d}x^n,
\]
其中$\partial y/\partial x$是坐标变换的Jacobian,这里选取的坐标变换是保向的,即Jacobian的行列式大于零。

体积形式必然同时正比于$\mathrm{d}y^1\wedge \cdots \wedge \mathrm{d}y^n$和$\mathrm{d}x^1\wedge \cdots \wedge \mathrm{d}x^n$,当然这是不够的,他还依赖于度规的选取。我们考虑一下度规的变化
\[
	g_{ij}'=g(\partial_i',\partial_j')=g\left(\frac{\partial x^k}{\partial y^i}\partial_k,\frac{\partial x^l}{\partial y^j}\partial_l\right)=\frac{\partial x^k}{\partial y^i}\frac{\partial x^l}{\partial y^j}g_{kl},
\]
为了消掉上面那个行列式,我们考虑一下上式的行列式
\[
	\det(g_{ij}')=\det\left(\frac{\partial x}{\partial y}\right)^2\det(g_{ij})=\det\left(\frac{\partial y}{\partial x}\right)^{-2}\det(g_{ij})
\]
这里出现了2,所以还要开方一下,则有
\[
	\sqrt{|\det(g_{ij}')|}=\det\left(\frac{\partial y}{\partial x}\right)^{-1}\sqrt{|\det(g_{ij})|}
\]
综上
\[
	\sqrt{|\det(g_{ij}')|}\mathrm{d}y^1\wedge \cdots \wedge \mathrm{d}y^n=\sqrt{|\det(g_{ij})|}\mathrm{d}x^1\wedge \cdots \wedge \mathrm{d}x^n,
\]
这是一个几何量,因此他就是体积形式。

Hodge星算子的定义和第一章里面的一样
\[
	\omega\wedge(\star \mu)=\langle \omega,\mu\rangle \mathrm{vol},
\]
通过耐心细致但不复杂的计算可以得到,

\pro 在伪Riemann流形$M$上,如果他的度规的负本征值个数为$s$,那么在$p$-形式上成立恒等式
	\[\star\star=(-1)^{s+p(n-p)}.\]
证明很类似在$\rr^{s+(n-s)}$上的证明。

有了Hodge星算子,可以在任意两个$p$-形式之间定义一个新的内积
\[
	(\omega,\mu)=\int \omega\wedge(\star \mu),
\]
很容易看出他是双线性的。

现在假如有一个$(p-1)$形式$\omega$,那么对任意的$p$-形式$\mu$,$(\dd \omega,\mu)$是可以定义的,这样,类似于定义转置算符,我们可以定义$\dd$的对偶算子$\dd^\star$如下
\[
	(\omega,\dd^\star\mu)=(\dd \omega,\mu).
\]
所以$\dd^\star$完成的是从$p$形式到$(p-1)$-形式的转变,只是是否存在还尚且未知。

对$\omega\wedge\star\mu$求外微分
\begin{align*}
	\dd (\omega\wedge\star\mu)&=\dd \omega\wedge\star\mu+(-1)^{p-1}\omega\wedge\dd(\star\mu)\\
	&=\dd \omega\wedge\star\mu+(-1)^{p-1}\omega\wedge\dd(\star\mu)\\
	&=\dd \omega\wedge\star\mu+(-1)^{p-1}(-1)^{s+(p-1)(n-p+1)}\omega\wedge\star\star\dd(\star\mu)\\
	&=\dd \omega\wedge\star\mu-(-1)^{s+(p+1)n+1}\omega\wedge\star(\star\dd\star\mu),
\end{align*}
两边积分后,左边由Stokes公式直接为0,所以
\[
	0=(\dd \omega,\mu)-(-1)^{s+(p+1)n+1}(\omega,\star\dd\star\mu),
\]
最后就求得了
\[
	\dd^\star=(-1)^{s+(p+1)n+1}\star\dd\star.
\]
所以前面写过的含源部分的Maxwell方程组$\star\dd \star \mathbf{F}=-4\pi\mathbf{j}$就等价于
\[
	\dd^\star \mathbf{F}=-4\pi\mathbf{j}.
\]

\para 定义Hodge-Laplace算子$\Delta=\dd^\star\dd+\dd\dd^\star$,他将$p$-形式变成$p$-形式。

他的一些性质直接计算即可,比如$\Delta\star=\star\Delta$.他是Laplace算子在流形上的推广,假如我们的度规是对角化的,记$g_i=g_{ii}$,则对光滑实函数$f$有,
\begin{align*}
	\Delta f&=\dd^\star\dd f+\dd\dd^\star f\\
	&=\sum_i(-1)^{s+1}\star\dd\star\left(\frac{\partial f}{\partial x^i}\dd x^i\right)\\
	&=\sum_i(-1)^{s+1}\star\dd\left((-1)^{i-1}\frac{\sqrt{|\det(g)|}}{g_i} \frac{\partial f}{\partial x^i}\dd x^1\wedge\cdots\hat{\dd x^i}\cdots\dd x^n\right)\\
	&=\sum_i(-1)^{s+1}\frac{\partial}{\partial x^i}\left(\frac{\sqrt{|\det(g)|}}{g_i} \frac{\partial f}{\partial x^i}\right)\frac{\star\mathrm{vol}}{\sqrt{|\det(g)|}}\\
	&=\sum_i\frac{(-1)^{s+1}}{\sqrt{|\det(g)|}}\frac{\partial}{\partial x^i}\left(\frac{\sqrt{|\det(g)|}}{g_i} \frac{\partial f}{\partial x^i}\right),
\end{align*}
其中$\dd\dd^\star f=0$是因为$\dd^\star f$是一个$n$-形式。

对于$\rr^n$,有$s=0$, $g_i=1$,则
\[
	\Delta f=-\sum_i\frac{\partial}{\partial x^i}\left(\frac{\partial f}{\partial x^i}\right),
\]
仅仅和我们熟知的Laplace算子差一个负号而已\footnote{在数学里,有些符号的差异是无法避免的。}。

对于$s=3$的$\rr^{3+1}$有
\[
	\Delta f=\frac{\partial}{\partial x^0}\left(\frac{\partial f}{\partial x^0}\right)-\sum_{i=1}^3\frac{\partial}{\partial x^i}\left(\frac{\partial f}{\partial x^i}\right),
\]
就是我们熟知的D'Alembert算子。

如果我们现在有一个矢量值形式,即$w\otimes \omega$,其中$w\in W$是一个矢量空间中的元素,而$\omega$是一个$p$-形式。如果$W$有一个非退化双线性函数$B:W\otimes W\to \rr$,那么对于两个$W$值$p$-形式,
\[
	v=\omega\otimes X,\quad w=\mu\otimes Y,
\]
其中$X,Y\in \lag$,定义他们的内积为
\[
	\langle v,w\rangle=B(X,Y) \langle\omega,\mu\rangle,
\]
其中$\langle\omega,\mu\rangle$则是$p$-形式之间的内积,即
\[
	\langle e^1\wedge\cdots \wedge e^p,f^1\wedge\cdots \wedge f^p\rangle =\det(g^{-1}(e^i,f^j)).
\]
有了上面这个内积的定义,我们可以拓展星算子定义到$W$值形式依然如
\[
	v\wedge(\star w)=\langle v,w\rangle \mathrm{vol}.
\]
特别地,如果$W=\lag$是一个Lie代数,$B$是这个Lie代数上面的非退化的双线性度量,那么我们就得到了$\lag$形式上的星算子定义。如果Lie群是半单的,那么Lie代数上面必然存在一个双不变度量,即Killing形式$(A,B)_K=\tr(\ad(A)\circ \ad(B))$。Killing形式本质上就是Lie群在双不变度量下的Ricci曲率,Killing形式本身是双不变的,而且在半单假设下,他是非退化的。


\section{Principal Bundle and Connection}
主丛是另一种纤维丛,他在规范理论里面处于基本的地位。

\para 设有纤维丛$(P,\pi,M,G)$,他被称为一个主$G$-丛如果

	\no{1} $G$是一个Lie群。

	\no{2} $G$自由右作用于$P$上:即存在运算$P\times G\to P$,且一旦存在$p$成立$p\cdot g=p$,则$g$是$G$的单位元。$M$微分同胚于$G$在$P$上面轨道的集合$P/G$.

	\no{3} 局部平凡化的时候,如果$p\in P$局部写作$(\pi(p),g)$,那么同一个轨道上的$p\cdot g'$的局部平凡化写作$(\pi(p),gg')$.其中$gg'$之间的乘法就是群乘法。


一个主丛自然地和一个矢量丛联系起来,通过群表示的方式。找一个矢量空间$V$,现在群$G$在$V$有表示,是左作用的。那么我们可以在$P\times V$上通过
\[
	(p,v)\cdot g=(p\cdot g,\rho(g^{-1})v)
\]
定义一个自由的右作用,其中$\rho$是$G$在$V$上的群表示。这样$P\times V/G\to P/G$就是一个矢量丛,他的纤维同构于$V$.这可以很简单检验,考虑$p\cdot G\in P/G$,那么他在$P\times V/G$中的原象就是$(p\cdot G,V)$.

如果考虑矢量丛$P\times V/G\to M\cong P/G$上两个不同平凡化之间的转移函数,就会发现转移函数还是$G$左作用于$V$.转而考察主丛$P\to M$,他的转移函数就是$G$的元素,由于满足
\begin{equation*}
	g_{\beta\alpha}g_{\alpha\beta}=e,\quad g_{\beta\gamma}g_{\alpha\beta}=g_{\alpha\gamma},\quad g_{\gamma\sigma}g_{\beta\gamma}g_{\alpha\beta}=g_{\alpha\sigma}.
\end{equation*}
所以转移函数的取值构成一个群,他是$G$的子群$H$,这样,就称呼$H$是主丛$P\to M$的结构群。

要在主丛上定义联络,Ehresmann联络就很方便。

\para 丛上的联络是指在主丛的切丛上光滑地指派一个子丛$H$,他在每一点都是垂直子空间的补空间,并且对于$R_g$诱导的切空间的映射满足$H_{p\cdot g}=(R_g)_*H_p$.


现在,我们来看一些主丛的例子,首先是球面上的主丛。在$\rr^n$中,一组标准正交基$\{e_i\}$称为一个标准正交标架,因为两组正交标架之间差一个正交变换$O(n)$,定义映射
\[
	\begin{array}{lccl}
		\pi:&O(n)&\to &S^{n-1}\\
		&\{e_i\}&\mapsto&e_n
	\end{array}
\]
当$e_n$固定时,$\{e_1,\cdots,e_{n-1}\}$是$e_n$的正交补空间中的标准正交标架,这些标架看到全体可以等同于$O(n-1)$,可验证$\pi$给出了$S^{n-1}$上的纤维丛,该纤维丛也是主丛,结构群是$O(n-1)$,在$O(n)$上的右作用为
\[
	\begin{array}{lcl}
		O(n)\times O(n-1)&\to& O(n)\\
		(A,B)&\mapsto& A\cdot \mathrm{diag}(B,1),
	\end{array}
\]
其中$\mathrm{diag}(B,1)$指的是对角为$B$和$1$的矩阵,乘法即矩阵乘法。同样地,如果考虑定向标准正交基,那么$SO(n)$就是$S^{n-1}$上的主丛,纤维是$SO(n-1)$.

再比如Riemann流形上的标架丛。设$M$是一个Riemann流形,$T_pM$中的一组标准正交基称为标架,令$F(M)$是流形各处标架的不交并,$F(M)$上可以自然定义微分结构使其变成一个可微流形,映射$\pi :F(M)\to M$显示了$F(M)$其实是一个主丛,纤维是$O(n)$.

\para 设Lie群$G$光滑右作用于流形$M$,在$G$的Lie代数$\mathfrak{g}$中任取矢量$A$,他都可以有一个$M$的单参变换群$R_{\exp(tA)}$,这个变换群的无穷小生成元是一个$M$上的矢量场,称为由$A$诱导的基本矢量场,记做$A^*$。或者将$(A^*)_p$看做曲线$\sigma(t)=R_{\exp(tA)}p=p\cdot\exp(tA)$在$\sigma(0)=p$处的切矢量$\dot\sigma(0)$.

当$G$为自由作用时,只要$A\neq 0$,则$A^*$处处非零。对于主丛$P$,$G$自由右作用于他,所以每一个矢量$A\in \mathfrak{g}$都诱导了$P$上的基本矢量场$A^*$,映射$A\to (A^*)_u$是$\mathfrak{g}\to T_uG=V_uP$的线性同构,所以$A^*$作为$TP$的切矢量场,他只有垂直分量。

设主丛$P$上有联络$H_p$,对于每一个$p\in P$,可以定义线性映射$\omega_p:T_p P\to \mathfrak{g}$通过对任意的$X\in T_p P$指定$\omega_p(X)=A\in \mathfrak{g}$为满足$(A^*)_p=\mathrm{p_v}(X)$的唯一矢量$A$,这样我们就得到了$P$上$\mathfrak{g}$值1-形式$\omega$,称为给定联络的联络形式。

很容易根据定义得到,$\forall A\in\mathfrak{g}$可以得到$\omega(A^*)=A$,由于$A^*\in V_pP$,或者$\mathrm{p_v}(A^*)=A^*$,那么自然$\omega(A^*)=A$.

$\omega$作为形式,自然可以拉回,尤其是右作用$R_g$的拉回$(R_g)^*$。

\pro 	联络形式满足:$(R_g)^*\omega=\mathrm{Ad}(g^{-1})\omega$.

\proof 使用拉回的定义,只要对任意的矢量场$X$考察$((R_g)^*\omega)(X)$就可以知道$(R_g)^*\omega$了。由于线性性,我们可以分成垂直方向和水平方向考虑,对于竖直方向,因为每一点都有$\mathfrak{g}\cong V_pP$,所以不妨将竖直矢量场看做一个基本矢量场$X=A^*$
\[
	((R_g)^*\omega)(A^*)=\omega((R_g)_*A^*).
\]

现在来求$(R_g)_*A^*$,
\[
	\left.\frac{\dd}{\dd t}\right|_{t=0}(R_gR_{\exp(tA)})(p)=\left.\frac{\dd}{\dd t}\right|_{t=0}(R_gR_{\exp(tA)}R_g^{-1})(p\cdot g)=\left.\frac{\dd}{\dd t}\right|_{t=0}(R_{g^{-1}\exp(tA)g})(p\cdot g),
\]
因为$g^{-1}\exp(tA)g=\exp(t\mathrm{Ad}(g^{-1})A)$,上面的等式就是说$\mathrm{Ad}(g^{-1})A$生成了基本矢量场$\left(R_g\right)_*A^*$.

综上
\[
	((R_g)^*\omega)(A^*)=\omega((R_g)_*A^*)=\mathrm{Ad}(g^{-1})A=\mathrm{Ad}(g^{-1})(\omega(A^*)),
\]
这就是说在竖直方向$(R_g)^*\omega=\mathrm{Ad}(g^{-1})\omega$.

对于水平方向,由于$\mathrm{p_v}(X)=0$所以$\omega(X)=0$,此外$(R_g)_*$由于将水平空间映到水平空间,所以$(R_g)_*X$依然是水平矢量场,于是$(R_g)^*\omega=\mathrm{Ad}(g^{-1})\omega$,因为两边都等于0.\qed

反之,如果给定了$P$上一个$\mathfrak{g}$值1-形式$\omega$,满足$\omega(A^*)=A$和$(R_g)^*\omega=\mathrm{Ad}(g^{-1})\omega$,那么可以定义一个联络通过
\[
	H_p=\ker(\omega_p)=\{X\in T_p P:\omega_p(X)=0\}.
\]
\section{Connection Form}
这里在局部计算主丛的联络形式,并将其和矢量丛里面的联络形式联系起来。

首先在主丛$P$有局部平凡化$(U_\alpha,\psi_\alpha)$,记微分同胚$\psi_\alpha(p)=(\pi(p),g_\alpha(p))$.那么在两个相交的平凡化上,转移函数满足
\[
	g_\beta(p)=g_{\beta\alpha}g_\alpha(p),
\]
并且主丛的定义要求平凡化要在同一个轨道里,所以
\[
	\psi_\alpha(p\cdot g)=(\pi(p),g_\alpha(p)g).
\]

在每一个$\alpha$,定义$U_\alpha$上的局部截面$\sigma_\alpha$为
\[
	\sigma_\alpha(x)=\psi_\alpha^{-1}(x,e),
\]
转移到其他平凡化里面
\[
	\sigma_\beta(x)=\psi_\beta^{-1}(x,e)=\psi_\alpha^{-1}(x,g_{\alpha\beta}(x))=\psi_\alpha^{-1}(x,e)\cdot g_{\alpha\beta}(x)=\sigma_\alpha(x)\cdot g_{\alpha\beta}(x).
\]

对每一个$\alpha$,通过联络形式$\omega$可以定义一个$\mathfrak{g}$值1-形式$\omega_\alpha$,那么通过截面$\sigma_\alpha:U_\alpha\to P$诱导的$\sigma_\alpha^*:T^*P\to T^*U_\alpha$,定义
\[
	\omega_\alpha=\sigma_\alpha^*(\omega).
\]
关于他在转移函数下的表现,考察$x\in U_\alpha\cap U_\beta$,设$X\in T_x M$,由截面在转移函数下的表现$\sigma_\beta(x)=R_{g_{\alpha\beta}(x)}\sigma_\alpha(x)$,求个导数(定义$L_p(g)=p\cdot g$)
\[
	(\sigma_\beta(x))_*X=(R_{g_{\alpha\beta(x)}})_*(\sigma_\alpha(x))_*X+\left(L_{\sigma_\alpha(x)}\right)_*(g_{\alpha\beta})_*X,
\]
上面式子的第两项是因为$g_{\alpha\beta}(x)$也是和$x$有关的。于是
\begin{align*}
	\omega_\beta(x)X&=((\sigma_\beta)^*\omega(x))X\\
	&=\omega(\sigma_\beta(x))(\sigma_\beta(x))_*X\\
	&=\omega(\sigma_\beta(x))((R_{g_{\alpha\beta}(x)}\sigma_\alpha(x))_*X)\\
	&=\omega(\sigma_\beta(x))\left((R_{g_{\alpha\beta(x)}})_*(\sigma_\alpha(x))_*X+\left(L_{\sigma_\alpha(x)}\right)_*(g_{\alpha\beta})_*X\right)\\
	&=\mathrm{Ad}(g_{\alpha\beta}^{-1})\left(\omega(\sigma_\alpha(x))(\sigma_\alpha(x))_*X\right)+\omega(\sigma_\beta(x))\left(\left(L_{\sigma_\alpha(x)}\right)_*(g_{\alpha\beta})_*X\right),\\
	&=\mathrm{Ad}(g_{\alpha\beta}^{-1})\left(\omega_\alpha(x)X\right)+L_{\sigma_\alpha(x)}^*(\omega)(g_{\alpha\beta}(x))\left((g_{\alpha\beta})_*X\right)\\
	&=\mathrm{Ad}(g_{\alpha\beta}^{-1})\left(\omega_\alpha(x)X\right)+g_{\alpha\beta}^*\circ L_{\sigma_\alpha(x)}^*(\omega)(x)\left(X\right)
\end{align*}
定义$\theta=L_{\sigma_\alpha(x)}^*(\omega)$,为了证实这个定义是合理的,我们需要检验这和选取的平凡化无关。选一个$Y\in T_g G$的元素,再重用一下符号$L$,将其看做群的左乘,那么$L_{g^{-1}}$的导数$(L_g^{-1})_* :T_g G\to T_e G=\mathfrak{g}$且$(L_g^{-1})_*Y=Y_0$.现在
\begin{align*}
	L_{\sigma_\alpha(x)}^*(\omega)(g)(Y)&=L_{\sigma_\alpha(x)}^*(\omega)(g)((L_g)_*Y_0)\\
	&=(\omega)(\sigma_\alpha(x)\cdot g)\left(L_{\sigma_\alpha(x)*}L_{g*}Y_0\right)\\
	&=(\omega)(\sigma_\alpha(x)\cdot g)\left(\left(L_{\sigma_\alpha(x)\cdot g}\right)_*Y_0\right),
\end{align*}
注意,$\left(L_{\sigma_\alpha(x)\cdot g}\right)_*Y_0$其实是$(\sigma_\alpha(x)\cdot g)\cdot \exp{(tY_0)}$的切矢量,所以他是$Y_0$诱导的基本矢量场的矢量$(Y_0)^*_{\sigma_\alpha(x)\cdot g}$,对于基本矢量场,按照联络形式的性质$\omega_{\sigma_\alpha(x)\cdot g}\left((Y_0)^*_{\sigma_\alpha(x)\cdot g}\right)=Y_0$,综上
\[
	\theta(Y)=L_{\sigma_\alpha(x)}^*(\omega)(g)(Y)=\omega_{\sigma_\alpha(x)\cdot g}(Y_0^*)=Y_0=\left(L_{g^{-1}}\right)_{*g}Y,
\]
所以$\theta$的定义是合理的。在Lie群理论里面,这个$\theta$称为典则左不变$\mathfrak{g}$值1-形式,或者Maurer-Cartan形式,称为左不变的是因为对任意的群元$g$都有$L_g^{*}\theta=\theta$,从定义看这是显然的。

最后$\omega_\alpha$在转移函数下的表现就是
\[
	\omega_\beta=\mathrm{Ad}(g_{\alpha\beta}^{-1})\omega_\alpha+g_{\alpha\beta}^*\theta,
\]
反之,给定了所有的$\omega_\beta$也就给出了联络形式,当然也给出了联络。

前面说过,一个主丛可以自然地通过表示和一个矢量丛(称为伴随丛)联系在一起。如果已知了表示$\rho:G\to \gl(V)$,我们就可以几乎把所有的$g\in G$换成$\rho(g)\in \gl(V)$而没什么问题,比如现在的矢量丛上的转移函数即$\varphi_{\alpha\beta}=\rho(g_{\alpha\beta})$.

给定主丛上的联络,即联络形式$\omega$,可以布局定义$\omega_\alpha$,由于$\rho:G\to \gl(V)$是Lie群同态,当然就诱导了Lie代数同态$\rho_*:\mathfrak{g}\to \mathfrak{gl}(V)$,那么$A_\alpha=\rho_*\omega_\alpha$就是$\mathfrak{gl}(V)$值1-形式。

因为一般线性群有直接继承自$\rr^{n^2}$的微分结构,所以
\[
	(L_g)_{*a}v=\lim_{t\to 0}\frac{1}{t}(L_g(a+tv)-L_g(a))=\lim_{t\to 0}\frac{1}{t}(L_g(tv))=L_g(v)=gv.
\]
其中$g$和$v$都是矩阵,矩阵乘矩阵还是矩阵,所以Lie代数也是矩阵的形式。设$\dd g=(\dd x_{ij})$,那么$v$就可以写成$\dd g(v)$,因为$\dd x_{ij}(v)=v_{ij}$,则
\[
	\theta=g^{-1}\dd g.
\]

所以一般线性群上的Maurer-Cartan形式即$g^{-1}\dd g$,将所有的$\lag$值形式$\omega$换成$\rho_*\omega$就好,那么按照这个原则
\begin{align*}
	A_\beta&=\rho(g_{\alpha\beta})^{-1}A_\alpha\rho(g_{\alpha\beta})+\rho(g_{\alpha\beta})^*\left(\rho(g)^{-1}\dd \rho(g)\right)\\
	&=\rho(g_{\alpha\beta})^{-1}A_\alpha\rho(g_{\alpha\beta})+\rho(g_{\alpha\beta})^{-1}\dd \rho(g_{\alpha\beta})\\
	&=\varphi_{\alpha\beta}^{-1}A_\alpha\varphi_{\alpha\beta}+\varphi_{\alpha\beta}^{-1}\dd \varphi_{\alpha\beta},
\end{align*}
这是前面我们已经见到过的变换方式。利用$A_\alpha$可以构造矢量丛上的联络,事实上,固定$V$的一组基$e_i$,那么可以在每一个$U_\alpha$上定义一个伴随丛的局部截面
\[
	\mu_{\alpha,i}(x)=\varphi_\alpha^{-1}(x,e_i)
\]
记$\mu_\alpha=(\mu_{\alpha,1},\cdots,\mu_{\alpha,n})$是一个行矢量,那么矩阵乘法给出
\[
	\mu_\alpha\cdot \rho(g_{\alpha\beta})=\mu_\alpha\cdot \varphi_{\alpha\beta}=\mu_\beta.
\]

将$A_\alpha$写成矩阵形式$A^i_{\alpha,j}$,我们可以定义些局部联络
\[
	D_\alpha\mu_{\alpha,i}=\mu_{\alpha,j}\otimes A^j_{\alpha,i},
\]
或者写成矩阵形式
\[
	D_\alpha\mu_{\alpha}=\mu_{\alpha}\otimes A_\alpha.
\]
如果$s=s^i\mu_{\alpha,i}$是$U_\alpha$中的截面,则定义局部联络算子
\[
	D_\alpha s=\mu_{\alpha,i}\otimes \dd s^i+s^iD_\alpha\mu_{\alpha,i}.
\]
容易通过$A_\alpha$和$\mu_\alpha$在转移函数下的行为证明在$U_\alpha\cap U_\beta\neq \varnothing$上,$D_\alpha=D_\beta$。这样,我们就给出了一个矢量丛上的联络$D$。

下面讨论主丛上的曲率。设$\eta_1$, $\eta_2$都是某个流形上的$\lag$值微分形式,取$\lag$的一组基$e_i$,所以
\[
	\eta_1=e_i\otimes \eta_1^i,\quad \eta_2=e_i\otimes \eta_2^i,
\]
定义,两个$\mathfrak{g}$值微分形式的对易为
\[
	[\eta_1,\eta_2]=[e_i,e_j]\otimes \eta_1^i\wedge\eta_2^i,
\]
容易验证这个定义和基的选取无关。根据定义,直接计算可以得到
\[
	[\eta_2,\eta_1]=(-1)^{\deg \eta_1\cdot \deg \eta_2+1}[\eta_1,\eta_2],\quad \dd[\eta_1,\eta_2]=[\dd\eta_1,\eta_2]+(-1)^{\deg \eta_1}[\eta_1,\dd\eta_2],
\]
此外对于两个1-形式,直接的计算也有
\[
	[\eta_1,\eta_2](X,Y)=[\eta_1(X),\eta_2(Y)]-[\eta_1(Y),\eta_2(X)].
\]
注意,根据上面的两个计算有
\[
	[\eta_1,\eta_1](X,Y)=[\eta_1(X),\eta_1(Y)]-[\eta_1(Y),\eta_1(X)]=2[\eta_1(X),\eta_1(Y)],
\]
此时一般是$[\eta_1,\eta_1]\neq 0$。类比矢量丛$F=\dd A+A\wedge A$,我们定义主丛上的曲率形式为
\[
	F=\dd \omega+\frac{1}{2}[\omega,\omega].
\]
曲率这样定义的合理性以及下面这个定理,不再详细叙述,$1/2$来自于$[\eta_1,\eta_1]$的计算,其余的可以自行检验两个表达式在矢量丛上的统一性。

\para Bianchi等式:$\dd F=[F,\omega]$.

% \section{形式化描述的场论}
% 这里开始使用上面的微分几何语言来形式化描述场论,当然,这只是一种形式化描述,不一定能概括所有的经典场论。

% 首先我们需要一个流形$M$,他被称为时空。时空为底流形,我们有(实或复的)矢量丛$E$,场是$E$的截面,场的空间记作$\mathcal{F}$。丛上的联络可以看作一个仿射空间丛的截面。对任意的情况,我们都可以定义一个运算映射
% \[
% \begin{array}{lcl}
% 	\mathcal{F}\times M&\to& E\\
% 	(\psi,x)&\mapsto &\psi(x).
% \end{array}
% \]
% Lagrangian
\section{Guage Field}
定义一个规范场论需要一个Lie群$G$,一个主$G$-丛$P$,底流形为$M$,此外Lie代数$\lag$上面有一个双不变度量,即$\langle \mathrm{Ad}_gX,\mathrm{Ad}_gY\rangle=\langle X,Y\rangle$对任意的$g\in G$和$X,Y\in \lag$都成立。

$P$上的一个联络被称为一个规范场,正如上面看到的,给出规范场只要给出其规范形式$A$即可,所以经常称呼$A$为规范场,但实际上,$A$不是坐标无关的,他依赖于局部化的选取,所以一个规范场是确定到一个联络上面的。在不同的局部化选取,或者说在不同规范下,规范场$A$的变换为
\[
	A_\beta=\mathrm{Ad}(g_{\alpha\beta}^{-1})A_\alpha+g_{\alpha\beta}^*\theta,
\]
如果$G$是一个矩阵群,那么变换写作
\[
	A_\beta=g_{\alpha\beta}^{-1}\cdot A_\alpha \cdot g_{\alpha\beta}+g_{\alpha\beta}^{-1}\cdot \dd g_{\alpha\beta}.
\]
规范场$A$的曲率$F_A$由下式定义:
\[
	F_A=\dd A+\frac{1}{2}[A,A].
\]
定一个规范,矩阵群下设$A=A_i\dd x^i$,那么
\[
	F=\frac{1}{2}\left(\partial_{[i}A_{j]}+[A_i,A_j]\right)\dd x^i\wedge \dd x^j.
\]

现在来看经典电动力学确实是一种规范场论,取$M$为$\rr^{3+1}$,$G=U(1)$,$P=M\times G$是一个平凡主$G$-丛。由于$\mathfrak{u}(1)=\rr$,所以所谓的$\mathfrak{u}(1)$值1-形式$A=A_i\dd x^i$其实就是普通的1-形式,其中$A_i$是标量函数。记规范场为$A=(\bm{A},\varphi)$,转移函数$\psi:M\to U(1)$的作用就变成了
\[
	A'=A+\psi^{-1} \dd \psi=A+\dd \ln(\psi),
\]
由于定义了同一个联络,所以也就是同一个场。此外,曲率$F_A$就是
\[
	F=\frac{1}{2}\partial_{[i}A_{j]}\dd x^i\wedge \dd x^j,
\]
他的矩阵当然就是熟悉的定义$F_{ij}=\partial_{[i}A_{j]}$,即电磁场张量。因为这里曲率是反对称的,他的$1+2+3=6$个独立分量确定了$\bm{E}$和$\bm{B}$的全部空间分量。Bianchi等式$\dd F_A=[F_A,A]$或在矢量丛下写作$DF_A=0$就告诉了我们第一组Maxwell方程。

第二组Maxwell方程的写出需要作用量,类比电磁场,一个纯经典规范场的作用量写作
\[
	S_{YM}(A)=-\frac{1}{2}\int_M F_A\wedge(\star F_A),
\]
他被称为Yang-Mills作用量。一般在定义星算子的时候,会选择Killing形式作为非退化双线型(加上半单假设)。经典电动力学的作用量中的Killing形式即两个实数相乘。如果含有源,则作用量写作
\[
	S(A)=-\int_M A\wedge\star j-\frac{1}{2}\int_M F_A\wedge\star F_A.
\]
如果考虑Lie群是矩阵群,那么可以在矢量丛下对其变分就得到了第二组Maxwell方程
\[
	\star D\star F_A=-j.
\]

\section{Orbital Geometry of the Adjoint Action}

上面看过了Lie代数的伴随表示,现在来看看Lie群的伴随作用,其中担任有趣的角色的就是极大交换子群,类似于Cartan子代数是极大交换子代数。

由于我们考虑是紧Lie群$G$,那么我们总可以得到他的Lie代数$\lag$上伴随作用下不变的内积,因此我们就可以定义$\lag$上面的一组正交基,选定正交基之后我们拓展到对应的左不变矢量场上,我们就得到了一个标架场,也就是确定了$G$上唯一的Riemann结构使得那些左不变矢量场在每一个点都相互正交。

这样定义的Riemann结构,由于标架场是左不变矢量场,所以左平移$(l_a)_*$是等距同构,而伴随作用因为我们定义的内积也是等距同构,那么右平移作为左平移和伴随作用的复合,也是等距同构。

伴随作用的轨道确定了Lie群的共轭类,而又因为伴随作用是等距同构,这就是说共轭类在几何结构上大致是相同的。这节中重要的定理即明确了这个事实。

\para 一个紧Lie群的环子群是一个紧的连通交换Lie子群。一个极大环子群就是说这个环子群不能真包含在其他的环子群里面。

$\mathrm{SO}(2)$是$\mathrm{SO}(3)$的极大环子群,$\mathrm{SO}(2)\times \mathrm{SO}(2)$是$\mathrm{SO}(4)$的极大环子群,环面$\mathrm{T}^n=\{\mathrm{diag}(e^{i\theta_1}$, $\cdots$, $e^{i\theta_n}):0\leq\theta_i<2\pi\}$是$U(n)$的极大环子群。由于$\mathrm{SO}(2)$同构于环$\mathrm{S}^1$,而$\mathrm{T}^n$同构于$\mathrm{S}^1\times\cdots\times\mathrm{S}^1$,所以环子群的名字是十分形象的。

\pro 令$T$是$G$的环子群以及$F(T,\lag)$或$F(T,G)$是$T$在$\lag$或者$G$上的伴随作用的不动点集,那么$T$是极大环子群当且仅当$\dim F(T,\lag)=\dim T$或者$F(T,G)$包含$T$作为其中之一的连通分支。

让$\mathfrak{t}$是$T$的Lie代数,那么每一个$t\in T$都满足$\mathrm{Ad}_t\mathfrak{t}=\mathfrak{t}$,这是Abel群的自然结果,所以
\[
	\mathfrak{t}\subset F(T,\lag).
\]
如果$T$不是极大的,那么存在一个$T_1$使得$T\subset T_1$,且他的Lie代数满足$\mathfrak{t}_1\subset F(T_1,\lag)$.所以
\[
	\dim T=\dim \mathfrak{t}<\dim \mathfrak{t}_1\leq \dim F(T,\lag).
\]

反过来,如果$\dim  F(T,\lag)>\dim T$,那么存在$X\in F(T,\lag)-\mathfrak{t}$使得分布$\{X,\mathfrak{t}\}$是$\lag$的一个Lie子代数,且上面的交换子都为$0$。于是存在一个Abel子群$H$其Lie代数为$\{X,\mathfrak{t}\}$,我们现在考虑$H$的闭包,他是一个紧集中的闭集,所以也是紧的(Hausdorff性是流形保证的),所以他是一个真包含$T$的环子群,这和$T$是极大环子群相悖。

至于剩下的关于Lie群的结论,显然来自于其Lie代数和Lie群的联系。

下面这个定理就是这节的主要内容。

\theo 令$T$是紧Lie群$G$的极大环子群,那么每个$g\in G$共轭于某个$T$的元素。

因此所有的极大环子群都是相互共轭的。

令$\varphi$是Lie群$G$的伴随表示在$T$上的限制$\varphi=\mathrm{Ad}|_T$,且$\mathfrak{t}$是$T$的Lie代数。因为$T$是极大环子群,所以从上一个命题可以推知$F(T,\lag)=\mathfrak{t}$,注意到极大环子群是Abel群,所以他的每一个复的不可约有限维表示都是一维的,则实的不可约有限维表示都是二维的,因此
\[
\varphi=\dim \mathfrak{t}\cdot 1\oplus \varphi_1\oplus\cdots \oplus \varphi_l.
\]
其中$1$表示为一维平凡表示,$\varphi_i$是一维非平凡不可约表示,由于$\varphi_i$都可以表示为一个幺正表示,也即$e^{i\theta}$,那么$\varphi_i$也可以看成一个非平凡的从$T$到$\mathrm{SO}(2)$的群同态,那么$\varphi_i$的核是余维度为1的闭Lie子群,他们的并$\cup\ker(\varphi_i)$的补集是$T$的一个开子流形,我们记做$W$.

令$t_0\in W$,则每一个$\varphi_i(t_0)$都是一个不平凡的旋转,因此$F(\varphi(t_0),\lag)=\mathfrak{t}$,令$G_{t_0}=\{g\in G:gt_0g^{-1}=t_0\}$是$t_0$的中心化子,以及$e^{sX}$是$G_{t_0}$的任意的单参子群,于是
\[
	e^{sX}=t_0e^{sX}t_0^{-1}=e^{s\mathrm{Ad}(t_0)X}
\]
对任意的$s\in\rr$都成立,于是$X\in F(\varphi(t_0),\lag)=\mathfrak{t}$.所以$G_{t_0}$的Lie代数就是$\mathfrak{t}$,那么他带单位元的连通分支就是$T$,因此$\dim G(t_0)=\dim G -\dim G_{t_0}=\dim G -\dim T$,其中$G(t_0)$是$t_0$的共轭作用的轨道,或者说是$t_0$的共轭类。我们下面要证明$G(t_0)$是$T$在$t_0$处的正交补。

注意到两个$t_0,t\in T$是可交换的,因此$\sigma_t:x\mapsto txt^{-1}$和$l_{t_0}$对任意的$x\in G$都是可以交换的。因此$(l_{t_0})_*$是$\lag$和$T_{t_0}G$对于伴随作用等价的映射,即$(l_{t_0})_*(\sigma_t)_*=(\sigma_t)_*(l_{t_0})_*$,这样$(l_{t_0})_*$就类似一个缠结映射。

因为
\[
\lag=\mathfrak{t}\oplus\mathfrak{t}^\bot,
\]
所以有
\[
\varphi|_\mathfrak{t}=\dim \mathfrak{t}\cdot 1,\quad \varphi|_{\mathfrak{t}^\bot}=\varphi_1\oplus\cdots \oplus \varphi_l.
\]
因为$(l_{t_0})_*\mathfrak{t}$就是$T$在$t_0$处的切空间,为了证明$G(t_0)$在$t_0$处的切空间就是$T$在$t_0$处的切空间的正交补,我们可以去证明$T$的伴随作用诱导在切空间上的作用不包含任意固定的方向。

\pro 令$H$是$G$的紧Lie子群,他们的Lie代数分别为$\mathfrak{h}$和$\lag$,则$\lag=\mathfrak{h}\oplus\mathfrak{h}^\bot$。那么在$G/H$基点处的切空间$T_0(G/H)$诱导的H作用等价于$H$的伴随作用在$\mathfrak{h}^\bot$上的限制。

设$p(x)=xH\in G/H$是正则投影,那么
\[
p\circ \sigma_h(x)=p(hxh^{-1})=hxh^{-1}\cdot H=hx\cdot H=l_h(x\cdot H)=l_h\circ p(x),
\]
对所有$h\in H$都成立。那么
\[
p_{*e}:\lag=\mathfrak{h}\oplus\mathfrak{h}^\bot\to T_0(G/H)
\]
是$\mathfrak{h}^\bot$和$T_0(G/H)$(作为$H$-线性空间)间的一个同构映射。

注意到作为群作用的共轭类,该作用的轨道就是共轭类,而给定元素的定点子群就是该元素的中心化子。所以$t_0$的轨道就是$G$和$t_0$的中心化子的商群,即$G(t_0)=G/G_{t_0}$,那么我们就得到了$\mathfrak{t}^\bot$和$T_0(G/G_{t_0})$之间的同构,而这和$(l_{t_0})_*\mathfrak{t}$的正交是显然的。

\theo $G$上的两个有限维表示$\psi_1,\psi_2$是等价的,当且仅当他们在极大环子群上的限制是等价的。

$\psi_1,\psi_2$等价和$\chi_{\psi_1}=\chi_{\psi_2}$是一个意思,那么因为每一个$g\in G$都共轭于某个$T$的元素,而共轭对迹没影响,所以$\chi_{\psi_1}=\chi_{\psi_2}$和$\chi_{\psi_1}|_T=\chi_{\psi_2}|_T$是一个意思,而$\chi_{\psi_1}|_T=\chi_{\psi_2}|_T$又和$\psi_1|_T,\psi_2|_T$等价是一个意思,所以$\psi_1,\psi_2$等价和$\psi_1|_T,\psi_2|_T$等价是一个意思。

注意到极大环子群是Abel群,所以他的有限维不可约表示都是一维的,而紧Lie群保证了完全可约性,那么我们只要确定了这些一维不可约表示,也就确定了整个Lie群上面的表示。

\para 群$W(G)=N_G(T)/T$被称为$G$的Weyl群,其中$N_G(T)$是$T$的正规化子,就是说是所有和$T$可交换的元素构成的群。

%\appendix
\renewcommand{\thepara}{\Alph{chapter}.\arabic{para}}

\chapter{Zorn引理与等价关系}

\section{Zorn引理}
\para 设$A$是一个集合,而$\leq$是$A$上的一个关系,如果满足

\no{1} $\forall x\in A$, $x\leq x$;

\no{2} 如果$x\leq y$以及$y\leq z$,则$x\leq z$;

\no{3} 如果$x\leq y$以及$y \leq x$,则$x=y$.\\则此时我们称$(A,\leq)$是一个偏序集,称$\leq$是$A$的一个偏序。为方便,有时候会直接说$A$是一个偏序集。

在记号上,如果$x\leq y$,则我们有时候也会记做$y\geq x$. 可以看到,如果$(A,\leq)$是一个偏序集,则$(A,\geq)$也是一个偏序集。

显而易见地,在实数域$\rr$上有一个自然的偏序,就是通常的小于等于,或者是大于等于。特别地一点是,实数域上任意两个数是可以比较的,即对$x$, $y\in \rr$,一定成立$x\leq y$或者$y\leq x$,但对于一般的偏序集而言,任意两个元素不一定可以比较。

\para 设$(A,\leq)$是一个偏序集,而$T$是一个$A$的子集,则$(T,\leq)$自然也是一个偏序集。如果$T$中的任意两个元素可以比较,这就称呼$T$是一个链。特别地,如果$A$本身就是一个链,则$A$称为一个全序集。因此链又叫做全序子集。

\para 所谓链$T\subset A$的上界,就是某个$a\in A$,使得$x\leq a$对任意的$x\in T$都成立。下界对应的就是使得$a\leq x$对任意的$x\in T$都成立的$a\in A$。显然,上下界不一定唯一。

\para 设$B$是$A$的一个子集,设$x\in B$,如果对于$B$中任意可以与$x$比较的元素$a$,都有$a\leq x$,则称呼$x$是$B$的一个极大元。同理可以有极小元。

极大元可能不存在,也可能存在很多个。比如设除去$1$的正整数集合$\zz^+-\{1\}=\{2$, $3$, $\cdots\}$,我们如下赋予一个偏序:如果$a$整除$b$(通常记做$a|b$),则$b\leq a$。那么任意的素数$p$都是$\zz^+$中的极大元,因为任意可以与$p$比较的元素都是$np$的形式,因为$p|np$,所以$np\leq p$.

\para 设$B$是$A$的一个子集,设$x\in B$,如果对于$B$中任意的元素$a$,都有$a\leq x$,则称呼$x$是$B$的一个最大元。同理可以有最小元。

最大元不一定存在,但如果存在显然只能有一个,如果有两个,$a$或者$b$,则$a\leq b$以及$b\leq a$,所以$a=b$. 对于全序集来说,极大元和最大元等价。

\theo Zorn引理:设$(A,\leq)$是一个非空偏序集,且其中的每一条链都存在上界,则$A$中存在极大元。

Zorn引理是选择公理的等价形式,这里不做证明了。

\section{等价关系}
\para 一个关系$\sim$被称为集合$S$上的等价关系,如果
\[
	a\sim a,
\]
\[
	a\sim b \Rightarrow b \sim a,
\]
\[
	a\sim b\text{ and } b\sim c \Rightarrow a \sim c,
\]
第一个被称为自反性,第二个被称为对称性,第三个被称为传递性。

\para 一旦给定集合$S$上的一个等价关系$\sim$,我们把与$a$等价的那些元素构成的子集记作$\bar{a}$. 从等价关系的三个性质可以看出:
\begin{itemize}
\item $a$与$b$等价当且仅当$\bar{a}=\bar{b}$. 
\item $\bigcup_{a\in S}\bar{a}=S$.
\item 如果$\bar{a}\cap \bar{b}\neq \varnothing$,则$\bar{a}=\bar{b}$.
\end{itemize}
这样的子集被称为等价类。从上面的第二点可以看出,不同的等价类不相交,所以通过等价关系,我们将$S$划分成了许多不相交的子集。满足上面性质的后两点的子集族被称为$S$的一个划分。

\para 反过来,如果我们给定了$S$的一个划分,则可以定义出一个等价关系:$a$和$b$等价当且仅当他们同处于一个划分$S_i$里面。而划分在这个等价关系里面就构成了等价类。所以等价关系与划分等价。

\end{document}
